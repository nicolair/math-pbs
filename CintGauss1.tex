\begin{enumerate}
 \item On rappelle que $a \in \left[ 1,2 \right]$. 
\begin{enumerate}
 \item Appliquons la formule de Taylor avec reste intégral à la fonction $t \mapsto e^{-at}$ à l'ordre 1 entre $0$ et $x$.
\begin{multline*}
 e^{-ax} = 1 - a x + \int_{0}^{x}(x - t)(-a)^2 e^{-at}\, dt
 = 1 - a x + a^2\int_{0}^{x}(x - t) e^{-at}\, dt \\
\Rightarrow \varphi(x) = a^2\int_{0}^{x}(x - t) e^{-at}\, dt. 
\end{multline*}
Pour étudier le signe de $\varphi$, séparons les cas $x\geq 0$ et $x \leq 0$.\newline
Pour $x\geq 0$, $\varphi(x)$ est clairement positif car les bornes d'intégration sont dans le \og bon sens\fg~ et $x-t\geq 0$ pour $t\in \left[ 0, x\right]$.\newline
Pour $x\leq 0$, on intervertit les bornes d'intégration : 
\[
 \varphi(x) = -a^2\int_{x}^0(x-t)e^{-at}\,dt = a^2\int_{x}^0\underset{\geq 0 }{\underbrace{(t-x)}}e^{-at}\,dt \geq 0.
\]
 \item  Interprétons l'hypothèse sur $x$:
\[
 x \geq -\frac{\ln 2}{a} \Leftrightarrow -ax \leq \ln 2 \Leftrightarrow e^{-ax} \leq 2.
\]
Introduisons cette inégalité pour majorer l'intégrale. Dans le cas $x\geq 0$,
\[
 0 \leq \varphi(x) \leq a^2 \int_0^x(t-x) 2 \,dt = a^2 x^2.
\]
Le calcul est analogue dans le cas $x\leq 0$. Comme tout est posifif, on écrit l'inégalité avec des valeurs absolues avant de diviser par $x$.
\[
 0 \leq \varphi(x) \leq a^2x^2 \Rightarrow \left| \varphi(x) \right| \leq a^2 |x|^2
 \Rightarrow 0 \leq \left|\frac{\varphi(x)}{x} \right| \leq a^2 |x|.
\]
\end{enumerate}

 \item 
\begin{enumerate}
 \item Avec les définitions,
\[
 h(0) = \int_0^1 \frac{dt}{1+t^2} + 0^2 = \frac{\pi}{4}.
\]

 \item Utilisons le changement de variable $u = xt$. Alors $du = x\,dt$ et
\[
 \int_{0}^{1}e^{-x^2t^2}\,dt = \int_{0}^{x} e^{-u^2}\,\frac{du}{x} = \frac{1}{x}\lambda(x) 
 \Rightarrow \lambda(x) = x \int_{0}^{1}e^{-x^2t^2}\,dt.
\]

 \item Avec la formule de dérivation admise en 1. et le résultat de la question précédente,
\begin{multline*}
 h'(x) = 2xf'(x^2) + 2\lambda'(x) \lambda(x)\\
 = - 2x\int_{0}^{1}e^{-x^2(1+t^2)}\,dt + 2 e^{-x^2} x \int_{0}^{1}e^{-x^2t^2}\,dt
 = 0 .
\end{multline*}
La fonction $h$ est donc constante sur l'intervalle $\left[ 0, +\infty\right[$ de valeur $\frac{\pi}{4}$. 

 \item La fonction à intégrer est positive. Avec $x >0$, on majore grossièrement en $t$
\[
 1+t^2 \geq 1 \Rightarrow
 \left\lbrace 
 \begin{aligned}
  e^{-x(1+t^2)} &\leq e^{-x} \\ \frac{1}{1+t^2} \leq 1
 \end{aligned}
\right. \Rightarrow
0 \leq f(x) \leq \int_0^1 e^{-x}\, dt = e^{-x}. 
\]
On en déduit que $f\rightarrow 0$ en $+\infty$.

 \item On passe à la limite en $+\infty$ dans la relation tirée de 2.c.
\[
 \frac{\pi}{4} = \underset{ \rightarrow 0}{\underbrace{f(x^2)}} + \lambda(x)^2 \Rightarrow \lambda(x) = \sqrt{\frac{\pi}{4} - f(x^2)} \rightarrow \frac{\sqrt{\pi}}{2}.
\]

\end{enumerate}

\end{enumerate}
