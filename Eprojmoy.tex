%<dscrpt>Moyennisation d'un projecteur.</dscrpt>
Cet exercice porte sur un procédé connu sous le nom de \emph{moyennisation d'un projecteur}.\footnote{Exercice 1 de e3a 2001 MP 2}
Soit $E$ un espace vectoriel r{\'e}el, $F$ un sous espace vectoriel de $E$ et $G$ un sous-groupe fini (avec $m$ éléments) du groupe des automorphismes de $E$.\newline
On suppose que le sous-espace $F$ est stable par les {\'e}l{\'e}ments de $G$ c'est {\`a} dire que :
\[
\forall g \in G , \forall x \in F :\; g(x)\in F.
\]

{\`A} tout {\'e}l{\'e}ment $u$ de $\mathcal{L}(E)$, on associe $u^+$ d{\'e}fini par
\[
u^+ = \frac{1}{m}\sum_{g\in G} g^{-1}\circ u \circ g.
\]
\begin{enumerate}
  \item Montrer que $u^+$ est un endomorphisme de $E$ commutant avec tout {\'e}l{\'e}ment $h$ de $G$.
  \item Calculer $(u^+)^+$.
  \item Soit $p$ un projecteur de $E$ avec $F = \Im (p)$.
  \begin{enumerate}
    \item Montrer que $\Im(p^+) = F$.
    \item Montrer que :
  \[
\forall (g,h)\in G^2,\;  g^{-1} \circ p \circ g \circ h^{-1} \circ p \circ h = h^{-1} \circ p \circ h.
  \]
    \item Montrer que $p^+$ est un projecteur.
    \item Montrer que le noyau de $p^+$ est stable par tout {\'e}l{\'e}ment $g$ de $G$.
  \end{enumerate}
\end{enumerate}