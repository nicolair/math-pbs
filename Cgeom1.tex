\begin{enumerate}
\item On rappelle les deux formules usuelles:
\begin{eqnarray*}
e^{ia} + e^{ib}&=& 2\cos \frac{a-b}{2}e^{i\frac{a+b}{2}}\\
e^{ia} - e^{ib}&=& 2i\sin \frac{a-b}{2}e^{i\frac{a+b}{2}}
\end{eqnarray*}
On en déduit les coordonnées d'un vecteur unitaire orthogonal à $(AB)$:
\[(\cos \frac{a+b}{2}, \sin \frac{a+b}{2})\]
et les coordonnées du milieu du segment $[A,B]$ 
\[(\cos \frac{a-b}{2}\cos \frac{a+b}{2}, \cos \frac{a-b}{2} \sin \frac{a+b}{2})\]

\item On écrit l'équation normale de la droite $(AB)$ en utilisant le milieu des deux points qui est sur la droite. Après calculs, l'équation de $(AB)$ devient :
\[x \cos \frac{a+b}{2} + y \sin \frac{a+b}{2} - \cos \frac{a-b}{2}=0\]
\item Le vecteur normal utilisé dans l'équation précédente étant unitaire, la distance est la valeur absolue de l'expression:
\begin{eqnarray*}
\mathrm{d}(Mt,(AB)) &=& \vert \cos t \cos \frac{a+b}{2} + \sin t \sin \frac{a+b}{2} - \cos \frac{a-b}{2}\vert \\
&=& \vert \cos (\frac{a+b}{2}-t) - \cos (\frac{a-b}{2})\vert \\
&=& \vert -2 \sin (b-t) \sin(a-t) \vert
\end{eqnarray*}
On ne peut pas écrire cette distance sans valeur absolue car le signe des sinus change suivant que l'on est sur l'un ou l'autre des deux arcs $(AB)$.
\item Lorsque l'on considère l'expression proposée par l'énoncé, on peut faire sortir la valeur absolue et simplifier
\[\left \vert \frac{( -2 \sin (b-t) \sin(a-t))( -2 \sin (d-t) \sin(c-t))} {( -2 \sin (a-t) \sin(c-t))( -2 \sin (d-t) \sin(b-t))}\right \vert = 1\]
\end{enumerate}