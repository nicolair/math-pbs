\def\Cos{\textnormal{Cos}}
\def\Sin{\textnormal{Sin}}

\begin{enumerate}
  \item Le point important dans cette question est de montrer que l'on sait que l'on ne sait rien ou presque des fonctions d'une variable complexe.\newline
Les seuls points au programme sur ce thème portent sur les fonctions rationnelles et la fonction exponentielle complexe.
\begin{enumerate}
 \item On utilise des propriétés de la fonction exponentielle complexe. L'ensemble des $z\in \C$ pour lesquels $\exp(z)=1$ est $2i\pi \Z$, on connait quelques valeurs particulères dont $\exp(i\pi)=-1$. Voir la \href{http:/back.maquisdoc.net/data/coursnicolair/C2002.pdf}{présentation axiomatique de la fonction exponentielle}. On en déduit (en utilisant la notation puissance) :
\begin{displaymath}
 z\in Z \Leftrightarrow e^{iz}+e^{-iz}=0\Leftrightarrow e^{2iz}=-1=e^{i\pi}
\Leftrightarrow e^{2iz-i\pi}=1
\Leftrightarrow z\in \frac{\pi}{2} + \pi\Z
\end{displaymath}
d'où 
\begin{displaymath}
 Z = \frac{\pi}{2} + \pi\Z
\end{displaymath}

 \item Les arcs sont réguliers car les dérivées de $\gamma_1$ et $\gamma_2$ ne s'annulent pas et les fonctions $\gamma_1$ et $\gamma_2$ prennent leurs valeurs dans $Z$ ce qui entraine que les $\Cos$ sont non nuls.
 \item L'angle orienté $\widehat{(\overrightarrow{g_{1}}'(t_0),\overrightarrow{g_{2}}'(t_0))}$ est un argument du quotient complexe
\begin{displaymath}
 \frac{\gamma_2'(t_0)}{\gamma_1'(t_0)}
\end{displaymath}
De même l'angle orienté $\widehat{(\overrightarrow{f_{1}}'(t_0),\overrightarrow{f_{2}}'(t_0))}$ est un argument du quotient complexe
\begin{displaymath}
  \frac{(\Sin \circ\gamma_2)'(t_0)}{(\Sin \circ\gamma_1)'(t_0)}
=
\frac{\Cos(\gamma_2(t_0))\gamma_2'(t_0)}{\Cos(\gamma_1(t_0))\gamma_1'(t_0)}
= \frac{\gamma_2'(t_0)}{\gamma_1'(t_0)}
\end{displaymath}
car $\gamma_1(t_0)=\gamma_2(t_0)$. Les angles sont donc égaux modulo $2\pi$.
\end{enumerate}
 
 \item Les définitions de l'énoncé montrent bien que $Ox$ est l'axe focal et $O$ le centre dans les deux cas. Les relations entre l'équation réduite et la distance centre-foyer sont des formules du \href{http:/back.maquisdoc.net/data/coursnicolair/C4893.pdf}{cours sur les coniques}.
\begin{align*}
 &F \text{ et } F' \text{ foyers de } \mathcal H_{a,b} \Leftrightarrow  a^2 + b^2 = 1 \\ 
&F \text{ et } F' \text{ foyers de } \mathcal E_{a,b} \Leftrightarrow  a^2 - b^2 = 1 
\end{align*}

\item
On paramètre la droite d'équation $y=y_0$ par $t \mapsto t+ iy_0$ avec $t\in \R$. On a alors pour $t\in\R$
\begin{eqnarray*}
\Sin (t+iy_0) &=& \frac1{2i} (e^{it-y_0} - e^{-it +y_0} ) \\
&=& \frac1{2i} \left( e^{-y_0} (\cos t + i\sin t ) - e^{y_0} (\cos t - i \sin t) \right) \\
&=& \frac{i}2 \cos t (e^{y_0} - e^{-y_0}) + \frac{i}{2i} \sin t (e^{y_0} + e^{-y_0} ) = a\sin t +i b \cos t
\end{eqnarray*}
avec
\begin{displaymath}
 a = \frac{e^{y_0} + e^{-y_0}}{2}=\ch(y_0) \text{ et } b = \frac{e^{y_0} - e^{-y_0}}2 = \sh(y_0)
\end{displaymath}
On reconna\^it la forme complexe d'un paramétrage de l'ellipse d'équation 
\begin{displaymath}
 \frac{x^2}{a^2} + \frac{y^2}{b^2} =1
\end{displaymath}
(Poser $t = u-\pi/2$ pour retrouver $(a \cos u, b \sin u)$.) Comme $a>b>0$, il s'agit bien de l'ellipse $\mathcal E _{a,b}$, ses foyers sont bien $F$ et $F'$ car
\begin{displaymath}
 a^2 - b^2 = \ch(y_0)^2 - \sh(y_0)^2 = 1
\end{displaymath}

\item 
Remarquons que $x^2=x_0^2$ équivaut \`a $x=x_0$ ou $x = -x_0$. Déterminons l'image par $\Sin$ de ces deux droites. On param\`etre la droite d'équation $x=x_0$ par $t \mapsto x_0 + it$ avec $t\in \R$. Alors pour $t\in \R$
\begin{eqnarray*}
\Sin (x_0 + it) &=& \frac1{2i} (e^{ix_0 - t } - e^{-ix_0 + t}) \\
&=& \frac1{2i} (e^{-t} (\cos x_0 + i \sin x_0)
-e^t (\cos x_0 - i \sin x_0)) \\
&=& \sin x_0 \frac{e^t + e^{-t}}2 + i \cos x_0 \frac{e^{t} - e^{-t}}2. 
\end{eqnarray*}
En posant $a = \sin x_0$ et $b = \cos x_0$, on reconna\^it un paramétrage d'une branche de l'hyperbole d'équation $\frac{x^2}{a^2}- \frac{y^2}{b^2} = 1$. En changeant $x_0$ en $-x_0$, on change le signe de $\sin x_0$ et on obtient ainsi les deux branches de l'hyperbole. 
\par
Les foyers sont $F$ et $F'$ car $c = \sqrt{a^2+b^2} = 1$.   

\item 
Soient $\mathcal H$ et $\mathcal E$ respectivement une hyperbole et une ellipse de foyers $F,F'$.
 Alors ces coniques sont centrées en $O$, milieu de $[F,F']$, d'axe focal $(FF') = Ox$, donc sont
  respectivement de la forme $\mathcal H_{a,b}$ et $\mathcal E_{a',b'}$ avec $a^2+b^2 =1$ et 
  $a'^2 - b'^2 =1$. On choisit $x_0 \in ]0,\pi/2[$ de sorte que $a = \sin x_0$ et $b = \cos x_0$.
L'hyperbole $\mathcal H$ est alors l'image par $\Sin$ des deux droites d'équations $x = \pm x_0$ 
d'apr\`es la question précédente. 
\par
On choisit aussi $y_0 >0$ tel que $a' = \frac{e^{y_0}+ e^{-y_0}}{2}$ et $b' = \frac{e^{y_0}- e^{-y_0}}{2}$. C'est possible en posant $y_0 = \ln (a'+ b')$. On a bien 
$$e^{y_0}+  e^{-y_0} = a'+b'  + \frac1{a'+b'} = a'+ b' + \frac{a'-b'}{a'^2-b'^2} = 2a'.$$
 De m\^eme 
$e^{y_0} - e^{-y_0} = 2b'$. On a alors que $\mathcal E$ est l'image par $\Sin$ de la droite d'équation $y=y_0$. Or les courbes paramétrées $t \mapsto \gamma_1(t) = x_0 + it$ (ou $x_0 -it$) et $t \mapsto \gamma_2(t) = t+iy_0$ sont orthogonales en tout point, donc leur image par $\Sin$ le sont aussi d'apr\`es la question 1. 
\end{enumerate}
