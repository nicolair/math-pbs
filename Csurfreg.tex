\begin{enumerate}
 \item D'après les propriétés des sinus et cosinus hyperboliques, pour $u$ réel et $\varepsilon\in\{-1,+1\}$,
\begin{displaymath}
 \left. 
\begin{aligned}
 a &= \varepsilon \ch u \\ b &= \sh u
\end{aligned}
\right\rbrace 
\Rightarrow a^2 - b^2 =1
\end{displaymath}
Réciproquement, si $a^2-b^2=1$ alors $a^2=1+b^2\geq 1$. D'après les propriétés de la fonction sinus hyperbolique, il existe $u\in \R$ tel que $b=\sh u$. On en déduit alors $\ch^2 u = a^2$ d'où $a=\varepsilon \ch u$ avec $\varepsilon = 1$ si $a>0$ et $\varepsilon = -1$ si $a<0$. 
 \item Par définition de $\mathcal{S}$, un point quelconque $M$ appartient à $\mathcal{S}$ si et seulement si il existe $H\in \mathcal H$ et $K\in D_0$ tels que $\overrightarrow{HK}\perp \overrightarrow k$ et $M\in (HK)$.\newline
D'après la première question, les points de $\mathcal{H}$ sont de la forme $H_\varepsilon(u)$ et il existe un seul point de $D_0$ tel que $\overrightarrow{H_\varepsilon(u)K}\perp \overrightarrow k$. On le note $K(u)$. Précisons les coordonnées :
\begin{displaymath}
 H_\varepsilon(u) : (1,2\varepsilon \ch u, 2\sh u)\hspace{0.5cm}
 K(u) : (0,0,2\sh u)
\end{displaymath}
 Un point $M$ appartient à $\mathcal{H}$ si et seulement si
\begin{displaymath}
 \exists \varepsilon\in\{-1,1\}, \exists u\in \R, \exists \lambda\in \R \text{ tels que } 
M= K(u)+\lambda\, \overrightarrow{K(u)H_\varepsilon(u)}
\end{displaymath}
On peut traduire cette condition en coordonnées:
\begin{displaymath}
 M\in\mathcal{S} \Leftrightarrow
\exists \varepsilon\in\{-1,1\}, \exists u\in \R, \exists \lambda\in \R \text{ tels que } 
\left\lbrace 
\begin{aligned}
 x(M) &= \lambda \\
 y(M) &= 2\lambda \varepsilon \ch u \\
 z(M) &= 2\sh u
\end{aligned}
\right. 
\end{displaymath}

 \item Former une équation cartésienne de $\mathcal{S}$ c'est trouver une relation entre les coordonnées assurant de l'existence des paramètres  dans la caractérisation paramétrique de la question précédente. On peut remplacer $\lambda$ par $x(M)$. La question 1 montre alors que
\begin{displaymath}
 M\in\mathcal{S} \Leftrightarrow
\left(\frac{y(M)}{x(M)} \right)^2 - z(M)^2 = 4 \Leftrightarrow
y(M)^2 = x(M)^2(z(M)^2+4) 
\end{displaymath}

\end{enumerate}
