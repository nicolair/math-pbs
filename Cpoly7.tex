\begin{enumerate}
 \item
\begin{enumerate}
 \item Le polynôme s'écrit
\begin{displaymath}
 P = X^2-(a_1+a_2)X+a_1a_2 = X^2-X-1
\end{displaymath}
 \item On effectue la division euclidienne de $A$ par $P$. On obtient $A=QP+R$ avec $Q=X^3-X-1$ et $R=1$. Comme $a_1$ et $a_2$ sont des racines de $P$, on en déduit que la valeur commune est $1$.
\end{enumerate}
 
 \item On considère 
\begin{displaymath}
 P=(X^2+1)^n=(X+i)^2(X-i)^n
\end{displaymath}
et son coefficient de degré $2p$.\newline
D'après la formule du binôme dans l'expression de gauche:
\begin{displaymath}
 c_{2p}(P)=\binom{n}{p}
\end{displaymath}
En utilisant l'expression de droite:
\begin{multline*}
 c_{2p}(P)= \sum _{k+l=2p}c_k((X+i)^n)c_l((X-i)^n)
=\sum _{k+l=2p}\binom{n}{k}(i)^{n-k}\binom{n}{l}(-i)^{n-l}\\
= \sum _{k=0}^{2p}(i)^{2n-2p}(-1)^{n-2p+k}\binom{n}{k}\binom{n}{2p-k}
\end{multline*}
Comme toute puissance d'exposant pair de $-1$ est égale à $1$:
\begin{displaymath}
 (i)^{2n-2p}(-1)^{n-2p+k} = (-1)^{2n-3p+k}=(-1)^{p+k}
\end{displaymath}
d'où, en identifiant les deux expressions de ce coefficient,
\begin{displaymath}
 \binom{n}{p} = \sum _{k=0}^{2p}(-1)^{p+k}\binom{n}{k}\binom{n}{2p-k}
\end{displaymath}

\end{enumerate}
