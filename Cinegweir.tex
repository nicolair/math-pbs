\subsection*{Partie I. Inégalités classiques.}
\begin{enumerate}
\item
\begin{enumerate}
 \item On veut montrer par récurrence que $1-\sum_{k=1}^n a_k \leq \prod_{k=1}^n(1-a_k)$.\newline
Pour $n=1$ et $a_1\in [0,1]$, l'inégalité est triviale. \newline
Soit $(a_1,\dots,a_{n+1})\in [0,1]^{n+1}$. Alors
\begin{multline*}
\prod\limits_{k=1}^{n}(1-a_k) 
= \left(\prod\limits_{k=1}^n(1-a_k)\right)\left(1-a_{n+1}\right)\\
\se \left(1-S_n\right)(1-a_{n+1})\text{ par hypothèse de récurrence et car } 1-a_{n+1} \se 0\\
\se 1-(\underset{ = S_{n+1}}{\underbrace{S_n + a_{n+1}}}) + S_n a_{n+1} \se 1 - S_{n+1}.
\end{multline*}
 \item On veut montrer par récurrence que $M_n (1 + S_n)\leq 1$ (car $1+S_n >0$).\newline
 Pour $n=1$ l'inégalité est facile:
 \[
  (1-a_1)(1+a_1) = 1 - a_1^2 \leq 1.
 \]
Montrons que l'inégalité à l'ordre $n$ entraine celle à l'ordre $n+1$.
\begin{multline*}
 M_{n+1}(1+S_{n+1})
 = M_n(1-a_{n+1})(1+S_n+a_{n+1})\\
 = M_n\left( \underset{\leq 1}{\underbrace{1-a_{n+1}^2}} + \underset{\leq 1}{\underbrace{(1-a_{n+1})}}S_n\right) 
 \leq M_n (1 + S_n) \leq 1.
\end{multline*}

\end{enumerate}

\item 
\begin{enumerate}
 \item En développant, l'inégalité $1+S_n \leq P_n$ est évidente car on néglige tous les autres termes du développement qui sont positifs. 
 \item On veut montrer par récurrence que, sous l'hypothèse $S_n<1$, $P_n(1-S_n)\leq 1$.\newline
\`A l'ordre $1$, l'inégalité est vérifiée:
\[
 P_1(1-S_1) = (1+a_1)(1-a_1) = 1 - a_1^2 \leq 1.
\]
Montrons que l'inégalité à l'ordre $n$ entraine celle à l'ordre $n+1$.
\begin{multline*}
 P_{n+1}(1-S_{n+1}) = P_n(1+a_{n+1})\left( 1- a_{n+1} - S_n\right) \\
 = P_n\left( (\underset{\leq 1}{\underbrace{1-a_{n+1}^2}}) -(\underset{> 1}{\underbrace{1+a_{n+1}}})S_n\right) 
 \leq P_n(1-S_n) \leq 1.
\end{multline*}
L'analogie entre les deux démonstrations est flagrante.\newline
En fait, on peut déduire directement la deuxième inégalité de la première.\newline
Pour tout $k\in\llbracket 1,n \rrbracket$, $a_k<1$ donc 
\[
 0\leq  1 - a_k^2 \leq 1 \Rightarrow M_n P_n \leq 1 \Rightarrow P_n \leq \frac{1}{S_n}. 
\]
D'autre part, d'après  l'inégalité de 1.a, comme $S_n<1$,
\[
 1 - S_n \leq M_n \Rightarrow \frac{1}{M_n}\leq \frac{1}{1-S_n} \Rightarrow P_n \leq \frac{1}{1-S_n}
\]
\end{enumerate}
 \item 
 \begin{enumerate}
  \item Voir le cours. L'inégalité de Cauchy-Schwarz résulte de ce que le discriminant du trinôme en $t$ est négatif ou nul car le trinôme ne prend que des valeurs positives.
  \item On applique l'inégalité de Cauchy-Schwarz avec des $n$-uplets bien choisis
\[
\forall i\;  \left. 
\begin{aligned}
 x_i &= \sqrt{a_i} \\ y_i &= \frac{1}{\sqrt{a_i}}
\end{aligned}
\right\rbrace \Rightarrow
n^2 = \left( \sum_{i=1}^n \sqrt{a_i}\frac{1}{\sqrt{a_i}}\right)^2
\leq \left( \underset{ = 1}{\underbrace{\sum_{i=1}^n \sqrt{a_i}^2}}\right) \left( \sum_{i=1}^n \frac{1}{a_i}\right).
\]
 \end{enumerate}
\end{enumerate}

\subsection*{Partie II. Images et fonctions.}
\begin{enumerate}
 \item Réduisons au même dénominateur
\[
 a+\frac{b}{z}+\frac{c}{1-z} = \frac{-az^2+(a-b+c)z+b}{z(1-z)}.
\]
Cela nous amène à considérer le système aux inconnues $u,v,w$:
\[
 (S) \;
 \left\lbrace 
\begin{aligned}
 -u =& 1 \\
 u -v + w =& 2 \\
 v =& 1
\end{aligned}
\right.  
\]
La réduction au même dénominateur montre que 
\[
 (a,b,c) \text{ solution de } (S) \Rightarrow a+\frac{b}{z}+\frac{c}{1-z} = \frac{(1+z)^2}{z(1-z)}.
\]
La résolution du système est immédiate. On en déduit
\[
 g(z) = \frac{(1+z)^2}{z(1-z)} = -1 + \frac{1}{z} + \frac{4}{1-z}.
\]

 \item D'après la décomposition précédente,
\[
 g'(z) = -\frac{1}{z^2} + \frac{4}{(1-z)^2} = \left(\frac{2}{1-z} + \frac{1}{z} \right) \left(\frac{2}{1-z} - \frac{1}{z} \right)
 = \frac{(z+1)(3z-1)}{z^2(1-z)^2}.
\]
On en déduit le tableau des variations avec $g(\frac{1}{3})=\frac{4^2}{2} = 8$
\begin{center}
\renewcommand{\arraystretch}{1.5}
\begin{tabular}{|c|ccccc|}\hline
     & $0$ &            & $\frac{1}{3}$ &            & $1$\\ \hline
$g'$ &     & -          & $0$           & +          & \\ \hline
$g$  &     & $\searrow$ &               & $\nearrow$ & \\
     &     &            & $8$           &            & \\ \hline
\end{tabular}
\end{center}

 \item Calcul de la dérivée de $f_z$:
\begin{multline*}
 f_z'(t) = K_z\left( -\frac{1}{t^2}(\frac{1}{1-z-t}-1) + (\frac{1}{t}-1)\frac{1}{(1-z-t)^2}\right)\\
 = \frac{K_z}{t^2(1-z-t)^2}\left( -(z+t)(1-z-t) +(1-t)t\right) 
 = \frac{K_z z \left(2t-(1-z)\right)}{t^2(1-z-t)^2}. 
\end{multline*}
On en déduit le tableau des variations avec 
\begin{center}
\renewcommand{\arraystretch}{1.5}
\begin{tabular}{|c|ccccc|}\hline
       & $0$ &            & $\frac{1-z}{2}$      &            & $1-z$\\ \hline
$f_z'$ &     & -          & $0$                  & +          & \\ \hline
$f_z$  &     & $\searrow$ &                      & $\nearrow$ & \\
       &     &            & $f_z(\frac{1-z}{2}) = g(z)$ &            & \\ \hline
\end{tabular}
\end{center}
car 
\[
f_z(\frac{1-z}{2})= \frac{1-z}{z}\left(\frac{2}{1-z}-1\right)^2  = \frac{(1+z)^2}{z(1-z)} = g(z).
\]

 \item
\begin{enumerate}
 \item cours: avec un $\exists$
 \item Pour un $z$ fixé, $x+y+1=1$ si et seulement si $y=1-z-x$ donc
\[
 \frac{(1-x)(1-y)(1-z)}{xyz} = f_z(x)
\]
On en déduit $\mathcal{P} = \bigcup_{z\in I}f_z(I_z)$ avec $f_z(I_z) = \left[ g(z), +\infty \right[$ d'après le tableau de la question 3. 
D'après le tableau de $g$:
\[
 \bigcup_{z\in I}\left[ g(z), +\infty\right[ = \left[ 8, +\infty \right[. 
\]

 \item En multipliant par $xyz$ on obtient
\[
 8 \leq \frac{(1-x)(1-y)(1-z)}{xyz} \Rightarrow 8xyz \leq (1-x)(1-y)(1-z).
\]
\end{enumerate}

\end{enumerate}


\subsection*{Partie III. Inégalité de Ky Fan.}
\begin{enumerate}
 \item Preuve de $\mathcal{F}_2$.
\begin{enumerate}
 \item On développe (les termes en $aba'b'$ se simplifient) puis on factorise:
\begin{multline*}
 (a+b)^2a'b'-(a'+b')^2ab = a^2ba'b' + b^2a'b' -a'^2ab - b'^2ab \\
 = aa'(ab'-a'b) + bb'(ba'-b'a) 
 = (ab'-a'b)(aa'-bb').
\end{multline*}
Si $a'=1-a$ et $b'=1-b$, cette relation devient
\[
 (a+b)^2(1-a)(1-b) -((1-a)+(1-b))ab = (a-b)^2(1-a-b).
\]
 \item D'après la question précédente, comme $1-a-b \geq 0$ pour $a$ et $b$ dans $\left] 0, \frac{1}{2}\right]$, 
\begin{multline*}
  (a+b)^2(1-a)(1-b) -((1-a)+(1-b))ab \geq 0 \\
  \Rightarrow  ((1-a)+(1-b))ab \leq (a+b)^2(1-a)(1-b)\\
 \Rightarrow \frac{ab}{(1-a)(1-b)} \leq \left( \frac{a + b}{(1-a) + (1-b)}\right)^2. 
\end{multline*}
Ce qui prouve l'inégalité de Ky Fan à l'ordre 2.
\end{enumerate}

 \item
\begin{enumerate}
 \item On veut montrer que $\mathcal{F}_{n}\Rightarrow \mathcal{F}_{2n}$. On se donne donc $2n$ nombres $a_i$ dans $\left] 0 , \frac{1}{2}\right]$. On les rassemble en deux groupes de $n$ et on introduit les objets sur lesquels portent les inégalités.
\begin{align*}
 &P_1 = \prod_{i=1}^{n}a_i,&    &S_1 = \sum_{i=1}^{n}a_i,&   &P_1' = \prod_{i=1}^{n}(1-a_i),&    &S_1' = \sum_{i=1}^{n}(1-a_i)\\ 
 &P_2 = \prod_{i=n+1}^{2n}a_i,& &S_2 = \sum_{i=n+1}^{n}a_i,& &P_2' = \prod_{i=n+1}^{2n}(1-a_i),& &S_2' = \sum_{i=n+1}^{2n}(1-a_i).
\end{align*}
Remarquons que les inégalités $a_i\leq \frac{1}{2}$ entrainent $S_1 \leq \frac{n}{2}$ et $S_2 \leq \frac{n}{2}$.\newline
Avec ces notations, on veut montrer
\[
 \frac{P_1\,P_2}{P_1'\,P_2'} \leq \left( \frac{S_1 + S_2}{S_1' + S_2'}\right)^{2n}. 
\]
On commence par appliquer l'inégalité KF à l'ordre $n$ pour chaque groupe
\[
 \frac{P_1\,P_2}{P_1'\,P_2'} \leq \left( \frac{S_1}{S_1'} \frac{S_2}{S_2'}\right)^{n}.
\]
Il suffit donc de montrer
\[
 \frac{S_1\,S_2}{S_1'\,S_2'} \leq \left( \frac{S_1 + S_2}{S_1' + S_2'}\right)^2. 
\]
C'est équivalent à la positivité d'une expression analogue à celle de 1.a.
\begin{multline*}
(S_1 + S_2)^2S_1'S_2' - (S_1'+ S_2')S_1S_0
 = (S_1S_2' - S_2 S_1')(S_1S_1' - S_2S_2')\\
 = n(S_1-S_2)^2(n-S_1-S_2)\text{ car }S_i' = n - S_i \\
 \geq 0 \text{ car $S_1$ et $S_2$ sont } \leq \frac{n}{2}.
\end{multline*}

 \item On veut montrer $\mathcal{F}_{n+1} \Rightarrow \mathcal{F}_{n}$. On se donne donc des $a_1, \cdots,a_{n}$ dans $\left]0, \frac{1}{2}\right]$.\newline
 Pour pouvoir exploiter l'hypothèse $\mathcal{F}_{n+1}$, on a besoin d'un $n+1$-ième nombre. Comme l'énoncé nous y invite, on considère
 \[
  a_n = \frac{S_n}{n}. 
 \]
Il est dans $\left]0, \frac{1}{2}\right]$ car c'est la moyenne des $a_i$ déjà définis qui y sont déjà.\newline
Utilisons des notations simplifiée analogues à celles définies dans la question précédente pour $S = S_n'$, $P = P_n$, $P' = P_n'$ et $a = a_{n+1}=\frac{S}{n}$.\newline
Appliquons KF à l'ordre $n+1$ avec la famille ainsi complétée.
\[
 \frac{P a}{P'(1-a)}
 \leq \left( \frac{S + a}{S' + 1 - a}\right)^{n+1} 
\]
Remplaçons $a$ par $\frac{S}{n}$ en remarquant que 
\[
 S+a = \frac{(n+1)S}{n}, \; 1-a = 1 - \frac{S}{n} = \frac{n-S}{n} = \frac{S'}{n}, \;
 S' + 1 -a = \frac{(n+1)S'}{n}.
\]
Il vient:
\[
 \frac{nPS}{nP'S'} \leq \left( \frac{(n+1)Sn}{n(n+1)S'}\right)^{n+1} 
 \Rightarrow 
 \frac{P}{P'} \leq \left( \frac{S}{S'}\right)^{n}. 
\]

 \item D'après $\mathcal{F}_2$ et $\mathcal{F}_n \Rightarrow \mathcal{F}_{2n}$, la propriété $\mathcal{F}_n$ est vérifiée par récurrence pour tous les entiers $n$ qui sont des puissances de $2$. Si $n$ n'est pas une puissnce de $2$, il existe $p$ tel que $n < 2^p$ et on peut déduire $\mathcal{F}_n$ de $\mathcal{F}_{2^p}$ et de $\mathcal{F}_{n+1}\Rightarrow \mathcal{F}_{n}$. 
\end{enumerate}

 
 \item
\begin{enumerate}
 \item On doit caractériser \og combien petit doit être $\lambda$\fg. Il doit être plus petit que tous c'est à dire plus petit que le plus petit.
\[
 \left( \forall i \in \llbracket 1,n \rrbracket, \; \lambda a_i \leq \frac{1}{2}\right) 
 \Leftrightarrow \lambda \leq \min A.
\]

 \item D'après la question précédente, on peut trouver des $\lambda$ assez petit pour que l'on puisse appliquer KF aux $\lambda a_i$. Appliquons la avec les notations 
\[
P = a_1 \cdots a_n,\;  S = a_1 + \cdots + a_n,\;  P_\lambda' = (1-\lambda a_1)\cdots(1-\lambda a_n).
\]
On obtient
\[
 \frac{\lambda^n P}{P_\lambda'} \leq \left( \frac{\lambda S}{n - \lambda S} \right)^n 
 \Rightarrow 
 \frac{P}{P_\lambda'} \leq \left( \frac{ S}{n - \lambda S} \right)^n 
\]
Pour $n$ et les $a_i$ fixés, quand $\lambda \rightarrow 0$, on a $P_\lambda' \rightarrow 1$ et $n - \lambda S \rightarrow n$. En passant à la limite dans l'inégalité, on obtient
\[
 P \leq \left( \frac{S}{n}\right)^n .
\]
la moyenne géométrique est inférieure ou égale à la moyenne arithmétique.
\end{enumerate}

\end{enumerate}
