\subsection*{Partie 1. Définitions - Exemples}
\begin{enumerate}
 \item \begin{enumerate}
 \item Comme la paramétrisation est normale, $z'(s)$ est l'affixe complexe de $\overrightarrow \tau (s)$. Le vecteur $\overrightarrow n$ est obtenu à partir de $\overrightarrow \tau$ par une rotation de $\frac{\pi}{2}$ qui se traduit par la multiplication par $i$ pour les affixes. La relation (2) est donc simplement la traduction complexe de la définition de la courbure par :
\begin{displaymath}
 \overrightarrow \tau ' = c \overrightarrow n
\end{displaymath}
Les deux relations sont donc équivalentes
\item Lorsque $c_0$ est un réel fixé, cherchons les solutions de l'équation différentielle
\begin{displaymath}
 z''=ic_0z'
\end{displaymath}
Si $c_0=0$, l'équation devient $z''=0$ dont les solutions sont de la forme
\begin{displaymath}
 s \rightarrow as +b
\end{displaymath}
avec $a$ et $b$ complexes fixés ($a$ de module $1$) pour que la paramétrisations soit normale. Le support de ces paramétrisations sont des droites.\newline
Si $c_0\neq 0$. D'après le cours sur les équations différentielles linéaires du premier ordre :
\begin{displaymath}
 z'(s)=z'(0)e^{ic_0s} \Rightarrow z(s)=z(0) - i\dfrac{z'(0)}{c_0}e^{ic_0s}
\end{displaymath}
Le support d'une telle courbe paramétrée est un cercle.\newline
En conclusion, le support d'une courbe paramétrée dont tous les points sont des sommets est une droite ou un cercle.

\item D'après le cours sur les équations différentielles linéaires du premier ordre, les solutions de
\begin{displaymath}
 y'(s) = \dfrac{i}{1+s^2}y
\end{displaymath}
 sont les fonctions de la forme
\begin{displaymath}
 \lambda e^{iF}
\end{displaymath}
où $F$ est une primitive de $s\rightarrow \frac{1}{1+s^2}$. Une telle primitive est $\arctan$. En tenant compte de la condition initiale $z'(0)=1$ on obtient donc
\begin{displaymath}
 z'(s)=e^{i\arctan s}
\end{displaymath}
Pour $\theta$ dans $]-\frac{\pi}{2}, \frac{\pi}{2}[$ (ce qui est le cas si $\theta=\arctan s$), on a
\begin{align*}
 \cos \theta = \dfrac{1}{\sqrt{\tan^2 \theta}}  & & 
\sin \theta = \dfrac{\tan \theta}{\sqrt{\tan^2 \theta}}
\end{align*}
Avec $\theta=\arctan s$, on déduit :
\begin{displaymath}
 z'(s)=\dfrac{1+is}{\sqrt{1+s^2}}
\end{displaymath}
Le calcul de $z$ revient à un calcul d'intégrale :
\begin{displaymath}
 z(s)= \underset{=z(0)}{i}
+\int_{0}^{s}\dfrac{1+iu}{\sqrt{1+u^2}} du
\end{displaymath}
On effectue un changement de variable $u=\sh t$ et on obtient
\begin{displaymath}
 z(s)= \argsh s +i(s-1)
\end{displaymath}
Il apparait alors que le support est le même que celui de la paramétrisation
\begin{displaymath}
 t \rightarrow 1 +i(\ch t -1)
\end{displaymath}
obtenue avec le changement de paramètre admissible $t=\sh s$.\newline
Ce support est une courbe appelée \emph{chaînette} qui ressemble de loin à une parabole mais n'en n'est pas une.
\end{enumerate}
 
\item 
\begin{enumerate}
 \item Pour la parabole, les calculs sont immédiats :
\begin{align*}
 \overrightarrow f'(t)=& \overrightarrow i + \dfrac{t^2}{2p}\overrightarrow j \\
\overrightarrow f''(t)=&  \dfrac{1}{p}\overrightarrow j 
\end{align*}
On en déduit que la paramétrisation n'est pas normale. La vitesse n'est pas de norme $1$ mais
\begin{align*}
 \Vert  \overrightarrow f'(t)\Vert =& \dfrac{\sqrt{p^2+t^2}}{p}\\
\overrightarrow{\tau}(t) =& \dfrac{p}{\sqrt{p^2+t^2}}\left( \overrightarrow i + \dfrac{t}{p}\overrightarrow j\right) \\
\overrightarrow{n} (t) =& \dfrac{p}{\sqrt{p^2+t^2}}\left(- \dfrac{t}{p}\overrightarrow i+ \overrightarrow i\right) 
\end{align*}
L'équation de la tangente en $f(t)$ est
\begin{displaymath}
 \det(\overrightarrow{f(t)M},\overrightarrow f'(t))=0
\Leftrightarrow
\begin{vmatrix}
 x-t & 1 \\
y-\dfrac{t^2}{2p} & \dfrac{t}{p} 
\end{vmatrix}
=0
\end{displaymath}

\item Le calcul de la courbure se fait à l'aide du déterminant :
\begin{displaymath}
 \gamma(f(t))=
\dfrac{\det(\overrightarrow f'(t),\overrightarrow f''(t))}{\Vert \overrightarrow f'(t)\Vert}
=p^2(p^2+t^2)^{-\frac{3}{2}}
\end{displaymath}
La dérivée de cette fonction
\begin{displaymath}
 -3p^2t(p^2+t^2)^{-\frac{5}{2}}
\end{displaymath}
s'annule seulement pour $t=0$ qui correspond à l'origine et au sommet de la parabole.\newline
Rappelons que pour une conique, le terme sommet désigne un point d'intersection avec l'axe focal.
\end{enumerate}

\item 
\begin{enumerate}
 \item Pour l'ellipse, les calculs sont immédiats :
\begin{align*}
 \overrightarrow f'(t)=& -a\sin t \overrightarrow i + b\cos t \overrightarrow j \\
\overrightarrow f''(t)=& -a\cos t \overrightarrow i - b\sin t \overrightarrow j 
\end{align*}
On en déduit que la paramétrisation n'est pas normale. La vitesse n'est pas de norme $1$ mais
\begin{align*}
 \Vert  \overrightarrow f'(t)\Vert =& \sqrt{a^2\sin^2 t+ b^2\cos^2 t}\\
\overrightarrow{\tau}(t) =& \dfrac{1}{\sqrt{a^2\sin^2 t+ b^2\cos^2 t}}
\left( -a\sin t \overrightarrow i + b\cos t \overrightarrow j\right) \\
\overrightarrow{n} (t) =& \dfrac{1}{\sqrt{a^2\sin^2 t + b^2\cos^2 t}}
\left(-b\cos t \overrightarrow i + a\sin t \overrightarrow j\right) 
\end{align*}
L'équation de la tangente en $f(t)$ est
\begin{displaymath}
 \det(\overrightarrow{f(t)M},\overrightarrow f'(t))=0
\Leftrightarrow
\begin{vmatrix}
 x-a\cos t & -a\sin t \\
y-b\sin t & b\cos t 
\end{vmatrix}
=0
\end{displaymath}

\item Le calcul de la courbure se fait à l'aide du déterminant :
\begin{displaymath}
 \gamma(f(t))=
\dfrac{\det(\overrightarrow f'(t),\overrightarrow f''(t))}{\Vert \overrightarrow f'(t)\Vert}
=ab(a^2\sin^2 t +b^2\cos^2 t)^{-\frac{3}{2}}
\end{displaymath}
La dérivée de cette fonction
\begin{displaymath}
 -3ab(a^2-b^2)\sin t \cos t(a^2\sin^2 t +b^2\cos^2 t)^{-\frac{5}{2}}
\end{displaymath}
s'annule seulement pour $t$ congru à $0$ modulo $\frac{\pi}{2}$. Il existe donc quatre sommets qui correspondent aux intersection avec les deux axes de symétrie. Pour une conique les intersections avec le petit-axe ne sont pas habituellement appelées des sommets.
\end{enumerate}
\end{enumerate}

\begin{figure}[h!t]
 \centering
 \input{Csommets_1.pdf_t}
 \caption{Droites passant par $M(w)$}
 \label{fig:Csommets_1}
\end{figure}

\subsection*{Partie 2. Courbe fermée strictement convexe}
\begin{enumerate}
 \item 
\begin{enumerate}
 \item La fonction $Y$ est continue, les valeurs $Y(u)$ et $Y(v)$ sont de signes opposés. D'après le théorème des valeurs intermédiaires, il existe $w$ entre $u$ et $v$ tel que $Y(w)=0$.\newline
Par définition, $M(w)$ est sur la droite $(M_1,M_2)$ et même sur le segment $[M_1,M_2]$. Considérons toutes les droites $D$ passant par $M(w)$ (Fig. \ref{fig:Csommets_1}). Parmi  ces droites doit se trouver la tangente en $M(w)$.
\begin{itemize}
 \item Si $D$ n'est pas $(M_1,M_2)$. Les deux points $M_1$ et $M_2$ ne sont pas dans le même des deux demi-plans ouverts définis par $D$. Chacun est dans un (différent) des deux.
\item Si $D$ est  $(M_1,M_2)$. Aucun des des deux points $M_1$ et $M_2$ n'est dans un des demi-plans \emph{ouverts} définis par $D$.
\end{itemize}
Or, d'après la convexité de la courbe, lorsque $D$ est la tangente en $M(w)$, les deux points devraient se trouver dans le même des deux demi-plans ouverts définis par la tangente. Il est donc impossible que $Y$ change de signe entre $s_1$ et $s_2$.
\item Montrer que tous les points $M(s)$ avec $s_1<s<s_2$ se trouvent dans le même des deux demi-plans définis par la droite $(M_1,M_2)$ est une reformulation de la question précédente. S'ils ne s'y trouvaient pas, la fonction $Y$ changerait de signe.\newline
On peut raisonner entre $s_2$ et $s_1+L$ comme entre $s_1$ et $s_2$. (Fig. \ref{fig:Csommets_2}) Il faut toutefois jistifier que la deuxième portion de courbe n'est pas du même côté de la droite. Cela se fait en considérant ce qui se passe autour de $M_2$. Comme $(M_1,M_2)$ ne peut être tangente, la tangente en $M_2$ traverse la droite $(M_1,M_2)$. Pour $s>s2$ le point se trouvera donc de l'autre côté.
\begin{figure}[ht]
 \centering
 \input{Csommets_2.pdf_t}
 \caption{Intersection avec une corde}
 \label{fig:Csommets_2}
\end{figure}
\end{enumerate}

\item 
\begin{enumerate}
 \item La fonction $c$ est continue et périodique de période $L$. Sa restriction à un segment de longueur $L$ est donc bornée et atteint ses bornes. Comme l'ensemble des valeurs prises par la fonction est aussi l'ensemble des valeurs prises par une restriction à un segment de longueur $L$, la fonction complète est donc bornée et atteint ses bornes.\newline
Notons $s_1$ un réel tel que
\begin{displaymath}
 c(s_1)=\min\left\lbrace c(s), s\in \R\right\rbrace 
\end{displaymath}
Comme la restriction de $c$ à $[s_1,s_1+L]$ atteint sa borne supérieure, il existe alors un $s_2\in[s_1,s_1+L]$ tel que
\begin{displaymath}
 c(s_2)=\max\left\lbrace c(s), s\in \R\right\rbrace 
\end{displaymath}
Les réels $s_1$ et $s_2$ réalisent des extréma absolus de la fonction $c$ et la fonction $c$ est dérivable dans $\R$. On en déduit que $c'(s_1)=c'(s_2)=0$. Les points $M(s_1)$ et $M(s_2)$ sont des sommets de la courbe.
\item On suppose dans cette question que $c'$ ne s'annule qu'en $s_1$ et $s_2$. Elle garde donc un signe constant dans $]s_1,s_2[$ et dans $]s_2,s_1+L[$. Comme $s_1$ réalise le minimum de $c$ et $s_2$ le maximum, la fonction est croissante sur $]s_1,s_2[$ et décroissante sur $]s_2,s_1+L[$. Les signes se combinent alors pour que l'intégrale (fonctions continues) soit positive.
\begin{displaymath}
 \int_{s_1}^{s_1+L}c'(s)Y(s)ds
= \int_{s_1}^{s_2}\underset{>0}{c'(s)}\underset{>0}{Y(s)}ds 
+ \int_{s_2}^{s_1+L}\underset{<0}{c'(s)}\underset{<0}{Y(s)}ds
> 0
\end{displaymath}
En fait on va montrer que cette intégrale est nulle ce qui conduit évidemment à une contradiction et à l'existence d'une troisième valeur $s_3$ où $c'$ s'annule.\newline
Le calcul utilise une intégration par parties et la définition de la courbure. Commençons par la courbure. On exprime la vitesse dans le repère indiqué par l'énoncé :
\begin{displaymath}\left. 
 \begin{aligned}
\overrightarrow \tau (s) =& X'(s)\overrightarrow i + Y'(s)\overrightarrow j \\ 
\overrightarrow n (s) =& -Y'(s)\overrightarrow i + X'(s)\overrightarrow j \\
\overrightarrow \tau '(s) =& c(s) \overrightarrow n(s)
 \end{aligned}
\right\rbrace 
\Rightarrow
\left\lbrace 
\begin{aligned}
 X''(s) =& -c(s)Y'(s)\\ \\
 Y''(s) =& c(s)X'(s)
\end{aligned}
\right. 
\end{displaymath}
Utilisons maintenant l'intégration par parties 
\begin{multline*}
\int_{s_1}^{s_1+L}c'(s)Y(s)ds = \underset{=0 \;\text{ par périodicité}}{\underbrace{\left[ c(s)Y(s)\right]_{s_1}^{s_1+L}}}
- \int_{s_1}^{s_1+L}c(s)Y'(s)ds  \\
= \int_{s_1}^{s_1+L}X''(s)ds = \left[ X'(s)\right]_{s_1}^{s_1+L} = 0 \;\text{ par périodicité} 
\end{multline*}

\begin{figure}[ht]
 \centering
\input{Csommets_3.pdf_t}
\caption{Graphe de $c$}
\label{fig:Csommets_3}
\end{figure}

\item On a montré à la question précédente l'existence d'un $s_3$ en lequel $c'$ s'annule. Si $c'$ ne change pas de signe en $s_3$, le raisonnement de la question précédente sur la stricte positivité de l'intégrale s'applique encore en contradiction avec le calcul qui montre qu'elle est nulle. La fonction $c'$ doit donc changer de signe en $s_3$ ce qui prouve que $s_3$ est un extrémum relatif pour $c$.\newline
Comme $s_1$ est le minimum absolu, $s_3$ ne peut être qu'un maximum relatif si $c'$ ne change pas de signe avant. Il existe alors forcément un minimum relatif $s_4$ entre $s_3$ et le maximum absolu $s_1+L$(Fig. \ref{fig:Csommets_3}).\newline
Ceci prouve l'existence d'un quatrième sommet.
\end{enumerate}
\end{enumerate}
\clearpage