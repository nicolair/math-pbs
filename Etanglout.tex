%<dscrpt>Calcul approché des valeurs de tan.</dscrpt>
\begin{enumerate}
\item Approximation de $\tan x$.\newline
Soit $a$ et $b$ deux réels fixés dans $]0,\frac{\pi}{4}[$, soit $x$ un réel quelconque dans $]0,\frac{\pi}{4}[$, montrer que dans
\[\{\tan(a+nb),n\in \Z\}\]
il existe un $u$ tel que $u\leq \tan x < u+2b$.
\item Mise en oeuvre matricielle.\newline
Soit $b\in ]0,\frac{\pi}{4}[$ et $\lambda$ un réel non nul quelconque, on pose $k=\tan b$ et on définit deux suites réelles $(x_n)_{n\in\N}$, $(y_n)_{n\in\N}$ par :
\begin{eqnarray*}
  \begin{pmatrix}
       x_{n+1}\\y_{n+1}
    \end{pmatrix} &=&
    \begin{pmatrix}
       1 & k \\ -k & 1
    \end{pmatrix}
    \begin{pmatrix}
       x_{n}\\y_{n}
    \end{pmatrix}  \\
  x_0 &=& \lambda \sin a \\
  y_0 &=& \lambda \cos a
\end{eqnarray*}
Calculer $\frac{x_n}{y_n}$
\item Intérêt d'une approche gloutonne.\newline
On considère trois nombres $b$, $b_0$, $L$ tels que
\[0<b_1<b_0<L\]
On considère les propositions suivantes :
\begin{quotation}
Pour tout $x\in [0,L]$, il existe un unique $n(x)$ entier tel que $x=n(x)b+r$ avec $0\leq r<b$. \newline
Pour tout $x\in [0,L]$, il existe un unique couple $(n_0(x),n_1(x))$ d'entiers tels que $x=n_0(x)b_0+r_0$ avec $0\leq r_0 < b_0$ et $r_0=n_1(x)b+r$ avec $0\leq r < b$.
\end{quotation}
\begin{enumerate}
\item Justifier ces propositions en précisant les entiers $n(x)$, $n_0(x)$, $n_1(x)$.
\item Calculer les intégrales suivantes pour $L=1$, $b_0=0.1$, $b=0.01$.
\[ \frac{1}{L}\int_0^L n(x)\,dx ,\medskip \frac{1}{L}\int_0^L (n_0(x)+n_1(x))\,dx
\]
\item Présenter una algorithme donnant une approximation par défaut de $\tan x$ pour $x \in [0,\frac{\pi}{4}]$ à 0.05 près en mons de 10 itérations.\newline
Comment s'interprètent les calculs de la question précédente ?
\end{enumerate}
\end{enumerate}
