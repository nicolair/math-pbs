%<dscrpt>Une suite d'intégrales.</dscrpt>
Pour tout entier naturel $n$, on définit une fonction $f_{n}$ sur $]0,1]$ par
\begin{displaymath}
\forall x\in ]0,1] : f_{n}(x)=\frac{1+x^{n}}{(1+x^{n+1})\sqrt{x}}
\end{displaymath}
Pour tout $a\in ]0,1]$, la restriction à $[a,1]$ de $f_n$ est clairement continue. On définit une fonction $J_n$  dans $]0,1]$ par :
\begin{displaymath}
 \forall a \in ]0,1] : J_n(a) = \int_a^1 f_n(x)dx
\end{displaymath}

\begin{enumerate}
\item On définit une fonction $g_n$ dans $[0,1]$ par:
\begin{displaymath}
 g_n(x)=
\left\lbrace 
\begin{aligned}
 &f_n(x) &\text{ si } &x\in ]0,1] \\
 &0 &\text{ si } &x=0
\end{aligned}
\right. 
\end{displaymath}
La fonction $g_n$ est-elle continue par morceaux dans $[0,1]$ ? Que peut-on en conclure ?
\item Montrer que, pour tout $a\in ]0,1]$, la suite $(J_{n}(a))_{n\in\N}$ est monotone. En déduire qu'elle est convergente. On note $J(a)$ sa limite.
\item Montrer que pour tout entier $n$, la fonction $J_n$ est monotone. Préciser son sens de variation. Montrer que, pour tout entier $n$, la fonction $J_n$ admet en $0$ une limite finie (notée $j_n$).
\item Dans cette question plus particulièrement, on citera très précisément les théorèmes utilisés
\begin{enumerate}
\item Montrer que
\begin{displaymath}
 \forall x\in ]0,1] : \dfrac{1}{\sqrt{x}}\leq f_n(x) \leq \dfrac{1+x^n}{\sqrt{x}}
\end{displaymath}
\item Pour $a\in]0,1]$, calculer $J(a)$.
\item Montrer que la suite $(j_n)_{n\in \N}$ converge vers 2.
\end{enumerate}
\item Montrer que 
\begin{displaymath}
 0 \leq j_n -2 \leq \dfrac{1}{(n+\frac{1}{2})(n+\frac{3}{2})}
\end{displaymath}
\end{enumerate}

