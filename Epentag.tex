%<dscrpt>Construction d'un pentagone régulier.</dscrpt>
Cet exercice porte sur la construction d'un pentagone régulier \footnote{D'après E.N.S.A.I.S 2005}.

Soit $\omega$ un nombre complexe tel que $\omega \neq 1$ et $\omega^5 = 1$. On considère les deux nombres complexes $\alpha$ et $\beta$ définis par :
\begin{align*}
 \alpha &= \omega + \frac{1}{\omega}  &   \beta &= -1 -\alpha 
\end{align*}
\begin{enumerate}
 \item 
 \begin{enumerate}
  \item Montrer que 
\begin{displaymath}
 \overline{\omega} = \frac{1}{\omega} = \omega^4
\end{displaymath}
Former une relation analogue pour $\omega^2$ au lieu de $\omega$. Que peut-on en déduire pour $\omega + \omega^4$ et $\omega^2 + \omega^3$?
  \item Montrer que 
  \begin{displaymath}
   1 + \omega + \omega^2 + \omega^3 + \omega^4 = 0
  \end{displaymath}
 \end{enumerate}

 \item 
 \begin{enumerate}
  \item Montrer que $\alpha$ et $\beta$ sont réels en exprimant $\alpha$ à l'aide d'une partie réelle.
  \item Simplifier $\alpha + \beta$ et $\alpha \beta$. En déduire une équation simple du second degré dont les solutions sont $\alpha$ et $\beta$.
 \end{enumerate}
  
\item Préciser le centre et les intersections avec les axes du cercle d'équation
\begin{align*}
 x^2 +y^2 +x -1 =0
\end{align*}

\item (En utilisant les résultats de terminale.) On suppose
\begin{displaymath}
 \omega = e^{\frac{2i\pi}{5}} = \cos \frac{2i\pi}{5} + i \sin \frac{2i\pi}{5}
\end{displaymath}
Montrer que les 3 points situés sur le cercle de centre 0 et de rayon 2 et dont les abscisses sont respectivement $\alpha$, $\beta$ et $2$  sont des sommets d'un pentagone régulier. En déduire une construction à la règle et au compas d'un pentagone régulier.
\end{enumerate}
