\begin{enumerate}
  \item On peut traduire les relations barycentriques par des {\'e}galit{\'e}s
  vectorielles dans la base $(\overrightarrow{AB},
  \overrightarrow{AC})$. Ces {\'e}galit{\'e}s donnent les coordonn{\'e}es des
  points dans le rep{\`e}re $(A,\overrightarrow{AB},
  \overrightarrow{AC})$. On obtient respectivement
  \begin{eqnarray*}
(\frac{\beta}{\alpha+\beta+\gamma},\frac{\gamma}{\alpha+\beta+\gamma})\\
(\frac{\beta'}{\alpha'+\beta'+\gamma'},\frac{\gamma'}{\alpha'+\beta'+\gamma'})\\
(\frac{\beta''}{\alpha''+\beta''+\gamma''},\frac{\gamma''}{\alpha''+\beta''+\gamma''})\\
  \end{eqnarray*}
  pour les points $X,X',X''$.

  D'apr{\`e}s le cours, la condition assurant l'alignement de trois
  points est la nullit{\'e} du d{\'e}terminant form{\'e} par les coordonn{\'e}es
  {\'e}crites en ligne et compl{\'e}t{\'e}es par une colonne de 1. En
  multipliant chaque ligne par $\alpha+\beta+\gamma$, on obtient
  la condition
  \[\left|\begin{array}{ccc}
    \alpha & \beta & \alpha+\beta+\gamma \\
    \alpha' & \beta' & \alpha'+\beta'+\gamma' \\
    \alpha'' & \beta'' & \alpha''+\beta''+\gamma''
  \end{array}\right| = 0\]
  On peut soustraire les deux premi{\`e}res colonnes {\`a} la troisi{\`e}me,
  la condition est donc
  \[\left|\begin{array}{ccc}
    \alpha & \beta & \gamma \\
    \alpha' & \beta' & \gamma' \\
    \alpha'' & \beta'' & \gamma''
  \end{array}\right| = 0\]

  \item
    \begin{enumerate}
      \item En introduisant respectivement $A"",B'',C''$ dans les
      parties droites des relations d{\'e}finissant ces points, on
      peut former les relations suivantes
\begin{eqnarray*}
\overrightarrow{AA''}+\overrightarrow{A''B'}+\overrightarrow{A''C'}=0\\
\overrightarrow{B B''}+\overrightarrow{B''C'}+\overrightarrow{B''A'}=0\\
\overrightarrow{CC''}+\overrightarrow{C''A'}+\overrightarrow{C''B'}=0\\
\end{eqnarray*}
 qui sont barycentriques. On en d{\'e}duit que $A'',B'',C''$ sont respectivement les
 barycentres de
 \begin{eqnarray*}
 ((A,-1),(B',1),(C',1))\\((A',-1),(B,-1),(C',1))\\((A',1),(B',1),(C,-1))
 \end{eqnarray*}
 Par associativit{\'e}, on obtient finalement que $A'',B'',C''$ sont respectivement les
 barycentres de $A,B,C$ avec les coefficients
\begin{eqnarray*}
  (-\beta+\gamma,1-\gamma,\beta)\\
  (\gamma,\alpha-\gamma,1-\alpha)\\
  (1-\beta,\alpha),-\alpha+\beta)
\end{eqnarray*}
La condition d'alignement est la nullit{\'e} d'un d{\'e}terminant, on le
transforme en remarquant que la somme des termes d'une ligne est 1
puis on d{\'e}veloppe suivant la premi{\`e}re colonne.
\begin{eqnarray*}
\left|\begin{array}{ccc}
    -\beta+\gamma & 1-\gamma & \beta\\
  \gamma & \alpha-\gamma & 1-\alpha\\
  1-\beta & \alpha & -\alpha+\beta
  \end{array}\right|  & =&
\left|\begin{array}{ccc}
    1 & 1-\gamma & \beta\\
  1 & \alpha-\gamma & 1-\alpha\\
  1 & \alpha & -\alpha+\beta
  \end{array}\right|\\
  &=&\sigma_2- \sigma_1 +1
\end{eqnarray*}
avec
\[\sigma_1=\alpha+\beta+\gamma,\sigma_2=\alpha\beta+\alpha\gamma+\beta\gamma\]
      \item Les {\'e}galit{\'e}s d{\'e}finissant $A',B',C'$ permettent
      d'exprimer ces points comme des barycentres de $A,B,C$. On utilise
      la question 1 pour traduire l'alignement
\begin{eqnarray*}
\left|\begin{array}{ccc}
    0 & \alpha & 1-\alpha\\
  1-\beta & 0 & \beta\\
  \gamma & 1-\gamma & 0
  \end{array}\right|  & =&
\left|\begin{array}{ccc}
    1 & \alpha & 1-\alpha\\
  1 &  0 & \beta\\
  1 & 1-\gamma & 0
  \end{array}\right|\\
  &=&\sigma_2- \sigma_1 +1
\end{eqnarray*}
Les deux conditions sont identiques. Les points $A'',B'',C''$ et
$A',B',C'$ sont simultan{\'e}ment align{\'e}s.
    \end{enumerate}
\end{enumerate}
