%<dscrpt>Une transformation proche de l'inversion.</dscrpt>
\'Etant donnés trois points $A$, $B$, $M$ respectivement d'affixe $a$, $b$, $m$ (avec $a\neq b$), l'objet de cet exercice est étudier la construction du point $M'$ d'affixe $m'$ vérifiant
\begin{displaymath}
 \frac{m'-a}{m'-b} + \frac{m-a}{m-b} = 0
\end{displaymath}
On considère le milieu $C$ de $[A,B]$, son affixe est notée $c$.
\begin{enumerate}
 \item En introduisant $c$ dans la relation définissant $m'$, former une expression simple pour $(m-c)(m'-c)$.
 \item En utilisant le théorème de l'arc capable, montrer que $M'$ est sur le cercle circonscrit à $A$, $B$, $M$. Quelle précision supplémentaire peut-on donner ?
 \item Dans le cas particulier où $a=1$ et $b=-1$, que devient la relation trouvée en 1. Comment peut-on construire $M'$ ?
 \item Montrer qu'il existe une similitude directe $S$ telle que l'affixe de $s(A)$ soit $1$ et celle de $s(B)$ soit $-1$. En déduire une construction de $M'$ dans le cas général.
\end{enumerate}
