\subsection*{Partie I. Propriétés de la famille de polynômes.}
\begin{enumerate}
\item Soit  $h>-1$. La suite  $\left((\ln(1+h)^{n+1}\right) _{n\in \N}$ est donc bien définie. Elle est bornée si et seulement si $|\ln(1+h)|\ie 1$ c'est à dire
\begin{displaymath}
\frac 1e-1\ie h\ie e-1 
\end{displaymath}

\item 
\begin{enumerate}
\item \label{2a} Pour $k\in\N^*$, les entiers $0, 1, \dots, k-1$ sont les racines de $\Gamma_k$ donc $\Gamma_k(x)=0$ pour $x\in \llbracket 0, k-1\rrbracket$. \newline
Pour $x\in \N \setminus \llbracket 0, k-1\rrbracket$, $\Gamma_k(x)=\frac{1}{k!}x(x-1)\dots(x-k+1)=\binom{x}{k}$\newline
Pour $x\in\Z\setminus \N$,
\begin{multline*}
\Gamma_k(x)=\frac{1}{k!}x(x-1)\dots(x-k+1)\\ =\frac{(-1)^k}{k!}(-x)(-x+1)\dots(-x+k-1)
=(-1)^k \binom{-x+k-1}{k} 
\end{multline*}
En particulier, on trouve $\Gamma_k(k)=\binom{k}{k}=1$ et $\Gamma_k(-1)=(-1)^k\binom{1+k-1}{k}=(-1)^k$.
\item Soit $x\in\R_+\setminus\N$ et $n\in\N$, alors $\Gamma_n(x)\neq 0$ et
\begin{displaymath}
 \frac{\Gamma_{n+1}(x)}{\Gamma_n(x)}=-\frac{x-n}{n+1}\se 0 \Leftrightarrow\quad \text{ssi}\quad n\se x
\end{displaymath}
La suite est donc de signe constant à partir de $n=\lceil x \rceil$.
\end{enumerate}
\item
\begin{enumerate}
\item Soit $i\in\llbracket 1, n\rrbracket$.\
 Pour $k\in\llbracket i+1, n\rrbracket$ $\Gamma_k(i)=0$ et comme démontré à la question \ref{2a}. D'où

\begin{displaymath}
 \sum_{k=0}^n(-1)^k\Gamma_k(i)=\sum_{k=0}^i(-1)^k\Gamma_k(i)=\sum_{k=0}^i (-1)^k\binom{i}{k}=0 
\end{displaymath}
\item Notons $P= \sum_{k=0}^n(-1)^k\Gamma_k$. Pour $k\in \N$, $\Gamma_k$ est de degré $k$. Le polynôme $P$ est donc de degré $n$ et de coefficient dominant $\frac{(-1)^n}{n!}$ qui est le coefficient dominant de $(-1)^n\Gamma_n$.\newline
D'autre part, d'après la question précédente les $i\in\llbracket 0, n\rrbracket$ sont racines de $P$. Comme $P$ est de degré $n$, ce sont exactement toutes ces racines et elles sont de multiplicité $1$.\newline
On en déduit que $$P=\frac{(-1)^n}{n!}(X-1)(X-2)\dots(X-n)$$ on reconnait $(-1)^n \,\widehat{\Gamma_n}(X-1)$ D'où
\begin{displaymath}
 \sum_{k=0}^n(-1)^k\Gamma_k = (-1)^n \,\widehat{\Gamma_n}(X-1)
\end{displaymath}
\end{enumerate}
\item
\begin{enumerate}
\item Calculons le développement en regroupant sous un même logarithme
\begin{multline*}
\ln\mu_n-\ln \mu_{n-1} = \ln\frac{\mu_n}{\mu_{n-1}} 
 = \ln\left(\frac n{n-1}\right)^ \rho\left|\frac{\Gamma_n(u)}{\Gamma_{n-1}(u)}\right|\\
= \ln\left(\frac n{n-1}\right)^ \rho\left|\frac{u-n+1}{n}\right|
= -\rho\ln\left(1-\frac 1n\right)+\ln\left(1-\frac{u+1}n\right)\quad\text{pour}\ n\se u+1\\
= -\rho\left(-\frac 1n-\frac{1}{2n^2} +o\left(\frac1{n^2}\right)\right)+ \left(-\frac {u+1}n-\frac{(u+1)^2}{2n^2} +o\left(\frac1{n^2}\right)\right)\\
= \frac {\rho-(u+1)}n+\frac{\rho+(u+1)^2}{2n^2} +o\left(\frac1{n^2}\right)
\end{multline*}

\item Pour $\rho=u+1$, comme admis dans l'énoncé, la suite $\left( \ln(\mu_n)\right) _{n\in \N}$ est convergente de limite $l(u)$, donc la suite $\left((\mu_n\right) _{n\in \N}$ est convergente de limite $e^{l(u)}\neq 0$ et donc $$\mu_n\sim e^{l(u)}$$
D'où
$$n^{u+1} \left|\Gamma_n(u)\right|\sim e^{l(u)}$$
et donc
$$ \left|\Gamma_n(u)\right|\sim e^{l(u)}n^{-(u+1)}$$

\item De l'équivalence $ \left|\Gamma_n(u)\right|\sim e^{l(u)}n^{-(u+1)}$, on peut déduire: \newline
$\bullet$ pour $u>-1$, $\left|\Gamma_n(u)\right|\to 0$ donc  
 $\left( \Gamma_n(u)\right)_{n\in \N}$ converge vers $0$.\newline
 $\bullet$ pour $u<-1$, $\left|\Gamma_n(u)\right|\to +\infty$ donc  
 $\left( \Gamma_n(u)\right)_{n\in \N}$ diverge.\newline
Pour $u\in\N$, à partir du moment où $n\se u$, $\Gamma_n(u)=0$ et donc $\left( \Gamma_n(u)\right)_{n\in \N}$ converge vers $0$ (stationnaire).
\end{enumerate}
\end{enumerate}

\subsection*{Partie II. Suite associée à une fonction.}
\begin{enumerate}
\item Soit $i\in\N$ et $k\se i$ alors $\Gamma_k(i)=0$ donc les équations
\begin{displaymath}
 \forall n\in\N,\; \forall i \in \llbracket 0, n\rrbracket,\hspace{0.5cm}f(i) - \sum_{k=0}^na_k\Gamma_k(i) = 0
\end{displaymath}
reviennent à 
\begin{displaymath}
 \forall i\in\N \hspace{0.5cm}f(i) - \sum_{k=0}^ia_k\Gamma_k(i) = 0 \quad (*)
\end{displaymath}

L'équation pour $i=0$ est $f(0)-a_0=0$, elle détermine $a_0$ de façon unique.\newline
Pour $n\in\N$, supposons $(a_0,\dots,a_n)$ déterminé de façon unique par les équations. Comme $\Gamma_{n+1}(n+1)=1$, l'équation pour $i=n+1$  donne
\begin{displaymath}
a_{n+1}=f(n+1) - \sum_{k=0}^na_k\Gamma_k(n+1) 
\end{displaymath}
Ce qui détermine un et un seul $a_ {n+1}$. On montre ainsi  par récurrence  qu'il existe  une unique suite $\left( a_k\right) _{k\in \N}$ de nombres réels vérifiant $(*)$.
\item Pour $n\in\N$ et $i \in \llbracket 0, n\rrbracket$,
\begin{displaymath}
 b^i-\sum_{k=0}^n(b-1)^k\Gamma_k(i)=b^i-\sum_{k=0}^i(b-1)^k\binom{i}{k}=0
\end{displaymath}
La suite $\left( (b-1)^n\right)_{n\in \N}$ est donc associée à la fonction $x \rightarrow b^x$.
\item 
\begin{enumerate}
\item La fonction $\varphi$ est $\mathcal{C}^{\infty}$  sur $I$ et s'annule en $i\in \llbracket 0, n\rrbracket$. En appliquant le théorème de Rolle sur les intervalles $[i, i+1]$ pour $i\in \llbracket 0, n-1\rrbracket$, on montre que $\varphi'$ s'annule en au moins $n-1$ points $x_1<\dots<x_{n-1}$ avec $i<x_i<i+1$. On applique ensuite le théorème de Rolle à $\varphi'$ sur les intervalles $[x_i, x_{i+1}]$ pour $i\in \llbracket 0, n-2\rrbracket$, \dots Par récurrence on montre que $\varphi^{(k)}$  s'annule en au moins $n+1-k$ réels positifs distincts.
\item D'après la question précédente $\varphi^{(n)}$  s'annule en au moins un réel positif, notons le $\lambda_n$.
Or $\varphi^{(n)}=f^{(n)}-a_n$, on en déduit que $a_n = f^{(n)}(\lambda_n)$.   
\end{enumerate}
\end{enumerate}

\subsection*{Partie III. Un exemple.}
\begin{enumerate}
\item On considère les équations caractérisant la suite $(a_n)$. On considère $f(0)-a_0=0$ on en déduit $$a_0=\frac 1\lambda$$
Puis, $f(1)-a_0-a_1=0$, donc $$a_1=\frac 1{1+\lambda}- \frac 1\lambda=\frac {-1}{\lambda(\lambda+1)}$$
Enfin $f(2)-a_0-2a_1-a_2=0$, d'où $$a_2=\frac 2{\lambda(\lambda+1)(\lambda+2)}$$

\item La fonction $x\mapsto (x+\lambda)r_n(x)$ est polynomiale de degré $n+1$ et s'annule en $0, \cdots, n$ par définition de $r_n$. D'après son degré, on a donc  trouvé toutes les racines du polynôme $x\mapsto (x+\lambda)r_n(x)$ et elles sont toutes simples. D'autre part, le coefficient dominant de  $x\mapsto (x+\lambda)r_n(x)$ est $-\frac{a_n}{n!}$. D'où
\begin{displaymath}
 \forall x\in I,\hspace{0.5cm} (x+\lambda)r_n(x) = -\frac{a_n}{n!}x(x-1)\dots(x-n)
\end{displaymath}
et donc 
\begin{displaymath}
 \forall x\in I,\hspace{0.5cm} r_n(x) = -(n+1)a_n\frac{\Gamma_{n+1}(x)}{x+\lambda}
\end{displaymath}
\begin{enumerate}

\item En évaluant  l'égalité  précédente en $x=n+1$, on trouve $(n+1+\lambda)r_n(n+1)=-(n+1)a_n$.\newline
D'autre part, par définition des $r_k$, on a 
\begin{displaymath}
 r_{n+1}=r_n-a_{n+1}\Gamma_{n+1}
\end{displaymath}
en évaluant l'égalité en  $x=n+1$, on trouve $a_{n+1}=r_n(n+1)$.\newline
D'où
\begin{displaymath}
 \forall n \in \N,\hspace{0.5cm} (n+1+\lambda)a_{n+1} = -(n+1)a_n
\end{displaymath}

\item Montrons le résultat par récurrence sur $n$.\newline
Pour $n=0$, $a_0=\frac 1\lambda=\frac {-1}{\lambda}=\frac {-1}{\Gamma-1(-\lambda)}$.\newline
Soit $n\in\N$, supposons le résultat vrai au rang $n$.
\begin{multline*}
(n+2)a_{n+1)} = (n+2)\frac{-(n+1)a_n}{n+1+\lambda}\quad\text{(par la question précédente)}\\
 = \frac{n+2}{(n+1+\lambda)\Gamma_{n+1}(-\lambda)}\quad\text{(par hypothèse de récurrence)}\\
 = \frac{-1}{\Gamma_{n+2}(-\lambda)}
\end{multline*}
D'où le résultat par récurrence.

\end{enumerate}
\item Soit $x> -\lambda$ fixé. On remarque que  $r_n(x)=f(x)- A_n(x)$.
\begin{eqnarray*}
\left|r_n(x)\right|&=&\left|\frac{\Gamma_{n+1}(x)}{\Gamma_{n+1}(-\lambda)(x+\lambda)}\right| \quad\text{par  2. et 3b.}\\
& \sim& \frac{K(x)}{K(-\lambda)}(n+1)^{-\lambda-x} \quad \text{par I4b.}\\
&\to& 0\quad\text{car } x> -\lambda
\end{eqnarray*}
Donc la suite $\left( A_n(x)\right)_{n\in \N}$  converge vers $f(x)$.
\end{enumerate}

\subsection*{Partie IV. Une expression du reste.}
\begin{enumerate}
 \item Par construction, la fonction $r_n$ s'annule en $0,1,\cdots,n$. Par définition $\Gamma_{n+1}$ s'annule aussi en ces points. Il en est donc de même pour $\psi$. De plus, on vérifie facilement que $\psi(u)$ aussi est nul. Comme $\psi$ s'annule $n+2$ fois, d'après le théorème de Rolle, sa dérivée s'annule $n+1$ fois (entre les zéros donc on reste bien dans $I$). On montre par récurrence que $\psi^{(k)}$ s'annule $n+2-k$ fois pour $k$ entre $0$ et $n+1$.

\item Quand on dérive $n+1$ fois, on fait disparaitre tous les $\Gamma_k$ de $r_n$. Si $v$ est tel que $\psi^{(n+1)}(v)=0$, il vérifie
\begin{displaymath}
 0 = f^{(n+1)}(v) - \frac{r_n(u)}{\Gamma_{n+1}(u)}\underset{=1}{\underbrace{\Gamma_{n+1}^{(n+1)}(v)}}
\Rightarrow
r_n(u) = f^{(n+1)}(v) \Gamma_{n+1}(u)
\end{displaymath}

\item Sous l'hypothèse de cette question :
\begin{displaymath}
 \left| r_n(u)\right| \leq M n \left|\Gamma_{n+1}(u)\right|\sim M K(u)n^{-u}
\end{displaymath}
d'après l'équivalent trouvé en I.4.b. On en déduit que $\left( r_n(u)\right) _{n\in \N}$ converge vers $0$.

\item Une fonction de classe $\mathcal{C}^{\infty}$ qui s'annule sur tous les entiers et qui n'est pas identiquement nulle ne vérifie certainement pas cette condition de majoration des ses dérivées.  
\end{enumerate}


\subsection*{Partie V. Un autre exemple.}
\begin{enumerate}
\item Lorsque $h=-1$, on peut utiliser la question I.3.b. On en tire:
\begin{displaymath}
 A_n(u) = \sum_{k=0}^n(-1)^k\Gamma_k(u) = (-1)^n \Gamma_n(u-1)
\end{displaymath}
La convergence a été discutée en I.4.c.
\begin{itemize}
 \item Si $u\geq 0$, la suite converge vers $0$.
 \item Si $u<0$, la suite diverge, sa valeur absolue tend vers $+\infty$.
\end{itemize}

\item
\begin{enumerate}
\item On a vu en question I.1.b. que la suite des $(-1)^n\Gamma_n(u)$ est de signe constant dès que $n\geq \lceil u \rceil$. Comme ici $h<0$, il en est de même de la suite des $h^n\Gamma_n(u)$.\newline
\`A partir de $\lceil u \rceil$, la suite des $A_n(u)$ est donc monotone.

\item D'après la question II.2., $\left( h_n\right) _{n\in \N}$ est la suite associée à la fonction $x\mapsto (h+1)^x$ définie dans $I = ]-1,+\infty[$. On peut exprimer le reste avec la question IV.2. Pour tout $n$, il existe un $v_n>-1$ tel que 
\begin{displaymath}
 (1+h)^u - A_n(u) = r_n(u) = (\ln(1+h))^{n+1}(1+h)^{v_n}\Gamma_{n+1}(u)
\end{displaymath}
car $(\ln(1+h))^{n+1}(1+h)^{x}$ est l'expression en $x$ de la dérivée $n$-ième de $x\mapsto (h+1)^x$.\newline
L'hypothèse $-1<h<0$ de la question entraîne $0<1+h<1$ donc $\ln(1+h)<0$ et $(1+h)^{v_n}>0$. Ainsi, à partir d'un certain rang, $r_n(u)$ est de signe constant et ce signe est le même que celui de la suite des $h^n\Gamma_n(u)$.
\begin{itemize}
 \item Si $r_n(u)$ et $h^n\Gamma_n(u)$ sont positifs. \`A partir de $\lceil u \rceil$, la suite des $A_n(u)$ est croissante et
\begin{displaymath}
 (1+h)^u - A_n(u) \geq 0 \Rightarrow A_n(u)\leq (1+h)^u
\end{displaymath}
La suite est donc majorée, elle converge.
\item Dans le cas négatif, la suite est décroissante et minorée, elle converge encore.
\end{itemize}

\item Lorsque $\frac{1}{e}-1\leq h <0$, d'après la question I.1., la suite $\left( (\ln(1+h))^{n+1}\right) _{n\in \N}$ est bornée. La suite $\left( (1+h)^{v_n}\right) _{n\in \N}$ l'est aussi car $\frac{1}{e}\leq 1+h <1$. La suite $\left( \Gamma_n(u)\right) _{n\in \N}$ converge vers $0$ d'après I.4.c. On en déduit que la suite des restes converge vers $0$ c'est à dire que la limite $g(u)=(1+h)^u$.
\end{enumerate}
 
\item
\begin{enumerate}
\item Dans cette situation, les $h^n\Gamma_n(u)$ changent de signe à chaque fois tout en décroissant en valeur absolue (à partir d'un certain rang). Les suites extraites d'indices pairs et impairs sont alors adjacentes.
\item  La suite complète converge vers leur limite commune car les indices pairs et impairs recouvrent tout $\N$.
\item L'expression du reste de la question 2.b. reste valable
\begin{displaymath}
 (1+h)^u - A_n(u) = r_n(u) = (\ln(1+h))^{n+1}(1+h)^{v_n}\Gamma_{n+1}(u)
\end{displaymath}
Cette fois, seul le facteur en $\Gamma$ change de signe.
\begin{itemize}
 \item Si $\Gamma_{n+1}(u)>0$. Alors $A_n(u)<A_{n+1}(u)$ et $A_n(u)< (1+h)^u$.\newline
 De plus, $(1+h)^u - A_{n+1}(u)$ est du signe de $A_{n+2}(u)$ c'est à dire négatif d'où
\begin{displaymath}
 A_n(u)< (1+h)^u < A_{n+1}(u)
\end{displaymath}
 \item Le raisonnement est analogue dans l'autre cas.
\end{itemize}
\end{enumerate}
Par passage à la limite dans l'encadrement, on obtient $g(u)=(1+h)^u$.
\item
\begin{enumerate}
\item Comme la suite des $\Gamma_n(u) = A_n(u) - A_{n-1}(u)$ diverge lorsque $u\leq-1$, la suite des $A_n(u)$ diverge aussi.
\item Pour $h=1$, l'étude faite en 3. reste valable et la suite converge vers $(1+h)^u = 2^u$.
\end{enumerate}
\end{enumerate}
