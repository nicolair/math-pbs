%<dscrpt>Autour de l'intégrale de Dirichlet.</dscrpt>
Dans tout le problème, on notera $f$ la fonction \emph{continue} dans $[0,+\infty[$ telle que
\begin{displaymath}
  \forall t>0,\hspace{0.5cm} f(t) = \frac{\sin t}{t}
\end{displaymath}
On désigne par $F$ la primitive de $f$ nulle en $0$.

\subsection*{I. \'Etudes locales en 0.}
\begin{enumerate}
  \item Justifier l'existence de $f$ et préciser sa valeur en $0$.
  \item Former un développement asymptotique en $0$ de la fonction 
\begin{displaymath}
t\rightarrow \frac{1}{\sin t}  
\end{displaymath}
définie dans $]0,\pi[$. Le reste devra être $o(t^2)$.
  \item Former un développement asymptotique en $0$ de la fonction 
\begin{displaymath}
t\rightarrow \frac{\cos t}{\sin^2 t}  
\end{displaymath}  
  définie dans $]0,\pi[$. Le reste devra être $o(t)$.
  \item Montrer que la fonction
\begin{displaymath}
t\rightarrow \frac{1}{t} - \frac{1}{\sin t}  
\end{displaymath}
définie dans $]0,\pi[$ admet un prolongement continu à $[0,\pi[$. Montrer que ce prolongement est de classe $\mathcal{C}^1$.
\end{enumerate}

\subsection*{II. Calcul de l'intégrale de Dirichlet.}
\begin{enumerate}
  \item Lemme de Riemann-Lebesgue. Soit $\varphi\in \mathcal{C}^{1}([0,\frac{\pi}{2}])$. Montrer que la suite
\begin{displaymath}
  \left( \int_0^{\frac{\pi}{2}}\varphi(t)\sin(nt)\, dt\right)_{n\in \N}
\end{displaymath}
converge vers $0$.
 \item 
\begin{enumerate}
  \item Pour $a$ et $b$ réels, exprimer $2\sin(a)\cos(b)$ comme une somme.
  \item Pour tout $t\in ]0,\pi[$, exprimer
\begin{displaymath}
  1 + 2\cos(2t) + 2\cos(4t) + \cdots + 2\cos(2nt)
\end{displaymath}
à l'aide d'un quotient de deux valeurs de la fonction $sin$.
\end{enumerate}
 
  \item Pour tout $n\in \N$, montrer que
\begin{displaymath}
  \int_0^{\frac{\pi}{2}}\frac{\sin\left( (2n+1)t\right) }{\sin(t)}\,dt = \frac{\pi}{2}
\end{displaymath}
  \item Montrer que 
\begin{displaymath}
  \left( F\left( (2n+1)\frac{\pi}{2}\right) \right)_{n\in \N} \rightarrow \frac{\pi}{2}
\end{displaymath}
  \item Soit $x>0$ et $n$ la partie entière de $\frac{x}{\pi}$.
\begin{enumerate}
  \item Montrer que
\begin{displaymath}
  \left|F(x) - F(\frac{\pi}{2}+n\pi)\right| \leq \frac{1}{n} 
\end{displaymath}
  \item Montrer que $F$ converge en $+\infty$ et préciser sa limite\footnote{Cette limite est appelée \emph{l'intégrale de Dirichlet}}.
\end{enumerate}
\end{enumerate}

\subsection*{III. \'Equation différentielle.}
Pour $n$ entier naturel, on considère l'équation différentielle
\begin{displaymath}
  (E_n):\hspace{1cm} y''(x) + y(x) = x^n
\end{displaymath}
où la fonction inconnue $y$ est à valeurs réelles et définie dans $\R$.\newline
On considère également la proposition
\begin{multline*}
  \mathcal{P}_n:\hspace{1cm} \text{ il existe } P_n \text{ et } Q_n \text{ dans } \R[X] \text{ tels que }\\
  \forall x>0:\;f^{(n)}(x)=\frac{P_n(x)\sin^{(n)}(x) + Q_n(x)\sin^{(n+1)}(x)}{x^{n+1}}
\end{multline*}
Les puissances entre parenthèses désignent des ordres de dérivation.
\begin{enumerate}
  \item
\begin{enumerate}
  \item   En utilisant une argumentation d'algèbre linéaire, montrer que $(E_n)$ admet une unique solution polynomiale et qu'elle est de degré $n$. On la désigne par $A_n$.
  \item Préciser l'ensemble des solutions de $(E_n)$. Quelle est la structure de cet ensemble dans l'espace des fonctions de $\R$ dans $\R$?
  \item On note 
\begin{displaymath}
  A_n = a_0 + a_1X + \cdots +a_n X^n
\end{displaymath}
Préciser $a_n$, $a_{n-1}$ et une relation entre $a_k$ et $a_{k+2}$ pour $k\in \llbracket 0, n-2 \rrbracket$. En déduire des expressions de $a_{n-2i}$ et $a_{n-2i-1}$ ne contenant que des factorielles. (On précisera les intervalles contenant $i$).
\end{enumerate}

  \item 
\begin{enumerate}
  \item Vérifier $\mathcal{P}_n$ pour $n=0,1,2$ en précisant les polynômes $P_n$ et $Q_n$.
  \item Montrer $\mathcal{P}_n$ pour tout entier $n$. Préciser les expressions de $P_{n+1}$ et $Q_{n+1}$ en fonction de $P_n$ et $Q_n$.
\end{enumerate}
  
  \item
\begin{enumerate}
  \item Montrer que $P_n$ et $Q_n$ sont à coefficients dans $\Z$, préciser le degré, la parité et le coefficient dominant de ces polynômes.
  \item Calculer $P_3$ et $Q_3$. 
\end{enumerate}

  \item Soit $U$ et $V$ deux polynômes vérifiant
\begin{displaymath}
\forall x>0,\hspace{0.5cm} U(x)\sin(x) + V(x)\cos(x) = 0  
\end{displaymath}
Montrer que $U$ et $V$ sont égaux au polynôme nul.

  \item 
\begin{enumerate}
\item En utilisant la formule de Leibniz et la relation
\begin{displaymath}
\forall x>0,\hspace{0.5cm} xf(x) = \sin(x)    
\end{displaymath}
former deux nouvelles relations entre $P_{n+1}, Q_{n+1}, P_n, Q_n$.
\item En déduire $P'_n = Q_n$ et que $P_n$ est l'unique solution polynomiale $A_n$ de l'équation différentielle $(E_n)$.
\end{enumerate}

\end{enumerate}
  

