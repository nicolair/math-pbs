\begin{enumerate}
\item  Dans les conditions de l'{\'e}nonc{\'e}, $s(n,0)=\frac{1-x^{n+1}}{%
1-x}$ tend vers $\frac{1}{1-x}=s_{0}(x)$.

\item \begin{enumerate} \item On peut {\'e}crire
\begin{eqnarray*}
(1-x)s(n,1)=1+2x+\cdots +(n+1)x^{n} \\
-x-2x^{2}-\cdots -nx^{n}-(n+1)x^{n+1} \\
=1+x+\cdots +x^{n}-(n+1)x^{n+1}
\end{eqnarray*}
On en d{\'e}duit que $s(n,1)$ converge et que sa limite $s_{1}(x)$
v{\'e}rifie $(1-x)s_{1}(x)=s_{0}(x)$ soit
$s_{1}(x)=\frac{1}{(1-x)^{2}}$.

\item  Exercice trait{\'e} en classe.
\end{enumerate}

\item \begin{enumerate} \item En d{\'e}veloppant $(1-x)s(n,r)$ on obtient
\begin{eqnarray*}
&+&\binom{r}{r}+\binom{r+1}{r}x+\binom{r+2}{r}x^{2}+\cdots
+\binom{r+n}{r}x^{n} \\
&-&\binom{r}{r}x+\binom{r+1}{r}x^{2}-\cdots
-\binom{r+n-1}{r}x^{n}-\binom{r+n}{r}x^{n+1} \\
&=&1+(\binom{r+1}{r}-\binom{r}{r})x+\cdots
+(\binom{r+n}{r}-\binom{r+n-1}{r})x^{n}\\
&\quad& -\binom{r+n}{r}x^{n+1} \\
&=&\binom{r-1}{r-1}+\binom{r}{r-1}x+\binom{r+1}{r-1}x^{2}+\cdots
+\binom{r+n-1}{r-1}x^{n}\\
&\quad&-\binom{r+n}{r-1}x^{n+1} \\
&=&s(n,r-1)-\binom{r+n}{r}x^{n+1}
\end{eqnarray*}

\item  La suite de terme g{\'e}n{\'e}ral $u_{n}^{r}x^{n}$ est une suite
usuelle qui converge vers 0. D'autre part, comme $r$ est fix{\'e}, $\binom{n%
}{r}x^{n}\sim \frac{n^{r}}{r!}x^{n}$donc la suite de terme g{\'e}n{\'e}ral $%
\binom{n}{r}x^{n}$converge vers 0. On peut donc passer {\`a} la limite
dans la relation du a. et obtenir $(1-x)s_{r}(x)=s_{r-1}(x)$. On
en d{\'e}duit
\[
s_{r}(x)=\frac{1}{(1-x)^{r+1}}\text{.}
\]
\end{enumerate}
\end{enumerate}
