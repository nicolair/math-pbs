%<dscrpt>Quelques propriétés d'une homographie.</dscrpt>
Soit $h$ l'application de $\C\setminus\{-i\}$ dans $\C$ définie par
\begin{displaymath}
 \forall z \in \C\setminus\{-i\}:\hspace{0.5cm} h(z)=\frac{z+2i}{1-iz}
\end{displaymath}
\begin{enumerate}
 \item Montrer que $h$ définit une bijection de $\C\setminus\{-i\}$ dans $\C\setminus\{i\}$.
 \item \'Etudier les $z$ tels que $h(z)=z$ (points fixes de $h$).\newline On en trouvera deux qui seront notés $p$ et $q$.
 \item Factoriser $h(z)-h(z')$ pour $z$ et $z'$ complexes autres que $-i$.
 \item Calculer $B(p,q,h(z),z)$ pour $z$ complexe différent de $p$ et $q$ avec
\begin{displaymath}
 B(m_1,m_2,m_3,m_4) = \frac{(m_2-m_1)(m_4-m_3)}{(m_4-m_1)(m_3-m_2)}
\end{displaymath}
On considère les points $P$, $Q$, $Z'$, $Z$ d'affixes $p$, $q$, $h(z)$, $z$. Que peut-on en déduire sur les mesures des angles orientés $(\overrightarrow{P Z},\overrightarrow{P Q})$ et $(\overrightarrow{Z' Z},\overrightarrow{Z' Q})$ ?
\end{enumerate}
