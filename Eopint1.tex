%<dscrpt>Opérateur intégral.</dscrpt>
On note $E=C^0(\R,\R)$ et on définit $F$ par
\begin{displaymath}
F=\{g\in C^1(\R,\R) \text{ telles que } g(0)=g'(0)=0\text{ et } g' \text{ d{\'e}rivable en 0} \} 
\end{displaymath}
ainsi que l'application
\begin{displaymath}
 \Phi : \left\lbrace 
\begin{aligned}
 E &\longrightarrow F \\
 f & \mapsto \Phi(f) \text{ noté }\Phi_f 
 =
 \left\lbrace 
 \begin{aligned}
 \R &\rightarrow \R \\ x &\mapsto \int_0^x tf(t) \, dt
 \end{aligned}
 \right. 
\end{aligned}
\right. 
\end{displaymath}
\begin{enumerate}
    \item Montrer que $E$ et $F$ sont des $\R$-espaces vectoriels.
    \item V{\'e}rifier que $\Phi_f \in F$ et que $\Phi$ est lin{\'e}aire.
    \item Soit $g \in F$. On d{\'e}finit $f$ dans $\R$ par 
\begin{displaymath}
\forall x\in \R,\; f(x)= \left\lbrace 
\begin{aligned}
 \frac{g'(x)}{x} &\text{ si } x\neq 0 \\
 g''(0) &\text{ si } x= 0 
\end{aligned}
\right. 
\end{displaymath}
      Montrer que $f\in E$ et en d{\'e}duire que $\Phi$ est surjective.
    \item Montrer que $\Phi$ est un isomorphisme.
\end{enumerate}
