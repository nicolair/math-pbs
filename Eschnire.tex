%<dscrpt>Borne inférieure d'un ensemble de nombres réels. Densité de Schnirelmann.</dscrpt>

Pour toute partie \footnote{D'après le problème 1 de l'ouvrage "Problèmes choisis de mathématiques supérieure" (Springer).} $A$ de $\N$ et tout entier $n\geq1$, on pose
\[S_n(A)=\card(A\cap \llbracket 1,n \rrbracket)\]
et on appelle \emph{densité de Schnirelmann} de $A$ le réel
\[\sigma (A)= \inf \{\frac{S_n(A)}{n}, n\geq 1\}\]
Si $A$ et $B$ sont deux parties de $\N$, on pose
\[A+B=\{a+b, a\in A , b\in B\}\]
\begin{enumerate}
\item \begin{enumerate}
  \item Justifier la définition de $\sigma (A)$ et montrer que $\sigma(A) \leq 1$.
  \item Que vaut $\sigma(A)$ si $1\not \in A$ ?
  \item Sous quelle condition a-t-on $\sigma (A)=1 $ ?
  \item Si $A \subset B$, comparer $\sigma (A)$ et $\sigma (B)$.
      \end{enumerate}
\item Calculer $\sigma (A)$ pour les parties suivantes :
\begin{enumerate}
  \item $A$ est une partie finie de $\N$.
  \item $A$ est l'ensemble des entiers impairs.
  \item Soit $k\geq 2$ entier fixé et $A$ l'ensemble des puissances $k$-ièmes d'entiers.
\begin{displaymath}
 A=\{m^k, m\in \N ^*\}
\end{displaymath}
      \end{enumerate}
\item Soit $A$ et $B$ deux parties de $\N$ contenant $0$, soit $n\geq 1$ un nombre entier. En considérant
\[
C=\{n-b, b\in\llbracket 0,n \rrbracket \cap B\}
\]
montrer que
\[
S_n(A)+S_n(B)\geq n \Rightarrow n\in A+B
\]
\item Soit $A$ et $B$ deux parties de $\N$ contenant $0$.
\begin{enumerate}
  \item Montrer que si $\sigma (A)+\sigma (B) \geq 1$ alors $A+B=\N$.
  \item Montrer que si $\sigma (A)\geq \frac{1}{2}$ alors tout entier est la somme de deux éléments de $A$.
\end{enumerate}

\end{enumerate} 