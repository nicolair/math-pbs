%<dscrpt>Astroïde.</dscrpt>
\textbf{{\'E}tude de l'astro{\"\i}de}\footnote{d'apr{\`e}s ESM Saint Cyr Math 2
Option M 1993}

 Soit $a$ un réel strictement positif fixé. Pour $t\in [-\pi,\pi]$, on appelle $M(t)$ le point de coordonn{\'e}es
\[(8a\cos^3t,8a\sin^3t)\]
dans un rep{\`e}re orthonorm{\'e} fix{\'e} $(O,\overrightarrow{i},\overrightarrow{j})$.\newline
Soit $(\mathcal{C})$ le support de la courbe param{\'e}tr{\'e}e $M$.
\begin{enumerate}
  \item
    \begin{enumerate}
     \item D{\'e}terminer les axes de sym{\'e}tries de $(\mathcal{C})$.
     \item {\'E}tudier et construire $(\mathcal{C})$. Pr{\'e}ciser les
     points stationnaires.
     \item Calculer la longueur totale de $(\mathcal{C})$. On admet que cette longueur $l$ est donn{\'e}e par
\[l=\int_{-\pi}^{\pi}\Vert\overrightarrow{M'}(t)\Vert dt\]
    \end{enumerate}
  \item
   \begin{enumerate}
     \item {\'E}crire une {\'e}quation de la tangente $D(t)$ en $M(t)$.
     \item Lorsque $M(t)$ n'est pas stationnaire, on note $A(t)$ et $B(t)$ les points d'intersection de
     $D(t)$ avec les axes. Calculer la longueur $A(t)B(t)$. Interpr{\'e}ter.
   \end{enumerate}

  \item Soit $t_0 \in [-\pi,\pi]$ et $P_0$ le point du cercle de centre O et de rayon
  $4a$ tel que 
\begin{displaymath}
(\overrightarrow{i},\overrightarrow{OP_0}) = t_0 
\end{displaymath}
(il s'agit d'un angle orienté)
   \begin{enumerate}
     \item Montrer que par $P_0$ passent en g{\'e}n{\'e}ral quatre tangentes {\`a} $(\mathcal{C})$.\newline
Montrer que trois de ces tangentes font deux {\`a} deux des angles {\'e}gaux. Que peut-on dire de la quatri{\`e}me ?
     \item Indiquer une construction g{\'e}om{\'e}trique de la droite $D(t_0)$ {\`a} partir du point $P_0$.
     \item Soit $H(t_0)$ la projection orthogonale de $O$ sur $D(t_0)$.\newline
Calculer
\begin{displaymath}
\overrightarrow{OH(t_0)}+\overrightarrow{OM(t_0)} 
\end{displaymath}
En d{\'e}duire une construction g{\'e}om{\'e}trique de $M(t_0)$.
   \end{enumerate}

\end{enumerate}
