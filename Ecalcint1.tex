%<dscrpt>Exercices d'application du cours de calcul intégral.</dscrpt>
Ce texte propose de simples applications du cours de calcul intégral.
\begin{enumerate}
  \item On considère la fonction $F$ définie dans $\R$ par:
\begin{displaymath}
  \forall x \in \R,\;F(x) = \int_{0}^{x}\ch(t)\sin(t)\,dt
\end{displaymath}
\begin{enumerate}
  \item Calculer $F(x)$ en utilisant deux intégrations par parties.
  \item Independamment du calcul précédent, retrouver l'expression de $F$ en développant à l'aide de exponentielles (réelles et complexes).
\end{enumerate}

\item Soit $a < b $ deux nombres réels. Pour $a < \alpha < \beta < b$ et $0 < x < y < 1$, on définit
\begin{displaymath}
  I(\alpha,\beta) = \int_{\alpha}^{\beta}\frac{dt}{\sqrt{(b-t)(t-a)}}, \hspace{0.5cm}
  J(x,y) = \int_{x}^{y}\frac{du}{\sqrt{u(1-u)}}
\end{displaymath}
\begin{enumerate}
  \item Effectuer dans l'intégrale $I(\alpha,\beta)$ le changement de variable
  \begin{displaymath}
    u = \frac{b-t}{b-a}
  \end{displaymath}
  
  \item  Effectuer dans l'intégrale $J(x,y)$ le changement de variable
  \begin{displaymath}
    u = \frac{1}{2} + \frac{1}{2}\sin \theta \hspace{0.5cm}\text{ avec } \theta \in \left[ -\frac{\pi}{2}, \frac{\pi}{2}\right] 
  \end{displaymath}
  
  \item Que se passe-t-il pour $I(\alpha,\beta)$ lorsque $\alpha$ tend vers $a$ et $\beta$ vers $b$?
\end{enumerate}

\end{enumerate}
