\begin{enumerate}
  \item 
\begin{enumerate}
  \item Soit $x\in \R$. Effectuons l'intégration par parties suivante:
$$\left \{ \begin{array}{ll}
            u_{1}'(t) = \ch (t)\\
            v_{1}(t) = \sin (t)
           \end{array}
  \right. \qquad \left \{ \begin{array}{ll}
                          u_{1}(t) = \sh (t)\\
                           v_{1}'(t) = \cos (t)
                          \end{array}
                  \right.$$
On a alors:
\begin{multline*}
 F(x) = \int_{0}^{x}u_{1}'(t)v_{1}(t)\ dt 
  = -\int_{0}^{x}u_{1}(t)v_{1}'(t)\ dt + \left [u_{1}(t)v_{1}(t)\right ]_{0}^{x} \\
  = -\int_{0}^{x}\sh (t)\cos (t)\ dt + \sh (x)\sin (x)
\end{multline*}

Calculons $\int_{0}^{x}\sh (t)\cos (t)\ dt$ à l'aide d'une intégration par parties. Posons:
$$\left \{ \begin{array}{ll}
            u_{2}'(t) = \sh (t)\\
            v_{2}(t) = \cos (t)
           \end{array}
  \right. \qquad \left \{ \begin{array}{ll}
                           u_{2}'(t) = \ch (t)\\
                           v_{2}'(t) = -\sin (t)
                          \end{array}
                 \right.$$
On a alors:
\begin{multline*}
 \int_{0}^{x}\sh (t)\cos (t)\ dt  = \int_{0}^{x}u_{2}'(t)v_{2}(t)\ dt
  = -\int_{0}^{x}u_{2}(t)v_{2}'(t)\ dt + \left [ u_{2}(t)v_{2}(t)\right ]_{0}^{x} \\
  = \int_{0}^{x}\ch (t)\sin (t)\ dt + \ch (x)\cos (x) - 1
\end{multline*}

On en déduit que $F(x) = -F(x) +\sh (x)\sin (x) - \ch (x)\cos (x) +1$ soit encore:
$$F(x) = \frac{\sh (x)\sin (x) - \ch (x)\cos (x) +1}{2}.$$

\item Pour tout $t\in \R$, $\sin (t) = \operatorname{Im}(e^{it})$ donc pour tout $x\in \R$:
$$F(x) = \operatorname{Im}\left ( \int_{0}^{x}\ch (t)e^{it}\ dt \right ).$$
Calculons l'intégrale complexe:
\begin{multline*}
 \int_{0}^{x}\ch (t)e^{it}\ dt  = \frac{1}{2}\left [ \int_{0}^{x}e^{(1+i)t} \ dt + \int_{0}^{x}e^{(-1+i)t}\ dt \right ] \\
  = \frac{1}{2} \left [ \frac{1}{1+i}\left ( e^{x}e^{ix}-1\right ) + \frac{1}{-1+i}\left ( e^{-x}e^{ix} - 1 \right ) \right ] \\
  = \frac{1}{4} \left[ \left(e^{x}\cos x - 1  + ie^{x}\sin x\right) (1-i) - \left( e^{-x}\cos x - 1 + ie^{-x}\sin x \right)(1+i) \right]
\end{multline*}
En prenant la partie imaginaire, on obtient:
\begin{multline*}
 F(x)  = \frac{1}{4} \left [ -(e^{x}\cos x-1)+e^{x}\sin x - \left ( e^{-x}\cos x - 1\right )-e^{-x}\sin x \right] \\
  = \frac{1}{4}\left[ \sin x\left( e^{x}-e^{-x} \right) -\cos x\left( e^{x}+e^{-x} \right) + 2\right]
  = \frac{\sin x\sh x - \cos x\ch x + 1}{2}
\end{multline*}
\end{enumerate}

\item
\begin{enumerate}
  \item Effectuons le changement de variables suivant:
$$\left \{ \begin{array}{ll}
            u = \frac{b-t}{b-a}\\
            du = -\frac{dt}{b-a}
           \end{array}
  \right.$$
Comme $b-t = (b-a)u$ et $t-a = (b-a)(1-u)$, on a, puisque $b-a\geq 0$: \\ 
$\sqrt{(b-t)(t-a)} = \sqrt{(b-a)^{2}u(1-u)} = (b-a)\sqrt{u(1-u)}$. Donc:
$$I(\alpha, \beta ) = \int_{\frac{b-\alpha}{b-a}}^{\frac{b-\beta}{b-a}}\frac{-(b-a)du}{(b-a)\sqrt{u(1-u)}} = 
\int_{\frac{b-\beta}{b-a}}^{\frac{b-\alpha}{b-a}}\frac{du}{\sqrt{u(1-u)}} = J\left ( \frac{b-\beta}{b-a}, \frac{b-\alpha}{b-a}   \right ).$$
\item Effectuons le changement de variables suivant:
$$\left \{ \begin{array}{ll}
            u = \frac{1}{2} + \frac{1}{2}\sin (\theta)\quad \text{ soit } \theta = \arcsin (2u-1)\\
            du = \frac{1}{2}\cos (\theta)d\theta
           \end{array}
 \right.$$
Ce changement de variables est possible puisque pour tout $u\in [x,y]$,
\begin{displaymath}
 -1<2x-1<2u-1<2y-1<1 
\end{displaymath}
On a: $u(1-u) = \frac{1}{4}(1-\sin(\theta))(1+\sin (\theta)) = \frac{1}{4}\cos^{2}(\theta)$. Comme $\cos (\theta)\geq 0$,
\begin{displaymath}
\sqrt{u(1-u)} = \sqrt{\frac{1}{4}\cos^{2}(\theta)} = \frac{1}{2}\cos(\theta)
\end{displaymath}
donc on a:
\begin{multline*}
J(x,y) = \int_{\arcsin(2x-1)}^{\arcsin (2y-1)}\frac{\cos (\theta)}{2\frac{1}{2}\cos(\theta)} 
= \int_{\arcsin(2x-1)}^{\arcsin (2y-1)}1 \ d\theta \\ = \arcsin (2y-1) - \arcsin (2x-1).  
\end{multline*}

\item On a pour tout $a<\alpha < \beta < b$:
\begin{displaymath}
I(\alpha, \beta ) = \arcsin \left ( \frac{2(b-\alpha)}{b-a}-1\right ) - \arcsin \left ( \frac{2(b-\beta)}{b-a}-1  \right)  
\end{displaymath}
Lorsque $\alpha$ tend vers $a$ et $\beta$ tend vers $b$, 
\begin{displaymath}
\left\lbrace  
\begin{aligned}
  \frac{b-\beta}{b-a} &\rightarrow 0 \\ \frac{b-\alpha}{b-a} &\rightarrow 1
\end{aligned}
\right. \Rightarrow
\left\lbrace  
\begin{aligned}
  2\frac{b-\beta}{b-a}-1 &\rightarrow -1 \\ 2\frac{b-\alpha}{b-a}-1 &\rightarrow 1
\end{aligned}
\right. \Rightarrow I(\alpha, \beta) \rightarrow \arcsin (1)-\arcsin (-1) = \pi. 
\end{displaymath}
Remarquons que 
\begin{displaymath}
 \frac{1}{\sqrt{(b - t)(t-a)}}\text{ tend vers } +\infty \text{ quand $t$ tend vers $a$ ou $b$.} 
\end{displaymath}
On s'attend à ce que l'intégrale $I(\alpha, \beta)$ soit de plus en plus grande lorsque $\alpha$ tend vers $a$ et $\beta$ tend vers $b$. Elle reste néanmoins majorée. \\
Géométriquement, l'aire de la partie du plan délimitée horizontalement par les droites d'équations $x=a$ et $x=b$ et verticalement par l'axe des abscisses et par le 
graphe de la fonction $t\mapsto \frac{1}{\sqrt{(b - t)(t-a)}}$ est finie. 
\end{enumerate}
\end{enumerate}



