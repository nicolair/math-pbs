%<dscrpt> Manipulations de sommes.</dscrpt>
\begin{enumerate}
\item Calculer les sommes suivantes 
    \begin{displaymath}
  F =\sum_{k=0}^{n}k(k!), \hspace*{1cm} B = \sum_{k=0}^{n}\frac{1}{k+1}\binom{n}{k}.
    \end{displaymath}
    
\item Pour tout entier $n\geq 1$, on note
\begin{displaymath}
  P_n = \prod_{k=1}^n\frac{2k-1}{2k}.
\end{displaymath}
\begin{enumerate}
  \item Montrer par récurrence que 
  \begin{displaymath}
    P_n < \frac{1}{\sqrt{2n+1}}.
  \end{displaymath}
  \item En remarquant que
\begin{displaymath}
  P_n = \prod_{k=1}^n\frac{2k-1}{2k}\times \frac{(2k)}{(2k)},
\end{displaymath}
exprimer $P_n$ uniquement avec des factorielles et une puissance de $2$. En déduire une expression de $P_n$ faisant intervenir un coefficient du binôme.
  \item Soit $k$ entier tel que $0\leq k < n$. Montrer que $\binom{2n}{k} < \binom{2n}{k+1}$. Que peut-on en déduire pour $\binom{2n}{n}$? Montrer que
\begin{displaymath}
  \frac{2^{2n}}{2n + 1} \leq \binom{2n}{n} \leq \frac{2^{2n}}{\sqrt{2n+1}}.
\end{displaymath}

\end{enumerate}
\end{enumerate}
