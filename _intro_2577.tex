À côté des problèmes d'entrainement ou d`évaluation habituels, cet ouvrage rassemble des problèmes de mathématiques présentant des résultats significatifs.  

Ces documents m'ont servi de matériau pour produire des textes (surtout de devoir en temps libre) proposés dans ma classe de MPSI (Versailles). Ils couvrent l'essentiel de ce qui doit être abordé par un étudiant débutant : calculs, analyse, algèbre, géométrie, à l'exception des probabilités. Je ne me suis pas pour autant attaché à respecter strictement un programme officiel d'un niveau particulier.

Ces problèmes sont des outils d'enseignement. Ils doivent permettre à un lecteur déterminé et aventureux, équipé seulement de son savoir de \og mathématiques élémentaires\fg, d'enrichir sa culture et d'élargir le paysage qu'il peut parcourir.

Au delà des MPSI, cet ouvrage s'adresse aussi aux étudiants de premier cycle intéressés par les mathématiques pour elles-mêmes, ainsi qu'aux enseignants ou aux futurs enseignants préparant un concours (CAPES ou agrégation) de mathématiques. 

Les solutions sont détaillées et rédigées avec précision. Comme les problèmes ne sont pas faciles, le lecteur ne doit pas hésiter à faire des aller-retours entre énoncé et solution pour réactiver sa recherche. S'agissant d'un ouvrage d'enseignement et non de recherche, je n'ai pas établi de bibliographie formelle mais j'ai cité chaque fois que possible les sources dont je me suis inspiré.

Je remercie mes générations d'étudiants pour avoir travaillé sur ces sujets et signalé nombre de coquilles et d'erreurs, avec une mention particulière pour la promotion 2016-2017. Je remercie enfin très chaleureusement mon ami Saman pour son scrupuleux travail de relecture et les innombrables échanges, coups de téléphone, mels et discussions; sans lui je ne serais pas allé au bout.

Les textes rassemblés dans cet ouvrage sont tirés de la base de données pédagogiques maquisdoc.net que j'alimente depuis des années et qui est toujours active. 

Un version papier de cet ouvrage a été édité en 2017 par Les Editions Universitaires Européennes. La version en ligne tient compte des modifications introduites depuis.

