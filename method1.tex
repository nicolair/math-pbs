%<dscrpt>Fichier de déclarations Latex à inclure au début d'un texte.</dscrpt>

\documentclass[a4paper,landscape,twocolumn]{article}
\usepackage[hmargin={.8cm,.8cm},vmargin={2.4cm,2.4cm},headheight=13.1pt]{geometry}

%includeheadfoot,scale=1.1,centering,hoffset=-0.5cm,
\usepackage[pdftex]{graphicx,color}
\usepackage{amsmath}
\usepackage{amssymb}
\usepackage{stmaryrd}
\usepackage[french]{babel}
%\selectlanguage{french}
\usepackage{fancyhdr}
\usepackage{floatflt}
\usepackage{ucs}
\usepackage[utf8]{inputenc}
\usepackage[T1]{fontenc}
%\usepackage[latin1]{inputenc}
\usepackage[pdftex,colorlinks={true},urlcolor={blue},pdfauthor={remy Nicolai}]{hyperref}

\usepackage{wrapfig}
%\usepackage{subcaption}
\usepackage{subfig}


%pr{\'e}sentation du compteur de niveau 2 dans les listes
\makeatletter
\renewcommand{\labelenumii}{\theenumii.}
\makeatother

%dimension des pages, en-t{\^e}te et bas de page
%\pdfpagewidth=20cm
%\pdfpageheight=14cm
%   \setlength{\oddsidemargin}{-2cm}
%   \setlength{\voffset}{-1.5cm}
%   \setlength{\textheight}{12cm}
%   \setlength{\textwidth}{25.2cm}
   \columnsep=0.4cm
   \columnseprule=0.3pt

%En tete et pied de page
\pagestyle{fancy}
\lhead{MPSI B}
\rhead{\today}
%\rhead{25/11/05}
\lfoot{\tiny{Cette création est mise à disposition selon le Contrat\\ Paternité-Partage des Conditions Initiales à l'Identique 2.0 France\\ disponible en ligne http://creativecommons.org/licenses/by-sa/2.0/fr/
} }
\rfoot{\tiny{Rémy Nicolai \jobname}}

\newcommand{\baseurl}{http://back.maquisdoc.net/data/devoirs_nicolair/}
\newcommand{\textesurl}{http://back.maquisdoc.net/data/devoirs_nicolair/}
\newcommand{\exosurl}{http://back.maquisdoc.net/data/exos_nicolair/}
\newcommand{\coursurl}{http://back.maquisdoc.net/data/cours_nicolair/}
\newcommand{\mwurl}{http://back.maquisdoc.net/data/maple_nicolair/}


\newcommand{\N}{\mathbb{N}}
\newcommand{\Z}{\mathbb{Z}}
\newcommand{\C}{\mathbb{C}}
\newcommand{\R}{\mathbb{R}}
\newcommand{\K}{\mathbf{K}}
\newcommand{\Q}{\mathbb{Q}}
\newcommand{\F}{\mathbf{F}}
\newcommand{\U}{\mathbb{U}}
\newcommand{\p}{\mathbb{P}}
\renewcommand{\P}{\mathbb{P}}

\newcommand{\card}{\operatorname{Card}}
\newcommand{\Id}{\operatorname{Id}}
\newcommand{\Ker}{\operatorname{Ker}}
\newcommand{\Vect}{\operatorname{Vect}}
\newcommand{\cotg}{\operatorname{cotan}}
\newcommand{\cotan}{\operatorname{cotan}}
\newcommand{\sh}{\operatorname{sh}}
\newcommand{\argsh}{\operatorname{argsh}}
\newcommand{\argch}{\operatorname{argch}}
\newcommand{\ch}{\operatorname{ch}}
\newcommand{\tr}{\operatorname{tr}}
\newcommand{\rg}{\operatorname{rg}}
\newcommand{\rang}{\operatorname{rg}}
\newcommand{\Mat}{\operatorname{Mat}}
\newcommand{\MatB}[2]{\operatorname{Mat}_{\mathcal{#1}}(#2)}
\newcommand{\MatBB}[3]{\operatorname{Mat}_{\mathcal{#1} \mathcal{#2}}(#3)}
\renewcommand{\cot}{\operatorname{cotan}}
\renewcommand{\Re}{\operatorname{Re}}
\newcommand{\Ima}{\operatorname{Im}}
\renewcommand{\Im}{\operatorname{Im}}
\renewcommand{\th}{\operatorname{th}}
\newcommand{\repere}{$(O,\overrightarrow{i},\overrightarrow{j},\overrightarrow{k})$}
\newcommand{\repereij}{$(O,\overrightarrow{i},\overrightarrow{j})$}
\newcommand{\zeron}{\llbracket 0,n\rrbracket}
\newcommand{\unAn}{\llbracket 1,n\rrbracket}

\newcommand{\IntEnt}[2]{\llbracket #1 , #2 \rrbracket}

\newcommand{\absolue}[1]{\left| #1 \right|}
\newcommand{\fonc}[5]{#1 : \begin{cases}#2 &\rightarrow #3 \\ #4 &\mapsto #5 \end{cases}}
\newcommand{\depar}[2]{\dfrac{\partial #1}{\partial #2}}
\newcommand{\norme}[1]{\left\| #1 \right\|}
\newcommand{\norm}[1]{\left\Vert#1\right\Vert}
\newcommand{\scal}[2]{\left\langle {#1} , {#2} \right\rangle}
\newcommand{\abs}[1]{\left\vert#1\right\vert}
\newcommand{\set}[1]{\left\{#1\right\}}
\newcommand{\se}{\geq}
\newcommand{\ie}{\leq}
\newcommand{\trans}{\,\mathstrut^t\!}
\renewcommand{\abs}[1]{\left\vert#1\right\vert}
\newcommand{\GL}{\operatorname{GL}}
\newcommand{\Cont}{\mathcal{C}}
\newcommand{\Lin}{\mathcal{L}}
\newcommand{\M}{\mathcal{M}}

\newcommand\empil[2]{\genfrac{}{}{0pt}{}{#1}{#2}}

% à utiliser avec \value{enumi} pour pouvoir reprendre le compte dans un enumerate fermé
\newcounter{numquestion}

\begin{document}

\chead{Méthodologie}

\section{Boîte à outils}
Votre \og boîte à outils\fg~ mathématiques personnelle doit contenir\bigskip
\begin{itemize}
 \item des objets: nombres, fonctions, ensembles, équations, systèmes, ... dont vous maitrisez les concepts et le vocabulaire
 \item des résultats de cours (propositions, théorèmes) dont vous maitrisez les noms, les conclusions, les hypothèses
 \item des techniques : raisonnement par analyse-synthèse, changement de noms dans les sommations, méthode du pivot pour les systèmes d'équations, IPP, changement de variables pour les intégrales, ...
 \item des exemples de mise en oeuvre des techniques : résolution d'exercices en classe, corrigés de DM ou de DS, preuves de théorème... 
 \item des conseils pratiques: de rédaction, de présentation, techniques : pas de $lim$ ni de $arg$, ne pas commencer un calcul en bas de page, énoncer de préférence un résultat avant de le montrer, ne rédigez jamais directement, ... 
 \item un \og purgatoire\fg~ contenant les objets, résultats, techniques, ... que vous savez ne pas maitriser (encore)
\end{itemize}
\bigskip
Vous devez pouvoir mobiliser le contenu de votre boîte à outils, en parler à vos camarades ou à moi même.\newline
Vous devez être capable, devant une situation particulière, de chercher un élément de votre boîte à outils qui vous semble avoir un rapport avec la situation (même de manière très vague) et, éventuellement, être conscient du fait que vous n'en trouvez pas.

\section{Travail personnel}
Par travail personnel, j'entends celui que vous produisez hors de la classe en travaillant sur le cours, les corrigés de DM ou de DS, les énoncés de DM.\newline
Le but de votre travail personnel est d'enrichir et de mettre à jour votre boite à outils. \bigskip
\begin{itemize}
 \item Sur le cours. 
 \begin{itemize}
 \item Cherchez à comprendre et à vous approprier.
 \item Classez dans la boîte y compris en considérant que certaines parties sont secondaires ou à placer dans le purgatoire.
 \end{itemize}
 
 \item Sur les exos. 
 \begin{itemize}
 \item Repérez ce à quoi ils correspondent dans le cours récent et traitez les avec.
 \item Ne perdez pas votre temps: rejetez toute méthode qui n'utiliserait pas le cours récent. Il ne s'agit pas de résoudre l'exo mais de l'utiliser pour alimenter la boite. 
 \item Ne vous préocupez pas trop de la rédaction, elle sera abordée lors de la correction en classe.
 \item Ne cherchez pas d'autres exos que ceux qui sont proposés, reprenez plutôt les anciens, intégrez les dans votre boîte.
 \end{itemize}
 
 \item Sur les DM.
 \begin{itemize}
 \item Ne vous placez pas dans les conditions d'un DS, travaillez avec vos notes. 
 \item Repérez dans les questions ce qui évoque quelque chose dans votre boîte. 
 \item Contrairement à un DS, cherchez d'abord les questions en liaison avec les objets du purgatoire. 
 \item Vous pouvez chercher plus longtemps qu'en DS mais toujours avec votre cours, ne perdez pas de temps avec une solution originale. 
 \item Rédigez bien les questions faciles, ne recopiez pas les corrigés, vous n'avez pas à tout faire.
 \end{itemize}
 
 \item Révision, préparation d'un DS.
 \begin{itemize}
  \item Il s'agit de mettre à jour dans votre boîte ce qui porte sur le domaine de révision ou de préparation. 
  \item Reprendre cours, exos et DM: mettre à jour votre boîte en séparant bien ce que que vous savez faire et ce que vous savez ne pas savoir (purgatoire).
  \item Retravaillez votre cours, pour essayer de comprendre des éléments que vous n'aviez pas compris.
  \item Si vous y parvenez, reprenez vos notes et corrigés pour retrouver ce qui s'y rapporte et bien vous l'approprier.
 \end{itemize}

\end{itemize}


\section{Travail en temps limité}
Les textes de problème utilisés comme outils d'évaluation en temps limité ne sont pas destinés à être traités en entier.\newline
Le but de votre travail en temps limité est de faire ce que vous savez déjà faire et de le montrer clairement.
\begin{itemize}
 \item Lire le texte complètement et repérer les questions qui évoquent quelque chose dans votre boîte à outils.
 \item Commencer par traiter la question qui vous semble la plus facile parmi celles qui évoquent un élément de la boîte. 
 \item Dans le traitement d'une question, penser à utiliser les résultats des questions précédentes et les outils de votre boite selon leurs règles d'utilisation. Veillez à mobiliser les conseils techniques.
 \item \'Ecarter les solutions originales ou utilisant des outils ne figurant pas dans votre boîte.
 \item \'Ecarter les situations en rapport avec le purgatoire. Ce n'est pas le moment de réfléchir sur ces éléments. Cela doit se faire lors de votre travail personnel.
 \item Une fois que vous avez trouvé, vous devez rédiger et présenter. C'est un travail distinct qui est souvent plus long que la recherche car il s'agit de produire un texte qui sera lu par quelqu'un d'autre et qui doit respecter certaines conventions: encadrer les résultats, dégager les étapes, ... Ici encore, veillez à bien mobiliser les conseils.
 \item Ne cherchez pas longtemps une même question. Jamais plus de quelques minutes. Si vous ne trouvez plus rien et qu'il est encore tôt, revenez sur les questions que vous avez déjà traitées et rédigez les à nouveau en respectant mieux les conseils, avec une présentation parfaite. \`A la fin, relisez vous encore en corrigeant les fautes d'orthographe en veillant aux enchainements logiques, ...  
\end{itemize} 


\end{document}

