%<dscrpt>Sommes imbriquées</dscrpt>
\begin{enumerate}
  \item Soit $a$ et $b$ des entiers tels que $0\leq a \leq b$, énoncer et démontrer une expression factorisée de
\begin{displaymath}
  \sum_{k=a}^{b}k
\end{displaymath}

  \item Pour $n$ naturel non nul, on considère la somme double
\begin{displaymath}
  S_2(n) = \sum_{i=1}^{n}\left( \sum_{j=i}^{n}\frac{i}{j}\right) 
\end{displaymath}
\begin{enumerate}
  \item Représenter, dans un plan rapporté à un repère avec les $i$ en abscisses et les $j$ en ordonnées l'ensemble des points de coordonnées $(i,j)$ pour les couples d'indices de la somme.
  \item Calculer $S_2(n)$ en échangeant les sommations.
\end{enumerate}
\item Pour $n$ naturel non nul, calculer la somme triple 
\begin{displaymath}
S_3(n) =  \sum_{i=1}^{n}\left( \sum_{j=i}^{n}\left( \sum_{k=j}^{n} \frac{i}{jk}\right)\right) 
\end{displaymath}
\item Pour tous $n$ et $k$ naturels non nuls, montrer que
\begin{displaymath}
  \sum_{1\leq i_1 \leq i_2 \leq \cdots \leq i_k\leq n}\frac{i_1}{i_2 i_3 \cdots i_k}
  = \frac{n(n+2^k-1)}{2^k}
\end{displaymath}
la somme porte sur les $k$-uplets $(i_1,\cdots,i_k)\in \llbracket 1,n \rrbracket^k$ tels que $1\leq i_1 \leq \cdots \leq i_k \leq n$.
\end{enumerate}
