%<dscrpt>Sommes de nombres complexes. Noyaux de Dirichlet et de Fejer.</dscrpt>
Pour $n$ entier naturel sup{\'e}rieur ou {\'e}gal 1 et $\theta$ r{\'e}el, on
pose \footnote{D'apr{\'e}s CCMP 2001 2 eme {\'e}preuve}
\begin{displaymath}
  D_n(\theta)=\sum_{k=-n+1}^{n-1}e^{ik\theta} \hspace{1.5cm}
  F_n(\theta)=\frac{1}{n}\sum_{j=1}^n D_j(\theta) 
\end{displaymath}
\begin{enumerate}
  \item Sans chercher à calculer $D_n$, montrer que
\begin{displaymath}
F_n(\theta)=\sum_{k=-n+1}^{n-1}(1-\frac{|k|}{n})e^{ik\theta}  
\end{displaymath}
  \item Pour $\theta \not \in 2\pi\Z$, en calculant $D_n(\theta)$ à l'aide d'une somme de termes en progresssion géométrique, exprimer $nF_n(\theta)$ comme le carr{\'e} d'un quotient de sinus. (ne pas chercher à utiliser la première question)
\end{enumerate}
