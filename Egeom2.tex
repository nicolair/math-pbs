%<dscrpt>Intersection d'une famille de droites.</dscrpt>
Soit $a$, $b$, $b^\prime$ trois réels strictement positifs tels que $b\neq b^\prime$. Dans un plan muni d'un repère, on définit trois points $A$, $B$, $B^\prime$ par leurs coordonnées 
\begin{align*}
 A : (a,0) & & B : (0,b) & & B^\prime : (0,b^{\prime})
\end{align*}
Soit $\Delta$ une droite passant par l'origine, on note $m$ la pente de $\Delta$ (lorsque cette pente existe).\newline
On note $\Delta^{\prime}$ la droite symétrique de $\Delta$ par rapport à l'axe $Ox$ du repère.\newline
Lorsque les droites se coupent, on note $M$ le point commun à $\Delta$ et à la droite $(AB)$ et $M^{\prime}$ le point commun à $\Delta^{\prime}$ et à la droite $(AB^{\prime})$.
\begin{enumerate}
 \item Construire sur dessin la droite $(MM^\prime)$ à partir d'une droite $\Delta$.
 \item Préciser les valeurs de $m$ pour lesquelles existent les points $M$ et $M^\prime$. Calculer alors les coordonnées de $M$ et $M^\prime$ et montrer que :
 \begin{displaymath} \left\vert
% use packages: array
\begin{array}{ll}
(b+am)x-ab & a(b+b^\prime) \\ 
(b+am)y-abm & -2bb^\prime -am(b^\prime -b)
 \end{array}  \right\vert
= 0
 \end{displaymath}
est une équation de $(MM^\prime)$.
 \item On se propose de montrer qu'il existe un unique point (noté $P$) appartenant à \emph{toutes les droites} $MM^\prime$.
  \begin{enumerate}
 \item Montrer que si $P$ existe, il est forcément sur la droite $(Oy)$ et que son ordonnée  $p$ vérifie 
\begin{displaymath}
 \frac{2}{p} = \frac{1}{b} + \frac{1}{b^\prime} 
\end{displaymath}
 \item La question a. prouve une partie de ce que l'on souhaitait montrer, laquelle ? Achever la démonstration. 

 \item Montrer qu'il existe un réel $\lambda$ tel que
\begin{align*}
 \overrightarrow{OB} = \lambda \overrightarrow{OB^\prime} & & \overrightarrow{PB} = -\lambda \overrightarrow{PB^\prime}
\end{align*}
\end{enumerate}

\item On suppose que $a$, $b$, $b^\prime$ varient de telle sorte que $P$ soit fixe et que la droite $(AB^\prime)$ reste parallèle à la droite d'équation 
\begin{displaymath}
 x + y = 0
\end{displaymath}
\begin{enumerate}
 \item Calculer $a$ et $b$ en fonction de $b^\prime$ et $p$.
 \item Montrer que les droites joignant $B$ au milieu de $[AB^\prime]$ passent par un unique point (noté $Q$) lorsque $b^\prime$ varie. Préciser les coordonnées de $Q$.
\end{enumerate}


\end{enumerate}



