%<dscrpt>Des exercices sur des calculs dans R et des fonctions usuelles.</dscrpt>
\begin{enumerate}
\item
\begin{enumerate}
\item  Pour $t$ réel, exprimer 
\begin{displaymath}
\frac{e^{2t}-1}{e^{2t}+1}  
\end{displaymath}
à l'aide des fonctions trigonométriques hyperboliques.

\item Pour tout $\varphi$ r{\'e}el non congru à $\frac{\pi}{2}$ modulo $\pi$, simplifier 
\begin{displaymath}
\frac{\tan ^{2}\varphi -1}{\tan ^{2}\varphi+1}
\end{displaymath}

\item  Montrer que, pour tout $t$ r{\'e}el,
\begin{displaymath}
\arccos (\th (t))+2\arctan (e^{t})=\pi
\end{displaymath}

\item  On considère, pour $x\in \left] 0,\frac{\pi }{2}\right[ $, l'{\'e}quation
\begin{displaymath}
\ch t=\frac{1}{\cos x}
\end{displaymath}
d'inconnue $t$. Montrer qu'elle admet une seule solution positive que l'on exprimera {\`a} l'aide de $\frac{\pi }{4}$, $\frac{x}{2}$, $\tan $, $\ln $.

\item  Former et démontrer, suivant les valeurs de $t$, une formule reliant
\begin{displaymath}
\arcsin (\frac{1}{\ch t})\text{ et }\arccos (\th t)
\end{displaymath}
\end{enumerate}

\item  Simplifier l'{\'e}criture des deux nombres r{\'e}els
\begin{align*}
(7+5\sqrt{2})^{\frac{1}{3}}-(-7+5\sqrt{2})^{\frac{1}{3}}
& &\left( \frac{13+5\sqrt{17}}{2}\right) ^{\frac{1}{3}}-\left( \frac{-13+5\sqrt{17}}{2}\right) ^{\frac{1}{3}}
\end{align*}

\item  Lin{\'e}ariser
\begin{displaymath}
\cos x\cos 2x\cos 3x\sin 2x
\end{displaymath}

\item  Montrer que
\begin{displaymath}
\arctan (1+x)-\arctan x=\arctan (\frac{1}{1+x+x^{2}})
\end{displaymath}
\end{enumerate}
