%<dscrpt>Polynômes réels à valeurs positives.</dscrpt>
On dira qu'un polyn{\^o}me est \emph{réel} lorsque que tous ses coefficients sont réels. On dira qu'il est \emph{positif} lorsqu'il est r{\'e}el, non nul et que tous ses coefficients sont positifs ou nuls. On dira qu'il est \emph{à valeurs positives} lorsqu'il est r{\'e}el et que
\begin{displaymath}
\forall x\in \R\text{, }\widetilde{P}(x)\geq 0
\end{displaymath}

\begin{enumerate}
\item  Déterminer, pour les polyn{\^o}mes suivants, s'ils vérifient ou non les propriétés définies au dessus
\begin{displaymath}
X^{2}+X+1,\quad X^{2}-X+1,\quad X^{2}+2X+\frac{1}{2}
\end{displaymath}

\item  Soit $m$ un entier non nul. D{\'e}terminer l'ensemble des $\theta $ de $\left] 0,\pi \right[$ vérifiant
\begin{displaymath}
\sin 2\theta \geq 0,\;\sin 3\theta \geq 0 ,\;\cdots, \; \sin (m+1)\theta \geq 0
\end{displaymath}

\item  Soit $C=X^{2}+c_{1}X+c_{2}$ un polyn{\^o}me r{\'e}el sans racine r{\'e}elle. Ses racines complexes sont notées $u_{1}$ et $u_{2}$ avec $\Im u_{1}>0$. On note $\phi \in ]-\pi , \pi]$ l'argument principal de $u_{1}$ et $r$ son module.

\begin{enumerate}
\item  Soit $m\in \N^*$. Montrer que le polyn{\^o}me
\begin{displaymath}
D_{m}=(X^{m+1}-u_{1}^{m+1})(X^{m+1}-u_{2}^{m+1})
\end{displaymath}
est r{\'e}el. Sous quelle condition sur $m$ et $\phi $ est-il positif?

\item  Préciser, sous la forme d'un produit, un polyn{\^o}me $B_{m}$ tel que $B_{m}C=D_{m}$.

\item  On pose $B_{m}=b_{0}+b_{1}X+\cdots +b_{2m}X^{2m}$. Pour $k\in\left\{ 0,\cdots ,m\right\}$, exprimer $b_{k}$ et $b_{2m-k}$ sous une forme trigonom{\'e}trique simple.

\item Montrer qu'il existe un entier $m$ tel que $B_m$ et $D_m$ soient positifs.
\end{enumerate}

\item  Soit $C$ un polyn{\^o}me r{\'e}el, de coefficient dominant strictement positif, sans racine r{\'e}elle. Montrer qu'il existe deux polyn{\^o}mes positifs $B$ et $D$ tels que $BC=D$.
\end{enumerate}
