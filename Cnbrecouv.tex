\begin{enumerate}
  \item Exemple. Choisissons $A=\{1,2,3\}$, $B=\{x,y,z,t\}$
  \[f=\begin{pmatrix}
    1 & \rightarrow & \{x,y\} \\
    2 & \rightarrow & \{x\} \\
    3 & \rightarrow & \emptyset \
  \end{pmatrix}\]
  alors
\[\varphi=\begin{pmatrix}
    x & \rightarrow & \{1,2\} \\
    y & \rightarrow & \{1\} \\
    z & \rightarrow & \emptyset \\
    t & \rightarrow & \emptyset
  \end{pmatrix}\]
  \item On peut reformuler les d{\'e}finitions de $\Phi_{AB}$ et de $\Phi_{BA}$
  \begin{eqnarray*}
  \forall f \in \mathcal{F}(A,\mathcal{P}(B)), \forall a\in A,
  \forall b \in B : a\in (\Phi_{AB}(f))(b) \Leftrightarrow b\in
  f(a) \\
  \forall g \in \mathcal{F}(B,\mathcal{P}(A)), \forall a\in A,
  \forall b \in B : b\in (\Phi_{BA}(g))(b) \Leftrightarrow a\in
  g(b)
  \end{eqnarray*}
  Par cons{\'e}quent, $\forall f \in \mathcal{F}(A,\mathcal{P}(B)), \forall a\in A,
  \forall b \in B$ on a
  \begin{eqnarray*}
  b\in(\Phi_{BA}\circ \Phi_{AB}(f))(a) & \Leftrightarrow & b\in(\Phi_{BA}(
  \Phi_{AB}(f)))(a)\\
  & \Leftrightarrow & a\in \Phi_{AB}(f))(b)\\
  & \Leftrightarrow & b \in f(a)
  \end{eqnarray*}
  Ceci {\'e}tant valable pour tous les $f \in
  \mathcal{F}(A,\mathcal{P}(B))$ on a bien
  \[\Phi_{BA}\circ \Phi_{AB}=Id_{\mathcal{F}(A,\mathcal{P}(B))}\]
  On d{\'e}montre de la m{\^e}me mani{\`e}re que
  \[\Phi_{AB}\circ \Phi_{BA}=Id_{\mathcal{F}(B,\mathcal{P}(A))}\]
  On tire la surjectivit{\'e} de $\Phi_{AB}$ de la premi{\`e}re {\'e}quation et
  l'injectivit{\'e} de la deuxi{\`e}me. Les applications $\Phi_{AB}$ et
  $\Phi_{BA}$ sont donc des bijections r{\'e}ciproques l'une de
  l'autre.\newline
  Supposons $A$ et $B$ finis avec respectivement $\alpha$ et $\beta$
  {\'e}l{\'e}ments. D'apr{\`e}s le cours sur le d{\'e}nombrement des applications
  et celui des parties d'un ensemble, on peut {\'e}crire
  \[\textrm{Card}(\mathcal{F}(A,\mathcal{P}(B)))=\textrm{Card}(\mathcal{F}(A,\mathcal{P}(B)))
  =(2^\beta)^\alpha=2^{\alpha \beta}\]
  \item On suppose maintenant que $A=\{1,\cdots , n\}$
    et $B=E$. Si $f$ est une application de $A$ dans
    $\mathcal{P}(E)$, on pose
    \[E_1=f(1),E_2=f(2),\cdots, E_n=f(n), \varphi=\Phi_{AB}(f)\]
     \begin{enumerate}
        \item D{\'e}signons par $\Omega_i$ l'ensemble des parties
        de $A$ contenant $i$ alors
        \begin{eqnarray*}
        x\in \varphi ^{-1}(\Omega_i) & \Leftrightarrow &
        \varphi(x)\in \Omega_i \\
        & \Leftrightarrow & i\in \varphi (x) \\
        & \Leftrightarrow & x \in f(i)
        \end{eqnarray*}
        Donc $\varphi ^{-1}(\Omega_i)=f(i)$.
        \item Utilisons encore la relation fondamentale
        \[\forall a \in A, \forall b \in B : a\in \varphi (b) \Leftrightarrow b\in f(a)\]
        On peut {\'e}crire
        \begin{eqnarray*}
        \emptyset \in \varphi(E) & \Leftrightarrow & \exists b \in
        B \textrm{ tq }\varphi(b)=\emptyset \\
        & \Leftrightarrow & \exists b \in
        B \textrm{ tq } \forall a \in A : a\not \in \varphi(b)\\
        & \Leftrightarrow & \exists b \in
        B \textrm{ tq } \forall a \in A : b\not \in f(a)\\
        & \Leftrightarrow & \exists b \in
        B \textrm{ tq } b\not \in \Omega_1 \cup \cdots \cup
        \Omega_n
        \end{eqnarray*}
        L'{\'e}quivalence des n{\'e}gations des deux propositions fournit
        la relation demand{\'e}e.\newline
        On en d{\'e}duit que le nombre de $n$-recouvrements est aussi
        le nombre d'applications de $E$ vers
        $\mathcal{P}(E)-{\emptyset}$ soit (si $p$ est le nombre
        d'{\'e}l{\'e}ments de $E$) :
        \[(2^p-1)^n\]
     \end{enumerate}
\end{enumerate}
