\begin{enumerate}
  \item Le système de relations vérifié par les $a_i$ demandés est triangulaire:
\begin{align*}
  a_0  &       &      &       &      &       &      &       &     & = & 1\\
  a_0  & {}+{} & a_1  &       &      &       &      &       &     & = & 1\\
  a_0  & {}+{} & 2a_1 & {}+{} & a_2  &       &      &       &     & = & 2 \\
  a_0  & {}+{} & 3a_1 & {}+{} & 3a_2 & {}+{} & a_3  &       &     & = & 6\\
  a_0  & {}+{} & 4a_1 & {}+{} & 6a_2 & {}+{} & 4a_3 & {}+{} & a_4 & = & 24 
\end{align*}
On en tire $a_0=1$, $a_1=0$, $a_2=1$, $a_3=2$, $a_4=9$. Comme le système est triangulaire avec un coefficient $1$ devant $a_n$, le dernier $a_n$ s'exprime en fonction des $a_k$ précédents et il est entier.
\item On exprime les coefficients du binôme avec des factorielles
\begin{multline*}
\binom{m}{k} \binom{n}{m}
= \frac{m!}{k!\, (m-k)!} \, \frac{n!}{m!\,(n-m)!}
= \frac{n!}{k!\, (m-k)!\, (n-m)!}\\
=\frac{n!}{k!\, (n-k)!}\,\frac{(n-k)!}{(m-k)!\, (n-m)!}
= \binom{n}{k} \binom{n-k}{m-k}
\end{multline*}
Cela permet de faire apparaitre une formule du binôme
\begin{multline*}
\sum_{m \in \llbracket k,n \rrbracket} \binom{m}{k}\binom{n}{m}\,z^m
= \binom{n}{k}\sum_{m \in \llbracket k,n \rrbracket} \binom{n-k}{m-k}z^{m}\\
= \binom{n}{k}\left( \sum_{m \in \llbracket k,n \rrbracket} \binom{n-k}{m-k}z^{m-k}\right) z^k
= \binom{n}{k}\left( \sum_{i \in \llbracket 0,n-k \rrbracket} \binom{n-k}{i}z^{i}\right) z^k\\
= \binom{n}{k}(1+z)^{n-k} z^k 
\end{multline*}
\item La sommation sur le triangle s'écrit avec des doubles sommes
\begin{displaymath}
\sum_{(m,k)\in \mathcal{T}} t_{m,k} 
= \sum_{m\in \llbracket 0, n\rrbracket} \left(\sum_{k\in \llbracket 0, m\rrbracket} t_{m,k}\right)
= \sum_{k\in \llbracket 0, n\rrbracket} \left(\sum_{m\in \llbracket k, n\rrbracket} t_{m,k}\right)   
\end{displaymath}
\item On écrit la somme à droite comme une double somme et on intervertit les sommations:
\begin{multline*}
\sum_{m\in \llbracket 0,n \rrbracket}m!\,\binom{n}{m}(-1)^m
= \sum_{m\in \llbracket 0,n \rrbracket}\left(\sum_{k\in \rrbracket0,m\llbracket}\binom{m}{k}a_k \right) \,\binom{n}{m}(-1)^m\\
= \sum_{k\in \llbracket 0,n \rrbracket}a_k\left(\sum_{m\in \llbracket k,n\rrbracket}\binom{m}{k} \,\binom{n}{m}(-1)^m\right)
= \sum_{k\in \llbracket 0,n \rrbracket}a_k \binom{k}{n}(1-1)^{n-k}(-1)^k
\end{multline*}
d'après la question 2 avec $z=-1$. Le seul indice $k$ qui contribue de manière non nulle à la somme est donc $k=n$. On en tire
\begin{displaymath}
\sum_{m\in \llbracket 0,n \rrbracket}m!\,\binom{n}{m}(-1)^m
=a_n \binom{n}{n} (-1)^n = (-1)^n a_n 
\end{displaymath}

\end{enumerate}
