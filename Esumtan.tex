%<dscrpt>Fonctions usuelles, relations entre coefficients et racines, somme de tan</dscrpt>
Soit $n$ un entier naturel non nul et $k$ un entier relatif. On
pose
\begin{eqnarray*}
P &=&\sum_{p=0}^{E(\frac{n}{2})}(-1)^{p}\binom{n}{2p}X^{p} \\
x_{k} &=&\frac{(2k-1)\pi }{2n}
\end{eqnarray*}

\begin{enumerate}
\item  Soit $x$ un nombre r{\'e}el qui n'est pas congru {\`a} $\frac{\pi }{2}
$ modulo $\pi $, montrer que
\[
\cos nx=\cos ^{n}x\widetilde{\,P}(\tan ^{2}x)
\]

\item  Montrer que
\[
\left\{ \tan ^{2}x_{k},k\in \Z\right\} =\left\{ \tan
^{2}x_{k},k\in \left\{ 1,2,\cdots ,E(\frac{n}{2})\right\} \right\}
\]

\item  D{\'e}terminer l'ensemble des racines de $P$.

\item  Soit $m$ un entier naturel non nul, exprimer les sommes et produits
suivants en fonction de $m$%
\begin{eqnarray*}
&&\sum_{k=1}^{m}\tan ^{2}\frac{(2k-1)\pi }{4m} \\
&&\prod_{k=1}^{m}\tan ^{2}\frac{(2k-1)\pi }{4m} \\
&&\sum_{k=1}^{m}\tan ^{2}\frac{(2k-1)\pi }{4m+2} \\
&&\prod_{k=1}^{m}\tan ^{2}\frac{(2k-1)\pi }{4m+2}
\end{eqnarray*}
\end{enumerate}
