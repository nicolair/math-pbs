%<dscrpt>Exercice sur des sous-espaces supplémentaires.</dscrpt>
Soit $A$ et $B$ deux sous-espaces vectoriels supplémentaires d'un $\K$-espace vectoriel $E$ de dimension finie. Soit $f$ une application linéaire de $A$ dans $B$ et $g$ une application définie dans $A$ par :
\begin{displaymath}
 \forall a \in A : g(a)= a +f(a)
\end{displaymath}
\begin{enumerate}
 \item Préciser l'ensemble d'arrivée de $g$. L'application $g$ est-elle linéaire ?
\item Montrer que $g$ est injective.
\item Montrer que $g(A)$ est un suppléméntaire de $B$.
\end{enumerate}
