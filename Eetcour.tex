%<dscrpt>Etude d'une courbe paramétrée.</dscrpt>
\subsubsection*{{\'E}tude d'une courbe plane\footnote{Partie I de
Centrale Sup{\'e}lec 2001 MP 1}}
\begin{enumerate}
  \item Soit $z=\rho e^{i\theta}$ un nombre complexe ($\rho$ et $\theta$
  r{\'e}els et $\rho$ non nul). D{\'e}terminer en fonction de $\rho$ et
  $\theta$ une forme trigonom{\'e}trique du nombre complexe
  \[\xi=ze^{-z}\]
  \item D{\'e}montrer que la fonction $u$ d{\'e}finie sur $]0,1]$ par
  \[u(t)=\frac{1+\ln t}{t}\]
  d{\'e}finit une bijection de $]0,1]$ sur $]-\infty,1]$. Montrer que
  sa bijection r{\'e}ciproque est de classe $\mathcal{C}^1$ sur
  $]-\infty,1[$.
  \item En d{\'e}duire l'existence d'une fonction
  $\theta\mapsto r(\theta)$ de $\Bbb{R}$ vers $]0,1]$ telle que
  pour tout r{\'e}el $\theta$
  \[r(\theta)e^{-r(\theta)\cos \theta}=\frac{1}{e}\]
  \item Montrer que $r$ est une fonction continue,
  $2\pi$-p{\'e}riodique et paire. D{\'e}montrer qu'elle est de classe
  $\mathcal{C}^1$ sur $]0,2\pi[$ et que,
  \[\forall \theta \in ]0,2\pi[ : r'(\theta)=\frac{r(\theta)^2\sin \theta}{r(\theta)\cos \theta -1}\]
  \item Donner un {\'e}quivalent simple {\`a} droite de 0 de la fonction (not{\'e}e
  $w$)
  d{\'e}finie par
  \[h\rightarrow 1-u(1-h)\]
  La fonction $u$ est d{\'e}finie en 2.. Gr{\^a}ce {\`a} une expression de
  $1-\cos \theta$ {\`a} l'aide de $u(r(\theta))$, en d{\'e}duire que {\`a}
  droite de 0:
  \[r(\theta)=1-\theta +o(\theta)\]
  Prouver que $r$ est de
\end{enumerate}
