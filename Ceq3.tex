Introduisons les polynômes symétriques élémentaires
\begin{align*}
 \sigma_1 = x_1+x_2+x_3 & & \sigma_2 = x_1x_2+x_1x_3+x_2x_3 & & \sigma_3=x_1x_2x_3
\end{align*}
et utilisons les pour exprimer chacune des trois relations.\newline
La première relation donne immédiatement $\sigma_1^2 -2\sigma_2 = 1$. \newline
On obtient la deuxième relation en réduisant au même dénominateur $x_1x_2x_3$:
\begin{displaymath}
 \frac{\sigma_1}{\sigma_3}=\frac{3}{2}
\end{displaymath}
Pour la troisième, on commence par regrouper par trois les termes qui ont le même dénominateur. Cela fait apparaitre le $\sigma_1$ au numérateur que l'on peut mettre en facteur:
\begin{displaymath}
 (x_1+x_2+x_3)\left( \frac{1}{x_1}+\frac{1}{x_2}+\frac{1}{x_3}\right) =6
\end{displaymath}
On réduit au même dénominateur $x_1x_2x_3$ ce qui donne directement: $\frac{\sigma_1\sigma_2}{\sigma_3}=6$.\newline
\`A partir des deux dernières relations, on obtient $\sigma_2=4$.\newline
En remplaçant dans la première, on obtient
\begin{displaymath}
 \sigma_1^2 = 9
\end{displaymath}
Le dernier s'obtient à partir de $\sigma_1$: $\sigma_3 = \frac{2}{3}\sigma_1$. 
On a donc deux possibilités:
\begin{align*}
 \sigma_1&=3 & \sigma_3&=2 & \sigma_2&= 4 \\
 \sigma_1&=-3 & \sigma_3&=-2 & \sigma_2&= 4 
\end{align*}
Les nombres $x_1$, $x_2$, $x_3$ sont donc (à permutation près) les racines des polynômes
\begin{displaymath}
 X^3-3X^2+4X-2, \hspace{1cm} X^3+3X^2+4X+2
\end{displaymath}
après factorisation à l'aide de la racine évidente $1$ ou $-1$ avec une division par $X-1$ ou $X+1$, on obtient les triplets $(1, 1 + i,1 - i)$ et $(-1,-1 + i,-1 - i)$.

