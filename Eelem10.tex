%<dscrpt>Des exercices sur des calculs dans R et des fonctions usuelles.</dscrpt>
\begin{enumerate}
\item  R{\'e}soudre
\[
\left( \frac{z+i}{z-i}\right) ^{3}+\left( \frac{z+i}{z-i}\right) ^{2}+\left(
\frac{z+i}{z-i}\right) +1=0
\]

\item  Si $a_{0},a_{1},\cdots ,a_{n}$ sont des nombres complexes et
\[
P(z)=a_{0}+a_{1}z+\cdots +a_{n}z^{n}
\]
comment s'exprime $P(z)-P(-z)$ ?\newline
R{\'e}soudre
\[
z^{7}+\binom{7}{2}z^{5}+\binom{7}{4}z^{3}+\binom{7}{6}z=0
\]

\item  D{\'e}terminer l'ensemble des r{\'e}els $x$ tels que
\[
\sin x+\sin 2x<\sin 3x
\]

\item  Montrer que
\[
\arctan 1+\arctan 2+\arctan 3=\pi
\]


\item  Pr{\'e}senter dans un tableau les valeurs de $\arccos (\left| \cos
x\right| )$ et $\arcsin (\left| \sin x\right| )$ dans les intervalles $%
\left[ 0,\frac{\pi }{2}\right] $, $\left[ \frac{\pi }{2},\pi \right] $, $%
\left[ \pi ,\frac{3\pi }{2}\right] $, $\left[ \frac{3\pi }{2},2\pi \right] $

\end{enumerate}
