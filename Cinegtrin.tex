\begin{enumerate}
  \item Inégalité de Cauchy-Schwarz.
\begin{enumerate}
  \item En développant $T$, il se met naturellement sous la forme d'un trinôme du second degré
\begin{displaymath}
  T(t) = \left( \sum_{i=1}^nb_i^2\right)t^2 -2\left( \sum_{i=1}^na_ib_i\right)t + \left( \sum_{i=1}^na_i^2\right) 
\end{displaymath}
Son discriminant s'exprime comme
\begin{displaymath}
  \Delta = 4\left( \left( \sum_{i=1}^na_ib_i\right)^2 - \left( \sum_{i=1}^na_i^2\right)\left( \sum_{i=1}^nb_i^2\right)\right) 
\end{displaymath}

  \item Pour tout $t$ réel, $T(t)\geq 0$. L'équation $T(z)=0$ admet donc \emph{au plus} une solution réelle. Le discriminant du trinôme est donc négatif ou nul. Comme tout est strictement positif, on obtient bien (en divisant par le carré de la somme des produits)
\begin{displaymath}
\left( \sum_{i=1}^na_ib_i\right)^2 - \left( \sum_{i=1}^na_i^2\right)\left( \sum_{i=1}^nb_i^2\right)\leq 0
\Rightarrow
1 \leq \frac{\left( \sum_{i=1}^na_i^2\right)\left( \sum_{i=1}^nb_i^2\right)}{\left( \sum_{i=1}^na_ib_i\right)^2}
\end{displaymath}

\end{enumerate}
  \item Un autre trinôme.
\begin{enumerate}
  \item La définition de $T$ entraine $T(0) = \sum_{i=1}^{n}a_i^2 > 0$ et 
  \begin{displaymath}
T(1) = \sum_{i=1}^{n}(a_i-\lambda b_i)(a_i-\mu b_i)
  \end{displaymath}
Or $\lambda < \frac{a_i}{b_i} < \mu$ pour tout $i$ donc $(a_i-\lambda b_i)> 0$ et $(a_i-\mu b_i)< 0$. On en déduit $T(1)< 0$ comme somme de termes strictement négatifs.
  \item Comme le trinôme change cette fois de signe, l'équation associée admet deux solutions réelles donc le discriminant doit être strictement positif. Développons pour exprimer ce discriminant (noté encore $\Delta$)
\begin{multline*}
T(t) = \lambda \mu\left( \sum_{i=1}^{n}b_i^2\right)t^2 -(\lambda + \mu)\left( \sum_{i=1}^{n}a_ib_i\right)t + \left( \sum_{i=1}^{n}a_i^2\right) \\
\Rightarrow
\Delta = (\lambda + \mu)^2\left( \sum_{i=1}^{n}a_ib_i\right)^2 -4\lambda \mu\left( \sum_{i=1}^{n}b_i^2\right)\left( \sum_{i=1}^{n}a_i^2\right)>0
\end{multline*}
On en déduit:
\begin{displaymath}
\frac{\left( \sum_{i=1}^{n}a_i^2\right)\left( \sum_{i=1}^{n}b_i^2\right)}{\left( \sum_{i=1}^{n}a_ib_i\right)^2}<\frac{(\lambda + \mu)^2}{4\lambda \mu}  
\end{displaymath}

  \item Lors des questions précédentes, les $a$, $b$, $A$, $B$ ne sont pas intervenus. Seuls les $\lambda$ et $\mu$ avec leur propriété fondamentale 
\begin{displaymath}
\forall i\in \{1,\cdots,n\}; \hspace{0.5cm} \lambda < \frac{a_i}{b_i} < \mu  
\end{displaymath}
  ont figurés dans le raisonnement. En fait, on peut \emph{fabriquer} de tels $\lambda$ et $\mu$ à partir des $a, A, b, B$. \`A cause des propriétés élémentaires des inégalités (tous les nombres étant strictement positifs):
\begin{displaymath}
\forall i\in \{1,\cdots,n\}; \hspace{0.5cm} \frac{a}{B} < \frac{a_i}{b_i} < \frac{A}{b}  
\end{displaymath}
Avec $\lambda = \frac{a}{B}$ et $\mu = \frac{A}{b}$,
\begin{displaymath}
\frac{(\lambda + \mu)^2}{4\lambda \mu} =\frac{1}{4}\left(\frac{\lambda + \mu}{\sqrt{\lambda \mu}} \right)^2
=\frac{1}{4}\left(\sqrt{\frac{\lambda}{\mu}} +\sqrt{\frac{\mu}{\lambda}}\right)^2  
=\frac{1}{4}\left(\sqrt{\frac{ab}{AB}} +\sqrt{\frac{AB}{ab}}\right)^2
\end{displaymath}
\end{enumerate}
\item Lorsque les $\lambda$ et $\mu$ sont respectivement le plus petit et le plus grand des $\frac{a_i}{b_i}$, l'inégalité n'est plus forcément valable. En effet, certains termes dans la somme exprimant $T(1)$ peuvent être nuls et il est même possible que \emph{tous} les termes soient nuls si les quotients ne prennent que deux valeurs.\newline
Dans un tel cas, $T(1)$ est nul et on ne peut rien déduire sur le discriminant du trinôme.\newline
Considérons par exemple $n=3$ avec $a_1=a_2=1$, $a_3=2$, $b_1=b_2=b_3=1$. Alors:
\begin{multline*}
\sum_{i=1}^{n}a_i^2 = 6,\; \sum_{i=1}^{n}b_i^2 = 3,\; \sum_{i=1}^{n}a_ib_i = 4,\;
\frac{\left( \sum_{i=1}^{n}a_i^2\right)\left( \sum_{i=1}^{n}b_i^2\right)}{\left( \sum_{i=1}^{n}a_ib_i\right)^2} = \frac{9}{8} \\
a=1,\; A=2,\; b=B=1,\; \lambda =1,\; \mu = 2,\; \frac{(\lambda + \mu)^2}{4\lambda \mu} = \frac{9}{8},\\
\frac{1}{4}\left(\sqrt{\frac{ab}{AB}} +\sqrt{\frac{AB}{ab}}\right)^2=\frac{1}{4}(\frac{1}{\sqrt{2}}+\sqrt{2}) = \frac{9}{8}
\end{multline*}
On constate dans ce cas particulier que l'inégalité devient une égalité. Dans tous les cas, l'inégalité large est valable. la justification est de nature analytique. Imaginons des suites $\left( \lambda_n\right)_{n\in \N}$ et $\left( \mu_n\right)_{n\in \N}$ qui convergent respectivement vers $\lambda$ et $\mu$ avec $\lambda_n < \lambda$ et $\mu< \mu_n$. L'inégalité stricte sera alors valable avec les $\lambda_n$ et $\mu_n$. En passant à la limite, on obtiendra une inégalité large. 
\end{enumerate}
