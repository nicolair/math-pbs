\begin{enumerate}
\item Notons $S_{0}$ la surface d'équation $xy+yz+zx=0$. Comme cette équation est homogène de degré 2, toutes les droites passant par l'origine et un point de $S_{0}$ sont sur $S_{0}$. Montrons que ce sont les seules.

Soit $A_{1}(x_{1},y_{1},z_{1})$ et $A_{2}(x_{2},y_{2},z_{2})$ deux points de la surface. La droite $(A_{1}A_{2})$ est l'ensemble des barycentres de $A_{1}$ et $A_{2}$ avec des coefficients $\lambda,1-\lambda$ ($\lambda$ réel quelconque). Les coordonnées d'un tel barycentre $M_{\lambda}$ sont
\[(\lambda x_{1}+(1-\lambda)x_{2},\lambda y_{1}+(1-\lambda)y_{2},\lambda y_{1}+(1-\lambda)y_{2})\]
Ecrivons que $ M_{\lambda}$ est sur $S_{0}$ en regroupant les termes suivant $\lambda$ et en tenant compte des équations vérifiées par $A_{1}$ et $A_{2}$. On obtient
\[\lambda^{2}0+\lambda(1-\lambda)(x_{1}y_{2}+y_{1}z_{2}+z_{1}x_{2}+ y_{1}x_{2}+z_{1}y_{2}+x_{1}z_{2})+(1-\lambda)^{2}0=0\]
On en déduit qu'une droite dans $S_{0}$ passant par $A_{1}(x_{1},y_{1},z_{1})$ est l'intersection de $S_{0}$ avec le plan d'équation
\[(y_{1}+z_{1})x+(z_{1}+x_{1})y+(x_{1}+y_{1})z=0\]
Comme l'origine est dans cette intersection, la droite passant par l'origine est la seule possible.
\item Pour la surface de niveau -1, le calcul barycentrique est analogue. Il conduit à
\begin{eqnarray*}
-\lambda^{2}+\lambda(1-\lambda)(x_{1}y_{2}+y_{1}z_{2}+z_{1}x_{2}+ y_{1}x_{2}+z_{1}y_{2}+x_{1}z_{2})-(1-\lambda)^{2}=-1\\
x_{1}y_{2}+y_{1}z_{2}+z_{1}x_{2}+ y_{1}x_{2}+z_{1}y_{2}+x_{1}z_{2}=-2
\end{eqnarray*}
Pour la surface de niveau +1, ontrouve de même
\begin{eqnarray*}
\lambda^{2}+\lambda(1-\lambda)(x_{1}y_{2}+y_{1}z_{2}+z_{1}x_{2}+ y_{1}x_{2}+z_{1}y_{2}+x_{1}z_{2})+(1-\lambda)^{2}=1\\
x_{1}y_{2}+y_{1}z_{2}+z_{1}x_{2}+ y_{1}x_{2}+z_{1}y_{2}+x_{1}z_{2}=2
\end{eqnarray*}
Les droites qui passent par un point donné de la surface sont donc l'intersection de la surface par un certain plan qui dépend du point. En fait ce plan est le plan tangent à la surface. Pour la surface de niveau -1 par un point passent deux droites, pour la surface de niveau 1 l'intersection est formée par lepoint tout seul. Il ne passe par de droite incluse dans la surface. Pour le prouver il faut faire un changement de variable. Voir la feuille de calculs Maple quadrique.mws qui permet aussi d'obtenir les figures suivantes.
\begin{center}
\includegraphics[width=4.5cm]{sreglC1.eps}
\hfill
\includegraphics[width=4.5cm]{sreglC2.eps}
\includegraphics[width=4.5cm]{sreglC3.eps}
\hfill
\includegraphics[width=4.5cm]{sreglC4.eps}
\end{center}
\end{enumerate}

