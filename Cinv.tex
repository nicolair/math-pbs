\subsection*{Partie I. Images}
\begin{enumerate}
 \item Introduisons le point $U$ d'affixe $u$ pour décrire les objets proposés.
\begin{itemize}
 \item $\mathcal D(u)$ est la droite passant par $U$ et orthogonale à la droite $(OU)$.
\item $\Delta (u)$ est la droite $(OU)$.
\item $\mathcal C^*(u)$  est le cercle de centre $U$ et de rayon $|u|$ (donc passant par $O$) mais privé du point $O$.
\item $\mathcal C(u,r)$ est le cercle de centre $U$ et de rayon $r$. Comme $|u|\neq r$, le point $O$ n'est pas dans ce cercle.
\end{itemize}

\item Comme on est en début d'année et que cette question est isolée, on fournit une rédaction détaillée. Dans un autre contexte, un tel résultat serait considéré comme évident.\newline
Remarquons d'abord que, pour tous les complexes non nuls $z$, $\mathcal I \circ \mathcal I (z)=z$.\newline 
Supposons $M\in \mathcal I (\mathcal E)$, il existe alors un $A\in \mathcal E$ tel que $M=\mathcal I (A)$. Donc $\mathcal I(M)=A$ est un élément de $\mathcal E$.\newline
Réciproquement, si $\mathcal I(M)\in \mathcal E$, il existe alors un $A\in \mathcal E$ (à savoir $\mathcal I (M)$ tel que $M=\mathcal I (A)$.

\item Les résultats sont présentés dans la table \ref{tab:images} (Images). Les calculs sont détaillés seulement pour $\mathcal D(u)$ et $\mathcal C(u,r)$. La troisième ligne se déduit de la première par 2.
\begin{table}[h]
 \centering
 \renewcommand{\arraystretch}{2.5}
 \begin{tabular}{|c|c|}
\hline
 $\mathcal E$ & $\mathcal I(\mathcal E)$ \\
\hline
 $\mathcal D(u)$ & $\mathcal C^*(\frac{1}{2\overline{u}})$\\
\hline  $\Delta(u)$ & $\Delta(u)$\\
\hline $\mathcal C^*(u)$ & $\mathcal D(\frac{1}{2\overline{u}})$\\
\hline $\mathcal C(u,r)$ & $\mathcal C \left(\frac{u}{|u|^2-r^2},\frac{r}{\left\vert |u|^2-r^2\right\vert } \right) $\\
\hline
\end{tabular}
 \caption{Images}
 \label{tab:images}
\end{table}

\begin{itemize}
 \item Image de $\mathcal D(u)$. Un point $M$ d'affixe $z\in\mathcal D(u)$ si et seulement si :
\begin{multline*}
  \frac{\frac{1}{\overline{z}}-u}{u}\in i\R \Leftrightarrow \frac{1-\overline{z}u}{\overline{z}u}\in i\R \Leftrightarrow
z\overline{u}(1-\overline{z}u) \in i\R \Leftrightarrow z\overline{u} - |\overline{z}u|^2\in i\R \\ 
\Leftrightarrow \Re(z\overline{u}) - |\overline{z}u|^2 =0 \Leftrightarrow |z\overline{u} -\frac{1}{2}|^2=\frac{1}{4}
\Leftrightarrow\left\vert z-\frac{1}{2\overline{u}}\right\vert=\frac{1}{2|u|} 
\end{multline*}

\item Image de $\mathcal C(u,r)$. Un point $M$ d'affixe $z\in\mathcal C(u,r)$ si et seulement si :
\begin{multline*}
 \left\vert \frac{1}{\overline{z}}-u\right\vert^2 = r^2 
\Leftrightarrow \left\vert 1-u\overline{z}\right\vert^2 = r^2|\overline{z}|^2
\Leftrightarrow 1 -2 \Re(u\overline{z}) + |u|^2|z|^2=r^2 |z|^2 \\
\Leftrightarrow (|u|^2-r^2)|z|^2\Re(u\overline{z}) +1 = 0
\Leftrightarrow |z|^2Re\left( \frac{u}{|u|^2-r^2}\overline{z}\right)+ \frac{1}{|u|^2-r^2} = 0 \\
\Leftrightarrow \left \vert z - \frac{u}{|u|^2-r^2}\right\vert = \frac{|u|^2}{(|u|^2-r^2)^2}-\frac{1}{|u|^2-r^2}=\frac{r^2}{(|u|^2-r^2)^2}
\end{multline*}
\end{itemize}
\end{enumerate}

\subsection*{Partie II}
\begin{enumerate}
 \item Les conditions d'orthogonalité sont immédiates à obtenir. Elles sont rassemblées dans la table \ref{tab:orthog} (Orthogonalités).
\begin{table}[h]
 \centering
 \renewcommand{\arraystretch}{2.2}
 \begin{tabular}{|c|c|}
\hline
 $\mathcal D(u) \perp \mathcal D(v)$      & $\Re\left(u\overline{v}\right)=0$ \\ \hline
 $\mathcal D(u) \perp \Delta(v)$          & $\Im \left( u\overline{v}\right) =0$ \\ \hline
 $\mathcal D(u) \perp \mathcal C^*(v)$    & $\Re\left(\frac{v}{u}\right) =1$ \\ \hline
 $\mathcal D(u) \perp \mathcal C(v,\rho)$ & $\Re\left(\frac{v}{u}\right) =1$ \\ \hline
 $\mathcal C^*(u) \perp \mathcal C^*(v)$  & $\Re \left( u\overline{v}\right) =0$\\ \hline
 $\mathcal C(u,r) \perp \mathcal C^*(v)$  & $ |u-v|^2 = r^2 +|v|^2$ \\
\hline
\end{tabular}
 \caption{Orthogonalités}
 \label{tab:orthog}
\end{table}
On peut remarquer que l'énoncé n'envisage pas les 10 cas possibles. Deux cas en particulier manquent et qui seront utiles dans la suite:
\begin{displaymath}
  \Delta(u) \perp \mathcal C(v,r) \Leftrightarrow \frac{v}{u} \in \R \hspace{0.5cm}\text{ et }\hspace{0.5cm}
  \Delta(u) \perp \mathcal C^*(v) \Leftrightarrow \frac{v}{u} \in \R 
\end{displaymath}

\item Les conditions d'orthogonalités sont lues dans les tableaux précédents et rassemblées dans la table \ref{tab:consverorth}.
\begin{table}[h]
 \centering
 \renewcommand{\arraystretch}{2.5}
 \begin{tabular}{|c|c|c|c|c|c|c|}
\hline
cas & $\mathcal E$      & $\mathcal E'$        & cond. orth.                       & $\mathcal I (\mathcal{E})$              & $\mathcal I (\mathcal{E}')$             & cond. orth. image     \\ \hline
1   & $\mathcal D(u)$   & $\mathcal D(v)$      & $\Re\left(u\overline{v}\right)=0$ & $\mathcal C^*(\frac{1}{2\overline{u}})$ & $\mathcal C^*(\frac{1}{2\overline{v}})$ & $\Re\left( \frac{1}{u\overline{v}}\right)=0$ \\ \hline
2   & $\mathcal D(u)$   & $\mathcal C^*(v)$    & $\Re \left(\frac{v}{u}\right)=1$  & $\mathcal C^*(\frac{1}{2\overline{u}})$ & $\mathcal D(\frac{1}{2\overline{v}})$   & $\Re\left( \frac{\overline{v}}{\overline{u}}\right)=1$ \\ \hline
3   & $\mathcal C^*(u)$ & $\mathcal C^*(v)$    & $\Re\left(u\overline{v}\right)=0$ & $\mathcal D(\frac{1}{2\overline{u}})$   & $\mathcal D(\frac{1}{2\overline{v}})$   & $\Re\left( \frac{1}{u\overline{v}}\right)=0$ \\ \hline
4   & $\mathcal C^*(u)$ & $\mathcal C(v,\rho)$ &                                   & $\mathcal D(\frac{1}{2\overline{u}})$   & $\mathcal C(w,r)$           & $\Re\left(w \, 2\overline{u}\right)=1$\\ \hline
5   & $\Delta(u)$       & $\mathcal C^*(v)$    & $\Im\left( \frac{v}{u}\right) =0$ & $\Delta(u)$                             & $\mathcal D(\frac{1}{2\overline{v}})$   & $\Im\left( \frac{u}{2v}\right)=0$ \\ \hline
6   & $\Delta(u)$       & $\mathcal C(v,\rho)$ & $\Im\left(\frac{v}{u}\right) =0$  & $\Delta(u)$                             & $\mathcal C(w,r)$                       & $\Im\left(\frac{w}{u}\right) =0$ \\ \hline
\end{tabular}
 \caption{Conservation de l'orthogonalité}
 \label{tab:consverorth}
\end{table}
\begin{displaymath}
\text{pour les cas 4 et 6: }\hspace{0.5cm}  w = \frac{v}{|v|^2 - \rho^2} \hspace{1cm} r = \frac{\rho}{|v|^2 - \rho^2}
\end{displaymath}
Les cas 1 et 3 résultent de la relation
\begin{displaymath}
  \frac{1}{u\overline{v}} = \frac{1}{|uv|^2}\overline{u}v = \frac{1}{|uv|^2}\overline{u\overline{v}}
\end{displaymath}
Le cas 2 vient de la conservation de la partie réelle par conjugaison.\newline
Dans le cas 4, transformons la condition non écrite
\begin{displaymath}
  |u - v|^2 = \rho^2 + |u|^2 \Leftrightarrow -2\Re\left(u\overline{v}\right) + |v|^2 = \rho^2 
  \Leftrightarrow 2\Re\left(u\frac{\overline{v}}{|v|^2 - \rho^2}\right) = 1
\end{displaymath}
qui est bien la condition voulue avec le $w$ donné.\newline
Le cas 5 se déduit de
\begin{displaymath}
  \Im\left(\frac{u}{2v}\right) = -\Im\left(\overline{\frac{u}{2v}}\right) = 
  -\frac{|u|^2}{2|v|^2}\Im\left(\frac{v}{u}\right) 
\end{displaymath}
Le cas 6 vient de ce que $\frac{w}{u}$ est égal à $\frac{v}{u}$ multiplié par un nombre réel.

\item Notons $\mathcal C$ le cercle passant par $O$ (donc de type $\mathcal C^*(u)$.\newline
Par hypothèse, il est orthogonal à $\mathcal C_1$ et $\mathcal C_2$ qui sont de types $\mathcal C(u,r)$. D'après la question précédentes, les images sont orthogonales. Les images de $\mathcal C_1$ et $\mathcal C_2$ sont des cercles qui ne passent pas par $O$, l'image de $\mathcal C$ est une droite $\mathcal D$ qui ne passe pas par $O$. Les orthogonalités des images se traduisent par le fait que les centres des cercles images de $\mathcal C_1$ et $\mathcal C_2$ sont sur $\mathcal D$.\newline
La droite des centres est du type $\Delta (u)$ et elle est orthogonale à $\mathcal C_1$ et $\mathcal C_2$. Les images sont encore orthogonales et la droite des centre est sa propre image. On en déduit que les centres des images de $\mathcal C_1$ et $\mathcal C_2$ sont sur la droite des centres.\newline
Les centres des cercles images sont donc confondus au point d'intersection de la droite des centres avec l'image de $\mathcal C$.
\end{enumerate}
