\begin{enumerate}
 \item \begin{enumerate}
 \item D'après les propriétés usuelles du produit scalaire :
\begin{displaymath}
  \mathcal{P}(\overrightarrow{a}^\bot) \cap \mathcal{P}(\overrightarrow{b}^\bot) = 
\mathcal D (\overrightarrow{a}^\bot \wedge \overrightarrow{b}^\bot)
\end{displaymath}
De plus ici $\overrightarrow c = -\overrightarrow a - \overrightarrow b$ donc tout vecteur orthogonal à $\overrightarrow a$ et $\overrightarrow b$ est aussi orthogonal à $\overrightarrow c$ ce qui se traduit par :
\begin{align*}
 \mathcal{P}(\overrightarrow{a}^\bot) \cap \mathcal{P}(\overrightarrow{b}^\bot) &\subset \mathcal{P}(\overrightarrow{c}^\bot) \\
 \mathcal{P}(\overrightarrow{a}^\bot) \cap \mathcal{P}(\overrightarrow{b}^\bot) \cap \mathcal{P}(\overrightarrow{c}^\bot) &=  \mathcal{P}(\overrightarrow{a}^\bot) \cap \mathcal{P}(\overrightarrow{b}^\bot) 
= \mathcal D (\overrightarrow{a}^\bot \wedge \overrightarrow{b}^\bot)
\end{align*}

\item \'Equation normale du plan $\mathcal P (\overrightarrow a , \overrightarrow b)$ :
\begin{displaymath}
 \overrightarrow u \in \mathcal P (\overrightarrow a , \overrightarrow b) \Leftrightarrow
(\overrightarrow u / \overrightarrow a \wedge \overrightarrow b )= 0 
\end{displaymath}
On en déduit d'après le cours :
\begin{displaymath}
 d(M,\mathcal P (\overrightarrow a , \overrightarrow b)) =
 \frac
   {\left \vert (\overrightarrow u / \overrightarrow a \wedge \overrightarrow b ) \right \vert}
   {\left\|\overrightarrow a \wedge \overrightarrow b \right\|}
\end{displaymath}
\end{enumerate}

\item Plans \og hauteurs\fg.
\begin{enumerate}
 \item Le plan hauteur issu de $\overrightarrow u$ est orthogonal à $\mathcal P (\overrightarrow v , \overrightarrow w)$ c'est à dire qu'il contient le vecteur $\overrightarrow v \wedge \overrightarrow w$. Il doit aussi contenir $\overrightarrow u$. Un vecteur orthogonal au plan hauteur issu de $\overrightarrow u$ est donc 
\begin{displaymath}
 (\overrightarrow v \wedge \overrightarrow w)\wedge \overrightarrow u
\end{displaymath}
 \item Preuve de l'identité de Jacobi.\newline
Les termes se simplifient deux à deux en sommant les doubles produits vectoriels :
\begin{align*}
 (\overrightarrow u \wedge \overrightarrow v)\wedge \overrightarrow w &=
\underset{\blacksquare}{(\overrightarrow u / \overrightarrow w)\overrightarrow v}
-\underset{\blacktriangle}{(\overrightarrow v / \overrightarrow w)\overrightarrow u} \\
 (\overrightarrow v \wedge \overrightarrow w)\wedge \overrightarrow u &=
\underset{\blacklozenge}{(\overrightarrow v / \overrightarrow u)\overrightarrow w}
-\underset{\blacksquare}{(\overrightarrow w / \overrightarrow u)\overrightarrow v} \\
 (\overrightarrow w \wedge \overrightarrow u)\wedge \overrightarrow v &=
\underset{\blacktriangle}{(\overrightarrow w / \overrightarrow v)\overrightarrow u}
-\underset{\blacklozenge}{(\overrightarrow u / \overrightarrow v)\overrightarrow w}
\end{align*}
Chacun des trois vecteurs de l'identité de Jacobi est orthogonal à un des plans hauteurs. La question 1. montre alors que l'intersection des trois plans est la droite $\mathcal D _h$ dirigée par le vecteur
\begin{displaymath}
 \left( (\overrightarrow u \wedge \overrightarrow v)\wedge \overrightarrow w\right) \wedge
\left( (\overrightarrow v \wedge \overrightarrow w)\wedge \overrightarrow u\right) 
\end{displaymath}
\end{enumerate}

\item Plans \og bissecteurs\fg.
\begin{enumerate}
 \item En utilisant les équations normale et le résultat de cours donnant la distance d'un point à un plan, on obtient que le point $M$ est équidistant de $\mathcal P (\overrightarrow u , \overrightarrow v)$ et $\mathcal P (\overrightarrow w , \overrightarrow u)$ si et seulement si :
\begin{equation*}
 \frac{\left\vert (\overrightarrow m / \overrightarrow u \wedge \overrightarrow v)\right\vert}
   {\left\Vert \overrightarrow u \wedge \overrightarrow v \right\Vert}
= \frac{\left\vert (\overrightarrow m / \overrightarrow w \wedge \overrightarrow u)\right\vert }
   {\left\Vert \overrightarrow w \wedge \overrightarrow u \right\Vert}
\end{equation*}
ou encore, pour $\varepsilon\in \{-1,+1\}$:
\begin{equation*}
 \frac{ (\overrightarrow m / \overrightarrow u \wedge \overrightarrow v)}
   {\left\Vert \overrightarrow u \wedge \overrightarrow v \right\Vert}
= \varepsilon\,
\frac{(\overrightarrow m / \overrightarrow w \wedge \overrightarrow u) }
   {\left\Vert \overrightarrow w \wedge \overrightarrow u \right\Vert}
\end{equation*}
Ce qui s'écrit
\begin{displaymath}
 (\overrightarrow m / \overrightarrow \alpha_\varepsilon ) = 0
\;\text{ avec }\;  \overrightarrow \alpha_\varepsilon =
\frac{1}{\left\Vert \overrightarrow u \wedge \overrightarrow v \right\Vert}\overrightarrow u \wedge \overrightarrow v 
+
\frac{\varepsilon}{\left\Vert \overrightarrow w \wedge \overrightarrow u \right\Vert}\overrightarrow w \wedge \overrightarrow u 
\end{displaymath}
On obtient donc deux plans bissecteurs respectivement orthogonaux à $\overrightarrow \alpha_1$ et $\overrightarrow \alpha_{-1}$

\item Calculons les produits scalaires :
\begin{align*}
 (\overrightarrow \alpha_\varepsilon / \overrightarrow v) 
&= \frac{\varepsilon}{\left\Vert \overrightarrow w \wedge \overrightarrow u \right\Vert}(\overrightarrow w \wedge \overrightarrow u / \overrightarrow v) 
= \varepsilon
    \frac{\det(\overrightarrow u , \overrightarrow v , \overrightarrow w)}
          {\left\Vert \overrightarrow w \wedge \overrightarrow u \right\Vert} \\
 (\overrightarrow \alpha_\varepsilon / \overrightarrow w) 
&= \frac{\varepsilon}{\left\Vert \overrightarrow u \wedge \overrightarrow v \right\Vert}(\overrightarrow u \wedge \overrightarrow v / \overrightarrow w) 
= \frac{\det(\overrightarrow u , \overrightarrow v , \overrightarrow w)}
          {\left\Vert \overrightarrow w \wedge \overrightarrow u \right\Vert} 
\end{align*}
Ces deux produits scalaires sont donc de signe opposés uniquement pour 
\begin{displaymath}
 \overrightarrow a = \overrightarrow \alpha_{-1}
= \frac{1}{\left\Vert \overrightarrow u \wedge \overrightarrow v \right\Vert}\overrightarrow u \wedge \overrightarrow v 
-\frac{1}{\left\Vert \overrightarrow w \wedge \overrightarrow u \right\Vert}\overrightarrow w \wedge \overrightarrow u
\end{displaymath}

\item On déduit les autres vecteurs orthogonaux aux plans bissecteurs en permutant les lettres. Ils se simplifient deux par deux dans la sommation :
\begin{align*}
 \overrightarrow a &=
\underset{\blacksquare}{\frac{1}{\left\Vert \overrightarrow u \wedge \overrightarrow v \right\Vert}\overrightarrow u \wedge \overrightarrow v }
-\underset{\blacktriangle}{\frac{1}{\left\Vert \overrightarrow w \wedge \overrightarrow u \right\Vert}\overrightarrow w \wedge \overrightarrow u} \\
 \overrightarrow b &=
\underset{\blacklozenge}{\frac{1}{\left\Vert \overrightarrow v \wedge \overrightarrow w \right\Vert}\overrightarrow v \wedge \overrightarrow w }
-\underset{\blacksquare}{\frac{1}{\left\Vert \overrightarrow u \wedge \overrightarrow v \right\Vert}\overrightarrow u \wedge \overrightarrow v} \\
 \overrightarrow c &=
\underset{\blacktriangle}{\frac{1}{\left\Vert \overrightarrow w \wedge \overrightarrow u \right\Vert}\overrightarrow w \wedge \overrightarrow u }
-\underset{\blacklozenge}{\frac{1}{\left\Vert \overrightarrow v \wedge \overrightarrow w \right\Vert}\overrightarrow v \wedge \overrightarrow w} \\
\end{align*}
La question 1. montre ici que l'intersection des trois plans bissecteurs (intérieurs) est une droite $\mathcal D_b$ dirigée par :
\begin{multline*}
 \left(
\frac{1}{\left\Vert \overrightarrow u \wedge \overrightarrow v \right\Vert}\overrightarrow u \wedge \overrightarrow v 
-
\frac{1}{\left\Vert \overrightarrow w \wedge \overrightarrow u \right\Vert}\overrightarrow w \wedge \overrightarrow u
\right)  \\
\wedge 
\left( 
\frac{1}{\left\Vert \overrightarrow v \wedge \overrightarrow w \right\Vert}\overrightarrow v \wedge \overrightarrow w
-
\frac{1}{\left\Vert \overrightarrow u \wedge \overrightarrow v \right\Vert}\overrightarrow u \wedge \overrightarrow v
\right) 
\end{multline*}
\end{enumerate}

\item Plans \og médiateurs\fg.
\begin{enumerate}
 \item En décomposant à l'aide du projeté orthogonal, on obtient que
\begin{displaymath}
 \Vert \overrightarrow m \Vert ^2
 = d(M,\mathcal D(\overrightarrow v))^2 +d(M,\mathcal P(\overrightarrow v ^\bot))^2 
 = d(M,\mathcal D(\overrightarrow w))^2 +d(M,\mathcal P(\overrightarrow w ^\bot))^2
\end{displaymath}
On en déduit qu'un point est à égale distance des droites si et seulement si il est à égale distance des plans.

\item \'Ecrivons qu'un point $M$ est à égale distance des droites en écrivant qu'il est à égale distance des plans (avec les équations normales) :
\begin{displaymath}
 \frac{\left| (\overrightarrow m / \overrightarrow v)\right|}{\Vert v \Vert}
=
 \frac{\left| (\overrightarrow m / \overrightarrow w)\right|}{\Vert w \Vert}
\end{displaymath}
ce qui s'écrit encore, avec $\varepsilon \in \{-1,+1\}$,
\begin{align*}
 (\overrightarrow m / \overrightarrow \alpha_\varepsilon) &= 0
  & \text{ avec } &  &
 \overrightarrow \alpha_\varepsilon &= 
 \frac{1}{\Vert \overrightarrow v \Vert}\overrightarrow v
+ \frac{\varepsilon}{\Vert \overrightarrow w \Vert}\overrightarrow w
\end{align*}
On obtient donc deux plans médiateurs associés aux deux vecteurs orthogonaux $\overrightarrow\alpha_{-1}$ et $\overrightarrow\alpha_{1}$.
\item Exprimons les produits scalaires avec des $\cos$ :
\begin{equation*}
 (\overrightarrow\alpha_\varepsilon / \overrightarrow v) = 
\Vert \overrightarrow v \Vert + \varepsilon \frac{(\overrightarrow w / \overrightarrow v)}{\Vert \overrightarrow w \Vert} =
\Vert \overrightarrow v \Vert(1+\varepsilon \cos \delta) = 
\Vert \overrightarrow v \Vert \varepsilon(\varepsilon + \cos \delta)
\end{equation*}
où $\delta$ est l'écart angulaire entre $\overrightarrow v$ et $\overrightarrow w$. De même
\begin{displaymath}
  (\overrightarrow\alpha_\varepsilon / \overrightarrow w) = 
\Vert \overrightarrow v \Vert (\varepsilon + \cos \delta)
\end{displaymath}
On en déduit que l'unique vecteur pour lequel les produits scalaires sont de signe opposés est
\begin{displaymath}
 \overrightarrow a = \overrightarrow \alpha_{-1} =
 \frac{1}{\Vert \overrightarrow v \Vert}\overrightarrow v
- \frac{1}{\Vert \overrightarrow w \Vert}\overrightarrow w
\end{displaymath}
\item Les vecteurs $\overrightarrow b$ et $\overrightarrow c$ s'obtiennent par permutation circulaire. Les termes se simplifient deux par deux lorsque l'on somme les trois. L'intersection des plans médiateurs est donc une droite $\mathcal D_m$ dirigée par
\begin{displaymath}
 \left(  \frac{1}{\Vert \overrightarrow v \Vert}\overrightarrow v
- \frac{1}{\Vert \overrightarrow w \Vert}\overrightarrow w
\right) 
\wedge
 \left(  \frac{1}{\Vert \overrightarrow w \Vert}\overrightarrow w
- \frac{1}{\Vert \overrightarrow u \Vert}\overrightarrow u
\right) 
\end{displaymath}
\end{enumerate}

\item Expression des vecteurs directeurs des droites.
\begin{itemize}
 \item Plans hauteurs. Avec des doubles produits vectoriels, et après avoir mis en facteur
\begin{displaymath}
 (\overrightarrow u / \overrightarrow w)(\overrightarrow v / \overrightarrow u)(\overrightarrow w / \overrightarrow v)
\end{displaymath}
on trouve :
\begin{displaymath}
 \frac{\overrightarrow{u}\wedge \overrightarrow{v}}{\overrightarrow{u}. \overrightarrow{v}} +
\frac{\overrightarrow{v}\wedge \overrightarrow{w}}{\overrightarrow{v}. \overrightarrow{w}} +
\frac{\overrightarrow{w}\wedge \overrightarrow{u}}{\overrightarrow{w}. \overrightarrow{u}}
\end{displaymath}
\item Plans bissecteurs. En utilisant la linéarité du produit vectoriel et après avoir multiplié par
\begin{displaymath}
 \Vert \overrightarrow{u}\wedge \overrightarrow{v} \Vert
\Vert \overrightarrow{v}\wedge \overrightarrow{w} \Vert
\Vert \overrightarrow{w}\wedge \overrightarrow{u} \Vert
\end{displaymath}
et mis en facteur 
\begin{displaymath}
 \det(\overrightarrow u,\overrightarrow v,\overrightarrow w)
\end{displaymath}
on trouve 
\begin{displaymath}
 \Vert \overrightarrow{v}\wedge \overrightarrow{w} \Vert \overrightarrow{u} +
\Vert \overrightarrow{w}\wedge \overrightarrow{u} \Vert \overrightarrow{v} +
\Vert \overrightarrow{u}\wedge \overrightarrow{v} \Vert \overrightarrow{w}
\end{displaymath}
\item Plans médiateurs. En utilisant la linéarité du produit vectoriel, on obtient directement :
\begin{displaymath}
 \frac{1}{\Vert \overrightarrow{u}\Vert \Vert \overrightarrow{v} \Vert}\overrightarrow{u}\wedge \overrightarrow{v} +
\frac{1}{\Vert \overrightarrow{v}\Vert \Vert \overrightarrow{w} \Vert}\overrightarrow{v}\wedge \overrightarrow{w} +
\frac{1}{\Vert \overrightarrow{w}\Vert \Vert \overrightarrow{u} \Vert}\overrightarrow{w}\wedge \overrightarrow{u}
\end{displaymath}


\end{itemize}



\end{enumerate}
