%<dscrpt>Théorème de Bang.</dscrpt>
L'objet de ce problème est de montrer que lorsque les faces d'un tétraèdre ont la même aire alors elles sont isométriques (théorème de Bang).\newline
On se place dans un espace euclidien orienté de dimension trois.
\begin{enumerate}
 \item \'Ecart angulaire.\newline
Soit $\overrightarrow{u}$ et $\overrightarrow{v}$ deux vecteurs non nuls. On appelle écart angulaire entre $\overrightarrow{u}$ et $\overrightarrow{v}$ le nombre
\begin{displaymath}
 \theta = \arccos \frac{(\overrightarrow{u}/\overrightarrow{v})}{\Vert\overrightarrow{u}\Vert\, \Vert\overrightarrow{v}\Vert}
\end{displaymath}
Montrer que 
\begin{displaymath}
 \Vert \overrightarrow{u}\wedge\overrightarrow{v}\Vert = \Vert\overrightarrow{u}\Vert\, \Vert\overrightarrow{v}\Vert\,\sin\theta
\end{displaymath}

 \item Intersection d'une sphère et d'un plan.\newline
Soit $O$ un point fixé et $\overrightarrow{\beta}$, $\overrightarrow{\gamma}$ deux vecteurs non colinéaires de norme $1$. On note $\mathcal{P}$ l'ensemble des points $M$ tels que 
\begin{displaymath}
 (\overrightarrow{\gamma}+\overrightarrow{\beta}/\overrightarrow{OM})+ (\overrightarrow{\gamma}/\overrightarrow{\beta})+1 = 0
\end{displaymath}
\begin{enumerate}
 \item Quelle est la nature de $\mathcal{P}$? Exprimer la distance de $O$ à $\mathcal{P}$ en fonction de $(\overrightarrow{\gamma}/\overrightarrow{\beta})$. Former à l'aide de $\overrightarrow \beta$ et $\overrightarrow \gamma$ des exemples de points dans $\mathcal{P}$ et sur la sphère centrée en $O$ et de rayon $1$.
 \item En déduire qu'il existe des vecteurs $\overrightarrow{\delta}$ de norme $1$ tels que
\begin{displaymath}
 (\overrightarrow{\gamma}/\overrightarrow{\beta}) +(\overrightarrow{\delta}/\overrightarrow{\beta}) +(\overrightarrow{\gamma}/\overrightarrow{\delta}) = -1
\end{displaymath}

\item Montrer qu'il existe des vecteurs $\overrightarrow{\alpha}$, $\overrightarrow{\delta}$ de norme $1$ et vérifiant
\begin{displaymath}
 \overrightarrow \alpha = \overrightarrow \beta + \overrightarrow \gamma + \overrightarrow \delta 
\text{ et }
\det(\overrightarrow{\beta},\overrightarrow{\gamma},\overrightarrow{\delta}) > 0
\end{displaymath}
\end{enumerate}

\item Un système d'équations vectorielles.\newline 
Dans cette question, on se donne trois vecteurs $\overrightarrow{\beta}$, $\overrightarrow{\gamma}$, $\overrightarrow{\delta}$ non coplanaires c'est à dire que $\det(\overrightarrow{\beta},\overrightarrow{\gamma},\overrightarrow{\delta})\neq0$. On considère un système $\mathcal{S}$ de trois équations aux inconnues vectorielles $(\overrightarrow{b}$, $\overrightarrow{c}$, $\overrightarrow{d})$ :
\begin{displaymath}
 (\mathcal{S})
\hspace{1cm}
\left\lbrace  
\begin{aligned}
 \overrightarrow{b}\wedge \overrightarrow{c} &= \overrightarrow{\delta} \\
 \overrightarrow{c}\wedge \overrightarrow{d} &= \overrightarrow{\beta} \\
 \overrightarrow{d}\wedge \overrightarrow{b} &= \overrightarrow{\gamma}
\end{aligned}
\right. 
\end{displaymath}

\begin{enumerate}
 \item 
Montrer que si $\mathcal{S}$ admet des solutions alors $\det(\overrightarrow{\beta},\overrightarrow{\gamma},\overrightarrow{\delta}) > 0$. On note $\Delta$ ce déterminant.

\item Montrer que si $\Delta$ est strictement positif alors le système admet deux triplets solutions
\begin{align*}
&(
 \frac{1}{\sqrt{\Delta}}\overrightarrow{\delta}\wedge \overrightarrow{\gamma},
 \frac{1}{\sqrt{\Delta}}\overrightarrow{\beta}\wedge \overrightarrow{\delta},
 \frac{1}{\sqrt{\Delta}}\overrightarrow{\gamma}\wedge \overrightarrow{\beta},
)\\
&(
 -\frac{1}{\sqrt{\Delta}}\overrightarrow{\delta}\wedge \overrightarrow{\gamma},
 -\frac{1}{\sqrt{\Delta}}\overrightarrow{\beta}\wedge \overrightarrow{\delta},
 -\frac{1}{\sqrt{\Delta}}\overrightarrow{\gamma}\wedge \overrightarrow{\beta},
)
\end{align*}
\end{enumerate}

\item
\begin{enumerate}
\item Soit $A$, $B$, $C$, $D$ quatre points non coplanaires. On pose 
\begin{align*}
 \overrightarrow{\delta}=\overrightarrow{AB}\wedge \overrightarrow{AC} ,& &
 \overrightarrow{\alpha}=\overrightarrow{BC}\wedge \overrightarrow{BD} ,& &
 \overrightarrow{\beta}=\overrightarrow{CD}\wedge \overrightarrow{CA} ,& &
 \overrightarrow{\gamma}= - \overrightarrow{DA}\wedge \overrightarrow{DB}
\end{align*}
Que peut-on dire de la direction et des longueurs de ces vecteurs? Montrer qu'ils vérifient une certaine relation à préciser. Cette relation ne contiendra ni $A$, ni $B$, ni $C$ ni $D$.

 \item  On se donne quatre vecteurs $\overrightarrow{\alpha}$, $\overrightarrow{\beta}$, $\overrightarrow{\gamma}$, $\overrightarrow{\delta}$ tels que
\begin{displaymath}
 \det(\overrightarrow{\beta},\overrightarrow{\gamma},\overrightarrow{\delta}) > 0,\hspace{1cm}
\overrightarrow{\alpha} - \overrightarrow{\beta} - \overrightarrow{\gamma} - \overrightarrow{\delta} = \overrightarrow{0}
\end{displaymath}
Montrer qu'il existe quatre points non coplanaires $A$, $B$, $C$, $D$ tels que 
\begin{align*}
 \overrightarrow{\delta}=\overrightarrow{AB}\wedge \overrightarrow{AC} ,& &
 \overrightarrow{\alpha}=\overrightarrow{BC}\wedge \overrightarrow{BD} ,& &
 \overrightarrow{\beta}=\overrightarrow{CD}\wedge \overrightarrow{CA} ,& &
 \overrightarrow{\gamma}= - \overrightarrow{DA}\wedge \overrightarrow{DB}
\end{align*}
Quelle condition supplémentaire doit-on imposer aux vecteurs donnés pour que les quatre faces du tétraèdre $(A,B,C,D)$ aient la même aire ?
\end{enumerate}

\item Soit $\overrightarrow \beta$ et $\overrightarrow \gamma$ deux vecteurs unitaires et non colinéaires.
\begin{enumerate}
\item Montrer qu'il existe au moins un tétraèdre $(A, B,C,D)$ tel que toutes les faces aient la même aire et que 
\begin{displaymath}
 \overrightarrow{\beta}=\overrightarrow{CD}\wedge \overrightarrow{CA} \hspace{0.5cm}
 \overrightarrow{\gamma}= - \overrightarrow{DA}\wedge \overrightarrow{DB}
\end{displaymath}
Dans la suite, $A$, $B$, $C$, $D$ est un tel tétraèdre et on adopte les notations de la question 4. 
\item Exprimer l'écart angulaire entre $\overrightarrow \alpha$ et $\overrightarrow \beta$ en fonction de l'écart angulaire entre $\overrightarrow \gamma$ et $\overrightarrow \delta$. En déduire que $AB=CD$.
\item Montrer que $AD=BC$ et en déduire que les faces $ABC$ et $ACD$ sont isométriques. 
\end{enumerate}
\end{enumerate}
