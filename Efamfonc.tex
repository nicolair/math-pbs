%<dscrpt>Famille de fonctions bijectives.</dscrpt>
Pour tout $\alpha > 0$, on définit une fonction $\varphi_\alpha$ dans $[-1,1]$ par:
\begin{displaymath}
  \varphi_\alpha (x) = -1 + 2 \left( \frac{1 + x}{2}\right)^\alpha 
\end{displaymath}
\begin{enumerate}
  \item Montrer que $\varphi_\alpha$ définit une bijection de $[-1,1]$ dans $[-1,1]$.
  \item Tracer sur la même figure le graphe d'un $\varphi_\alpha$ avec $\alpha <1$, le graphe de $\varphi_1$ et celui d'un $\varphi_\alpha$ avec $\alpha >1$.
  \item Discuter, selon les $x\in [-1,+1]$ des limites en $0$ et $+\infty$ de la fonction
\begin{displaymath}
  \alpha \mapsto -1 + 2 \left( \frac{1 + x}{2}\right)^\alpha
\end{displaymath}
  \item Montrer que, pour tout $u\in ]-1, 1[$, il existe un unique $\alpha > 0$ tel que $\varphi_\alpha(0) = u$. On désigne par $\gamma(u)$ cet unique $\alpha$. Donner une expression de $\gamma(u)$ avec des fonctions usuelles et déterminer les limites de $\gamma$ en $-1$ et $1$.
\end{enumerate}
