%<dscrpt>Borne inférieure et addition parallèle. </dscrpt>
On définit la \emph{somme parallèle}\footnote{d'après X 99 PC 1}
de deux réels strictement positifs par:
\[
 \forall (a,b) \in \left] 0,+\infty\right[ ^{2} ,\; a//b=\frac{ab}{a+b}.
\]

\begin{enumerate}
\item Cette opération est-elle commutative, associative, admet-elle un élément neutre?

\item Soit $x$ un réel quelconque.
Montrer que
\[(a//b)x^2=\inf \{ay^{2}+bz^{2},(y,z)\in \R^{2}\,\text{ tq }\,y+z=x\}\]
Cette borne inférieure est-elle un plus petit élément?\newline
Si oui, pour quels couples $(y_0,z_0)$ la relation $(a//b)x^{2}=ay_0^{2}+bz_0^{2}$ est-elle satisfaite ?

\item Interpréter physiquement les résultats de la question précédente en prenant pour $y$ et $z$ les intensités des courants électriques qui traversent des résistances $a$ et $b$ montées en parall{\`e}le.

\item Soit $a$, $b$, $c$, $d$ des réels strictement positifs et $x$ un réel quelconque. Montrer que
\[(a//c)x^{2}+(b//d)x^{2}\leq ((a+b)//(c+d))x^{2}.\]
Interpréter physiquement cette inégalité.

\item Soient $\alpha_1,\alpha_2,\ldots,\alpha_k$ et $\beta_1,\beta_2,\ldots,\beta_k$ des réels strictement positifs. Montrer que
\[\sum_{i=1}^{k}(\alpha_i//\beta_i)\leq \left (\sum_{i=1}^{k}\alpha_i\right)// \left (\sum_{i=1}^{k}\beta_i\right).\]
\end{enumerate}
