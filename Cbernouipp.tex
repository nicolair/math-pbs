
\begin{enumerate}
 \item \begin{enumerate}
            \item \begin{itemize}
                   \item[\textbullet] Unicité: Soient $F_{1}, F_{2}$ deux telles fonctions. Comme $(F_{1}-F_{2})' = 0$, il existe $\lambda \in \R$ tel que $F_{1}-F_{2} = \lambda$. Alors:
                   \[ \pi \lambda = \int_{0}^{\pi}\lambda \ dx = \int_{0}^{\pi}(F_{1}(x) - F_{2}(x))\ dx = \int_{0}^{\pi}F_{1}(x)\ dx - \int_{0}^{\pi}F_{2}(x)\ dx = 0.\]
                   Donc $F_{1} = F_{2}$.
                   \item[\textbullet] Existence: une fonction continue sur un segment possède toujours une primitive. Soit $G$ une primitive de $f$. Posons $C = \displaystyle{\frac{1}{\pi}\int_{0}^{\pi}G(x)\ dx}$ et $F = G-C$. Alors
            $\displaystyle{\int_{0}^{\pi}F(x)\ dx = \int_{0}^{\pi}G(x)\ dx - \pi C = 0}$.
                  \end{itemize}
          
            \item Construisons la suite $(B_{n})$ par récurrence. Soit $n\in \N$, supposons qu'il existe une unique famille $(B_{0}, ..., B_{n})$ de fonctions vérifiant $(i)$, $(ii)$ et $(iii)$. 
            D'après la question précédente, il existe une unique fonction $B_{n+1}$ telle que 
            
            $B_{n+1}' = B_{n}$ et $\displaystyle{\int_{0}^{\pi}B_{n+1}(x)\ dx = 0}$. 
            
            \item Soit $n\geq 2$. Alors:
            \[ B_{n}(\pi) - B_{n}(0) = \int_{0}^{\pi}B_{n}'(x)\ dx = \int_{0}^{\pi}B_{n-1}(x)\ dx = 0
            \Rightarrow B_{n}(\pi) = B_{n}(0).
            \]
            
            \item Comme $B_{0} = 1$, il existe $\lambda$ tel que pour tout $x\in \R$, $B_{1}(x) = x + \lambda$. Comme $\displaystyle{\int_{0}^{\pi}x + \lambda \ dx = \frac{\pi^{2}}{2} + \lambda \pi = 0}$
            alors $\lambda = \displaystyle{-\frac{\pi}{2}}$. Donc $B_{1}(x) = \displaystyle{x - \frac{\pi}{2}}$. 
            
            Il existe donc $\mu \in \R$ tel que pour tout $x\in \R$, $B_{2}(x) = \displaystyle{\frac{x^{2}}{2}-\frac{\pi x}{2} + \mu}$. Intégrons $B_{2}$ entre $0$ et $\pi$:
            \[ 0 = \int_{0}^{\pi}B_{2}(x)\ dx = \frac{\pi^{3}}{6} - \frac{\pi^{3}}{4} + \mu \pi  = -\frac{\pi^{3}}{12} + \mu \pi\]
            donc $\mu = \displaystyle{\frac{\pi^{2}}{12}}$. Ainsi, $\displaystyle{B_{2}(x) = \frac{x^{2}}{2} - \frac{\pi x}{2} + \frac{\pi^{2}}{12}}$.
           \end{enumerate}
           

\item \begin{enumerate}
           \item Effectuons une première intégration par parties en posant:
           \[ \left \{ \begin{array}{ll}
                        u(x) = B_{2}(x)\\
                        v'(x) = \cos(2nx)
                       \end{array}
               \right. \qquad \left \{ \begin{array}{ll}
                                        u'(x) = B_{2}'(x) = B_{1}(x)\\
                                        v(x) = \frac{1}{2n}\sin(2nx)
                                       \end{array}
                                \right.\]
    \begin{multline*}
      I_{1,n}  = \left [ u(x)v(x)\right ]_{0}^{\pi}-\int_{0}^{\pi}u'(x)v(x) \ dx \notag 
       = \left [ \frac{B_{2}(x)\cos(2nx)}{2n}\right ]_{0}^{\pi} \\ 
       - \frac{1}{2n}\int_{0}^{\pi}B_{1}(x)\sin (2nx)\ dx\notag 
       = -\frac{1}{2n}\int_{0}^{\pi}B_{1}(x)\sin (2nx)\ dx. \notag
    \end{multline*}
    Effectuons une seconde intégration par parties:
    \[ \left \{ \begin{array}{ll}
                 u(x) = B_{1}(x)\\
                 v'(x) = \sin (2nx)
                \end{array}
          \right. \qquad \left \{ \begin{array}{ll}
                                   u'(x) = B_{0}(x)\\
                                   v(x) = -\frac{1}{2n}\cos(2nx)
                                  \end{array}
                          \right.\]
     Ainsi:
     \begin{multline*}
       I_{1,n}   = -\frac{1}{2n}\left [ u(x)v(x)\right ]_{0}^{\pi} + \frac{1}{2n}\int_{0}^{\pi}u'(x)v(x)\ dx\notag\\
       = \frac{1}{4n^{2}}(B_{1}(\pi) - B_{1}(0)) - \frac{1}{4n^{2}}\int_{0}^{\pi}\sin (2nx)\ dx \notag 
        = \frac{B_{1}(\pi) - B_{1}(0)}{4n^{2}}\notag 
        = \frac{\pi}{4n^{2}}\notag
     \end{multline*}
     
     \item Calculons $\displaystyle{I_{p+1, n} = \int_{0}^{\pi}B_{2p+2}(x)\cos(2nx)\ dx}$ par parties:
     \[ \left \{ \begin{array}{ll}
                  u(x) = B_{2p+2}(x)\\
                  v'(x) = \cos(2nx)
                 \end{array}
         \right. \qquad \left \{ \begin{array}{ll}
                                  u'(x) = B_{2p+1}(x)\\
                                  v(x) = \frac{1}{2n}\sin(2nx)
                                 \end{array}
                        \right.\]

  \begin{align}
    I_{p+1,n} &  = \frac{B_{2p+2}(\pi) - B_{2p+2}(0)}{2n} - \frac{1}{2n}\int_{0}^{\pi}B_{2p+1}\sin(2nx)\ dx \notag \\
    & = -\frac{1}{2n}\int_{0}^{\pi}B_{2p+1}\sin (2nx)\ dx\notag
  \end{align}
  Effectuons encore une intégration par parties:
  \[ \left \{ \begin{array}{ll}
               u(x) = B_{2p+1}(x)\\
               v'(x) = \sin (2nx)
              \end{array}
      \right. \qquad \left \{ \begin{array}{ll}
                               u'(x) = B_{2p}(x)\\
                               v(x) = -\frac{1}{2n}\cos(2nx)
                              \end{array}
                      \right. \]
 On obtient:
\[
  I_{p+1,n}  = \frac{B_{2p+1}(\pi) - B_{2p+1}(0)}{2n} - \frac{1}{4n^{2}} \int_{0}^{\pi}B_{2p}(x)\cos(2nx)\ dx 
   = -\frac{1}{4n^{2}}I_{p,n}
\]
 
 
\item Un récurrence immédiate donne le résultat.
\end{enumerate}


\item Soit $t\in ]0, \pi[$. Alors:
\begin{multline*}
 \sum_{k=1}^{n}\cos(2kt) = \operatorname{Re}\left ( \sum_{k=1}^{n}e^{2ikt}\right )\notag 
  = \operatorname{Re}\left ( e^{2it}\frac{1-e^{2int}}{1-e^{2it}}\right )\notag\\
  = \operatorname{Re}\left (\frac{ e^{i(n+2)t}}{e^{it}}\frac{e^{-int}-e^{int}}{e^{-it}-e^{it}}\right )\notag
  = \operatorname{Re}\left ( e^{i(n+1)t}\frac{\sin (nt)}{\sin (t)}\right ) \notag
  = \cos((n+1)t)\frac{\sin (nt)}{\sin (t)}\notag
\end{multline*}
Or, 
\begin{multline*}
\sin((2n+1)t) - \sin (t) = \sin ((n+1)t + nt) - \sin((n+1)t - nt) \\ = 
2\cos((n+1)t)\sin (nt) \text{ car }\sin (a+b) - \sin (a-b) = 2\cos(a)\sin (b)) 
\end{multline*}
 donc:
\[ 
\sum_{k=1}^{n}\cos(2kt) = \frac{\sin((2n+1)t) -\sin (t)}{2\sin (t)} = \frac{\sin ((2n+1)t)}{2\sin (t)} - \frac{1}{2}.
\]


\item Soit $p\in \N^{*}$, soit $x\in ]0, \pi[$. Alors:
\[ \frac{\sin (2n+1)x}{\sin (x)} = 2\sum_{k=1}^{n}\cos(2kx) + 1\]
donc:
\begin{multline*}
  f_{p}(x)\sin ((2n+1)x)  = (B_{2p}(x)-B_{2p}(0))\frac{\sin ((2n+1)x)}{\sin (x)}\notag \\
   = (B_{2p}(x)-B_{2p}(0))+ 2\sum_{k=1}^{n}(B_{2p}(x)-B_{2p}(0))\cos(2kx) \notag\\
   = (B_{2p}(x)-B_{2p}(0))+ 2\sum_{k=1}^{n}B_{2p}(x)\cos(2kx) - 2\sum_{k=1}^{n}B_{2p}(0)\cos(2kx)\notag
\end{multline*}
Cette égalité reste vraie pour $x=0$ et $x=\pi$. Ainsi:
\begin{multline*}
 \int_{0}^{\pi}f_{p}(x)\sin((2n+1)x)\ dx  = \int_{0}^{\pi}(B_{2p}(x)-B_{2p}(0))\ dx + 2\sum_{k=1}^{n}I_{p,k} \\
 - 2\sum_{k=1}^{n}B_{2p}(0)\underbrace{\int_{0}^{\pi}\cos(2kx)\ dx}_{=0} \notag\\
  = -\pi B_{2p}(0)+ \sum_{k=1}^{n}\frac{(-1)^{k-1}\pi}{2^{2p-1}k^{2p}} + 0\notag
\end{multline*}
donc:
\[ \int_{0}^{\pi}f_{p}(x)\ \sin ((2n+1)x)\ dx = -\pi B_{2p}(0) + \frac{\pi (-1)^{p-1}}{2^{2p-1}}\sum_{k=1}^{n}\frac{1}{k^{2p}}.\]


\item Effectuons un intégration par parties:
\begin{multline*}
 \int_{0}^{\pi}f(x)\sin ((2n+1)x)\ dx = \left [ -\frac{f(x)\cos((2n+1)x)}{2n+1}\right ]_{0}^{\pi}\\
 + \frac{1}{2n+1}\int_{0}^{\pi}f'(x)\cos((2n+1)x)\ dx.
\end{multline*}
     
D'une part:
\[ \left [ -\frac{f(x)\cos((2n+1)x)}{2n+1}\right ]_{0}^{\pi} = \frac{f(0)}{2n+1} + \frac{f(\pi)}{2n+1} \xrightarrow[n\to + \infty]{}0.\]
D'autre part, comme $f'$ est bornée, il existe $M\in \R_{+}$ tel que pour tout $x\in [0, \pi]$, $-M\leq f'(x) \leq M$ donc: $-M \leq f'(x)\cos((2n+1)x) \leq M$ donc:
\[ -\frac{M\pi}{2n+1} \leq \frac{1}{2n+1}\int_{0}^{\pi}f'(x)\cos((2n+1)x)\ dx \leq \frac{M\pi }{2n+1}\]
donc d'après le théorème des gendarmes: 
\[ \frac{1}{2n+1}\int_{0}^{\pi}f'(x)\cos((2n+1)x)\ dx \xrightarrow[n\to + \infty]{}0.\]

D'où le résultat.

\item \begin{enumerate}
           \item D'après la question 4:
\[ \sum_{k=1}^{n}\frac{1}{k^{2p}} = \frac{2^{2p-1}(-1)^{k-1}}{\pi}\int_{0}^{\pi}f_{p}(x)\sin ((2n+1)x)\ dx + 2^{2p-1}(-1)^{p-1} B_{2p}(0). \]
D'après la question précédente:
\[ \int_{0}^{\pi}f_{p}(x)\sin((2n+1)x)\ dx \xrightarrow[n\to + \infty]{}0\]
donc:
\[ \sum_{k=1}^{n}\frac{1}{k^{2p}}\xrightarrow[n\to + \infty]{}(-1)^{p-1}2^{2p-1}B_{2p}(0).\]
         
           \item D'après la question 1.d, $B_{2}(0) = \displaystyle{\frac{\pi^{2}}{12}}$ donc:
           \[ \sum_{k=1}^{n}\frac{1}{k^{2}} \xrightarrow[n\to + \infty]{}(-1)^{1-1}2^{2-1}\frac{\pi^{2}}{12} = \frac{\pi^{2}}{6}.\]
          \end{enumerate}


\end{enumerate}

