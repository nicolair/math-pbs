\begin{enumerate}
  \item Rappel de cours.
\begin{displaymath}
\forall b \in E, \forall X \subset F,\hspace{.5cm}
\left\lbrace 
\begin{aligned}
  b\in g(X) &\Leftrightarrow \exists x \in X \text{ tel que } b=g(x)\\
  b\in f^{-1}(X) &\Leftrightarrow f(b) \in X
\end{aligned}
\right. 
\end{displaymath}
\begin{displaymath}
\forall y \in X, \forall A \subset E,\hspace{.5cm}
\left\lbrace 
\begin{aligned}
  y\in f(A) &\Leftrightarrow \exists a \in A \text{ tel que } y=f(a)\\
  y\in g^{-1}(A) &\Leftrightarrow g(y) \in A
\end{aligned}
\right. 
\end{displaymath}

  \item Dans cette question, on suppose qu'il existe $B\subset E$ telle que $E\setminus B = g(F\setminus(f(B))$. La partie $B$ appartient donc à $\mathcal{H}$ en vérifiant l'égalité au lieu d'une seule inclusion.
\begin{enumerate}
  \item La fonction $f_1$ est obtenue à partir de $f$ en restreignant l'espace de départ à la partie $B$. Elle est donc injective comme $f$. Comme l'espace d'arrivée est limité à $f(B)$, la fonction $f_1$ est automatiquement surjective. La démonstration est la même pour $g_1$ avec la remarque que $g(F\setminus f(B)) = E\setminus B$  est la propriété fondamentale de $B$ qui est admise. 
  \item Considérons $\psi \circ \varphi$.
\begin{multline*}
\forall a\in E,
\left\lbrace 
\begin{aligned}
a\in B \Rightarrow \varphi(a) = f_1(a)\in f(B) \Rightarrow \psi(\varphi(a))=f_1^{-1}(f_1(a)) = a \\
a\notin B \Rightarrow \varphi(a) = g_1^{-1}(a)\notin f(B) \Rightarrow \psi(\varphi(a))=g_1(g_1^{-1}(a)) = a 
\end{aligned}
\right. \\
\Rightarrow \psi \circ \varphi = \Id_{|E}.
\end{multline*}
On montre de la même manière que $\varphi \circ \psi = \Id_{|F}$. On en déduit que les deux applications sont bijectives et bijections réciproques l'une de l'autre.
\end{enumerate}

  \item 
\begin{enumerate}
  \item On se propose de démontrer
\begin{displaymath}
A \in \mathcal{H} \Leftrightarrow \left( \forall x \in F, x \notin f(A) \Rightarrow g(x) \notin A \right).
\end{displaymath}
Supposons $A \in \mathcal{H}$. Pour tout $x\in F$, si $x\notin f(A)$ alors $x\in F\setminus f(A)$ donc 
\begin{displaymath}
g(x)\in g(F\setminus f(A)) \subset E \setminus A \text{ (car $A \in \mathcal{H}$)} \Rightarrow g(x) \notin A.
\end{displaymath}
Réciproquement, supposons
\begin{displaymath}
\left( \forall x \in F, x \notin f(A) \Rightarrow g(x) \notin A \right) .
\end{displaymath}
On doit montrer que $A \in \mathcal{H}$, c'est à dire que $g(F\setminus f(A))\subset E\setminus A$.
\begin{displaymath}
\forall b \in g(F\setminus f(A)), \,\exists x\in F\setminus f(A) \text{ tq } b = g(x) .  
\end{displaymath}
Or, par hypothèse
\begin{displaymath}
  x\notin f(A) \Rightarrow b = g(x) \notin A .
\end{displaymath}
On a donc bien prouvé que $g(F\setminus f(A))\subset E\setminus A$.
  \item L'implication de la question précédente est équivalente à sa contraposée 
\begin{displaymath}
  \left( x \notin f(A) \Rightarrow g(x) \notin A \right)
  \Leftrightarrow
  \left( g(x) \in A \Rightarrow x \in f(A) \right).
\end{displaymath}
Or $g(x)\in A \Leftrightarrow x \in g^{-1}(A)$. L'implication à droite est donc équivalente à la simple inclusion $g^{-1}(A)\subset f(A)$. D'où la nouvelle caractérisation:
\begin{displaymath}
A \in \mathcal{H} \Leftrightarrow g^{-1}(A)\subset f(A) .
\end{displaymath}
\end{enumerate}

  \item
\begin{enumerate}
  \item Utilisons la caractérisation de la question précédente.
\begin{displaymath}
  g^{-1}(\emptyset) = \emptyset \subset \emptyset = f(\emptyset) \Rightarrow \emptyset \in \mathcal{H}.
\end{displaymath}
Comme $E\setminus g(F)$ est formé des éléments \emph{qui n'ont pas d'antécédents} par $g$, l'image réciproque $g^{-1}(E\setminus g(F))$ est vide. Elle est donc incluse dans n'importe quoi, en particulier dans $f(E\setminus g(F))$ donc $E\setminus g(F)\in \mathcal{H}$.
  \item La partie $B$ est définie comme l'union de toutes les parties de $E$ qui sont des éléments $\mathcal{H}$. Cela se traduit par:
\begin{displaymath}
\forall c\in E, \; c\in B \Leftrightarrow \exists A\in \mathcal{H} \text{ tq } c\in A .
\end{displaymath}
On en tire, pour tout $x\in F$,
\begin{multline*}
x\in g^{-1}(B) \Leftrightarrow g(x)\in B 
\Leftrightarrow \exists A\in \mathcal{H} \text{ tq } g(x) \in A \\
\Leftrightarrow \exists A\in \mathcal{H} \text{ tq } x \in g^{-1}(A)
\Leftrightarrow x \in \bigcup_{A\in \mathcal{H}}g^{-1}(A) .
\end{multline*}
Donc $g^{-1}(B) = \bigcup_{A\in \mathcal{H}}g^{-1}(A)$.\newline
On prouve séparément les deux inclusions pour les images directes.
\begin{displaymath}
\left( \forall A\in \mathcal{H}, A \subset B\right) 
\Rightarrow \left( \forall A\in \mathcal{H}, f(A) \subset f(B)\right)
\Rightarrow \bigcup_{A \in \mathcal{H}}f(A) \subset f(B) .
\end{displaymath}
Pour l'autre inclusion, considérons un élément $x$ quelconque de $f(B)$. Il existe $b\in B$ tel que $x=f(b)$. Par définition de $B$, il existe $A_0\in \mathcal{H}$ tel que $b\in A_0$ donc $x=f(b)\in f(A_0)\subset \bigcup_{A\in \mathcal{H}}f(A)$.\newline
Pour montrer que $B$ appartient elle même à $\mathcal{H}$, on utilise la question 3.b.:
\begin{displaymath}
  g^{-1}(B) = \bigcup_{A \in \mathcal{H}}g^{-1}(A) \subset \bigcup_{A \in \mathcal{H}}f(A) = f(B) .
\end{displaymath}
Cette inclusion résulte de la caractérisation de 3.b. appliquée aux $A$ dans $\mathcal{H}$ dans un sens. En utilisant pour $B$ la caractérisation dans l'autre sens, on obtient
\begin{displaymath}
 g^{-1}(B) \subset f(B) \Rightarrow B \in \mathcal{H}.
\end{displaymath}
\end{enumerate}

  \item 
\begin{enumerate}
  \item On va démontrer la contraposée c'est à dire $B \cup \left\lbrace a\right\rbrace \in \mathcal{H} \Rightarrow a \in B$.\newline
En effet $B \cup \left\lbrace a\right\rbrace \in \mathcal{H}$ entraine $B \cup \left\lbrace a\right\rbrace \subset B$ par définition de $B$ comme union des éléments de $\mathcal{H}$. Mais comme $a\in B\cup \left\lbrace a\right\rbrace \subset B$, on a bien $a\in B$.
  \item Comme $B\in \mathcal{H}$, on connait déjà une inclusion: $g(F\setminus f(B)) \subset E \setminus B$.\newline
Pour démontrer l'autre inclusion, considérons dans $E$ un $a\notin B$.\newline
D'après la question précédente, $B \cup \left\lbrace a\right\rbrace \notin \mathcal{H}$, donc
\begin{displaymath}
  g^{-1}(B)\cup g^{-1}(\{a\}) \not\subset f(B)\cup \{f(a)\}
\Rightarrow g^{-1}(\{a\}) \not\subset f(B)\cup \{f(a)\}.
\end{displaymath}
car $g^{-1}(B)\subset f(B)$ (à cause de $B\in \mathcal{H}$).
\end{enumerate}
Le fait que $g^{-1}(\left\lbrace a\right\rbrace$ \emph{ne soit pas} incluse dans une certaine partie de $F$ montre qu'elle n'est pas vide. Il existe donc un antécédent $x\in F$ tel que $a=g(x)$. Alors:
\begin{displaymath}
  \{x\} = g^{-1}(\{a\}) \not\subset f(B)\cup \{f(a)\} \Rightarrow x \notin f(B) \Rightarrow a=g(x)\in g(F\setminus f(B)).
\end{displaymath}
On a bien montré que $a\notin B \Rightarrow a\in g(F\setminus f(B))$ ce qui prouve la deuxième inclusion.

\item Sous les hypothèses du théorème, les questions 4. et 5. montrent l'existence d'une partie $B$ vérifiant la propriété admise en question 2. et qui permet de fabriquer les deux bijections réciproques l'une de l'autre.
\end{enumerate}
