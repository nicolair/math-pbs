\begin{enumerate}
 \item
\begin{enumerate}
 \item Par définition, $\omega_4=i$ et, si $a=(a_0,a_1,a2,a_3)$,
\begin{displaymath}
 F(a)= (a_0+a_1+a_2+a_3, a_0+ia_1-a_2-ia_3,a_0-a_1+a_2-a_3,a_0-ia_1-a_2+ia_3)
\end{displaymath}

 \item D'après la question précédente,
\begin{displaymath}
 M_4=
\begin{pmatrix}
 1 & 1  & 1  & 1  \\ 
 1 & i  & -1 & -i \\
 1 & -1 & 1  & -1 \\
 1 & -i & -1 & i
\end{pmatrix}
\end{displaymath}

 \item Après calculs, on trouve $M_4 \overline{M_a}=4I_4$. On en déduit que $M_4$ est inversible d'inverse $\frac{1}{4}\overline{M_4}$ puis que $F_4$ est bijective. La bijection réciproque $F_4^{-1}$ est l'endomorphisme de $\C^n$ dont la matrice dans la base canonique est $\frac{1}{4}\overline{M_4}$.
 \item Après calculs, on trouve
\begin{displaymath}
 M_4^2= 4
\begin{pmatrix}
 1 & 0 & 0 & 0  \\ 
 0 & 0 & 0 & 1 \\
 0 & 0 & 1  & 0 \\
 0 & 1 & 0 & 0
\end{pmatrix},
\hspace{0.5cm}
M_4^4 = 16 I_4
\end{displaymath}
\end{enumerate}
 
 \item Dans toute cette question, on note $\omega$ au lieu de $\omega_n$.
\begin{enumerate}
 \item Comme $F_n$ est un endomorphisme de $\C^n$, pour prouver que $F_n$ est bijective, il suffit de montrer que $F_n$ est injective.\newline
Si $F_n(a)=0_{\C^n}$, le polynôme $A$ associé à ce $n$-uplet $a$ est nul en $1,\omega_n,\cdots,\omega_n^{n-1}$. Il admet $n$ racines distinctes en étant de degré au plus $n-1$, il est donc nul.
 \item D'après la définition de $F_n$,
\begin{displaymath}
 M_n=
\begin{pmatrix}
 1      & 1              & 1                & \cdots & 1                     \\
 1      & \omega         & \omega^2         & \cdots & \omega^{n-1}          \\
 1      & (\omega)^2     & (\omega^2)^2     & \cdots & (\omega^{n-1})^2      \\
 \vdots & \vdots         &  \vdots          & \vdots &  \vdots               \\
 1      & (\omega)^{n-1} & (\omega^2)^{n-1} & \cdots & (\omega^{(n-1)})^{n-1}
\end{pmatrix}
\end{displaymath}
On en déduit que le terme d'indice $(i,j)$ de cette matrice est
\begin{displaymath}
 (\omega^{j-1})^{i-1}=\omega^{(i-1)(j-1)}
\end{displaymath}

 \item Soit $i$ et $j$ deux entiers entre $-n$ et $n$. La somme proposée est formée de termes en progression géométrique de raison $\omega^{j-i}$. Par définition de $\omega$,
\begin{multline*}
 \omega^{j-i}
\left\lbrace 
\begin{aligned}
 &=1& &\text{ si } i\equiv j \mod n\\
 &\in \U_n\setminus\{1\}& &\text{ si } i-j \not \equiv j \mod n
\end{aligned}
\right. \\
\Rightarrow 
\sum_{k=0}^{n-1}\omega^{(i-j)k}=
\left\lbrace
\begin{aligned}
 &n& &\text{ si } i\equiv j \mod n\\
 &0& &\text{ si }  i-j \not \equiv j \mod n
\end{aligned}
\right. \\
\text{car, pour } \omega^{j-i}\neq 1,\;\sum_{k=0}^{n-1}\omega^{(i-j)k}=\frac{1-(\omega^{i-j})^n}{1-\omega^{i-j}}=0.
\end{multline*}

 \item Par définition du produit matriciel, pour $i$ et $j$ entre $1$ et $n$, $i-j$ est entre $-n+1$ et $n-1$, le terme $(i,j)$ de $M_n\overline{M_n}$ est
\begin{multline*}
 \sum_{k=1}^{n-1}\omega^{(i-1)(k-1)}\bar\omega^{(k-1)(j-1)}
= \sum_{k=1}^{n-1}\omega^{(i-1-j+1)(k-1)}\text{ car } \bar\omega = \omega^{-1}\\
= \sum_{k=1}^{n-1}\omega^{(i-j)(k-1)}
= \left\lbrace
\begin{aligned}
 &n &\text{ si } i=j\\
 &0 &\text{ si } i\neq j
\end{aligned}
\right. 
\end{multline*}
On en déduit que $M_n \overline{M_n}=nI_n$ puis que $F_n^{-1}$ est l'endomorphisme de $\C^n$ dont la matrice dans la base canonique est  $\frac{1}{n}\overline{M_n}$.
 \item Le terme $(i,j)$ de $M_n^2$ est
\begin{multline*}
 \sum_{k=1}^{n-1}\omega^{(i-1)(k-1)}\omega^{(k-1)(j-1)}
= \sum_{k=1}^{n-1}\omega^{(i-j+2)(k-1)}\\
= \left\lbrace
\begin{aligned}
 &n & &\text{ si } i+j-2 \equiv 0 \mod n\\
 &0 & &\text{ sinon }
\end{aligned}
\right. 
\end{multline*}
Pour $i$ et $j$ entre $1$ est $n$, on peut exprimer $i$ en fonction de $j$ pour que la congruence soit vérifiée. On doit avoir $i=1$ si $j=1$ et $i=n-j+2$ pour $j$ entre $2$ et $n$. Un seul terme est non nul par colonne et il est égal à $n$:
\begin{displaymath}
 M_n^2= n
\begin{pmatrix}
 1      & 0      & 0      & \cdots  & 0      \\
 0      & 0      & 0      & \cdots  & 1      \\
 \vdots & \vdots & \vdots & \diagup & \vdots \\
 0      & 0      & 1      &         & 0      \\
 0      & 1      & 0      & \cdots  & 0
\end{pmatrix}
\end{displaymath}
Cette matrice permet de préciser les images des vecteurs de la base canonique (attention au décalage d'indice)
\begin{displaymath}
 F_n^2(e_0)=n e_0,\hspace{1cm}\forall j\in\{1,\cdots,n-1\},\; F_n^2(e_j)=ne_{n-j}
\end{displaymath}
On en déduit que $F_n^4 = n^2 \Id_{\C^n}$.
\end{enumerate}

 \item
\begin{enumerate}
 \item Les $n$-uplets $F_n(e_1+e_{n-1})$ et $F_n(e_1-e_{n-1})$ correspondent à la somme et la différence des colonnes $2$ et $n$ de la matrice $A_n$. Comme $\omega^{n-1}=\omega^{-1}=\overline{\omega}$, les parties réelles et imaginaires apparaissent conduisant à
\begin{displaymath}
 F_n(e_1+e_{n-1})=2u,\hspace{1cm}F_n(e_1-e_{n-1})=2iv
\end{displaymath}
En composant par $F_n$ (qui est $\C$-linéaire), il vient
\begin{multline*}
\left. 
\begin{aligned}
 2F_n(u) &= n\left(F_n^2(e_1)+F_n^2(e_{n-1}) \right)= n(e_{n-1}+e_1)\\
 2iF_n(v)&= n\left(F_n^2(e_1)-F_n^2(e_{n-1}) \right)= n(e_{n-1}-e_1)
\end{aligned}
 \right\rbrace \\ \Rightarrow
\left\lbrace 
\begin{aligned}
 F_n(u) &= \frac{n}{2}(e_1 + e_{n-1}) \\ F_n(v) &= \frac{in}{2}(e_1 - e_{n-1}) 
\end{aligned}
\right. 
\end{multline*}

 \item D'après les calculs précédents, 
\begin{multline*}
 F_n\left( \frac{\sqrt{n}}{2}(e_1+e_{n-1})+u\right) =\frac{\sqrt{n}}{2}2u+F_n(u)
=\sqrt{n}u+\frac{n}{2}(e_1+e_{n-1})\\=\sqrt{n}\left(\frac{\sqrt{n}}{2}(e_1+e_{n-1})+u \right) 
\end{multline*}
Les autres calculs sont analogues et conduisent à
\begin{align*}
 F_n(u_+)= \sqrt{n}u_+&,& F_n(u_-)= -\sqrt{n}u_-&,&
F_n(v_+)= i\sqrt{n}v_+&,& F_n(v_-)= -i\sqrt{n}v_-
\end{align*}

\end{enumerate}

 \item
\begin{enumerate}
 \item Avec les conventions de l'énoncé,
\begin{displaymath}
 \left. 
\begin{aligned}
 A&= a_0+a_1X+a_2 X^2+\cdots \\
 B&= a_0 + a_2X + a_4X^2+\cdots \\
 C&= a_1+ a_3X+a_5X^2 + \cdots
\end{aligned}
\right\rbrace \Rightarrow
A = \widehat{B}(X^2) + X\widehat{C}(X^2)
\end{displaymath}

 \item La première relation demandée est obtenue simplement en remplaçant $X$ par $\omega$ dans la relation $A = \widehat{B}(X^2) + X\widehat{C}(X^2)$ de la question précédente. Pour les puissances au delà de $\frac{n}{2}$,
\begin{multline*}
 A(\omega^{\frac{n}{2}+k})=
a_0+a_1\omega^{\frac{n}{2}+k}+a_2\omega^{n+2k}+a_3\omega^{3(\frac{n}{2}+k)}+a_4\omega^{2n+4k}+\cdots \\
= \left( a_0 + a_2\omega'^k +a_4\omega'^{2k} +\cdots \right) 
+ \omega^{\frac{n}{2}+k}\left( a_1 + a_3\omega'^k +a_5\omega'^{2k} +\cdots \right) \text{ car } \omega^n =1\\
= B(\omega'^k)-\omega^kC(\omega'^k)\text{ car } \omega^{\frac{n}{2}}=-1 . 
\end{multline*}

 \item Il faut bien faire attention ici à la relation $n=2^N$. La récursivité est relative au passage de $N$ à $N-1$ c'est à dire de $n$ à $\frac{n}{2}$.
La méthode envisagée pour le calcul de $F_n(a)$ est la suivante
\begin{itemize}
 \item On forme à partir de $a$ les deux $\frac{n}{2}$-uplets $b$ et $c$ en séparant les indices pairs et impairs.
 \item On calcule $F_{\frac{n}{2}}(a)$ et $F_{\frac{n}{2}}(b)$.
 \item On renvoie $F_n(a)$ calculé à partir de $F_{\frac{n}{2}}(a)$ et $F_{\frac{n}{2}}(b)$ avec les formules de 4.b.
\end{itemize}

 \item Organisation du calcul et évaluation de la complexité.
\begin{itemize}
 \item La formation de $b$ et $c$ se fait sans opérations (au sens de l'énoncé)
 \item Le calcul récursif de $F_{\frac{n}{2}}(a)$ et $F_{\frac{n}{2}}(b)$ se fait avec $2U_{N-1}$ opérations.
 \item On effectue $\frac{n}{2}$ multiplication pour calculer les $\omega^k C(\omega'^k)$. On effectue ensuite $2\times \frac{n}{2}$ opérations (ajouter-soustraire) pour calculer $A(\omega^{k})$ et $A(\omega^{\frac{n}{2}+k})$. 
\end{itemize}
Comme $\frac{n}{2}=2^{N-1}$, on obtient donc bien en tout
\begin{displaymath}
 u_N = 2u_{N-1} + 3\times 2^{N-1}
\end{displaymath}

 \item Comme l'énoncé nous y invite, posons $v_N= u_N2^{-N}$. On en tire $u_N=2^{N}v_N$ avec lequel on réécrit la relation de récurrence. On peut diviser par $2^{N}$ et obtenir une suite arithmétique
\begin{multline*}
 2^Nv_N=2^Nv_{N-1}+3\times 2^{N-1}
\Rightarrow v_N=v_{N-1}+\frac{3}{2}\Rightarrow v_N =\frac{3N}{2}\\
\Rightarrow u_N=\frac{3}{2}N\,2^N=\frac{3}{2\ln 2}n\ln n
\end{multline*}

\end{enumerate}

 \item
\begin{enumerate}
 \item D'après la définition, $F_n(r)$ est le produit terme à terme de $F_n(p)$ et $F_n(q)$. Avec les notations précisées par l'énoncé, on peut écrire $F_n(r)=F_n(p)F_n(q)$.
 \item Nombre d'opérations pour la formule usuelle du produit de deux polynômes. Présentons ces nombres d'opérations dans un tableau pour la première moitié des coefficients. Le nombre d'opérations sera le même par symétrie pour les coefficients suivants.
\begin{center}\renewcommand{\arraystretch}{1.8}
\begin{tabular}{|c|c|c|} \hline
degré & $*$ & $+$\\ \hline
$0$ & $1$ & $0$\\ \hline
$1$ & $2$ & $1$\\ \hline
$2$ & $3$ & $2$\\ \hline
$\vdots$ & $\vdots$ & $\vdots$\\ \hline
$\frac{n}{4}-1$ & $\frac{n}{4}$ & $\frac{n}{4}-1$ \\ \hline
\end{tabular}
\end{center}
On en déduit après calculs que le nombre total d'opérations par cette méthode est $\frac{1}{16}n^2$.
 \item Il est clair d'après la question a. et le fait que $F_n$ est bijective que l'on calcule ainsi le $2n$-uplet attaché au produit des polynômes $P$ et $Q$.\newline
\'Evaluons le nombre d'opérations effectuées.
\begin{itemize}
 \item $2u_N$ opérations pour le calcul de $F_n(p)$ et $F_n(q)$.
 \item $n=2^N$ opérations pour le produit terme à terme
 \item $u_N$ opérations pour le calcul de la transformée inverse qui est analogue au direct.
\end{itemize}
On obtient donc $3u_N + 2^N=\frac{9}{2\ln 2}n\ln n +n$ opérations.
 \item Pour de grands $n$, la méthode utilisant la transformée de Fourier discète est beaucoup plus rapide car $\frac{9}{2\ln 2}n\ln n +n$ est beaucoup plus petit que $\frac{1}{16}n^2$.
\end{enumerate}

\end{enumerate}
