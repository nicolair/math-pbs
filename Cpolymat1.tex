\subsection*{Partie I.}
\begin{enumerate}  
  \item La substitution de $X$ par un polynôme de degré $1$ ne change pas le degré. La fonction est bien à valeurs dans $\R_2[X]$. Les propriétés de la substitution entrainent la linéarité. 

  \item Les propriétés de la substitution entrainent la linéarité.

  \item Pour former la matrice, on calcule les images des trois vecteurs de base.
\begin{align*}
 f(1) =& 1 \\
 f(X) =& \frac{1}{2}\left( \frac{X}{2}+\frac{X+1}{2}\right)=\frac{2X+1}{4} \\
 f(X^2) =& \frac{1}{2}\left( \frac{X^2}{4}+\frac{(X+1)^2}{4}\right)=\frac{2X^2+2X+1}{8}
\end{align*}
On en déduit la matrice demandée :
\renewcommand{\arraystretch}{1.3}
\begin{displaymath}
 \Mat_{\mathcal B}f_{_2}=
\begin{pmatrix}
 1 & \frac{1}{4} & \frac{1}{8} \\
 0 & \frac{1}{2} & \frac{1}{4} \\
 0 & 0 & \frac{1}{4} 
\end{pmatrix}
\end{displaymath}

 \item La matrice  de $f$ dans $\mathcal B$ est triangulaire supérieure avec des réels non nuls sur la diagonale, elle est donc de rang $3$ ce qui prouve que $f$ est un automorphisme.

 \item Le noyau $\ker \varphi$ est formé par les polynômes de degré inférieur à 2 et admettant $1$ comme racine c'est à dire divisibles par $X-1$ soit 
\begin{displaymath}
 \ker \varphi = (X-1)\R_1[X] , \hspace{0.5cm} \text{base :}\; (X-1, X^2 - X)
\end{displaymath}
On en déduit $\dim(\ker \varphi) = \dim (\R_1[X])=2$. On peut remarquer que $\ker \varphi$ est un hyperplan de $\R_2[X]$ (noyau d'une forme linéaire non nulle).

  \item L'application $\varphi$ est une forme linéaire donc elle n'est pas injective car son noyau est un hyperplan (un \og gros\fg~sous-espace) mais elle est surjective car elle n'est pas nulle et à valeurs dans le corps de base.
\end{enumerate}

\subsection*{Partie II.}
\begin{enumerate}
  \item La base canonique de $\R$ est la famille $(1)$ à un seul élément. Dans cette base, un nombre réel $x$ considéré comme un vecteur s'écrit $x = x\, 1$, son unique coordonnée est donc $x$. La matrice d'une forme linéaire $\varphi$ est donc la ligne des images par $\varphi$ des vecteurs de base. On en déduit
\begin{displaymath}
 L =
\left( 
 \varphi(1) \hspace{0.3cm} \varphi(X) \hspace{0.3cm} \varphi(X^2)
\right) 
=
\left( 
 1 \hspace{0.3cm} 1 \hspace{0.3cm} 1
\right) 
\end{displaymath}

 \item 
\begin{enumerate}
\item Notons $P_0 = 1$, $P_1 = -2X + 1$, $P_2 = 6X^2 - 6X + 1$ de sorte que $\mathcal{B}' = (P_0, P_1, P_2)$. Cette famille étant échelonnée, elle est libre dans $\R_2[X]$. C'est donc une base car $\R_2[X]$ est de dimension 3. 

\item Par définition d'une matrice de passage,
\renewcommand{\arraystretch}{1.}
\begin{displaymath}
 Q = P_{\mathcal{B}, \mathcal{B}'}=
\begin{pmatrix}
 1 & 1 & 1 \\ 0 & -2 & -6 \\ 0 & 0 & 6
\end{pmatrix}
\end{displaymath}

\item Comme toute matrice de passage entre deux bases, $Q$ est inversible. Pour calculer son inverse, on exprime la base canonique de $\R_2[X]$ :
\renewcommand{\arraystretch}{1.3}
\begin{displaymath}
\left. 
\begin{aligned}
 P_0 =& 1 \\ P_1 =& -2X + 1 \\ P_2 =& 6X^2 - 6X + 1
\end{aligned}
\right\rbrace \Rightarrow
\left\lbrace 
\begin{aligned}
 1 =& P_0 \\ X=& \frac{1}{2}P_0 - \frac{1}{2}P_1 \\ X^2 =& \frac{1}{6}P_2 -\frac{1}{2}P_1 + \frac{1}{3}P_0 
\end{aligned}
\right. 
\Rightarrow 
Q^{-1}=
\begin{pmatrix}
 1 & \frac{1}{2} & \frac{1}{3}\\ 0 & -\frac{1}{2} & -\frac{1}{2} \\ 0 & 0 & \frac{1}{6} 
\end{pmatrix} 
\end{displaymath}
\end{enumerate}
 
 \item 
\begin{enumerate}
\item  Les calculs de I.3. montrent que $A = \Mat_{\mathcal{B}} f$. D'après la formule de changement de base, $D = \Mat_{\mathcal{B}'} f = Q^{-1} A Q $ avec
\begin{multline*}
Q^{-1} A =
\begin{pmatrix}
 1 & \frac{1}{2} & \frac{1}{3}\\ 0 & -\frac{1}{2} & -\frac{1}{2} \\ 0 & 0 & \frac{1}{6} 
\end{pmatrix} 
\begin{pmatrix}
 1 & \frac{1}{4} & \frac{1}{8} \\
 0 & \frac{1}{2} & \frac{1}{4} \\
 0 & 0 & \frac{1}{4} 
\end{pmatrix}
=
\begin{pmatrix}
 1 & \frac{1}{2} & \frac{1}{3} \\ 
 0 & -\frac{1}{4} & -\frac{1}{4} \\
 0 &  0          & \frac{1}{24}
\end{pmatrix} \\
\Rightarrow
D = \Mat_{\mathcal{B}'} f=
\begin{pmatrix}
 1 & \frac{1}{2} & \frac{1}{3} \\ 
 0 & -\frac{1}{4} & -\frac{1}{4} \\
 0 &  0          & \frac{1}{24}
\end{pmatrix}
\begin{pmatrix}
 1 & 1 & 1 \\ 0 & -2 & -6 \\ 0 & 0 & 6
\end{pmatrix}
=
\begin{pmatrix}
 1 & 0 & 0 \\ 0 & \frac{1}{2} & 0 \\ 0 & 0 & \frac{1}{4}
\end{pmatrix}
\end{multline*}

\item On en déduit $A = Q D Q^{-1}$ ce qui entraine $A^n = Q D^n Q^{-1}$ après simplification des $Q^{-1}Q$ intermédiaires.

\item En fait, on cherche la matrice de $\varphi \circ f^n$. La composition des applications linéaires se traduit par une multiplication matricielle lorsque les bases correspondent. La matrice cherchée est
\begin{displaymath}
 L\,Q\,D^n\,Q^{-1}
\end{displaymath}
Il s'agit d'une matrice ligne.
\end{enumerate}

 \item Considérons un $P = a + bX + cX^2 \in \R_2[X]$ et exprimons matriciellement une valeur de la suite
\renewcommand{\arraystretch}{1.}
\begin{displaymath}
\varphi(f^n(P)) =  L\,Q\,D^n\,Q^{-1}\,C \hspace{0.5cm} \text{avec }
C=
\begin{pmatrix}
 a \\ b \\ c
\end{pmatrix}
\text{ et }
D^n=
\begin{pmatrix}
 1 & 0 & 0 \\ 0 & 2^{-n} & 0 \\ 0 & 0 & 2^{-2n}
\end{pmatrix}
\end{displaymath}
La suite étudiée s'exprime comme une combinaison linéaire d'une suite constante et de deux suites géométriques qui convergent vers $0$. Elle est donc convergente. Calculer la limite, revient à remplacer par $0$ les suites qui tendent vers $0$ dans $D^n$ ce qui rend très simples les calculs matriciels.
\renewcommand{\arraystretch}{1.3}
\begin{displaymath}
\begin{pmatrix}
 1 & 0 & 0 \\ 0 & 0 & 0 \\ 0 & 0 & 0
\end{pmatrix}
\begin{pmatrix}
 1 & \frac{1}{2} & \frac{1}{3}\\ 0 & -\frac{1}{2} & -\frac{1}{2} \\ 0 & 0 & \frac{1}{6} 
\end{pmatrix} 
=
\begin{pmatrix}
 1 & \frac{1}{2} & \frac{1}{3}\\ 0 & 0 & 0 \\ 0 & 0 & 0 
\end{pmatrix} 
\end{displaymath}
\begin{displaymath}
\begin{pmatrix}
 1 & 1 & 1 \\ 0 & -2 & -6 \\ 0 & 0 & 6
\end{pmatrix}
\begin{pmatrix}
 1 & \frac{1}{2} & \frac{1}{3}\\ 0 & 0 & 0 \\ 0 & 0 & 0 
\end{pmatrix} 
=
\begin{pmatrix}
 1 & \frac{1}{2} & \frac{1}{3}\\ 0 & 0 & 0 \\ 0 & 0 & 0 
\end{pmatrix} 
\end{displaymath}
\begin{displaymath}
\begin{pmatrix}
 1 & 1 & 1
\end{pmatrix}
\begin{pmatrix}
 1 & \frac{1}{2} & \frac{1}{3}\\ 0 & 0 & 0 \\ 0 & 0 & 0 
\end{pmatrix} 
=
\begin{pmatrix}
 1 & \frac{1}{2} & \frac{1}{3} 
\end{pmatrix} 
\end{displaymath}
On en déduit que $\left( \varphi(f^n(P))\right)_{n\in \N}$ converge vers
\begin{displaymath}
\begin{pmatrix}
 1 & \frac{1}{2} & \frac{1}{3} 
\end{pmatrix} 
\begin{pmatrix}
 a \\ b \\ c
\end{pmatrix}
= a + \frac{b}{2} + \frac{c}{3}
= \int_0^1 \widetilde{P}(t)\,dt.
\end{displaymath}
\end{enumerate}

\subsection*{Partie III.}
\begin{enumerate}
 \item Notons $\mathcal{P}_n$ la proposition à démontrer.
\begin{displaymath}
\mathcal{P}_n: \hspace{0.5cm} \left( \forall p \in \R_2[X], \; f^n(P) = \frac{1}{2^n}\sum_{k=0}^{2^n-1}\widehat{P}(\frac{X+k}{2^n})\right)  
\end{displaymath}
et montrons par récurrence que les propositions $\mathcal{P}_n$ sont vraies pour $n\geq 1$.\newline
Pour $n=1$, il s'agit de la définition même de $f$.\newline
Montrons que $\mathcal{P}_n$ entraine $\mathcal{P}_{n+1}$. Pour $k\in \llbracket 0 , 2^{n}-1 \rrbracket$, introduisons des notations polynomiales
\begin{displaymath}
 Q_k = \widehat{P}(\frac{X+k}{2^n})
\Rightarrow 
\left\lbrace 
\begin{aligned}
\widehat{Q_k}(\frac{X}{2}) =& \widehat{P}(\frac{X + 2k}{2^{n+1}}) \\
\widehat{Q_k}(\frac{X+1}{2}) =& \widehat{P}(\frac{X + 2k+1}{2^{n+1}}) 
\end{aligned}
\right. 
\end{displaymath}
En sommant sur les $k<2^{n}$, on obtient, avec les $2k$ et $2k+1$ tous les $k'<2^{n+1}$. Par linéarité,
\begin{multline*}
 f^{n+1}(P)
 = \frac{1}{2^n}\sum_{k=0}^{2^n -1} f(Q_k)
 = \frac{1}{2^{n+1}}\sum_{k=0}^{2^n -1}\left( \widehat{P}(\frac{X + 2k}{2^{n+1}}) + \widehat{P}(\frac{X + 2k+1}{2^{n+1}}) \right) \\
 = \frac{1}{2^{n+1}}\sum_{k'=0}^{2^{n+1} -1} \widehat{P}(\frac{X + k'}{2^{n+1}})
\end{multline*}
en regroupant dans la même somme les termes pairs $2k$ et le termes impairs $2k+1$.

 \item D'après la formule précédente, $\varphi(f^n(P))$ s'exprime comme une somme de Riemann (vers la droite) :
\begin{displaymath}
\varphi(f^n(P)) =  
\frac{1}{2^n}\sum_{k=0}^{2^n-1}\widetilde{P}(\frac{k+1}{2^n})
\end{displaymath}
 attachée à la subdivision régulière de pas $2^{-n}$. La suite des sommes de Riemann converge vers l'intégrale car la fonction polynomiale est continue.\newline
Dans ce raisonnement, c'est la continuité qui joue le rôle important. La limite vers l'intégrale sera donc valable pour des polynômes de n'importe quel degré. 
\end{enumerate}
