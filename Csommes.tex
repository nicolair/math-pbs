\begin{enumerate}
  \item Calcul de $F$. On remarque que $(k+1)! - k! = k\,k!$ et que l'on peut faire commencer la somme à $k=1$ car la contribution de $k=0$ est nulle. On en déduit, par sommation télescopique,
\begin{displaymath}
  F =(2! -1!) + (3! - 2!) + \cdots + ((n+1)! -n!) = (n+1)! -1.
\end{displaymath}

Calcul de $B$. Introduisons la fonction polynomiale
\begin{displaymath}
  f(x) = (1+x)^n = \sum_{k=0}^{n}\binom{n}{k}x^k.
\end{displaymath}
et sa primitive nulle en $0$:
\begin{displaymath}
  F(x) = \frac{1}{n+1}\left( (1+x)^{n+1} -1\right) = \sum_{k=0}^{n}\frac{1}{k+1}\binom{n}{k}x^{k+1} .
\end{displaymath}
On a alors
\begin{displaymath}
  B = F(1) = \frac{2^{n+1}-1}{n+1}.
\end{displaymath}

  \item 
\begin{enumerate}
  \item Pour tout entier $n$, on note $\mathcal{I}_n$ l'inégalité à démontrer.\newline
Si $n= 1: \; P_n = \frac{1}{2}$ et $\frac{1}{\sqrt{2n+1}} = \frac{1}{\sqrt{3}}$. Comme $\frac{1}{4}< \frac{1}{3}$, on a bien $P_n < \frac{1}{\sqrt{2n+1}}$ dans ce cas.\newline
On veut maintenant montrer que, pour tout entier $n$, $\mathcal{I}_n \Rightarrow \mathcal{I}_{n+1}$. On peut retrouver dans $P_{n+1}$ les facteurs constituant $P_n$.
\begin{displaymath}
  P_{n+1} = P_n \,\frac{2n+1}{2(n+1)} \underset{\text{d'après }\mathcal{I}_n}{\underbrace{< \frac{1}{\sqrt{2n+1}}}} \frac{2n+1}{2(n+1)}
  = \frac{\sqrt{2n+1}}{2(n+1)} .
\end{displaymath}
Pour montrer que $\mathcal{I}_n \Rightarrow \mathcal{I}_{n+1}$, il suffit donc de vérifier que
\begin{displaymath}
  \frac{\sqrt{2n+1}}{2(n+1)} \leq \frac{1}{\sqrt{2n+3}}.
\end{displaymath}
Or cette inégalité est équivalente à
\begin{displaymath}
  \sqrt{2n+1} \sqrt{2n+3} \leq 2(n+1)
\end{displaymath}
et celle ci est une conséquence de
\begin{displaymath}
  (2(n+1))^2 -(2n+1)(2n+3) = 1 \geq 0 \text{ pour } n\geq 1.
\end{displaymath}

  \item Le numérateur de $P_n$ est le produit des impairs consécutifs alors que le dénominateur est le produit des pairs. Avec la remarque de l'énoncé, tous les entiers figurent au numérateur alors qu'au dénominateur, seuls les pairs figurent et ils y sont deux fois. Dans le produit des pairs, en mettant les $2$ en facteur, on retrouve une factorielle. On en tire:
\begin{displaymath}
  P_n = \frac{(2n)!}{(2^n n!)^2} = \frac{(2n)!}{2^{2n}(n!)^2} \;\text{ et }\;
  P_n = 2^{-2n} \binom{2n}{n}.
\end{displaymath}

  \item Formons le quotient des deux coefficients avec $0 \leq k < n$:
\begin{displaymath}
  \frac{\binom{2n}{k+1}}{\binom{2n}{k}}
= \frac{(2n)(2n-1)\cdots (2n-k)}{(k+1)!}\, \frac{k!}{(2n)(2n-1)\cdots (2n-k+1)}
= \frac{2n-k}{k+1}.
\end{displaymath}
avec $\frac{2n-k}{k+1}<1$ car $2n-k-k-1=2(n-k)-1>0$.
\end{enumerate}
On en déduit que la suite des $\binom{2n}{k}$ est strictement croissante de $0$ à $n$. D'après la formule $\binom{2n}{2n-k} = \binom{2n}{k}$, les mêmes valeurs se retrouvent au delà de $n$ donc $\binom{2n}{n}$ est le plus grand des coefficients $\binom{2n}{k}$ pour $k$ entre $0$ et $2n$.\newline
La partie droite de l'encadrement à prouver résulte de l'inégalité de a. et de l'expression de b.\newline
La partie gauche résulte d'une formule du binôme majorée simplement:
\begin{displaymath}
  2^{2n} = (1+1)^{2n} = \sum_{k=0}^{2n} \binom{2n}{k} \leq 
\underset{\text{nb de termes}}{\underbrace{(2n+1)}}\times
\underset{\text{le plus gd des termes}}{\underbrace{\binom{2n}{n}}}.
\end{displaymath}

\end{enumerate}
