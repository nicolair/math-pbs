%<dscrpt>Exercice sur une fonction dérivable.</dscrpt>
\begin{enumerate}
 \item Soit $f$ une fonction de classe $\mathcal{C}^1$ dans un intervalle $[a,b]$ telle que $f'$ soit dérivable dans $]a,b[$. Montrer qu'il existe $d$ dans $]a,b[$ tel que
\begin{displaymath}
 f(b)-f(a)-(b-a)f'(a) = \frac{(b-a)^2}{2}f''(d)
\end{displaymath}
On considèrera obligatoirement une fonction $\varphi$ de la forme :
\begin{displaymath}
 \varphi(t) = f(t)-f(a)-(t-a)f'(a)-K(t-a)^2
\end{displaymath}
pour un réel $K$ bien choisi.
\item Soit $G$ une fonction de classe $\mathcal{C}^2$ dans $[0,+\infty[$. On définit une fonction $g$ par
\begin{displaymath}
 \forall x\in [0,+\infty[:\;
g(x)=
\left\lbrace
\begin{aligned}
 &-G'(0) &\text{ si } x=0 \\
 &\frac{1}{x}\left( G(x^2)-G(x)\right) &\text{ si } x\neq 0 \\
\end{aligned}
 \right. 
\end{displaymath}
\begin{enumerate}
 \item Les fonctions définies dans $\left] 0, + \infty\right[$ 
\begin{displaymath}
x \mapsto \frac{G(x) - G(0)}{x},\hspace{1cm} x \mapsto \frac{G(x^2) - G(0)}{x} 
\end{displaymath}
admettent elles des limites en $0$? 
 \item Montrer que $g$ est continue dans $[0,+\infty[$.
 \item Comment se traduit le résultat de la question 1 appliqué à la fonction $G$ avec $a=0$ et $b=x$? Et avec $a=0$ et $b=x^2$? 
 \item Montrer que $g$ est de classe $\mathcal C^1$ dans  $[0,+\infty[$ et préciser $g'(0)$.
\end{enumerate}

\end{enumerate}
