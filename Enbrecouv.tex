%<dscrpt>Nombre de recouvrements.</dscrpt>

Soit $E$ un ensemble fini, on appelle \emph{$n$-recouvrement} de
$E$ un $n$-uplet 
$$(E_1, \cdots ,E_n)$$
 de parties de $E$ dont la
r{\'e}union est $E$.\newline Soient $A$ et $B$ deux ensembles
quelconques, on consid{\`e}re une application $\Phi_{AB}$ d{\'e}finie de
la mani{\`e}re suivante
\begin{eqnarray*}
\mathcal{F}(A,\mathcal{P}(B)) & \rightarrow &
\mathcal{F}(B,\mathcal{P}(A))\\
f & \rightarrow & \ \Phi_{AB} (f)= \varphi
\end{eqnarray*}
avec
\[\forall b \in B, \, \varphi(b)=\{a\in A , b\in f(a)\}\]
\begin{enumerate}
  \item Pr{\'e}ciser pour un exemple de $A$,$B$, $f$ de votre choix la
  fonction $\varphi$.
  \item Explicitez $\Phi_{AB}\circ \Phi_{BA}$ et $\Phi_{BA}\circ
  \Phi_{AB}$. Montrer que $\Phi_{AB}$ est une bijection. V{\'e}rifier,
  en les calculant autrement, que $\mathcal{F}(A,\mathcal{P}(B))$
  et $\mathcal{F}(B,\mathcal{P}(A))$ ont le m{\^e}me nombre d'{\'e}l{\'e}ments
  lorsque $A$ et $B$ sont finis.
    \item Dans toute la suite, on suppose que $A=\{1,\cdots , n\}$
    et $B=E$. Si $f$ est une application de $A$ dans
    $\mathcal{P}(E)$, on pose
    \[E_1=f(1),E_2=f(2),\cdots, E_n=f(n), \varphi=\Phi_{AB}(f)\]
     \begin{enumerate}
        \item D{\'e}signons par $\Omega_i$ l'ensemble des parties
        de $A$ contenant $i$. Montrer que
        \[E_i=\varphi^{-1}(\Omega_i)\]
        \item Montrer que $E_1\cup \cdots \cup E_n = E$ si et
        seulement si l'ensemble vide n'est pas un {\'e}l{\'e}ment de
        $\varphi (E)$. En d{\'e}duire le nombre de $n$-recouvrements
        de $E$. Montrer que
        \[\forall i \in \{1, \cdots ,n \} , E_i\neq \emptyset \Leftrightarrow \bigcup _{x\in E} \varphi (x) = \{1,\cdots ,n\} \]
     \end{enumerate}

\end{enumerate}
