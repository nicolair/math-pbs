\subsubsection*{Exercice 3}
\begin{enumerate}
\item Exprimons $\cos \theta$ à l'aide d'exponentielles. Il vient
\begin{displaymath}
  1-\cos\theta e^{i\theta}=1-\frac{1}{2}e^{2i\theta}-\frac{1}{2}=-i e^{i\theta}\sin \theta
\end{displaymath}
On en déduit que le module est $|\sin \theta |$. L'argument est
\begin{displaymath}
\theta - \frac{\pi}{2} \text{ si } \sin \theta > 0,  \hspace{1cm}
\theta + \frac{\pi}{2} \text{ si } \sin \theta < 0
\end{displaymath}

\item Notons $A_{n}$ et $B_{n}$ les deux sommes proposées. Ce sont les parties réelles de sommes de suites géométriques de raison respectivement $\cos\theta e^{i\theta}$ et $\frac{1}{\cos\theta} e^{i\theta}$. Lorsque $\theta$ n'est pas un multiple de $\frac{\pi}{2}$, les raisons sont différentes de 1, on en déduit
\begin{multline*}
A_{n} = \Re\left(\cos\theta \,e^{i\theta}\frac{1- e^{i n \theta}(\cos\theta)^{n}}{1-\cos\theta e^{i\theta}}\right) 
 = \Re\left(\cos\theta \,e^{i\theta}\frac{1- e^{i n \theta}(\cos\theta)^{n}}{-i e^{i\theta}\sin \theta }\right) \\
 = \frac{(\cos \theta)^{n+1}\sin(n\theta)}{\sin\theta}
\end{multline*}
\begin{displaymath}
B_{n} = \Re\left(\frac{1-\frac{ e^{i(n+1)\theta}}{(\cos\theta)^{n+1}}}{1-\frac{e^{i\theta}}{\cos\theta}}\right) 
 = \Re\left(\frac{1-\frac{ e^{i(n+1)\theta}}{(\cos\theta)^{n+1}}}{-i\tan\theta}\right)
 = \frac{\sin(n+1)\theta}{(\cos\theta)^{n}\sin\theta}
\end{displaymath}

\end{enumerate}
