%<dscrpt>Polynômes matrices et intégrales.</dscrpt>
On note $\mathcal B=(1,X,X^2)$ la base canonique du $\R$-espace vectoriel des polynômes à coefficients réels de degré inférieur ou égal à 2 noté $\R_2[X]$. Lorsque $P$ et $Q$ sont deux polynômes et $a$ un réel, le polynôme $\widehat{P}(Q)$ est obtenu en substituant $Q$ à $X$ dans l'expression de $P$. Le réel $\widetilde{P}(a)$ est obtenu en substituant $a$ à $X$ dans l'expression de $P$. \newline
On définit\footnote{d'après \'Epreuve toute filière du concours commun 2009 des écoles des mines d'Albi, Alès, Douai, Nantes} les deux applications suivantes :
\begin{align*}
 f:
\left\lbrace 
\begin{aligned}
 &\R_2[X] \rightarrow \R_2[X] \\
&P \rightarrow \frac{1}{2}
\Bigl[
\widehat{P}(\frac{X}{2}) + \widehat{P}(\frac{X+1}{2})
\Bigr]
\end{aligned}
\right. 
 & &
\varphi :
\left\lbrace
\begin{aligned}
& \R_2[X] \rightarrow \R \\
& P \rightarrow \widetilde{P}(1)
\end{aligned}
\right. 
\end{align*}
On rappelle aussi que l'on note $f^0=\Id_{\R_2[X]}$ et, pour tout $n\in \N^*$, $f^n=f\circ f^{n-1}$.
\subsection*{Partie I.}
\begin{enumerate}
 \item Vérifier que $f$ est bien à valeurs dans $\R_2[X]$ et montrer que $f$ est linéaire.
\item Montrer que $\varphi$ est linéaire.
\item \'Ecrire la matrice de $f$ dans la base $\mathcal B$ de $\R_2[X]$.
\item L'application $f$ est elle injective? surjective ?
\item Déterminer une base de $\ker \varphi$. Quelle est la dimension de $\ker \varphi$ ?
\item L'application $\varphi$ est-elle injective ? surjective ?
\end{enumerate}

\subsection*{Partie II}
On considère la matrice $A\in \mathcal M_3(\R)$ et la famille $\mathcal B'$ de $\R_2[X]$ : 
\renewcommand{\arraystretch}{2}
\begin{displaymath}
 A = 
\begin{pmatrix}
 1& \dfrac{1}{4} & \dfrac{1}{8} \\
 0& \dfrac{1}{2} & \dfrac{1}{4} \\
 0& 0 & \dfrac{1}{4} 
\end{pmatrix}
\hspace{1cm}
\mathcal B' = \bigr(1,-2X+1,6X^2-6X+1 \bigl)
\end{displaymath}
\begin{enumerate}
  \item \'Ecrire la matrice de $\varphi$ dans les bases canoniques de $\R_2[X]$ et $\R$. On notera $L$ cette matrice.
 
  \item 
\begin{enumerate}
  \item Justifier que la famille $\mathcal B'$ est une base de $\R_2[X]$.
  \item \'Ecrire la matrice de passage $Q$ de $\mathcal B$ à $\mathcal B'$.
  \item Justifier que $Q$ est inversible et calculer son inverse.
\end{enumerate}

\item 
\begin{enumerate}
 \item \'Ecrire la matrice $D$ de $f$ dans $\mathcal B'$.
 \item Pour tout $n\in \N$, exprimer $A^n$ en fonction de puissances de $Q$ et $D$.
 \item Pour tout $n\in \N$, exprimer, dans les bases canoniques de $\R_2[X]$ et $\R$, la matrice de l'application de $\R_2[X]$ dans $\R$ définie par :
\begin{displaymath}
 P \mapsto \varphi(f^n(P))
\end{displaymath}
en fonction de $L$ et de puissances de $Q$ et $D$.
\end{enumerate}

\item Montrer que 
\begin{displaymath}
 \forall P\in \R_2[X] :\;
\left( \varphi(f^n(P))\right)_{n\in \N} \rightarrow  \int_0^1\widetilde{P}(t)dt
\end{displaymath}
\end{enumerate}

\subsection*{Partie III}
\begin{enumerate}
 \item Montrer que :
\begin{displaymath}
 \forall P\in \R_2[X], \forall n\in \N :
f^n(P) = \frac{1}{2^n}\sum_{k=0}^{2^n-1}\widehat{P}(\frac{X+k}{2^n})
\end{displaymath}
\item En déduire une deuxième démonstration (indépendante de celle de la partie II) de
\begin{displaymath}
 \forall P\in \R_2[X] :
\left( \varphi(f^n(P))\right)_{n\in \N} \rightarrow  \int_0^1\widetilde{P}(t)dt
\end{displaymath}
Ce résultat est-il toujours valable sans restriction sur le degré?
\end{enumerate}
