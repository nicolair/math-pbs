%<dscrpt>Nombres rationnels : suites de Farey, cercles de Ford.</dscrpt>
\subsection*{Notations}

Le repr{\'e}sentant irr{\'e}ductible d'un nombre rationnel $x$ est une fraction $\frac{p}{q}$ telle que $x=\frac{p}{q}$ avec $p\in \Z$, $q\in \N^{*}$, $p$ et $q$ sans diviseur commun. On dira alors que $p$ est le \emph{num{\'e}rateur }et $q$ le \emph{d{\'e}nominateur }de $x$.
\newline
On convient que $\frac{0}{1}$ est le repr{\'e}sentant irr{\'e}ductible de $0$ et $\frac{1}{1}$ celui de $1$.\newline
Si $\frac{p}{q}$ et $\frac{p^{\prime }}{q^{\prime }}$ sont des repr{\'e}sentants irr{\'e}ductibles de nombres rationnels, on d{\'e}finit le \emph{m{\'e}dian} de ces nombres (not{\'e} $\mu (\frac{p}{q},\frac{p^{\prime}}{q^{\prime }})$) en posant
\begin{displaymath}
 \mu (\frac{p}{q},\frac{p^{\prime }}{q^{\prime }})=\frac{p+p^{\prime }}{q+q^{\prime }}
\end{displaymath}

\subsection*{Question pr{\'e}liminaire}
Montrer que $\frac{p}{q}<\frac{p^{\prime }}{q^{\prime }}$ entraine $\frac{p}{q}<\mu (\frac{p}{q},\frac{p^{\prime }}{q^{\prime }})<\frac{%
p^{\prime }}{q^{\prime }}$

\subsection*{Partie I.\quad M{\'e}dians et suites de Farey}
Pour tout entier $n$, on d{\'e}finit par r{\'e}currence un ensemble $\mathcal{M}_{n}$ de la mani{\`e}re suivante :

\begin{quote}
$\mathcal{M}_{0}=\left\{ 0,1\right\}$

$\mathcal{M}_{n+1}$ s'obtient à partir de $\mathcal{M}_{n}$ en ajoutant le m{\'e}dian entre deux termes cons{\'e}cutifs.
\end{quote}
Par exemple 
\begin{displaymath}
 \mathcal{M}_{1}=\left\{ 0,\frac{1}{2},1\right\}
\end{displaymath}
On d{\'e}finit aussi (pour tout entier $n$) l'ensemble $\mathcal{F}_{n}$ des rationnels écrits sous forme irréductible par :
\begin{displaymath}
 \frac{p}{q} \in \mathcal{F}_{n} \Leftrightarrow 0\leq p\leq q\leq n
\end{displaymath}

\begin{enumerate}
\item  \begin{enumerate}
 \item Pr{\'e}ciser $\mathcal{M}_{2}$, $\mathcal{M}_{3}$, $\mathcal{M}_{4}$ et les 10 premiers éléments de $\mathcal{M}_{5}$.
 \item Dans chacune des listes précédentes, pour chaque $\mathcal{M}_{i}$ correspondant, entourer les éléments de $\mathcal{F}_{i-1}$.
 \item Quel est le nombre d'{\'e}l{\'e}ments de $\mathcal{M}_{n}$?
\end{enumerate}

\item  Soient $\frac{p}{q}$, $\frac{p^{\prime }}{q^{\prime }}$, $\frac{p^{\prime \prime }}{q^{\prime \prime }}$ des repr{\'e}sentants
irr{\'e}ductibles de nombres rationnels tels que $p^{\prime }q-pq^{\prime}=1$.

\begin{enumerate}
\item  Montrer que 
\begin{displaymath}
 \frac{p}{q}<\frac{p^{\prime \prime }}{q^{\prime \prime }}<\frac{p^{\prime }}{q^{\prime }} \Rightarrow q+q^{\prime }\leq q^{\prime
\prime }
\end{displaymath}

\item  Soit $n\geq \max (q,q^{\prime })$, montrer que si 
\begin{displaymath}
 \left] \frac{p}{q},\frac{p^{\prime }}{q^{\prime }}\right[ \cap \mathcal{F}_{n}=\emptyset
\end{displaymath}
c'est {\`a} dire $\frac{p}{q}$ et $\frac{p^{\prime }}{q^{\prime }}$ sont cons{\'e}cutifs dans $\mathcal{F}_{n}$, alors
\begin{displaymath}
 \left] \frac{p}{q},\frac{p^{\prime }}{q^{\prime }}\right[ \cap \mathcal{F}_{n+1}
\subset \left\{ \mu (\frac{p}{q},\frac{p^{\prime }}{q^{\prime }})\right\}
\end{displaymath}
\end{enumerate}
\item Soit $x<y$ deux éléments consécutifs de $\mathcal{F}_{n+1}$.
 \begin{enumerate}
   \item  Montrer que  $x\not\in \mathcal{F}_{n}$ et $y\not\in \mathcal{F}_{n}$ est impossible.
   \item Montrer que si $x\in \mathcal{F}_{n}$ et $y\in \mathcal{F}_{n+1}$, il existe alors $z \in \mathcal{F}_{n}$ tel que $x$ et $z$ soient consécutifs dans $\mathcal{F}_{n}$ et $y=\mu(x,z)$. Quels sont les autres cas possibles?
  \end{enumerate} 

\item  Montrer la proposition suivante notée $\mathcal{P}_{n}$ pour tout entier $n\geq 2$.

Soit $x<y$ deux éléments consécutifs de $\mathcal{F}_{n}$ :
\begin{itemize}
 \item il existe un entier $i<n$ tel que $x$ et $y$ sont consécutifs dans $\mathcal{M}_{i}$ 
 \item $x=\frac{p}{q} \text{ et } x=\frac{p^\prime}{q^\prime} \text{ entraine } p^{\prime }q-pq^{\prime }=1$
\end{itemize}
On peut remarquer que cela entraine $\mathcal F _n \subset \mathcal M_{n-1}$.

\item  Soient $\frac{p}{q}$ et $\frac{p^{\prime }}{q^{\prime }}$ cons{\'e}cutifs dans $\mathcal{F}_{n}.$

\begin{enumerate}
\item  Montrer que $\frac{p}{q}$ et $\frac{p^{\prime }}{q^{\prime }}$ sont
cons{\'e}cutifs dans tous les $\mathcal{F}_{r}$ tels que
\[
\max (q,q^{\prime })\leq r\leq q+q^{\prime }-1
\]

\item  Montrer que
\[
\left] \frac{p}{q},\frac{p^{\prime }}{q^{\prime }}\right[ \cap \mathcal{F}%
_{q+q^{\prime }}=\left\{ \mu (\frac{p}{q},\frac{p^{\prime }}{q^{\prime }}%
)\right\}
\]
\end{enumerate}

\item  Soient $\frac{p}{q}$ et $\frac{p^{\prime }}{q^{\prime }}$ tels que $p^{\prime }q-pq^{\prime }=1$, montrer qu'ils sont cons{\'e}cutifs dans tous les $\mathcal{F}_{r}$ pour
\[
\max (q,q^{\prime })\leq r\leq q+q^{\prime }-1
\]
\end{enumerate}

\subsection*{Partie II.\quad Cercles de Ford.}

Soit $\mathcal{C}$ un cercle de centre $C$ de coordonn{\'e}es $(x,y)$ et $\mathcal{C}^{\prime }$ un cercle de centre $C^{\prime }$ de coordonn{\'e}es $ (x^{\prime },y^{\prime })$. On pourra utiliser que $\mathcal{C}$ et $\mathcal{C}^{\prime }$ sont tangents si et seulement si $CC^{\prime}=r+r^{\prime }$ c'est {\`a} dire
\[
(x-x^{\prime })^{2}+(y-y^{\prime })^{2}=(r+r^{\prime })^{2}
\]

On s'int{\'e}resse aux cercles tangents {\`a} l'axe des $x$ et situ{\'e}s au dessous de cet axe. Si $u$ est l'abcisse du point de contact et si $r$ est le rayon du cercle alors les coordonn{\'e}es du centre sont $(u,-r)$.\newline
On dira qu'un tel cercle est \emph{un cercle de Ford}.\newline
Plus particuli{\`e}rement, si $\frac{p}{q}$ est le repr{\'e}sentant irr{\'e}ductible d'un nombre rationnel $x$, le cercle  $\mathcal{C}_{x}$ est défini par son centre et son rayonle cercle de centre de coordonn{\'e}es
\begin{displaymath}
\text{ centre } C_{x}: \left( \frac{p}{q}, -\frac{1}{q^2}\right), \hspace{0.5cm}\text{ rayon }: \frac{1}{2q^{2}}
\end{displaymath}
Dans cette partie, $\frac{p}{q}$ et $\frac{p^{\prime }}{q^{\prime }}$ avec $\frac{p}{q}<\frac{p^{\prime }}{q^{\prime }}$ sont les repr{\'e}sentants irr{\'e}ductibles de deux nombres rationnels.
\begin{enumerate}
\item  Donner une condition n{\'e}cessaire et suffisante assurant que $\mathcal{C}_{\frac{p}{q}}$ et $\mathcal{C}_{\frac{p^{\prime }}{q^{\prime }}}$ sont tangents.

\item  Pr{\'e}ciser les cercles de Ford tangents {\`a} $\mathcal{C}_{\frac{p}{q}}$ et $\mathcal{C}_{\frac{p^{\prime }}{q^{\prime }}}$ (donner les coordonn{\'e}es du centre)

\item  On suppose que $\mathcal{C}_{\frac{p}{q}}$ et $\mathcal{C}_{\frac{p^{\prime }}{q^{\prime }}}$ sont tangents, pr{\'e}ciser le cercle de Ford tangent {\`a} $\mathcal{C}_{\frac{p}{q}}$ et $\mathcal{C}_{\frac{p^{\prime }}{q^{\prime }}}$ et dont le point de contact avec l'axe des $x$ est entre $\frac{p}{q}$ et $\frac{p^{\prime }}{q^{\prime }}$.

\item  Comment se pr{\'e}sentent les cercles $\mathcal{C}_{x}$ pour $x\in \mathcal{F}_{n}$ ?
\end{enumerate}

\subsection*{Partie III.\quad Approximation de Dirichlet}
Soit $x$ un nombre irrationnel et $Q$ un entier naturel non nul. En consid{\'e}rant les approximations par exc{\`e}s et par d{\'e}faut de $x$ dans $\mathcal{F}_{Q}$, montrer qu'il existe un rationnel $\frac{a}{q}$ tel que
\[
q\leq Q\text{ et }\left| x-\frac{a}{q}\right| \leq \frac{1}{q(Q+1)}
\]
