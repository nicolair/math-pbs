%<dscrpt>Définition géométrique d'une opération dans le plan. Associativité. Récurrence linéaire.</dscrpt>
Soit $A$ le point $(1,0)$ et $B$ le point $(0,1)$ dans le plan $\R^2$. Avec ces points fixés, on forme une opération (notée $*$) dans le plan.\newline
Soit $M_0=(x_0,y_0)$ et $M_1=(x_1,y_1)$. \newline
On construit $P_0$ par les conditions :
\begin{eqnarray*}
(P_0M_0)\parallel Ox \\
P_0 \in (AB)
\end{eqnarray*}
On construit $Q_0$ par les conditions :
\begin{eqnarray*}
(P_0 Q_0) \parallel (M_1B)\\
Q_0 \in (AM_1)
\end{eqnarray*}
On construit le point $M_2=M_0*M_1$ en imposant que $M_0P_0Q_0M_2$ est un parallélogramme.
\begin{enumerate}
\item On note $(x_2,y_2)$ le point $M_2$. Calculer les coordonnées de $P_0$ et $Q_0$. Montrer que
\[\left\lbrace  
\begin{array}{lcl}
x_2 & = & x_0+x_1y_0 \\ 
y_2 & = & y0 y1 
\end{array}
\right. \]
\item Montrer que $*$ est associative, admet un élément neutre et que si $y\neq0$, $(x,y)$ admet un inverse.
\item On définit une suite $(M_n)_{n\in\N}$ par $M_0$, $M_1$ et 
\[M_n=M_{n-1}*M_{n-2}\]
pour tout entier $n\geq 2$. On pose $M_n=(x_n,y_n)$, déterminer $y_n$ en fonction de $y_0$ et $y_1$.
\end{enumerate}

