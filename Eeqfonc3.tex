%<dscrpt>Equation fonctionnelle.</dscrpt>
On cherche\footnote{d'après Leçons sur quelques équations fonctionnelles E Picard 1928. Voir \href{\baseurl Aeqfonc2.pdf}{Aeqfonc2.pdf}} les fonctions deux fois dérivables dans $\R$ et à valeurs complexes vérifiant l'équation fonctionnelle
\begin{displaymath}
 \forall (x,y)\in\R^2 : f(x+y) + f(x-y) = 2f(x)f(y)\hspace{2cm}(1)
\end{displaymath}
\begin{enumerate}
 \item Soit $f$ une fonction qui n'est pas la fonction nulle et vérifiant la relation.
\begin{enumerate}
 \item Montrer que $f(0)=1$ et que $f$ est paire.
\item Montrer que 
\begin{displaymath}
 \forall(x,y)\in \R^2 : f(x)f^{\prime\prime}(y)=f^{\prime\prime}(x)f(y)
\end{displaymath}
\item Montrer que
\begin{displaymath}
 \forall x\in \R : f(x)=\dfrac{1}{2}\left( e^{\lambda x} + e^{-\lambda x}\right) 
\end{displaymath}
où $\lambda$ est une racine carrée (complexe) de $f^{\prime\prime}(0)$.
\end{enumerate}
\item \begin{enumerate}
 \item Montrer que pour tout nombre complexe $\lambda$, la fonction définie par :
\begin{displaymath}
 \forall x\in \R : f(x)=\dfrac{1}{2}\left( e^{\lambda x} + e^{-\lambda x}\right) 
\end{displaymath}
vérifie l'équation fonctionnelle.

\item Quelles sont les fonctions à valeurs réelles qui vérifient la relation ?
\end{enumerate}

\end{enumerate}
