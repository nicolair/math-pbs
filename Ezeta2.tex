%<dscrpt>Autour de zeta(2).</dscrpt>
Ce problème porte sur le nombre $\zeta(2)$\footnote{d'après Concours commun mines Albi-Alès-Douai-Nantes, 2002, MPSI}.
Pour $p\in \C$, lorsque la suite
\begin{displaymath}
 \left(1+\frac{1}{2^{p}}+\frac{1}{3^{p}}+\cdots++\frac{1}{n^{p}} \right)_{n\in \N^*}
\end{displaymath}
converge, sa limite est notée $\zeta(p)$ (fonction \emph{zeta} de Riemann). \\ L'objet de ce problème est de montrer que $\zeta(2)=\displaystyle \frac{\pi^2}{6}$ et que ce nombre est irrationnel.
\subsection*{Partie I - Convergence de la suite}
Dans cette partie, pour tous entiers naturels non nuls $p$ et $n$, on pose~:
\begin{displaymath}
 S_n(p)=\sum_{k=1}^n\frac{1}{k^p}.
\end{displaymath}
\begin{enumerate}
 \item Montrer que, pour tout entier $k\geq 1$,
\begin{displaymath}
 \frac{1}{(k+1)^p}\leq \int_k^{k+1}\frac{dx}{x^p}\leq \frac{1}{k^p}.
\end{displaymath}
\item Montrer que, pour tout entier $n\geq2$,
\begin{displaymath}
 S_n(p)-1\leq \int_1^{n}\frac{dx}{x^p}\leq S_{n-1}(p).
\end{displaymath}
\item Montrer qu'une primitive de la fonction
\begin{displaymath}
 \left\lbrace
\begin{aligned}
 \left[ 1,+\infty\right[   & \longrightarrow  \R \\
 t &\longmapsto \displaystyle \frac{1}{t^p}
\end{aligned}
\right.
\end{displaymath}
est majorée si et seulement si $p\geq 2$.
\item Montrer que la suite $\left( S_n(p)\right) _{n\in \N^*}$ converge si et seulement si $p\geq 2$.
\end{enumerate}

\subsection*{Partie II - Calcul de la limite}
Dans cette partie, on définit une fonction $h$ dans $\R$ et une fonction $\varphi$ dans $[0,\pi]$ par:
\begin{displaymath}
 \forall t\in \R,\;h(t)=\frac{1}{2\pi}t^2 -t,
\hspace{1cm} \varphi(t)=
\left\lbrace
\begin{aligned}
 &-1 &\text{ si }& t=0\\
 &\frac{h(t)}{2\sin \displaystyle \frac{t}{2}} &\text{ si }& t\in ]0,\pi ]
\end{aligned}
\right.
\end{displaymath}
\begin{enumerate}
 \item Montrer que $\varphi\in \mathcal C^{1}(\left[ 0,\pi\right] )$.
 \item Calculer $\displaystyle \int_0^\pi h(t)\cos(kt)\,dt$ pour tout $k$ entier naturel non nul.
 \item Pour $t\in \left] 0,\pi\right]$, donner une expression factorisée de $\displaystyle \sum_{k=1}^n\cos(kt)$. En déduire une constante $\lambda$ (à préciser) telle que:
\begin{displaymath}
 \forall t\in \left] 0,\pi\right],\;
\sum_{k=1}^n\cos(kt)=
\frac{\displaystyle \sin\left(\left(n+\frac{1}{2}\right)t \right) }{\displaystyle 2\sin \left(\frac{t}{2}\right)} -\lambda.
\end{displaymath}
\item Montrer à l'aide d'une intégration par parties que, pour toute fonction $\psi\in \mathcal C^1(\left[0,\pi\right])$, la suite $\displaystyle \left( \int_0^\pi\psi(t)\sin\left( \left(n+\frac{1}{2}\right)t\right)\,dt \right) _{n\in \N}$ converge vers $0$.
\item Montrer que
\begin{displaymath}
 \zeta(2)=\frac{\pi^2}{6}.
\end{displaymath}

\end{enumerate}

\subsection*{Partie III - Irrationalité}
Dans cette partie, on veut montrer que $\pi^2$ est irationnel. On va raisonner par l'absurde en supposant que $\displaystyle \pi^2=\frac{a}{b}$ avec $a$ et $b$ naturels non nuls. On pose aussi~:
\begin{displaymath}
 \forall n\in \N^*, \forall x\in \R, \ 
f_n(x)=\frac{1}{n!}x^n(1-x)^n.
\end{displaymath}
\begin{enumerate}
 \item Dans cette question, $n\in \N^*$.
\begin{enumerate}
 \item Montrer qu'il existe $n+1$ entiers $e_n, e_{n+1},\cdots,e_{2n}$ tels que \[f_n(x)=\frac{1}{n!}\sum\limits_{i=n}^{2n}e_ix^i.\]
 \item Montrer que pour tout $k \in \mathbb{N}$, $f^{(k)}(0)$ et $f^{(k)}(1)$ sont des entiers. \\ (On pourra remarquer que $f_n(x)=f_n(1-x)$).
\end{enumerate}
 \item Pour tout $n$ entier naturel non nul, on définit une fonction $F_n$ et une fonction $g_n$ dans $\R$ par~:
\begin{align*}
 F_n(x)&=
b^n\left(
\pi^{2n}f_n(x)-\pi^{2n-2}f_n^{(2)}(x)+\pi^{2n-4}f_n^{(4)}(x)- \cdots +(-1)^nf^{(2n)}(x)
 \right), \\
g_n(x) &= F_n'(x)\sin(\pi x)-\pi F_n(x)\cos(\pi x).
\end{align*}
\begin{enumerate}
 \item Montrer que $F_n(0)$ et $F_n(1)$ sont entiers.
 \item Montrer que $g_n'(x)=\pi^2a^nf_n(x)\sin(\pi x)$ et que $A_n$ est entier, où
\begin{displaymath}
 A_n = \pi \int_0^1a^nf_n(x)\sin(\pi x)\,dx.
\end{displaymath}
\end{enumerate}
\item
\begin{enumerate}
 \item Montrer qu'il existe un entier naturel $n_0$ tel que pour tout entier $n \geq n_0$, on ait \[\displaystyle \frac{a^n}{n!}<\frac{1}{2}. \]
 \item Montrer que $\displaystyle0\leq f_n(x)\leq \frac{1}{n!}$ pour tout $x\in[0,1]$.
 \item Que peut-on dire de $A_n$ lorsque $n\geq n_0$ ? \\ En déduire que $\pi^2$ est irrationnel.
 \item Comment peut-on en déduire que $\pi$ est irrationnel ?
\end{enumerate}
\end{enumerate}

\bigskip

\begin{it}Pour information.\newline
On sait depuis le 18ème siècle que $\zeta(p)$ est irrationnel pour $p$ pair ($p \geq 2$). On a démontré seulement en 1979 (Apéry) que $\zeta(3)$ est irrationnel. Pour les autres entiers impairs, la conjecture reste non démontrée.\end{it}