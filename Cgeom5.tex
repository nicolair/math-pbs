Exprimons les aires à l'aide de déterminants et utilisons 
\begin{displaymath}
 \overrightarrow{AG}+\overrightarrow{BG}+\overrightarrow{CG}=\overrightarrow{0}
\end{displaymath}
ainsi que les propriétés du déterminant (bilinéaire et antisymétrique):
\begin{multline*}
 \mathcal A(ABG) = \frac{1}{2}|\det(\overrightarrow{AB},\overrightarrow{AG})| 
= \frac{1}{2}|\det(\overrightarrow{AG}+\overrightarrow{GB},\overrightarrow{AG})|
= \frac{1}{2}|\det(\overrightarrow{GB},\overrightarrow{AG})| \\
= \frac{1}{2}|\det(\overrightarrow{CG},\overrightarrow{AG}+\overrightarrow{GC})|
= \frac{1}{2}|\det(\overrightarrow{CG},\overrightarrow{AC})|= \mathcal A(CAG)\\
= \frac{1}{2}|\det(\overrightarrow{GB},\overrightarrow{AG})|
= \frac{1}{2}|\det(\overrightarrow{GB},\overrightarrow{GB}+\overrightarrow{GC})|\\
= \frac{1}{2}|\det(\overrightarrow{GB},\overrightarrow{GC})|
= \frac{1}{2}|\det(\overrightarrow{GB},\overrightarrow{BG}+\overrightarrow{GC})|
= \frac{1}{2}|\det(\overrightarrow{BG},\overrightarrow{BC}|
= \mathcal A (BCG)
\end{multline*}
