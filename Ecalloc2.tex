%<dscrpt>Exercice sur un développement limité.</dscrpt>
\footnote{D'après Concours Commun Centrale Supelec 2000 PC épreuve 1}
Soit $m$ un entier supérieur ou égal à $1$. Tous les développements limités se font en $0$.
\begin{enumerate}
  \item Soit $\lambda$ un réel non nul. {\'E}crire le développement limité à l'ordre $m$ en 0 de la fonction
\begin{displaymath}
 x \rightarrow e^{\lambda x}
\end{displaymath}

  \item {\'E}crire le développement limité très simple à l'ordre $m$ en 0 de la fonction
\begin{displaymath}
x \rightarrow (e^ x -1 )^m 
\end{displaymath}

  \item Donner une autre expression du développement limité de la fonction
\begin{displaymath}
 x \rightarrow (e^ x -1 )^m
\end{displaymath}
En déduire, pour les entiers $j$ entre 1 et $m$, la valeur de
\begin{displaymath}
 \sum_{k=1}^m (-1)^{m-k}\binom{m}{k} k^j
\end{displaymath}
\end{enumerate}
