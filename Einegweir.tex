%<dscrpt>Encadrement des produits de Weierstrass</dscrpt>
Pour tous $n \in \N^*$ et $(a_1,\cdots,a_n)\in \left] 0, 1\right[^n$, on appelle \emph{produits de Weierstrass} les expressions 
\[
 P_n = \prod_{k=1}^{n}(1 + a_k), \hspace{0.5cm} M_n = \prod_{k=1}^{n}(1 - a_k). 
\]
On note aussi $S_n = a_1 + a_2 + \cdots + a_n$. \newline
L'objet de ce problème est de présenter des inégalités faisant intervenir ces objets ou des expressions analogues. Les parties sont indépendantes entre elles.
\subsection*{Partie I. Inégalités classiques.}
\begin{enumerate}
\item Encadrement de $M_n$.
\begin{enumerate}
 \item Montrer par récurrence que $1 - S_n \leq M_n$.
 \item Montrer par récurrence que $M_n \leq \frac{1}{1 + S_n}$.
\end{enumerate}
\item Encadrement de $P_n$.
\begin{enumerate}
 \item Montrer que $1 + S_n \leq P_n$.
 \item On suppose $S_n < 1$, montrer que $P_n \leq \frac{1}{1 - S_n}$.
\end{enumerate}
\item 
\begin{enumerate}
 \item Pour $(x_1,\cdots,x_n)\in \left]0, + \infty\right[^n$ et $(y_1,\cdots,y_n)\in \left]0, + \infty\right[^n$, montrer l'inégalité de Cauchy-Schwarz
\[
 \left( \sum_{i=1}^n x_i y_i\right)^2 \leq \left( \sum_{i=1}^n x_i^2\right)  \left( \sum_{i=1}^n y_i^2\right)
\]
en utilisant 
\[
 t \mapsto \sum_{i=1}^n (t x_i + y_i)^2.
\]
 \item Soit $(a_1,\cdots,a_n)\in \left] 0, 1\right[^n$ tel que $a_1+\cdots + a_n=1$. En remarquant que $a_i = (\sqrt{a_i})^2$, montrer que 
\[
 n^2 \leq \sum_{i=1}^n \frac{1}{a_i}
\]
en utilisant l'inégalité de Cauchy-Schwarz.
\end{enumerate}


\end{enumerate}

\subsection*{Partie II. Images et fonctions.}
Dans cette partie, on note $I = \left] 0,1 \right[$ et on considère
\[
 \mathcal{P} = \left\lbrace \frac{(1-x)(1-y)(1-z)}{xyz} \text{ tq } (x,y,z)\in I^3 \text{ et } x + y + z =1\right\rbrace.
\]

\begin{enumerate}
 \item En considérant un système d'équations aux inconnues réelles $u$, $v$, $w$, préciser des réels $a$, $b$, $c$ tels que 
\[
 \forall z \in I,\; \;
 \frac{(1+z)^2}{z(1-z)} = a + \frac{b}{z} + \frac{c}{1-z}.
\]
L'unicité du triplet $(a,b,c)$ n'est pas demandée. Dans votre rédaction, toute proposition induisant l'unicité dévalorisera votre copie. Soyez attentif au sens des implications que vous écrirez.
 \item Former le tableau de variations de la fonction $g$ de $I$ dans $\R$ définie par :
\[
 \forall z \in I,\; \; g(z) = \frac{(1+z)^2}{z(1-z)} .
\]
 \item Pour tout $z \in \left] 0,1 \right[$, on définit une fonction $f_z$ de $\left] 0, 1 -z \right[$ (noté $I_z$) dans $\R$ par 
\[
 \forall t \in I_z, \;f_z(t) = K_z\left( \frac{1}{t} - 1\right) \left( \frac{1}{1-z-t} -1\right) \text{ avec } K_z = \frac{1}{z}-1. 
\]
Former le tableau de variations de $f_z$.

 \item
 \begin{enumerate}
  \item Rappeler la définition de $f_z(I_z)$ avec des quantificateurs. 
  \item Montrer que 
\[
 \mathcal{P} = \bigcup_{z \in I}f_z(I_z) = \bigcup_{z \in I}\left[ \,g(z), + \infty\right[ = \left[8,+\infty \right[.
\]
 \item Montrer que 
\[
 \forall (x,y,z)\in I^3,\; x+y+z = 1 \Rightarrow 8xyz \leq (1-x)(1-y)(1-z).
\]
 \end{enumerate}
\end{enumerate}

\subsection*{Partie III. Inégalité de Ky Fan.}
Dans cette partie, on veut montrer l'\emph{inégalité de Ky Fan} pour tout $n\in N^*$:
\[
 \mathcal{F}_n:\hspace{1cm} \forall (a_1,\cdots, a_n)\in \left] 0, \frac{1}{2}\right]^n,\;
 \frac{\prod_{i=1}^n a_i}{\prod_{i=1}^n (1-a_i)} \leq 
 \left( \frac{\sum_{i=1}^n a_i}{\sum_{i=1}^n (1 - a_i)}\right)^n 
\]
\begin{enumerate}
 \item Preuve de $\mathcal{F}_2$.
 \begin{enumerate}
  \item Pour $a, b, a', b'$ réels, développer et factoriser $(a+b)^2a'b' - ab(a'+b')^2$.\newline 
Que devient cette relation si $a' = 1-a$ et $b' = 1-b$?
  \item Montrer $\mathcal{F}_2$.
 \end{enumerate}

  \item Soit $n\in \N^*$ avec $n\geq 2$.
  \begin{enumerate}
   \item Montrer que $\mathcal{F}_n \Rightarrow \mathcal{F}_{2n}$.
   \item Montrer que $\mathcal{F}_{n+1} \Rightarrow \mathcal{F}_{n}$. Pour $a_1, \cdots, a_n$ fixés, on pourra considérer $\frac{S_n}{n}$.
   \item Montrer $\mathcal{F}_n$.
  \end{enumerate}
 
 \item Dans cette question $(a_1, \cdots, a_n)\in \left] 0, + \infty\right[^n$. On note 
\[
 A = \left\lbrace \frac{1}{2a_i}, i \in \llbracket 1,n \rrbracket \right\rbrace. 
\]
\begin{enumerate}
 \item Soit $\lambda > 0$. Traduire la proposition
\[
 \forall i \in \llbracket 1,n \rrbracket, \; \lambda a_i \leq \frac{1}{2}
\]
 par une inégalité faisant intervenir $\max A$ ou $\min A$.
 \item Rappeler la définition de la moyenne arithmétique et de la moyenne géométrique de $a_1, \cdots, a_n$. En utilisant l'inégalité de Ky Fan (à l'exlusion de toute autre méthode), montrer une inégalité entre ces deux moyennes.
\end{enumerate}

   
\end{enumerate}

