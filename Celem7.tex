\subsubsection*{Exercice 1}
\begin{itemize}
\item Traitement de
\begin{displaymath}
S(x)=2\arctan \sqrt{\frac{1-x}{1+x}}+\arcsin x  
\end{displaymath}
Il est clair que le domaine est $]-1,1]$. On introduit $\theta=\arccos x $ dans le premier terme
\begin{displaymath}
\frac{1-x}{1+x}=\frac{2\cos^2\frac{\theta}{2}}{2\sin^2\frac{\theta}{2}}=\tan^2 \frac{\theta}{2} 
\end{displaymath}
Par définition de $\arccos$,
\begin{displaymath}
 \theta \in [0,\pi[ ,\hspace{0.5cm} \frac{\theta}{2}\in[0,\frac{\pi}{2}[ , \hspace{0.5cm}\tan \frac{\theta}{2}\geq 0
\end{displaymath}
 Donc $\arctan (\tan \frac{\theta}{2})=\frac{\theta}{2}$ d'où
$$S(x)=\arccos x+\arcsin x=\frac{\pi}{2}$$

\item Traitement de 
\begin{displaymath}
 S(x)=\arctan\frac{1}{2x^2}-\arctan\frac{x}{x+1}+\arctan\frac{x-1}{x}
\end{displaymath}
Remarquons que $S$ est défini dans $\R$ privé de -1 et 0. Notons
\begin{align*}
a=\arctan\frac{1}{2x^2}  & &
b=-\arctan\frac{x}{x+1}  & &
c=\arctan\frac{x-1}{x}
\end{align*}
et utilisons la formule 
$$\tan(a+b+c)=\frac{a+b+c-abc}{1-ab-ac+bc}$$
le numérateur est
$$ \frac{1}{2x^2}-\frac{x}{x+1}+\frac{x-1}{x}+\frac{x-1}{2x^2(x+1)}=0$$
La somme des trois arctan est donc un multiple de $\pi$ mais lequel?\newline
Cette valeur est constante à l'intérieur de chacun des trois intervalles formant le domaine de $S$. En considérant les limites à l'infini, on trouve

$S(x)=0-\frac{\pi}{4}+\frac{\pi}{4}=0$ dans $]-\infty,-1[$ et dans $]0,\infty[$.

Lorsque $x\in ]-1,0[$, une rapide étude de signes montre que $a,b,c$ sont dans $]0,\frac{\pi}{2}[$. La seule valeur possible pour $S(x)$ est donc $\pi$.
\end{itemize}
%
\subsubsection*{Exercice 2}
\begin{enumerate}
\item On trouve facilement $f(p-1)=p^2-p+1$
\item En utilisant la factorisation de $p^3-1$ et la question précédente, on obtient
\[\frac{ p^3-1}{ p^3+1}=\frac{(p-1)(p^2+p+1)}{(p+1)(p^2-p+1)}=\frac{(p-1)f(p)}{(p+1)f(p-1)}=\frac{p(p-1)f(p)}{p(p+1)f(p-1)}\]
On choisira donc \[u_{p}=\frac{p(p+1)}{p^2+p+1}\] Que pourrait -on choisir d'autre?
\item Chaque facteur s'exprimant comme un quotient de deux termes consécutifs d'une même suite, les simplifications s'enchaînent et conduisent à
\[\prod_{p=2}^{n}\frac{p^3-1}{p^3+1}=\frac{u_{1}}{u_{n}}=\frac{2(n^2+n+1)}{3n(n+1)}\]
\end{enumerate}
