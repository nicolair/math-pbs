\begin{enumerate}
 \item Les égalités sont vraies, peu importe le nom de l'indice de sommation.
 
 \item Implicitement, l'énoncé nous invite à écrire $T$ sous la forme
\begin{multline*}
T = \left(\sum_{i\in \llbracket 1,n\rrbracket}a_i x_i\right) \left( \sum_{j \in \llbracket 1,n\rrbracket} b_j y_j\right)  
 - \left(\sum_{i\in \llbracket 1,n\rrbracket}a_j y_j\right) \left( \sum_{j \in \llbracket 1,n\rrbracket} b_i x_i\right) \\
 = \sum_{(i,j)\in \llbracket 1,n \rrbracket^2}(a_ib_j - a_jb_i) x_i y_j
\end{multline*}
L'expression demandée est donc $t_{i,j} = a_ib_j - a_jb_i$. 

 \item On peut remarquer dans l'expression précédente que $t_{i,j} = 0$ si $i=j$ et que $t_{j,i} = -t_{i,j}$ si $i\neq j$.\newline
 \`A priori, le domaine de sommation de l'expression développée est le carré $\llbracket 1,n \rrbracket^2$. En fait, on peut enlever la diagonale ($i=j$) et rassembler les termes symétriques par rapport à cette diagonale en exploitant l'antisymétrie des $t_{i,j}$ que l'on vient de remarquer.\newline
 Le domaine se réduit donc au \og triangle\fg~ formé par les couples $(i,j)$ tels que $i<j$:
 \begin{multline*}
  T = \sum_{\substack{(i,j)\in \llbracket 1,n\rrbracket^2 \\ i < j}}(t_{i,j}x_i y_j + t_{j,i}x_j y_i)
  = \sum_{\substack{(i,j)\in \llbracket 1,n\rrbracket^2 \\ i < j}}t_{i,j}(x_i y_j - x_j y_i) \\
  = \sum_{\substack{(i,j)\in \llbracket 1,n\rrbracket^2 \\ i < j}} (a_ib_j - a_jb_i) (x_i y_j - x_j y_i) 
 \end{multline*}
  
 \item On se place dans $\R$ avec $x=b$ et $y=a$, alors:
\[
 A_X = B_Y = \sum_{i\in \llbracket 1,n \rrbracket}a_ib_i,\hspace{0.5cm} A_Y = \sum_{i \in \llbracket 1,n \rrbracket}a_i^2
 , \hspace{0.5cm} B_X = \sum_{i \in \llbracket 1,n \rrbracket}b_i^2.
\]
L'identité de Cauchy-Binet devient
\begin{multline*}
\left( \sum_{i\in \llbracket 1,n \rrbracket}a_ib_i\right)^2 
 - \left( \sum_{i \in \llbracket 1,n \rrbracket}a_i^2\right) \left( \sum_{i \in \llbracket 1,n \rrbracket}b_i^2\right)\\
 = \sum_{\substack{(i,j)\in \llbracket 1,n\rrbracket^2 \\ i < j}} (a_ib_j - a_jb_i) (b_i a_j - b_j a_i)
 = - \sum_{\substack{(i,j)\in \llbracket 1,n\rrbracket^2 \\ i < j}} (a_ib_j - a_jb_i)^2 \leq 0.
\end{multline*}
On en déduit 
\[
\left( \sum_{i\in \llbracket 1,n \rrbracket}a_ib_i\right)^2 
 \leq \left( \sum_{i \in \llbracket 1,n \rrbracket}a_i^2\right) \left( \sum_{i \in \llbracket 1,n \rrbracket}b_i^2\right). 
\]

puis l'inégalité de Cauchy-Schwarz en prenant les racines carrées. On remarque qu'il est indispensable d'être dans $\R$ pour que les carrés soient positifs.

 \item Cette fois on se place dans $\C$ et on choisit $x= \overline{b}$ et $y=\overline{a}$.
\[
 A_X = \sum_{i\in \llbracket 1,n \rrbracket}a_i\,\overline{b_i},\hspace{0.3cm}
 B_Y = \sum_{i\in \llbracket 1,n \rrbracket}b_i\,\overline{a_i} = \overline{A_X},\hspace{0.3cm}
 A_Y = \sum_{i\in \llbracket 1,n \rrbracket}|a_i|^2 ,\hspace{0.3cm}
 B_X = \sum_{i\in \llbracket 1,n \rrbracket}|b_i|^2.
\]
L'identité de Binet-Cauchy devient
\begin{multline*}
 \left|\sum_{i\in \llbracket 1,n \rrbracket}a_i\,\overline{b_i}\right|^2 
- \left( \sum_{i\in \llbracket 1,n \rrbracket}|a_i|^2 \right) \left(\sum_{i\in \llbracket 1,n \rrbracket}|b_i|^2 \right)\\
  = \sum_{\substack{(i,j)\in \llbracket 1,n\rrbracket^2 \\ i < j}} (a_ib_j - a_jb_i) \overline{(b_i a_j - b_j a_i)}
  = - \sum_{\substack{(i,j)\in \llbracket 1,n\rrbracket^2 \\ i < j}} |a_ib_j - a_jb_i|^2
\end{multline*}
qui conduit à l'identité demandée. On peut remarquer que cela donne une version complexe (avec des modules) de l'inégalité de Cauchy-Schwarz
\begin{multline*}
 \forall (a_1,\cdots, a_n)\in \C^n,\; \forall (b_1,\cdots, b_n)\in \C^n \\
 \left|\sum_{i\in \llbracket 1,n \rrbracket}a_i\,\overline{b_i}\right|^2 
\leq \left( \sum_{i\in \llbracket 1,n \rrbracket}|a_i|^2 \right) \left(\sum_{i\in \llbracket 1,n \rrbracket}|b_i|^2 \right). 
\end{multline*}
On se place encore dans $\C$ mais on choisit cette fois $b = \overline{a}$ et $y=\overline{x}$.
\[
 A_X = \sum_{i\in \llbracket 1,n \rrbracket}a_i\,x_i,\hspace{0.3cm}
 B_Y = \overline{A_X},\hspace{0.3cm}
 A_Y = \sum_{i\in \llbracket 1,n \rrbracket}a_i\,\overline{x_i},\hspace{0.3cm}
 B_X = \overline{A_Y}.
\]
L'identité de Binet-Cauchy devient
\begin{multline*}
 \left|\sum_{i\in \llbracket 1,n \rrbracket}a_i\,x_i\right|^2 
 - \left|\sum_{i\in \llbracket 1,n \rrbracket}a_i\, \overline{x_i}\right|^2 
 = \sum_{\substack{(i,j)\in \llbracket 1,n\rrbracket^2 \\ i < j}} 
 \underset{ = 2i \Im(a_i\overline{a_j})}{\underbrace{(a_i\overline{a_j} - a_j\overline{a_i})}} \,
 \underset{ = 2i \Im(x_i \overline{x_j})}{\underbrace{(x_i \overline{x_j} - x_j \overline{x_i})}}
 \\
 = - 4 \sum_{\substack{(i,j)\in \llbracket 1,n\rrbracket^2 \\ i < j}} \Im(a_i\overline{a_j})\Im(x_i \overline{x_j}).
\end{multline*}
Ceci conduit à la relation demandée. 

\end{enumerate}
