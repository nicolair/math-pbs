Cet ouvrage a été publié en 2014 chez l'éditeur "In Libro Veritas". Depuis cette date, les textes qui le composent ont continués à être mis à jour. La préface de l'ouvrage publié est reproduite au dessous.



La collection "MATH\'EMATIQUES EN MPSI" propose des documents pédagogiques (recueils de problèmes corrigés, livres de cours) en complément de ceux distribués en classe.\newline
Les ouvrages de la collection sont  disponibles sur internet. En fait, ils sont \emph{produits en ligne} à partir d'une base de données (le \emph{maquis documentaire}) accessible à l'adresse
\begin{center}
 \href{http://maquisdoc.net}{http://maquisdoc.net}
\end{center}
Cette base est conçue pour être très souple. Elle accompagne les auteurs et les utilisateurs en leur permettant de travailler librement et au jour le jour.

Il est devenu impossible de travailler sans internet (y compris pour rédiger des problèmes de mathématiques) mais il est également impossible de ne travailler que sur écran. Le papier garde donc toute sa validité et la publication de livres sous la forme imprimée habituelle (à coté d'autres types de services) est encore totalement justifiée.\newline
En revanche, le modèle économique de l'édition est devenu obsolète pour de tels ouvrages péri-scolaires produits à partir de structures web. L'éditeur (\emph{In Libro Veritas}) a accepté de diffuser cette collection sous licence Creative Commons. Les auteurs peuvent ainsi user plus libéralement de leur droit d'auteur et offrir davantage de liberté aux lecteurs.

\begin{center}
 \textbf{"\emph{Problèmes d'automne}"}
\end{center}
est un recueil de problèmes corrigés.\newline
Les énoncés portent sur le "programme de début d'année" qui nous est imposé par les textes officiels. Certains textes sont de simples exercices, en particulier ceux relatifs aux calculs usuels (complexes, trigonométriques) ou aux équations différentielles. D'autres, en particulier en géométrie, sont moins immédiats. Certains sont complexes mais aucun n'est abstrait, en cela ils ne sont pas tout à fait représentatifs de ce qui demandé le reste de l'année mais forment un socle indispensable pour les problèmes de géométrie des concours.\newline
Ces textes sont indiscutablement plus difficiles que ceux du type bac et leur abord peut dérouter. L'étudiant, surtout en début d'année, ne doit pas se condamner à trouver. La lecture de la solution, après un temps de recherche assez court, s'avèrera plus rentable qu'un acharnement infructueux. Il faut aussi ne pas se contenter de trouver mais s'obliger à repérer les tournures qui cristallisent les idées, à reproduire les présentations qui valorisent la copie. Il faut rédiger !


D'autres ouvrages de la collection proposent des textes portant sur l'ensemble du programme : très simples (\emph{Problèmes basiques}) ou moins immédiats (\emph{Problèmes d'approfondissement}).
 
