%<dscrpt>Espace de matrices, introduction au polynôme caractéristique.</dscrpt>
On note $E$ l'ensemble des matrices de $\mathcal M _3 (\R)$ de la forme 
\begin{displaymath}
 M(a_1,a_2,a_3,b) = 
\begin{pmatrix}
 a_1 & b & b \\
 b & a_2 & b \\
 b & b & a_3
\end{pmatrix}
\end{displaymath}

\subsection*{PARTIE I}
\begin{enumerate}
 \item Montrer que $E$ est un sous-espace vectoriel de $\mathcal M _3 (\R)$. Préciser sa dimension. Pour $a$, $a^\prime$, $b$, $b^\prime$ éléments de $\C$, calculer le produit matriciel
\begin{displaymath}
 M(a,a,a,b) M(a^\prime,a^\prime,a^\prime,b^\prime)
\end{displaymath}
\item Comment doit-on choisir les nombres complexes $a$, $a^\prime$, $b$, $b^\prime$ pour que 
\begin{displaymath}
 M(a,a,a,b) M(a^\prime,a^\prime,a^\prime,b^\prime) = I
\end{displaymath}
En déduire, lorsque $M(a,a,a,b)$ est inversible, une expression de la matrice inverse.
\item Résoudre dans $\C^3$ le système d'équations linéaires :
\begin{displaymath}
 \left\lbrace 
\begin{aligned}
 ax + y + z &= a^2 -3 \\
x + ay + z &= 2a -4 \\
x + y + az &=-2
\end{aligned}
\right. 
\end{displaymath}
où $a$ est un élément de $\C$.
\end{enumerate}


\subsection*{PARTIE II}
Dans cette partie, on pose 
\begin{displaymath}
 A = M(1,1,1,-1)
\end{displaymath}
\begin{enumerate}
 \item Montrer que pour tout entier naturel non nul $n$, il existe deux entiers naturels $u_n$ et $v_n$ tels que 
\begin{displaymath}
 A^n = u_n A + v_n I
\end{displaymath}
Préciser les relations de récurrence permettant d'exprimer $u_n$ et $v_n$ en fonction de $u_{n-1}$ et $v_{n-1}$ pour $n\geq 2$.
\item Déterminer $u_n$ et $v_n$ puis $A_n$ en fonction de $n$.
\item On pose $Q = 2I -A$. Calculer $Q^n$ pour $n$ entier naturel non nul et retrouver à partir de cette relation l'expression de $A_n$.
\end{enumerate}



\subsection*{PARTIE III}
Dans cette partie, on pose $B_{a}=M(1+a,1,1-a,-1)$ avec $a$ réel non nul.
\begin{enumerate}
\item Montrer que la fonction
\begin{eqnarray*}
	\R &\rightarrow& \R\\
	\lambda &\rightarrow& \det (B_{a}-\lambda I_{3})
\end{eqnarray*}
est polynomiale. On note $P_{a}$ le polynôme associé.
\item Calculer $P_{a}$. Montrer qu'il admet trois racines $\lambda_{1},\lambda_{2},\lambda_{3}$ vérifiant
$$\lambda_{1}<0< \lambda_{2}<2<\lambda_{3}$$

\item Soit $\lambda$ un nombre réel, montrer que $\lambda \in \{\lambda_{1},\lambda_{2},\lambda_{3}\}$ si et seulement si il existe une matrice colonne non nulle $X \in \mathcal{M}_{3,1}(\R)$ telle que $$B_{a}X=\lambda X$$

\item Soit $\lambda \in \{\lambda_{1},\lambda_{2},\lambda_{3}\}$, déterminer les $X$ tels que $B_{a}X=\lambda X$. Préciser la colonne $X$ dont la première ligne est $2-\lambda$
\end{enumerate}

