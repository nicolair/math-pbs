%<dscrpt>Points de franchissement pour une fonction continue sur un segment.</dscrpt>
\subsection*{Partie I. Pr{\'e}ambule}

Dans tout le probl{\`e}me, $u$ d{\'e}signera un nombre r{\'e}el.

On consid{\`e}re l'ensemble $\mathcal{C}_{u}$ des fonctions continues $f$ d{\'e}finies sur $[ 0,1] $ et {\`a} valeurs
r{\'e}elles telles que $f(0)\neq u$, $f(1)\neq u$ et $f$ n'est constante {\'e}gale {\`a} $u$ sur aucun intervalle ouvert de $[0,1] $.

Soit $f$ une fonction appartenant {\`a} $\mathcal{C}_{u}$.

\begin{itemize}
\item  On appelle point de \emph{franchissement vers le haut} du niveau $u$, un point $t_{0}$ de l'intervalle $] 0,1[ $ tel qu'il existe $\varepsilon >0$ tel que
\begin{eqnarray*}
\forall t \in ] t_{0}-\varepsilon ,t_{0}[ &:& f(t)<u \\
\forall t \in ] t_{0},t_{0}+\varepsilon [ &:& f(t)>u
\end{eqnarray*}

\item  On appelle point de \emph{franchissement vers le bas} du niveau $u$, un point $t_{0}$ de l'intervalle $] 0,1[ $ tel qu'il existe $\varepsilon >0$ tel que
\begin{eqnarray*}
\forall t \in ] t_{0}-\varepsilon ,t_{0}[ &:& f(t)>u \\
\forall t \in ] t_{0},t_{0}+\varepsilon [ &:& f(t)<u
\end{eqnarray*}

\item  On dit que $f$ admet un point de \emph{franchissement }du niveau $u$ en $t_{0}\in ] 0,1[ $ si et seulement si pour tout $\varepsilon >0$, il existe $t_{1}$ et $t_{2}$
dans $] t_{0}-\varepsilon,t_{0}+\varepsilon [ $ tels que
\[
(f(t_{1})-u)(f(t_{2})-u)<0.
\]
\end{itemize}

\begin{enumerate}
\item  Montrer qu'un point de franchissement vers le haut est un point de franchissement. Montrer qu'un point de franchissement vers le bas est un
point de franchissement.

\item  Montrer que si $t_{0}$ est un point de franchissement du niveau $u$ alors $f(t_{0})=u$.

\item  Exemples {\'e}l{\'e}mentaires. On suppose $u=0$. Indiquer si le point $t=\frac{1}{2}$ est un point de franchissement vers le haut, vers le bas ou
n'est pas un point de franchissement dans les cas suivants
\[
\begin{tabular}{ll}
a.\quad $f(x)=x-\frac{1}{2}$ & d.\quad $f(x)=(x-\frac{1}{2})^{2}$ \\
b.\quad $f(x)=(x-\frac{1}{2})^{3}$ & e.\quad $f(x)=| x-\frac{1}{2}| $ \\
c.\quad $f(x)=2(x-\frac{1}{2})+| x-\frac{1}{2}| $ &
\end{tabular}
\]

\item  Il existe des points de franchissement qui ne sont ni vers le haut ni vers le bas. Montrer que sur tout intervalle ouvert contenant un tel point, $f$ prend une infinit{\'e} de fois la valeur $u$. Construire une fonction pr{\'e}sentant un tel point de franchissement.

\item  On appelle point de \emph{tangence} du niveau $u,$ un point $t_{0}\in ] 0,1[ $ tel que $f(t_{0})=u$ et tel qu'en ce point $f$ n'admette
pas de point de franchissement de niveau $u$. Montrer qu'un point de tangence est un extr{\'e}mum local.

\item  Soit $t_{1}$ et $t_{2}$ deux points de $[ 0,1] $. Montrer que si $f(t_{1})>u$ et $f(t_{2})<u$, il y a au moins un point de
franchissement entre $t_{1}$ et $t_{2}$. Cette question n{\'e}cessite une r{\'e}daction particuli{\`e}rement rigoureuse.
\end{enumerate}

\subsection*{Partie II. Polygonation}

Soit $f\in \emph{C}_{u}.$ On suppose que pour tout entier $n$ non nul;
\[
\forall k\in \{ 0,\cdots ,2^{n}\} :f(\frac{k}{2^{n}})\neq u
\]
On consid{\`e}re alors la fonction $f_{n}$ appel{\'e}e polygonation d'ordre $n$ de $f$ d{\'e}finie par

\begin{itemize}
\item  $f_{n}$ est affine sur tout intervalle $[ \frac{k}{2^{n}},\frac{k+1}{2^{n}}] ,$ $k\in \{ 0,\cdots ,2^{n}-1\} $

\item  $\forall k\in \{ 0,\cdots ,2^{n}\} :f_{n}(\frac{k}{2^{n}} )=f(\frac{k}{2^{n}})$
\end{itemize}

\begin{enumerate}
\item  Montrer que $f_{n}\in \mathcal{C}_{u}$. On note $F_{u}$ et $F_{u}^{n}$ respectivement les nombres de points de franchissement de $u$ par la
fonction $f$ et par la fonction $f_{n}$. Ces nombres peuvent {\^e}tre infinis.

\item  Montrer que $F_{u}^{n}$ est fini et que $F_{u}^{n}\leq F_{u}$ lorsque $F_{u}$ est fini.

\item  Montrer que la suite $(F_{u}^{n})_{n\in \mathbf{N}}$ est croissante.

\item  On suppose que $F_{u}$ est fini.

\begin{enumerate}
\item  Montrer qu'il existe $F_{u}$ intervalles ouverts disjoints $I_{1},\cdots I_{F_{u}}$ qui contiennent chacun un point de franchissement du
niveau $u$ par $f$ et un seul.

\item  Montrer que pour tout $i\in \{ 1,\cdots ,F_{u}\} $, il existe deux intervalles ouverts $J_{i}$ et $K_{i}$ tels que
$J_{i}\subset I_{i},$ $K_{i}\subset I_{i}$, $f-u>0$ sur $J_{i}$, $f-u<0$ sur $K_{i}$.

\item  Montrer que $(F_{u}^{n})_{n\in \mathbf{N}}$ converge vers $F_{u}$.
\end{enumerate}

\item  On suppose $F_{u}$ infini. Montrer que $(F_{u}^{n})_{n\in \mathbf{N}}$ diverge vers $+\infty $.

\item Donner un exemple de fonction admettant un seul point de franchissement et qui n'est ni vers le haut ni vers le bas.\newline
 Montrer que lorsque $F_{u}$ est infini il existe un point de franchissement de $u$ qui n'est ni vers le haut ni vers le bas.
\end{enumerate}
