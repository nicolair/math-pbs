%<dscrpt>Autour des nombres de Catalan.</dscrpt>
Ce problème porte sur \emph{les nombres de Catalan} définis ici par récurrence par les relations
\begin{displaymath}
C_0=1,\hspace{1cm}\forall n\in \N,\; C_{n+1} = \sum_{i=0}^{n}C_i\,C_{n-i}  
\end{displaymath}
Cette notation $C_n$ sera valable dans tout le problème.

\subsection*{Partie I. Boîte à outils.}
\begin{enumerate}
  \item Calculer 
\begin{displaymath}
  C_1,\hspace{0.3cm} C_2,\hspace{0.3cm} C_3,\hspace{0.3cm} C_4,\hspace{0.3cm}
  H_2 =
  \begin{vmatrix}
    C_0 & C_1 \\ C_1 & C_2
  \end{vmatrix}
  \hspace{0.3cm}
  H_3 =
  \begin{vmatrix}
    C_0 & C_1 & C_2 \\ C_1 & C_2 & C_3 \\ C_2 & C_3 & C_4
  \end{vmatrix}
\end{displaymath}

\item On admet ici la formule de Stirling $n! \sim \sqrt{2\pi}\, n^n e^{-n} \sqrt{n}$.\\ En déduire une suite équivalente à $\left( \binom{2n}{n}\right)_{n\in \N} $.

\item Soit $p$ et $q$ réels strictement positifs tels que $p+q=1$.
\begin{enumerate}
  \item Montrer que $4pq\leq 1$. Dans quel cas a-t-on l'égalité ?
  \item Montrer que 
\begin{displaymath}
  1-\sqrt{1-4pq} = 2\min(p,q)
\end{displaymath}
\end{enumerate} 

\item Pour $k\in \N$, on pose $w_k = \int_{0}^{\frac{\pi}{2}}\cos^{2k}t\, dt$.
\begin{enumerate}
  \item Préciser $w_0$.
  \item Montrer que $2kw_k = (2k-1) w_{k-1}$ pour $k\geq 1$.
  \item Montrer que 
\begin{displaymath}
  w_k = \frac{\pi}{2}\, \binom{2k}{k}\, 4^{-k}
\end{displaymath}
\end{enumerate}

\end{enumerate}



\subsection*{Partie II. Marches sur $\N$.}
Soit $l\in \N$, une \emph{marche} sur $\N$ de longueur $l$ est une famille $(a_0,a_1,\cdots,a_l)$ de nombres naturels tels que
\begin{displaymath}
  a_0 = 0,\hspace{0.5cm}\forall i\in \llbracket 0, l-1 \rrbracket: |a_{i+1}-a_i|=1
  \text{ (on notera $\alpha_i = a_{i+1}-a_i$ )}
\end{displaymath}
Un \emph{circuit} de longueur $l$ est une marche $(a_0,a_1,\cdots,a_l)$ telle que $a_l=0$. Un circuit est dit \emph{strict} lorsque $a_k\neq 0$ pour tous les $k\in \llbracket 1,l-1\rrbracket$.\\
Pour $n\in \N$, on note $c_n$ le nombre de circuits de longueur $2n$ et $c'_n$ le nombre de circuits stricts de longueur $2n$.
\begin{enumerate}
  \item Soit $(a_0,a_1,\cdots,a_l)$ une marche sur $\N$. Montrer que 
\begin{displaymath}
  a_l = \text{Nombre de $\alpha_i$ égaux à $1$} 
  - \text{ Nombre de $\alpha_i$ égaux à $-1$}
\end{displaymath}
En déduire que la longueur d'un circuit est toujours paire.

\item Préciser $c_0$, $c_1$, $c'_1$, $c'_2$, . Montrer que $c'_n = c_{n-1}$ pour $n\geq 1$.

\item Soit $n\geq 1$ et $k\in \llbracket 1, n-1\rrbracket$. On considère les circuits $(a_0,a_1,\cdots,a_{2n})$ tels que
\begin{displaymath}
a_{2k}=0 \text{ et } a_i\neq 0 \text{ pour } 0< i < 2k  
\end{displaymath}
\begin{enumerate}
  \item Exprimer le nombre de ces circuits avec un $c$ et un $c'$ puis avec deux $c$.
  \item Exprimer $c_n$ comme une somme de $c$. En déduire que, pour tout $n\in \N^*$,
\begin{displaymath}
  c_n = C_n, \hspace{0.5cm} c'_n = C_{n-1}
\end{displaymath}
\end{enumerate}

\end{enumerate}


\subsection*{Partie III. Développements et série génératrice.}
Dans cette partie, on note $f$ la fonction définie dans $[0,1[$ par $f(x) =\sqrt{1-x}$. 
\begin{enumerate}

\item Soit $n$ naturel supérieur ou égal à 1, montrer que
\begin{displaymath}
\forall t\in [0,1[,\hspace{0.5cm}
f^{(n)}(t) = -\frac{(n-1)!}{2^{2n-1}}\binom{2n-2}{n-1}(1-t)^{\frac{1}{2}-n}
\end{displaymath}

\item Développement limité.
\begin{enumerate}
  \item Montrer que $f$ admet en $0$ le développement limité suivant à un ordre $n\geq 1$ 
\begin{multline*}
f(x)=\sqrt{1-x}=a_0+a_1x+\cdots+a_nx^n +o(x^n)\\
\text{ avec } a_0 = 1 \text{ et }
a_n = -\frac{2}{4^n n}\binom{2n-2}{n-1} \text{ pour } n\geq 1
\end{multline*}
\item Montrer que, pour $n\geq 2$ entier naturel,
\begin{displaymath}
  \sum_{k=0}^n a_ka_{n-k} = 0
\end{displaymath}
\end{enumerate}

\item Développement en série.
\begin{enumerate}
\item Rappeler sans démonstration la formule de Taylor avec reste intégral appliquée à $f$ entre $0$ et $x\in[0,1[$. Le reste intégral sera noté $r_n(x)$.
\item Montrer que pour $0\leq t \leq x <1$:
\begin{displaymath}
  0\leq \frac{x-t}{1-t} \leq x
\end{displaymath}
\item Montrer que, pour $x\in [0,1[$, la suite $\left( r_n(x)\right)_{n\in \N}$ converge vers $0$.
\item Montrer que, pour $x\in [0,1[$, la série $\sum a_nx^n$ est convergente avec
\begin{displaymath}
  \sum_{n\geq 0} a_nx^n = \sqrt{1-x}
\end{displaymath}
\end{enumerate}

\item On définit une fonction $\varphi$ dans $]0,\frac{1}{4}[$ par :
\begin{displaymath}
\forall y \in ]0,\frac{1}{4}[,\hspace{0.5cm}     \varphi(y) =  \frac{1-\sqrt{1-4y}}{2y}
\end{displaymath}
\begin{enumerate}
  \item Montrer que $\varphi$ se prolonge par continuité en $0$. On note encore $\varphi$ la fonction prolongée. Préciser la valeur $\varphi(0)$.
  \item Exprimer, pour $n\in \N$, des coefficients $c_n$ en fonction des $a_n$ pour que
  \begin{displaymath}
\forall y \in [0,\frac{1}{4}[,\hspace{0.5cm}    \varphi(y) = \sum_{n\geq 0} c_ny^n
  \end{displaymath}
\end{enumerate}

\item Montrer que, pour tout entier $n\geq 2$ et tout $y \in [0,\frac{1}{4}[$,
\begin{displaymath}
  c_n = C_n = \frac{1}{n+1}\binom{2n}{n},\hspace{0.5cm}
\frac{1-\sqrt{1-4y}}{2y} = \sum_{n\geq 0} C_n\,y^n
\end{displaymath}

\end{enumerate}



\subsection*{Partie IV. L'espoir du retour.}
Soit $l\in \N$ et $p$, $q$ des réels strictement positifs tels que $p+q=1$.\\
On note $\mathcal{M}_l$ l'ensemble des marches sur $\N$ de longueur $l$. On définit, pour $k$ entier entre $1$ et $l-1$ des fonctions $A_k$ et $D_k$:
\begin{displaymath}
  A_k((a_0,a_1,\cdots,a_l))= a_k,\hspace{0.5cm}
  D_k((a_0,a_1,\cdots,a_l))= a_{k+1} - a_k
\end{displaymath}
On probabilise l'ensemble $\mathcal{M}_l$ en définissant une fonction $p$ à l'aide de probabilités conditionnelles:
\begin{align*}
  &p_{|A_k \neq 0}(D_k = 1) = p & & p_{|A_k \neq 0}(D_k = -1) = q \\
  &p_{|A_k = 0}(D_k = 1) = 1 & & p_{|A_k = 0}(D_k = -1) = 0
\end{align*}
Pour $n\in \N^*$ tel que $2n<l$, on note $R_n$ l'événement \og la marche reprend la valeur $0$ pour la première fois en $2n$\fg.
\begin{enumerate}
  \item Calcul de $p(R_n)$.
\begin{enumerate}
\item Soit $(a_0,\cdots,a_{2n})$ un circuit strict de longueur $2n$. Préciser la probabilité de l'événement
\begin{displaymath}
(A_1 = a_1)\cap (A_2 = a_2)\cap \cdots \cap (A_{2n} = a_{2n})   
\end{displaymath}
\item Montrer que $p(R_n)= C_{n-1} p^{n-1}q^{n}$.
\end{enumerate}

\item Préciser une suite simple équivalente à $\left( p(R_n)\right)_{n\in \N^*}$. En déduire la convergence de la série $\sum_{n\geq1}p(R_n)$. Que représente la somme de cette série?

\item On suppose ici $p\neq \frac{1}{2}$ et $q\neq \frac{1}{2}$.
\begin{enumerate}
  \item En distinguant deux cas, calculer $\sum_{n\geq 1} p(R_n)$. (on trouvera deux expressions très simples)
  \item Montrer que la série $\sum np(R_n)$ est convergente. Que représente sa somme en termes probabilistes ? Montrer que 
\begin{displaymath}
\sum_{n\geq 1} np(R_n) = \frac{q}{Q-P}\text{ avec } P=\min(p,q), Q = \max(p,q)  
\end{displaymath}
\end{enumerate}

\end{enumerate}

\subsection*{Partie V. Matrices de Hankel.}
Dans cette partie, on définit une fonction $\rho$ dans $]0,4]$ par :
\begin{displaymath}
  \forall t \in ]0,4],\hspace{0.5cm} 
  \rho(t) = \frac{1}{2\pi} \frac{\sqrt{4t-t^2}}{t}
\end{displaymath}
\begin{enumerate}
  \item Soit $a\in ]0,4]$. Effectuer le changement de variable $t = 2 + 2\sin \theta$ avec $\theta \in [-\frac{\pi}{2},\frac{\pi}{2}]$ dans l'intégrale $\int_{a}^{4}\rho(t)\, dt$. En déduire la limite en $0$ de la fonction $a\mapsto \int_{a}^{4}\rho(t)\, dt$. On note $u_0$ cette limite.

  \item Pour $n\in \N^*$, on pose $u_n = \int_0^4 t^n\rho(t)\,dt$.
\begin{enumerate}
  \item Justifier l'existence de cette intégrale. Montrer que 
\begin{displaymath}
  u_n
= \frac{2^n}{\pi}\int_{-\frac{\pi}{2}}^{\frac{\pi}{2}}(1+\sin \theta)^{n-1}\cos^2 \theta \, d\theta
\end{displaymath}
 \item Montrer que
\begin{displaymath}
  u_n
= \frac{4^{n+1}}{\pi} \int_{0}^{\frac{\pi}{2}} \cos^{2n}u \,\sin^2 u \, du
\end{displaymath}
\item Montrer que $u_n=C_n = \frac{1}{n+1}\binom{2n}{n}$.
\end{enumerate}
  
\item Soit $l\in \N^*$, montrer que l'application
\begin{displaymath}
\left\lbrace 
\begin{aligned}
  &\R_l[X]^2 &\rightarrow &\R \\
  &(P,Q) &\mapsto &(P/Q) = \int_{0}^{4}P(x)Q(x)\rho(x)\,dx
\end{aligned}
\right. 
\end{displaymath}
est un produit scalaire sur $\R_l[X]$. \`A quoi sert dans cette preuve la question 1 de cette partie?

\item  On admet ici que la famille de polynômes $(P_0, P_1,\cdots P_l)$ avec
\begin{displaymath}
  \forall n \in \llbracket 0, l \rrbracket, \hspace{0.5cm}
P_n = \sum_{k=0}^n \binom{n+k}{n-k} (-X)^k  
\end{displaymath}
est une base orthonormée de $\R_l[X]$ pour le produit scalaire de la question précédente.
\begin{enumerate}
  \item Préciser la matrice de passage de $(1,X,X^2,X^3)$ vers $(P_0,P_1,P_2,P_3)$ (bases de $\R_3[X]$). On note $M$ cette matrice. Que représente $M^{-1}$? et $\trans M^{-1} M^{-1}$ ? 
  \item Soit $n$ entier naturel, montrer que 
\begin{displaymath}
\begin{vmatrix}
C_0    & C_1     & \cdots & C_n     \\
C_1    & C_2     & \cdots & C_{n+1} \\
\vdots &         &        & \vdots  \\
C_{n}  & C_{n+1} & \cdots & C_{2n}
\end{vmatrix}
=1
\end{displaymath}


\end{enumerate}


\end{enumerate}