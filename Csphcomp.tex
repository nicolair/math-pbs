\begin{enumerate}
 \item Le déterminant attaché au système linéaire $(S)$ est
\begin{displaymath}
 \begin{vmatrix}
  x   & x   & 1 \\
  y   & y   & i \\
  z-1 & z+1 & 0
 \end{vmatrix}=
 \begin{vmatrix}
  x   & 0   & 1 \\
  y   & 0   & i \\
  z-1 & 2 & 0
 \end{vmatrix}=
-2(ix+y)
\end{displaymath}
Lorsque $ix+y\neq 0$, le système admet un unique triplet solution que l'on calcule par les formules de Cramer :
\begin{align*}
&\begin{vmatrix}
 1 & x   & -1 \\
 i & y   & i  \\
 0 & z+1 & 0
\end{vmatrix}=
\begin{vmatrix}
 1 & x   & 0 \\
 i & y   & 2i  \\
 0 & z+1 & 0
\end{vmatrix} = -2i(z+1)
\\
&\begin{vmatrix}
 x   & 1   & -1 \\
 y   & i   & i  \\
 z-1 & 0   & 0
\end{vmatrix} =
\begin{vmatrix}
 x   & 1   & 0 \\
 y   & i   & 2i  \\
 z-1 & 0   & 0
\end{vmatrix} = 2i(z-1)
\\
&\begin{vmatrix}
 x   & x   & 1 \\
 y   & y   & i  \\
 z-1 & z+1 & 0
\end{vmatrix} = 
\begin{vmatrix}
 x   & 0 & 1 \\
 y   & 0 & i  \\
 z-1 & 2 & 0
\end{vmatrix} = -2(ix-y)
\end{align*}
L'unique triplet solution est donc
\begin{displaymath}
 \left(
\frac{z+1}{x-iy}, -\frac{z-1}{x-iy}, \frac{x+iy}{x-iy}\right) 
\end{displaymath}
Lorsque $ix+y=0$, on remplace $y$ par $-ix$. Le système devient :
\begin{displaymath}
 \left\lbrace 
\begin{aligned}
 x\alpha + x \beta - \gamma &= 1 \\
 x\alpha + x \beta - \gamma &= -1 \\
 (z-1)\alpha + (z+1)\beta &=0
\end{aligned}
\right. 
\end{displaymath}
Il est donc sans solution.

 \item Un triplet $(x,y,z)$ est dans $\Im \Phi$ si et seulement si il existe des complexes $u$ et $v$ tels que
\begin{displaymath}
 \left\lbrace 
\begin{aligned}
 x &= \frac{1+uv}{u+v}\\
 y &= i\,\frac{1-uv}{u+v}\\
 z &= \frac{u-v}{u+v}
\end{aligned}
\right. 
\Leftrightarrow
\left\lbrace 
\begin{aligned}
 xu + xv - uv &= 1\\
 yu + yv + iuv &= i \\
 (z-1)u + (z+1)v &= 0
\end{aligned}
\right. 
\end{displaymath}
Autrement dit $(x,y,z)\in \Im \Phi$ si et seulement si il existe un triplet solution de $(S)$ de la forme $(u,v,uv)$. Cela est vrai si et seulement si $ix+y\neq 0$ et
\begin{multline*}
 \left( \frac{z+1}{x-iy}\right)\left( -\frac{z-1}{x-iy}\right)
= \frac{x+iy}{x-iy}
\Leftrightarrow
1-z^2 = (x+iy)(x-iy)\\
\Leftrightarrow x^2+y^2+z^2 = 1
\end{multline*}
 La fonction $\Phi$ est donc une paramétrisation de la \og sphère complexe \fg. Il faut bien noter que $x$ et $y$ étant des nombres complexes, $x-iy$ \emph{n'est pas} le conjugué de $x+iy$.

\item On va démontrer que $\Phi(u,v)\in\R^3$ si et seulement si $u\overline{v}=1$.

Examinons d'abord la troisième composante.
\begin{multline*}
 \frac{u-v}{u+v}\in \R \Leftrightarrow (u-v)(\overline{u}-\overline{v})\in \R
\Leftrightarrow |u|^2 - |v|^2 -v\overline{u}+u\overline{v}\in \R \\
\Leftrightarrow -v\overline{u}+u\overline{v}\in \R 
\Leftrightarrow 2i \Im(u\overline{v})\in \R 
\Leftrightarrow u\overline{v} \in \R
\end{multline*}
Lorsque $v\neq 0$, cela revient à $\frac{u}{v}\in \R$, on pose alors $\lambda =\frac{u}{v}\in \R$ soit $u=\lambda v$. Examinons la première composante :
\begin{multline*}
 \frac{1+uv}{u+v}=\frac{1+\lambda v^2}{(1+\lambda)v}\in \R
\Leftrightarrow (1+\lambda v^2)\overline{v}\in \R 
\Leftrightarrow \overline{v} + \lambda |v|^2 v\in \R \\
\Leftrightarrow(-1+\lambda |v|^2)\beta =0 \text{ avec } \beta=\Im v
\end{multline*}
Si $\beta=0$ alors $v$ et $u$ sont réels.\newline
Réciproquement, lorsque $u$ et $v$ sont réels, les premières et troisièmes composantes sont réelles. Il est clair que $\Phi(u,v)$ est réel si et seulement si la deuxième composante est nulle soit $uv=1$.

Si $\lambda = \frac{1}{|v|^2}$ alors $u\overline{v}=1$ on remplace en utilisant $u=\frac{1}{\overline{v}}$ et
\begin{align*}
 \frac{1+uv}{u+v}=\frac{1+\frac{v}{\overline{v}}}{\frac{1}{\overline{v}}+v}
= \frac{\overline{v}+v}{1+|v|^2} = \frac{2\Re(v)}{1+|v|^2}\in \R \\
i\frac{1-uv}{u+v}=i\frac{1-\frac{v}{\overline{v}}}{\frac{1}{\overline{v}}+v}
= i\frac{\overline{v}-v}{1+|v|^2} = \frac{2\Im(v)}{1+|v|^2}\in \R 
\end{align*}

Si $v=0$ alors
\begin{displaymath}
 \Phi(u,0)=\left( \frac{1}{u},\frac{i}{u},1\right) 
\end{displaymath}
ce triplet ne peut pas être formé uniquement de nombres réels.\\
On a donc bien démontré le résultat annoncé.
\end{enumerate}
