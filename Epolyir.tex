%<dscrpt>Sur les polynômes irréductibles à coefficients rationnels.</dscrpt>
Dans tout ce problème $\Q[X]$ (respectivement $\Z[X]$) désigne l'ensemble des polynômes à une indéterminée, à coefficients dans $\Q$ (respectivement $\Z$). Ces ensembles sont des anneaux commutatifs pour les lois $+$ et $\times$ usuelles sur les polynômes.\\
Un polynôme $\Phi$ de $\Q[X]$ (respectivement $\Z[X]$) est dit irréductible sur $\Q$ (respectivement $\Z$) s'il n'est ni constant, ni de la forme $\Phi=PQ$ avec $P$, $Q$ dans $\Q[X]$ (respectivement $\Z[X]$) et $\deg(P)\se 1$, $\deg(Q)\se 1$. 

\subsection*{Partie 1. Exemples.}
\begin{enumerate}
\item Montrer que les polynômes $X^2-X-1$ et $X^3-X-1$ sont irréductibles sur $\Z$.
\item Soit $n\in\N^*$ et $a_1,\dots,a_n$ des entiers relatifs deux à deux distincts. Définissons $\Phi$ par
\begin{displaymath}
 \Phi=(X-a_1)\dots(X-a_n)-1
\end{displaymath}
On remarque que $\Phi \in\Z[X]$. Le but de la question est de montrer qu'il est irréductible sur $\Z$. Supposons qu'il existe $P$ et $Q$ dans $\Z[X]$ de degré supérieur ou égal à $1$ et vérifiant $\Phi=PQ$.
\begin{enumerate}
\item Montrer que $a_1,\dots,a_n$ sont des racines de $P+Q$.
\item En déduire que $\Phi=-P^2$.
\item Conclure. 
\end{enumerate}
\item Soit $n$ un entier naturel impair et soient $a_1,\dots,a_n$ des entiers relatifs deux à deux distincts. Montrer que $(X-a_1)\dots(X-a_n)+1$ est irréductible dans $\Z$.
\end{enumerate}

\subsection*{Partie 2. Lemme de Gauss.}
Soit $P=\sum_{k=0}^na_kX^k$ un polynôme non nul de $\Z[X]$. On définit le contenu de $P$, noté $c(P)$ par $$c(P)=\text{pgcd}(a_0,\dots, a_n)$$
Le polynôme $P$ est dit primitif si $c(P)=1$.\\
Soient $P$ et $Q$ dans $\Z[X]$.
\begin{enumerate}
\item On suppose dans cette question que $P$ et $Q$ sont primitifs.\\
Notons $P=\sum\limits_{k=0}^na_kX^k$, $Q=\sum\limits_{k=0}^mb_kX^k$ et $PQ=\sum\limits_{k=0}^rc_kX^k$. \\
Soit $p$ un nombre premier.
\begin{enumerate} 
\item Montrer qu'il existe un plus petit entier  $k\in\{0,\dots,n\}$ tel que $p$ ne divise pas $a_k$. Notons le $k_0$.\\
Notons, de même, $k_1$ le plus petit entier  $k\in\{0,\dots,m\}$ tel que $p$ ne divise pas $b_k$.
\item Montrer que $p$ ne divise pas $c_{k_0+k_1}$
\item En déduire que $c(PQ)=1$.
\end{enumerate}
\item  Montrer que $c(PQ)=c(P)c(Q)$ (lemme de Gauss).
\end{enumerate} 

\subsection*{Partie 3. Critère d'Eisenstein.}
\begin{enumerate}
\item Soit $\Phi\in\Z[X]$ pour lequel il existe des polynômes $P$ et $Q$ de degré supérieur ou égal à $1$ et à coefficients rationnels tels que $\Phi=PQ$. Montrer qu'il existe deux polynômes $P_0$ et $Q_0$ de $\Z[X]$ proportionnels respectivement à $P$ et $Q$ et tels que $\Phi=P_0Q_0$. 
\item Soit $\Phi\in\Z[X]$. Montrer que $\Phi$ est irréductible sur $\Q$ si et seulement si il est irréductible sur $\Z$.
\item Soit $\Phi=\sum\limits_{k=0}^na_kX^k$ un polynôme non constant de $\Z[X]$. On suppose qu'il existe un nombre premier $p$ tel que   : 
\begin{itemize}
\item[$\bullet$]$p$ ne divise pas $a_n$
\item[$\bullet$]$p$ divise $a_i$ pour $i\in\{0,\dots, n-1\}$
\item[$\bullet$]$p^2$ ne divise pas $a_0$
\end{itemize}
Montrer que $\Phi$ est irréductible sur $\Q$.
\end{enumerate}

