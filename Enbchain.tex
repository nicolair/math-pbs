%<dscrpt>Exercice sur la théorie des ensembles.</dscrpt>
Soit $E$ un ensemble quelconque et $p$ un entier naturel non nul,
on
d{\'e}finit une application $\Phi $%
\begin{eqnarray*}
\mathcal{P}(E)^{p} &\longrightarrow &\mathcal{F}(E,\left\{
0,\cdots
,p\right\} ) \\
(A_{1},A_{2},\cdots ,A_{p}) &\rightarrowtail &\varphi
_{A_{1},A_{2},\cdots ,A_{p}}
\end{eqnarray*}
en posant
\[
\forall x\in E,\quad \varphi _{A_{1},A_{2},\cdots
,A_{p}}(x)=Card\left\{ i\in \left\{ 1,\cdots ,p\right\} \text{
tels que }x\in A_{i}\right\}
\]

\begin{enumerate}
\item  Exemple. Dans cette question seulement, $E=\left\{ 1,2,3,4,5\right\} $%
, $p=3$, $A_{1}=\left\{ 1,2,3\right\} $, $A_{2}=\left\{ 3,4,5\right\} $, $%
A_{3}=\left\{ 2\right\} $. Pr{\'e}cisez la fonction $\varphi
_{A_{1},A_{2},A_{3}}$.

\item  On d{\'e}finit une partie $\mathcal{U}$ de $\mathcal{P}(E)^{p}$ en
posant :
\[
(A_{1},A_{2},\cdots ,A_{p})\in \mathcal{U}\Longleftrightarrow
A_{1}\subset A_{2}\subset \cdots \subset A_{p}=E
\]
Montrer que la restriction de $\Phi $ {\`a} $\mathcal{U}$ d{\'e}finit une
bijection de $\mathcal{U}$ dans $\mathcal{F}(E,\left\{ 1,\cdots ,p\right\} )$%
. En d{\'e}duire le nombre d'{\'e}l{\'e}ments de $\mathcal{U}$ lorsque $E$
contient $n$ {\'e}l{\'e}ments.
\end{enumerate}
