\subsection*{Exercice 1}
\begin{enumerate}
  \item Expressions de $\tan \frac{\pi}{8}$ et $\tan \frac{3\pi}{8}$ avec des racines carrées.
\begin{enumerate}
  \item Lorsque $x \not \equiv \frac{\pi}{2} \mod \pi$ et $x \not \equiv \pi \mod 2\pi$, on peut écrire
\[
  \tan x = \frac{2 \tan \frac{x}{2}}{1 - \tan^2 \frac{x}{2}}.
\]
 \item Introduisons $\epsilon = \pm 1$ pour rendre compte synthétiquement de $\tan \frac{\pi}{4} = 1$ et $\tan \frac{3\pi}{4} = -1$. En utilisant la formule précédente, considèrons l'équation d'inconnue $t$ réelle
\[
  \epsilon = \frac{2t}{1-t^2} \Leftrightarrow 1 = \epsilon \, \frac{2t}{1-t^2} \Leftrightarrow t^2 + 2\epsilon t - 1 = 0\;(E_\epsilon).
\]
Alors $\tan \frac{\pi}{8}$ est la solution positive de l'équation $(E_\epsilon)$ avec $\epsilon = 1$ et $\tan \frac{3\pi}{8}$ est la solution positive de l'équation avec $\epsilon = -1$.\newline
Les solutions de l' équation $(E_\epsilon)$ sont $-\epsilon \pm \sqrt{2}$. On en déduit
\[
  \tan \frac{\pi}{8} = -1 + \sqrt{2}, \hspace{0.5cm} \tan \frac{3\pi}{8} = 1 + \sqrt{2}.
\]
Les autres solutions sont
\[
  1 - \sqrt{2} = \tan \left( -\frac{\pi}{8}\right),\hspace{0.5cm}
  -1 - \sqrt{2} = \tan \left( -\frac{3\pi}{8}\right).
\]
\end{enumerate}
Comme les arguments sont entre $-\frac{\pi}{2}$ et $\frac{\pi}{2}$, on obtient le tableau 
\begin{center}
\renewcommand{\arraystretch}{1.5}
\begin{tabular}{|l|c|c|c|c|} \hline
solutions & $-1-\sqrt{2}$     & $1-\sqrt{2}$     & $-1+\sqrt{2}$   & $1 + \sqrt{2}$    \\ \hline
$\arctan$ & $-\frac{3\pi}{8}$ & $-\frac{\pi}{8}$ & $\frac{\pi}{8}$ & $\frac{3\pi}{8}$ \\ \hline
\end{tabular}
\end{center}

\item
\begin{enumerate}
  \item On suppose que $t$ n'est pas congru {\`a} $\frac{\pi }{8}$ modulo $\frac{\pi }{4}$ ni {\`a} $\frac{\pi }{2}$ modulo $\pi $ donc $\tan 4t$ et $\tan t$
sont bien d{\'e}finis. Avec la formule du binôme:
\begin{multline*}
\cos 4t = \Re (\cos t+i\sin t)^{4} = \cos ^{4}t-6\cos^{2}t\sin^{2}t+\sin ^{4}t \\
 = \cos ^{4}t\,\left(1-6\tan ^{2}t+\tan^{4}t\right),
\end{multline*}
\begin{multline*}
\sin 4t = \Im (\cos t+i\sin t)^{4} = 4\cos ^{3}t\sin t-4\cos t\sin^{3}t\\
 = \cos ^{4}t\,\left(4\tan t-4\tan ^{3}t\right).
\end{multline*}
On en d{\'e}duit
\[
\tan 4t=\frac{4\tan t-4\tan ^{3}t}{1-6\tan ^{2}t+\tan ^{4}t}.
\]

  \item On sait que $\cos 4t = 0 \Leftrightarrow t \equiv \frac{\pi}{8} \mod \frac{\pi}{4}$. Pour de tels $t$, $\cos t \neq 0$ et $\tan t$ est racine de $1-6x^2 + x^4=0$. On en déduit que 
\[
  -1-\sqrt{2} = \tan(-\frac{3\pi}{8}) < 1 - \sqrt{2} = \tan(-\frac{\pi}{8}) < - 1 + \sqrt{2} = \tan\frac{\pi}{8} <  1 + \sqrt{2} = \tan\frac{3\pi}{8} 
\]
sont quatre racines distinctes de cette équation. Ce sont les seules car l'équation est de degré 4. 
  
  \item Pour $x\notin \left\{ -1 - \sqrt{2}, 1 - \sqrt{2}, -1 + \sqrt{2}, 1+\sqrt{2}\right\}$, la fraction est définie. De plus, en notant $t=\arctan x$ ce qui entraine $x=\tan t$, la question a. montre que
\[
\frac{4x^{2}-4x^{3}}{1-6x^{2}+x^{4}} = \tan 4t.
\]
On ne peut pas en déduire que 
\[
\arctan \left(\frac{4x^{2}-4x^{3}}{1-6x^{2}+x^{4}}\right) = 4t = 4\arctan x
\]
car $4t$ n'est pas toujours dans $\left] -\frac{\pi}{2}, \frac{\pi}{2} \right[$ suivant les valeur de $t$. Il faut considérer divers cas et ajouter le bon multiple de $\pi$.

\begin{description}
  \item[.] Si $x<-1 - \sqrt{2}$ alors $t\in \left] -\frac{\pi}{2},-\frac{3\pi }{8}\right[$, 
  $4t\in \left] -2\pi ,-\frac{3\pi}{2}\right[$, $4t+2\pi \in \left] -\frac{\pi }{2},\frac{\pi}{2}\right[$ .
  \item[.] Si $-1 - \sqrt{2}<x< 1 - \sqrt{2}$ alors $t\in \left] -\frac{3\pi }{8},-\frac{\pi }{8}\right[$,
  $4t\in \left] -\frac{3\pi }{2},-\frac{\pi }{2}\right[ $, $4t+\pi \in \left] -\frac{\pi }{2}, \frac{\pi }{2}\right[$.
  \item[.]  Si $1-\sqrt{2}<x<-1+\sqrt{2}$ alors $t\in \left]-\frac{\pi }{8},\frac{\pi }{8}\right[$,
  $4t\in \left] - \frac{\pi}{2},-\frac{\pi }{2}\right[$.
  \item[.] Si $-1 + \sqrt{2 } < x <1+\sqrt{2}$ alors $t\in \left] \frac{\pi }{8},\frac{3\pi }{8}\right[$,
  $4t\in \left]\frac{\pi }{2},\frac{3\pi }{2}\right[ $, $4t-\pi \in \left] -\frac{\pi }{2},\frac{\pi }{2}\right[$.
  \item[.] Si $1+\sqrt{2} < x$ alors $t\in \left] \frac{3\pi }{8},\frac{\pi }{2}\right[$,
  $4t\in \left] \frac{3\pi }{2},2\pi \right[ $, $4t-2\pi \in \left] -\frac{\pi }{2},\frac{\pi }{2}\right[$.
\end{description}
Finalement :
\[
\arctan \frac{4x^{2}-4x^{3}}{1-6x^{2}+x^{4}} = 4\arctan x \;
\left\{
\begin{aligned}
&+2\pi  &\text{si} & & &x < -1 - \sqrt{2}  \\
&+\pi   &\text{si} & &-1 - \sqrt{2} < &x < 1 - \sqrt{2}  \\
&+0     &\text{si} & &1 - \sqrt{2} < &x < -1 + \sqrt{2}  \\
&-\pi   &\text{si} & &-1 + \sqrt{2} < &x < 1 + \sqrt{2} \\
&-2\pi  &\text{si} & &1 + \sqrt{2} < &x 
\end{aligned}
\right.
\]
\end{enumerate}
\end{enumerate}

\subsection*{Exercice 2}
\begin{enumerate}
\item  A l'aide de la formule de r{\'e}currence, on obtient
imm{\'e}diatement
\begin{eqnarray*}
P_{2}(t) &=&2t^{2}-1 \\
P_{3}(t) &=&2t(2t^{2}-1)-t=4t^{3}-3t \\
P_{4}(t) &=&2t(4t^{3}-3t)-(2t^{2}-1)=8t^{4}-8t^{2}+1 \\
Q_{2}(t) &=&2t \\
Q_{3}(t) &=&2t(2t)-1=4t^{2}-1 \\
Q_{4}(t) &=&2t(4t^{2}-1)-2t=8t^{3}-4t
\end{eqnarray*}

\item  On remarque que les formules sont vraies pour $n=0$ et $1$. En effet:
\[
\forall x \in \R, \; 
\left\lbrace
\begin{aligned}
  &P_0(\cos x) = 1 = \cos(0x), &P_1(\cos x) = \cos x = \cos(1x) \\
  &\sin x \,Q_0(\cos x) = 0 = \sin(0x), &\sin x \, Q_1(\sin x) = \sin x = \sin(1x) 
\end{aligned}
\right. .
\]
Pour tout $n\in \N$, considérons la propriété $\mathcal{C}_n$:
\[
  \forall x \in \R, \hspace{0.5cm}
  \left\lbrace
    \begin{aligned}
      &P_{n}(\cos x) = \cos (nx),& &\sin x\,Q_{n}(\cos x)=\sin (nx) \\
      &P_{n+1}(\cos x) = \cos ((n+1)x),& &\sin x\,Q_{n+1}(\cos x) = \sin ((n+1)x)
    \end{aligned}
  \right.
\]
Comme on a vu que $\mathcal{C}_0$ est vraie, il s'agit de montrer que $\left( \forall n \in \N, \; \mathcal{C}_n \Rightarrow \mathcal{C}_{n+1}\right)$. En fait, par définition de la proposition, il suffit de démontrer
\[
  \mathcal{C}_n \Rightarrow
  \left\lbrace
  \begin{aligned}
    P_{n+2}(\cos x)         &= \cos ((n+2)x) \\
    \sin x\,Q_{n+2}(\cos x) &= \sin ((n+2)x)
  \end{aligned}
  \right. .
\]
Utilisons les formules de transformation de somme en produit :
\[
\begin{aligned}
\cos ((n+2)x)+\cos nx &= 2\cos x\cos ((n+1)x) \\
\sin ((n+2)x)+\sin nx &= 2\cos x\sin ((n+1)x)
\end{aligned}
\]
d'o{\`u}
\[
\cos ((n+2)x)=2\cos xP_{n+1}(\cos x)-P_{n}(\cos x)=P_{n+2}(\cos x)
\]
d'apr{\`e}s la formule d{\'e}finissant $P_{n+2}$. De m{\^e}me,
\[
\sin ((n+2)x)=2\cos x\sin x\,Q_{n+1}(\cos x)-\sin x\,Q_{n}(\cos
x)=\sin xQ_{n+2}(x)
\]
d'apr{\`e}s la formule d{\'e}finissant $Q_{n+2}$.

\item  D{\'e}rivons l'égalité fonctionelle $P_n(\cos x) = cantberns (nx)$ de la pr{\'e}c{\'e}dente. Il vient
\[
-\sin xP_{n}^{\prime }(\cos x)=-n\sin nx.
\]
Ce qui s'{\'e}crit encore
\[
-\sin xP_{n}^{\prime }(\cos x)=-n\sin x\,Q_{n}(\cos x).
\]
Lorsque $x\in \left] 0,\pi \right[ $, $\sin x$ est non nul et on obtient $ P_{n}^{\prime }(\cos x)=n\,Q_{n}(\cos x)$. Pour tout r{\'e}el $t$ dans $\left] -1,1\right[ $, il existe un $x\in \left] 0,\pi \right[ $ tel que $t=\cos x$, on en d{\'e}duit la formule demand{\'e}e.
\end{enumerate}
