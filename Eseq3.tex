%<dscrpt>Suite définie par récurrence.</dscrpt>
Soit $u$ un r{\'e}el strictement positif, la suite $(u_n)_{n\in\N}$ est d{\'e}finie par les relations
\begin{displaymath}
u_0 =u, \hspace{1cm}
\forall n \in \N: \; u_{n+1}=\ln (1+u_n).
\end{displaymath}
Soit $\lambda $ un r{\'e}el non nul, la suite $(v_n)_{n\in \N}$ est d{\'e}finie par
\[
\forall n\in \N : \quad v_n=u_{n+1}^\lambda -u_n^\lambda.
\]

\begin{enumerate}
\item Former le tableau de variation de la fonction $x\rightarrow \ln (x+1) -x$.\newline
Soit $(x_n)_{n\in \N}$ une suite qui converge vers 0. Préciser (sans démonstration) des suites équivalentes pour $(e^{x_n}-1)_{n\in \N}$ et $(\ln (1+x_n)-x_n)_{n\in \N}$
\item  Soit $(w_{n})_{n\in \N}$ une suite de r{\'e}els qui converge vers un nombre $C$ non nul. Montrer que
\[
w_{1}+w_{2}+\cdots +w_{n}\sim n\,C .
\]
(r{\'e}diger la d{\'e}monstration)

\item  Les suites $(u_{n})_{n\in \N}$ et $(v_{n})_{n\in \N}$
sont-elles bien d{\'e}finies ?

\item  Montrer que $(u_{n})_{n\in \N}$ converge, pr{\'e}ciser sa limite.

\item  A-t-on $u_{n+1}\sim u_{n}$ ? Justifier.

\item  Montrer que
\[
v_{n}\sim -\frac{\lambda }{2}u_{n}^{\lambda +1}.
\]
On pourra utiliser que, pour $x$ au voisinage de $0$,
\begin{displaymath}
 \ln(1+x) = x -\frac{x^2}{2} +o(x^2).
\end{displaymath}

\item  En utilisant une valeur de $\lambda $ bien choisie, trouver un {\'e}quivalent simple de $u_{n}$.
\end{enumerate}
