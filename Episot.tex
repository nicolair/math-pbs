%<dscrpt>Autour des nombres de Pisot.</dscrpt>
Dans ce problème, lorsque $P$ est un polynôme, on notera simplement $P(a)$ le résultat de la substitution de $X$ par $a$ dans $P$.

Un \emph{nombre de Pisot} est un nombre réel strictement plus grand que $1$ et qui est racine d'un polynôme unitaire de degré au moins $1$, à coefficients dans $\Z$ et dont toutes les autres racines dans $\C$ sont de module strictement plus petit que $1$.\newline
On peut le reformuler ainsi: $\theta\in \R$ est un nombre de Pisot si et seulement si $\theta>1$ et s'il existe $P\in \Z[X]$ de coefficient dominant $1$ tel que 
\begin{displaymath}
 \forall u\in \C:
\left. 
\begin{aligned}
 u\neq \theta \\ P(u)=0
\end{aligned}
\right\rbrace \Rightarrow |u|<1
\end{displaymath}

\subsection*{Partie I. Exemples}
On note\footnote{Ce réel $\theta_0$ est aussi appelé \emph{nombre d'or}.} $\theta_0=\frac{1}{2}(1+\sqrt{5})$ et, pour tout $n\in \N^*$, 
\begin{displaymath}
 P_n = X^n(X^2-X-1)+X^2-1
\end{displaymath}

\begin{enumerate}
 \item En considérant $X^2-X-1$, montrer que $\theta_0$ est un nombre de Pisot.
 \item \'Etude de $P_1=X^3-X-1$.
\begin{enumerate}
 \item Montrer que $P_1$ a une unique racine réelle\footnote{Ce réel $\theta_1$ est appelé aussi nombre \emph{d'argent} ou nombre \emph{plastique}.}, notée $\theta_1$ et appartenant à $]1,\sqrt{2}[$ .
 \item Montrer que $\theta_1$ est un nombre de Pisot.
 \item Montrer que $\frac{\theta_1}{\theta_1-1}=1+\theta_1+\theta_1^2$.
\end{enumerate}
 \item \'Etude de $P_2=X^4-X^3-1$.
 \begin{enumerate}
  \item Former le tableau de variations de la fonction d'une variable réelle associée à $P_2$. En déduire que $P_2$ admet deux racines réelles notées $\alpha$ et $\theta_2$ avec $\alpha < \theta_2$. Où se placent-elles par rapport à $-1$, $0$, $1$ ?
  \item Montrer que $P_2(-\frac{1}{\theta_2})$ est strictement négatif.
  \item Montrer que $\theta_2$ est un nombre de Pisot.
 \end{enumerate}
\item 
\begin{enumerate}
 \item Pour tout $n\in\N^*$, montrer que $P_n$ admet une racine réelle dans $]1,\theta_0[$.\newline
 On \emph{admet} qu'elle est unique et que c'est un nombre de Pisot. On le note  $\theta_n$.
 \item Calculer puis factoriser le reste de la division de $P_{n+1}$ par $P_n$ en traitant à part les cas $n=1$ et $n=2$. En déduire que $\left( \theta_n\right) _{n\in \N^*}$ est strictement croissante.
 \item Montrer que $\left(\theta_n\right) _{n\in \N^*}$ converge vers $\theta_0$.
\end{enumerate}

\end{enumerate}

\subsection*{Partie II. Algorithme d'Euclide.}
Soit $y\neq 1$ un nombre complexe.
\begin{enumerate}
 \item  Calculer la suite, commençant par $X^3-X-1, X^2-y$, des polynômes obtenus par l'algorithme d'Euclide.
 \item 
\begin{enumerate}
 \item Caractériser de deux manières la propriété : $X^3-X-1$ et $X^2-y$ ont une racine en commun dans $\C$.
 \item En déduire que $\theta_1^2$ est un nombre de Pisot et préciser un polynôme unitaire de $\Z[X]$ dont il est la seule racine de module strictement plus grand que $1$.
\end{enumerate}
\item Montrer que $\theta_1^3$ est un nombre de Pisot et préciser le polynôme de $\Z[X]$ dont il est la racine de module strictement plus grand que $1$.
\end{enumerate}


\subsection*{Partie III. Puissances presque entières.}
Soit $a_1$, $a_2$, $a_3$ trois nombres complexes non nuls. On note $S_0=3$ et, pour tout $n$ non nul dans $\Z$, $S_n=a_1^n+a_2^n+a_3^n$.\newline
Dans cette partie, $a_1$, $a_2$, $a_3$ sont les trois racines complexes du polynôme  $P_1=X^3-X-1$ défini en I et le nombre de Pisot $\theta_1$ défini en I.2 et racine de $P_1$ sera désigné par $\theta$.
\begin{enumerate}
\item
\begin{enumerate}
 \item Calculer $S_1$, $S_2$, $S_{-1}$.
 \item Calculer $S_3$ et montrer que la suite $\left(S_n\right) _{n\in \N}$ vérifie une relation de récurrence à préciser.
\end{enumerate}

\item 
\begin{enumerate}
 \item Montrer que, pour tout $x$ réel, $|\sin x|\leq |x|$.
 \item Montrer que, pour tout $k$ et $n$ dans $\N$,
\begin{displaymath}
 \sin^2(\pi \theta^k)\leq \frac{4\pi^2}{\theta^k},\hspace{1cm}
\sum_{k=0}^n \sin^2(\pi \theta^k)\leq \frac{4\pi^2\theta}{\theta -1}
\end{displaymath}
\end{enumerate}

\item Soit $u>0$. Montrer que la suite $\left( \prod_{k=0}^n\cos(u\theta^{-k}\right) _{n\in \N}$ est convergente. Dans toute la fin de cette partie, sa limite est notée $\Gamma(u)$.

\item Soit $n$ naturel non nul et $s_1,s_n,\cdots,s_n$ dans $]0,1[$. Montrer que
\begin{displaymath}
 (1-s_1)(1-s_2)\cdots(1-s_n)\geq 1-\left(s_1+s_2+\cdots+s_n\right)  
\end{displaymath}

\item 
\begin{enumerate}
 \item Montrer que $\left( \prod_{k=0}^n\cos^2(\pi\theta^{-k}\right) _{n\in \N}$ converge vers un réel $A>0$.
 \item Montrer que $\left( \prod_{k=0}^n\cos^2(\pi\theta^{k}\right) _{n\in \N}$ converge vers un réel $B>0$.
 \item Montrer que $\left( \Gamma(\pi\theta^{m})^2\right) _{m\in \N}$ converge vers $AB$. En déduire que $\Gamma$ ne converge pas vers $0$ en $+\infty$.
\end{enumerate}

 \end{enumerate}
