%<dscrpt>Hyperplan d'endomorphismes stable par composition.</dscrpt>
Dans cet exercice, $E$ est un $\R$-espace vectoriel de dimension finie $n$ et $\mathcal{A}$ est un hyperplan de $\mathcal{L}(E)$. Les éléments de $\mathcal{A}$ sont donc des endomorphismes de $E$.\newline
On suppose de plus que $\mathcal{A}$ est \emph{stable} pour la composition des endomorphismes:
\begin{displaymath}
  \forall (a,a')\in \mathcal{A}^2,\; a\circ a' \in \mathcal{A}
\end{displaymath}
On se propose de démontrer par l'absurde que $\Id_E\in \mathcal{A}$. On suppose donc que $\Id_E\notin \mathcal{A}$.
\begin{enumerate}
  \item 
\begin{enumerate}
  \item Quelle est la dimension de $\mathcal{A}$? Montrer que $\mathcal{A}$ et $\Vect(\Id_E)$ sont supplémentaires dans $\mathcal{L}(E)$.
  \item Montrer que, pour tout endomorphisme $f\in \mathcal{L}(E)$, il existe un unique nombre réel (noté $p(f)$) tel que $f - p(f)\Id_E \in \mathcal{A}$.
  \item La question précédente définit une application $p$ de $\mathcal{L}(E)$ dans $\R$. Montrer qu'elle est linéaire et qu'elle vérifie
\begin{displaymath}
  \forall(f,g)\in \mathcal{L}(E)^2,\; p(f\circ g) = p(f) p(g)
\end{displaymath}
\end{enumerate}

  \item Montrer que, si $f\in \mathcal{L}(E)$ et $f^2 \in \mathcal{A}$, alors $f\in \mathcal{A}$.
  
  \item Soit $(e_1,e_2, \cdots, e_n)$ une base de $E$. Pour tout couple $(i,j)\in \llbracket 1,n \rrbracket$, on définit $f_{i,j}$ comme étant l'unique endomorphisme de $E$ vérifiant
\begin{displaymath}
\forall k \in \llbracket 1, n \rrbracket, \;
f_{i,j}(e_k)=
\left\lbrace 
\begin{aligned}
  &0   &\text{ si } k\neq j \\
  &e_i &\text{ si } k = j
\end{aligned}
\right. 
\end{displaymath}
\begin{enumerate}
  \item Quel est l'endomorphisme $f_{1,1} + f_{2,2} + \cdots + f_{n,n}$ ?
  \item Pour $i$, $j$, $k$, $l$ dans $\llbracket 1,n \rrbracket$, calculer $f_{i,j} \circ f_{k,l}$.
  \item Soit $i$ et $j$ dans $\llbracket 1,n \rrbracket$. Montrer que 
\begin{displaymath}
  i \neq j \Rightarrow f_{i,j} \in \mathcal{A}
\end{displaymath}
  \item En déduire que $f_{i,i}\in \mathcal{A}$ pour tous les $i$ entre $1$ et $n$. Conclure.
\end{enumerate}

\end{enumerate}
