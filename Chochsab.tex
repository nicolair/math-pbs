\begin{enumerate}
 \item Notons $q$ le cardinal de l'ensemble $X$, à chaque $y\in X$, on peut associer une fonction $f_y$ définie dans $X$ par :
\begin{displaymath}
 \forall x\in X : f_y(x)=
\left\lbrace
\begin{aligned}
 1 &\text{ si } x=y \\
 0 &\text{ si } x\neq y 
\end{aligned}
\right. 
\end{displaymath}
Notons
\begin{displaymath}
 X = \left\lbrace y_1,y_2,\cdots, y_q\right\rbrace 
\end{displaymath}
Alors tout élément $f\in\mathcal F(X,\C)$ s'écrit de manière unique sous la forme
\begin{displaymath}
 f = f(y_1)f_{y_1}+f(y_2)f_{y_2}+ \cdots + f(y_q)f_{y_q}
\end{displaymath}
Ce qui assure que
\begin{displaymath}
 \left( f_{y_1},f_{y_2}, \cdots , f_{y_q} \right) 
\end{displaymath}
est une base de $\mathcal F(X,\C)$ et donc que
\begin{displaymath}
 \dim \left( \mathcal F(X,\C) \right) = q
\end{displaymath}
On en déduit
\begin{displaymath}
 p\leq q
\end{displaymath}
car $p$ est la dimension de $V$ qui est un sous-espace vectoriel de $\mathcal F(X,\C)$.
\item \begin{enumerate}
 \item Remarquons d'abord que, d'après la question précédente,
\begin{displaymath}
\dim \left( \mathcal F(A,\C)\right) =p=\dim V 
\end{displaymath}
L'espace de départ et l'espace d'arrivée de la fonction restriction $R$ ont donc la même dimension.\newline
Pour montrer que $R$ est un isomorphisme, il suffit de montrer qu'elle est injective ou surjective. On va montrer qu'elle est injective.\newline
Soit $f\in \ker R$. C'est une fonction de $X$ dans $\C$ qui est nulle sur tous les éléments de $A$. On veut montrer que c'est la fonction nulle c'est à dire qu'elle est prend aussi la valeur $0$ pour les éléments de $X$ qui ne sont pas dans $A$.\newline
Considérons un tel élément $x\in X-A$ et l'élément $x^*$ qui lui est associé dans $V^*$. Comme $(x^*_1,\cdots, x^*_p)$ est une base de $V^*$, il existe $\lambda_1,\cdots,\lambda_p$ dans $\C$ tels que
\begin{displaymath}
 x^* = \lambda_1x_1^* + \cdots + \lambda_px_p^* 
\end{displaymath}
Prenons alors la valeur en $f\in V$ :
\begin{displaymath}
 x^*(f) = \lambda_1x_1^*(f) + \cdots + \lambda_px_p^*(f) 
\end{displaymath}
Ce qui s'écrit encore :
\begin{displaymath}
 f(x) = \lambda_1f(x_1) + \cdots + \lambda_pf(x_p)= 0 
\end{displaymath}
car $f$ est nulle sur $A$.
\item Considérons, comme en 1, les fonctions $f_1,\cdots,f_p$ définies dans $A$ par :
\begin{displaymath}
 \forall (i,j)\in \{1,\cdots,\}^2 : f_i(x_j)=\left\lbrace 
\begin{aligned}
 1 &\text{ si } i=j\\
 0 &\text{ si } i\neq j
\end{aligned}
\right. 
\end{displaymath}
La famille $(f_1,\cdots,f_p)$ est une base de $\mathcal F(A,\C)$, comme $R$ est un isomorphisme, il existe une base de $V$
\begin{displaymath}
 (v_1,\cdots,v_p)
\end{displaymath}
définie par
\begin{displaymath}
 \forall i\in\{1,\cdots,p\} : R(v_i)=f_i
\end{displaymath}
Ces relations traduisent exactement les conditions demandées
\begin{displaymath}
 \forall (i,j)\in\{1,2,\cdots \}^2 : v_i(x_j)=
\left\lbrace 
\begin{aligned}
 1 &\text{ si } i=j \\
 0 &\text{ si } i\neq j
\end{aligned}
\right. 
\end{displaymath}
\end{enumerate}
\item \begin{enumerate}
\item Comme $V$ est de dimension $p$ il contient des vecteurs non nuls c'est à dire des fonctions non identiquement nulles. Soit $v\in V$ l'une d'entre elles. Comme cette fonction n'est pas identiquement nulle, il existe $x\in X$ tel que 
\begin{displaymath}
f(x)\neq 0 \Leftrightarrow x^*(f)\neq 0 
\end{displaymath}
Ce qui siginifie que $x^*$ n'est pas le vecteur nul de $V^*$. La famille $(x^*)$ est donc libre.
\item Considérons une famille $(x_1,\cdots, x_q)$ vérifiant
\begin{align*}
 &(x^*_1,\cdots, x^*_q) \text{ libre }\\
 &\forall x\in X : (x^*_1,\cdots, x^*_q,x^*) \text{ liée }
\end{align*}
Il en existe d'après la question précédente.\newline
Comme $(x^*_1,\cdots, x^*_q)$ est une famille libre de $V^*$ qui est de dimension $p$ égale à celle de $V$ (résultat de cours), on a forcément $q\leq p$.\newline
D'autre part, les conditions entrainent aussi que, pour tout $x\in X$,
\begin{displaymath}
 x^* \in \Vect\left( x^*_1,\cdots, x^*_q\right) 
\end{displaymath}
Autrement dit:
\begin{displaymath}
 \forall x\in X, \exists (\lambda_1(x),\cdots,\lambda_q(x))\in\C^q \text{ tel que }
x^* = \lambda_1(x)x_1^* + \cdots + \lambda_q(x)x_q^*
\end{displaymath}
Ceci définit des \emph{fonctions} $(\lambda_1,\cdots,\lambda_q)$ de $X$ dans $\C$.\newline
Prenons alors la valeur en $v\in V$
\begin{align*}
 &\forall x\in X,\forall v\in V : &x^*(v) &= \lambda_1(x)x_1^*(v) + \cdots + \lambda_q(x)x_q^*(v)\\
&\forall x\in X,\forall v\in V : &v(x) &= \lambda_1(x)v(x_1) + \cdots + \lambda_q(x)v(x_q)\\
&\forall v\in V : &v &= v(x_1)\lambda_1 + \cdots + v(x_q)\lambda_q
\end{align*}
ce qui entraine
\begin{displaymath}
 v\in \Vect (\lambda_1, \cdots , \lambda_q) 
\end{displaymath}
pour tous les $v\in V$. On en déduit que $\dim V =p \leq q$ et donc que $p=q$. La famille libre $(x^*_1,\cdots, x^*_p)$ est alors une base de $V^*$.
\end{enumerate}
\end{enumerate}