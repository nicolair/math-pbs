Le corrigé de mat6 est incomplet. Il manque la rédaction des questions \og faciles\fg.
\subsection*{I. Convexité}
\begin{enumerate}
 \item
 \item 
 \item
 \item 
 \begin{enumerate}
  \item Montrons que $e_1$ est extrémal car le raisonnement sera le même pour $e_2$ et $e_3$.\newline
  Supposons $e_1 \in \left[ a, b\right]$ (c'est à dire combinaison convexe de $a$ et $b$) avec $a=(a_1,a_2,a_3)$ et $b=(b_1,b_2,b_3)$ dans $\mathcal{T}$ donc tous les $a_i$, $b_j$ entre 0 et 1. Il s'agit de montrer que $e_1$ est $a$ou $b$.\newline
  Chaque coordonnée de $e_1$ est combinaison convexe des coordonnées correspondantes de $a$ et $b$. Donc
\[
  1 \in \left[ a_1,b_1\right] \Rightarrow
1 = b_1  \Rightarrow b = e_1 \text{ car } b_1 + b_2 + b_3 = 1 \text{ avec les $b_i$ entre 0 et 1.}
\]
En fait on démontre aussi que $a=e_1$ avec $0 \in \left[ a_2,b_2\right]$ et $0 \in \left[ a_3,b_3\right]$.

  \item La condition $x_t \in \mathcal{T}$ se traduit par un système de 4 inégalités car la somme des coordonnées est toujours 1.
\begin{multline*}
\left. 
\begin{aligned}
 0 \leq x_1 + t \leq 1 \\ 0 \leq x_2 - t \leq 1
\end{aligned}
\right\rbrace 
\Leftrightarrow t \in \left[ \max(-x_1,x_2-1),\min(x_2,1-x_1)\right] \\
\Leftrightarrow t \in \left[ -\min(x_1,1 - x_2),\min(x_2,1-x_1)\right]
\end{multline*}
Posons $\rho = \min(x_1,1 - x_2,x_2,1-x_1) = \min(x_2, 1-x_1)$ avec les hypothèses.\newline
Alors $0 < t < \rho$ entraine $x_t$ et $x_{-t}$ dans $\mathcal{T}$ et différents de $x$. Or
\[
 x = \frac{1}{2}x_t + \frac{1}{2}x_{-t} \in \left] x_t, x_{-t}\right[ 
\]
ce qui signifie que $x$ n'est pas un point extrémal.

  \item Si $x$ n'est pas un des $e_i$, au moins une de ses coordonnées est dans $\left] 0,1\right[$ . Il doit en exister une autre car la somme vaut $1$. On se retrouve dans une situation analogue au a. ce qui entraine que $x$ n'est pas extrémal. Le seuls points extrémaux sont donc les $e_i$.
 \end{enumerate}

\end{enumerate}

\subsection*{II. Matrices bistochastiques}
\begin{enumerate}
 \item
 \begin{enumerate}
  \item 
  \item
 \end{enumerate}

 \item 
 \begin{enumerate}
  \item 
  \item
 \end{enumerate}

 \item Opérations sur les matrices bistochastiques.
  \begin{enumerate}
  \item 
  \item
  \item 
  \item
 \end{enumerate}

 
 \item On veut montrer qu'une matrice de permutation est un point extrémal de l'ensemble des matrices bistochastiques.\newline
Montrons d'abord que la matrice identité (qui est une matrice de permutation particulière) est extrémale. Supposons $I \in \left[ A,B \right]$ avec $A$ et $B$ bistochastiques et 
\[
 I = t\,A + (1-t)\,B \text{ avec } 0 \leq t \leq 1. 
\]
On veut montrer que $I$ est $A$ ou $B$. Logiquement, cela revient à montrer
\[
 B \neq I \Rightarrow A = I.
\]
Supposons $B \neq I$. Il existe $i$ tel que $b_{i i} < 1$. Pour simplifier, on supposera $b_{1 1} < 1$.\newline
Comme $1$ (terme $1 1$ de $I$) est entre $a_{1 1 }$ et $b_{1 1}$ et que $a_{1 1}$ est entre $0$ et $1$, on a  $a_{1 1} = 1$. 
La somme des termes de la première ligne ou de la première colonne de $A$ vaut 1, donc $a_{1 1} = 1$ est le seul terme non nul de cette ligne et de cette colonne.\newline
Il s'agit de montrer que $t=1$. Pour cela considérons le terme $1 1$ de l'égalité matricielle
\[
 \left. 
 \begin{aligned}
  t < 1 \\ b_{1 1} < 1
 \end{aligned}
\right\rbrace \Rightarrow 
t a_{1 1} + (1-t)b_{1 1} = t + (1-t)b_{1 1} < t < 1
\]
en contradiction avec le fait que ce terme est égal au terme $1 1$ de $I$ c'est à dire $1$. On en déduit $t=1$ et $A=I$.\newline
Il reste à montrer que n'importe quelle matrice de permutation $P_\sigma$ est extrémale.\newline
Soit $P_\sigma \in \left[ A, B\right]$ avec $A$ et $B$ extrémales. On peut tout multiplier par la matrice de permutation inverse $P_{\sigma^{-1}}$. Donc $I \in \left[ A', B'\right]$ avec $A' = P_{\sigma^{-1}}\,A$ et $B' = P_{\sigma^{-1}}\,B$. Les matrices $A'$ et $B'$ sont bistochastiques comme produits de matrices bistochastiques (II.3.b). Comme $I$ est extrémale, elle est égale à $A'$ ou $B'$. Par exemple
\[
 I = A' = P_{\sigma^{-1}}\,A \Rightarrow P_\sigma = A.
\]

 \item \'Etude de $\mathcal{E}$: matrices dont les sommes des termes par ligne et colonne sont nulles.
 \begin{enumerate}
  \item On vérifie les propriétés assurant que $\mathcal{E}$ est un sous-espace et que $\Phi$ est linéaire. 
  \item On veut montrer que $\Phi$ est injective. Soit $M \in \mathcal{E}$ dans $\ker \Phi$. Le bloc $(n-1)\times(n-1)$ en haut à gauche de $M$ est nul. Comme pour les $n-1$ premières lignes, la somme des termes est nulle, on obtient que les $n-1$ premiers termes de la colonne $n$ sont nuls. Le dernier terme de cette colonne est aussi nul car la somme est nulle. On raisonne de même avec les $n-1$ premières colonnes et on obtient que les $n-1$ premiers termes de la dernière ligne sont nuls. Ainsi $M$ est la matrice nulle, ce qui assure l'injectivité de $\Phi$.
  
  \item On veut montrer que $\Phi$ est surjective. Soit $A\in \mathcal{M}_{n-1}$ quelconque. On veut trouver $M\in E$ tel que $\Phi(M)=A$.\newline
Par définition, de $\Phi$, le bloc en haut à gauche de $M$ doit être $A$:
\[
 \forall (i,j) \in \llbracket 1, n-1\rrbracket^2, \; m_{i j} = a_{i j}.
\]
Il reste à définir $m_{1 n}, \cdots, m_{n-1 n}$, $m_{n 1}, \cdots, m_{n n-1}$ et $m_{n n}$.\newline
On définit les premières valeurs avec les premières lignes et colonnes
\[
 \forall k \in \llbracket 1,n-1 \rrbracket, \; m_{k n} = - \sum_{i=1}^{n-1}a_{k i}, \; m_{n k} = - \sum_{i=1}^{n-1}a_{i k}.
\]
On peut définir $m_{n n}$ avec la dernière ligne:
\[
 m_{n n} = - \sum_{i=1}^{n-1}a_{n i}.
\]
Mais il faut alors vérifier que la somme des termes de la dernière colonne est bien nulle.
\[
\sum_{k=1}^{n} m_{k n} = \sum_{k=1}^{n}\left( - \sum_{i=1}^{n-1}a_{k i}\right) 
= -\sum_{i=1}^{n-1}\underset{ = 0}{\underbrace{\left(\sum_{k=1}^{n} a_{k i}\right)}}  = 0
\Rightarrow M \in \mathcal{E}.
\]

  \item Comme $\Phi$ est un isomorphisme, il conserve la dimension donc $\dim (\mathcal{E}) = (n-1)^2$.
 \end{enumerate}
 
 \item 
  \begin{enumerate}
  \item Soit $B$ bistochastique avec $2n$ coefficients non nuls. Il existe donc $2n$ couples $(i,j)$ tels que $b_{i j}\neq 0$. Notons $V$ le sous-espace vectoriel engendré par les matrices élémentaires $E_{i j}$ attachées à ces couples. Par définition $\dim V = 2n$. Donc:
\[
 \dim \mathcal{E} + \dim V = (n-1)^2 + 2n = n^2 + 1 > n^2 = \dim \mathcal{M}_n(\R).
\]
On en déduit que $\mathcal{E} \cap  V$ ne se réduit pas à la matrice nulle. Notons $E$ une matrice non nulle dans cette intersection et posons $B_t = B + t E$. Les matrices $B$ et $B_t$ coincident pour tous les couples d'indices sauf sur les $2n$ couples $(i,j)$ pour lesquels $b_{i j} > 0$. 
Pour un tel couple, le terme $i,j$ de $B_t$ est $b_{i j} + t$. Il existe donc un intervalle ouvert contenant $0$ assez petit pour que tous ces termes soient strictement positifs. Comme $B\in \mathcal{E}$ les sommes des termes des lignes et des colonnes sont les mêmes pour $B$ et $B_t$ donc $B_t$ est bien bistochastique. On a alors
\[
 B = \frac{1}{2}B_t + \frac{1}{2}B_{-t} \Rightarrow B \in \left] B_t, B_{-t}\right[ 
\]
Ce qui entraine que $B$ n'est pas extrémale.
   
  \item Soit $B$ bistochastique avec au plus $2n -1$ coefficients non nuls. Il existe donc une colonne $j$ contenant au plus un coefficient $b_{i j} \neq 0$. Comme la somme des termes de la colonne $j$ est $1$, on a $b_{i j} = 1$. La somme des termes de la ligne $i$ vaut $1$ donc tous les autres $b_{i k}$ sont nuls.
 \end{enumerate}

 \item Pour montrer que toute matrice bistochastique extrémale est une matrice de permutation, on raisonne par récurrence sur la taille $n$ des matrices. On a vu que la propriété est vraie pour $n=2$.\newline
 Considérons une matrice $n\times n$ bistochastique extrémale. D'après 6.a. elle admet au plus $2n-1$ coefficients non nuls et d'après 6.b. il existe un couple $(i,j)$ tel que $b_{i j}=1$ soit le seule terme non nulde la ligne $i$ et de la colonne $j$.\newline
 Notons $B'$ la matrice extraite obtenue à partir de $B$ en supprimant la ligne $i$ et la colonne $j$. Par construction, elle est toujours bistochastique et extrémale mais de taille $(n-1)\times(n-1)$. D'après l'hypothèse de récurrence, $B'$ est une matrice de permutation. Donc $B'$ admet un seul terme égal à $1$ par ligne et par colonne. La matrice complète $B$ vérifie aussi cette propriété donc c'est une matrice de permutation.
\end{enumerate}

\subsection*{III. Majorisation}
\begin{enumerate}
 \item Par définition de $S_X$,
\begin{multline*}
 X \in S_Y \Leftrightarrow \exists \phi \in \mathfrak{S}_n \text{ tq } X = P_\phi Y
 \Leftrightarrow \exists \phi \in \mathfrak{S}_n \text{ tq } X = \underset{ = P_{\phi \circ \sigma^{-1}}}{\underbrace{P_\phi P_{\sigma^{-1}}}} P_\sigma Y\\
 \Leftrightarrow \exists \phi' \in \mathfrak{S}_n \text{ tq } X = P_{\phi'}  P_\sigma Y
 \Leftrightarrow X \in S_{ P_\sigma Y}.
\end{multline*}
On en déduit $S_Y = S_{ P_\sigma Y}$ donc $C_Y = C_{ P_\sigma Y}$.\newline
Si $X$ est une combinaison convexe de $P_\sigma Y$, alors $P_\theta X$ est aussi une combinaison convexe de permutés de $Y$ car pour chaque $\sigma$, $P_\theta P_\sigma Y = P_\phi Y$ avec $\phi = \theta \circ \sigma$. On en déduit 
\[
 X \in C_Y \Leftrightarrow P_\theta X \in C_Y = C_{ P_\sigma Y}.
\]

 \item Il s'agit simplement de la définition de $C_Y$ comme ensemble des combinaisons complexes des permutés de $Y$. Pour chaque permutation $\sigma$, $\pi(\sigma)$ est le coefficient du $P_\sigma Y$ dans la combinaison. 
Par linéarité, posons
\[
 B = \sum_{\sigma \in \mathfrak{S}_n}\pi(\sigma)P_\sigma.
\]
Quels sont les $\sigma$ qui contribuent au terme $b_{i j}$? Uniquement ceux tels que $\sigma(j) = i$. On en déduit 
\[
 b_{i j} = \sum_{\sigma \text{ tq } \sigma(j) = i}\pi(\sigma).
\]

 \item D'après la question 1., on ne change pas une relation $X \in C_Y$ en multipliant $X$ et $Y$ par des matrices de permutations. On peut permuter les coefficients de $X$ et de $Y$ pour les ranger par ordre décroissant.
 
 \item
 \begin{enumerate}
  \item On suppose $X = BY$ avec $B$ bistochastique. Pour tout $j\in \llbracket 1, n-1 \rrbracket$,
\[
 \sum_{k=1}^{j}x_k = \sum_{k=1}^{j}\sum_{i=1}^{n}b_{k i} y_i 
 = \sum_{i=1}^{n}\left( \sum_{k=1}^{j}b_{k i}\right)y_i 
 = \sum_{i=1}^{n} c_i y_i\; \text{ avec } c_i = \sum_{k=1}^{j}b_{k i}. 
\]
Avec les notations de l'énoncé,
\[
 \sum_{k=1}^{j}x_k = \sum_{k=1}^{n} c_k y_k\; \text{ avec } c_k = \sum_{i=1}^{j}b_{i k}.
\]
Les $c_k$ sont positifs car $B$ est positive et $c_k = \sum_{i=1}^{j}b_{i k} \leq \sum_{i=1}^{n}b_{i k} = 1$. De plus,
\[
 \sum_{k=1}^{n} c_k = \sum_{k=1}^{n} \sum_{i=1}^{j}b_{i k} = \sum_{i=1}^{j} \underset{ = 1}{\underbrace{\sum_{k=1}^{n}b_{i k}}} = j.
\]

  \item La relation que l'énoncé nous conseille de considérer est vraie car la parenthèse à droite du $y_j$ est nulle à cause la question précédente $\sum_{k=1}^{n} c_k = j$. Suivons le conseil.
\begin{multline*}
  \sum_{k=1}^{j}x_k - \sum_{k=1}^{j} y_k =
 \sum_{k=1}^nc_k y_k + y_j\left(-\left( \sum_{k=1}^{n}c_k\right)  + j \right) - \sum_{k=1}^jy_k \\
= \sum_{k=1}^nc_k (y_k - y_j) + \sum_{k=1}^j(y_j - y_k)
= \sum_{k=1}^j \underset{ \leq 0}{\underbrace{(c_k - 1)}} \, \underset{ \geq 0}{\underbrace{(y_k - y_j)}} + \sum_{k= j + 1}^nc_k \underset{\leq 0}{\underbrace{(y_k - y_j)}}
\leq 0\\
\Rightarrow \sum_{k=1}^{j}x_k   \leq \sum_{k=1}^{j} y_k .
\end{multline*}
Ceci étant valable pour tous les $j$, on a bien $X \prec Y$.

  \item On a vu en question 2 que $X \in C_Y$ entraine qu'il existe $B$ bistochastique telle que $X = B Y$. La question précédente entraine $X \prec Y$.

 \end{enumerate}

 \item
 \begin{enumerate}
  \item Si $X$ et $Y$ sont deux colonnes égales sauf pour un seul indice, ils ne peuvent avoir la même somme de termes. Ici, la somme des termes de $X$ et $Y$ sont égales à $1$ donc $X\neq Y$ entrainent qu'au moins deux des coefficients sont distincts donc $p \geq 2$.
  
  \item Par définition $i$ et $j$ sont le plus petit et le plus grand des $k$ tels que $a_k \neq b_k$ donc $i < j$ car il existe au moins deux de ces $k$.\newline
  Si $i=1$ alors $x_1 \leq y_1$ donc $x_i < y_i$. Si $i > 1$, on a
\[
\left. 
\begin{aligned}
 k < i &\Rightarrow x_k = y_k \\
 x_1 + \cdots + x_{i} &\leq y_1 + \cdots + y_i 
\end{aligned}
\right\rbrace 
\Rightarrow x_i \leq y_i \Rightarrow x_i < y_i\;\text{ car } x_i \neq y_i.
\]
De l'autre côté,
\begin{multline*}
\left. 
\begin{aligned}\
 x_1 + \cdots + x_{j-1} &\leq y_1 + \cdots + y_{j-1} \\
 x_1 + \cdots + x_n &= y_1 + \cdots + y_n =1
\end{aligned}
\right\rbrace 
\Rightarrow
y_j + \cdots + y_n \leq x_j + \cdots + x_n \\
\Rightarrow\;y_j \leq x_j \text{ car } k > j \Rightarrow x_k = y_k  
\Rightarrow y_j < x_j \text{ car } x_j \neq y_j.
\end{multline*}
En utilisant la décroissance des $x$ et des $y$, il vient
\[
 y_j < x_j \leq x_i < y_i.
\]

  \item Comme $x_i$ est entre $y_j$ et $y_i$, on peut l'écrire comme une combinaison convexe.
\[
 x_i = \lambda y_i + (1-\lambda)y_j
 \Leftrightarrow
 \lambda = \frac{x_i -y_j}{y_i -y_j} \in \left] 0,1 \right[ . 
\]
On pose $B_\lambda = \lambda I_n + (1-\lambda)P_{(i j)}$ et $Y' = B_\lambda Y$.\newline
Les coefficients de $Y'$ sont égaux à ceux de $Y$ sauf pour les indices $i$ et $j$:
\[
 y_i' = \lambda y_i + (1-\lambda)y_j,\; y_j' = \lambda y_j + (1-\lambda)y_i.
\]
Avec $y_i' = x_i$ par choix du $\lambda$ et
\[
 y_j' = y_i + \lambda (y_j - y_i) = y_i - x_i + y_j \Rightarrow x_i - y_i + y_j' - y_j = 0.
\]
Cette dernière relation servira plus loin
Examinons les sommes partielles en supposant $i <j$:
\[
 l < i \Rightarrow \sum_{k=1}^{l} y_k' = \sum_{k=1}^{l} y_k \geq \sum_{k=1}^{l} x_k.
\]
Pour la somme jusqu'à $i$ utilisons que $y_i' = x_i$:
\[
 \sum_{k=1}^{i} y_k' = \sum_{k=1}^{i-1} y_k + y_i' 
 = \sum_{k=1}^{i-1} y_k + x_i 
 \geq  \sum_{k=1}^{i-1} x_k + x_i = \sum_{k=1}^{i} x_k.
\]
Pour les sommes jusqu'à $j-1$ l'inégalité est valable car les nouveaux $y_k'$ sont égaux au $y_k$. Examinons la somme jusqu'à $j$.
\[
 \sum_{k=1}^{j} y_k' = \sum_{k=1}^{j}y_k + \underset{ = 0}{\underbrace{ x_i - y_i + y_j' - y_j}}
 = \sum_{k=1}^{j} y_k \geq \sum_{k=1}^{j} x_k.
\]
On a bien vérifié que $X\prec Y'$ le nombre de $k$ tels que $x_k \neq y'_k$ est strictement plus petit que $p$ à cause de $y_i'=x_i$.
 \end{enumerate}

 \item On peut proposer une démonstration algorithmique de l'implication $X \prec Y \Rightarrow X \in C_Y$ en utilisant la question précédente. Il existe des matrices bistochastiques $B_1, B_2, \cdots$ et des colonnes $Y_1 = B_1 Y$, $Y_2 = B_2 Y_1$ telles $X \prec Y_k$ tant que $X\neq Y_k$. Le nombre d'indices distincts décroit strictement donc il existe un $k$ tel que 
 \[
  X = Y_k = B_k\,B_{k-1}\, \cdots \, B_1\, Y 
 \]
avec $B = B_k\,B_{k-1}\, \cdots \, B_1$ bistochastique (produit de matrices bistochastiques) donc $X \in C_Y$.
\end{enumerate}
