%<dscrpt>Un exercice sur les produits vectoriels.</dscrpt>
Soient $a,b,c,d$ quatre vecteurs de $\R^3$ tels que $a$ et $b$ ne
soient pas colinéaires, $c$ est orthogonal à $a$ et $d$ est
orthogonal à $b$.\\
On cherche une condition nécessaire et suffisante pour qu'il existe
une solution au système, d'inconnue $v$ :
\begin{align*}
(*) &
\left\{
 \begin{aligned}
            a\wedge v  = & c \\
            b\wedge v  = & d
\end{aligned}
\right. 
\end{align*}
\begin{enumerate}
  \item Calculer de deux manières le déterminant de la famille  $(a,b,v)$. En déduire que pour qu'il existe une solution, il est  nécessaire que 
\begin{displaymath}
a.d+b.c=0 
\end{displaymath}
où $x.y$ désigne le produit scalaire des vecteurs $x$ et $y$.
  \item \begin{enumerate}
  \item
  Montrer que l'équation $a\wedge v=c$ possède au moins une  solution. Exprimer toutes les solutions de cette équation
  en fonction d'une solution $w$ et des données de l'exercice.
  \item Montrer que la condition $a.d+b.c=0$ est une condition suffisante pour qu'il existe une solution au système $(*)$.
          \end{enumerate}
\end{enumerate}
