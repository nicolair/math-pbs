%<dscrpt>Equations différentielles : recollement</dscrpt>
Soit $a$ un r{\'e}el fix{\'e} non nul et $I$ un intervalle de $\R$. On consid{\`e}re l'{\'e}quation différentielle
\begin{equation*}
(E_I)\hspace{0.5cm}  \forall x \in I:\hspace{0.25cm} y''(x)-4y(x)=\pi-4a|x|
\end{equation*}
où l'inconnue $y$ est une fonction définie et deux fois dérivable dans $I$.
\begin{enumerate}
  \item
\begin{enumerate}
 \item  D{\'e}terminer les ensembles de solutions de $(E_{]0,+\infty[})$ et de $(E_{]-\infty,0[})$. On les notera respectivement $\mathcal{S}_+$ et $\mathcal{S}_-$.
 \item Soit $v$ et $w$ deux nombres réels quelconques. Déterminer la solution $z_0$ dans $(E_{]-\infty,0[})$ telle que 
\begin{displaymath}
 \text{ en }0^-:\hspace{0.5cm} z_0 \rightarrow v \text{ et } z_0' \rightarrow w 
\end{displaymath}

\end{enumerate}
  \item On note $\mathcal{S}$ l'ensemble des solutions de $(E_\R)$ 
\begin{enumerate}
 \item Dans les théorèmes de cours sur les équations différentielles linéaires du second ordre à coefficients constants, quelle hypothèse doit vérifier le second membre ? Doit-il être continu ou dérivable ?\newline
 Que peut-on en conclure pour les éléments de  $\mathcal{S}_+$ ?
  \item La fonction définie dans $\R$ par
\begin{displaymath}
 x\mapsto -\frac{\pi}{4}+a|x|
\end{displaymath}
est-elle dans  $\mathcal{S}$ ?
\end{enumerate}

\item Déterminer  $\mathcal{S}$.
\end{enumerate}
