%<dscrpt>Coniques, enveloppes, cercles directeurs.</dscrpt>
Dans ce problème, la partie V est indépendante des autres.

Un plan est rapporté à un repère $\mathcal R = (A,\overrightarrow{i},\overrightarrow{j})$. On note
\begin{displaymath}
 \overrightarrow{e}_\theta = \cos \theta \overrightarrow{i}+ \sin \theta \overrightarrow{j}
\end{displaymath}
Soit $a$ et $c$ deux réels strictement positifs. On introduit, pour $\theta \in ]-\pi,\pi]$,  les éléments suivants:
\begin{itemize}
 \item le point $B$ de coordonnées $(2c,0)$
\item le point $C_\theta$  défini par :
\begin{displaymath}
 \overrightarrow{AC_\theta} = 2a \overrightarrow{e}_\theta 
\end{displaymath}
\item la médiatrice $\mathcal{D}_\theta$  de $[B,C_\theta]$.
\item le milieu $H_\theta$ de $[B,C_\theta]$.
\item le point d'intersection $M_\theta$ (lorsqu'il existe) des droites $\mathcal{D}_\theta$ et $(AC_\theta)$.
\item l'ensemble $\mathcal C$ des points $M_\theta$.
\item le nombre $\theta_0 = \arccos \frac{a}{c}$ lorsque $a<c$.
\end{itemize}
Ces notations sont valables dans tout le problème.
\subsubsection*{Partie I. Situation géométrique. Calculs complexes.}
Lorsque $M$ est un point quelconque du plan, on note $z(M)$ l'affixe de $M$ relative au repère $\mathcal R$.
\begin{enumerate}
 \item Faire une figure dans chacun des trois cas :
\begin{align*}
 0<c<a &,& 0<c=a &,& 0<a<c
\end{align*}
\item Quel est l'ensemble $\mathcal C$ lorsque $c=a$ ? Dans toute la suite du problème, on supposera $c\neq a$.
\item Calculer les affixes de $A$, $B$, $C_\theta$, $H_\theta$. Quel est l'ensemble des $H_\theta$ lorsqe $\theta$ décrit $]-\pi,\pi]$ ?
\item 
\begin{enumerate}
 \item Justifier l'existence d'un $\lambda \in \R$ tel que $z(M_\theta)=\lambda e^{i\theta}$.
 \item Montrer que
\begin{displaymath}
 (\lambda-a-ce^{-i\theta})(a-ce^{i\theta}) \in i \R
\end{displaymath}
\item Préciser les $\theta$ pour lesquels $M_\theta$ existe et calculer $z(M_\theta)$.
\end{enumerate}
\item Dans les cas $c<a$ puis $a<c$, former le tableau des signes de
\begin{displaymath}
 \frac{a^2-c^2}{a-c\cos \theta}
\end{displaymath}
pour $\theta \in ]-\pi,\pi]$
\end{enumerate}
\subsubsection*{Partie II. Définition bifocale.}
\begin{enumerate}
 \item Montrer que
\begin{displaymath}
 \left \Vert \overrightarrow{M_\theta C_\theta}\right\Vert = \left \Vert \overrightarrow{M_\theta B}\right\Vert 
\end{displaymath}
\item En discutant de la position relative des trois points alignés $A$, $M_\theta$ et $C_\theta$, montrer que $\mathcal C$ est une conique de foyers $A$ et $B$. On fera une figure dans chaque cas.
\end{enumerate}
\subsubsection*{Partie III. Enveloppes.}
Dans cette partie, $I$ est un intervalle de $\R$ et $U$,$V$,$W$ sont trois fonctions dérivables définies dans $I$ (à valeurs dans $\R$) telles que, pour tous les $t$ de $I$:
\begin{align*}
 U(t)V^\prime (t) - U^\prime (t)V(t)\neq 0 &,& (U^\prime (t),V^\prime(t))\neq (0,0)
\end{align*}
 On définit les droites $\Delta_t$ et $\Delta^\prime _t$ par leurs équations :
\begin{align*}
 \Delta_t &:& U(t)x+V(t)y+W(t)=0 \\
\Delta^\prime_t &:& U^\prime(t)x+V^\prime(t)y+W^\prime(t)=0 
\end{align*}
\begin{enumerate}
 \item Montrer que les droites $\Delta_t$ et $\Delta^\prime_t$ se coupent. On note $f(t)$ leur point d'intersection. La courbe paramétrée $f$ est appelée \emph{enveloppe} de la famille de droites. Le support de la courbe paramétrée $f$ est noté $\mathcal F$.
\item Montrer que $\Delta_t$ est tangente en $f(t)$ à $\mathcal F$.
\item Exemple. Soit $\mathcal C_1$ un cercle de centre $O$ (coordonnées $(0,0)$) et de rayon $R$ et $S$ un point de coordonnées $(s,0)$. On supposera $0<s<R$.
\begin{enumerate}
\item Soit $M$ un point de coordonnées $(R\cos t, R\sin t)$, former l'équation cartésienne de la droite (notée $\Delta_t$) perpendiculaire à $(SM)$ et passant par $M$.
\item Former la courbe paramétrée $f$ enveloppe de ces droites comme en 1.
\end{enumerate}
\end{enumerate}
\subsubsection*{Partie IV. Médiatrices.}
\begin{enumerate}
 \item Former l'équation de $\mathcal D _\theta$.
\item On note $\mathcal{D}^\prime _\theta$ l'équation obtenue à partie de $\mathcal D _\theta$ en dérivant chaque coefficient (comme en III). Montrer que $\mathcal D _\theta$ et $\mathcal D ^\prime _\theta$ se coupent en $M_\theta$. En déduire que $\mathcal D _\theta$ est tangente en $M_\theta$ à $\mathcal C$.
\item Calculer le produit des distances
\begin{displaymath}
 d(A,\mathcal D _\theta)d(B,\mathcal D _\theta)
\end{displaymath}
 des points $A$ et $B$ à la droite $\mathcal D _\theta$.
\item Quelle est la courbe formée par la projection d'un foyer sur les tangentes à une hyperbole ou une ellipse ? (Une telle courbe est appelée podaire)
\end{enumerate}
\subsubsection*{Partie V. Podaire d'un foyer sur une parabole.}
Soit $\mathcal P$ une parabole de foyer $F$ et de directrice $\mathcal D$. On choisit un repère dans lequel :
\begin{itemize}
 \item l'équation de $\mathcal D$ est
\begin{displaymath}
 x+\frac{p}{2}=0
\end{displaymath}
\item les coordonnées de $F$ sont $(\frac{p}{2},0)$
\end{itemize}
\begin{enumerate}
 \item \begin{enumerate}
 \item Former l'équation cartésienne de $\mathcal P$.
 \item Montrer que $\mathcal P$ est le support de la courbe paramétrée $u \rightarrow M_u$ définie dans $\R$ où $M_u$ est le point de coordonnées $(2pu^2,2pu)$.
\end{enumerate}
\item \'Ecrire une équation cartésienne de la tangente $\mathcal T_u$ en $M_u$ à $\mathcal P$.

On note dans toute la suite $K_u$ le projeté orthogonal de $M_u$ sur $\mathcal D$.

\item Calculer la distance de $F$ à $\mathcal T_u$, calculer la distance de $K_u$ à $\mathcal T_u$. Que peut-on en déduire ?
\item  Quel est l'ensemble des projetés orthogonaux de $F$ sur les tangentes à la parabole ?
\end{enumerate}

