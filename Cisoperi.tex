\subsection*{Partie I. Calculs préliminaires.}
\begin{enumerate}
 \item 
\begin{enumerate}
 \item Après réduction au même dénominateur, l'expression qu'on nous demande d'évaluer se met sous la forme
\begin{displaymath}
 \frac{\sin\left( (k+\varepsilon)\frac{\pi}{n}\right) }{\cos\left(\frac{\pi}{n} \right) }
\end{displaymath}
de plus, $\cos\left(\frac{\pi}{n} \right) >0$ et $k+\varepsilon \in \llbracket 0,n \rrbracket$ entraîne $\sin\left( (k+\varepsilon)\frac{\pi}{n}\right)\geq 0$.
 \item Formons le tableau de variations de la fonction $f$ définie dans $[0,\frac{\pi}{2}[$ par $f(x)= \tan x -x$. Elle est $\mathcal{C}^\infty$ dans son domaine avec $f'(x) = \tan^2x\geq 0$. Cette fonction est donc croissante donc $f(x)\geq 0$ pour tous les $x$ du domaine. Le cas particulier des $x$ de la forme $\frac{\pi}{n}$ conduit à l'inégalité demandée. 
\end{enumerate}

 \item Dans cette question $Z = (z_0,\cdots, z_{n-1})$ est un polygone régulier.
\begin{enumerate}
  \item Un polygone régulier direct est équilatéral car tous les côtés ont la même longueur.
\begin{displaymath}
 |z_{j+1}-z_j| = |a||\omega^{j+1}-\omega^j| = |a||\omega-1| = 2|a|\sin\left( \frac{\pi}{n}\right)   
\end{displaymath}
Le calcul est analogue et le résultat est le même dans le cas d'un polygone régulier indirect.
  \item Un polygone régulier est l'image par similitude d'un polygone régulier inscrit sur le cercle unité. Il est inscrit dans le cercle de centre $b$ et de rayon $|a|$.
  
  \item On obtient $b$ en sommant les affixes des points du polygone
\begin{displaymath}
\sum_{k=0}^{n-1}z_i = a\left( \sum_{k=0}^{n-1} \omega ^k \right) + n b = b  \text{ car } \sum_{k=0}^{n-1} \omega ^k = \frac{\omega^n -1}{\omega -1} = 0
\end{displaymath}
On en déduit
\begin{displaymath}
b = \frac{1}{n}\sum_{k=0}^{n-1}z_i \text{ et } z_0 = a  + b \Rightarrow a =z_0-b = \frac{1}{n}\left((n-1)z_0 -z_1 -\cdots -z_{n-1} \right)   
\end{displaymath}

\end{enumerate}
 
\item  Le polygone à quatre côtés proposé dans l'exemple est équilatéral car le calcul montre que tous les côtés ont la même longueur $\sqrt{5}$. On peut remarquer que c'est un losange (diagonales perpendiculaires).\newline
Si ce losange était un polygone régulier, son $b$ serait nul d'après la question précédente. Il apparait alors que les $|z_i -b|$ valent $1$ ou $2$ au lieu d'être égaux à un même $|a|$. Le losange de l'exemple est donc équilatéral mais non régulier. Les deux notions ne sont pas équivalentes: un polygone régulier est équilatéral mais un polygone équilatéral n'est pas forcément régulier. 

 \item La quantité qu'on nous demande d'évaluer est une somme de termes en progression géométrique. La raison est $\omega^p$ qui est toujours une racine $n$-ième de l'unité. Lorsque cette raison n'est pas $1$ la somme est donc nulle. On conclut donc:
\begin{displaymath}
 S(p)=\frac{1}{\sqrt{n}}\sum_{k=0}^{n-1}\left( \omega^p\right)^k=
\left\lbrace  
\begin{aligned}
 0 &\text{ si } p\not \equiv 0 \mod n \\
 \sqrt{n} &\text{ si } p \equiv 0 \mod n
\end{aligned}
\right. 
\end{displaymath}

 \item Transformons le membre de gauche de l'égalité à vérifier
\begin{multline*}
  \frac{1}{\sqrt{n}}\sum_{k=0}^{n-1}\left( \omega^j\right)^k \widehat{z}_k 
=   \frac{1}{n}\sum_{k=0}^{n-1}\left( \omega^j\right)^k
\left(
\sum_{l=0}^{n-1}\left( \bar{\omega}^k\right)^lz_l
\right)
= \frac{1}{n-1}\sum_{k=0}^{n} \sum_{l=0}^{n-1} \omega^{k(j-l)}z_l\\
= \frac{1}{n}\sum_{l=0}^{n-1} \sum_{k=0}^{n-1} \left( \omega^{j-l}\right)^kz_l
= \frac{1}{n}\sum_{l=0}^{n-1} S(j-l)z_l
=z_j
\end{multline*}
car la seule $S(j-l)$ non nulle correspond à $l=j$ et elle vaut $\sqrt{n}$.
\end{enumerate}

\subsection*{Partie II. Inégalité isopérimétrique.}
\begin{enumerate}
 \item Comme $\overline{z_{k+1}}\,\overline{\overline{z_k}} = \overline{z_{k+1}\overline{z_k}}$, on a $A(\overline{Z})=-A(Z)$.\newline
D'autre part:
\begin{displaymath}
(z_{k+1}+c)\overline{(z_k+c)}=
z_{k+1}\overline{z_k} + |c|^2 + z_{k+1}\overline{c} + c \overline{z_k} 
\end{displaymath}
donc
\begin{displaymath}
 A(Z+c)= A(Z) 
+ \Im\left(\overline{c}\sum_{k=0}^{n-1}z_{k+1}\right)
+ \Im\left(c\sum_{k=0}^{n-1}\overline{z_{k}}\right) 
\end{displaymath}
Or, comme $z_n=z_0$,
\begin{displaymath}
 \sum_{k=0}^{n-1}z_{k+1} = \sum_{k=0}^{n-1}z_{k}
\end{displaymath}
et les complexes dans les parties imaginaires sont conjugués. Les parties imaginaires s'annulent et $A(Z+c) = A(Z)$.
 \item Dans cette question, $Z$ est un polygone régulier.
\begin{enumerate}
 \item On a déjà vu que pour un polygone régulier $|z_{k+1}-z_k|= 2|a|\sin\left( \frac{\pi}{n}\right)$. On en tire 
\begin{align*}
 L(Z) = 2 n|a|\sin\left( \frac{\pi}{n}\right) & &
 E(Z) = 4 n|a|^2\sin^2\left( \frac{\pi}{n}\right)
\end{align*}
On sait d'après la question 1 de cette partie que le $b$ disparaitra du calcul de $A$ pour un polygone régulier. On a donc, dans le cas direct:
\begin{displaymath}
 A(Z) = 
\frac{|a|^2}{2}\Im\left( \sum_{k=0}^{n-1}\omega^{k+1 - k}\right)
= \frac{n}{2}|a|^2\Im(\omega)
= \frac{n}{2}|a|^2\sin\left( \frac{2\pi}{n}\right) 
\end{displaymath}
Dans le cas direct, le résultat est affecté d'un $-1$. 
 \item D'après les résultats de la question précédente, après simplification,
\begin{align*}
 \frac{|A(Z)|}{L(Z)^2}
= \frac{\sin\left(\frac{\pi}{n} \right) }{8n\sin^2\left(\frac{\pi}{n} \right)}
= \frac{ \cos\left(\frac{\pi}{n} \right)}{4n\sin\left(\frac{\pi}{n} \right)}
 & &
\frac{|A(Z)|}{E(Z)} = \frac{ \cos\left(\frac{\pi}{n} \right)}{4\sin\left(\frac{\pi}{n} \right)}
 & &
\frac{L(Z)^2}{E(Z)} = n
\end{align*}
\end{enumerate}

 \item On peut utiliser l'inégalité de Cauchy-Schwarz pour majorer $L(Z)$,
\begin{displaymath}
 L(Z)=
\sum_{k=0}^{n-1}1\times |z_{k+1}-z_k|
\leq \sqrt{\sum_{k=0}^{n-1}1^2}\sqrt{\sum_{k=0}^{n-1}|z_{k+1}-z_k|^2}
 = \sqrt{n}\sqrt{E(Z)}
\end{displaymath}
On en déduit l'inégalité demandée en élevant au carré.\newline
L'égalité se produit si les $|z_{k+1}-z_k|$ sont proportionnels à la suite constante $1$ c'est à dire égaux entre eux ou encore si et seulement si le polygone est équilatéral.
 \item 
\begin{enumerate}
 \item Notons $j$ la variable locale à la somme définissant $A$ pour utiliser plus commodément la question I.5. Comme toutes les sommes se font sur des indices entre $0$ et $n-1$, on se permet de ne pas écrire les ensembles sur lesquels on somme pour alléger un peu les formules
\begin{displaymath}
 z_{j+1}\overline{z_j} = \frac{1}{n}\sum_{k,l}
\left(\omega^{j+1} \right)^k\widehat{z}_k
\left(\omega^{-j} \right)^l\overline{\widehat{z}_l} 
=\frac{1}{n}\sum_{k,l}
\left(\omega^{k-l} \right)^j\omega^k \widehat{z}_k \overline{\widehat{z}_l} 
\end{displaymath}
En sommant sur $j$, on peut encore utiliser I.4.
\begin{displaymath}
\sum_{j} z_{j+1}\overline{z_j} 
= \frac{1}{n}\sum_{k,l}
\underset{n \text{ si } k=l\text{ sinon }0}{
  \left( 
    \sum_{j} \left(\omega^{k-l} \right)^j
  \right)
} 
\omega^k \widehat{z}_k \overline{\widehat{z}_l}
=\sum_{k} \omega^k \widehat{z}_k \overline{\widehat{z}_k}
\end{displaymath}
En prenant la partie imaginaire, on a bien:
\begin{displaymath}
 A(Z) = \frac{1}{2}\sum_{k=0}^{n-1}\sin\left(\frac{2k\pi}{n} \right)|\widehat{z}_k|^2 
\end{displaymath}
Pour $E$, on commence par examiner la somme des carrés des modules
\begin{displaymath}
 |z_j|^2 = \frac{1}{n}\sum_{k,l}
\left(\omega^{j} \right)^k\widehat{z}_k
\left(\omega^{-j} \right)^l\overline{\widehat{z}_l} 
= \frac{1}{n}\sum_{k,l}
\left(\omega^{k-l} \right)^j \widehat{z}_k \overline{\widehat{z}_l}
\end{displaymath}
On en déduit comme plus haut avec I.4.,
\begin{displaymath}
 \sum_j |z_j|^2 = \sum_k |\widehat{z}_k|^2
\end{displaymath}
Par circularité, cette somme est aussi celle de $1$ à $n$. On en déduit alors, en utilisant encore le travail fait sur $\sum_{j} z_{j+1}\overline{z_j}$:
\begin{multline*}
 E(Z)= \sum_j \left( |z_{j+1}|^2 + |z_{j}|^2 -2\Re\left(z_{j+1}\overline{z_j} \right) \right) \\
= 2 \sum_j |\widehat{z}_{j}|^2 
-2\Re\left(\sum_j  z_{j+1}\overline{z_j}\right) 
= 2 \sum_j |\widehat{z}_{j}|^2 
-2\sum_k \cos\left(\frac{2k\pi}{n} \right)|\widehat{z}_k|^2\\
= 2\sum_k\left(1-\cos\left( \frac{2k\pi}{n}\right)  \right)|\widehat{z}_{k}|^2 
= 4\sum_k \sin^2\left( \frac{k\pi}{n}\right)|\widehat{z}_{k}|^2
\end{multline*}

 \item La relation demandée est facile à prouver. Elle vient de la sommation des relations
\begin{multline*}
 \sin^2\left( \frac{k\pi}{n}\right) - \tan\left( \frac{\pi}{n}\right)\frac{1}{2}\sin\left( \frac{2k\pi}{n}\right)\\
= \sin^2\left( \frac{k\pi}{n}\right) 
- \tan\left( \frac{\pi}{n}\right)\sin\left( \frac{k\pi}{n}\right)\cos\left( \frac{k\pi}{n}\right)\\
= \sin\left( \frac{k\pi}{n}\right)
\left( 
\sin\left( \frac{k\pi}{n}\right) - \tan\left( \frac{\pi}{n}\right)\cos\left( \frac{k\pi}{n}\right)
\right) 
\end{multline*}

 \item Comme $0\leq \frac{k\pi}{n} \leq \pi$ pour les $k$ entre $0$ et $n$, on a $\sin\left( \frac{k\pi}{n}\right) \geq0$. Le résultat de la question I.1.a. avec $\varepsilon=1$, montre (avec la question précédente et $E(Z)>0$) que
\begin{displaymath}
 E(Z)-4\tan\left( \frac{\pi}{n}\right)A(Z)\geq 0
\Rightarrow 
\frac{A(Z)}{E(Z)}\leq \frac{1}{4\tan\left( \frac{\pi}{n}\right)}
\end{displaymath}
De même avec $\varepsilon =1$,
\begin{displaymath}
 E(Z)+4\tan\left( \frac{\pi}{n}\right)A(Z)\geq 0
\Rightarrow 
-\frac{A(Z)}{E(Z)}\leq \frac{1}{4\tan\left( \frac{\pi}{n}\right)}
\end{displaymath}
On en déduit la majoration avec la valeur absolue de $A(Z)$. 
\end{enumerate}

 \item
\begin{enumerate}
 \item \'Evaluons $A(Z_1)$ en notant par de simples $\cdots$ les termes de $A(Z_0)$ qui ne sont pas modifiés
\begin{multline*}
 A(Z_1)=
\frac{1}{2}\Im
\left( 
\left(z_j+\lambda(z_{j+1}-z_{j-1}) \right)\overline{z_{j-1}}
+ z_{j+1}\overline{\left(z_j+\lambda(z_{j+1}-z_{j-1}) \right)}
+\cdots 
\right) \\
=A(Z_0)
+  \frac{\lambda}{2}\Im
\left( 
(z_{j+1}-z_{j-1})\overline{z_{j-1}}\right)
+ z_{j+1}\overline{(z_{j+1}-z_{j-1})}\\
=A(Z_0)
+  \frac{\lambda}{2}\Im
\left( 
z_{j+1}\overline{z_{j-1}}
- z_{j+1}\overline{z_{j-1}}\right)
= A(Z_0)
\end{multline*}
  
 \item D'après la propriété fondamentale de $Z_0$ et la question précédente (conservation du $A$), on déduit que pour tous les $Z_1$ définis comme dans la question précédente avec $j$ et $\lambda$ arbitraires:
\begin{multline*}
 \left. 
\begin{aligned}
 A(Z_1) &= A(Z_0)\\
\frac{|A(Z_1)|}{L(Z_1)^2} &\leq \frac{|A(Z_0)|}{L(Z_0)^2}
\end{aligned}
\right\rbrace \Rightarrow
L(Z_0) \leq L(Z_1) \\
\Rightarrow
 |z_{j+1}-z_j|+|z_j-z_{j-1}| < |z_{j+1}-z'_j|+|z'_j-z_{j-1}|
\end{multline*}
Il reste à montrer que cela ne peut se produire que si $Z_0$ est équilatéral.\newline
Comme les longueurs se conservent par similitude, on peut supposer que $z_{j+1}=k$ et $z_{j-1}=-k$ avec $k$ réel. On note $a=\Re(z_j)$ et $b=\Im(z_j)$. On peut alors prendre $z'_j=z_j+t$ avec $t$ réel car $z_{j+1}-z_{j1}$ est réel. L'expression se simplifie
\begin{displaymath}
 f(t)= |z_{j+1}-z'_j|+|z'_j-z_{j-1}|
= \sqrt{(k-a-t)^2+b^2}+\sqrt{(k+a+t)^2+b^2}
\end{displaymath}
et on doit avoir $f(0)<f(t)$ pour tout $t \neq 0$. Ceci entraîne $f'(0)=0$.\newline
Or
\begin{displaymath}
 f'(0)=
-\frac{k-a}{|k-z|}+\frac{k+a}{|k+z|}
\end{displaymath}
 D'où
\begin{multline*}
 (k-a)|k+z|=(k+a)|k-z|
\Rightarrow
(k-a)^2\left((k+a)^2+b^2 \right) =  (k+a)^2\left((k-a)^2+b^2 \right)\\
\Rightarrow
(k-a)^2 = (k+a)^2
\Rightarrow a = 0 \\
\Rightarrow |z_{j+1}-z_j|= \sqrt{k^2 + b^2} = \sqrt{(-k)^2 + b^2} =|z_j-z_{j+1}|
\end{multline*}
Ceci étant valable pour tous les $j$, le polygone $Z_0$ est bien équilatéral.

 \item Comme le polygone $Z_0$ est équilatéral, il réalise un cas d'égalité dans la formule de Cauchy-Schwarz de la question II.3. On en déduit que $L(Z_0)^2=nE(Z_0)$. Utilisons alors l'inégalité de 4.c.
\begin{displaymath}
 \frac{|A(Z)|}{L(Z)^2} 
\leq \frac{|A(Z_0)|}{L(Z_0)^2}
= \frac{|A(Z_0)|}{nE(Z_0)}\leq \frac{1}{4n\tan\left(\frac{\pi}{n} \right) }
\leq \frac{1}{4\pi}
\end{displaymath}
d'après la deuxième partie de la question I.1.
\end{enumerate}
\end{enumerate}
