%<dscrpt>Propriétés du reste d'un développement fonctionnel.</dscrpt>
Ce problème porte sur des propriétés de $R_n(\varphi)$ défini par :
\begin{displaymath}
 \frac{\varphi}{2} =\sin \varphi - \frac{1}{2}\sin (2\varphi) + \frac{1}{3}\sin (3\varphi) +\cdots + \frac{(-1)^{n+1}}{n}\sin (n\varphi) + R_n(\varphi)
\end{displaymath}
où $n$ est un entier naturel non nul et $\varphi \in [-\frac{\pi}{2},\frac{\pi}{2}]$.

\subsection*{Partie I. Aspect euclidien.}
On définit un produit scalaire $(./ .)$ dans $E=\mathcal{C}(I,\R)$ avec $I=[-\pi,\pi]$ en posant :
\begin{displaymath}
 \forall (f,g)\in E^2:\; (f/ g) = \int_{-\pi}^{\pi}f(t)g(t)\,dt
\end{displaymath}
La démonstration qu'il s'agit bien d'un produit scalaire n'est pas demandée.\newline
On définit les fonctions $u$ et (pour tout entier naturel $n$) $c_n$, $s_n$ en posant:
\begin{displaymath}
 \forall t\in I:\; u(t) = \frac{t}{2}, \; c_n(t)=\cos(nt), \; s_n(t)=\sin(nt)
\end{displaymath}
\begin{enumerate}
 \item Calculer, pour tout couple $(m,n)$ d'entiers naturels
\begin{displaymath}
 (c_m / c_n), \; (c_n / s_m), \; (s_n / s_m)
\end{displaymath}
Soit $n$ naturel fixé, que peut-on conclure pour la famille formée par toutes les fonctions $c_k$ et $s_k$ pour  $k$ entre 0 et $n$ ?

\item Calculer $(u/c_0)$ et, pour tout entier $k$ non nul,
\begin{displaymath}
\frac{(u / c_k)}{\Vert c_k \Vert^2},\hspace{1cm} \frac{(u / s_k)}{\Vert s_k \Vert^2}, 
\end{displaymath}
Que peut-on en conclure pour la fonction $R_n$ ?
\end{enumerate}

\subsection*{Partie II. Reste intégral.}
Dans cette partie, $n\in \N^*$, $x$ est un réel strictement positif et 
\begin{displaymath}
 \varphi = \arctan \frac{1}{x} ,\quad y = x+ \sqrt{1+x^2}
\end{displaymath}
\begin{enumerate}
 \item \begin{enumerate}
 \item Montrer que
\begin{displaymath}
 \arctan^{(n)}(x)=(-1)^{n-1}(n-1)!\,(\sin n\varphi)(\sin \varphi)^n
\end{displaymath}
 \item Former (en le démontrant) le développement limité de $\arctan$ en $0$ à l'ordre $n$. En déduire $\arctan^{(n)}(0)$.
\end{enumerate}

\item \begin{enumerate}
 \item Exprimer $\arctan x$ en fonction de $\varphi$.
 \item Exprimer $\arctan y$ en fonction de $\varphi$.
 \item Exprimer $y- x$ en fonction de $\varphi$.
\end{enumerate}

\item Former le développement de Taylor avec reste intégral de la fonction $\arctan$ entre $x$ et $y$ à l'ordre $n$.
\item Montrer que
\begin{displaymath}
 R_n(\varphi)= \int_{x}^y \frac{(y-t)^n}{n!}\arctan^{(n+1)}t\, dt
\end{displaymath}
\item Effectuez le changement de variable $ \theta = \arctan \frac{1}{t}$ dans l'expression intégrale de $R_n$.

\end{enumerate}


\subsection*{Partie III. Reste de Lagrange et convergence.}
Les relations entre $x$, $y$ et $\varphi$ sont les mêmes que pour la partie II. On se propose de montrer que la suite $(R_n(\varphi))_{n\in \N}$ converge vers $0$.
\begin{enumerate}
 \item Montrer que pour tout $n$ naturel, il existe un $z_n(\varphi)\in [x,y]$ tel que
\begin{displaymath}
 R_n(\varphi) = \frac{(y-x)^{n+1}}{(n+1)!}\arctan^{(n+1)} z_n(\varphi)
\end{displaymath}

\item Montrer qu'il existe un $\theta_\varphi\in [\frac{\varphi}{2},\varphi]$ tel que
\begin{displaymath}
 R_n(\varphi) = \frac{(-1)^n}{(n+1)} \frac{(\sin(n+1)\theta_\varphi)(\sin \theta_\varphi)^{n+1}}{(\sin \varphi)^{n+1}}
\end{displaymath}

\item Montrer que la suite $(R_n(\varphi))_{n\in \N}$ converge vers $0$.
\end{enumerate}


