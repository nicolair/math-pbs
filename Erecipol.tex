%<dscrpt>Polynômes réciproques.</dscrpt>
On consid{\`e}re l'application $f$ de $\mathbf{C}^{*}$ dans $\mathbf{C}$ d{\'e}finie par $f(z)=z+\frac{1}{z}$.

\begin{enumerate}
\item
\begin{enumerate}
\item  Montrer que $f$ est surjective.

\item  Soit $\Gamma $ l'ensemble des nombres complexes de module 1. D{\'e}terminer l'image $\mathcal{F}$ de $\Gamma $ par $f$.

Montrer que l'image r{\'e}ciproque de $\mathcal{F}$ par $f$ est $\Gamma $.

\item  Soit $D=\left\{ z\in \Bbb{C}\:\mathrm{ tq }\:0<\left| z\right|<1\right\} $, montrer que l'application $g$ qui co{\"\i}ncide avec $f$ mais d{\'e}finie dans $D$ et {\`a} valeurs dans $\Bbb{C}-\mathcal{F}$ est une bijection.
\end{enumerate}

\item
\begin{enumerate}
\item  Montrer que, pour tout entier $n$ positif ou nul, il existe un unique polyn{\^o}me $P_{n}$ tel que :
$$\forall z \in \Bbb{C}^{*},\, f(z^{n})=\tilde{P}_{n}(f(z))$$ et que
$$\forall n \geq 1 ,\, P_{n+1}=XP_{n}-P_{n-1}$$
\item  Expliciter $P_{0}$, $P_{1}$, $P_{2}$, $P_{3}$.

\item  D{\'e}terminer le degr{\'e} de $P_{n}$ et {\'e}tudier la parit{\'e}
de $P_{n}$.
\end{enumerate}

\item  Soit $n$ et $k$ deux entiers naturels; calculer la d{\'e}riv{\'e}e de la fonction qui {\`a} $x\in \left] -2,+2\right[ $ associe $2\cos (n\arccos \frac{x}{2})$.

Pr{\'e}ciser la valeur de cette d{\'e}riv{\'e}e en $2\cos \frac{(2k+1)\pi }{2n}$ pour $k\in \left\{ 0,\ldots ,n-1\right\} $.

\item  Pour tout entier $n$ sup{\'e}rieur ou {\'e}gal {\`a} 1, on
consid{\`e}re l'{\'e}quation polynomiale dans $\Bbb{C}$ :
\begin{equation}
\tilde{P}_{n}(x)=0
\end{equation}

\begin{enumerate}
\item  R{\'e}soudre l'{\'e}quation (1). V{\'e}rifier qu'elle admet $n$
solutions distinctes. Class{\'e}es par ordre \emph{d{\'e}croissant}, elles seront not{\'e}es $x_{n,k}$ ; $k$ variant de 0 {\`a} $n-1$:
\[
x_{n,0}\geq x_{n,1}\geq \cdots \geq x_{n,k}\geq \cdots \geq x_{n,n-1}
\]

\item  On suppose $n\geq 2$, montrer que
\[
\forall k\in \left\{ 0,\ldots ,n-2\right\} ,\mathrm{ }x_{n,k}>x_{n-1,k}>x_{n,k+1}
\]
\end{enumerate}


Etant donn{\'e} un nombre complexe $b$, on consid{\`e}re, pour tout entier $n\geq 2$, l'{\'e}quation polynomiale dans $\Bbb{C}$
\begin{equation}
\tilde{P}_{n}(x)=b
\end{equation}

\item On suppose que $b$ n'appartient pas {\`a} l'intervalle r{\'e}el $\left[ -2,+2\right] $. Montrer que (2) admet $n$ racines distinctes.

\item On suppose que $b\in \left[ -2,+2\right] $ et on pose $\theta=\arccos \frac{b}{2}$.

\begin{enumerate}
\item  Exprimer les solutions de (2) en fonction de $\theta $.

\item  Pour quelles valeurs de $b$ l'{\'e}quation (2) admet-elle des racines doubles ?
\end{enumerate}
\end{enumerate}
