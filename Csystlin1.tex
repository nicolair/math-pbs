\begin{enumerate}
  \item Question de cours.
Le système est de Cramer lorsque le déterminant est  non nul:
\begin{displaymath}
  \begin{vmatrix}
A & B \\ C & D    
  \end{vmatrix}
=AD - BC \neq 0
\end{displaymath}
Les formules de Cramer donnent alors l'unique couple solution:
\begin{displaymath}
  (\frac{  \begin{vmatrix}
\lambda & B \\ \mu & D    
  \end{vmatrix}
}{  \begin{vmatrix}
A & B \\ C & D    
  \end{vmatrix}
},
\frac{  \begin{vmatrix}
A & \lambda \\ C & \mu    
  \end{vmatrix}
}{  \begin{vmatrix}
A & B \\ C & D    
  \end{vmatrix}
}
)
\end{displaymath}

  \item Pour le système $\mathcal{S}_0$, le déterminant est
\begin{displaymath}
  \frac{2}{a(a+1)}\left( \frac{1}{b(c+1)} - \frac{1}{c(b+1)}\right)
  = \frac{2(c-b)}{a(a+1)b(b+1)c(c+1)} \neq 0
\end{displaymath}
De plus
\begin{displaymath}
\renewcommand{\arraystretch}{1.8}
\begin{vmatrix}
1 & \frac{2}{(a+1)(b+1)} \\ 1& \frac{2}{(a+1)(c+1)}    
\end{vmatrix}
= \frac{2}{(a+1)}\left( \frac{1}{c+1}-\frac{1}{b+1}\right)
= \frac{2(b-c)}{(a+1)(b+1)(c+1)}
\Rightarrow
u = -abc
\end{displaymath}
\begin{displaymath}
\renewcommand{\arraystretch}{1.8}
\begin{vmatrix}
 \frac{1}{ab} & 1\\ \frac{1}{ac} & 1    
\end{vmatrix}
= \frac{1}{a}\left( \frac{1}{b}-\frac{1}{c}\right)
= \frac{c-b}{abc}
\Rightarrow
v = \frac{1}{2}(a+1)(b+1)(c+1)
\end{displaymath}
L'unique couple solution est
\begin{displaymath}
  (-abc, \frac{1}{2}(a+1)(b+1)(c+1))
\end{displaymath}

  \item On transforme $\mathcal{S}$ en un système équivalent par la méthode du pivot standard avec l'opération élémentaire codée par
\[
L_2 \longleftarrow L_2 -\frac{a-1}{b-1}L_1 .
\]
On obtient
\begin{multline*}
  \left(\frac{1}{b}-\frac{a-1}{b-1}\times\frac{1}{a} \right)y 
 +\left(\frac{1}{b+1}-\frac{a-1}{b-1}\times\frac{1}{a+1} \right)z
 = 1-\frac{a-1}{b-1} \\
 \Leftrightarrow
 \frac{b-a}{ab(b-1)}y + \frac{2(b-a)}{(a+1)(b-1)(b+1)}z = \frac{b-a}{b-1}
 \Leftrightarrow
\frac{1}{ab}\,y + \frac{2}{(a+1)(b+1)}\,z = 1 
\end{multline*}
Pour  $L_3 \leftarrow L_3-\frac{a-1}{c-1}L_1$, les calculs sont analogues en échangeant $b$ et $c$. On obtient
\begin{displaymath}
\frac{1}{ac}\,y + \frac{2}{(a+1)(c+1)}\,z = 1.
\end{displaymath}
Ces transformations montrent que si $(x,y,z)$ est solution de $\mathcal{S}$ alors $(y,z)$ est solution de $\mathcal{S}_0$. D'après la question 2
\begin{multline*}
\left. 
\begin{aligned}
y &= -abc \\ z &= \frac{1}{2}(a+1)(b+1)(c+1)  
\end{aligned}
\right\rbrace   
\Rightarrow
x = (a-1)\left(1+bc -\frac{1}{2}(b+1)(c+1) \right) \\
= \frac{1}{2}(a-1)(1+bc-b-c)=\frac{1}{2}(a-1)(b-1)(c-1)
\end{multline*}
L'unique triplet solution de $\mathcal{S}$ est donc
\begin{displaymath}
  \left( \frac{1}{2}(a-1)(b-1)(c-1), -abc, \frac{1}{2}(a+1)(b+1)(c+1)\right) 
\end{displaymath}


  
\end{enumerate}
