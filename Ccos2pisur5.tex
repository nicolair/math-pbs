\begin{enumerate}
  \item Comme $1$ n'est pas solution de l'équation $(1)$ et que
\begin{displaymath}
  (z-1)(1+z+z^2+z^3+z^4)=z^5 - 1.
\end{displaymath}
L'ensemble des solutions de $(1)$ est $\U_5 \setminus\{1\}$.\newline
L'équation $(2)$ est du second degré, de discriminant $5$, ses solutions sont
\begin{displaymath}
  \frac{1}{2}\left(-1-\sqrt{5} \right),\hspace{0.5cm} \frac{1}{2}\left(-1+\sqrt{5} \right).
\end{displaymath}

  \item Soit $u$ un nombre complexe non nul, alors:
\begin{multline*}
u+\frac{1}{u}\text{ solution de $(2)$ }
\Leftrightarrow
u^2+2+\frac{1}{u^2}+u+\frac{1}{u}-1=0 \\
\Leftrightarrow u^4 +u^2+ 1 + u^3 +u =0 \text{ (en multipliant par }u^2\neq 0) 
\Leftrightarrow u\text{ solution de $(1)$ }.
\end{multline*}

  \item D'après le cours sur les racines de l'unité, les solutions de $(1)$ s'écrivent aussi
  \begin{displaymath}
    \omega = e^{\frac{2i\pi}{5}},\; \omega^2= e^{\frac{4i\pi}{5}},\; \omega^3=\omega^{-2}=\overline{\omega^2},\; \omega^4=\omega^{-1}=\overline{\omega}.
  \end{displaymath}
Elles sont conjuguées deux par deux. Donc, lorsque $u$ décrit cet ensemble, $u+\frac{1}{u}$ prend seulement deux valeurs:
\begin{displaymath}
  2\cos \frac{2\pi}{5}\hspace{0.5cm} 2\cos \frac{4\pi}{5}.
\end{displaymath}

  \item D'après les questions précédentes, $2\cos \frac{2\pi}{5}$ est solution de $(2)$ donc
\begin{displaymath}
  2\cos \frac{2\pi}{5}=\frac{1}{2}\left(-1-\sqrt{5} \right)\hspace{0.5cm}\text{ou}\hspace{0.5cm} 
  2\cos \frac{2\pi}{5}=\frac{1}{2}\left(-1+\sqrt{5} \right).
\end{displaymath}
Or, $0 < \frac{2\pi}{5} < \frac{\pi}{2}$ donc les propriétés de $\cos$ entraînent que 
\begin{displaymath}
  \cos \frac{2\pi}{5}=\frac{1}{4}\left(-1+\sqrt{5} \right).
\end{displaymath}

\end{enumerate}
