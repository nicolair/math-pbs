%<dscrpt>Intégrales de 0 à l'infini.</dscrpt>
Dans cette partie\footnote{Extrait de ECRIN 1996} $p$ et $q$ sont des entiers tels que $0\leq q<p$.
\begin{enumerate}
\item  Montrer que la fonction 
\[t\rightarrow \frac{t^{2q}}{1+t^{2p}}\]
 est
int\'{e}grable dans $\left[ 0,+\infty \right[ $. On notera $I(p,q)$
l'int\'{e}grale 
\[\int_{0}^{+\infty }\frac{t^{2q}}{1+t^{2p}}dt\]
\item  D\'{e}terminer la d\'{e}composition en \'{e}l\'{e}ments simples dans $%
\mathbf{C}$ de 
\[
R=\frac{X^{2q}}{1+X^{2p}}
\]
On notera $\Omega $ l'ensemble des p\^{o}les.

\item  Pour $w\in \Omega $ fix\'{e}, pr\'{e}ciser une primitive de $%
t\rightarrow \frac{1}{w-t}$.

\item  Montrer que 
\[
I(p,q)=-\frac{i\pi }{4p}\sum_{w\in \Omega }\sigma (w)w^{2q+1}
\]
o\`{u} $\sigma (w)\in \left\{ -1,+1\right\} $ est le signe de la partie
imaginaire de $w$.

\item  On pose $z_{p}=e^{\frac{i\pi }{2p}}$.

\begin{enumerate}
\item  Exprimer l'ensemble $\Omega$  à l'aide des $z_p^{2k+1}$ pour
$0\leq k \leq p-1$.
\item  Exprimer $I(p,q)$ \`{a} l'aide des $z_{p}$ et montrer que 
\[
I(p,q)=\frac{\pi }{2p\sin \left( \frac{2q+1}{2p}\pi \right) } 
\]
\end{enumerate}

\item  Pour $\alpha =\frac{2p}{2q+1}$, calculer $\int_{0}^{+\infty }\frac{dt%
}{1+t^{\alpha }}$ apr\`{e}s avoir justifi\'{e} l'int\'{e}grabilit\'{e}.

\item  Formuler une propri\'{e}t\'{e} de la fonction définie dans $ ] 1,+\infty
[$
\[\alpha \rightarrow \int_{0}^{+\infty }\frac{dt}{1+t^{\alpha }}\]
(sans d\'{e}monstration) permettant de d\'{e}duire
facilement la valeur de 
\[\int_{0}^{+\infty }\frac{dt}{1+t^{\alpha }}\]
pour $\alpha $ r\'{e}el quelconque strictement plus grand que 1
\end{enumerate}

