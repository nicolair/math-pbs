\begin{enumerate}
\item  D{\'e}composons en {\'e}l{\'e}ments simples la fraction 
\begin{displaymath}
\frac{4X-3}{X(X-2)(X+1)} .
\end{displaymath}
Tous les pôles sont simples:
\begin{multline*}
 \widetilde{\frac{4X-3}{(X-2)(X+1)}}(0) = \frac{3}{2},\hspace{0.5cm}
 \widetilde{\frac{4X-3}{X(X+1)}}(2) = \frac{5}{6}, \hspace{0.5cm}
 \widetilde{\frac{4X-3}{X(X-2)}}(-1) = -\frac{7}{3} \\
 \Rightarrow
\frac{4X-3}{X(X-2)(X+2)}=\frac{1}{6}\left( \frac{9}{X} + \frac{5}{X-2} - \frac{14}{X+1} \right)
\end{multline*}
\item On en d{\'e}duit
\begin{multline*}
\sum_{k=3}^{n}\frac{4k-3}{k(k-2)(k+1)} 
= \frac{1}{6}\left( 9\sum_{k=3}^{n}\frac{1}{k} + 5\sum_{k=3}^{n}\frac{1}{k-2} - 14\sum_{k=3}^{n}\frac{1}{k+1}\right)
\\
= \frac{1}{6}\left( 9\sum_{k=3}^{n}\frac{1}{k} + 5\sum_{k=1}^{n-2}\frac{1}{k} - 14\sum_{k=4}^{n+1}\frac{1}{k}\right)
\end{multline*}
Comme $9+ 5 - 14 = 0$, les termes des sommes entre $4$ et $n-2$ disparaissent. Il reste :
\[
\sum_{k=3}^{n}\frac{4k-3}{k(k-2)(k+1)}
= \frac{1}{6}\left( 9 \,\frac{1}{3} + 5(1+\frac{1}{2}+\frac{1}{3}) + \varepsilon_{n}\right)
\]
o{\`u} $\varepsilon _{n}$ est une somme de termes qui tendent vers 0. On en d{\'e}duit que
\[
\left( \sum_{k=3}^{n}\frac{4k-3}{k(k-2)(k+1)}\right)_{n \geq 3} \rightarrow 
\frac{1}{6}\left( 3 + \frac{55}{6}\right) 
=\frac{73}{36}.
\]
\end{enumerate}
