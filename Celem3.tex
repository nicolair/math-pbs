\begin{enumerate}
 \item \begin{enumerate}
 \item Par définition de la tangente hyperbolique:
\begin{displaymath}
\frac{e^{2t}-1}{e^{2t}-1} = \frac{e^{t}-e^{-t}}{e^{t}+e^{-t}} = \th t
\end{displaymath}

\item Avec les relations usuelles de trigonométrie circulaire:
\begin{displaymath}
\frac{\tan ^{2}\varphi - 1}{\tan ^{2}\varphi+1} = \cos^2\varphi \,(\frac{\sin^2 \varphi}{\cos^2 \varphi} -1)
= \sin^2\varphi - \cos^2 \varphi = -\cos 2\varphi
\end{displaymath}

\item On cherche à montrer que
\begin{displaymath}
  \pi -2\arctan(e^t) = \arccos(\th t)
\end{displaymath}
Posons $\theta = \pi -2\arctan(e^t)$.\newline
A-t-il le bon cosinus?\newline
Utilisons la question pr{\'e}c{\'e}dente avec $\varphi = \arctan(e^t)$ donc $\tan \varphi = e^t$:
\begin{displaymath}
\cos \theta = -\cos (2\arctan (e^{t}))= \frac{e^{2t}-1}{e^{2t}+1}=\th t
\end{displaymath}

Comme $e^{t}$ est strictement positive, $\arctan (e^{t})\in \left[ 0,\frac{\pi }{2}\right]$. On en tire que
\begin{displaymath}
2\arctan (e^{t})\in \left[ 0,\pi \right] \Rightarrow \theta \in \left[ 0,\pi \right] 
\end{displaymath}
Il est donc dans le bon intervalle et on peut conclure
\begin{displaymath}
\pi -2\arctan(e^t) = \arccos(\th t) \Rightarrow  \arccos(\th t) + 2\arctan(e^t) = \pi
\end{displaymath}


\item D'apr{\`e}s l'expression de la fonction $\ch$, le réel strictement positif $t$ est solution de l'équation proposée si et seulement si $e^{t}$ est une solution plus grande que 1 de l'équation d'inconnue $z$
\begin{displaymath}z^{2}-\frac{2z}{\cos x}+1=0\end{displaymath}
Cette équation s'étudie sans problème, son discriminant est 
\begin{displaymath}
  \frac{4}{\cos^2 x} -4 = 4 \tan^2 x
\end{displaymath}
ses racines sont
\begin{displaymath}
\frac{1+ \sin x}{\cos x}, \hspace{0.5cm}\frac{1 - \sin x}{\cos x}  
\end{displaymath}
Elles sont positives et leur produit est 1. Une seule est plus grande que 1, c'est :
\begin{displaymath}
 \frac{1+\sin x}{\cos x}
\end{displaymath}
 On en d{\'e}duit
\begin{displaymath}
t=\ln (\frac{1+\sin x}{\cos x}) 
\end{displaymath}
On peut transformer cette expression, en posant $y=\frac{\pi}{2} - x$ et en passant à $\frac{y}{2}$:
\begin{displaymath}
\frac{1+\sin x}{\cos x} = \frac{1+\cos y}{\sin y} = \frac{2\cos^2 \frac{y}{2}}{2\sin\frac{y}{2}\cos\frac{y}{2}} 
= \frac{\cos\frac{y}{2}}{\sin \frac{y}{2}}  = \tan(\frac{\pi}{2}-\frac{y}{2}) = \tan(\frac{\pi}{4} + \frac{x}{2})
\end{displaymath}
L'unique solution est donc
\begin{displaymath}
  \ln\left( \tan(\frac{\pi}{4} + \frac{x}{2})\right) 
\end{displaymath}

\item Calculons $\cos (\arcsin (\frac{1}{\ch t}))$ en remarquant que le $\cos $ d'un $\arcsin $ est toujours positif ;
\begin{displaymath}
\cos (\arcsin (\frac{1}{\ch t}))=\sqrt{1-\frac{1}{\ch ^{2}t}}=\sqrt{\frac{\ch ^{2}t-1}{\ch ^{2}t}}=\left| \th t\right|
\end{displaymath}
Comme $\arcsin (\frac{1}{\ch t})$ et $\pi -\arcsin (\frac{1}{\ch t})$ sont dans $[0,\frac{\pi }{2}]\subset [0,\pi]$, on peut conclure.\newline 
Pour $t\geq 0$,
\begin{displaymath}
\cos (\arcsin (\frac{1}{\ch t}))=\th t \Rightarrow \arcsin (\frac{1}{\ch t})=\arccos (\th t)
\end{displaymath}
Pour $t\leq 0$,
\begin{displaymath}
\cos (\arcsin (\frac{1}{\ch t}))=-\th t \Rightarrow \pi - \arcsin (\frac{1}{\ch t})=\arccos (\th t)   
\end{displaymath}

\end{enumerate}

\item  Posons $a=(7+5\sqrt{2})^{\frac{1}{3}}$, $b=(-7+5\sqrt{2})^{\frac{1}{3}}$ et utilisons l'identit{\'e}
\begin{displaymath}
a^{3}-b^{3}=(a-b)(a^{2}+ab+b^{2})=(a-b)((a-b)^{2}+3ab)
\end{displaymath}
Comme $a^{3}-b^{3}=14$ et $ab=1$, on en d{\'e}duit que le nombre $x$ que l'on nous demande de simplifier est racine de
\begin{displaymath}
x^{3}+3x-14=(x-2)(x^{2}+2x+7)
\end{displaymath}
donc $x=2$ car $x^{2}+2x+17$ est sans racine r{\'e}elle.\newline
De m{\^e}me, si $a=\left( \frac{13+5\sqrt{17}}{2}\right) ^{\frac{1}{3}}$ et $b=\left( \frac{-13+5\sqrt{17}}{2}\right) ^{\frac{1}{3}}$, $a^{3}-b^{3}=13$
et $ab=4$, on en d{\'e}duit que le nombre $x$ que l'on nous demande de simplifier est racine de
\begin{displaymath}
x^{3}+12x-13=(x-1)(x^{2}+x+13)
\end{displaymath}
donc $x=1$ car $x^{2}+x+13$ est sans racine r{\'e}elle.
 
\item En utilisant les formules de transformation de produits en sommes, on obtient
\begin{displaymath}
\frac{1}{8}(\sin 2x+\sin 6x+\sin 8x)
\end{displaymath}

\item D'après la formule donnant la tangente d'une somme,
\begin{displaymath}
\tan (\arctan (1+x)-\arctan x)=\frac{1+x-x}{1+(1+x)x}=\frac{1}{1+x+x^{2}}
\end{displaymath}
On en tire qu'il existe un entier $k$ tel que
\begin{displaymath}
  \arctan (1+x)-\arctan x = \arctan \frac{1}{1+x+x^{2}} + k\pi
\end{displaymath}
Il s'agit maintenant de montrer que $k$ est nul.\newline
Remarquons d'abord que $1+x+x^{2}>0$ pour tous les r{\'e}els $x$ donc.
\begin{displaymath}
  \arctan \frac{1}{1+x+x^{2}} \in \left]0,\frac{\pi}{2}\right[
\end{displaymath}

D'autre part, par croissance et d{\'e}finition de la fonction $\arctan $, 
\begin{displaymath}
\arctan (1+x)-\arctan x\in \left[ 0,\pi \right[   
\end{displaymath}
On en déduit
\begin{displaymath}
k\pi = \arctan (1+x)-\arctan x - \arctan \frac{1}{1+x+x^{2}} \left]-\frac{\pi}{2}, \pi\right[
\end{displaymath}
On en tire $k=0$ donc
\begin{displaymath}
\arctan (1+x)-\arctan x=\arctan \frac{1}{1+x+x^{2}}
\end{displaymath}
\end{enumerate}