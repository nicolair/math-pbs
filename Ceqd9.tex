Comme il s'agit d'une application directe des méthodes du cours, on présente directement les fonctions constituant les ensembles de solutions. Pour chaque équation, $\lambda$ décrit $\R$.
\begin{align}
 x &\rightarrow 2 +\frac{\lambda}{2+x} & \lambda \in \R \\
 x &\rightarrow \frac{\arctan x + \lambda}{2x} & \lambda \in \R \\
 x &\rightarrow \lambda \cos 3x + \mu \sin 3x + \frac{1}{9}x + \frac{1}{18}e^{3x} & (\lambda,\mu) \in \R^2 
\end{align}
Les calculs pour obtenir une solution particulière pour la dernière équation sont plus compliqués. On travaille d'abord avec le second membre $xe^{3x}$. Pour lequel on cherche une solution particulière de la forme $(ax+b)e^{3x}$. On obtient 
\begin{align*}
 a&=\frac{1}{10}(3-i) & b&=\frac{4i-1}{(3+i)^2}=\frac{1}{50}(8+19i)
\end{align*}
Il reste ensuite à prendre la partie réelle et à multiplier par $10$
\begin{align}
 x &\rightarrow \lambda e^x  + \mu e^{-\frac{x}{2}} + (3\cos x +\sin x)x +\frac{1}{5}(8\cos x -19 \sin x)& (\lambda,\mu) \in \R^2 
\end{align}

