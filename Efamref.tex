%<dscrpt>Familles de vecteurs et de reflexions.</dscrpt>
Dans tout le problème\footnote{d'après Centrale Supélec 2 PC 2005}, on désigne par $n$ un entier égal à 2 ou 3 et par  $A$ une matrice $(n,n)$ symétrique dont les coefficients $a_{ij}$ sont des \emph{entiers naturels non nuls}. Les coefficients de la diagonale principale de $A$ sont des 1.\newline
On désigne par $M$ la matrice réelle carrée d'ordre $n$ et de coefficient $m_{ij}$ défini par :
\[m_{ij}=-\cos \frac{\pi}{a_{ij}}\]
Dans le cas $n=2$, on notera
\begin{eqnarray*}
 a=a_{12}=a_{21},& m=m_{12}=m_{21}=-\cos \frac{\pi}{a}
\end{eqnarray*}
On désigne par $E$ un espace vectoriel euclidien orienté de dimension $n$ dont le produit scalaire est noté $(./.)$. On se propose d'étudier les  bases $\mathcal{B}=(e_1,\cdots,e_n)$ telles que:
\begin{displaymath}
\forall (i,j)\in \llbracket 1,n \rrbracket^2,\hspace{0.5cm} (e_i/e_j)=m_{ij} 
\end{displaymath}
On dira alors que $\mathcal B$ vérifie la propriété $\mathcal M$.

\subsubsection*{Partie I. Existence d'une famille vérifiant $\mathcal M$.}
\begin{enumerate}
 \item Calculer le déterminant de $M$ pour $n$ égal à 2 ou 3.
\item Montrer que s'il existe une base $\mathcal B$ vérifiant $\mathcal M$ alors $a_{ij}\geq 2$ pour tous les couples $(i,j)$ tels que $i\neq j$.
\begin{quote}
  Dans toute la suite du problème, on suppose $a_{ij}\geq 2$ pour tous les couples $(i,j)$ tels que $i\neq j$.
\end{quote}
\item Cas $n=2$. Construire une base directe vérifiant $\mathcal M$.
\item Cas $n=3$. On veut  construire une base directe $\mathcal{B}=(e_1,e_2,e_3)$ vérifiant $\mathcal M$. Soit $(a_1,a_2,a_3)$ une base orthonormée directe de $E$, on pose $e_1=a_1$.
\begin{enumerate}
 \item Préciser un vecteur $e_2\in\Vect (a_1,a_2)$ tel que
\begin{align*}
 (e_1,e_2,a_3)  \text{ directe et }
 (e_1/e_2)=m_{12} 
\end{align*}

\item L'ensemble des vecteurs $x$ de $E$ tels que
\begin{displaymath}
 \left\lbrace 
\begin{aligned}
 (x/e_1) &= m_{13}\\
 (x/e_2) &= m_{23}
\end{aligned}
\right. 
\end{displaymath}
forme une droite affine $\mathcal D$.\newline
Quelle est sa direction? Calculer les coordonnées dans $(a_1,a_2)$ du point d'intersection $\mathcal D$ avec le plan $\Vect (a_1,a_2)$. En déduire la distance du vecteur nul à la droite $\mathcal D$.
\item Traduire par une propriété géométrique faisant intervenir $\mathcal D$ l'existence d'un vecteur $e_3$ tel que $(e_1,e_2,e_3)$ vérifie $\mathcal M$.
\item Montrer que si $\det M >0$ il existe une base $\mathcal B$ vérifiant $\mathcal M$.
\end{enumerate}
\item Cas particulier $n=3$ et
\begin{displaymath}
 A= \begin{bmatrix}
 1 & 3 & 2 \\
3 & 1 & 4 \\
2 & 4 & 1
\end{bmatrix}
\end{displaymath}
Préciser $M$ et montrer qu'il existe une base $\mathcal B$ vérifiant $\mathcal M$.
\end{enumerate}

\subsubsection*{Partie II. Famille de réflexions.}
Dans cette partie, $\mathcal B =(e_1,\cdots,e_n)$ est une base directe vérifiant $\mathcal M$.
On désigne par $\sigma_i$ la \emph{réflexion} telle que
\[\sigma_i(e_i)=-e_i\]
\begin{enumerate}
 \item On considère deux vecteurs $x$ et $y$ de $E$ admettant pour coordonnées dans $\mathcal B$ respectivement $(x_1,\cdots,x_n)$ et $(y_1,\cdots,y_n)$. Comment peut-on traduire matriciellement qu'ils sont orthogonaux ?

 \item Cas $n=2$. \begin{enumerate}
 \item Former les matrices $S_1$, $S_2$, $T$ dans $\mathcal B$ de $\sigma_1$, $\sigma_2$ et $\tau=\sigma_1 \circ \sigma_2$. Pour trouver $S1$, on pourra par exemple considérer le vecteur $me_1-e_2$ qui est orthogonal à $e_1$.
\item Soit $\mathcal{C}=(a_1,a_2)$ une base orthonormée directe avec $a_1=e_1$. Former les matrices dans $\mathcal C$ de $\sigma_1$, $\sigma_2$ et $\tau$. En déduire la nature et les éléments géométriques de $\tau$.
\end{enumerate}
\item Cas $n=3$. Former les matrices  $S_1$, $S_2$, $S_3$ dans $\mathcal B$ de $\sigma_1$, $\sigma_2$, $\sigma_3$. On pourra par exemple considérer les vecteurs $m_{12}e_1-e_2$ et $m_{13}e_1-e_3$ qui sont orthogonaux à $e_1$.
\item Cas particulier $n=3$ et
\begin{displaymath}
 A= \begin{bmatrix}
 1 & 3 & 2 \\
3 & 1 & 4 \\
2 & 4 & 1
\end{bmatrix}
\end{displaymath}
\begin{enumerate}
 \item Former la matrice $T$ de $\tau=\sigma_1 \circ \sigma_2 \circ \sigma_3$ dans $\mathcal B$.
 \item Déterminer un vecteur unitaire $u$ tel que $\tau(u)=-u$ puis une base orthonormée directe $\mathcal{D}=(u,v,w)$. On choisira un vecteur $v$ combinaison linéaire de $e_1$ et $e_3$.
 \item Former la matrice de $\tau$ dans $\mathcal D$. En déduire sa nature et ses éléments géométriques.
\end{enumerate}
\end{enumerate}

