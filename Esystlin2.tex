%<dscrpt>Système d'équations linéaires avec deux paramètres.</dscrpt>
On considère deux systèmes de trois équations linéaires aux inconnues réelles $(x,y,z)$ dépendant des paramètres réels $a$ et $b$
\begin{displaymath}
(S):\;\left\lbrace 
\begin{aligned}
  ax + by + z &= 1 \\ x + aby + z &= b \\ x + by + az &= 1
\end{aligned}
\right. 
\hspace{1cm}
(S')\;
\left\lbrace 
\begin{aligned}
 &x + &z + &aby &=& b \\ &  &(a-1)z + &b(1-a)y &=& 1 - b \\ &  &  &b(1-a)(2+a)y &=& 2 -b -ab
\end{aligned}
\right. 
\end{displaymath}
\begin{enumerate}
  \item Préciser les opérations élémentaires assurant que les deux systèmes sont équivalents.
  \item Préciser, suivant les valeurs des paramètres $a$ et $b$ les ensembles de solutions. Lorsque le système admet un unique triplet solution, on ne cherchera pas à le calculer.
\end{enumerate}
