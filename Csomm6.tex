\begin{enumerate}
 \item La somme $s_1 + \cdots + s_{n-1}$ contient 
\begin{displaymath}
 1 + 2 + \cdots + (n-1) = \frac{(n-1)n}{2} \text{ termes.}
\end{displaymath}
On paramètre les impairs successifs $1, 3, 5, \cdots$ sous la forme $2k-1$ avec $k$ prenant les valeurs $1, 2, 3, \cdots$. Le plus grand terme de la somme $s_1 + s_2 + \cdots + s_{n-1}$ est donc 
\begin{displaymath}
 2\,\frac{(n-1)n}{2} -1 = (n-1)n - 1
\end{displaymath}
On remarque que ce nombre est bien impair car $n(n-1)$ est pair (parmi deux nombres consécutifs, un est pair).

 \item Le plus petit terme de $s_n$ est le nombre impair qui suit le plus grand des sommes précédentes, c'est donc
\begin{displaymath}
 (n-1)n + 1
\end{displaymath}
Comme $s_n$ est formée par les $n$ nombres impairs qui suivent, on obtient
\begin{displaymath}
 s_n =  
\sum_{k=1}^n\left( (n-1)n + 2k-1\right)
\end{displaymath}

 \item Le calcul de $s_n$ n'est pas difficile
\begin{displaymath}
 s_n = \sum_{k=1}^n\left( (n^2-n-1) + 2k\right)
= n(n^2-n-1) + 2\,\frac{n(n+1)}{2}
= n^3
\end{displaymath}
Ce résultat est connu sous le nom de théorème de Nicomachus.

\item La somme de \emph{tous} les entiers entre $1$ et $n(n+1)$ est 
\begin{displaymath}
 \frac{n(n+1)\left( n(n+1)+1\right) }{2}
\end{displaymath}
La somme étendue seulement aux entiers pairs est égale à 2 fois celle de tous les entiers entre $1$ et $\frac{n(n+1)}{2}$. Elle est donc égale à
\begin{displaymath}
 2\,\frac{\frac{n(n+1)}{2}(\frac{n(n+1)}{2}+1)}{2}
 = \frac{1}{2}n(n+1)\left( \frac{n(n+1)}{2}+1\right) 
\end{displaymath}

D'après la question 1, le plus grand terme de $s_n$ est $n(n+1)-1$. Le théorème de Nicomachus montre alors que 
\begin{multline*}
 1^3 + 2^3 + \cdots + n^3
= \text{ somme des entiers impairs entre $1$ et $n(n+1)-1$}  \\
= \left( \text{ somme de tous les entiers entre $1$ et $n(n+1)$}\right) \\
- \left( \text{ somme des entiers pairs entre $1$ et $n(n+1)$}\right)\\
= \frac{n(n+1)\left( n(n+1)+1\right) }{2} - \frac{n(n+1)}{2}\left( \frac{n(n+1)}{2}+1\right)\\
= \frac{n(n+1)}{2}\left( n(n+1)+1 - \frac{n(n+1)}{2} - 1\right) 
= \left( \frac{n(n+1)}{2}\right)^2 
\end{multline*}

\end{enumerate}
