\subsection*{Partie I}
\begin{enumerate}
 \item On va montrer que, pour toute matrice $A\in\mathcal T^+$ :
 $A^3$ est la matrice nulle.\newline
En effet
\begin{displaymath}
 A=
\begin{pmatrix}
 0 & a & b \\
 0 & 0 & c \\ 
 0 & 0 & 0
\end{pmatrix}
\Rightarrow
A^2=
\begin{pmatrix}
 0 & 0 & ac \\
 0 & 0 & 0 \\ 
 0 & 0 & 0
\end{pmatrix}
\Rightarrow
A^3=
\begin{pmatrix}
 0 & 0 & 0 \\
 0 & 0 & 0 \\ 
 0 & 0 & 0
\end{pmatrix}
\end{displaymath}
\item \begin{enumerate}
 \item 
Soit $D$ une matrice diagonale dont les termes diagonaux sont $\alpha$, $\beta$, $\gamma$ et commutant avec toutes les triangulaires supérieures strictes.
\begin{align*}
 &D \text{ commute avec }
\begin{pmatrix}
 0 & 1 & 0\\
0 & 0 & 0 \\
0 & 0 & 0
\end{pmatrix}
\Rightarrow
\begin{pmatrix}
 0 & \alpha & 0\\
0 & 0 & 0 \\
0 & 0 & 0
\end{pmatrix}
=
\begin{pmatrix}
 0 & \beta & 0\\
0 & 0 & 0 \\
0 & 0 & 0
\end{pmatrix}
\Rightarrow \alpha =\beta \\
 &D \text{ commute avec }
\begin{pmatrix}
 0 & 0 & 0\\
0 & 0 & 1 \\
0 & 0 & 0
\end{pmatrix}
\Rightarrow
\begin{pmatrix}
 0 & 0 & 0\\
0 & 0 & \beta \\
0 & 0 & 0
\end{pmatrix}
=
\begin{pmatrix}
 0 & 0 & 0\\
0 & 0 & \gamma \\
0 & 0 & 0
\end{pmatrix}
\Rightarrow \beta =\gamma
\end{align*}
On doit donc avoir $\alpha=\beta=\gamma$ soit $D\in \Vect(I_3)$. Réciproquement, toute matrice de la forme $\lambda I_3$ commute avec toute autre matrice.
\item Comme $D$ commute avec $A$, on peut utiliser la formule du binöme. Cette formule est très simple car les puissances de $A$ sont nulles à partir de l'exposant $3$. On obtient :
\begin{displaymath}
 (D+A)^n = D^n + nD^{n-1}A + \frac{n(n-1)}{2}D^{n-2}A^2
\end{displaymath}
\end{enumerate}
\end{enumerate}
\subsection*{Partie II}
\begin{enumerate}
 \item Il est bien évident par définition que $\mathcal E$ est non vide, pour montrer que c'est un sous-groupe de $GL_3(\R)$, on doit vérifier que :
\begin{itemize}
 \item toute matrice de $\mathcal E$ est inversible et que son inverse est encore dans $\mathcal E$.
\item le produit de deux matrices de $\mathcal E$ est dans $\mathcal E$.
\end{itemize}
Une matrice de $\mathcal E$ est triangulaire supérieure avec des $1$ sur la diagonale. Son rang est donc $3$ ce qui prouve son inversibilité. On ne cherche pas à montrer tout de suite que la matrice inverse est dans $\mathcal E$. Cela sera rendu plus facile par la stabilité suivante.\newline
On vérifie facilement que
\begin{displaymath}
 M(a,b)M(a',b')=M(a+a',b+b'+aa')
\end{displaymath}
On en déduit la stabilité pour le produit.\newline
On remarque que $I_3=M(0,0)$. On en déduit que $M(-a,-b+a^2)=M(a,b)^{-1}$. Ce qui prouve la stabilité par inversion qui manquait.
\item \begin{enumerate}
 \item Appliquons la relation de définition avec $x$ quelconque et $y=0$. On obtient
\begin{displaymath}
 \widehat{M}(x)\widehat{M}(0)= \widehat{M}(x)
\end{displaymath}
Comme $\widehat{M}(x)$ est inversible, cela entraine $\widehat{M}(0)=I_3$ donc $f(0)=0$. Si on applique ensuite la relation fondamentale à $x$ et $-x$ (dans les deux sens), on obtient
\begin{displaymath}
 \widehat{M}(x)^{-1}=\widehat{M}(-x)
\end{displaymath}
\item On peut réécrire avec des $f$ l'expression du produit trouvée en 1.
\begin{displaymath}
 M(x,f(x))M(y,f(y))=M(x+y,f(x)+f(y)+xy)
\end{displaymath}
On en déduit:
\begin{displaymath}
 \widehat{M}(x)\widehat{M}(y)=\widehat{M}(x+y)
\Leftrightarrow
f(x)+f(y)+xy = f(x+y)
\end{displaymath}
\item La vérification est immédiate par le calcul.
\item Pour $y\neq 0$ et $x$ quelconque, en utilisant $f(0)=0$, on peut écrire la relation fonctionnelle sous la forme suivante :
\begin{displaymath}
 \frac{f(x+y)f(x)}{y}=\frac{f(y)-f(0)}{y}+x
\end{displaymath}
Si on suppose $f$ dérivable dans $\R$, en prenant la limite en $0$ des fonctions de $y$, on obtient:
\begin{displaymath}
 \forall x\in \R : f'(x)=f'(0)+x
\end{displaymath}
Posons $m=f'(0)$, en intégrant on retrouve bien que $f$ est de la forme
\begin{displaymath}
 f(x)= \frac{1}{2}x^2 +mx
\end{displaymath}
 La constante d'intégration est nulle car on doit avoir $f(0)=0$.
\end{enumerate}

\end{enumerate}
