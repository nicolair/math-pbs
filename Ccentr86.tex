\begin{enumerate}
\item  La fonction $\left( \frac{\sin t}{t}\right) ^{2n}$ se prolonge par
continuit\'{e} en 0 avec la valeur 1. Pour tout $t\in $ $\left[ 1,+\infty
\right[ $, on peut \'{e}crire 
\[
\left| \left( \frac{\sin t}{t}\right) ^{2n}\right| \leq \frac{1}{t^{2n}}
\]
Comme $2n\geq 2$, la fonction $\frac{1}{t^{2n}}$ est int\'{e}grable sur $%
\left[ 1,+\infty \right[ $. On en d\'{e}duit l'int\'{e}grabilit\'{e} sur $%
\left] 0,+\infty \right[ $.

\item  Il est bien connu (\'{e}tude de fonction ou convexit\'{e}) que $\sin
t\leq t$ pour tout $t$ positif. L'\'{e}galit\'{e} se produisant seulement en
0. Dans l'intervalle $\left[ a,\frac{\pi }{2}\right] $, cette
in\'{e}galit\'{e} assure $\left| \sin t\right| <t$ car le sin est positif.
Au del\`{a} de $\frac{\pi }{2}$ on peut \'{e}crire $\left| \sin t\right|
\leq 1<\frac{\pi }{2}\leq t$. Ceci assure que la fonction $t\rightarrow 
\frac{\sin t}{t}$ est major\'{e}e strictement par 1. Comme elle converge
vers 0 en +$\infty $, elle atteint sa borne sup\'{e}rieure sur $\left[
a,+\infty \right[ $. Cette borne $M(a)$ est donc strictement plus petite que
1.\newline
Pour obtenir l'in\'{e}galit\'{e} demand\'{e}e dans l'int\'{e}grale, on isole
un 2 dans l'exposant 
\begin{multline*}
\int_{a}^{+\infty }\left( \frac{\sin t}{t}\right) ^{2n}dt = \int_{a}^{+\infty}\left( \frac{\sin t}{t}\right)^{2n-2}\left( \frac{\sin t}{t}\right)
^{2}dt \\
\leq M(a)^{2n-2}\int_{a}^{+\infty }\frac{\sin ^{2}t}{t^{2}}dt
\end{multline*}
Si $a\geq 1$, on majore le $\sin $ par 1 d'o\`{u} $\int_{a}^{+\infty }\frac{%
\sin ^{2}t}{t^{2}}dt\leq $ $\int_{1}^{+\infty }\frac{1}{t^{2}}dt=1$.\newline
Si $a\in \left] 0,1\right[ $, on coupe en deux morceaux et dans le premier
on utilise $\left| \sin t\right| <t$ : 
\begin{multline*}
\int_{a}^{+\infty }\frac{\sin ^{2}t}{t^{2}}dt =
\int_{a}^{1}\frac{\sin ^{2}t}{t^{2}}dt+\int_{1}^{+\infty }\frac{\sin ^{2}t}{t^{2}}dt \\
\leq \int_{a}^{1}1dt+\int_{1}^{+\infty }\frac{1}{t^{2}}dt = 2-a \leq 2.
\end{multline*}

\item  On effectue le changement de variable $t=a\sin u$ dans $J_{n}(a)$. On
en d\'{e}duit 
\[
J_{n}(a)=aI_{2n+1} 
\]

\item  Formons le tableau de variations de la fonction $\phi (t)=\sin t-t+%
\frac{1}{6}t^{3}$ pour $t\in \left] 0,\sqrt{6}\right] $. Les
d\'{e}riv\'{e}es sont : $\phi ^{\prime }(t)=\cos t-1-\frac{1}{2}t^{2}$, $%
\phi ^{\prime \prime }(t)=-\sin t+t$%
\[
\begin{tabular}{|l|l|}
\hline
& $0\qquad \sqrt{6}$ \\ \hline
$\phi ^{\prime \prime }$ & $\quad +$ \\ \hline
$\phi ^{\prime }$ & $0\quad \nearrow $ \\ \hline
$\phi $ & $0\quad +$ \\ \hline
\end{tabular}
\]

\item  D'apr\`{e}s les questions pr\'{e}c\'{e}dentes : $u_{n}=\int_{a}^{+%
\infty }\left( \frac{\sin t}{t}\right) ^{2n}dt\geq \int_{a}^{\sqrt{6}}\left( 
\frac{\sin t}{t}\right) ^{2n}dt\geq \int_{a}^{+\infty }\left( 1-\frac{1}{6}%
t^{2}\right) ^{2n}dt=\sqrt{6}I_{4n+1}$.

\item 
\begin{enumerate}
\item  Posons $\phi _{\lambda }(t)=\sin t-t-\frac{1}{\lambda ^{2}}t^{3}$ et
d\'{e}rivons : 
\[
\phi _{\lambda }^{\prime }(t)=\cos t-1+\frac{3}{\lambda ^{2}}t^{2}\text{, }%
\phi _{\lambda }^{\prime \prime }(t)=-\sin t+\frac{6}{\lambda ^{2}}t\text{, }%
\phi _{\lambda }^{\prime \prime \prime }(t)=-\cos t+\frac{6}{\lambda ^{2}} 
\]
Mais$\frac{6}{\lambda ^{2}}<1$ car $\lambda >\sqrt{6},$ il existe donc (par
continuit\'{e} en 0 de $\cos $) un $\mu $ assez petit pour que $\cos t>\frac{%
6}{\lambda ^{2}}$ lorsque $t\in \left[ 0,\mu \right] .$ On peut aussi
supposer $\mu <\lambda $. Le tableau de variation de $\phi _{\lambda }$ dans 
$\left[ 0,\mu \right] $ montre alors que $\phi _{\lambda }$ est n\'{e}gative
dans cet intervalle ce qui d\'{e}montre l'in\'{e}galit\'{e} demand\'{e}e.

\item  En d\'{e}coupant l'int\'{e}grale puis en utilisant les relations
d\'{e}montr\'{e}es : 
\begin{multline*}
u_{n} = \int_{0}^{+\infty }\left( \frac{\sin t}{t}\right)^{2n}dt\\
=\int_{0}^{\mu }\left( \frac{\sin t}{t}\right) ^{2n}dt + \int_{\mu
}^{+\infty }\left( \frac{\sin t}{t}\right) ^{2n}dt
\end{multline*}
\begin{multline*}
\int_{0}^{\mu }\left( \frac{\sin t}{t}\right) ^{2n}dt
 \leq \int_{0}^{\mu}\left( 1-\frac{1}{\lambda ^{2}}t^{2}\right) ^{2n}dt \\
\leq \int_{0}^{\lambda}\left( 1-\frac{1}{\lambda ^{2}}t^{2}\right) ^{2n}dt 
= J_{n}(\lambda ) = \lambda I_{4n+1}
\end{multline*}
\begin{displaymath}
\int_{\mu }^{+\infty }\left( \frac{\sin t}{t}\right) ^{2n}dt 
\leq 2M(\mu)^{2n-2}
\end{displaymath}
On obtient la relation demand\'{e}e 
\[
u_{n}\leq J_{n}(\lambda )+2M(\mu )^{2n-2} 
\]
\end{enumerate}

\item  Les questions pr\'{e}c\'{e}dentes montrent que, pour tout $\lambda >%
\sqrt{6}$, il existe un $\mu \in \left] 0,\lambda \right[ $ tel que 
\begin{equation}
\sqrt{6}\leq \frac{u_{n}}{I_{4n+1}}\leq \lambda +\frac{2M(\mu )^{2n-2}}{%
I_{4n+1}(\lambda )}
\end{equation}
De plus, l'\'{e}nonc\'{e} nous donne un \'{e}quivalent de $I_{n}$ : $%
I_{4n+1}\sim \sqrt{\frac{\pi }{2(4n+1)}}\sim \sqrt{\frac{\pi }{8n}}$
d'o\`{u} 
\[
\frac{2M(\mu )^{2n-2}}{I_{4n+1}}\sim K_{\mu }\rho _{\mu }^{n}\sqrt{n} 
\]
o\`{u} $K_{\mu }$ est une constante positive (qui d\'{e}pend de $\mu )$ et $%
\rho =M(\mu )^{2}\in \left] 0,1\right[ $.\newline
Consid\'{e}rons un nombre $\varepsilon >0$ arbitraire, fixons un $\lambda
\in \left] \sqrt{6},\sqrt{6}+\frac{1}{2}\varepsilon \right[ $. L'encadrement
(1) est alors v\'{e}rifi\'{e} pour un certain $\mu $.\newline
Lorsque $\lambda $ et $\mu $ sont fix\'{e}s, $(K_{\mu }\rho _{\mu }^{n}\sqrt{%
n})_{n\in \mathbf{N}}$ tend vers 0 quand $n$ tend vers l'infini. Il existe
donc un entier $N$ tel que 
\[
\frac{2M(\mu )^{2n-2}}{I_{4n+1}}\leq \frac{1}{2}\varepsilon 
\]
lorsque $n\geq N$. En injectant cela dans (1), on obtient que 
\[
\sqrt{6}\leq \frac{u_{n}}{I_{4n+1}}\leq \sqrt{6}+\frac{1}{2}\varepsilon +%
\frac{1}{2}\varepsilon 
\]
C'est \`{a} dire que $(\frac{u_{n}}{I_{4n+1}})_{n\in \mathbf{N}}$ converge
vers $\sqrt{6}$. Finalement : 
\[
u_{n}\sim \sqrt{6}I_{4n+1}\sim \sqrt{\frac{6\pi }{2(4n+1)}}\sim \sqrt{\frac{%
3\pi }{4n}} 
\]
\end{enumerate}
