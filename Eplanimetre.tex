%<dscrpt>Planimètre, ou comment une roulette permet de mesurer une aire.</dscrpt>
\begin{figure}
   \centering
   \input{Eplanimetre_1.pdf_t}
   \caption{planim{\`e}tre}
   \label{fig:Eplanimetre_1}
\end{figure}
Un \emph{planimètre} est un dispositif mécanique permettant de mesurer l'aire d'une portion de plan définie par une courbe fermée.\newline
Ce dispositif est formé de deux tiges rigides. La première (de longueur $R$) peut tourner autour d'un point fixé. La deuxième (de longueur $r$) peut tourner autour de l'extrémité de la première. Une roulette est fixée perpendiculairement à la deuxième tige (en bleu sur la figure \ref{fig:Eplanimetre_1}). L'extrémité de la deuxième tige décrit une courbe fermée formant le bord d'un domaine $D$.\newline
Lors du déplacement du point autour de la courbe, la roulette glisse et tourne. On souhaite montrer que l'aire de la partie de plan définie par la courbe est proportionnelle à la distance parcourue par la roulette en tournant.
\begin{enumerate}
\item Quel est l'ensemble des points que peut-atteindre l'extrémité de la deuxième tige ?
\item Préciser une partie du plan $\Omega$ telle que si $m\in \Omega$, il existe un unique couple $(\alpha(m),\beta(m))$ de réel entre $0$ et $\pi$ tels que $m$ soit l'extrémité de la deuxième tige avec $\alpha(m)$ égal à l'angle de la première tige avec l'axe horizontal et $\beta(m)$ égal à l'angle de la deuxième tige avec l'axe horizontal.\newline
Dans toute la suite on se place dans cette partie $\Omega$ du plan. Les \emph{fonctions} $\alpha$, $\beta$, $x$, $y$ sont définies dans $\Omega$ et à valeurs réelles ($x(m)$ et $y(m)$ sont les coordonnées de $m$). On suppose aussi que la courbe fermée est dans $\Omega$.
\item Exprimer $x$ et $y$ en fonction de $\alpha$ et $\beta$. En déduire l'expression de l'élément de surface $dx\wedge dy$ en fonction de $d\alpha$ et $d\beta$.
\item Soit $\overrightarrow{u}(m)$ un vecteur unitaire perpendiculaire à la deuxième tige et $\omega$ la forme differentielle
\[\omega_{m}(\overrightarrow{v})=(\overrightarrow{u}(m)/\overrightarrow{v})\]
Exprimer $\omega$ en fonction de $d\alpha$ et $d\beta$. Conclure
\end{enumerate}
\begin{figure}
   \centering
   \input{Eplanimetre_2.pdf_t}
   \caption{Système de coordonnées $(\alpha,\beta)$}
   \label{fig:Eplanimetre_2}
\end{figure}

