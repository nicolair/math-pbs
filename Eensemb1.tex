%<dscrpt>Etude d'une injectivité dans un contexte ensembliste.</dscrpt>
Soit $E$ un ensemble et $A$, $B$ deux parties fixées de $E$. On définit une fonction $f$ par 
\begin{displaymath}
  f:
\left\lbrace 
\begin{aligned}
  \mathcal{P}(E) &\rightarrow \mathcal{P}(E)\times \mathcal{P}(E) \\
  X &\mapsto f(X) = (A\cap X, B \cup X)
\end{aligned}
\right. 
\end{displaymath}
\begin{enumerate}
  \item Préciser $f(A)$, $f(A\cup B)$, $f(\emptyset)$, $f(B\cap \overline{A})$. Que peut-on en déduire si $f$ est injective ?
  \item Soit $X$ une partie de $E$, montrer que 
\begin{displaymath}
X = (X \cap B) \cup \left( (X\cup B) \cap \overline{B}\right)   
\end{displaymath}

  \item Dans cette question, on suppose $B \subset A$.
\begin{enumerate}
  \item Montrer que :
\begin{displaymath}
\forall (X,Y)\in \mathcal{P}(E)^2, \; A \cap X = A \cap Y \Rightarrow B \cap X = B \cap Y
\end{displaymath}
  \item Montrer que $f$ est injective.
\end{enumerate}

\end{enumerate}
