\begin{enumerate}
 \item Par définition de $\theta$ comme une $\arctan$, $\cos \theta >0$ et $x-1=\tan \theta$ donc
\begin{displaymath}
 T(x) = \frac{1+ \tan \theta}{\sqrt{2}\sqrt{\tan^2\theta +1}}=\frac{|\cos\theta|}{\sqrt{2}}(1+\tan \theta)
=\frac{1}{\sqrt{2}}(\cos \theta + \sin\theta)
=\sin\left( \theta + \frac{\pi}{4}\right) 
\end{displaymath}
 
 \item L'expression précédente de $T(x)$ comme un $\sin$ montre que $-1\leq T(x) \leq 1$ pour tous les $x$ réels.

 \item D'après les questions précédentes, $\arcsin(T(x)) = \arcsin(\sin(\theta +\frac{\pi}{4}))$.\newline
Lorsque $\theta +\frac{\pi}{4}\in [-\frac{\pi}{2},\frac{\pi}{2}]$, on a évidemment $\arcsin(T(x)) = \theta +\frac{\pi}{4}$. Pour quels $x$ cela se produit-il?\newline
Si $x\leq 2$, alors $x-1\leq 1$ donc $\theta = \arctan(x-1) \in ]-\frac{\pi}{2},\frac{\pi}{4}]$ donc $\theta +\frac{\pi}{4}\in ]-\frac{\pi}{4},\frac{\pi}{2}]$ et
\begin{displaymath}
 \arcsin(T(x)) = \arctan(x-1) +\frac{\pi}{4}
\end{displaymath}
Si $x> 2$, alors $x-1 > 1$ donc $\theta = \arctan(x-1) \in [\frac{\pi}{4},\frac{\pi}{2}]$ donc $\theta +\frac{\pi}{4}\in [\frac{\pi}{2},\frac{3\pi}{4}[$ et
$\theta +\frac{\pi}{4}$ a le même $\sin$ que 
\begin{displaymath}
 \pi -\left( \theta +\frac{\pi}{4}\right) \in
]-\frac{\pi}{4},\frac{\pi}{2}]
\end{displaymath}
On en déduit  
\begin{displaymath}
 \arcsin(T(x)) = \frac{3\pi}{4} -\arctan(x-1) 
\end{displaymath}



\end{enumerate}
