%<dscrpt>Famille de courbes paramétrées en polaire.</dscrpt>
Soit $a$ un r{\'e}el strictement positif donn{\'e} et $(O,\overrightarrow i,\overrightarrow j)$ un repère fixé.\newline
Pour tout réel $\varphi$, on définit $\overrightarrow e_{\varphi}$ par
\begin{displaymath}
 \overrightarrow e_\varphi = \cos\varphi \overrightarrow i + \sin\varphi \overrightarrow j
\end{displaymath}
\`A tout nombre r{\'e}el $\alpha $, on associe la courbe paramétrée
\begin{displaymath}
 t \rightarrow 0 + r_\alpha(t)\overrightarrow e_{\theta(t)}
\end{displaymath}
avec
\begin{align*}
r_\alpha(t)=\frac{a\cos \alpha }{\cos ^2\frac{2t}3\sin (\frac t3-\alpha)}
& & \theta (t)=t. 
\end{align*}
On note $\mathcal C_\alpha$ le support de cette courbe.
\begin{enumerate}
\item  Comparer les courbes $\mathcal{C}_{\alpha }$, $\mathcal{C}_{\alpha+\pi }$, $\mathcal{C}_{-\alpha }$.

\item  Pour $\alpha $ donn{\'e} entre 0 et $\frac{\pi }{2}$, {\'e}tudier les branches infinies de $\mathcal{C}_{\alpha }$. On d{\'e}terminera la position de la courbe par rapport {\`a} l'asymptote.

\item  Construire la courbe $\mathcal{C}_{\frac{\pi }{4}}$.

\item  En utilisant $\frac{1}{r_\alpha}$ et ses d{\'e}riv{\'e}es, former une {\'e}quation caract{\'e}risant les valeurs du param{\`e}tre qui correspondent aux points non bir{\'e}guliers.\newline
Quel est l'ensemble de ces points lorsque $\alpha $ varie ?
\end{enumerate}
