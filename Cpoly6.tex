Si $P$ est un polynôme unitaire dont les racines sont $x,y,z$, il s'écrit
\[P=(X-x)(X-y)(X-z)=X^3-\sigma_1 X^2 + \sigma_2 X - \sigma_3\]
avec
\begin{displaymath}
\sigma_1 = x+y+z, \hspace{0.5cm}
\sigma_2 = xy+yz+xz, \hspace{0.5cm}
\sigma_3 = xyz 
\end{displaymath}
On utilise alors
\begin{displaymath}
x^2+y^2+z^2 = \sigma_1 ^2 -2\sigma_2 , \hspace{0.5cm}
\frac{1}{x}+\frac{1}{y}+\frac{1}{z} = \frac{\sigma_2}{\sigma_3}
\end{displaymath}
On en déduit que
\begin{displaymath}
 \sigma_1=2\;,\;\sigma_2=-1\;,\;\sigma_3=-2
\end{displaymath}
lorsque $x$, $y$, $z$ vérifient le système de l'énoncé. Le polynôme cherché est
\begin{displaymath}
X^3-2X^2-X+2 
\end{displaymath}
Les trois racines sont $-1$, $1$, $2$. Ce sont donc (à permutation près) les valeurs de $x$, $y$, $z$ lorsqu'ils vérifient le système de l'énoncé.