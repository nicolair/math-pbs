%<dscrpt>Equations différentielles linéaires et bijections réciproques.</dscrpt>
\begin{enumerate}
  \item
    \begin{enumerate}
     \item La fonction
     \begin{eqnarray*}
     ]0,\pi[ & \rightarrow & ]0,4[\\
     t& \rightarrow &2(1-\cos t)
     \end{eqnarray*}
     est bijective, d{\'e}rivable et sa d{\'e}riv{\'e}e ne s'annule pas dans
     $]0,\pi[$. On note $f$ sa bijection r{\'e}ciproque, pr{\'e}ciser la
     d{\'e}riv{\'e}e de $f$
     \item La fonction
     \begin{eqnarray*}
     ]0,+\infty[ & \rightarrow & ]-\infty ,0[\\
     t& \rightarrow & 2(1-\ch t)
     \end{eqnarray*}
     est bijective, d{\'e}rivable et sa d{\'e}riv{\'e}e ne s'annule pas dans
     $]0,\pi[$. On note $g$ sa bijection r{\'e}ciproque, pr{\'e}ciser la
     d{\'e}riv{\'e}e de $g$

    \end{enumerate}

  \item
    \begin{enumerate}
     \item Calculer les r{\'e}els $a$ et $b$ tels que pour tout r{\'e}el
     $x$ diff{\'e}rent de 0 et de 4, on ait
     \[\frac{x-2}{x(x-4)}=\frac{a}{x}+\frac{b}{x-4}\]
     \item Pr{\'e}ciser  dans chaque intervalle l'ensemble des solutions de l'{\'e}quation
     diff{\'e}rentielle
     \[x(x-4)y'+(x-2)y=0\]
     \item Quelles sont les fonctions continues et d{\'e}rivables dans
     $\R$ et v{\'e}rifiant l'{\'e}quation dans
     $\R$.
    \end{enumerate}

  \item

    \begin{enumerate}
     \item Montrer que la fonction $\frac{f}{\sin \circ f}$ est
     solution dans $]0,4[$ d'une {\'e}quation diff{\'e}rentielle lin{\'e}aire du premier ordre {\`a} pr{\'e}ciser.
     \item Montrer que la fonction $\frac{g}{\mathrm{sh\,} \circ g}$ est
     solution dans $]-\infty,0[$  d'une {\'e}quation diff{\'e}rentielle lin{\'e}aire du premier ordre {\`a} pr{\'e}ciser.
    \end{enumerate}

\end{enumerate}
