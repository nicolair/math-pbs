Pour tous les exercices, on se place dans un espace $\mathcal{E}$ muni d'un repère orthonormé direct $\mathcal R = (O,(\overrightarrow i , \overrightarrow j , \overrightarrow k))$ dont les fonctions coordonnées sont notées $x$, $y$, $z$.
\subsection*{Exercice 1.}
L'énoncé donne le plan $\mathcal{P}$ de manière paramétrique. On peut lire sur cette définition les coordonnées d'un point $A$ et de vecteurs $\overrightarrow u$  et $\overrightarrow v$ tels que
\begin{displaymath}
 \mathcal{P} = A +\Vect(\overrightarrow u , \overrightarrow v)\text{ avec }
A :
\begin{pmatrix}
 2\\3\\1
\end{pmatrix}
\;
\overrightarrow u :
\begin{pmatrix}
 1\\-1\\2
\end{pmatrix}
\;
\overrightarrow v :
\begin{pmatrix}
 -1\\2\\1
\end{pmatrix}
\end{displaymath}
On en déduit l'équation de $\mathcal{P}$ puis la distance d'un point à ce plan avec la formule du cours.
\begin{multline*}
 M\in \mathcal{P}\Leftrightarrow
\det(\overrightarrow{AM},\overrightarrow u, \overrightarrow v)=0
\Leftrightarrow
\begin{vmatrix}
 x(M)-2 & 1  & -1 \\
 y(M)-3 & -1 & 2  \\
 z(M)-1 & 2  & 1
\end{vmatrix} =0\\
\Leftrightarrow 
-5x(M)-3y(M)+z(M)+18=0
\end{multline*}
\begin{displaymath}
 d(M,\mathcal{P})=\frac{\left\vert -5x(M)-3y(M)+z(M)+18 \right\vert}{\sqrt{35}}
\end{displaymath}

\subsection*{Exercice 2.}
Un système d'équations pour une droite est constitué par les équations de deux plans dont l'intersection est égale à la droite. Ici, la droite $\mathcal{D}'$ dont on demande un système d'équations est la projection de $\mathcal{D}$ sur $\mathcal P$. On choisira donc $\mathcal{P}$ comme premier plan contenant $\mathcal D'$.\\
Pour l'autre plan, on choisit d'adjoindre à $\mathcal D$ un vecteur $\overrightarrow n$ normal à $\mathcal P$. On note $\Pi$ ce plan. Pour trouver une équation de $\Pi$, on forme une définition paramétrique de $\mathcal D$.
\begin{displaymath}
 M\in \mathcal{D}\Leftrightarrow
\left\lbrace 
\begin{aligned}
 x(M) + y(M) + z(M) -1 = 0 \\
 x(M) - y(M)-2z(M) = 0
\end{aligned}
\right. \\
\Leftrightarrow
\begin{pmatrix}
 x(M)\\y(M)\\z(M)
\end{pmatrix}
=
\begin{pmatrix}
 0\\2\\-1
\end{pmatrix}
+x(M)
\begin{pmatrix}
 1\\-3\\2
\end{pmatrix}
\end{displaymath}
On en déduit
\begin{displaymath}
\mathcal{D} = A+\Vect(\overrightarrow u),\; \Pi = A+\Vect(\overrightarrow u, \overrightarrow n)
\text{ avec }
A:
\begin{pmatrix}
 0\\2\\-1
\end{pmatrix}
\;
\overrightarrow u:
\begin{pmatrix}
 1\\-3\\2
\end{pmatrix}
\;
\overrightarrow n:
\begin{pmatrix}
 1\\2\\3
\end{pmatrix}
\;
\end{displaymath}
Les coordonnées de $\overrightarrow n$ ont été obtenues en lisant directement sur l'équation de $\mathcal{P}$. On en déduit l'équation de $\Pi$
\begin{multline*}
 M\in \Pi\Leftrightarrow \det(\overrightarrow{AM},\overrightarrow u, \overrightarrow n)=0
\Leftrightarrow
\begin{vmatrix}
 x(M)    & 1  & 1  \\
 y(M) -2 & -3 & 2  \\
 z(M)+1  & 2  & 3
\end{vmatrix} = 0 \\
\Leftrightarrow
-13x(M)-y(M)+5z(M)+ 7 =0
\end{multline*}
puis un système d'équations de $\mathcal{D}'$.
\begin{displaymath}
 M\in \mathcal{D}'\Leftrightarrow M \in \mathcal{P}\cap \Pi
\Leftrightarrow
\left\lbrace 
\begin{aligned}
 x(M) + 2y(M) + 3z(M)+6 &= 0\\
 -13x(M)-y(M)+5z(M)+7 &= 0
\end{aligned}
\right. 
\end{displaymath}

Une autre méthode est possible pour former l'équation du deuxième plan. Elle repose sur la notion de \emph{faisceau linéaire de plans}.\newline
Donnons nous deux réels $\alpha$ et $\beta$ et formons l'équation d'un plan $\Pi_{\alpha, \beta}$ à partir des deux plans définissant $\mathcal D$.
\begin{align*}
 \mathcal D &:
\left\lbrace 
\begin{aligned}
 &x+y+z-1 =0  & &\times \alpha \\
 &x-y-2z = 0  & &\times \beta
\end{aligned}
\right. \\
\Pi_{\alpha, \beta} &: 
(\alpha + \beta)x + (\alpha-\beta)y + (\alpha -2\beta)z -\alpha =0
\end{align*}
D'après sa définition même, ce plan contient la droite $\mathcal D$. Il est orthogonal à $\overrightarrow n_{\alpha , \beta}$ de coordonnées
\begin{displaymath}
 \overrightarrow n_{\alpha , \beta} :
\begin{pmatrix}
 \alpha + \beta \\ \alpha - \beta \\ \alpha -2 \beta
\end{pmatrix}
\end{displaymath}
Pour que $\Pi_{\alpha, \beta} = \Pi$, il suffit de choisir $\alpha$ et $\beta$ tels que $\overrightarrow n$ soit orthogonal à $\overrightarrow n_{\alpha , \beta}$.
\begin{displaymath}
 (\overrightarrow n_{\alpha , \beta}/\overrightarrow n) = 
(\alpha + \beta) +2( \alpha - \beta) + 3(\alpha -2 \beta)=6\alpha -7\beta = 0
\end{displaymath}
 
On choisit $\alpha =7$, $\beta=6$ et on obtient :
\begin{displaymath}
 \Pi : 13x+y-5z-7
\end{displaymath}


Remarque pour le lecteur qui a en charge d'évaluer des copies sur ce texte.\\
Ce type d'exercice est désagréable à évaluer car on pourrait qualifier la question de \emph{semi-ouverte}. Elle admet plusieurs réponses correctes. La validation d'une réponse peut nécessiter un calcul de la part du correcteur. On peut ici faire faire un calcul de rang à un logiciel de calcul formel. Si $\varphi$ et $\psi$ sont les deux équations proposées par le corrigé et $\varphi_1$, $\psi_1$ deux équations proposées par une copie, la réponse sera exacte si et seulement si
\begin{displaymath}
 \rg(\varphi,\psi,\varphi_1,\psi_1)=2
\end{displaymath}

\subsection*{Exercice 3.}
\begin{enumerate}
 \item On remarque que $\overrightarrow J$ et  $\overrightarrow K$ sont unitaires et orthogonaux. On en déduit que la base $(\overrightarrow I, \overrightarrow J, \overrightarrow K)$ est orthonormée directe si et seulement si $(\overrightarrow J, \overrightarrow K , \overrightarrow I)$ est orthonormée directe si et seulement si 
\begin{displaymath}
 \overrightarrow I = \overrightarrow J \wedge \overrightarrow K
\end{displaymath}
On calcule en coordonnées:
\begin{displaymath}
\frac{1}{\sqrt{2}}
\begin{pmatrix}
 1\\-1\\0
\end{pmatrix}
 \wedge
\frac{1}{\sqrt{3}}
\begin{pmatrix}
 1 \\1\\1
\end{pmatrix}
=
\frac{1}{\sqrt{6}}
\begin{pmatrix}
 -1 \\ -1 \\ 2
\end{pmatrix}
\end{displaymath}
On en déduit
\begin{displaymath}
 \overrightarrow I =
\frac{1}{\sqrt{6}}
\left( 
-\overrightarrow i - \overrightarrow j + 2\overrightarrow k
\right) 
\end{displaymath}
\item Pour tout point $M$:
\begin{displaymath}
 \overrightarrow{OM}=X(M)\overrightarrow I + Y(M)\overrightarrow J + Z(M)\overrightarrow K 
\end{displaymath}
Comme la base $(\overrightarrow I, \overrightarrow J, \overrightarrow K)$ est orthonormée directe, on en déduit
\begin{displaymath}
 \left\lbrace 
\begin{aligned}
 X &= (\overrightarrow{OM}/\overrightarrow I)\\
 Y &= (\overrightarrow{OM}/\overrightarrow J)\\
 Z &= (\overrightarrow{OM}/\overrightarrow K)
\end{aligned}
\right. 
\end{displaymath}
Comme on connait toutes les coordonnées, on peut calculer:
\begin{displaymath}
 \left\lbrace 
\begin{aligned}
 X(M) &= \frac{1}{\sqrt{6}}(-x-y+2z)\\
 Y(M) &= \frac{1}{\sqrt{2}}(x-y)\\
 Z(M) &= \frac{1}{\sqrt{3}}(x+y+z)
\end{aligned}
\right. 
\end{displaymath}

 \item Une des équations fait intervenir le carré de la distance que l'on peut exprimer dans l'un ou l'autre des deux repères orthonormés
\begin{displaymath}
 x(M)^2+y(M)^2+z(M)^2
= \Vert \overrightarrow{OM}\Vert
= X(M)^2+Y(M)^2+Z(M)^2
\end{displaymath}
On en déduit 
\begin{displaymath}
 M\in \mathcal{C}\Leftrightarrow
\left\lbrace
\begin{aligned}
 Z(M) &= \sqrt{3} \\
 X(M)^2 + Y(M)^2 &= 2
\end{aligned}
\right. 
\end{displaymath}
On en tire que le cercle est de rayon $\sqrt{2}$. Soit $C$ son centre, les coordonnées de $C$ dans le repère $\mathcal{R}'$ sont évidentes, on en déduit les coordonnées dans le premier repère
\begin{displaymath}
 \left\lbrace 
\begin{aligned}
 X(C) &= 0\\
 Y(C) &= 0\\
 Z(C) &= \sqrt{3}
\end{aligned}
\right. 
\Leftrightarrow
 \left\lbrace 
\begin{aligned}
 -x(C)-y(C) +2z(C)&= 0\\
 x(C)-y(C) &= 0\\
 x(C)+y(C)+z(C) &= 3
\end{aligned}
\right. 
\Leftrightarrow
x(C)=y(C)=z(C)=1
\end{displaymath}

\end{enumerate}

\subsection*{Exercice 4.}
\begin{enumerate}
 \item \`A partir des équations, on obtient facilement une définition paramétrique de $\mathcal{D}$.
\begin{displaymath}
 M\in \mathcal{D}\Leftrightarrow
\begin{pmatrix}
 x(M)\\y(M)\\z(M)
\end{pmatrix}
=
\begin{pmatrix}
 2\\1\\0
\end{pmatrix}
+z(M)
\begin{pmatrix}
 1\\-1\\1
\end{pmatrix}
\end{displaymath}
On en déduit
\begin{displaymath}
 \mathcal D = A +\Vect(\overrightarrow u) \text{ avec }
A :
\begin{pmatrix}
 2\\1\\0
\end{pmatrix}
\;
\overrightarrow u : 
\begin{pmatrix}
 1\\-1\\1
\end{pmatrix}
\end{displaymath}
Cela permet d'exprimer le projeté $H$ de $M$ sur $\mathcal{D}$ à l'aide de la formule de cours
\begin{displaymath}
 \overrightarrow{AH}=
\frac{(\overrightarrow{AM}/\overrightarrow u)}{\Vert \overrightarrow u \Vert}\overrightarrow u
\end{displaymath}
puis le symétrique de $M$ :
\begin{displaymath}
 s(M) = M + 2\overrightarrow{MH}
= M + 2\overrightarrow{MA} + 2\overrightarrow{AH}
\end{displaymath}
Ce qui donne en coordonnées :
\begin{multline*}
\begin{pmatrix}
 x(s(M))\\y(s(M))\\z(s(M))
\end{pmatrix}
=
\begin{pmatrix}
 x(M)\\y(M)\\z(M)
\end{pmatrix}
+2\begin{pmatrix}
 2-x(M)\\1-y(M)\\-z(M)
\end{pmatrix}
+\frac{2}{3}\left(x(M)-y(M)+z(M) \right)
\begin{pmatrix}
 1\\-1\\1
\end{pmatrix}
 \\
=
\frac{1}{3}
\begin{pmatrix}
 10 -x(M) -2 y(M) +2z(M)\\
 8 -2x(M)-y(M) -2z(M)\\
 -2 + 2x(M) -2y(M) - z(M)
\end{pmatrix}
\end{multline*}

 \item Les calculs précédents permettent de former l'équation du plan $\mathcal{P}'$ symétrique de $\mathcal{P}$ par rapport à $\mathcal{D}$.
\begin{displaymath}
 M\in \mathcal P'\Leftrightarrow s(M) \in \mathcal{P}
\Leftrightarrow
x(s(M))-y(s(M))-3z(s(M))=1
\end{displaymath}
Après calculs :
\begin{displaymath}
 M\in \mathcal P'\Leftrightarrow
-5x(M)+5y(M)+7z(M)+5 = 0
\end{displaymath}
\end{enumerate}
