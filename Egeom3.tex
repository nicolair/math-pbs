%<dscrpt>Forme complexe d'une équation de droite dans un plan.</dscrpt>
Soit $a$, $b$, $c$, $z^\prime$ quatre nombres complexes. On considère l'équation $(E)$ d'inconnue $z$
\begin{align*}
 az+b\overline{z}+c = z^\prime & & (E)
\end{align*}
\begin{enumerate}
 \item \begin{enumerate}
 \item Montrer que $z$ est solution de $(E)$ si et seulement si
\begin{displaymath}
% use packages: array
\left\lbrace 
\begin{array}{lll}
 az+b\overline{z} &=& z^\prime -c \\
 \overline{b}z + \overline{a}\overline{z} &=& \overline{z^\prime} - \overline{c}
\end{array}
\right. 
\end{displaymath}
\item Résoudre $(E)$ dans le cas $|a|\neq |b|$.
\end{enumerate}
\item On suppose dans cette question $|a|= |b|\neq 0$ et on pose alors
\begin{align*}
 a=re^{i\alpha} &, & b=re^{i\beta}
\end{align*}
avec $\alpha$ et $\beta$ réels et $r$ réel strictement positif.
\begin{enumerate}
 \item Montrer que si $(E)$ admet des solutions alors le point $M^\prime$ d'affixe $z^\prime$ est sur la droite (notée $D^\prime$) passant par $C$ et orthogonale à $(AB)$. \newline(on pourra introduire $\frac{\alpha + \beta}{2}$)
 \item Montrer que si $M^\prime$ d'affixe $z^\prime$ est sur la droite $D^\prime$, alors l'ensemble des points $M$ d'affixe $z$ tels que $z$ soit solution de $(E)$ est une droite (notée $D_{z^\prime}$). Préciser un vecteur directeur de $D_{z^\prime}$ et l'angle $(D^\prime,D_{z^\prime})$.
\end{enumerate}
\item On considère la fonction $f_{a,b,c}$ définie de $\C$ dans $\C$ par 
\begin{displaymath}
 f_{a,b,c}(z) = az + b\overline{z}+c
\end{displaymath}
\begin{enumerate}
 \item Discuter suivant $a$, $b$, $c$ des caractères injectif, surjectif, bijectif de $f_{a,b,c}$.
 \item \'Etudier les points invariants de $f_{a,b,c}$ c'est à dire les points $M$ dont l'affixe $z$ vérifie
\begin{displaymath}
 f_{a,b,c}(z)=z
\end{displaymath}

\end{enumerate}

\end{enumerate}
