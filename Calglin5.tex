
\begin{enumerate}
\item  Pour calculer le d{\'e}terminant de la matrice
\[
A-\lambda I_{n}=\left[
\begin{array}{cccc}
-\lambda  & 1 & \cdots  & 1 \\
1 & \ddots  &  & \vdots  \\
\vdots  &  & \ddots  & 1 \\
1 & \cdots  & 1 & -\lambda
\end{array}
\right]
\]
on remarque que chaque colonne est la somme de la colonne fix{\'e}e
\[
C=\left[
\begin{array}{c}
1 \\
\vdots  \\
1
\end{array}
\right] \] et de  $C_{i}=-(\lambda +1)E_{i}$ o{\`u} $E_{i}$ est la
colonne dont tous les termes sont nuls sauf celui de la i
$\grave{e}me$ ligne qui vaut 1. En d{\'e}veloppant par
multilin{\'e}arit{\'e} on obtient une somme de 2$^{n}$
d{\'e}terminants. Tous ceux o{\`u} figure deux fois $C$ sont nuls;
il n'en reste plus que $n+1$.
\begin{eqnarray*}
\det (A-\lambda I_{n}) &=&\det (C_{1},\cdots ,C_{n}) +
\sum_{i=1}^{n}\det (C_{1},\cdots ,C,\cdots
,C_{n})\\
&=&(-1)^{n}(\lambda +1)^{n}+n(-1)^{n-1}(\lambda
+1)^{n-1} \\
&=&(-1)^{n}(\lambda +1)^{n-1}(\lambda -n+1)
\end{eqnarray*}
L'ensemble des racines de ce polyn{\^o}me est donc $\left\{
-1,n-1\right\} $

\item  La matrice $A$ est inversible car d'apr{\`e}s le calcul pr{\'e}c{\'e}%
dent
\[\det A=(-1)^{n}(1-n).\]

Pour calculer l'inverse de $A,$ on va exprimer les colonnes
$E_{i}$ en fonction des colonnes $C_{j}(A)$.

On remarque d'abord que $C_{i}=C-E_{i}$. D'autre part, en sommant
toutes les
colonnes, on obtient
\[C_{1}(A)+\cdots +C_{n}(A)=(n-1)C\]
d'o{\`u}
\[C=\frac{1}{n-1}(C_{1}(A)+\cdots +C_{n}(A))\]
 et
\begin{eqnarray*}
E_{i}&=&\frac{1}{n-1}C_{1}(A)+\cdots + \frac{1}{n-1}C_{i-1}(A)\\
&\phantom{f}& \quad +(\frac{1}{n-1}%
-1)C_{i}(A)+\frac{1}{n-1}C_{i+1}(A)+\cdots \frac{1}{n-1}C_{n}(A)
\end{eqnarray*}
On en d{\'e}duit
\[
A^{-1}=\frac{1}{n-1}\left[
\begin{array}{cccc}
2-n & 1 & \cdots  & 1 \\
1 & \ddots  &  & \vdots  \\
\vdots  &  & \ddots  & 1 \\
1 & \cdots  & 1 & 2-n
\end{array}
\right]
\]

\item  Remarquons que $A$ et $A^{-1}$ (d'apr{\`e}s la formule pr{\'e}c{\'e}%
dente) s'{\'e}crivent encore
\[
A^{-1}=\frac{1}{n-1}B-I,\quad A=B-I
\]
Ce qui est la formule propos{\'e}e par l'{\'e}nonc{\'e} pour $k=1$
ou $-1$. On pourrait tenter un raisonnement par r{\'e}currence.
J'utiliserai plutot la
formule du bin{\^o}me exploitant le fait que $A$ et $B$ commutent et que $%
B^{k}=n^{k-1}B$ pour $k$ entier naturel non nul.

Pour tout $k$ dans $\Bbb{N}^{*}$%
\[
A^{k}=(-1)^{k}I_{n}+\sum_{i=1}^{k}C_{k}^{i}(-I_{n})^{k-i}B^{i}=(-1)^{k}I_{n}+%
\underset{=\frac{1}{n}\left( (n-1)^{k}-(-1)^{k}\right) }{\underbrace{%
\sum_{i=1}^{k}C_{k}^{i}(-1)^{k-i}n^{i-1}}}B
\]
d'o{\`u} la formule demand{\'e}e pour $k\geq 0$.

Pour les puissances n{\'e}gatives posons $l=-k\in \Bbb{N}^{*}$%
\begin{eqnarray*}
A^{k}
&=&(-1)^{l}I_{n}+\sum_{i=1}^{l}C_{l}^{i}(-I_{n})^{l-i}(\frac{1}{n-1}B)^{i}\\
&=&(-1)^{l}I_{n}+\sum_{i=1}^{l}C_{l}^{i}(-1)^{l-i}\frac{n^{i-1}}{%
(n-1)^{i}}B
\end{eqnarray*}
\begin{eqnarray*}
\sum_{i=1}^{l}C_{l}^{i}(-1)^{l-i}\frac{n^{i-1}}{(n-1)^{i}}
 &=&\frac{1}{n}%
\left( (\frac{n}{n-1}-1)^{l}-(-1)^{l}\right)\\
&=&\frac{1}{n}\left( (n-1)^{k}-(-1)^{k}\right)
\end{eqnarray*}
\end{enumerate}
