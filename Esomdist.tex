%<dscrpt>Autour de la somme des distances d'un point variable à des points fixés.</dscrpt>
\begin{figure}
	\begin{center}
	\input{Esomdist_1.pdf_t}
	\end{center}
\caption{Ensemble $X$ dans un plan.}
\end{figure} 

Dans une espace de dimension 3, on considère un ensemble 
\begin{displaymath}
X = \left\lbrace P_1, P_2, \cdots ,P_n \right\rbrace  
\end{displaymath}
de $n\geq 3$ points fixés distincts et non alignés.\newline
Pour tout point $M$ de l'espace, on pose
\begin{displaymath}
 d(M) = \sum _{i=1}^{n} \Vert \overrightarrow{MP_i} \Vert
\end{displaymath}
et lorsque $M$ n'est pas dans $X$
\begin{displaymath}
 \overrightarrow{V}(M) = \sum _{i=1}^{n} \frac{1}{\Vert \overrightarrow{MP_i} \Vert} \overrightarrow{MP_i}
\end{displaymath}
lorsque $M$ n'est pas dans $X$ et $\overrightarrow v$ est un vecteur quelconque, on pose
\begin{displaymath}
 q_M(\overrightarrow v) =
 \sum _{i=1}^{n} \frac{\Vert \overrightarrow v\Vert ^2}{\Vert \overrightarrow{MP_i} \Vert} -
 \sum _{i=1}^{n} \frac{(\overrightarrow{MP_i} / \overrightarrow v)^2}{\Vert \overrightarrow{MP_i} \Vert ^3} 
\end{displaymath}

On dira que $d$ \emph{est minimale en un point} $A$ lorsque $d(A)\leq d(M)$ pour tous les points $M$ de l'espace.
\begin{enumerate}
 \item Montrer que 
\begin{displaymath}
q_M(\overrightarrow v) =
 \sum _{i=1}^{n} \frac{\Vert\overrightarrow{MP_i}\wedge \overrightarrow v\Vert ^2}{\Vert \overrightarrow{MP_i} \Vert ^3} 
\end{displaymath}


 \item \'Etant donné trois points $M$, $N$ et $P$, montrer l'inégalité
\begin{displaymath}
 \Vert \overrightarrow{NP} \Vert - \Vert \overrightarrow{MP} \Vert \leq 
\frac{(\overrightarrow{NM}/\overrightarrow{NP})}{\Vert \overrightarrow{NP} \Vert}
\end{displaymath}
\item Montrer que s'il existe un point $N$ tel que $\overrightarrow{V}(N) = \overrightarrow 0$ alors $d$ est minimale au point $N$.

\item S'il existe un entier $k$ entre $1$ et $n$ tel que
\begin{displaymath}
 \left \Vert \sum _{i\in \{1,\cdots n\}-\{k\}} \frac{1}{\Vert \overrightarrow{P_kP_i} \Vert} \overrightarrow{P_kP_i} \right \Vert \leq 1
\end{displaymath}
montrer que $d$ est minimale au point $P_k$.
\item Si $N$ est le milieu d'un segment $[M_0, M_1]$, établir l'inégalité
\begin{displaymath}
 2d(N) \leq d(M_0) + d(M_1)
\end{displaymath}
En déduire l'unicité du point où $d$ est minimale lorsqu'un tel point existe.
\item Trouver les points où $d$ est minimale dans les cas suivants
\begin{enumerate}
 \item $n=3$ et $X$ est formé par les trois sommets d'un triangle équilatéral.
 \item $n=3$ et $X$ est formé par les trois sommets d'un triangle dont l'un des angles est compris entre $\frac{2\pi}{3}$ et $\pi$.
 \item $n=8$ et $X$ est formé par les huit sommets d'un cube.
 \item $n=4$ et $X=\{P_1,P_2,P_3,P_4\}$ où les points $P_i$ sont définis par leurs coordonnées dans un repère orthonormé
\begin{align*}
 P_1:(0,0,0) &,& P_2:(1,0,0) &,& P_3:(0,1,0) &,& P_4:(x,y,z)
\end{align*}
où $x$, $y$, $z$ vérifient
\begin{eqnarray*}
  x^2 + y^2 +z^2 &=& 1 \\
  x +y &\leq& -1 
\end{eqnarray*}
(On pourra calculer 
$\Vert \overrightarrow{P_1 P_2}+ \overrightarrow{P_1 P_3}+\overrightarrow{P_1 P_4}\Vert$)
\end{enumerate}
\item On fixe un point $M$ et un vecteur $\overrightarrow u$ et on pose pour tout $t$ réel
\begin{align*}
 M_t = M+t\overrightarrow u &,& f(t)= d(M_t)
\end{align*}
Montrer que 
\begin{align*}
 f^\prime (t) = -(\overrightarrow u/ \overrightarrow V (M_t)) &,&
 f^{\prime \prime} (t) = q_{M_t}(\overrightarrow u)
\end{align*}
Que peut-on en déduire lorsque $d$ est minimale en un point $M$ qui n'est pas dans $X$ ?
\end{enumerate}
