%<dscrpt>Suites des solutions d'une famille d'équations.</dscrpt>
On d{\'e}finit, pour tout entier $n\geq 1$, une fonction $f_n$ de $\R $ dans $\R$ en posant
\[
\forall x\in \R,\quad f_n(x)=x^n+x^{n-1}+\cdots +x^2+x-1
\]
\begin{enumerate}
\item  Montrer qu'il existe un unique r{\'e}el $a_{n}$ strictement positif tel que $f_{n}(a_{n})=0$.

\item  Montrer que $(a_{n})_{n\in \N^{*}}$ est monotone, en d{\'e}duire sa convergence.

\item  Montrer que $a_{2}\in \left] 0,1\right[ $. En d{\'e}duire la convergence et la limite de
\[(a_{n}^{n+1})_{n\in \N^{*}}\]
puis la limite $l$ de $(a_{n})_{n\in \N^{*}}$.\newline 
On pourra montrer que $a_{n}=\frac{1}{2}(1+a_{n}^{n+1})$.

\item  Pr{\'e}ciser, suivant $x \in ]0, 1[$ et $x \neq \frac{1}{2}$, la limite de $(f_{n}(x))_{n\in \N^{*}}$. En d{\'e}duire directement,
sans utiliser 2 la convergence et la limite $l$ de $(a_{n})_{n\in \N^{*}}$.\newline
Pour tout $\varepsilon >0$, on pourra considérer les suites $\left( f_n(\frac{1}{2}-\varepsilon)\right)_{n\in \N^*}$ et $\left( f_n(\frac{1}{2}+\varepsilon)\right)_{n\in \N^*}$. 

\item  Trouver un {\'e}quivalent simple {\`a} la suite $(a_{n}-l)_{n \in \N^*}$.\newline
On pourra {\'e}tudier d'abord la limite de $((2a_{n})^{n+1})_{n\in \N^{*}}$.
\end{enumerate}
