\begin{enumerate}
 \item
\begin{enumerate}
 \item On étudie la section par le plan d'équation $x=b$.\newline
Si $b=0$, cette section est la droite $Oz$ d'équations $x=0$ et $y=0$.
Si $b\neq 0$ cette section est une hyperbole dont un système d'équations est :
\begin{displaymath}
 x=b \text{ et } \frac{y^2}{4b}-\frac{z^2}{4}=1
\end{displaymath}
 \item La deuxième équation dans le système de la question a. est une équation réduite de l'hyperbole. L'axe focal de l'hyperbole est la droite dont un système d'équations est $x=b$ et $z=0$. Le centre est le point de coordonnées $(b,0,0)$.\newline
Les sommets sont les points de coordonnées $(b,2b,0)$ et $(b,-2b,0)$.\newline
Lorsque $b$ décrit $\R^*$, ils forment l'union des deux droites passant par $O$ et dirigées par $\overrightarrow i + 2 \overrightarrow j$ et $\overrightarrow i - 2 \overrightarrow j$.\newline
La distance centre-foyer est donné par $c^2 = 4b^2+4$. Les foyers sont donc les points de coordonnées

\begin{displaymath}
 (b,2\sqrt{b^2+1},0)\hspace{0.5cm}(b,-2\sqrt{b^2+1},0)
\end{displaymath}
Lorsque $b$ décrit $\R^*$, ils forment une hyperbole d'équations 
\begin{displaymath}
 \left\lbrace 
\begin{aligned}
 &z=0\\
&\frac{y^2}{4}-x^2=1
\end{aligned}
\right.  
\end{displaymath}
\end{enumerate}
 
 \item On note $\Gamma_c$ la section par le plan d'équation $y=c$ (avec $c\neq0$). L'équation de $\Gamma_c$ est
\begin{displaymath}
 c^2 = x^2(z^2+4)
\end{displaymath}

 \item
\begin{enumerate}
 \item Soit $M(t)$ le point de coordonnées 
\begin{displaymath}
 (u(t)=\frac{c}{2}\cos t,c, v(t) = 2\tan t)
\end{displaymath}
Alors:
\begin{displaymath}
 u(t)^2(v(t)^2+4)=\frac{c^2}{4}\cos^2t\times 4(1+\tan^2t)=c^2 \Rightarrow M(t)\in \Gamma_c
\end{displaymath}
Réciproquement, soit $M\in \Gamma_c$ de coordonnées $(x(M),y(M))$. On doit avoir $x(M)\neq 0$.\newline
Si $x(M)>0$, il existe $t\in]-\frac{\pi}{2},\frac{\pi}{2}[$ tel que $y(M)=2\tan t$. Alors
\begin{displaymath}
 x^2(M)=\frac{c^2}{4(1+\tan^2t)}=\frac{c^2}{4}\cos^2 t \Rightarrow x(M)=\frac{c}{2}\cos t
\end{displaymath}
car $x(M)$ et $\cos t$ sont strictement positifs donc $M=M(t)$.\newline
Lorsque $x(M)<0$, il existe un $t\in ]\frac{\pi}{2},3\frac{\pi}{2}[$ tel que $y(M)=2\tan t$ et on montre de même que $M=M(t)$.
 \item Pour calculer la courbure, on commence par calculer les dérivées
\begin{multline*}
 \begin{aligned}
  &u(t)=\frac{c}{2}\cos t & v(t)=2\tan t \\
  &u'(t) = -\frac{c}{2}\sin t & v(t)=\frac{2}{\cos^2 t} \\
  &u''(t) = -\frac{c}{2}\cos t & v(t)=\frac{4\sin t}{\cos^3 t} \\
 \end{aligned} \\
\Rightarrow
\Delta(t)=-2c\frac{\sin^2t}{\cos^3t}+\frac{c}{\cos t}
= \frac{c}{\cos^3 t}(1-3\sin^2t)
\end{multline*}

 \item  Il existe quatre points d'inflexion pour $\Gamma_c$. Ils correspondent aux $t$ tels que $3\sin^2t =1$. Lorsque $c$ décrit $\R^*$ il forment les quatre droites déterminées paramétriquement par
\begin{displaymath}
 (\pm \frac{c}{\sqrt{6}},c,\pm \sqrt{2})
\end{displaymath}

\end{enumerate}

\end{enumerate}
