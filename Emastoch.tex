%<dscrpt>Matrices stochastiques</dscrpt>

\subsection*{Partie 1: Un exemple en dimension 2}

Soient $\alpha, \beta \in ]0,1[$. On note:
\[ A(\alpha, \beta) = \begin{pmatrix}
                       1-\alpha & \alpha \\
                       \beta & 1-\beta
                      \end{pmatrix}.\]
On note $\lambda = 1-\alpha -\beta$. 


\begin{enumerate}
\item \begin{enumerate}
  \item Calculer $A\begin{pmatrix} 1 \\ 1 \end{pmatrix}$ et $A\begin{pmatrix} -\alpha \\ \beta \end{pmatrix}$. 

  \item Expliciter une matrice $P\in \GL_{2}(\R)$ telle que:
\[
P^{-1}AP = \begin{pmatrix}  1 & 0\\ 0 & \lambda \end{pmatrix}.
\]
Déterminer également $P^{-1}$.

  \item Montrer que pour tout $n\in \N$:
\[ 
A^{n} = \frac{1}{\alpha  + \beta} 
  \begin{pmatrix}
      \beta + \alpha \lambda^{n} & \alpha (1-\lambda^{n})\\
      \beta (1-\lambda^{n}) & \alpha + \beta \lambda^{n}
   \end{pmatrix}.
\]

\end{enumerate}


\item Soit $(\Omega, \P)$ un espace probabilisé et $n+1$ variables aléatoires $X_{0}, ..., X_{n}$ définies dans $\Omega$ et à valeurs dans $\{ 0, 1\}$. On suppose que pour tout $k\in \llbracket 0, n-1\rrbracket$:
\[ \P(X_{k+1} = 1|X_{k} = 0) = \alpha \quad \text{ et } \quad \P(X_{k+1} = 0|X_{k} = 1) = \beta.\]

\begin{enumerate}
 \item Montrer que pour tout $k\in \llbracket 0, n-1\rrbracket$:
 \[ \begin{pmatrix}
     \P(X_{k+1} = 0)\\
     \P(X_{k+1} = 1)
    \end{pmatrix} = A(\alpha, \beta)^{T}\begin{pmatrix}
                                                 \P(X_{k} = 0)\\
                                                 \P(X_{k} = 1)
                                        \end{pmatrix}.\]
  \item Montrer que:
  \[ \P(X_{n} = 0) = \frac{\beta + \alpha \lambda^{n}}{\alpha + \beta}\P(X_{0} = 0) + \frac{\alpha(1- \lambda^{n})}{\alpha + \beta}\P(X_{n} = 1)\]
   \[ \P(X_{n} = 0) = \frac{\beta (1 -  \lambda^{n})}{\alpha + \beta}\P(X_{0} = 0) + \frac{\alpha  + \beta \lambda^{n}}{\alpha + \beta}\P(X_{n} = 1)\]
  \item Déterminer $\P(X_{n} = X_{0})$. 
  \item Posons $\displaystyle{r = \min \left ( \frac{\alpha}{\alpha + \beta},\ \frac{\beta}{\alpha  + \beta}\right )}$. A l'aide d'une étude de la fonction 
  
  $\Phi (x) = x + (1-x)\lambda^{n}$, montrer que:
  \[ \P(X_{n} = X_{0}) \geq r + (1-r)(1-\alpha - \beta)^{n}.\]
\end{enumerate}




\item Soit $l\in \N^{*}$. Considérons une famille $(X_{k}^{i})_{\empil{0\leq k\leq n}{1\leq i\leq l}}$ de variables aléatoires  définies sur $\Omega$ à valeurs dans $\{ 0, 1\}$ telles que pour tout
$k\in \llbracket 0, n\rrbracket$, les variables aléatoires $X_{k}^{1}, ..., X_{k}^{l}$ soient mutuellement indépendantes. On suppose que pour tout $i\in \llbracket 1, l\rrbracket$ et pour tout 
$k\in \llbracket P, n-1\rrbracket$:
\[ \P(X_{k+1}^{i} = 1|X_{k}^{i} = 0) = \alpha \quad \text{ et } \quad \P(X_{k+1}^{i} = 0|X_{k}^{i} = 1) = \beta.\]
On note $Q_{n}$ la probabilité de l'événement $\{ \forall i\in \llbracket 1, l\rrbracket,\  X_{n}^{i} = X_{0}^{i} \}$. Monter que:
\[ Q_{n} \geq \left [ r + (1-r)(1-\alpha - \beta)^{n}\right ]^{l}.\]




\item On souhaite transmettre un message constitué de $l$ bits $(a_{1}, ..., a_{l})\in \{ 0, 1\}^{l}$ à l'aide d'un réseau de communication constitué de $n$ relais numérotés de $1$ à $n$.
Pour être transmis, le message doit passer successivement par les $n$ relais. Au passage de chaque relais, chaque bit du message a une probilité $\alpha$ d'être modifié. On suppose que les relais agissent 
indépendamment les uns des autres et indépendamment sur chaque bit.

Soit $\varepsilon >0$. Déterminer un entier $n_{c}$ tel que pour tout $n\geq n_{c}$, la probabilité pour que le message soit erroné après le passage du $n$-ème relais soit supérieure ou égale à $\varepsilon$. 

\end{enumerate}




\subsection*{Partie 2: Spectre d'une matrice stochastique}


Soit $n\geq 2$, soit $A=(a_{i,j})_{1\leq i,j\leq n}\in \M_{n}(\R)$. On dit que $A$ est une matrice stochastique (respectivement strictement stochastique) si tous ses coefficients $a_{i,j}$ sont positifs ou nuls (respectivement strictement positifs) et si:
\[\forall i\in \llbracket 1, n\rrbracket ,\ \sum_{j=1}^{n}a_{i,j} = 1.\]
Dans toute cette partie on notera $U\in \M_{n,1}(\R)$ la matrice colonne dont tous les coefficients valent $1$.



\begin{enumerate}


 \item Soit $A = (a_{i,j})_{1\leq i,j\leq n}$ une matrice stochastique (respectivement strictement stochastique). Montrer que pour tout $(i,j)\in \llbracket 1, n\rrbracket^{2}$:
 \[ 0\leq a_{i,j}\leq 1 \quad (\text{  respectivement } 0 < a_{i,j} < 1).\]
 
 
 
 \item Montrer qu'une matrice $A\in \M_{n}(\R)$ à coefficients positifs ou nuls est stochastique si et seulement si $AU = U$. 
 
 
 
 
 
 \item Montrer que le produit de deux matrices stochastiques (respectivement stochastiques) est stochastique (respectivement stochastique). 
 
 
 
 
 
 \item Soit $A = (a_{i,j})_{1\leq i,j\leq n}\in \M_{n}(\C)$, soit $\lambda \in \C$. On suppose qu'il existe $X\in \M_{n,1}(\C)\setminus \{ 0 \}$ telle que $AX = \lambda X$. Montrer qu'il existe 
 $i\in \llbracket 1, n\rrbracket$ tel que:
 \[ \abs{\lambda -a_{i,i}}\leq \sum_{\empil{j=1}{j\neq i}}^{n}\abs{a_{i,j}}.\]
 
  
 
 \item Soit $A\in \M_{n}(\R)$ une matrice stochastique, soit $\lambda \in \C$. On suppose qu'il existe 
 
 $X\in \M_{n,1}(\C)\setminus \{ 0 \}$ tel que $AX = \lambda X$. 
 \begin{enumerate}
  \item Montrer que $\abs{\lambda}\leq 1$.  
 \item Supposons $A$ strictement stochastique. Montrer que si $\lambda \neq 1$, alors $\abs{\lambda} < 1$. 
 \end{enumerate}

 
 
 
 \item Soit $A = (a_{i,j})_{1\leq i,j\leq n}\in \M_{n}(\C)$. On dit que $A$ est à diagonale strictement dominante si:
 \[ \forall i\in \llbracket 1, n\rrbracket,\ \abs{a_{i,i}} > \sum_{\empil{j=1}{j\neq i}}^{n}\abs{a_{i,j}}.\]
 Montrer qu'une matrice à diagonale strictement dominante est inversible. 
 
 

 
 \item Soit $A\in \M_{n}(\R)$ une matrice strictement stochastique. 
 \begin{enumerate}
  \item Notons $A_{1} = (a_{i,j})_{1\leq i,j\leq n-1}\in \M_{n-1}(\R)$ la matrice obtenue en supprimant la dernière ligne et la dernière colonne de $A$. Montrer que $A_{1}-I_{n-1}$ est à diagonale strictement
  dominante. 
  \item Montrer que $\Ker(A-I_{n})$ est de dimension $1$. 
 \end{enumerate}
 
 

 


\end{enumerate}



