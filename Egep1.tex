%<dscrpt>Géométrie plane.</dscrpt>
Dans un plan muni d'un rep{\`e}re orthonorm{\'e}
$(O,\overrightarrow{i},\overrightarrow{j})$, pour tout $m>0$, on
note $D_m$ la droite d'{\'e}quation $y=mx$. Soit $M$ un point du plan,
$A$ est sa projection sur l'axe des $x$ et $B$ sa projection sur
l'axe des $y$. Par $A$ et $B$ on m{\`e}ne les parall{\`e}les {\`a} $D_m$ qui
coupent respectivement l'axe des $y$ en $A'$ et l'axe des $x$ en
$B'$.On appelle $M'$ le point qui se projette en $A'$ et $B'$ sur
les axes. On note $\mathcal{T}_m$ l'application du plan dans lui
m{\^e}me qui {\`a} $M$ associe $M'$.
\begin{center}
\includegraphics[width=10cm,height=8cm]{Egep1f1.pdf}
\end{center}
\begin{enumerate}
  \item Calculer les coordonn{\'e}es $(\alpha ',\beta ')$ de $M'$ en
  fonction des coordonn{\'e}es $(\alpha,\beta)$ de $M$. Montrer que
  $\mathcal{T}_m$ est bijective et calculer $\mathcal{T}_m ^{-1}$
  \item Montrer que la droite $(M\mathcal{T}_m(M))$ quand elle est
  d{\'e}finie a une direction ind{\'e}pendante de $M$ et que le milieu $P$
  de $([M\mathcal{T}_m(M)]$ d{\'e}crit une droite $\delta_m$ que l'on
  d{\'e}terminera. Quelle est la nature de $\mathcal{T}_m$ ? Peut-elle
  {\^e}tre une sym{\'e}trie orthogonale ?
  \item Le r{\'e}el $m$ {\'e}tant fix{\'e}, on suppose que $M$ d{\'e}crit une
  droite passant par $O$.
    \begin{enumerate}
      \item Que peut-on dire de l'ensemble des droites $(AB)$ ?
      \item Que peut-on dire de l'ensemble des droites $(A'B')$ ?
    \end{enumerate}

  \item On suppose que $M$ est fix{\'e} et que $m$ d{\'e}crit l'ensemble
  des r{\'e}els strictement positifs.
    \begin{enumerate}
      \item D{\'e}terminer l'ensemble
      \[\mathcal{H}=\{\mathcal{T}_m(M), m>0\}\]
      \item On suppose que $M$ n'est pas situ{\'e} sur les axes.
      Montrer que la tangente en $\mathcal{T}_m(M)$ {\`a}
      $\mathcal{H}$ est l'image de la droite $(A'B')$ par une
      homoth{\'e}tie de centre $O$ que l'on pr{\'e}cisera.
    \end{enumerate}

  \item Soit $(m_n)_{n\in \Bbb{N}}$ une suite infinie de r{\'e}els non
  nuls et $M$ un point fix{\'e} du plan de coordonn{\'e}es $(\alpha,\beta)$ avec $\alpha \beta \neq
  0$.Pour tout entier $n$, on pose
  \[M_n=\mathcal{T}_{m_n}(M)\]
  Trouver une condition n{\'e}cessaire et suffisante sur la suite $(m_n)_{n\in
  \Bbb{N}}$ pour que les deux suites de coordonn{\'e}es de $M_n$
  convergent.
\end{enumerate}
