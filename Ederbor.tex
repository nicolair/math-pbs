%<dscrpt>Fonctions à dérivées bornées.</dscrpt>
L'objet de ce problème \footnote{d'après Centrale-Supélec 2001 PC Maths 1} est de montrer le résultat suivant.
\begin{quote}
 Lorsque $f\in\mathcal C^n(\R)$ est telle que $f$ et $f^{(n)}$ soient bornées alors, pour tous les $k$ entre $1$ et $n-1$, la fonction $f^{(k)}$ est bornée.
\end{quote}
Pour cela, on introduit des matrices et on utilisera un résultat relatif à ces matrices.\newline
Soit $m$ un entier non nul, on définit une matrice carrée $V_m\in \mathcal M_m(\R)$  et une matrice ligne $L_m\in\mathcal M_{1,m}(\R)$ par :
\begin{displaymath}
 V_m = 
\begin{bmatrix}
 1 & 1^2 & \cdots & 1^m \\
 2 & 2^2 & \cdots & 2^m \\
 \vdots & \vdots & \vdots & \vdots \\
 m & m^2 & \cdots & m^m
\end{bmatrix}
\end{displaymath}
\begin{displaymath}
L_m =
\begin{bmatrix}
 (-1)^{m-1}\dbinom{m}{1} & \cdots & (-1)^{m-k}\dbinom{m}{k} & \cdots & (-1)^0\dbinom{m}{m}
\end{bmatrix} 
\end{displaymath}
On se propose de montrer \emph{de deux manières différentes }que :
\begin{displaymath}
 L_m V_m =
\begin{bmatrix}
 0 & 0 & \cdots & 0 & m!
\end{bmatrix}
\end{displaymath}

\subsection*{Partie I}
Soit $E$ le $\R$-espace vectoriel des polynômes à coefficients réels de degré inférieur ou égal à $m$ (y compris le polynôme nul) et divisibles par $X$.
\begin{enumerate}
 \item Montrer que, pour tout $i$ entre $1$ et $m$, il existe un unique polynôme $\Lambda_i\in E$ tel que :
\begin{displaymath}
 \forall j\in \{1,\cdots, m\} : \widetilde{\Lambda_i}(j) = \delta_{i,j}
\end{displaymath}
\item Préciser, pour $i$ entre $1$ et $m$, le coefficient dominant de $L_i$.
\item Montrer que $\mathcal L = (\Lambda_1,\cdots,\Lambda_m)$ est une base de $E$. Soit $P\in E$, préciser la matrice $\Mat_\mathcal L P$ des coordonnées. Que peut-on en déduire pour $V_m$ ?
\item Montrer que l'application $\varphi$
\begin{displaymath}
 \varphi :
\left\lbrace 
\begin{aligned}
 E &\rightarrow \R \\
 P &\rightarrow \widetilde{P^{(m)}}(0)
\end{aligned}
\right. 
\end{displaymath}
est une forme linéaire. Préciser $\Mat_\mathcal L (\varphi)$.
\item Montrer sans calcul que $V_m$ est inversible. Quel renseignement la question 2. nous donne-t-elle sur $V_m^{-1}$ ?
\item Démontrer la formule annoncée :
\begin{displaymath}
 L_m V_m =
\begin{bmatrix}
 0 & 0 & \cdots & 0 & m!
\end{bmatrix}
\end{displaymath}
\end{enumerate}

\subsection*{Partie II}
Dans cette partie $m$ est un entier naturel non nul. Par développement limité à l'ordre $m$ en $0$ on entend un développement limité dont le reste est $o(x^m)$.
\begin{enumerate}
 \item Soit $k\in\{1,\cdots,m\}$. Former le développement limité à l'ordre $m$ en $0$ de
\begin{displaymath}
 x \rightarrow e^{kx}
\end{displaymath}
\item Former le développement limité à l'ordre $m$ en $0$ de
\begin{displaymath}
 x \rightarrow (e^{x}-1)^m
\end{displaymath}
\item Soit $j$ un entier entre $1$ et $m$. \'Ecrire le terme $1,j$ du produit matriciel $L_m V_m$.
\item Démontrer la formule annoncée :
\begin{displaymath}
 L_m V_m =
\begin{bmatrix}
 0 & 0 & \cdots & 0 & m!
\end{bmatrix}
\end{displaymath}
\end{enumerate}

\subsection*{Partie III}
\begin{enumerate}
 \item Soit $f\in\mathcal C^2(\R)$ telle que $|f|$ est bornée par un réel $M_0$ et $|f^{(2)}|$ bornée par un réel $M_2$.
\begin{enumerate}
\item Soit $x$ un réel quelconque et $h$ un réel strictement positif quelconque. \'Ecrire les formules de Taylor avec reste de Lagrange à l'ordre deux entre $x$ et $x+h$ puis entre $x$ et $x-h$.
\item En déduire :
\begin{displaymath}
 \forall x\in \R, \forall h> 0 : |f'(x)|\leq
\dfrac{M_0}{h}+\dfrac{M_2}{2}h
\end{displaymath}
\item En déduire que $|f'|$ est bornée par
\begin{displaymath}
 \sqrt{2M_0M_2}
\end{displaymath}
\end{enumerate}

\item Soit $f\in\mathcal C^3(\R)$ telle que $|f|$ est bornée par un réel $M_0$ et $|f^{(3)}|$ bornée par un réel $M_3$.
\begin{enumerate}
\item Soit $x$ un réel quelconque et $h$ un réel strictement positif quelconque. \'Ecrire les formules de Taylor avec reste de Lagrange à l'ordre trois entre $x$ et $x+h$ puis entre $x$ et $x-h$.
\item En déduire que $|f'|$ est bornée par
\begin{displaymath}
 \dfrac{1}{2}\left(9M_0^2M_3 \right)^{\frac{1}{3}} 
\end{displaymath}
\item La fonction $|f''|$ est-elle bornée ?
\end{enumerate}
\end{enumerate}

\subsection*{Partie IV}
Soit $f\in \mathcal C^n(\R)$, on suppose $|f|$ bornée par $M_0$ et $|f^{(n)}|$ bornée par $M_n$. On définit aussi $K_h$ par :
\begin{displaymath}
 K_h = 2M_0 + \dfrac{((n-1)h)^n}{n!}M_n
\end{displaymath}
Soit $x$ un réel quelconque et $h$ un réel strictement positif quelconque. On définit des réels $y_1,y_2,\cdots,y_{n-1}$ par :
\begin{displaymath}
 \begin{bmatrix}
  y_1 \\ y_2 \\ \vdots \\ y_{n-1}
 \end{bmatrix}
= V_{n-1}\,
\begin{bmatrix}
 \dfrac{h}{1!}f'(x)\vspace{5pt} \\ \dfrac{h^2}{2!}f''(x) \vspace{5pt}\\ \vdots \vspace{5pt} \\ \dfrac{h^{n-1}}{(n-1)!}f^{(n-1)}(x)
\end{bmatrix}
\end{displaymath}
\begin{enumerate}
 \item Montrer que,
\begin{displaymath}
 \forall i \in\{1,\cdots n-1\} : |y_i|\leq K_h
\end{displaymath}
\item Montrer que
\begin{displaymath}
 \sum_{i=1}^{n-1}(-1)^{n-1-i}\dbinom{n-1}{i}y_i = h^{n-1}f^{(n-1)}(x)
\end{displaymath}
\item Montrer que $|f^{(n-1)}|$ est bornée par
\begin{displaymath}
 \left( \dfrac{2}{h}\right)^{n-1}K_h
\end{displaymath}
\item En déduire que pour tout entier $k$ entre $1$ et $n-1$ la fonction $|f^{(k)}|$ est bornée.
\end{enumerate}


