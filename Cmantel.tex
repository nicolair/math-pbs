Question préliminaire.\newline
Il s'agit d'une question de cours portant sur la démonstration de l'inégalité de Cauchy-Schwarz.\newline
On définit une fonction $\varphi$ de $\R$ dans $\R$ par :
\begin{displaymath}
 \forall t\in \R :
\varphi(t) = \sum_{i=1}^{n}\left( a_i + t b_i\right)^2 
\end{displaymath}
Sous cette forme, il est bien clair que $\varphi$ ne prend que des valeurs positives ou nulles. D'autre part, $\varphi$ est en fait polynomiale du second degré. Après développement :
\begin{displaymath}
 \forall t\in \R :
\varphi(t) = \left(\sum_{i=1}^{n} {a_i}^2\right)t^2 + 2t \left(\sum_{i=1}^{n} a_i b_i\right) + 
\left(\sum_{i=1}^{n} {b_i}^2\right) 
\end{displaymath}
 Le discriminant de cette expression du second degré doit être négatif ou nul pour qu'elle reste toujours positive ou nulle. Cela conduit à l'inégalité demandée.
\subsection*{Partie I. Graphes orientés}
\begin{enumerate}
 \item 
\begin{enumerate}
 \item En examinant la figure, on remplit facilement le tableau suivant :
\begin{center}
\renewcommand{\arraystretch}{1.3}
\begin{tabular}{|c|c|c|c|c|}\hline
$s$ & $V_+(s)$    & $d_+(s)$ & $V_-(s)$    & $d_-(s)$\\ \hline
$1$ & $\{3\}$     & $1$      & $\{6\}$     & $1$     \\ \hline
$2$ & $\emptyset$ & $0$      & $\emptyset$ & $0$     \\ \hline
$3$ & $\emptyset$ & $0$      & $\{1\}$     & $1$     \\ \hline
$4$ & $\{5\}$     & $1$      & $\{5\}$     & $1$     \\ \hline
$5$ & $\{4\}$     & $1$      & $\{4\}$     & $1$     \\ \hline
$6$ & $\{1\}$     & $1$      & $\emptyset$ & $0$     \\ \hline
\end{tabular}
\end{center}

 \item On déduit du tableau précédent :
\begin{align*}
 \mathcal{S}_-=\{1,3,4,5\} & & \mathcal{S}_+=\{1,4,5,6\} & & \mathcal{S}=\{1,3,4,5,6\} 
\end{align*}
 
 \item Le graphe n'est pas conservatif car, par exemple, $d_-(6)=0$ et $d_+(6)=1)$.
\end{enumerate}

 \item 
\begin{enumerate}
 \item Il est bien clair que les deux ensembles proposés sont des ensembles d'arêtes. Ils sont donc inclus dans $\mathcal{A}$.\newline
Réciproquement, pour toute arête $a\in \mathcal A$, il existe des sommets $s_1$ et $s_2$ tels que $a=(s_1,s_2)$. Avec les définitions données au début, $s_1$ est un sommet initial donc dans $\mathcal S_+$ avec $s_2\in V_+(s_1)$ et 
\begin{displaymath}
(s_1,s_2)\in \left\lbrace (s_1,s'),s'\in V_+(s_1)\right\rbrace
\Rightarrow
a=(s_1,s_2)\in \bigcup_{s\in \mathcal S_+} \left\lbrace (s,s'),s'\in V_+(s)\right\rbrace
\end{displaymath}
De même, $s_2$ est un sommet final donc dans $S_-$ avec $s_1\in V_-(s_2)$ et
\begin{displaymath}
(s_1,s_2)\in \left\lbrace (s,s_2),s\in V_-(s_2)\right\rbrace
\Rightarrow
a=(s_1,s_2)\in \bigcup_{s'\in \mathcal S_-} \left\lbrace (s,s'),s'\in V_+(s)\right\rbrace
\end{displaymath}
On a donc :
\begin{displaymath}
 \bigcup_{s\in \mathcal S_+} \left\lbrace (s,s'),s'\in V_+(s)\right\rbrace
=
\bigcup_{s'\in \mathcal S_-} \left\lbrace (s,s'),s'\in V_+(s)\right\rbrace
= \mathcal A
\end{displaymath}

 \item D'après la question précédente, l'union des ensembles $\left\lbrace (s,s'),s'\in V_+(s)\right\rbrace$ pour $s$ décrivant $\mathcal S_+$ est égale à $\mathcal A$.\newline
 Chacun de ces ensembles contient $d_+(s)$ éléments car il est en bijection avec $V_+(s)$ par l'application $s'\rightarrow (s,s')$.\newline
Pour deux $s$ distincts, ces ensembles sont disjoints car les premiers termes des couples sont distincts.\newline
Ces ensembles constituent une \emph{partition} de $\mathcal A$. On en déduit que le nombre d'éléments dans $\mathcal A$ est la somme des nombres d'éléments de ces ensembles, c'est à dire
\begin{displaymath}
 \sharp\, \mathcal A = \sum _{s\in \mathcal S_+}d_+(s)
\end{displaymath}
 La démonstration de l'autre égalité est analogue.
\end{enumerate}

 \item Dans cette question, il faut bien réaliser que les sommes à gauche des égalités portent sur des ensembles d'arêtes et non de sommets. On utilise encore la partition de $\mathcal{A}$ de la question $2.a$ et on regroupe les arêtes qui ont un même sommet final :
\begin{displaymath}
 \sum_{(s,s')\in\mathcal A}d_+(s') = \sum_{s'\in \mathcal S_-}\left(\sum_{s\in V_-(s')}d_+(s') \right) 
= \sum_{s'\in \mathcal S_-} \sharp\,V_-(s')\, d_+(s')
= \sum_{s'\in \mathcal S_-} d_-(s')\, d_+(s')
\end{displaymath}
 La démonstration est la même pour l'autre égalité en regroupant cette fois les arêtes qui ont un même sommet initial $s$.

 \item 
\begin{enumerate}
 \item 
Si un graphe orienté $\mathcal A$ contient un triangle $s_1,s_2,s_3$ alors, par définition, $(s_3,s_1)$ est une arête avec $s_2\in V_+(s_1)$ et $s_2\in V_-(s_3)$ donc il existe une arête $(s,s')$ (avec $s=s_3$ et $s'=s_1$) telle que $V_+(s')\cap V_-(s)$ est non vide.\newline
Réciproquement, si $(s,s')$ est une arête telle que $V_+(s')\cap V_-(s)$ contient un élément $w$, il est immédiat que $(s',w,s)$ est un triangle du graphe.
 
 \item Si le graphe ne contient pas de triangle, alors pour toute arête $a=(s,s')$, on doit avoir $V_+(s')\cap V_-(s)$ vide. On en tire
\begin{displaymath}
\left. 
\begin{aligned}
 & \sharp\left( V_+(s')\cup V_-(s)\right) = d_+(s') + d_-(s) \\
 & V_+(s')\cup V_-(s) \subset \mathcal S
\end{aligned}
\right\rbrace \Rightarrow
T(a) = d_+(s) + d_-(s) \leq \sharp\, \mathcal S
\end{displaymath}
\end{enumerate}

 \item Comme le graphe est conservatif, on peut confondre les types de sommets et les nombres d'arêtes $d$. On notera en particulier $d(s)=d_-(s)=d_+(s)$ pour tout sommet $s$. On déduit de la question 3. que
\begin{displaymath}
 \sum_{a\in\mathcal A}T(a) =\sum_{(s,s')\in \mathcal A}d_-(s) + \sum_{(s,s')\in \mathcal A}d_+(s')
= 2 \sum_{s\in \mathcal S}d(s)^2
\end{displaymath}
D'autre part, d'après 2.b, on peut majorer le nombre d'arêtes avec l'inégalité de Cauchy-Schwarz.
\begin{displaymath}
 \sharp\, A =\sum_{s\in \mathcal S}d(s) =\sum_{s\in \mathcal S}\left( d(s)\times 1\right) 
\leq
\sqrt{\left( \sum_{s\in \mathcal S}d(s)^2\right) \left( \sum_{s\in \mathcal S} 1\right)}
\end{displaymath}
On en tire
\begin{displaymath}
 \sharp\, A \leq \sqrt{\frac{1}{2}\left( \sum_{a\in\mathcal A}T(a)\right)  \sharp\,\mathcal S}
\end{displaymath}
Ce qui conduit à l'inégalité demandée.

 \item Considérons un graphe conservatif qui ne contient pas de triangle, on peut appliquer les inégalités de 5. et 4.b:
\begin{multline*}
 \sharp\, A \leq \sqrt{\frac{1}{2}\left( \sum_{a\in\mathcal A}T(a)\right)  \sharp\,\mathcal S}
\leq \sqrt{\frac{1}{2} \left( \sharp\, \mathcal A \times \sharp\, \mathcal S\right) \sharp\,\mathcal S}\\
\Rightarrow
\sqrt{2\,\sharp\,\mathcal A} \leq \sharp\,\mathcal S 
\Rightarrow
\sharp\,\mathcal A \leq \frac{1}{2} (\sharp\,\mathcal S)^2
\end{multline*}
On en tire le théorème de Mantel. Si un graphe orienté conservatif vérifie $2\,\sharp\,\mathcal A > (\sharp \,\mathcal S)^2$, alors il ne contient pas de triangle.
\end{enumerate}


\subsection*{Partie II. Graphes non orientés}
\begin{enumerate}
 \item \`A chaque paire $\{s,s'\}$ d'un graphe non orienté $\mathcal O$ on peut associer \emph{les deux couples} $(s,s')$, $(s',s)$. Constituons un graphe orienté $\mathcal A$ avec tous ces couples. Les ensembles de sommets sont alors les mêmes mais le graphe orienté contient deux fois plus d'arêtes. $\sharp\,\mathcal A = 2\,\sharp\, \mathcal O$. Le graphe orienté $\mathcal A$ est automatiquement conservatif. 
 \item On peut appliquer le théorème de Mantel au graphe orienté conservatif $\mathcal A$ associé au graphe non orienté $\mathcal O$. On en déduit que si
\begin{displaymath}
 \sharp \mathcal O >\frac{(\sharp\, \mathcal S)^2}{4}
\end{displaymath}
alors le graphe $\mathcal{O}$ ne contient pas de triangle.
 \item Classons les arêtes d'un graphe bipartite complet suivant le sommet de départ dans $\mathcal E_1$. Il y a toujours $n_2$ arêtes issues d'un sommet donné. On en déduit que le nombre d'arêtes est $n_1n_2$ où $n_1$ et $n_2$ sont les nombres d'éléments de $\mathcal E_1$ et $\mathcal E_2$.
 \item Notons $p$ la partie entière de $\frac{n}{2}$ de sorte que $n=2p$ si $n$ est pair et $n=2p+1$ si $n$ est impair.\newline
On peut remarquer alors que 
\begin{displaymath}
 \left. 
\begin{aligned}
 n^2 &= 4p^2     &\text{ si $n$ pair} \\
 n^2 &= 4p^2 +4p +1 &\text{ si $n$ pair} 
\end{aligned}
\right\rbrace 
\Rightarrow
\lfloor\frac{n^2}{4}\rfloor = 
\left\lbrace 
\begin{aligned}
&p^2 &\text{si $n$ pair}\\ 
&p^2 +p &\text{si $n$ impair}
\end{aligned}
\right. 
\end{displaymath}
Notons $n_1=p$ et définissons $n_2$ par:
\begin{displaymath}
 n_2 = 
\left\lbrace 
\begin{aligned}
&p &\text{si $n$ pair}\\ 
&p +1 &\text{si $n$ impair}
\end{aligned}
\right. 
\end{displaymath}
On peut donc partitionner un ensemble $E$ à $n$ éléments en deux parties $\mathcal E_1$ et $\mathcal E_2$ et former un graphe bipartite complet contenant $n_1n_2$ arêtes avec :
\begin{displaymath}
 n_1n_2 = 
\left\lbrace 
\begin{aligned}
&p^2 &\text{si $n$ pair}\\ 
&p^2 +p &\text{si $n$ impair}
\end{aligned}
\right\rbrace 
= \lfloor\frac{n^2}{4}\rfloor  
\end{displaymath}
Comme un graphe bipartite complet est évidemment sans triangle, on a montré que l'inégalité du theorème de Mantel assurant qu'un graphe est sans triangle est la meilleure possible. 
\end{enumerate}
