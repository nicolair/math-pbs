%<dscrpt>Déterminants de Cauchy.</dscrpt>
Soient $a_1, \cdots, a_n, b_1, \cdots, b_n$ des réels tels que
\begin{displaymath}
\forall (i,j)\in \llbracket 1,n \rrbracket^2,\; i\neq j \Rightarrow a_i \neq a_j \text{ et } b_i \neq b_j  
\end{displaymath}
Soit $C(a_1,b_1,a_2,b_2,\cdots,a_n,b_n)$ la matrice $n\times n$ (dite de Cauchy) dont le coefficient d'indice $i,j$ est $\frac{1}{a_i + b_j}$ et $c(a_1,b_1,a_2,b_2,\cdots,a_n,b_n)$ son déterminant.\newline
L'objet de cet exercice est d'obtenir, par deux méthodes différentes une expression factorisée de ce déterminant.
\begin{enumerate}
  \item Calculer $60^3 c(1,1,2,2,3,3)$.

  \item Opérations élémentaires.
\begin{enumerate}
  \item Préciser l'opération élémentaire et les factorisations montrant que
\begin{displaymath}
c(a_1,b_1,a_2,b_2)=
\frac{a_1-a_2}{(a_1+b_1)(a_1+b_2)}
\begin{vmatrix}
  1 & 1 \\ \frac{1}{a_2 + b_1} & \frac{1}{a_2 + b_2}
\end{vmatrix}
\end{displaymath}
En déduire l'expression factorisée de $c(a_1,b_1,a_2,b_2)$.
  \item On note $L_i$ la ligne $i$ de $C(a_1,b_1,a_2,b_2,\cdots,a_n,b_n)$. Pour $i\in \llbracket 2,n\rrbracket$ et $j\in \llbracket 1,n \rrbracket$, préciser le coefficient dans la colonne $j$ de $L_i - L_1$.
  \item Montrer que
\begin{displaymath}
c(a_1,b_1,\cdots,a_n,b_n) = \frac{\prod_{i=2}^n(a_1 - a_i)}{\prod_{j=1}^n(a_1 + b_j)}
\begin{vmatrix}
  1                 & 1                  & \cdots & 1 \\
 \frac{1}{a_2 + b_1}& \frac{1}{a_2 + b_2}& \cdots &\frac{1}{a_2 + b_n} \\
 \vdots             &   \vdots           &        & \vdots             \\
 \frac{1}{a_n + b_1}& \frac{1}{a_n + b_2}& \cdots &\frac{1}{a_n + b_n} 
 \end{vmatrix}
\end{displaymath}
  \item Montrer que
\begin{displaymath}
c(a_1,b_1,\cdots,a_n,b_n) = 
\frac{\prod_{i=2}^n(a_1 - a_i)\prod_{j=2}^n(b_1-b_j)}{\prod_{j=1}^n(a_1 + b_j)\prod_{i=2}^n(a_i + b_1)}  
c(a_2,b_2,\cdots,a_n,b_n)
\end{displaymath}
\end{enumerate}

\item Méthode algébrique.\newline
On considére l'application $F$ définie dans une partie de $\R$ par :
\begin{displaymath}
  x \mapsto F(x) = c(x,b_1,\cdots,a_{n},b_{n})
\end{displaymath}
\begin{enumerate}
  \item Montrer que $F$ est une fraction rationnelle. Préciser son degré et ses pôles.
  \item Montrer qu'il existe un réel $\lambda$ et des polynômes unitaires $A$ et $B$ tels que $F = \lambda \frac{A}{B}$. Préciser la forme factorisée de $A$ et $B$. Montrer que
\begin{displaymath}
  \lambda=
\begin{vmatrix}
  1                 & 1                  & \cdots & 1 \\
 \frac{1}{a_2 + b_1}& \frac{1}{a_2 + b_2}& \cdots &\frac{1}{a_2 + b_n} \\
 \vdots             &   \vdots           &        & \vdots             \\
 \frac{1}{a_n + b_1}& \frac{1}{a_n + b_2}& \cdots &\frac{1}{a_n + b_n}   
\end{vmatrix}
\end{displaymath}
  \item Comment retrouver la formule de la question 2.d. sans opérations élémentaires?
\end{enumerate}

\end{enumerate}
