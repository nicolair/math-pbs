%<dscrpt> Théorème de Nicomachus. </dscrpt>
Dans tout cet exercice, $n$ désigne un entier naturel supérieur ou égal à $2$.\newline
On considère la suite des nombres impairs que l'on somme par groupes de $1, 2, 3, \cdots$
\begin{displaymath}
 \underset{=s_1}{\underbrace{1}},\; \underset{=s_2}{\underbrace{3 ,\; 5}},\; \underset{=s_3}{\underbrace{7 ,\; 9 ,\; 11}},\;
 \underset{=s_4}{\underbrace{13 ,\; 15 ,\; 17 ,\; 19}},\;\cdots
\end{displaymath}
On définit ainsi des nombres $s_n$ avec $n$ entier naturel non nul. Les premières valeurs sont
\begin{displaymath}
 s_1=1, \hspace{0.5cm}s_2 = 3 + 5, \hspace{0.5cm} s_3 = 7 + 9 + 11, \hspace{0.5cm} s_4 = 13 + 15 + 17 + 19, \hspace{0.5cm} \cdots
\end{displaymath}
\begin{enumerate}
 \item Combien de termes (nombres impairs) la somme $s_1 + s_2 + \cdots + s_{n-1}$ contient-elle? Quel est le plus grand terme de cette somme?
 
 \item Quel est le plus petit terme de la somme $s_n$? (On le notera $t_n$) En déduire une expression de $s_n$ avec le symbole $\sum$.
 
 \item Former une expression très simple de $s_n$. (Théorème de Nicomachus)
 
 \item Préciser la somme des entiers de $1$ à $n(n+1)$ et celle des entiers \emph{pairs} entre $1$ et $n(n+1)$. En déduire, en utilisant le théorème de Nicomachus,
\begin{displaymath}
 1^3 + 2^3 + \cdots + n^3 = \left( \frac{n(n+1)}{2}\right)^2  
\end{displaymath}

\end{enumerate}
