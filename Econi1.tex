%<dscrpt>Ellipse: `bande de papier` et cercles de Chasles.</dscrpt>
\begin{figure}
	\begin{center}
	\input{Econi1_1.pdf_t}
	\end{center}
\caption{Bande de papier.}
\end{figure} 
Sur une bande de papier, on place trois points $Q$, $M$, $P$. Le point $M$ est entre $P$ et $Q$ avec $QM=a\geq MP=b>0$ comme sur la figure.\newline
Un repère orthonormé $(O,\overrightarrow{i},\overrightarrow{j})$ étant fixé, on fait glisser la bande de papier en plaçant le point $Q$ sur l'axe $(Oy)$ et le point $P$ sur l'axe $(Ox)$. On note $\mathcal E$ l'ensemble décrit par les points $M$.
\begin{enumerate}
    \item Construire quelques points de $\mathcal E$ sur une figure.
    \item On note $I$ le point du plan dont les projetés orthogonaux sur les axes sont $P$ et $Q$. On note  $\theta$ l'angle orienté $(\overrightarrow{i},\overrightarrow{OI})$.
    \begin{enumerate}
       \item Calculer $\Vert \overrightarrow{OI} \Vert$.
       \item  Calculer en fonction de $\theta$ les coordonnées de $P$, $Q$, $M$.
       \item Que peut-on en déduire pour $\mathcal E$ ?
    \end{enumerate}

\item (cercles de Chasles) On définit les points $I_\theta$ et $J_\theta$ par :
\begin{align*}
 \overrightarrow{OI_\theta}=(a+b)\overrightarrow{e}_\theta & &
 \overrightarrow{OJ_\theta}=(a-b)\overrightarrow{e}_{-\theta}
\end{align*}
Montrer que le milieu de $[I_\theta, J_\theta]$ est le point $M$ et que la droite $(I_\theta, J_\theta)$ est la normale en $M$ à $\mathcal E$. 
\end{enumerate}
