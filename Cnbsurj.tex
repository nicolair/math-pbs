
\subsubsection*{PARTIE I}

\begin{enumerate}
\item
\begin{itemize}
\item  Lorsque $p>n$, il n'existe pas d'applications surjective de $E_{n}$
vers $E_{p}$ : $S_{n,p}=0$.

\item  L'ensemble des applications surjectives de $E_{n}$ dans lui m{\^e}me
est aussi celui des bijections de $E_{n}$ : $S_{n,n}=n!$.

\item  L'application constante {\'e}gale {\`a} 1 est la seule application
surjective de $E_{n}$ vers $E_{1}$ : $S_{n,1}=1$.

\item  Une application quelconque de $E_{n}$ dans $E_{2}$ est
caract{\'e}ris{\'e}e par l'ensemble des ant{\'e}c{\'e}dents de 1. L'application
consid{\'e}r{\'e}e est surjective lorsque cet ensemble n'est
ni $\emptyset $ ni $E_{2}$. Comme $2^{n}$ est le nombre total de parties de $E_{n}$ : $S_{n,2}=2^{n}-2$. On peut aussi remarquer que toutes les applications sont surjectives sauf les deux applications constantes de valeur 0 et 1.
\end{itemize}

\item  Une application surjective de $E_{p+1}$ vers $E_{p}$ ne peut {\^e}tre
injective. Il existe donc un {\'e}l{\'e}ment $x$ de $E_{p}$ admettant
plusieurs ant{\'e}c{\'e}dents. Si $x$ admet strictement plus de 2
ant{\'e}c{\'e}dents, il reste dans $E_{p+1}$ strictement moins de $p-1$
{\'e}l{\'e}ments distincts qui ont forc{\'e}ment moins de $p-1$ images . M{\^e}me
en comptant $x$, le nombre total d'image est strictement plus
petit que $p$ en contradiction avec la surjectivit{\'e} de
l'application consid{\'e}r{\'e}e. Il existe donc un seul {\'e}l{\'e}ment de
$E_{p}$ admettant plusieurs ant{\'e}c{\'e}dents, en fait 2
exactement.\newline
 Une application surjective de $E_{p+1}$ dans
$E_{p}$ est donc caract{\'e}ris{\'e}e par un triplet $(A,x,g)$ o{\`u} $A$ est
une paire d'{\'e}l{\'e}ments de $E_{p+1}$, $x$ un {\'e}l{\'e}ment de $E_{p}$, $g$
une bijection de $E_{p+1}-A$ vers $E_{p}-\{x\}$. On en d{\'e}duit
\[
S_{p+1,p}=\binom{p+1}{2}\cdot p\cdot (p-1)!=\frac{p}{2}(p+1)!
\]
\end{enumerate}

\subsubsection*{PARTIE II}

\begin{enumerate}
\item  Transformons le produit de coefficients du bin{\^o}me :
\begin{eqnarray*}
\binom{p}{q}\binom{q}{k}&=&\frac{p!}{q!(p-q)!}\frac{q!}{k!(q-k)!}\\
&=&\frac{p!(p-k)!}{\underset{=(p-k-(q-k))!}{(p-q)!}k!(q-k)!(p-k)!}\\
&=&\binom{p}{k}%
\binom{p-k}{q-k}
\end{eqnarray*}
En injectant cette transformation dans la somme de l'{\'e}nonc{\'e} puis
en posant $r=q-k$, on obtient
\[
\sum_{q=k}^{p}(-1)^{q}\binom{p}{q}\binom{q}{k}=\binom{p}{k}%
\sum_{q=k}^{p}(-1)^{q}\binom{p-k}{q-k}=\binom{p}{k}(-1)^{k}%
\sum_{r=0}^{p-k}(-1)^{r}\binom{p-k}{r}
\]
Ceci est nul d'apr{\`e}s la formule du bin{\^o}me lorsque $0\leq k<p$.

\item  Classons les fonctions $f$ de $E_{n}$ vers $E_{p}$ suivant les
parties $f(E_{n})$. Quel est le nombre de fonctions $f$ telles que
$f(E_{n})$ soit une partie donn{\'e}e {\`a} $q$ {\'e}l{\'e}ments de $E_{p}$? Il y
en a
autant que de surjections de $E_{n}$ vers $E_{q}$ c'est {\`a} dire $S_{n,q}$%
. Comme $p^{n}$ est le nombre total de fonctions et que
$\binom{p}{q}$ est le nombre de parties {\`a} $q$ {\'e}l{\'e}ments de $E_{p}$,
la partition associ{\'e}e {\`a} cette classification montre que
\[
p^{n}=\sum_{q=0}^{p}\binom{p}{q}S_{n,q}
\]

\item  Calculons en utilisant la formule du 2.
\begin{eqnarray*}
(-1)^{p}\sum_{k=0}^{p}(-1)^{k}\binom{p}{k}k^{n}
&=&(-1)^{p}\sum_{k=0}^{p}(-1)^{k}\binom{p}{k}\sum_{q=0}^{k}\binom{k}{q}S_{n,q}\\
&=&(-1)^{p}\sum_{(k,q)\in
T}(-1)^{k}\binom{p}{k}\binom{k}{q}S_{n,q}
\end{eqnarray*}
o{\`u} $(k,q)\in T\Leftrightarrow k\in \left\{ 0,\ldots ,p\right\} $ et $%
q\in \left\{ 0,\ldots ,k\right\} \Leftrightarrow q\in \left\{
0,\ldots ,p\right\} $ et $k\in \left\{ q,\ldots ,p\right\} $ donc
\[
(-1)^{p}\sum_{k=0}^{p}(-1)^{k}\binom{p}{k}k^{n}=(-1)^{p}%
\sum_{q=0}^{p}S_{n,q}\sum_{k=q}^{p}(-1)^{k}\binom{p}{k}\binom{k}{q}
\]
avec $\sum_{k=q}^{p}(-1)^{k}\binom{p}{k}\binom{k}{q}=0$ sauf si
$p=q$ auquel cas il vaut $(-1)^{p}.$ On conclut
\[
S_{n,p}=(-1)^{p}\sum_{k=0}^{p}(-1)^{k}\binom{p}{k}k^{n}
\]

\item  Utilisons les formules pr{\'e}c{\'e}dentes en regroupant les
coefficients de $k^{n-1}$
\[
p(S_{n-1,p}+S_{n-1,p-1})=(-1)^{p}\sum_{k=0}^{p}(-1)^{k}p\left( \binom{p}{k}-%
\binom{p-1}{k}\right) k^{n-1}
\]
avec $p\left( \binom{p}{k}-\binom{p-1}{k}\right) =p\binom{p-1}{k-1}=k\binom{p%
}{k}.$ Finalement $p(S_{n-1,p}+S_{n-1,p-1})=S_{n,p}$.

\begin{itemize}
\item  Pour $n=p+1$, $S_{p+1,p}=p(S_{p,p}+S_{p,p-1})=pp!+pS_{p,p-1}$. A
partir de cette relation et de $S_{2,1}=1$, on obtient par
r{\'e}currence que $S_{p+1,p}=\frac{p}{2}(p+1)!$.

\item  De m{\^e}me, pour $n=p+2,$ supposons $S_{p+1,p-1}=\frac{(p-1)(3p-2)}{%
24}(p+1)!$ alors
\begin{eqnarray*}
S_{p+2,p}&=&p\left(
\frac{p}{2}(p+1)!+\frac{(p-1)(3p-2)}{24}(p+1)!\right)\\
&=&p(p+1)!\frac{3p^{2}+7p+2}{24}\\
&=&\frac{p(p+1)!(p+2)(3p+1)}{24}
\end{eqnarray*}
\end{itemize}

\item  La relation $p(S_{n-1,p}+S_{n-1,p-1})=S_{n,p}$ permet, comme pour le
triangle de Pascal de construire le tableau suivant
\[
\begin{tabular}{llllllll}
& 1 & 2 & 3 & 4 & 5 & 6 & 7 \\
1 & 1 & 0 & 0 & 0 & 0 & 0 & 0 \\
2 & 1 & 2 & 0 & 0 & 0 & 0 & 0 \\
3 & 1 & 6 & 6 & 0 & 0 & 0 & 0 \\
4 & 1 & 14 & 36 & 24 & 0 & 0 & 0 \\
5 & 1 & 30 & 150 & 240 & 120 & 0 & 0 \\
6 & 1 & 62 & 540 & 1560 & 1800 & 720 & 0 \\
7 & 1 & 126 & 1806 & 8400 & 16800 & 15120 & 5040
\end{tabular}
\]
\end{enumerate}

\subsubsection*{PARTIE III}

\begin{enumerate}
\item  A chaque surjection $f$ de $E_{n}$ dans $E_{p}$, on peut associer la
partition de $E_{n}$ en sous-ensembles non vides $\left\{
f^{-1}(\{1\}),f^{-1}(\{2\}),\cdots ,f^{-1}(\{p\})\right\}
$.\newline Deux applications $f$ et $g$ donneront la m{\^e}me
partition si et seulement si il existe une permutation $\sigma $
de $E_{p}$ telle que $g=\sigma \circ f $. On en d{\'e}duit
$A_{n,p}=\frac{1}{p!}S_{n,p}$.

\item  De la relation $p(S_{n-1,p}+S_{n-1,p-1})=S_{n,p}$, on d{\'e}duit
imm{\'e}diatement $A_{n,p}=A_{n-1,p-1}+pA_{n-1,p}$.

\item  D'apr{\'e}s 1., $A_{n,1}=A_{n,n}=1$. Pour obtenir la table des $%
A_{n,p}$, on divise par $p!$ la colonne $p$ du tableau pr{\'e}c{\'e}dent.
\end{enumerate}

\subsubsection*{PARTIE IV}

\begin{enumerate}
\item  On peut classer les bijections de $E_{n}$ selon le nombre $n-k$ de
leurs points fixes. Il y a $\binom{n}{n-k}=\binom{n}{k}$ choix de
$k$ points
fixes et $D_{k}$ bijections ayant $k$ points fixes. On en d{\'e}duit $%
n!=\sum_{k=0}^{n}\binom{n}{k}D_{k}$.

\item  Comme en II 3., calculons en permutant les sommations :
\begin{eqnarray*}
(-1)^{n}\sum_{k=0}^{n}(-1)^{k}\binom{n}{k}k!
&=&(-1)^{n}\sum_{k=0}^{n}(-1)^{k}\binom{n}{k}\sum_{q=0}^{k}\binom{k}{q}
D_{q}\\
&=&(-1)^{n}\sum_{q=0}^{n}\left(
\sum_{k=q}^{n}(-1)^{k}\binom{n}{k}\binom{k}{q}\right)  =D_{n}
\end{eqnarray*}
\end{enumerate}
