%<dscrpt>Exercices sur les fonctions usuelles, les nombres complexes et les sommations.</dscrpt>
\begin{enumerate}

%exo2
\item  Dans quelle partie de $\R$ la fonction $f$
\begin{displaymath}
 x \mapsto \arcsin (\frac{2x}{1+x^{2}})
\end{displaymath}
est elle définie? dérivable?\newline
Transformer $f(x)$ en introduisant $\theta = \arctan x$. Présenter dans un tableau les différentes expressions de $f(x)$ et les intervalles dans lesquels elles sont valides.\newline
Retrouver le tableau précédent en utilisant la dérivée.

%exo3
\item
\begin{enumerate}
\item Dans quel cas un nombre complexe admet-il un argument dans $\left] -\frac{\pi}{2}, \frac{\pi}{2}\right[$? 
\item  Soit $a \in \R$ et $t \in \left]-1,+1\right[ $. Calculer la partie réelle et la partie imaginaire de
\[\frac{e^{-ia}+t}{e^{-ia}-t}\]
Montrer que ce nombre complexe a un argument dans $\left] -\frac{\pi }{2},+ \frac{\pi }{2}\right[ $.
\item  On suppose toujours $t\in \left] -1,+1\right[ $ et on pose
\[M=\frac{1}{2}\ln \frac{1+2t\cos a+t^{2}}{1-2t\cos a+t^{2}}\quad N=\arctan
\frac{2t\sin a}{1-t^{2}}\]
Calculer $e^{S}$ pour $S=M+ i N$.
\item \'Enoncer et prouver un résultat analogue à celui de la question $c$ dans le cas $t>1$.
\end{enumerate}

%exo4
\item Préciser dans quelles parties de $\R$ les fonctions $\arcsin \circ \cos$, $\arccos \circ \sin$, $\sin \circ \arccos$ et $\cos \circ \arcsin$ sont définies. Tracer les graphes de ces fonctions.


%exo8
\item Soit $p$ et $q$ deux entiers non nuls, trouver une forme simplifi{\'e}e pour
\begin{displaymath}
 \sum _{k=0}^{p}\binom{p+q}{k} \binom{p+q-k}{p-k}.
\end{displaymath}


%exo10
\item Soit $n$ un entier strictement positif, calculer les sommes suivantes
\begin{displaymath}
\sum _{k=0}^{n}\frac{\sin(kx)}{\cos^{k}x},\hspace{1cm}
\sum _{k=0}^{n}\binom{n}{k}\sin(kx)
\end{displaymath}

\end{enumerate}
