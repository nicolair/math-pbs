

\begin{enumerate}
\item Un nombre complexe $z$ est solution si et seulement si il est diff{\'e}rent de $i$ et si $\frac{z+i}{z-i}$ est solution de $1+w+w^{2}+w^{3}=0$ dont les racines sont 
\begin{displaymath}
 U_{4}-\left\{ 1\right\}=\left\{ -1,i,-i\right\}
\end{displaymath}
D'autre part la fonction homographique $z\rightarrow \frac{z+i}{z-i}$ est une bijection de $\C-\left\{ i\right\}$ vers $\C-\left\{1\right\}$ dont la bijection r{\'e}ciproque est $w\rightarrow i\frac{w+1}{w-1}$. On en d{\'e}duit que l'ensemble des solutions est
\[
\left\{ 0,1,-1\right\}
\]

\item Il est {\'e}vident que $P(z)-P(-z)$ se r{\'e}duit
aux termes d'indice impair compt{\'e}s deux fois soit
\[
2(a_{1}z+a_{3}z^{3}+a_{5}z^{5}+\cdots )
\]
Pour $P(z)=(1+z)^{7}$ et gr{\^a}ce {\`a} la formule du bin{\^o}me,
l'{\'e}quation propos{\'e}e revient {\`a} $(1+z)^{7}=(1-z)^{7},$ soit comme $1$ n'est pas solution {\`a}
\[
\left( \frac{1+z}{1-z}\right) ^{7}=1
\]
ou encore {\`a} $\frac{1+z}{1-z}\in U_{7}$. L'homographie $z\rightarrow \frac{1+z}{1-z}$ est une bijection de $\C-\left\{ 1\right\}$ vers $\C-\left\{ -1\right\}$ dont la bijection r{\'e}ciproque est $w\rightarrow \frac{w-1}{w+1}$. De plus, si $w=e^{it}$
\[
\frac{e^{it}-1}{e^{it}+1}=i\tan \frac{t}{2}
\]
Comme $-1$ n'est pas racine 7${{}^\circ}$ toutes les racines de l'unit{\'e}
correspondent {\`a} des solutions de l'{\'e}quation propos{\'e}e dont
l'ensemble est
\[
\left\{ i\tan \frac{k\pi }{7},k\in \left\{ 0,\cdots ,6\right\} \right\}
\]

\item  Remarquons que $-\sin x+\sin 3x=2\sin x\cos x$.
\newline
 En utilisant $\sin 2x=2\sin x\cos x$, les in{\'e}quations suivantes sont {\'e}quivalentes

\[0<\sin x(2\cos ^{2}x-1-\cos x)\]
\[0< \sin x(\cos x-1)(2\cos x+1)\]
\[ \sin x(2\cos x+1)<0\]

On en d{\'e}duit  que l'ensemble des solutions est
\[
\bigcup_{k\in \Z}\left] -\frac{2\pi }{3}+2k\pi ,+2k\pi
\right[ \cup \left] \frac{2\pi }{3}+2k\pi ,\pi +2k\pi \right[
\]

\item Comme $\arctan 1=\frac{\pi }{4}$, il suffit de
prouver que $\arctan 2+\arctan 3=\frac{3\pi }{4}$.\newline
D'une part
$\arctan 2+\arctan 3\in \left[ \frac{\pi }{4}+\frac{\pi
}{4},\frac{\pi }{2}+\frac{\pi }{2}\right] =\left[ \frac{\pi
}{2},\pi \right] $.\newline D'autre part,
\[
\tan (\arctan 2+\arctan 3)=\frac{2+3}{1-2\times 3}=-1
\]
c'est {\`a} dire $\arctan 2+\arctan 3\in -\frac{\pi }{4}+\pi
\Z$. La seule possibilit{\'e} dans l'intervalle $\left[
\frac{\pi }{2},\pi \right] $ est $\frac{3\pi }{4}$.

\item On peut former le tableau suivant.
\begin{center}
\begin{tabular}{|c|c|c|c|c|}
\hline
& $\left[ 0,\frac{\pi }{2}\right] $ & $\left[ \frac{\pi }{2},\pi \right] $ &
$\left[ \pi ,\frac{3\pi }{2}\right] $ & $\left[ \frac{3\pi }{2},2\pi \right]
$ \\ \hline
$\sin $ & $+$ & $+$ & $-$ & $-$ \\ \hline
$\cos $ & $+$ & $-$ & $-$ & $+$ \\ \hline
$\left| \sin \right| $ & $\sin x$ & $\sin x$ & $-\sin x$ & $-\sin x$ \\
\hline
$\left| \cos \right| $ & $\cos x$ & $-\cos x$ & $-\cos x$ & $\cos x$ \\
\hline
$\arcsin (\left| \sin \right| )$ & $x$ & $\pi -x$ & $x-\pi $ & $2\pi -x$ \\
\hline
$\arccos (\left| \cos \right| )$ & $x$ & $\pi -x$ & $x-\pi $ & $2\pi -x$ \\
\hline
\end{tabular}
\end{center}
%}\end{floatingtable}
\end{enumerate}
