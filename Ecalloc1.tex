%<dscrpt>Etude d'une fonction, développement limité.</dscrpt>
Soit $f_1$ l'application de $\R^{*}$ dans $\R$ d{\'e}finie par
\[
\forall x\in \R^{*},f_1(x)=\left( x+\sqrt{x^2+1}\right) ^{\frac 1x}%
\text{.}
\]

\begin{enumerate}
\item
\begin{enumerate}
\item  Montrer que $f_{1}$ est prolongeable par continuit{\'e} en une fonction $f$ d{\'e}finie et continue sur $\R$.

\item  {\'E}tudier la parit{\'e} de $f$.

\item  Calculer un d{\'e}veloppement limit{\'e} {\`a} l'ordre 4 de $f$ en 0.

\item  {\'E}tudier le sens de variation de $f$ et les limites de $f^{\prime} $ en 0 et $+\infty $.

\item  Tracer sommairement le graphe de $f$. Pr{\'e}ciser la position de la courbe par rapport {\`a} la parabole d'{\'e}quation $y=e(1-\frac{x^{2}}{6})$ au voisinage de 0.
\end{enumerate}

\item  Soit $F$ la primitive de $f$ sur $\R^{+}$ nulle en 0. On d{\'e}finit la fonction $H$ sur $\R^{+}$ en posant
\[
H(x)=\left\{
\begin{array}{ccc}
f(0) & \text{si} & x=0 \\
\frac{F(x)}{x} & \text{si} & x>0
\end{array}
\right. .
\]

\begin{enumerate}
\item  Montrer que $H$ est continue en 0.

\item  Montrer que $H$ est deux fois d{\'e}rivable dans $\R^{+}$, pr{\'e}ciser $H^{\prime }(0)$ et $H^{\prime \prime }(0)$.

\item  Montrer que $H(x)\geq f(x)$ pour tous les $x$ de $\R^{+}$.

\item  Quelle est la limite de $H$ en $+\infty $?
\end{enumerate}

\item
\begin{enumerate}
\item  Montrer que $F$ admet une bijection r{\'e}ciproque $G$ d{\'e}finie, continue et strictement croissante sur $\R^{+}$. (on
pourra utiliser 2.c.)

\item  Montrer que $G$ est d{\'e}rivable. Montrer que, pour tout $u>0$, il existe $v$ dans $\left] 0,G(u)\right[ $ tel que
\[
\frac{G(u)}{u}=\frac{1}{f(v)}\text{.}
\]
\end{enumerate}
\end{enumerate}
