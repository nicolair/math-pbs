\begin{enumerate}
 \item Les pôles de $F$ sont les racines $n$-ièmes de $1$. Comme la fraction est de degré strictement négatif, il n'y a pas de partie entière. La fraction est la somme de ses parties polaires.
 \item Le développement limité de $x^k$ en $1$ est
\begin{displaymath}
 x^k = 1 +k(x-1) + o(x-1)
\end{displaymath}
En sommant les développements précédents, la somme des entiers consécutifs apparait et il vient:
\begin{displaymath}
 1+x+\cdots +x^{n-1} = n +\frac{n(n-1)}{2}(x-1)+o(x-1)
\end{displaymath}
On factorise par $n$ pour se ramener à un développement usuel
\begin{multline*}
 \frac{1}{(1+x+\cdots +x^{n-1})^2}=\frac{1}{n^2}\left(1+\frac{n-1}{2}(x-1)+o(x-1) \right)^{-2}\\
=\frac{1}{n^2} -\frac{n-1}{n^2}(x-1)+o(x-1) 
\end{multline*}

 \item Soit $u\in\U$. Si on substitue $uX$ à $X$ dans $F$, la fraction est conservée. La partie polaire relative au pôle $u$ devient
\begin{displaymath}
 \frac{\alpha(u)}{(uX-u)^2}+\frac{\beta(u)}{uX-u}=
\frac{\alpha(u)}{u^2(X-1)^2}+\frac{\beta(u)}{u(X-1)}
\end{displaymath}
qui est la partie polaire relative à $1$. On en déduit
\begin{displaymath}
 \alpha(u) = u^2\alpha(1),\hspace{1cm} \beta(u)=u\beta(1)
\end{displaymath}

 \item
\begin{enumerate}
 \item Par définition d'une partie polaire,
\begin{displaymath}
 F = \frac{\alpha(1)}{(X-1)^2} + \frac{\beta(1)}{X-1} +R
\end{displaymath}
où $R$ est une fraction qui n'admet pas de pôle en $1$. Comme
\begin{displaymath}
 X^n-1 = (X-1)(1+X+\cdots+X^{n-1})
\end{displaymath}
En multipliant $F$ par $(X-1)^2$, on obtient
\begin{displaymath}
 \frac{1}{(1+X+\cdots+X^{n-1})^2} = \alpha(1)+\beta(1)(X-1)+(X-1)^2R
\end{displaymath}
Comme $1$ n'est pas un pôle de $R$, la fonction attachée à $R$ admet une limite finie en $1$ dont la fonction attachée à$(X-1)^2R$ est négligeable en $1$ devant $x-1$. L'écriture proposée est donc bien un développement limité en $1$.
 \item En identifiant les développements limités obtenus en 2. et 4.a., on obtient
\begin{displaymath}
 \alpha(1)=\frac{1}{n^2},\hspace{1cm}\beta(1)=-\frac{n-1}{n^2}
\end{displaymath}
puis la décomposition en éléments simples
\begin{displaymath}
 F= \frac{1}{n^2}\sum_{u\in\U}\frac{u^2}{(X-u)^2} - \frac{n-1}{n^2}\sum_{u\in\U}\frac{u}{X-u}
\end{displaymath}
\end{enumerate}

 \item
\begin{enumerate}
 \item Si $1\leq k \leq n-1$ et $k'=n-k$, alors $1\leq k' \leq n-1$ et $\overline{w^k}=w^{k'}$.
 \item On regroupe les racines conjuguées. Dans le cas pair deux racines sont réelles ($1$ et $-1$) dans le cas impair $1$ est la seule racine.
\begin{align*}
 n=2p& &X^n-1 &= (X-1)(X+1)\prod_{k=1}^{p-1}\left(X^2-2\cos\frac{2k\pi}{n}X +1 \right) \\ 
 n=2p+1& &X^n-1 &= (X-1)\prod_{k=1}^{p}\left(X^2-2\cos\frac{2k\pi}{n}X +1 \right) 
\end{align*}

 \item Pour un pôle $u=e^{i\theta}$ non réel, 
\begin{displaymath}
 \frac{u^2}{(X-u)^2}= \frac{u^2(X^2-2\overline{u}X+\overline{u}^2)}{(X^2-2\cos\theta X +1)^2}
=\frac{u^2X^2-2uX+1}{(X^2-2\cos\theta X +1)^2}
\end{displaymath}
Les éléments simples relatifs à deux pôles conjugués sont eux mêmes conjugués. Les regrouper revient à prendre deux fois la partie réelle. Soit:
\begin{multline*}
 2\frac{\cos2\theta X^2 -2\cos\theta X +1}{(X^2-2\cos\theta X +1)^2}\\
= \frac{2\cos 2\theta }{X^2-2\cos\theta X +1} +4\sin^2\theta\frac{-2\cos \theta X +1}{(X^2-2\cos\theta X +1)^2}
\end{multline*}
Pour le résidu,
\begin{displaymath}
 \frac{u}{X-u}=\frac{u(X-\overline{u})}{X^2-2\cos\theta X +1}=\frac{uX-1}{X^2-2\cos\theta X +1}
\end{displaymath}
Le double de la partie réelle est 
\begin{displaymath}
 2\frac{\cos \theta X-1}{X^2-2\cos\theta X +1}
\end{displaymath}
On en déduit la décomposition dans $\R(X)$. On pose $\theta_k= \frac{2k\pi}{n}$.\newline
Dans le cas impair $n=2p+1$:
\begin{multline*}
 F=\frac{1}{n^2}\frac{1}{(X-1)^2}-\frac{n-1}{n^2}\frac{1}{X-1}\\
+\frac{4}{n^2}\sum_{k=1}^p\sin^2\theta_k\frac{-2\cos \theta_k X +1}{(X^2-2\cos\theta_k X +1)^2}\\
+\frac{2}{n^2}\sum_{k=1}^p\frac{(n-1)\cos \theta_k X +n-1+\cos 2\theta_k}{X^2-2\cos\theta_k X +1}
\end{multline*}
Dans le cas pair $n=2p$:
\begin{multline*}
 F=\frac{1}{n^2}\frac{1}{(X-1)^2}-\frac{n-1}{n^2}\frac{1}{X-1}
+\frac{1}{n^2}\frac{1}{(X+1)^2}+\frac{n-1}{n^2}\frac{1}{X+1}\\
+\frac{4}{n^2}\sum_{k=1}^{p-1}\sin^2\theta_k\frac{-2\cos \theta_k X +1}{(X^2-2\cos\theta_k X +1)^2}\\
+\frac{2}{n^2}\sum_{k=1}^{p-1}\frac{(n-1)\cos \theta_k X +n-1+\cos 2\theta_k}{X^2-2\cos\theta_k X +1}
\end{multline*}
\end{enumerate}

\end{enumerate}
