%<dscrpt>Orthocentre d'un triangle dont les points sont sur une hyperbole.</dscrpt>
On désigne par $x,y$ les fonctions coordonnées relatives à un repère orthonormé $(O,\overrightarrow{i},\overrightarrow{j})$ d'un plan P.\newline
Soit $\Gamma$\footnote{d'après E4A 2001 Maths 2} l'ensemble des points $M$ du plan tels que
\[x(M)y(M)=k\]
$k$ désignant un réel strictement positif fixé.
\begin{enumerate}
\item {\'E}crire l'équation de la hauteur issue de $A$ d'un triangle $(A,B,C)$.\newline
Le résultat sera exprimé à l'aide d'un déterminant faisant intervenir les coordonnées des points $A,B,C$.
\item On considère trois points $A,B,C$ de $\Gamma$ deux à deux distincts dont les abscisses sont notées $a,b,c$ respectivement.\newline
Déterminer les coordonnées $(\lambda,\mu)$ de l'orthocentre $H$ du triangle $(A,B,C)$. Vérifier que $H$ appartient à $\Gamma$.
\end{enumerate} 
