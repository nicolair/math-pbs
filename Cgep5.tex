\begin{enumerate}
 \item On sait que tout cercle est la ligne de niveau $0$ d'une fonction de la forme
\begin{displaymath}
 x^2 + y^2 + ax +by + c
\end{displaymath}
Il existe donc un cercle contenant le points $A_i$ si et seulement si il existe $a$, $b$, $c$ réels tels que
\begin{displaymath}
 \forall i\in \{1\cdots n\}:\hspace{0.5cm} ax(A_i) + by(A_i) +c = -(x(A_i)^2 + y(A_i)^2)
\end{displaymath}
Il s'agit donc d'un système de $n$ équations aux trois inconnues $a$, $b$, $c$.
 \item Pour $n=3$, le système devient
\begin{displaymath}
 \left\lbrace 
\begin{aligned}
 ax(A_1) + by(A_1) +c = -(x(A_1)^2 + y(A_1)^2)\\
 ax(A_2) + by(A_2) +c = -(x(A_2)^2 + y(A_2)^2)\\
 ax(A_3) + by(A_3) +c = -(x(A_3)^2 + y(A_3)^2)
\end{aligned}
\right. 
\end{displaymath}
Il admet une unique solution si et seulemnt si le déterminant est non nul soit
\begin{displaymath}
 \begin{vmatrix}
  x(A_1) & y(A_1) & 1 \\ x(A_2) & y(A_2) & 1 \\ x(A_3) & y(A_3) & 1 
 \end{vmatrix}
\neq 0
\end{displaymath}
On retrouve bien la condition d'alignement et le fait que par trois points passe un unique cercle (le cercle circonscrit) si et seulement si ils sont non alignés.\newline
Lorsqu'un cercle est la ligne de niveau $0$ d'une fonction de la forme
\begin{displaymath}
 x^2 + y^2 + ax +by + c
\end{displaymath}
les coordonnées du centre sont $(-\frac{a}{2},-\frac{b}{2})$. Les coordonnées du centre du cercle circonscrit à $A_1$, $A_2$, $A_3$ sont donc
\begin{displaymath}
 \left(
-\frac{1}{2}
\frac
{
 \begin{vmatrix}
  x(A_1)^2+y(A_1)^2 & y(A_1) & 1 \\ x(A_2)^2+y(A_2)^2 & y(A_2) & 1 \\ x(A_3)^2+y(A_3)^2 & y(A_3) & 1 
 \end{vmatrix}
}
{ \begin{vmatrix}
  x(A_1) & y(A_1) & 1 \\ x(A_2) & y(A_2) & 1 \\ x(A_3) & y(A_3) & 1 
 \end{vmatrix}
},
-\frac{1}{2}
\frac
{
 \begin{vmatrix}
  x(A_1) & x(A_1)^2+y(A_1)^2  & 1 \\ x(A_2) & x(A_2)^2+y(A_2)^2 & 1 \\ x(A_3) & x(A_3)^2+y(A_3)^2 & 1 
 \end{vmatrix}
}
{ \begin{vmatrix}
  x(A_1) & y(A_1) & 1 \\ x(A_2) & y(A_2) & 1 \\ x(A_3) & y(A_3) & 1 
 \end{vmatrix}
}
 \right) 
\end{displaymath}

 \item Notons $D_x$, $D_y$, $D_z$, $D$ les déterminants intervenant dans les formule de Cramer pour le système formé par les trois premières équations. Le cercle circoncrit aux trois premiers points a donc pour équation
\begin{displaymath}
 x^2+y^2+ax+by+c\text{ avec } a=\frac{D_x}{D}, b=\frac{D_y}{D}, c=\frac{D_z}{D}
\end{displaymath}
Par conséquent, les quatre points sont cocycliques si et seulement si
\begin{multline*}
 x(A_4)^2 + y(A_4)^2 + ax(A_4)+by(A_4)+c=0 \\
\Leftrightarrow 
(x(A_4)^2 + y(A_4)^2)D + x(A_4)D_x + y(A_4)D_y + cD_z=0
\end{multline*}
en multipliant par $D\neq0$.
\end{enumerate}
Examinons maintenant le déterminant proposé en le développant suivant la dernière ligne. On obtient
\begin{displaymath}
 -D_1x(A_4)+D_2y(A_4)-D_3+D_4(x(A_4)^2+y(A_4)^2)
\end{displaymath}
avec
\begin{displaymath}
 D_1 =
\begin{vmatrix}
 y(A_1) & 1 & x(A_1)^2+y(A_1)^2 \\ y(A_2) & 1 & x(A_2)^2+y(A_2)^2 \\ y(A_3) & 1 & x(A_3)^2+y(A_3)^2 
\end{vmatrix}
=-
\begin{vmatrix}
  -(x(A_1)^2+y(A_1)^2) & y(A_1) & 1 \\ -(x(A_2)^2+y(A_2)^2) & y(A_2) & 1 \\  -(x(A_3)^2+y(A_3)^2) & y(A_3) & 1 
\end{vmatrix}
=D_x
\end{displaymath}

\begin{displaymath}
 D_2 =
\begin{vmatrix}
 x(A_1) & 1 & x(A_1)^2+y(A_1)^2 \\ x(A_2) & 1 & x(A_2)^2+y(A_2)^2 \\ x(A_3) & 1 & x(A_3)^2+y(A_3)^2 
\end{vmatrix}
= +
\begin{vmatrix}
 x(A_1) & -(x(A_1)^2+y(A_1)^2) & 1 \\ x(A_2) & -(x(A_2)^2+y(A_2)^2) & 1 \\ x(A_3) & -(x(A_3)^2+y(A_3)^2)  & 1 
\end{vmatrix}
= D_y
\end{displaymath}


\begin{multline*}
 D_3 =
\begin{vmatrix}
 x(A_1) & y(A_1) & x(A_1)^2+y(A_1)^2 \\ x(A_2) & y(A_2) & x(A_2)^2+y(A_2)^2 \\ x(A_3) & y(A_3) & x(A_3)^2+y(A_3)^2 
\end{vmatrix} \\
= -
\begin{vmatrix}
 x(A_1) & y(A_1) & -(x(A_1)^2+y(A_1)^2) \\ x(A_2) & y(A_2) & -(x(A_2)^2+y(A_2)^2) \\ x(A_3) & y(A_3) & -(x(A_3)^2+y(A_3)^2) 
\end{vmatrix}
= -D_z
\end{multline*}
On retrouve donc exactement la condition précédente assurant la cocyclicité ce qui démontre l'équivalence.
