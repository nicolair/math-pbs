%<dscrpt>Relations entre coefficients et racines, éléments simples.</dscrpt>
Soit $A=(X+1)^{2n}-1 \in \R[X]$. On note \footnote{d'après Mines d'Albi 2000} le polynôme à coefficients réels 
\[
 P_{n}=\prod_{k=1}^{n}\sin \frac{k \pi}{2n}, \hspace{1cm} Q_{n}=\prod_{k=1}^{2n-1}\sin \frac{k \pi}{2n}.
\]
\begin{enumerate}
\item Montrer que l'on peut écrire $A=XB$ où $B$ est un polynôme dont on précisera le degré, le coefficient dominant et le coefficient noté $b_{0}$ du terme de degré 0.
\item Déterminer les racines de $A$ dans $\C$.
\item Montrer que
\[
P_{n}=\prod_{k=n+1}^{2n-1}\sin \frac{k \pi}{2n}.
\]
En déduire que $P_{n}=\sqrt{Q_{n}}$.
\item Calculer de deux façons le produit des racines de $B$. En déduire $Q_{n}$ puis $P_{n}$.
\item Déterminer la décomposition en éléments simples complexes de $F=\dfrac{1}{A}$.
\end{enumerate}

