%<dscrpt>Fichier de déclarations Latex à inclure au début d'un texte.</dscrpt>

\documentclass[a4paper,landscape,twocolumn]{article}
\usepackage[hmargin={.8cm,.8cm},vmargin={2.4cm,2.4cm},headheight=13.1pt]{geometry}

%includeheadfoot,scale=1.1,centering,hoffset=-0.5cm,
\usepackage[pdftex]{graphicx,color}
\usepackage{amsmath}
\usepackage{amssymb}
\usepackage{stmaryrd}
\usepackage[french]{babel}
%\selectlanguage{french}
\usepackage{fancyhdr}
\usepackage{floatflt}
\usepackage{ucs}
\usepackage[utf8]{inputenc}
\usepackage[T1]{fontenc}
%\usepackage[latin1]{inputenc}
\usepackage[pdftex,colorlinks={true},urlcolor={blue},pdfauthor={remy Nicolai}]{hyperref}

\usepackage{wrapfig}
%\usepackage{subcaption}
\usepackage{subfig}


%pr{\'e}sentation du compteur de niveau 2 dans les listes
\makeatletter
\renewcommand{\labelenumii}{\theenumii.}
\makeatother

%dimension des pages, en-t{\^e}te et bas de page
%\pdfpagewidth=20cm
%\pdfpageheight=14cm
%   \setlength{\oddsidemargin}{-2cm}
%   \setlength{\voffset}{-1.5cm}
%   \setlength{\textheight}{12cm}
%   \setlength{\textwidth}{25.2cm}
   \columnsep=0.4cm
   \columnseprule=0.3pt

%En tete et pied de page
\pagestyle{fancy}
\lhead{MPSI A - B}
\rhead{\today}
%\rhead{25/11/05}
\lfoot{\tiny{Cette création est mise à disposition selon le Contrat\\ Paternité-Partage des Conditions Initiales à l'Identique 2.0 France\\ disponible en ligne http://creativecommons.org/licenses/by-sa/2.0/fr/
} }
\rfoot{\tiny{Rémy Nicolai Benoît Saleur\jobname}}

\newcommand{\baseurl}{http://back.maquisdoc.net/data/devoirs_nicolair/}
\newcommand{\textesurl}{http://back.maquisdoc.net/data/devoirs_nicolair/}
\newcommand{\exosurl}{http://back.maquisdoc.net/data/exos_nicolair/}
\newcommand{\coursurl}{http://back.maquisdoc.net/data/cours_nicolair/}
\newcommand{\mwurl}{http://back.maquisdoc.net/data/maple_nicolair/}


\newcommand{\N}{\mathbb{N}}
\newcommand{\Z}{\mathbb{Z}}
\newcommand{\C}{\mathbb{C}}
\newcommand{\R}{\mathbb{R}}
\newcommand{\K}{\mathbf{K}}
\newcommand{\Q}{\mathbb{Q}}
\newcommand{\F}{\mathbf{F}}
\newcommand{\U}{\mathbb{U}}
\newcommand{\p}{\mathbb{P}}
\renewcommand{\P}{\mathbb{P}}

\newcommand{\card}{\operatorname{Card}}
\newcommand{\Id}{\operatorname{Id}}
\newcommand{\Ker}{\operatorname{Ker}}
\newcommand{\Vect}{\operatorname{Vect}}
\newcommand{\cotg}{\operatorname{cotan}}
\newcommand{\cotan}{\operatorname{cotan}}
\newcommand{\sh}{\operatorname{sh}}
\newcommand{\argsh}{\operatorname{argsh}}
\newcommand{\argch}{\operatorname{argch}}
\newcommand{\ch}{\operatorname{ch}}
\newcommand{\tr}{\operatorname{tr}}
\newcommand{\rg}{\operatorname{rg}}
\newcommand{\rang}{\operatorname{rg}}
\newcommand{\Mat}{\operatorname{Mat}}
\renewcommand{\cot}{\operatorname{cotan}}
\renewcommand{\Re}{\operatorname{Re}}
\newcommand{\Ima}{\operatorname{Im}}
\renewcommand{\Im}{\operatorname{Im}}
\renewcommand{\th}{\operatorname{th}}
\newcommand{\repere}{$(O,\overrightarrow{i},\overrightarrow{j},\overrightarrow{k})$}
\newcommand{\repereij}{$(O,\overrightarrow{i},\overrightarrow{j})$}
\newcommand{\zeron}{\llbracket 0,n\rrbracket}
\newcommand{\unAn}{\llbracket 1,n\rrbracket}

\newcommand{\IntEnt}[2]{\llbracket #1 , #2 \rrbracket}

\newcommand{\absolue}[1]{\left| #1 \right|}
\newcommand{\fonc}[5]{#1 : \begin{cases}#2 &\rightarrow #3 \\ #4 &\mapsto #5 \end{cases}}
\newcommand{\depar}[2]{\dfrac{\partial #1}{\partial #2}}
\newcommand{\norme}[1]{\left\| #1 \right\|}
\newcommand{\norm}[1]{\left\Vert#1\right\Vert}
\newcommand{\scal}[2]{\left\langle {#1} , {#2} \right\rangle}
\newcommand{\abs}[1]{\left\vert#1\right\vert}
\newcommand{\set}[1]{\left\{#1\right\}}
\newcommand{\se}{\geq}
\newcommand{\ie}{\leq}
\newcommand{\trans}{\,\mathstrut^t\!}
\renewcommand{\abs}[1]{\left\vert#1\right\vert}
\newcommand{\GL}{\operatorname{GL}}
\newcommand{\Cont}{\mathcal{C}}
\newcommand{\Lin}{\mathcal{L}}
\newcommand{\M}{\mathcal{M}}

\newcommand\empil[2]{\genfrac{}{}{0pt}{}{#1}{#2}}