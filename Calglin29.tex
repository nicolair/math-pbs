\begin{enumerate}
 \item Premières propriétés.
 \begin{enumerate}
  \item Avec les définitions de la dérivation et de la substitution polynomiale,
\[
 \Delta \circ D = D \circ \Delta \in \mathcal{L}(\R[X]).
\]

  \item Soit $P$ de degré $n$ et de coefficient dominant $a$. Les calculs sont immédiats, les termes de plus haut degré disparaissent: $D(P)$ et $\Delta(P)$ sont de degré $n-1$ et de coefficient dominant $na$.
  
  \item La famille $\mathcal{F}_n$ est libre car elle est constituée de polynômes de degrés échelonnés. C'est une base de $\R_n[X]$ car elle contient $n+1 = \dim (\R_n[X])$ vecteurs.\newline
  Les nombres de Stirling de première espèce sont les coordonnées des vecteurs de $\mathcal{F}$ dans $\mathcal{C}$. Ceux de deuxième espèce sont les coordonnées des vecteurs de $\mathcal{C}$ dans $\mathcal{F}$.
 \end{enumerate}

 \item Expressions récursives.
 \begin{enumerate}
  \item Comme $F_0 = 1$, $s(0,0) = \sigma(0,0) = 1$.\newline
  Pour $k\geq1$, $s(0,k)$ est le terme de degré $0$ de $F_k = X(X-1)\cdots$ c'est à dire $0$. De l'autre côté, $s(k,k)=1$ car c'est le coefficient dominant de $F_k = X(X-1)\cdots$.
  \[
   s(0,0) = 1, \forall k \in \N^*, s(0,k) = 0, \; s(k,k) = 1.
  \]
Par définition : $F_{k+1} = (X-k)F_k$. On en déduit
\begin{multline*}
 F_{k+1} = \sum_{i=0}^{k}s(i,k)(X-k)X^i
 = \sum_{i'=1}^{k+1}s(i'-1,k)X^{i'} - \sum_{i=0}^{k}ks(i,k)X^i \\
 = -k\underset{= 0}{\underbrace{s(0,i)}} + \sum_{i = 1}^{k}\left(s(i-1,k) - ks(i,k) \right) X^i + \underset{= 1}{\underbrace{s(k,k)}}X^{k+1}. 
\end{multline*}
On en déduit
\[
 \forall k \in \N^*, \forall i \in \llbracket 1,k \rrbracket, \; s(i,k+1) = s(i-1,k) - ks(i,k).
\]

  \item Pour $k\geq 1$, considérons la décomposition de $X^k$ dans $\mathcal{F}_n$. 
\[
 X^k = \sum_{i=0}^{k}\sigma(i,k)F_i.
\]
  En substituant $0$ à $X$, on obtient $\sigma(0,k)=0$. De l'autre côté, seul $F_k$ contient un $X^k$ donc $\sigma(k,k)=1$.
  \[
   \sigma(0,0) = 1, \forall k \in \N^*, \sigma(0,k) = 0, \; \sigma(k,k) = 1.
  \]
Par définition : $XF_i = (X-i +i)F_i = F_{i+1} + iF_i$. On en déduit:
\begin{multline*}
 X^{k+1} = \sum_{i=0}^{k}\sigma(i,k)(F_{i+1} + iF_i)
 = \sum_{i=1}^{k+1}\sigma(i-1,k)F_i + \sum_{i=0}^{k}\sigma(i,k)iF_i\\
 = \sum_{i=1}^{k}\left( \sigma(i-1,k) + i \sigma(i,k)\right) F_i + \sigma(k,k)F_{k+1}.
\end{multline*}
On en déduit
\[
 \forall k \in \N^*, \forall i \in \llbracket 1,k \rrbracket, \; \sigma(i,k+1) = \sigma(i-1,k) + i\sigma(i,k).
\]

  \item Les relations de la question précédente présentent une certaine analogie avec celles définissant les coefficients du binôme. Les $\sigma(i,4)$ forment la dernière ligne dans un tableau analogue au triangle de Pascal qui permet de les calculer récursivement.
\begin{center}
\renewcommand{\arraystretch}{1.5}
\vspace{0.2cm}
\begin{tabular}{|c|c|c|c|c|c|}\hline
$k\diagdown i$ & 0 & 1 & 2                   & 3 & 4 \\ \hline
0              & 1 &   &                     &   &    \\ \hline
1              & 0 & 1 &                     &   &    \\ \hline
2              & 0 & 1 & 1                   &   &   \\ \hline
3              & 0 & 1 & $2\times 1 + 1 = 3$ & 1 &   \\ \hline
4              & 0 & 1 & $2\times 3 + 1 = 7$ & $3\times 1 + 3 = 6$  & 1 \\ \hline
\end{tabular}
\end{center}

 \end{enumerate}

 \item Propriétés de $\Delta$.
 \begin{enumerate}
  \item On trouve $\Delta(F_k) = k F_{k-1}$. Avec en particulier $\Delta(F_0) = 0$.
  \item D'après la question précédente: $F_0 \in \ker(\Delta)$ donc $\Vect(F_0) \subset \ker(\Delta)$. Réciproquement, soit $P\in \ker(\Delta)$. Notons $n$ son degré, il se décompose dans $\mathcal{F}_n$:
\begin{multline*}
 P = \lambda_0 F_0 + \cdots + \lambda_n F_n 
 \Rightarrow 0 = \Delta(P) = \lambda_1 F_2 + 2\lambda_2 F_3 + \cdots + n\lambda_n F_{n-1}\\
 \Rightarrow \lambda_1 = \cdots = \lambda_n = 0
\end{multline*}
car la famille est libre. On en déduit $\ker(\Delta) = \Vect(F_0) = \R_0[X]$.\newline  
D'autre part 
\[
\Im(\Delta_n) =\Vect\left( \Delta(F_0), \cdots, \Delta(F_{n})\right) = \Vect\left( F_0, \cdots, F_{n-1}\right) = \R_{n-1}[X]. 
\]

  \item Soit $P$ un polynôme non nul de degré $n$ et $V=X\R_{n}[X]$.\newline
  C'est un sous-espace de $\R_{n+1}[X]$ supplémentaire de $\ker(\Delta_{n+1}) = \R_0[X]$. En effet, il est de dimension $n+1$ et son intersection avec $\R_0[X]$ se réduit au polynôme nul. D'après le théorème noyau image, $\Delta_{n+1}$ induit un isomorphisme entre $V$ et $\Im(\Delta_{n+1})=\R_n[X]$. Il existe donc un unique polyome $Q\in V$ c'est à dire divisible par $X$ tel que $\Delta(Q) = P$.\newline
  Pour calculer cet antécédent de $X^4$, utilisons $\mathcal{F}_4$:
\begin{multline*}
 X^4 = \sigma(1,4)F_1 + \sigma(2,4)F_2 + \sigma(3,4)F_3 + \sigma(4,4)F_4
 = F_1 + 7F_2 + 6F_3 + F_4 \\
 = \frac{1}{2}\Delta(F_2) + \frac{7}{3}\Delta(F_3) + \frac{6}{4}\Delta(F_4) + \frac{1}{5}\Delta(F_5)
 = \Delta\left( \frac{1}{2}F_2 + \frac{7}{3}F_3 + \frac{3}{2}F_4 + \frac{1}{5}F_5\right) 
\end{multline*}
L'unique antécédent de $X^4$ cherché est donc
\[
 Q = \frac{1}{2}F_2 + \frac{7}{3}F_3 + \frac{3}{2}F_4 + \frac{1}{5}F_5.
\]
Il est divisible par $F_2 = X(X-1)$.
 \end{enumerate}

 \item Application à un calcul de somme. D'après la question précédente: $X^4 = \widehat{Q}(X+1) - Q$. La somme est donc télescopique
 \[
  \sum_{i=1}^{n}i^4
  = \sum_{i=1}^{n}\left( \widetilde{Q}(i+1) - \widetilde{Q}(i)\right) 
  = \widetilde{Q}(n+1) - \widetilde{Q}(1) 
  = \widetilde{Q}(n+1)
 \]
car $\widetilde{F_2}(1)=\widetilde{F_3}(1)=\widetilde{F_4}(1)=\widetilde{F_5}(1)=0$. \newline
La fin du calcul ne présente pas d'intérêt et doit être sautée en temps limité:
\begin{multline*}
 \sum_{i=1}^{n}i^4 = (n+1)n\left[\frac{1}{2} + \frac{7}{3}(n-1) + \frac{3}{2}(n-1)(n-2) + \frac{1}{5}(n-1)(n-2)(n-3) \right] \\
 = \frac{1}{30}(n+1)n\left[ 15 + 70(n-1) + 45(n-1)(n-2) + 6(n-1)(n-2)(n-3)\right]\\
 = \frac{1}{30}(n+1)n\left[ 6n^3 + 9n^2 +n -1\right]
 = \frac{1}{30}(n+1)n(2n+1)(3n^2+3n-1).
\end{multline*}
La dernière factorisation s'effectuant à l'aide d'une division euclidienne.
 
 \item Somme d'opérateurs $D^i$.
 \begin{enumerate}
  \item L'expression fait bien intervenir une infinité d'endomorphismes mais pour une image particulière seul un nombre fini d'endormorphisme contribue à la somme. Plus précisément, pour un polynôme $P$ quelconque, notons $n$ son degré, alors:
\[
 \left( \sum_{i=1}^{+\infty}\frac{1}{i!}D^i\right)(P) = \sum_{i=1}^{n}\frac{1}{i!}D^i(P) 
\]
car les dérivées suivantes sont nulles.

  \item D'après la formule de Taylor pour les polynômes:
\[
 P = \sum_{i=0}^{n}\frac{\widetilde{D^i(P)}(a)}{i!}(X-a)^i.
\]
En substituant $X+a$ à $X$:
\[
 \widehat{P}(X+a) = \sum_{i=0}^{n}\frac{\widetilde{D^i(P)}(a)}{i!}X^i \hspace{0.5cm}(*).
\]
  \item Introduisons un polynôme $S$:
\[
 S = \sum_{i=0}^{n}\frac{1}{i!}D^i(P).
\]
En substituant $1$ à $X$ dans $(*)$, on obtient
\[
 \forall a \in \R, \; \widetilde{P}(a+1) = \sum_{i=0}^{n}\frac{\widetilde{D^i(P)}(a)}{i!} = \widetilde{S}(a).
\]
Comme ceci est valable pour tous les $a$ réels (une infinité), on déduit une égalité polynomiale
\[
 \widehat{P}(X+1) = S = P + \sum_{i=1}^n \frac{1}{i!}D^i(P).
\]
On peut conclure:
\[
 \left( \forall n \in \N, \forall P \in \R_n[X],\; \Delta(P) = \left(\sum_{i=1}^{n} \frac{1}{i!}D^i\right) (P)\right) 
 \Rightarrow \Delta = \sum_{i=1}^{+\infty} \frac{1}{i!}\,D^i.
\]
 \end{enumerate}

 \item Somme d'opérateurs $\Delta^i$.
 \begin{enumerate}
  \item D'après le calcul de $\Delta(F_k)$ déjà effectué,
\[
 \Delta^{i}(F_k) =
 \left\lbrace 
 \begin{aligned}
  &0 &\text{ si } i > k\\
  &k(k-1)\cdots (k-i+1)F_{k-i} &\text{ si } i\leq k
 \end{aligned}
\right. .
\]
Les $F_k$ sont divisibles par $X$ donc 
\[
 \widetilde{\Delta ^i(F_k)}(0) = 
 \left\lbrace 
 \begin{aligned}
  &0  &\text{ si } i \neq k \\
  &k! &\text{ si } i = k
 \end{aligned}
\right. .
\]

  \item On raisonne comme dans la démonstration de la formule de Taylor
\begin{multline*}
 P= \lambda_0 F_0 + \cdots + \lambda_nF_n \\
 \Rightarrow
 \widetilde{\Delta ^i(P)}(0) = \widetilde{\Delta ^i(\lambda_0 F_0 + \cdots + \lambda_nF_n)}(0)
 = \lambda_i i! F_0 
 \Rightarrow \lambda_i = \frac{\widetilde{\Delta ^i(P)}(0)}{i!}.
\end{multline*}

  \item De $\Delta(F_k)= k F_{k-1}$, on déduit $\Delta(E_k) = E_{k-1}$.\newline
  Notons $\lambda_0, \cdots, \lambda_n$ les coordonnées de $D(E_k)$ dans $\mathcal{F}$. D'après la question précédente
\begin{multline*}
 \lambda_{k-i} = \frac{1}{(k-i)!}\widetilde{(\Delta^{k-i} \circ D)(E_k)}(0) = \frac{1}{(k-i)!}\widetilde{(D \circ \Delta^{k-i})(E_k)}(0)\\
 = \frac{1}{(k-i)!}\widetilde{D(E_{i})}(0)
\end{multline*}
car $D$ et $\Delta$ commutent. Or
\begin{multline*}
 D(F_{i}) = \underset{i-1\text{ facteurs}}{\underbrace{(X-1)(X-2)\cdots}} + XD\left( (X-1)(X-2)\cdots\right) \\
 \Rightarrow
 \widetilde{D(E_{i})}(0) = \frac{(-1)^{i-1}(i-1)!}{i!}
 \Rightarrow
 \lambda_{k-i} = \frac{(-1)^{i-1}}{i\, (k-i)!}\\
 \Rightarrow
 D(E_k) = \sum_{i=1}^{k}\frac{(-1)^{i-1}}{i\, (k-i)!} F_{k-i}
 =\sum_{i=1}^{k}\frac{(-1)^{i-1}}{i}E_{k-i}.
\end{multline*}

  \item Comme $E_{k-i} = \Delta^i(E_k)$, les endomorphismes $D$  et $\sum_{i=0}^{+\infty}\frac{(-1)^{i-1}}{i}\Delta^i$ 
coïncident sur tous les polynomes $E_k$. Comme tout polynôme est une combinaison linéaire des $E_k$, on peut conclure:
\[
 D = \sum_{i=0}^{+\infty}\frac{(-1)^{i-1}}{i}\Delta^i.
\]

 \end{enumerate}

\end{enumerate}
