%<dscrpt>Sous-groupes centralisateurs.</dscrpt>
Dans cet exercice $G$ désigne un groupe dont l'opération est notée multiplicativement: pour tous  $a$ et $b$ de $G$ le produit de $a$ par $b$ est simplement noté $ab$. On ne suppose pas que le groupe soit commutatif.\newline
Pour une partie $A$ de $G$, on appelle \emph{centralisateur} de $A$ la partie $\mathcal C(A)$ de $G$ définie par :
\begin{displaymath}
 \forall x \in G, \; \left( x\in \mathcal C(A) \Leftrightarrow \forall a \in A, ax =xa\right) .
\end{displaymath}
Dans la suite de l'exercice, quand on demande de comparer deux parties de $G$, il s'agit de prouver une inclusion entre ces deux parties.\newline
La partie $A$ de $G$ est fixée pour la suite de l'exercice.
\begin{enumerate}
 \item Montrer que $\mathcal C(A)$ est un sous-groupe de $G$.
 \item  Soit $X$ et $Y$ deux parties de $G$ telles que $X\subset Y$. Comparer $\mathcal C(X)$ et $\mathcal C(Y)$.
 \item Soit $X$ une partie quelconque de $G$, comparer $X$ et $\mathcal C(\mathcal C(X))$.
 \item Montrer que 
\begin{displaymath}
\mathcal C( \mathcal C(\mathcal C(A))) = \mathcal C(A)
\end{displaymath}
\end{enumerate}
