%<dscrpt>Points cocycliques : quotient de distances d'un point à des cordes.</dscrpt>
Dans un plan rapporté à un repère orthonormé, les fonctions coordonnées sont notées $x$ et $y$. \newline
Les points $A$, $B$, $C$, $D$ respectivement d'affixes $e^{ia}$, $e^{ib}$, $e^{ic}$, $e^{id}$ sont sur le cercle unité.
\begin{enumerate}
\item En utilisant
\[e^{ia}+e^{ib}\;,\;e^{ia}-e^{ib}\]
préciser les coordonnées d'un vecteur unitaire orthogonal à la droite $(AB)$ et les coordonnées du milieu de $[A,B]$.
\item Former une équation de la droite  $(AB)$ (chaque coefficient contiendra seulement une fonction trigonométrique).
\item Soit $M_t$ le point d'affixe $e^{it}$, calculer la distance de $M_t$ à la droite $(AB)$. On la mettra sous la forme d'un produit de sinus. Peut-on écrire cette distance sans valeur absolue ?
\item Montrer que le quotient suivant est indépendant de $t$
\[\frac{d(M_t,(AB))\,d(M_t,(CD))}{d(M_t,(AC))\,d(M_t,(BD))}\]
\end{enumerate}