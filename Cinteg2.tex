\begin{enumerate}
\item  La fonction est continue et positive sur $\left] 0,+\infty \right[ .$
En $0$ elle est {\'e}quivalente {\`a} $\frac{1}{\sqrt{t}}$ qui est
int{\'e}grable sur $\left] 0,1\right] $ (primitive $2\sqrt{t}$
convergente en 0). En $+\infty $ elle est {\'e}quivalente {\`a}
$t^{-\frac{5}{2}}$ qui
est int{\'e}grable sur $\left[ 1,+\infty \right[ $ (primitive $-\frac{2}{3}%
t^{-\frac{3}{2}}$ convergente en $+\infty $).

Calcul de l'int{\'e}grale $I=\int_{0}^{+\infty }\frac{dt}{\sqrt{t}(1+t^{2})}$%
. Par le changement de variable $u=\sqrt{t}$, on obtient
\[
I=\int_{0}^{+\infty }\frac{2udu}{u(1+u^{4})}=2\int_{0}^{+\infty }\frac{du}{%
1+u^{4}}=\int_{-\infty }^{+\infty }\frac{du}{1+u^{4}}
\]

\emph{1}$^{\grave{e}re}$\emph{\ m{\'e}thode : fonctions {\`a} valeurs
complexes.}\newline Notons $z_{1},z_{2},z_{3},z_{4}$ les 4 p{\^o}les
de la fractions
\[
z_{1}=e^{i\frac{\pi }{4}},\ z_{2}=-z_{1},\ z_{3}=\overline{z}_{1},\ z_{4}=%
\bar{z}_{2}=-\overline{z}_{1}
\]
{\'E}crivons la d{\'e}composition en {\'e}l{\'e}ments simples
\[
\frac{1}{1+X^{4}}=\sum_{i=1}^{4}\frac{a_{i}}{X-z_{i}}
\]
avec $a_{2}=-a_{1}$, $a_{3}=\bar{a}_{1}$, $a_{4}=-\bar{a}_{1}$ car
la fraction est r{\'e}elle et paire et
\[
a_{1}=\frac{1}{4z_{1}^{3}}=\frac{-1}{4}z_{1}
\]
On en d{\'e}duit la primitive :
\[
\sum_{i=1}^{4}a_{i}\left\{ \ln \left| u-z_{i}\right| +i\arctan \frac{u-\mathrm{%
Re}z_{i}}{\mathrm{Im}z_{i}}\right\}
\]
La partie en $\arctan $ est born{\'e}e, la primitive converge {\`a}
l'infini la partie logarithmique converge donc vers $0$ {\`a}
l'infini. On en d{\'e}duit :
\[
I=\sum_{i=1}^{4}ia_{i}\text{Sgn}(\mathrm{Im}z_{i})\pi =i\pi
(a_{1}-a_{2}-a_{3}+a_{4})=2i\pi (a_{1}-\bar{a}_{1})=-4\pi
\mathrm{Im}a_{1}
\]
\[
I=\int_{0}^{+\infty }\frac{dt}{\sqrt{t}(1+t^{2})}=\frac{\pi
}{\sqrt{2}}
\]

\emph{2}$^{\grave{e}me}$\emph{\ m{\'e}thode : factorisation r{\'e}elle et
th{\'e}or{\`e}me de Bezout}\newline On commence par factoriser le
d{\'e}nominateur :
\[
1+u^{4}=(1+u^{2})^{2}-2u^{2}=(1+\sqrt{2}u+u^{2})(1-\sqrt{2}u+u^{2})
\]
Posons $A=1+\sqrt{2}u+u^{2}$, $B=1-\sqrt{2}u+u^{2}$ alors
\begin{eqnarray*}
A-B &=&2\sqrt{2}u,\quad B=1-2\sqrt{2}u(\frac{1}{2}-\frac{1}{2\sqrt{2}}u) \\
1 &=&(\frac{1}{2}-\frac{1}{2\sqrt{2}}u)\,A+(\frac{1}{2}+\frac{1}{2\sqrt{2}}%
u)\,B \\
\frac{1}{1+u^{4}} &=&\frac{\frac{1}{2}+\frac{1}{2\sqrt{2}}u}{1+\sqrt{2}%
u+u^{2}}+\frac{\frac{1}{2}-\frac{1}{2\sqrt{2}}u}{1-\sqrt{2}u+u^{2}}
\end{eqnarray*}
\begin{eqnarray*}
\frac{1}{1+u^{4}}
&=&\frac{1}{4\sqrt{2}}\frac{2u+\sqrt{2}}{1+\sqrt{2}u+u^{2}}+\frac{1}{4}\frac{1}{1+\sqrt{2}u+u^{2}}\\
&\phantom{f}& \quad -\frac{1}{4\sqrt{2}}\frac{2u-\sqrt{2}}{%
1-\sqrt{2}u+u^{2}}+\frac{1}{4}\frac{1}{1-\sqrt{2}u+u^{2}}
\end{eqnarray*}
On en d{\'e}duit la primitive
\[
\frac{1}{4\sqrt{2}}\ln \frac{1+\sqrt{2}u+u^{2}}{1-\sqrt{2}u+u^{2}}+\frac{1}{2%
\sqrt{2}}\arctan (\sqrt{2}u+1)+\frac{1}{2\sqrt{2}}\arctan
(\sqrt{2}u-1)
\]
En passant {\`a} la limite, on observe {\`a} nouveau l'annulation des
logarithmes et $I=\frac{\pi }{\sqrt{2}}$.

\item  La fonction est continue et positive sur $\left] 0,+\infty \right[ $.
En $0$ elle est {\'e}quivalente {\`a} $t^{1-\alpha }$ et en $+\infty $ {\`a}
$\frac{\pi }{2}t^{-\alpha }.$ Elle est int{\'e}grable sur $\left]
0,1\right] $ lorsque $1-\alpha >-1$ c'est {\`a} dire $\alpha <2$. Elle
est int{\'e}grable sur $\left[ 1,+\infty \right[ $ lorsque $-\alpha
<-1$ c'est {\`a} dire $\alpha >1$.\newline Finalement, la fonction est
int{\'e}grable lorsque $\alpha \in \left] 1,2\right[ $.\newline Soit
$J$ cette int{\'e}grale pour $\alpha =\frac{3}{2}$, une int{\'e}gration
par parties et un calcul de limite conduisent {\`a} l'int{\'e}grale de la
question 1.
\[
J=\underset{=0}{\underbrace{\left[ -2t^{-\frac{1}{2}}\arctan
t\right]
_{0}^{+\infty }}}+2\int_{0}^{+\infty }\frac{dt}{\sqrt{t}(1+t^{2})}=\sqrt{2}%
\pi
\]
\end{enumerate}
