%<dscrpt>Parties symétriques et antisymétriques d'une matrice.</dscrpt>
Soit $n \in \N^*$, on note $\mathcal{M} = \mathcal{M}_n(\R)$, $\mathcal{C} = \mathcal{M}_{n,1}(\R)$ (matrices colonnes), $\mathcal{L} = \mathcal{M}_{1,n}(\R)$ (matrices lignes), $\mathcal{O} = O_n(\R)$ (matrices orthogonales).\newline 
On note aussi $\mathcal{S}$ le sous-espace de $\mathcal{M}$ formé des matrices symétriques et $\mathcal{A}$ le sous-espace des matrices antisymétriques.\newline
Pour toute $M\in \mathcal{M}$, on définit sa \emph{partie symétrique} (notée $M_s$) et sa \emph{partie antisymétrique} (notée $M_a$) par
\[
 \forall M \in \mathcal{M}: \; M_s = \frac{1}{2}(M + \trans{M}), \;M_a = \frac{1}{2}(M - \trans{M}).
\]
On introduit les produits scalaires habituels dans $\mathcal{C}$ et $\mathcal{M}$ :
\[
\begin{aligned}
\forall (X,Y) \in \mathcal{C}^2, \; (X/Y) &= \trans{X} Y \\ 
\forall (A,B) \in \mathcal{M}^2, \; (A/B) &= \tr(\trans{A} B)
\end{aligned}
\]
Les normes sont notées $\Vert \:\Vert$. Le contexte permettra de distinguer entre la norme et le produit scalaire dans $\mathcal{C}$ ou dans $\mathcal{M}$.\newline
Une matrice $M\in \mathcal{M}$ est dite \emph{normale} si et seulement si $\trans{M} M = M \trans{M}$.
\subsection*{I. Parties symétriques et antisymétriques.}
\begin{enumerate}
 \item
 \begin{enumerate}
  \item Préciser des bases pour $\mathcal{S}$ et $\mathcal{A}$. En déduire leurs dimensions.
  \item Montrer que $\mathcal{S}^\bot = \mathcal{A}$ et $\mathcal{S} = \mathcal{A}^\bot$.
  \item Préciser les projections orthogonales sur $\mathcal{S}$ et $\mathcal{A}$ d'une matrice $M\in \mathcal{M}$.
 \end{enumerate}
 
 \item Soit $M\in \mathcal{M}$. Exprimer $\trans{M} M$ en fonction de $M_s$ et $M_a$. En déduire que $M$ est \emph{normale}) si et seulement $M_a M_s = M_s M_a$.

 \item Soit $M\in \mathcal{M}$ inversible. 
 \begin{enumerate}
  \item  Montrer que 
\[
 \forall (S,A) \in \mathcal{S} \times\mathcal{A}: \hspace{0.5cm} M S \trans{M} \in \mathcal{S}, \;  M A \trans{M} \in \mathcal{A}.
\]
  \item Montrer que
\[
 M_s = M (M^{-1})_s \trans{M}, \hspace{0.5cm} M_a = - M (M^{-1})_a \trans{M}.
\]
  \item Montrer que $\det(M_s) = (\det(M))^2 \det((M^{-1})_s)$.
 \end{enumerate}

\end{enumerate}

\subsection*{II. Sous-espaces stables.}
Soit $M\in \mathcal{M}$, on note $\mu_M$ l'application
\[
 \begin{aligned}
  \mathcal{C} &\rightarrow \mathcal{C} \\ X &\mapsto MX
 \end{aligned}
\]
On pourra désigner $\ker \mu_M$ par $\ker M$ et $\Im \mu_M$ par $\Im M$. Ce sont des sous-espaces de $\mathcal{C}$.\newline 
Soit $\mathcal{U}$ un sous-espace de $\mathcal{C}$ de dimension $r$ stable par $\mu_M$.\newline
Soit $(C_1, \cdots, C_n)$ une base orthonormée de $\mathcal{C}$ telle que $(C_1, \cdots, C_r)$ soit une base orthonormée de $\mathcal{U}$.\newline
On note $P \in \mathcal{M}$ la matrice dont les colonnes sont $(C_1, \cdots, C_n)$ et $Q \in \mathcal{M}_{n,r}(\R)$ celle dont les colonnes sont $(C_1, \cdots, C_r)$.
\begin{enumerate}
 \item Montrer que $\rg(M) = \rg(\mu_M)$.

 \item Changement de base.
 \begin{enumerate}
  \item Exprimer la matrice de $\mu_M$ dans la base $(C_1, \cdots, C_n)$ en fonction de $M$ et $P$. On la note $M'$.
  \item Exprimer la matrice $R$ de la restriction de $\mu_M$ à $\mathcal{U}$ dans la base $(C_1, \cdots, C_r)$ en fonction de $M$ et $Q$. En déduire que si $M$ est symétrique (respectivement antisymétrique) alors $R$ est symétrique (respectivement antisymétrique).
 \end{enumerate}  

 \item Dans cette question, on suppose que $M$ est normale.
 \begin{enumerate}
  \item Montrer que $M'$ est normale.
  \item On écrit $M'$ avec des blocs
\[
 M' = 
 \begin{pmatrix}
  M_1 & M_2 \\ M_3 & M_4
 \end{pmatrix}
\text{ avec } M_1 \in \mathcal{M}_{r}(\R).
\]
Montrer que $M_3$ est une matrice nulle. Montrer que $M_2$ est une matrice nulle. En déduire que $\mathcal{U}^{\bot}$ est stable par $\mu_M$ et que $M_1$ et $M_4$ sont normales.
 \end{enumerate}
 
 \item Dans cette question on s'intéresse aux matrices antisymétriques.
\begin{enumerate}
  \item Montrer que s'il existe une matrice antisymétrique inversible réelle de taille $m\times m$ alors $m$ est pair.
 
 \item Montrer que 
 \[
  \forall A \in \mathcal{A}, \forall X \in \mathcal{C}, \; \trans{X}A X = 0.
 \]
 
 \item Montrer que $\Im A = (\ker A)^\bot$ et que $\rg(A)$ est pair.
 
\end{enumerate}
\end{enumerate}

\subsection*{III. Sous-espaces stables irréductibles.}
Soit $M\in\mathcal{M}$ et $\lambda \in \C$.\newline
On dit que $\lambda$ est une \emph{valeur propre complexe} de $M$ si et seulement si 
\[
 \exists Z \in \mathcal{M}_{n,1}(\C) \text{ tq } 
 Z \neq
 \begin{pmatrix}
  0 \\ \vdots \\ 0
 \end{pmatrix}
\text{ et }
MZ = \lambda Z.
\]
La colonne $Z$ est alors un vecteur propre de valeur propre $\lambda$. On note $\text{Sp}_\C(M)$ l'ensemble des valeurs propres complexes de $M$. On admet que c'est une partie non vide de cardinal inférieur ou égal à $n$.\newline
On dit que $\lambda \in \R$ est une \emph{valeur propre réelle} de $M$ si et seulement si 
\[
 \exists X \in \mathcal{C} \text{ tq } 
 X \neq
 \begin{pmatrix}
  0 \\ \vdots \\ 0
 \end{pmatrix}
\text{ et }
MX = \lambda X.
\]
On note $\text{Sp}_\R(M)$ l'ensemble des valeurs propres réelles de $M$.

\begin{enumerate}
 \item Soit $M\in\mathcal{M}$. Soit $\lambda \in \C \setminus \R$ une valeur propre complexe non réelle et $Z$ un vecteur propre associé. On définit $X$ et $Y$ dans $\mathcal{C}$ tels que $Z = X + iY$ par:
\[
 Z = 
\begin{pmatrix}
 z_1 \\ \vdots \\ z_n
\end{pmatrix}, \;
 X = 
\begin{pmatrix}
 \Re(z_1) \\ \vdots \\ \Re(z_n)
\end{pmatrix}, \;
 Y = 
\begin{pmatrix}
 \Im(z_1) \\ \vdots \\ \Im(z_n)
\end{pmatrix}.
\]
\begin{enumerate}
 \item Montrer que $(X,Y)$ est une famille libre de $\mathcal{C}$.
 \item Montrer que $\mathcal{U} = \Vect(X,Y)$ est stable par $\mu_M$ et former la matrice de la restriction de  $\mu_M$ à $\mathcal{U}$ dans la base $(X,Y)$.
\end{enumerate}
\item 
\begin{enumerate}
 \item Soit $S\in \mathcal{S}$. Montrer que $\text{Sp}_\C(S) = \text{Sp}_\R(S)$.

 \item Soit $A\in \mathcal{A}$. Montrer que $\text{Sp}_\C(A) \subset i\R$.
\end{enumerate}

 
 \item Soit $P \in \mathcal{O}$. On note $S = P_s$ et $A = P_a$.
 \begin{enumerate}
  \item Montrer que $\text{Sp}_\R(P) \subset \left\lbrace -1, +1 \right\rbrace$. 
  \item Soit $\lambda$ une valeur propre complexe non réelle de $P$. Montrer qu'il existe $\theta \in \left] -\pi , \pi\right[$ tel que $\lambda = e^{i \theta}$. 
  \item Montrer que $\ker A = \ker(P-I) \stackrel{\bot}{\oplus} \ker(P+I)$.
 \end{enumerate}

\end{enumerate}

