%<dscrpt>Trigonométrie, nombres complexes, produit.</dscrpt>
\subsection*{Exercice 1}
Former le tableau des signes de $\cos(2x)$ et $2\cos x -1$ pour $x$ dans $]-\pi,\pi]$. Déterminer l'ensemble des $x$ de $]-\pi,\pi]$ tels que
\begin{displaymath}
 \cos x + \cos(3x) > \cos(2x)
\end{displaymath}

\subsection*{Exercice 2}
Soit $a$, $b$, $c$ trois nombres complexes de module $1$ et deux à deux distincts. On considère
\begin{displaymath}
 T = \frac{b(c-a)^2}{a(c-b)^2}
\end{displaymath}
\begin{enumerate}
 \item On pose $w = \frac{c-a}{c-b}$. Exprimer $\overline{w}$ en fonction de $a$, $b$, $c$ puis exprimer $T$ avec un module.
 \item Exprimer $T$ en utilisant des arguments $\alpha$, $\beta$, $\gamma$ de $a$, $b$, $c$.
 \item Interpréter géométriquement le résultat $T\in \R_+$ démontré de deux manières différentes dans les questions précédentes.
\end{enumerate}

\subsection*{Exercice 3}
Soit $n\in \N^*$ et $T_n=\{(i,j)\in\N^2\text{ tq } 1\leq i < j \leq n\}$. On considère $P_n=\prod_{(i,j)\in T_n}ij$. On se propose de calculer ce produit de deux manières différentes.
\begin{enumerate}
 \item On note $u_j=\prod_{i=1}^{j-1}(ij)$ et $v_j=(j!)^{j-1}$ pour $j\geq 2$ entier.
\begin{enumerate}
 \item Simplifier $\frac{v_j}{v_{j-1}}$ pour $j\geq 2$.
 \item Exprimer $u_j$ à l'aide d'une factorielle et d'une puissance.
 \item En déduire une expression de $P_n$.
\end{enumerate}
\item On pose
\begin{align*}
 &T'_n=\{(i,j)\in\N^2\text{ tq } 1\leq j < i \leq n\} & P'_n = \prod_{(i,j)\in T'_n}ij\\
 &D_n=\{(i,i)\in\N^2\text{ tq } 1\leq i \leq n\} &  \pi_n = \prod_{(i,j)\in D_n}ij
\end{align*}

\begin{enumerate}
 \item Calculer
\begin{displaymath}
 \Pi_n = \prod_{(i,j)\in \{1,\cdots n\}^2 }(ij)
\end{displaymath}
 \item Que vaut le produit $P_n\pi_nP'_n$ ?
 \item Montrer que $P_n=P'_n$.
 \item En déduire l'expression de $P_n$.
\end{enumerate}

\end{enumerate}
