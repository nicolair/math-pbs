%<dscrpt>Ellipse inscrite aux milieux des côtés d'un triangle.</dscrpt>
On se place dans un plan $\mathcal P$ rapporté à un repère orthonormé d'origine $O$. Dans tout le problème, $\alpha \in ]0,\pi[ -\{\frac{\pi}{2}\}$.\newline
On définit des applications $a$, $h$, $f$ de $\C$ dans $\C$ par les formules suivantes valables pour tout $z\in \C$:
\begin{eqnarray*}
 a(z) &=& z\cos^2\frac{\alpha}{2}+\overline{z}\sin^2\frac{\alpha}{2}\\
h(z) &=& \frac{2}{\sin \alpha}z \\
f(z) &=& h\circ a (z)
\end{eqnarray*}
On note $A$, $H$, $F$ les transformations du plan qui à un point d'affixe $z$ associent respectivement les points d'affixes $a(z)$, $h(z)$, $f(z)$.

On note $I$, $J$, $K$ les points respectivement d'affixes $1$, $j$, $j^2$ et  $U$, $V$, $W$ les points respectivement d'affixes $u=f(1)$, $v=f(j)$, $w=f(j^2)$.

On note enfin $\mathcal C$ le cercle de centre $O$ et de rayon $\frac{1}{2}$

\subsubsection*{Partie I}
\begin{enumerate}
 \item Soit $M$ un point du plan de coordonnées $(x,y)$, calculer les coordonnées des points $A(M)$, $H(M)$, $F(M)$. Préciser la nature des transformations $A$ et $H$.
\item Montrer que l'image de $\mathcal C$ par $F$ est une conique (notée $\mathcal E$). Préciser le genre et les foyers de $\mathcal E$.
\end{enumerate}

\subsubsection*{Partie II}
\begin{enumerate}
 \item Former les équations des droites $(IJ)$, $(IK)$, $(JK)$. Exprimer la distance d'un point $M$ de coordonnées $(x,y)$ à ces droites.
\item Montrer que $\mathcal C$ est le cercle inscrit dans le triangle $(IJK)$.
\item Montrer que chacun des segments $[UV]$, $[UW]$, $[VW]$ est tangent en son milieu à la conique $\mathcal E$.
\end{enumerate}

\subsubsection*{Partie III}
\begin{enumerate}
 \item Montrer que pour tout $z$ complexe :
\[f(z)=\frac{z}{\tan \frac{\alpha}{2}}+\overline{z}\tan \frac{\alpha}{2}\]
\item Calculer $u+v+w$  et $uv+uw+vw$.
\item En déduire les racines de $P^\prime(x)$ (dérivée formelle) du polynôme
\[P(x)=(x-u)(x-v)(x-w)\] 

Ceci démontre dans un cas particulier le théorème de van der Berg \footnote{d'après \emph{Polynomials}, Prasolov, Springer}:
\begin{quote}
 Lorsque les sommets d'un triangle forment les trois racines d'un polynôme, l'ellipse tangente au milieu de chaque côté admet pour foyers les racines du polynôme dérivé.
\end{quote} 
\end{enumerate}

\subsubsection*{Partie IV}
\begin{enumerate}
 \item Soit $(z_0,z_1,z_2)$ trois nombres complexes. On dit que $(z_0,z_1,z_2)$ vérifie $(*)$ lorsque :
\begin{displaymath}
% use packages: array
\left\lbrace \begin{array}{lll}
z_0+z_1+z_2 & = & 0 \\ 
z_0z_1+z_0z_2+z_1z_2 & = & -3
\end{array} \right. 
\end{displaymath}
Montrer que $(z_0,z_1,z_2)$ vérifie $(*)$ si et seulement si $z_1$ et $z_2$ sont les racines d'une certaine équation du second degré à préciser.
\item Former un triplet $(z_0,z_1,z_2)$ vérifiant $(*)$ avec $z_0=4$. Trouver un $\alpha$ tel que
\[\begin{array}{ccc}
z_0=f(1) & z_1=f(j) &z_2=f(j^2)
\end{array}
\]
\end{enumerate}


