
\begin{enumerate}
\item \begin{enumerate}
 \item Il s'agit d'une question de cours. La distance d'un point $M$ à une droite $\mathcal D$ passant par un point $A$ et de vecteur directeur $\overrightarrow u$ est 
\begin{displaymath}
 d(M,\mathcal D)=\dfrac{\Vert \overrightarrow{AM}\wedge \overrightarrow u\Vert}{\Vert \overrightarrow u\Vert}
\end{displaymath}
En effet, notons $K_M$ le projeté orthogonal de $M$ sur $\mathcal D$ et décomposons dans le produit vectoriel
\begin{displaymath}
 \overrightarrow{AM}=\overrightarrow{AK_M}+\overrightarrow{K_MM}
\end{displaymath}
Comme $\overrightarrow{AK_M}$ est colinéaire à $\overrightarrow u$, il disparait dans le produit vectoriel et on obtient
\begin{displaymath}
 \dfrac{\Vert \overrightarrow{AM}\wedge \overrightarrow u\Vert}{\Vert \overrightarrow u\Vert}
=\dfrac{\Vert \overrightarrow{K_MM}\wedge \overrightarrow u\Vert}{\Vert \overrightarrow u\Vert}
= \dfrac{\Vert \overrightarrow{K_MM}\Vert \Vert \overrightarrow u\Vert \sin \theta}{\Vert \overrightarrow u\Vert}
= \Vert \overrightarrow{K_MM}\Vert =  d(M, D)
\end{displaymath}
car l'écart angulaire $\theta$ entre $\overrightarrow{k_MM}$ et $\overrightarrow u$ est $\frac{\pi}{2}$.
\item Il s'agit encore d'une question de cours. On introduit $H$ et $H'$ dans $\overrightarrow{AA'}$. Les termes en $\overrightarrow{AH}$ et $\overrightarrow{H'A'}$ disparaissent car les vecteurs sont respectivement colinéaires à $\overrightarrow u$ et $\overrightarrow u'$
\begin{displaymath}
 \dfrac{\det(\overrightarrow{AA'},\overrightarrow u,\overrightarrow{u'})}{\Vert \overrightarrow u \wedge \overrightarrow{u'} \Vert}
=\dfrac{\vert\det(\overrightarrow{HH'},\overrightarrow u,\overrightarrow{u'})\vert}{\Vert \overrightarrow u \wedge \overrightarrow{u'} \Vert}
=\dfrac{\vert(\overrightarrow u \wedge \overrightarrow{u'} / \overrightarrow{HH'})\vert}{\Vert \overrightarrow u \wedge \overrightarrow{u'}\Vert}
= \Vert \overrightarrow{HH'}\Vert
\end{displaymath}
car $\overrightarrow{HH'}$ et $\overrightarrow u \wedge \overrightarrow{u'}$ sont colinéaires.
\end{enumerate}
\begin{figure}[ht]
 \centering
\input{Csurfbiss_1.pdf_t}
\caption{Intersection avec le plan $x=0$}
\label{fig:Csurfbiss_1}
\end{figure}
\begin{figure}[ht]
 \centering
\input{Csurfbiss_2.pdf_t}
\caption{Intersection avec le plan $y=0$}
\label{fig:Csurfbiss_2}
\end{figure}
\begin{figure}[ht]
 \centering
\input{Csurfbiss_3.pdf_t}
\caption{Intersection avec le plan $z=0$}
\label{fig:Csurfbiss_3}
\end{figure}
\item \begin{enumerate}
 \item On utilise la formule de la question 1.a. dans ce cas particulier pour former l'équation de $X$
\begin{align*}
 &\text{Coordonnées de }\overrightarrow{AM}\wedge \overrightarrow{u} :&
\begin{pmatrix}
 x\\y\\z
\end{pmatrix}
 \wedge
\begin{pmatrix}
 0\\1\\0
\end{pmatrix}
=&
\begin{pmatrix}
 -z\\0\\x
\end{pmatrix}
\\
 &\text{Coordonnées de }\overrightarrow{A'M}\wedge \overrightarrow{u'} :&
\begin{pmatrix}
 x\\y\\z-1
\end{pmatrix}
 \wedge
\begin{pmatrix}
 1\\0\\0
\end{pmatrix}
=&
\begin{pmatrix}
 0\\z-1\\-y
\end{pmatrix}
\end{align*}
On en déduit l'équation de $X$:
\begin{displaymath}
 x^2+z^2=(z-1)^2+y^2 \Leftrightarrow
x^2=1+y^2-2z
\end{displaymath}
\item 
\begin{itemize}
 \item L'intersection de $X$ avec le plan $yOz$ d'équation $x=0$ est une parabole d'équation $-2z+1+y^2=0$ (figure :\ref{fig:Csurfbiss_1}).
\item L'intersection de $X$ avec le plan $xOy$ d'équation $z=0$ est une parabole d'équation $x^2=-2z+1$ (figure :\ref{fig:Csurfbiss_2}).
\item L'intersection de $X$ avec le plan $xOy$ d'équation $z=0$ est une hyperbole d'équation $x^2=1+y^2$ (figure :\ref{fig:Csurfbiss_3}). 
\end{itemize}
\end{enumerate}

\begin{figure}[ht]
 \centering
\input{Csurfbiss_4.pdf_t}
\caption{Repère pour la question 3.}
\label{fig:Csurfbiss_4}
\end{figure}

\item \begin{enumerate}
 \item On choisit $\overrightarrow K = \overrightarrow I \wedge \overrightarrow J$ pour que la base $(\overrightarrow I , \overrightarrow J , \overrightarrow K)$ soit orthonormée directe.
\item Par définition, $\overrightarrow I$ et $\overrightarrow J$ sont des vecteurs directeurs des bissectrices des droites dirigées par $\overrightarrow u$ et $\overrightarrow u'$ et $O'$ est le milieu de $(H,H')$.\newline
Suivant comment est orienté $\overrightarrow K$, il existe $\varepsilon\in\{-1,+1\}$ tel que les coordonnées de $H$ et $H'$ soient respectivement 
\begin{displaymath}
 \begin{pmatrix}
  0\\0\\ \varepsilon\frac{d}{2}
 \end{pmatrix}
\text{ et }
 \begin{pmatrix}
  0\\0\\ \varepsilon'\frac{d}{2}
 \end{pmatrix}
\end{displaymath}
avec $\varepsilon'=-\varepsilon$. Sur la figure \ref{fig:Csurfbiss_4} par exemple $\varepsilon =-1$.
Il existe alors un réel $m$ tel que $\overrightarrow u'$ soit colinéaire à $\overrightarrow I +\varepsilon' m\overrightarrow J$ et $\overrightarrow u$ soit colinéaire à $\overrightarrow I +\varepsilon m\overrightarrow J$. Sur la figure  par exemple ce réel $m$ est strictement positif.\newline
Avec ces conventions, les équations de $D$ et $D'$ s'écrivent respectivement:
\begin{align*}
 \left\lbrace 
\begin{aligned}
 y =& \varepsilon m x\\
 z =& \varepsilon \frac{d}{2}
\end{aligned}
\right. 
& &
 \left\lbrace 
\begin{aligned}
 y =& \varepsilon' m x\\
 z =& \varepsilon' \frac{d}{2}
\end{aligned}
\right. 
\end{align*}
Par exemple dans le cas de la figure :
\begin{align*}
 \text{équation de $D$ : }
\left\lbrace 
\begin{aligned}
 y =& - m x\\
 z =& -\frac{d}{2}
\end{aligned}
\right. 
 & &
 \text{équation de $D'$ : }
\left\lbrace 
\begin{aligned}
 y =& + m x\\
 z =& \frac{d}{2}
\end{aligned}
\right. 
\end{align*}

\item Calculons d'abord la distance d'un point à une droite dont les équations sont de la forme de la question précédente en conservant le $\varepsilon$ :
\begin{displaymath}
 \begin{pmatrix}
  x\\y\\z-\varepsilon \frac{d}{2}
 \end{pmatrix}
\wedge
\begin{pmatrix}
 1\\ \varepsilon m \\0
\end{pmatrix}
=
\begin{pmatrix}
 \varepsilon mz -\frac{md}{2}\\ z-\varepsilon \frac{d}{2} \\ \varepsilon mx -y
\end{pmatrix}
\end{displaymath}
en égalant les normes, on forme l'équation de $X$:
\begin{multline*}
 (mz-\frac{md}{2})^2+(z-\frac{d}{2})^2+(mx-y)^2
=
(mz+\frac{md}{2})^2+(z+\frac{d}{2})^2+(mx+y)^2
\\
\Leftrightarrow
(m^2+1)dz+2mxy=0
\end{multline*}

\end{enumerate}

\item Soit $M$ un point de $X$ dont les coordonnées sont $(x_0,y_0,z_0)$ dans le repère de la question précédente. Cherchons les conditions à imposer aux coordonnées $(u,v,w)$ d'un vecteur $\overrightarrow a$ pour que, pour tous $\lambda$ réels, le point $M_\lambda = A+\lambda \overrightarrow a$ appartienne à $X$.\newline
On introduit les coordonnées de $M_\lambda$ dans l'équation de $X$ et on réordonne suivant les puissances de $\lambda$ :
\begin{multline*}
 M_\lambda \in X \Leftrightarrow
(m^2+1)d(z_0+\lambda w)+2m(x_0+\lambda u )(y_0+\lambda v)=0 \\
\Leftrightarrow
\underset{=0}{\underbrace{(m+1)dz_0+2mx_0y_0}} + \left( (m^2+1)dw+2mx_0v +2muy_0\right) \lambda +2muv\lambda ^2 =0
\end{multline*}
ceci est réalisé \emph{pour tous les} $\lambda$ lorsque $u$, $v$, $w$ annulent les coefficients soit
\begin{align*}
 \left\lbrace 
\begin{aligned}
 u =& 0 \\
(m^2+1)dw+2mx_0v =& 0
\end{aligned}
\right. 
 & & 
 \left\lbrace 
\begin{aligned}
 v =& 0 \\
(m^2+1)dw+2my_0u =& 0
\end{aligned}
\right. 
\end{align*}
Par chaque point $M$ de $X$ dont les coordonnées sont $(x_0,y_0,z_0)$ passent donc deux droites incluses dans $X$ dont les vecteurs directeurs ont pour coordonnées :
\begin{align*}
 \begin{pmatrix}
  0 \\ (m^2+1)d \\ -2mx_0
 \end{pmatrix}
& & 
\begin{pmatrix}
 (m^2+1)d  \\ 0 \\-2my_0
\end{pmatrix}
\end{align*}

\item \begin{enumerate}
 \item Décomposons le carré de la norme en utilisant les propriétés du produit scalaire et réordonnons suivant les puissance de $\lambda$
\begin{multline*}
 d(P_\lambda,D)^2 = \left\Vert(\overrightarrow(AM)+\lambda \overrightarrow w) \wedge \overrightarrow u\right\Vert^2 \\
= \left\Vert \overrightarrow{AM}\wedge \overrightarrow u\right\Vert ^2
 +\left(\overrightarrow{AM}\wedge \overrightarrow u / \overrightarrow{w}\wedge \overrightarrow u\right) \lambda
 +\left\Vert \overrightarrow{w}\wedge \overrightarrow u\right\Vert ^2 \lambda^2
\end{multline*}

\item Le coefficient de $\lambda$ dans le développement précédent s'exprime à l'aide d'un produit mixte
\begin{displaymath}
 \left(\overrightarrow{AM}\wedge \overrightarrow u / \overrightarrow{w}\wedge \overrightarrow u\right)
=
\det(\overrightarrow{AM}, \overrightarrow u , \overrightarrow{w}\wedge \overrightarrow u)
\end{displaymath}
On en déduit que les relations $(1)$ et $(2)$ entraînent l'égalité des coefficients de $\lambda$ et $\lambda^2$. Lorsque de plus $M\in X$ alors 
\begin{displaymath}
 \left\Vert \overrightarrow{AM}\wedge \overrightarrow u\right\Vert =
\left\Vert \overrightarrow{A'M}\wedge \overrightarrow u'\right\Vert
\end{displaymath}
Alors , pour tous $\lambda$, le point $P_\lambda$ est à égale distance de $D$ et de $D'$ donc sur $X$.
\item Transformons la relation $(1)$ :
\begin{multline*}
 \left\Vert \overrightarrow{w}\wedge \overrightarrow u\right\Vert ^2
=
\left\Vert \overrightarrow{w}\wedge \overrightarrow u'\right\Vert ^2
\Leftrightarrow
\left( 
\overrightarrow{w}\wedge \overrightarrow u + \overrightarrow{w}\wedge \overrightarrow u'
/
\overrightarrow{w}\wedge \overrightarrow u - \overrightarrow{w}\wedge \overrightarrow u'
\right) =0\\
\Leftrightarrow
\left( 
\overrightarrow{w}\wedge (\overrightarrow u + \overrightarrow u') 
/
\overrightarrow{w}\wedge (\overrightarrow u - \overrightarrow u')
\right) =0 \\
\Leftrightarrow
\det(
\overrightarrow w
,
\overrightarrow u + \overrightarrow u'
,
\overrightarrow{w}\wedge (\overrightarrow u - \overrightarrow u')
)=0 \\
\Leftrightarrow
\det(
(\overrightarrow u' - \overrightarrow u)\wedge \overrightarrow{w},
\overrightarrow u + \overrightarrow u' ,
\overrightarrow w
)=0 \\
\Leftrightarrow
\left( 
\left( (\overrightarrow u' - \overrightarrow u)\wedge \overrightarrow w\right) 
\wedge(\overrightarrow u + \overrightarrow u')
/
\overrightarrow w
\right) =0
\end{multline*}
On utilise ensuite la formule du double produit vectoriel
\begin{displaymath}
\left(  (\overrightarrow u' - \overrightarrow u)\wedge \overrightarrow w \right) 
\wedge(\overrightarrow u + \overrightarrow u')
=
\underset{=0}{\underbrace{\left( \overrightarrow u - \overrightarrow u'/\overrightarrow u + \overrightarrow u'\right)}} \overrightarrow w
-
\left( \overrightarrow w / \overrightarrow u + \overrightarrow u'\right)(\overrightarrow u' - \overrightarrow u) 
\end{displaymath}
car $\overrightarrow u$ et $\overrightarrow u'$ sont unitaires. La condition s'écrit alors :
\begin{displaymath}
 \left( \overrightarrow w / \overrightarrow u + \overrightarrow u'\right) 
\left( \overrightarrow w / \overrightarrow u - \overrightarrow u'\right) = 0
\end{displaymath}
Traduisant exactement $\overrightarrow w\in \mathcal P^+ \cup \mathcal P^+$
\item Exprimons le déterminant intervenant dans la condition $(2)$ comme un produit scalaire contre $\overrightarrow w$.
\begin{multline*}
 \det(\overrightarrow{AM},\overrightarrow u , \overrightarrow w \wedge \overrightarrow u)
= \det(\overrightarrow u \wedge \overrightarrow w , \overrightarrow u ,\overrightarrow{AM}) 
= \left( (\overrightarrow u \wedge \overrightarrow w)\overrightarrow u/ \overrightarrow{AM}\right) \\
=\left(
\Vert \overrightarrow u\Vert^2\overrightarrow w-(\overrightarrow w/\overrightarrow u)\overrightarrow u 
/
\overrightarrow{AM}
\right) 
=(\overrightarrow w / \overrightarrow{AM})- (\overrightarrow w/\overrightarrow u) (\overrightarrow u/\overrightarrow{AM}) \\
= \left( 
\overrightarrow w
/
\overrightarrow{AM} -(\overrightarrow u/\overrightarrow{AM})\overrightarrow u
\right) 
\end{multline*}
Un calcul analogue est évidemment valable pour $A$ et $\overrightarrow u'$. Les deux déterminants sont égaux si et seulement si $\overrightarrow w$ est dans le plan (noté $\Pi$) orthogonal à la différence des deux vecteurs soit
\begin{displaymath}
 \overrightarrow{AA'}-(\overrightarrow u / \overrightarrow{AM})\overrightarrow u
                     +(\overrightarrow u' / \overrightarrow{A'M})\overrightarrow u'
\end{displaymath}

\item Avec les traductions de chaque condition que l'on vient de trouver, il apparait que la droite $M+\Vect(\overrightarrow w)$ est incluse dans $X$ si et seulement si $\overrightarrow w \in \Pi \cap \mathcal P^+$ ou $\overrightarrow w \in \Pi \cap \mathcal P^-$. On retrouve bien que , par chaque point de $X$ passent deux droites incluses dans $X$.
\end{enumerate}

\end{enumerate}
