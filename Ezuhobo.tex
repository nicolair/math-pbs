%<dscrpt>Transformations de Zhukowskii et de Bohlin. Plan complexe, coniques, équations différentielles.</dscrpt>
Soit $\mathcal P$ un plan euclidien muni d'un repère orthonormé $(O,\overrightarrow{i},\overrightarrow{j})$. On note $x$ et $y$ les fonctions coordonnées dans ce repère.
On pourra dans tout le problème identifier un point de $\mathcal P$ avec son affixe complexe.  \newline
Ainsi $x(z), y(z)$ désignent aussi bien les coordonnées du point d'affixe $z$ que les parties réelle et imaginaire de $z$ 
\subsection*{Préliminaire}
Montrer que l'image d'une conique d'excentricité $e$ par une similitude directe est une conique. Préciser un foyer et l'excentricité de la conique image.
\subsection*{I. Transformation de Zhukowskii}
On considère l'application $Z$ définie par :
\begin{eqnarray*}
\C^* \longrightarrow \C \\
z\rightarrow z+\frac{1}{z}
\end{eqnarray*}
\begin{enumerate}
\item \begin{enumerate}
 \item Soit $r$ et $\theta$ deux nombres réels avec $r>0$. Calculer la partie réelle et la partie imaginaire de 
\[Z(re^{i\theta})\]
\item Pour quels nombres complexes $u$, l'équation
\begin{displaymath}
 z+\dfrac{1}{z} = u
\end{displaymath}
admet-elle une solution réelle strictement positive ? 
\end{enumerate}
\item \begin{enumerate}
\item Soit $C_r$ le cercle de centre $O$ et de rayon $r \neq 1$. Former l'équation cartésienne de l'image $\mathcal{E}_r$ par la transformation $Z$ de ce cercle.
\item Quelle est la nature de $\mathcal{E}_r$ ? Préciser les foyers et l'excentricité.
\item  Dessiner $\mathcal{E}_2$ 
\end{enumerate}
\item \begin{enumerate}
\item Soit $D_\theta$ la demi-droite d'extrémité $O$ et dirigée par $\overrightarrow{e}_\theta$ (vecteur de coordonnées $(\cos \theta, \sin \theta)$ avec $\theta \not \equiv 0 \;(\frac{\pi}{2})$.  Former l'équation cartésienne de l'image $\mathcal{H}_\theta$ par la transformation $Z$ de cette demi-droite.
\item Quelle est la nature de $\mathcal{H}_\theta$ ? Préciser les foyers, l'excentricité et un vecteur directeur pour chaque asymptote.
\item  Dessiner $\mathcal{H}_{\frac{\pi}{3}}$ 
\end{enumerate}
\item Déterminer les images des demi-droites $\mathcal D_0$, $\mathcal D_\frac{\pi}{2}$, $\mathcal D_\pi$, $\mathcal D_{-\frac{\pi}{2}}$.
\end{enumerate}
\subsection*{II. Coniques de Hooke}
Soit $K$ un nombre réel non nul, $k$ est un nombre complexe tel que $k^2=K$ on considère l'équation differentielle
\begin{equation}
z''-Kz=0 \label{eq:1}
\end{equation}
Une solution de cette équation est une application définie dans $\R$ et à valeurs dans $\C$, c'est donc une courbe paramétrée. On se propose de donner quelques propriétés du support d'une telle courbe.
\begin{enumerate}
\item Préciser l'ensemble des solutions de (\ref{eq:1}).
\item Soit $A$, $B$, $g$ des nombres complexes non nuls, $\alpha$ est un argument de $A$, $\beta$ est un argument de $B$.
\begin{enumerate}
\item Déterminer les $g$ complexes tels que $g^2=AB$.
\item Montrer que 
\[(\frac{A}{g})^2\in \R \Leftrightarrow \alpha \equiv \beta \; (\pi) \Leftrightarrow A\overline{B}\in \R\]
Montrer que 
\[\vert\frac{A}{g}\vert=1\Leftrightarrow |A|=|B|\]
\item  Soit $S$ la similitude définie par :
\[w \rightarrow S(w)=gw\]
Calculer
\[S\circ Z(\frac{A}{g}e^{kt}) \]
\end{enumerate}
\item Soit $z$ une solution de (\ref{eq:1}) de la forme
\[z(t)=Ae^{kt}+Be^{-kt}\]
avec $A$ et $B$ complexes non nuls. Attention ici $z$ désigne une fonction.\newline
En distinguant des conditions sur $A$, $B$, $K$, montrer que le support de $z$ est soit une ellipse, soit un segment, soit une branche d'hyperbole, soit une demi-droite, soit une droite.
Pour une ellipse, on précisera les foyers, le demi grand axe, le demi petit axe.\newline
Pour un cercle on précisera le centre et le rayon.\newline
Pour une branche d'hyperbole, on précisera les foyers, les vecteurs directeurs des asymptotes, la distance du centre au sommet.\newline
Pour une demi-droite on précisera l'extrémité et le vecteur directeur.
\item Soit $z_0$ et $z'_0$ deux nombres complexes, on adopte ici les notations de la question précédente.
\begin{enumerate}
\item Calculer $A$ et $B$ pour que $z$ soit la solution de (\ref{eq:1}) vérifiant $z(0)=z_0$, $z'(0)=z'_0$.
\item Calculer $AB$ et $A\overline{B}$.
\item Pour $K<0$, que doit-on imposer à $z_0$ et $z'_0$ pour que $A\neq 0$, $B\neq 0$, $|A|\neq |B|$? 
\item Pour $K>0$, que doit-on imposer à $z_0$ et $z'_0$ pour que $A\neq 0$, $B\neq 0$, $A\overline{B}\not\in\R$?
\end{enumerate}
  
\end{enumerate}
\subsection*{III. Transformation de Bohlin}
Dans cette partie, on reprend les notations de I. On introduit aussi deux nouvelles transformations.
La transformation de Bohlin est l'application $B$ définie par
\begin{eqnarray*}
\C \longrightarrow \C \\
z\rightarrow z^2
\end{eqnarray*}
On considère aussi la translation $T$ définie par :
\begin{eqnarray*}
\C \longrightarrow \C \\
z\rightarrow z+2
\end{eqnarray*}
\begin{enumerate}
\item \begin{enumerate}
\item Préciser $B(C_r)$ et $B(D_\theta)$.
\item Préciser l'image par $B$ d'une droite d'équation $y=c$ avec $c\neq 0$ puis d'une droite quelconque (ne passant pas par l'origine) 
\end{enumerate}
\item Montrer que
\[B\circ Z = T\circ Z \circ B\]
\item Préciser $B(\mathcal{E}_r)$ et $B(\mathcal{H}_r)$.
\item Question facultative et culturelle. Hors barême.\newline
Soit $z$ une solution de l'équation différentielle (\ref{eq:1}). On admet qu'il existe une fonction $\varphi$ continue et dérivable telle que pour tout réel $t$:
\[\varphi'(t)=|z(t)|^2\]
On définit une courbe paramétrée $w$ en posant pour tout réel $t$:
\[w(\varphi(t))=z(t)^2\]
Quel est le support de $w$? Calculer $w''(\varphi(t))$. Interpréter physiquement.

\end{enumerate}
