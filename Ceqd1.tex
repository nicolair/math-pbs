\begin{enumerate}
 \item 
\begin{enumerate}
 \item Les calculs sont immédiats car les solutions des équations avec second membre sont évidentes. On trouve que $\mathcal{S}_+$ est constitué des fonctions
\begin{displaymath}
\left\lbrace  
\begin{aligned}
 ]0,+\infty| &\rightarrow \R \\
 x &\mapsto Ae^{2x} + Be^{-2x} -\frac{\pi}{4} + ax
\end{aligned}
\right. 
\end{displaymath}
où $A$ et $B$ sont des réels quelconques.\newline
De même $\mathcal{S}_-$ est constitué des fonctions
\begin{displaymath}
\left\lbrace  
\begin{aligned}
 ]-\infty,0| &\rightarrow \R \\
 x &\mapsto Ae^{2x} + Be^{-2x} -\frac{\pi}{4} - ax
\end{aligned}
\right. 
\end{displaymath}
où $A$ et $B$ sont des réels quelconques.
 \item Toutes les solutions dans $\mathcal{S}_-$ sont de la forme indiquées au dessus. Elles admettent donc (ainsi que leur dérivée) une limite en $0^-$. Il s'agit donc de résoudre un système aux inconnues $A$ et $B$.
\begin{displaymath}
 \left\lbrace 
\begin{aligned}
 A + B &= \frac{\pi}{4}+v\\ 2A -2B &= a +w
\end{aligned}
\right. 
\Leftrightarrow
\left\lbrace 
\begin{aligned}
 A &= \frac{1}{8}\left( \pi +2a+4v+2w\right)\\ 
 B &= \frac{1}{8}\left( \pi -2a+4v-2w\right)
\end{aligned}
\right. 
\end{displaymath}

\end{enumerate}
 \item 
\begin{enumerate}
 \item Dans les théorèmes de cours sur les équations différentielles du second ordre à coefficients constants, \emph{le second membre doit être supposé continu}. Il est inutile de le supposer dérivable. On peut donc utiliser ces théorèmes pour l'équation $(E_\R)$. En particulier, d'après le théorème sur le problème de Cauchy, si on se donne un $x_0>0$ et des réels $u$, $v$ quelconques, il existe une unique solution $w_0$ dans $\mathcal{S}_\R$ telle que $w_0(x_0)=u$ et $w_0'(x_0)=v$. On en déduit que l'on peut prolonger toute solution dans $\mathcal{S}_{]0,+\infty[}$ à une solutions dans $\mathcal{S}_\R$.
 \item La fonction donnée n'est pas une solution car la fonction valeur absolue n'est pas dérivable en $0$.
\end{enumerate}
\item L'étude de l'équation homogène est la même que dans la question 1. Le problème est ici de trouver une solution dans $\R$ de l'équation complète. On choisit de prolonger la fonction 
\begin{displaymath}
 \left\lbrace 
\begin{aligned}
 ]0,+\infty[ &\rightarrow \R \\ x &\mapsto -\frac{\pi}{4} +ax
\end{aligned}
\right. 
\end{displaymath}
D'après la question 2., on sait qu'un tel prolongement $w$ existe. Par continuité à droite de $0$, on a
\begin{displaymath}
 w(0) = -\frac{\pi}{4}\hspace{0.5cm} w'(0) = a
\end{displaymath}
On peut alors obtenir l'expression de $w$ à gauche de $0$ avec la question 1.b.
\begin{displaymath}
\left. 
\begin{aligned}
 v &= -\frac{\pi}{4} \\ w &= a
\end{aligned}
\right\rbrace 
\Rightarrow
\left\lbrace 
\begin{aligned}
 A &= \frac{a}{2} \\ B &= -\frac{a}{2} 
\end{aligned}
\right. 
\end{displaymath}
On en déduit l'expression de $w$ dans $\R$
\begin{displaymath}
 w(x)=
\left\lbrace 
\begin{aligned}
&a \sh(2x) - \frac{\pi}{2} -ax &\text{ si } x< 0 \\
& - \frac{\pi}{2} +ax &\text{ si } x\geq 0 
\end{aligned}
\right. 
\end{displaymath}
On peut remarquer qu'il est inutile de démontrer que $w$ est dérivable en $0$ car on \emph{sait} d'après les théorèmes du cours qu'il existe un prolongement solution et ce prolongement \emph{doit être}  cette fonction.
Les solutions dans $\mathcal{S}$ sont les fonctions définies dans $\R$ par :
\begin{displaymath}
 w(x) + Ae^{2x} + Be^{-2x}
\end{displaymath}
où $A$ et $B$ sont des réels quelconques.
\end{enumerate}
