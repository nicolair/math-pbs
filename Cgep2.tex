Un carr{\'e} dont un des sommets est $0$ est toujours form{\'e} par $0,$ $%
r(0),$ $r^2(0),$ $r^3(0)$ o{\`u} $r$ est une rotation affine d'angle
$\frac
\pi 2$. Une telle rotation est de la forme $z\rightarrow r(z)=iz+u$ o{\`u} $%
u $ est un nombre complexe non nul (pour que les sommets soient
distincts de 0).\newline Les racines $z_{1,}$ $z_2,$ $z_3$ de
$X^3+aX^2+bX+c$ forment, avec 0, un carr{\'e} si et seulement si il
existe un nombre complexe non nul $u$ tel que
\[
X^3+aX^2+bX+c=(X-r(0))(X-r^2(0))(X-r^3(0))
\]
Comme $r(0)=u$, $r^2(0)=(1+i)\,u$, $r(0)=(1+i+i^2)\,u=i\,u$,
\[
(X-u)(X-(1+i)\,u)(X-iu)=X^3-2iX^2u-2X^2u+3iXu^2-iu^3+u^3
\]
l'{\'e}galit{\'e} polyn{\^o}miale se traduit par le syst{\`e}me
\[
\left\{
\begin{array}{l}
a=-2(1+i)u \\
b=3iu^2 \\
c=(1-i)u^3
\end{array}
\right.
\]
L'existence d'un $u$ v{\'e}rifiant ces trois {\'e}quations est {\'e}quivalente
aux conditions
\[
b=3i(\frac{-a}{2(1+i)})^2=\frac
38a^2,c=(1-i)(\frac{-a}{2(1+i)})^3=\frac 1{16}a^3
\]
soit finalement
\[
8b=3a^2\text{, }16c=a^3
\]
