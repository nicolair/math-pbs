\subsection*{Partie II}
\textbf{II.1. Espérance et variance}
\begin{enumerate}
  \item 
  \begin{enumerate}
    \item   Par définition, $E(X_k) = \p(X_k=1) - \p(X_k = -1) = 0$. Comme les variables sont mutuellement indépendantes:
\[
  X_k \neq X_l \Rightarrow E(X_k X_l) = E(X_k) E(X_l) = 0.
\]
    \item Introduisons le conjugué pour exprimer le carré du module puis utilisons la linéarité de l'espérance et la question précédente: 
\begin{multline*}
  |Z|^2 = \sum_{(k,l)\in \llbracket 1,n \rrbracket^2}X_kX_l\, \xi^{k-l} 
  = n + 2 \sum_{(k,l)\in \llbracket 1,n \rrbracket^2 \atop k < l}X_kX_l\, \xi^{k-l} \\
  \Rightarrow
  E(|Z|^2) = n + 2\sum_{(k,l)\in \llbracket 1,n \rrbracket^2 \atop k < l} \underset{ = 0}{\underbrace{E(X_k X_l)}}\, \xi^{k-l} 
   = n
\end{multline*}
car $X_k^2$ est la variable constante (certaine) de valeur $1$.
  \end{enumerate}

  \item 
  \begin{enumerate}
    \item Comme les variables sont mutuellement indépendantes et centrées,
\[
  E(X_i X_j X_k X_l) \neq 0 \Rightarrow  \card \left\lbrace X_i, X_j, X_k, X_l \right\rbrace < 4 .
\]
On sait que $X_i \neq X_j$ car $i < j$ et $X_k \neq X_l$ car $k < l$. Les deux paires $\left \lbrace X_i, X_j\right\rbrace$ et $\left \lbrace X_k, X_l\right\rbrace$ ne peuvent pas être disjointes.\newline
Si $X_k = X_j$ alors $k = j$ donc $j < l$ donc 
\[
  E(X_i X_j X_k X_l) = E(X_i X_j X_l) = 0.
\]
On en déduit $X_k = X_i$ et $E(X_i X_j X_k X_l) = E(X_j X_l)$ serait nul si $X_j \neq X_l$. On doit donc avoir $i=k$ et $j=l$.

    \item Comme en 1.b.
\begin{multline*}
|Z|^4 = \left( n + 2 \sum_{(i,j)\in \llbracket 1,n \rrbracket^2 \atop i < j}X_i X_j \, \xi^{i-j}\right)
        \left( n + 2 \sum_{(k,l)\in \llbracket 1,n \rrbracket^2 \atop k < l}X_kX_l\, \xi^{k-l}\right)\\
= n^2 + 4n \sum_{(i,j)\in \llbracket 1,n \rrbracket^2 \atop i < j}X_i X_j \, \xi^{i-j} + 
4 \sum_{(i,j,k,l)\in \llbracket 1,n \rrbracket^4 \atop i< j, \; k < l} X_i X_j X_kX_l\, \xi^{i-j+k-l}  \\
\Rightarrow E \left(|Z|^4\right) = n^2 + 4\,  \frac{n(n-1)}{2} = 3n^2 - 2n
\end{multline*}
car dans la dernière somme, seuls les quadruplets $(i,j,i,j)$ contribuent vraiment et chacun pour la valeur 1.\newline
Avec $E(|Z|^2) = n$, on obtient
\[
  V(|Z|^2) = V(|Z|^4) - E(|Z|^2) = 2n(n-1).
\]
  \end{enumerate}
\end{enumerate}
  
\textbf{II.2. Inégalités de concentration.}
\begin{enumerate}
  \item Comme $Z$ est une variable aléatoire à valeurs positives, on peut lui appliquer l'inégalité de Markov:
\[
  \p(|Z|^2 \geq t) \leq \frac{E(|Z|^2)}{t} = \frac{n}{t}.
\]

  \item 
  \begin{enumerate}
    \item Introduisons une fonction $f$ définie dans $\R$ et calculons sa dérivée 
\[
  f(x) = \ch(x) e^{-\frac{x^2}{2}}
  \Rightarrow 
  f'(x) = \left( \sh(x) - x\ch(x)\right)e^{-\frac{x^2}{2}}.
\]
Le signe de $f'(x)$ est celui de $g(x)$ avec $g(x) = \sh(x) - x\ch(x)$. Alors
\[
  g'(x) = -x \sh(x) \leq 0.
\]
La fonction $g$ est décroissante dans $\R$, nulle en $0$ donc positive pour les négatifs et négative pour les positifs. La fonction $f$ admet en minimum absolu en $0$ de valeur 1. On en déduit
\[
  \forall x \in \R, \; \ch(x) \leq e^{-\frac{x^2}{2}}.
\]

    \item On linéarise:
\[
  \sum_{k=0}^{n-1}\cos^2\left(\frac{2k\pi}{n} \right)
  =\sum_{k=0}^{n-1}\left( \frac{1}{2} + \frac{1}{2}\cos\left(\frac{4k\pi}{n} \right) \right)
  = \frac{n}{2}.
\]

    \item Comme les variables $X_k$ sont mutuellement indépendantes, les variables $e^{\theta \cos \left( \frac{2k\pi}{n}\right) X_k}$ le sont aussi. Remarquons que 
\[
  E\left(e^{\theta \cos \left( \frac{2k\pi}{n}\right) X_k}\right) 
  = \frac{1}{2}
  \left( 
      e^{\theta \cos \left( \frac{2k\pi}{n}\right)} 
    + e^{-\theta \cos \left( \frac{2k\pi}{n}\right)}
  \right)
  = \ch \left( \theta \cos \left( \frac{2k\pi}{n}\right)\right).
\]
    
    On en déduit
\begin{multline*}
  E(e^{\theta X}) = \prod_{k=1}^{n} 
            E\left(
                   e^{\theta  X_k\cos\left( \frac{2k\pi}{n}\right)}
             \right)
  = \prod_{k=1}^{n} \ch \left( \theta \cos \left( \frac{2k\pi}{n}\right)\right)\\
  \leq \prod_{k=1}^{n} 
      e^{ \frac{
                \left(\theta \cos \left( \frac{2k\pi}{n}\right)\right)^2
               }{2}
        }
  = e^{
       \frac{\theta^2}{2}\sum_{k=1}^{n} \cos^2 \left( \frac{2k\pi}{n}\right)
      }
  = e^{\frac{n\theta^2}{4}}.
\end{multline*}

  \end{enumerate}


\end{enumerate}
