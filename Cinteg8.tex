\begin{enumerate}
\item La primitive de la fonction à intégrer est évidente:
\[
 \int_{0}^{1}\frac{dx}{(1+x)^2} = \left[ -\frac{1}{1 + x}\right]_{0}^{1} = -\frac{1}{2} + 1  = \frac{1}{2}. 
\]


\item  On d{\'e}finit la fonction $f$ dans $\left[ 0,1\right]$ par 
\[
 \forall x \in \left[ 0,1\right], \; f(x)=\frac{1}{(1+x)^{2}}.
\]
C'est une fonction continue qui permet d'interpr{\'e}ter chaque $a_{n}$ comme une somme de Riemann dont la suite converge vers une intégrale
\[
a_{n} = \frac{1}{n}\sum_{k = 0}^{n - 1}\frac{1}{(1+\frac{k}{n})^{2}}
 = \frac{1}{n}\sum_{k = 0}^{n - 1}f(\frac{k}{n})
\Rightarrow 
(a_{n})_{n\in \N^*} \rightarrow \int_{0}^{1}\frac{1}{(1+x)^{2}}dx = \frac{1}{2}.
\]

\item Soit $F \in \mathcal{C}^2(\left[ a,b\right])$.  
\begin{enumerate}
 \item La formule de Taylor avec reste intégral appliquée à $F$ entre $a$ et $b$ à l'ordre $1$ s'écrit:
\[
 F(b) = F(a) + (b-a)F'(a) + R \; \text{ avec }\; R = \int_{a}^{b}(b-t)F''(t)\,dt .
\]

 \item Comme $F''$ est continue, elle est bornée sur le segment et atteint ses bornes 
\[
 m = \min_{\left[ a,b\right]} F'', \hspace{0.5cm} M = \max_{\left[ a,b\right]} F''.
\]
Les bornes $a < b$ sont \og dans le bon sens\fg~ avec $b - t \geq 0$, on peut intégrer l'encadrement. Il vient
\begin{multline*}
 m \int_a^b(b-t)\,dt \leq R \leq M \int_a^b(b-t)\,dt \; \text{ avec } \int_a^b(b-t)\,dt = \frac{1}{2}(b-a)^2 \\
 \Rightarrow \frac{2}{(b-a)^2} R \in \left[ m,M\right] 
 \Rightarrow \exists c \in \left[ a,b\right]\; \text{ tq } R = \frac{(b-a)^2}{2}\,F''(c). 
\end{multline*}

\end{enumerate}

\item  
\begin{enumerate}
 \item D'apr{\`e}s les questions pr{\'e}c{\'e}dentes, $b_{n}$ est la diff{\'e}rence entre l'int{\'e}grale de $f$ et une de ses sommes de Riemann. Découpons l'intégrale avec la subdivision r{\'e}guli{\`e}re :
\[
b_{n} = \sum_{k = 0}^{n - 1}\left( \int_{\frac{k}{n}}^{\frac{k + 1}{n}}f(x)\,dx\; - \frac{1}{n}f(\frac{k}{n})\right).
\]
Notons $F$ une primitive de $f$ et appliquons le résultat de la question 3.b. (reste de Lagrange) {\`a} $F$ entre $\frac{k}{n}$ et $\frac{k + 1}{n}$.
\[
\exists c_{k}\in \left[ \frac{k}{n},\frac{k + 1}{n}\right]\;\text{ tq }\; 
F(\frac{k + 1}{n}) = F(\frac{k}{n}) + \frac{1}{n}f(\frac{k}{n}) + \frac{1}{2n^{2}}f^{\prime }(c_{k}).
\]
Ceci s'{\'e}crit encore
\begin{multline*}
\int_{\frac{k}{n}}^{\frac{k + 1}{n}}f(x)\,dx - \frac{1}{n}f(\frac{k}{n}) 
= \frac{1}{2n^{2}}f^{\prime }(c_{k})
\Rightarrow b_n = \sum_{k = 0}^{n - 1} \frac{1}{2n^2}f'(c_{k}) \\
\Rightarrow 2nb_n  = \frac{1}{n} \sum_{k = 0}^{n - 1} f'(c_{k}).
\end{multline*}
Comme $c_k \in \left[ \frac{k}{n}, \frac{k + 1}{n}\right]$ , la dernière somme est une somme de Riemann de $f'$ attachée à la subdivision régulière. 

 \item Comme $f'$ est continue, la somme de Riemann précédente converge vers l'intégrale
\[
(2nb_{n})_{n\in \N^*} \rightarrow \int_{0}^{1}f'(t)\,dt = f(1)-f(0) = -\frac{3}{4}.
\] 
On peut réécrire ces limites avec des développements:
\[
 b_n = -\frac{3}{8n} + o(\frac{1}{n}), \hspace{0.5cm} b_n = \frac{1}{2} - a_n
 \Rightarrow a_n = \frac{1}{2} - b_n = \frac{1}{2} + \frac{3}{8n} + o(\frac{1}{n}).
\]

\end{enumerate}


\end{enumerate}