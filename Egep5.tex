%<dscrpt>Système linéaire et cercle circonscrit.</dscrpt>
\textit{Cet exercice doit se traiter uniquement à l'aide de systèmes d'équations linéaires.}\newline
Il porte sur la ccocyclicité d'une famille de point (existence d'un cercle passant par ces points) et sur le cercle circonscrit à trois points.

On désigne par $x$ et $y$ les fonctions coordonnées relatives à un repère orthonormé fixé d'un plan. 
\begin{enumerate}
 \item Soit $n$ un entier naturel non nul et $A_1,A_2,\cdots A_n$ des points du plan. Traduire à l'aide d'un système d'équations linéaires l'existence d'un cercle contenant les points $A_1,A_2,\cdots A_n$. On précisera très clairement le nombre d'équations et les inconnues.

 \item Ici $n=3$. Former avec un déterminant une condition caractérisant l'existence d'un unique cercle passant par $A_1$, $A_2$, $A_3$. Lorsque cette condition est vérifiée, exprimer les coordonnées du centre de ce cercle.
 \item On admet que la régle de développement d'un déterminant suivant une ligne ou une colonne s'étend au cas des déterminants $4\times 4$. On suppose $A_1$, $A_2$, $A_3$ non alignés, montrer que $4$ points $A_1$, $A_2$, $A_3$, $A_4$ sont cocycliques si et seulement si
\begin{displaymath}
 \begin{vmatrix}
  x(A_1) & y(A_1) & 1 & x(A_1)^2+y(A_1)^2 \\
  x(A_2) & y(A_2) & 1 & x(A_2)^2+y(A_2)^2 \\
  x(A_3) & y(A_3) & 1 & x(A_3)^2+y(A_3)^2 \\
  x(A_4) & y(A_4) & 1 & x(A_4)^2+y(A_4)^2 
 \end{vmatrix}
= 0
\end{displaymath}
  
\end{enumerate}


