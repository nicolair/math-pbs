%<dscrpt>Algèbre euclidienne, dérivées partielles.</dscrpt>
Dans tout le problème, on convient d'identifier une matrice carrée d'ordre 1 à son unique coefficient et l'espace $\R^3$ à l'espace des matrices à trois lignes et une colonne. Dans les parties A et B, cet espace est muni de son produit scalaire usuel.\newline
Si $A$ est une matrice réelle 3,3, on appelle \emph{noyau} de $A$ l'ensemble des vecteurs colonnes $X$ tels que
\[AX=\begin{pmatrix}
0\\
0\\
0 \end{pmatrix}
\]

Pour tout $(a,b)\in\R^2$, on pose
\[
M=\begin{pmatrix}
a & -a & b\\
-a & a & b\\
b & b & 0
\end{pmatrix}
\]
L'ensemble des matrices de la forme $M(a,b)$ avec $(a,b)\in\R^2$ est noté $E$.
\subsection*{A. Généralités}
\begin{enumerate}
\item Justifier que $E$ est un sous-espace vectoriel de $\mathcal{M}_3(\R)$. En donner une base et la dimension.
\item Pour quels réels $\lambda$ la matrice $M(a,b)-\lambda I_3$ est-elle non inversible ? Pour chaque $\lambda$ trouvé préciser une base du noyau (tous les vecteurs seront unitaires).
\item Préciser une matrice $P$ telle que $M(a,b)=PD\,{^tP}$ avec
\[
D=\begin{pmatrix}
2a & 0 & 0\\
0 & b\sqrt 2 & 0\\
0 & 0 & -b\sqrt 2
\end{pmatrix}
\]
\end{enumerate}
\subsection*{B. Matrices orthogonales de $E$}
\begin{enumerate}
\item Déterminer, parmi les matrice $M(a,b)$ de $E$, celles qui sont orthogonales.
\item On note
\[
A=\frac{1}{2}\begin{pmatrix}
-1 & 1 & \sqrt 2\\
1 & -1 & \sqrt 2\\
\sqrt 2 & \sqrt 2 & 0
\end{pmatrix}
\]
Justifier que l'endomorphisme $\psi$ de $\R^3$, admettant $A$ pour matrice dans la base canonique est une isométrie vectorielle. En préciser la nature et les éléments géométriques.
\end{enumerate}
\subsection*{C. Construction de nouveaux produits scalaires sur $\R^3$}
{\'E}tant donnés trois réels $\lambda$, $a$ et $b$, on pose
\[N=\lambda I_3 +M(a,b)\]
Pour tous vecteurs $U$, $V$ de $\R^3$, on pose $\phi(U,V)={^tU}NV$.\newline
On souhaite déterminer une condition nécessaire et suffisante, portant sur $\lambda$, $a$, $b$ pour que $\phi$ soit un produit scalaire sur $\R^3$.
\begin{enumerate}
\item Sans déterminer explicitement $\phi(U,V)$, montrer que $\phi$ est une application à valeurs dans $\R$ bilinéaire et symétrique.
\item On pose
\[
P=\frac{1}{2}\begin{pmatrix}
\sqrt 2 & 1 & 1\\
-\sqrt 2 & 1 & 1\\
0 & \sqrt 2 & -\sqrt 2
\end{pmatrix}
\]
 et $Z={^tP}U$ et on notera
 \[
 Z=\begin{pmatrix}
 z_1\\
 z_2\\
 z_3
 \end{pmatrix}
 \]
 Montrer que
 \[\phi(U,U)={^t}Z(\lambda I_3 + D)Z=(\lambda+2a)z_1^2+(\lambda+b\sqrt 2)z_2^2+(\lambda-b\sqrt 2)z_3^2\]
 \item Montrer que si $\lambda > \max \{-2a,|b|\sqrt 2\}$ alors $\phi$ est un produit scalaire sur $\R^3$.
 \item {\'E}tudier la réciproque.
\end{enumerate}
\subsection*{D. {\'E}tude des points critiques d'une fonction de deux variables}
Dans cette partie, $a=-1$, $b=1$ et $\lambda=2$ de sorte que
\[N=2I_3+M(-1,1=
\begin{pmatrix}
 1 & 1 & 1\\
 1 & 1 & 1\\
 1 & 1 & 2
 \end{pmatrix}\]
Pour tout couple $(x,y)$ de réels, on pose
 \[
 U=\begin{pmatrix}
 x\\
 xy\\
 y
 \end{pmatrix}
 \]
Puis $f((x,y))=\phi(U,U)={^tU}NU$.
\begin{enumerate}
\item Montrer que $f$ est de classe $\mathcal{C}^2$ et admet exactement deux points critiques que l'on précisera.
\item {\`A} l'aide de la partie C., montrer que l'un des points critiques (que l'on précisera) correspond à un minimum global de $f$.
\item Former, pour $t$ au voisinage de $0$, un équivalent simple à la fonction
\[t \rightarrow f((-2+t,-1+t))\]\\
Que peut-on en déduire pour l'autre point critique ?
\end{enumerate} 
