\begin{enumerate}
  \item La transformation du syst{\`e}me par la m{\'e}thode du pivot (avec
  $c\neq 0$)
  conduit aux syst{\`e}mes {\'e}quivalents

  \begin{eqnarray*}
  \left \{
\begin{array}{cccc}
  cx& & -az&=m\\
   &-cy&+bz &=1\\
  -bx&+ay& &=n
\end{array}
  \right. \\
\left \{
\begin{array}{cccc}
  cx& & -az&=m\\
   &-cy&+bz &=1\\
  &+ay&-\frac{ab}{c}z &=n+\frac{b}{c}m
\end{array}
  \right. \\
\left \{
\begin{array}{cccc}
  cx& & -az&=m\\
   &-cy&+bz &=1\\
  && 0&=n+\frac{b}{c}m+\frac{a}{c}
\end{array}
  \right.
  \end{eqnarray*}
Le syst{\`e}me admet des solutions si et seulement si
\[cn+bm+a=0\]
Dans ce cas, l'ensemble des solutions est form{\'e} par les triplets
\[(\frac{m}{c},-\frac{1}{c},0)+z(\frac{a}{c},\frac{b}{c},1)\]
o{\`u} $z$ est un r{\'e}el arbitraire.
  \item Le produit des deux matrices donne
  $\phantom{}^tAA=(a^2+b^2+c^2)I_4$. On en d{\'e}duit que la matrice
  $A$ est inversible d'inverse
  \[A^{-1}=\frac{1}{a^2+b^2+c^2}\phantom{}^tA\]
  Par cons{\'e}quent, l'{\'e}quation $AX=C$ d'inconnue $X$ admet une
  unique solution
  \begin{eqnarray*}
  A^{-1}C&=&\frac{1}{a^2+b^2+c^2}\phantom{}^tA C\\
  &=&\frac{1}{a^2+b^2+c^2}
       \left(
        \begin{array}{c}
          -cm+bn+ap\\
          c-an+bp\\
          -b+am+cp\\
          a+bm+cn
        \end{array}
       \right)
  \end{eqnarray*}
  \item Notons $S_1$ le syst{\`e}me de la question 1. et $S_2$ celui de la question 2.\newline
  Il est {\'e}vident que si $\left(
        \begin{array}{c}
          x\\
          y\\
          z\\
          0
        \end{array}
       \right)$ est une solution de $S_2$ alors $(x,y,z)$ est
       solution de $S_1$; dans ce cas on a $p=ax+by+cz$.
       R{\'e}ciproquement, si $(x,y,z)$ est solution de $S1$ et
       $p=ax+by+cz$ alors $\left(
        \begin{array}{c}
          x\\
          y\\
          z\\
          0
        \end{array}
       \right)$ est solution de $S_2$.\newline
       Ainsi, en utilisant l'expression des solutions de $S_2$
       trouv{\'e}e en 2., on peut exprimer les solutions de $S_1$ en
       fonction d'un nouveau param{\`e}tre $\mu$. Ces solutions sont
       \[
       \frac{1}{\alpha^2}(cm-bn,-c+an,b-am)+\mu(a,b,c)
       \]
\end{enumerate}
