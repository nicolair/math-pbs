%<dscrpt>Etude d'une suite récurrente</dscrpt>
On note $I=]0,\frac{1}{\sqrt{6}}[$ et  $f$ la fonction définie sur $I$ par $f(x)=x-2x^3$. Soit $(u_n)_{n\in \N^{*}}$ la suite définie par :
\[
u_1=\frac{1}{10}\;,\; u_{n+1}=u_n-2u_n^3\]
\begin{enumerate}
\item 
\begin{enumerate}
	\item Déterminer les variations de $f$ puis comparer $f(I)$ à $I$.
	\item Montrer que $(u_n)_{n\in \N^{*}}$ est monotone.
	\item Montrer que $(u_n)_{n\in \N^{*}}$ est convergente et déterminer sa limite.
\end{enumerate}
\item Théorème de Ces\`aro.\newline
Soit $(v_n)_{n\in \N^{*}}$ une suite qui converge vers un réel $l$. On définit alors la suite $(M_n)_{n\in \N^{*}}$ par :
\[M_n=\frac{v_1+v_2+\cdots +v_n}{n}\]
$M_n$ est la moyenne arithmétique des $n$ premiers termes de la suite.
\begin{enumerate}
	\item Traduire à l'aide de quantificateurs le fait que $(v_n)_{n\in \N^{*}}$ converge vers $l$.
	\item Soit $n$ un entier non nul et $p$ un entier tel que $1\leq p\leq n$. Montrer que
	\[
	\vert M_n - l \vert \leq \frac{1}{n}\sum_{k=1}^p \vert v_k - l\vert  + \max_{p <k \leq n} \vert v_k - l\vert
	\]
	\item Conclure avec soin que si $(v_n)_{n\in \N^{*}}$ converge vers $l$ alors $(M_n)_{n\in \N^{*}}$ converge aussi vers $l$.
\end{enumerate}
\item Application à la recherche d'un équivalent.
\begin{enumerate}
	\item Déterminer la limite  en $0$ de
	\[\frac{1}{(x-2x^3)^2}-\frac{1}{x^2}\]
	En déduire la limite de $(v_n)_{n\in \N^{*}}$ définie par :
	\[v_n=\frac{1}{u_{n+1}^2} - \frac{1}{u_{n}^2}\]
	\item Donner un équivalent de la suite $(u_n)_{n\in \N^{*}}$. 
\end{enumerate}
\end{enumerate}