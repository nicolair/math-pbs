%<dscrpt>Equation différentielles linéaires d'ordres 2 et 4.</dscrpt>
Dans cet exercice la présentation des calculs sera valorisée. Vous devrez en particulier utiliser plusieurs lignes affectées de coefficients (comme dans le cours) et veiller à ce que tout calcul tienne sur une seule page.\newline
Les résultats doivent être encadrés et présentés avec les notations de l'énoncé.\newline
Dans cet exercice, $I= ]0, + \infty[$ et, sauf mention explicite, toutes les fonctions considérées sont définies dans $I$ et à valeurs dans $\R$. On utilise des \emph{puissances symboliques} pour désigner les dérivées successives. Pour une fonction $z$ quatre fois dérivable:
\begin{displaymath}
  z' = z^{(1)}, \hspace{0.4cm} z'' = (z')' = z^{(2)},\hspace{0.4cm} z^{(3)} = \left( z''\right)', \hspace{0.4cm} z^{(4)} = \left( z^{(3)}\right)'  
\end{displaymath}
On considère les équations différentielles suivantes
\begin{align*}
  (E)& & \forall t \in I,\; tx''(t) - 2x'(t) -tx(t) = 0 \\
  (E_4)& & x^{(4)} -2 x^{(2)} + x = 0 \\
  (E_2)& & x'' -x = 0
\end{align*}
Leurs ensembles de solutions (définies dans $I$) sont respectivement notés $\mathcal{S}$, $\mathcal{S}_4$ et $\mathcal{S}_2$. 
\begin{enumerate}
  \item De $(E)$ à $(E_4)$. Soit $z\in \mathcal{S}$ une solution de $(E)$. 
\begin{enumerate}
  \item Montrer que $z$ est de classe $\mathcal{C}^{\infty}$.
  \item Exprimer $2z'$ en fonction de $t$, $z^{(4)}(t)$ et $z''(t)$.
  \item Montrer que $\mathcal{S} \subset \mathcal{S}_4$.
\end{enumerate}

  \item \'Etude de $(E_4)$.
\begin{enumerate}
  \item Si $z$ est définie dans $I$ par $z(t)= e^{\lambda t}$ pour $\lambda \in \R$, calculer $z^{(4)} -2 z^{(2)} + z$.
  \item On définit les fonctions $z_1$, $z_2$, $z_3$, $z_4$
\begin{displaymath}
\forall t \in I, \hspace{0.5cm} z_1(t) = te^t,\;z_2(t) = e^t,\;z_3(t) = te^{-t},\;z_4(t) = e^{-t}  
\end{displaymath}
Montrer que ces fonctions sont des solutions de $(E_4)$. Que peut-on dire des fonctions $az_1 + bz_2 + cz_3 + dz_4$, pour tout $(a,b,c,d)\in \R^4$?
\end{enumerate}
\item Solutions de $(E)$. 
On admet ici que 
\begin{displaymath}
z \in \mathcal{S}_4  \Rightarrow \exists (a,b,c,d)\in \R^4 \text{ tq } z = az_1 + bz_2 + cz_3 + dz_4  
\end{displaymath}
Déterminer l'ensemble des solutions de $(E)$.

  \item On veut démontrer ici la propriété admise à la question 3.
\begin{enumerate}
  \item Soit $z\in \mathcal{S}_4$, montrer que $z'' -z \in \mathcal{S}_2$.
  \item Montrer que 
\begin{displaymath}
z \in \mathcal{S}_4  \Rightarrow \exists (a,b,c,d)\in \R^4 \text{ tq } z = az_1 + bz_2 + cz_3 + dz_4  
\end{displaymath}
\end{enumerate}
  \item Recollement. On considère l'équation différentielle
\begin{displaymath}
    (\overline{E})\hspace{0.5cm} \forall t \in \R,\; tx''(t) - 2x'(t) -tx(t) = 0 \\
\end{displaymath}
Existe-t-il des fonctions deux fois dérivables dans $\R^*$ et continues en $0$ solutions de cette équation? S'il en existe comment s'écrivent-t-elles et de combien de paramètres dépendent-t-elles?
\end{enumerate}
