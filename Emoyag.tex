%<dscrpt>Moyennes arithmétique et géométrique.</dscrpt>
Ce problème présente la preuve de Cauchy de l'inégalité entre moyennes arithmétique et géométrique.

Pour un entier $n$ non nul et $n$ nombres réels strictement positifs $a_{1},a_{2},\cdots ,a_{n}$ on définit :
\begin{itemize}
\item  la moyenne arihtmétique $\mathcal{A}(a_{1},a_{2},\cdots ,a_{n})$ de $a_{1},a_{2},\cdots ,a_{n}$ en posant
\begin{displaymath}
\mathcal{A}(a_{1},a_{2},\cdots ,a_{n})=\frac{1}{n}\sum_{i=1}^{n}a_{i} 
\end{displaymath}

\item  la moyenne géométrique $\mathcal{G}(a_{1},a_{2},\cdots,a_{n}) $ de $a_{1},a_{2},\cdots ,a_{n}$ en posant
\begin{displaymath}
 \mathcal{G}(a_{1},a_{2},\cdots ,a_{n})=\left( \prod_{i=1}^{n}a_{i}\right) ^{\frac{1}{n}}
\end{displaymath}
\end{itemize}

\begin{enumerate}
\item  Montrer que $\mathcal{A}(a_{1},a_{2})\geq \mathcal{G}(a_{1},a_{2})$.

\item  Montrer par récurrence sur $m$ que :
\begin{displaymath}
\forall m\in \N , \forall (a_{1},a_{2},\cdots ,a_{2^{m}})\in (\R_{+}^{*})^{2^{m}}, \;
\mathcal{A}(a_{1},a_{2},\cdots,a_{2^{m}})\geq \mathcal{G}(a_{1},a_{2},\cdots ,a_{2^{m}})
\end{displaymath}
Ceci prouve que la moyenne géométrique est inférieure à la moyenne arithmétique lorsque le nombre de réels mis en jeu est une puissance de 2. On se propose maintenant d'étendre cette inégalité pour un nombre quelconque de réels.

\item  Soit $n$ un entier non nul et $a_{1},a_{2},\cdots ,a_{n}$ des nombres réels strictement positifs. Il existe un entier $m$ tel que $2^{m}\leq n<2^{m+1}$. Notons $a=\mathcal{A}(a_{1},a_{2},\cdots ,a_{n})$ et étendons la définition des $a_{i}$ en posant $a_{i}=a$ pour tous les $i$ entiers entre $n+1$ et $2^{m+1}$.

\begin{enumerate}
\item  Calculer $\mathcal{A}(a_{1},a_{2},\cdots ,a_{2^{m+1}})$.

\item  Montrer que $\mathcal{A}(a_{1},a_{2},\cdots ,a_{n})\geq \mathcal{G}%
(a_{1},a_{2},\cdots ,a_{n})$.
\end{enumerate}

\end{enumerate}