%<dscrpt>Inégalité, produits scalaires, variables aléatoires.</dscrpt>
Dans cet exercice, ($\Omega, \mathbb{P})$ désigne un espace probabilisé fini tel que
\begin{displaymath}
\forall A\in \mathcal{P}(\Omega)\setminus \{ \emptyset \}, \; \mathbb{P}(A)>0  
\end{displaymath}
On note $E = \R^{\Omega}$ l'ensemble des variables aléatoires réelles sur $\Omega$ et on considère dans $E$ une famille $X_{1},...,X_{n}$ de variables aléatoires centrées réduites mutuellement indépendantes. 

\begin{enumerate}
 \item Pour tout couple $(X,Y)\in E^{2}$, on pose $<X,Y> = E(XY)$.
\begin{enumerate}
  \item Montrer que $<.,.>$ est un produit  scalaire sur $E$.
  \item Montrer que la famille $(X_1,\cdots,X_n)$ est orthonormée.
\end{enumerate}

 \item Soit $Z:\Omega \to \R_{+}$ une variable aléatoire telle que $E(Z)>0$. Pour chaque $t$ dans $[0,E(Z)]$, on introduit des variables aléatoires $Z_t^+$ et $Z_t^-$ définies par
\begin{displaymath}
\forall \omega\in \Omega,\hspace{0.5cm}
Z_t^+(\omega) = 
\left\lbrace 
\begin{aligned}
  &Z(\omega) &\text{ si } Z(\omega)>t \\ &0 &\text{ si } Z(\omega)\leq t
\end{aligned}
\right.,\hspace{0.5cm}
Z_t^-(\omega) = 
\left\lbrace 
\begin{aligned}
  &Z(\omega) &\text{ si } Z(\omega)\leq t \\ &0 &\text{ si } Z(\omega)> t
\end{aligned}
\right.
\end{displaymath}

 \begin{enumerate}
  \item Montrer que $E(Z_t^-) \leq t$. 
  \item Montrer que $E(Z_t^+) \leq \sqrt{E(Z^{2})}\, \sqrt{\mathbb{P}(Z>t)}$.
  \item En déduire que:
  $$\mathbb{P}(Z>t) \geq \frac{(E(Z)-t)^{2}}{E(Z^{2})}.$$
 \end{enumerate}
 
 \item On munit $\R^{n}$ du produit scalaire canonique $(.|.)$.\newline
Pour $a = (a_{1},...,a_{n})\in \R^{n}$ fixé, on définit des variables aléatoires $\overrightarrow{V}$ à valeurs dans $\R^n$ et $Z$ à valeurs dans $\R^+$:
\begin{displaymath}
\overrightarrow{V} = \left( a_1X_1(\omega),a_2X_2(\omega),\cdots, a_nX_n(\omega)\right),\hspace{0.5cm}
Z = \norm{\overrightarrow{V}}^2
\end{displaymath}
 \begin{enumerate}
  \item Montrer que $E(Z) = \norm{a}^{2}$ et que $E(Z^{2}) = \displaystyle{\sum_{1,\leq i,j\leq n}a_{i}^{2}a_{j}^{2}E(X_{i}^{2}X_{j}^{2})}$.
  \item Soit $\mu >0$ tel que pour tout $i\in \llbracket 1, n\rrbracket$, $E(X_{i}^{4})\leq \mu^{4}$. Montrer que $E(Z^{2})\leq \mu^{4}\norm{a}^{4}$. 
 \end{enumerate}

 \item à reprendre, le résultat obtenu est complètement trivial.\newline
 On suppose dans cette question que
\begin{displaymath}
\forall i \in \unAn, \;  \mathbb{P}(X_{i}=1) = \mathbb{P}(X_{i}=-1) = \frac{1}{2}
\end{displaymath}
Montrer que pour tout $t\in ]0,1[$:
\begin{displaymath}
 \mathbb{P}\left ( \norm{\overrightarrow{V}} > t\norm{a} \right ) \geq (1-t^{2})^{2} 
\end{displaymath}

\end{enumerate}
