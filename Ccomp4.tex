\begin{enumerate}
  \item Soit $w\in \C$. On note $x = \Re(w)$ et $y = \Im(w)$. On peut écrire
\[
  e^{iw} = e^{i(x+iy)} = e^{ix - y} = \underset{> 0}{\underbrace{e^{-y}}}\, e^{ix}.
\]
On en déduit que $x$ est un argument de $e^{iw}$ et que $\left|e^{iw}\right| = e^{-y}$. 
  \item 
  \begin{enumerate}
    \item Les deux exponentielles sont inverses l'une de l'autre:
\[
  e^{iz} + e^{-iz} - 2 
  = \left( e^{i\frac{z}{2}}\right)^2 + \left( e^{-i\frac{z}{2}}\right)^2 - 2 \left( e^{i\frac{z}{2}}\right) \left( e^{-i\frac{z}{2}}\right)
  = \left( e^{i\frac{z}{2}} - e^{-i\frac{z}{2}}\right)^2
\]

    \item Pour calculer $D$ on commence par faire apparaitre un carré sous le module avec la question précédente avant de développer le carré du module d'une différence ou d'une somme selon la formule du cours.
\begin{displaymath}
 D = \frac{1}{2}\left| e^{i\frac{z}{2}} - e^{-i\frac{z}{2}} \right|^2 \\
= \frac{1}{2}\left( |e^{iz}| + |e^{-iz}| -2\Re(e^{i\frac{z}{2}} \overline{e^{-i\frac{z}{2}}})\right) .
\end{displaymath}

Utilisons les parties réelles et imaginaires.
\begin{displaymath}
 e^{iz} = e^{ib -a} = e^{-a}e^{ib} \Rightarrow |e^{iz}| = e^{-a}.
\end{displaymath}
De même, $|e^{-iz}| = e^{a}$. De plus
\begin{displaymath}
e^{i\frac{z}{2}} \overline{e^{-i\frac{z}{2}}} = e^{i\frac{z}{2}} e^{i\frac{\overline{z}}{2}} 
= e^{i\frac{a}{2}-\frac{b}{2} + i\frac{a}{2} + \frac{b}{2}}
= e^{ia}
\Rightarrow
\Re(e^{i\frac{z}{2}} \overline{e^{-i\frac{z}{2}}}) = \cos a.
\end{displaymath}
On en déduit 
\begin{displaymath}
D= \frac{e^{b} + e^{-b}}{2} - \cos a.
\end{displaymath}
Le calcul est analogue pour la somme et conduit à
\begin{displaymath}
S = \frac{e^{b} + e^{-b}}{2} + \cos a.
\end{displaymath}
En les sommant, on obtient finalement
\begin{displaymath}
 D + S = e^{b} + e^{-b}.
\end{displaymath}

  \end{enumerate}

\end{enumerate}

