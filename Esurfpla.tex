%<dscrpt>Intersection d'une surface par des plans.</dscrpt>
Dans un espace muni d'un repère orthonormé direct $\mathcal R = (O,(\overrightarrow i ,\overrightarrow j ,\overrightarrow k))$, on considère l'ensemble $\mathcal{S}$ des points $M$ dont les coordonnées $(x(M), y(M), z(M))$ dans $\mathcal{R}$ vérifient
\begin{displaymath}
 y(M)^2 = x(M)^2(z(M)^2+4)
\end{displaymath}
\begin{enumerate}
 \item Soit $b\in \R$.
\begin{enumerate}
 \item En traitant à part le cas $b=0$, déterminer la nature de la courbe $\mathcal C_b$ intersection de $\mathcal S$ par le plan d'équation $x=b$.
 \item Déterminer les ensembles constitués par les sommets et les foyers de $\mathcal C_b$ lorsque $b$ décrit $\R^*$.  
\end{enumerate}
\item On note $\Gamma_c$ l'intersection de $\mathcal S$ par le plan d'équation $y=c$ (pour $c\neq 0$).\\ Former une équation de $\Gamma_c$.
\item On définit une courbe paramétrée 
\begin{displaymath}
 t\in \left] -\frac{\pi}{2},\frac{\pi}{2}\right[ \cup \left] \frac{\pi}{2},\frac{3\pi}{2}\right[
\rightarrow
M(t) \text{ de coordonnées } (u(t),c,w(t))
\end{displaymath}
avec
\begin{align*}
 u(t) = \frac{c}{2}\cos t  & & w(t)=2\tan t 
\end{align*}
La \emph{courbure} en $M(t)$ est définie par
\begin{displaymath}
\Delta(t) = u'(t) w''(t) - u''(t)w'(t)
\end{displaymath}
\begin{enumerate}
 \item Montrer que $\Gamma_c$ est le support de la courbe paramétrée $M$.
 \item Calculer $\Delta(t)$.
 \item On dit que $M(t_0)$ est un point d'inflexion si et seulement si $\Delta(t)$ s'annule en changeant de signe en $t_0$. Déterminer l'ensemble constitué par les points d'inflexion des $\Gamma_c$ lorsque $c$ décrit $\R$.
\end{enumerate}

\end{enumerate}
