\subsection*{Partie I.}
\begin{enumerate}
 \item On peut remarquer que 
\begin{displaymath}
 T_b \circ T_{-b} = \Id\\Z = T_{-b} \circ T_b
\end{displaymath}
On en déduit que $T_b$ est bijective de bijection réciproque $T_{-b}$.
\begin{itemize}
 \item Si $b=0$. $T_b$ est l'identité les orbites sont les singletons. Il en existe une infinité.
 \item Si $|b|=1$. Il existe une seule orbite qui est $\Z$ lui même.
 \item Si $|b|> 1$. Les orbites sont les classes de congruence modulo $|b|$. Il en existe $|b|$ à savoir les orbites de $0$, de $1, \cdots$, de $|b|-1$. 
\end{itemize}
On vérifie facilement que $T_b^m = T_{mb}$.

 \item
\begin{enumerate}
 \item La relation $\preceq$ est réflexive car $a=F^0(a)$.\newline
 La relation $\preceq$ est transitive car si $b = F^p(a)$ et $c=F^q(b)$ alors $c = F^q(F^p(a)) = F^{p+q}(a)$.\newline
 En revanche ce n'est pas une relation d'ordre car elle n'est pas forcément antisymétrique. Considérons une bijection $F$ qui échange $1$ et $2$.
 \begin{displaymath}
 F(1) = 2 \Rightarrow 1 \preceq 2,\hspace{1cm} F(2) = 1 \Rightarrow 2 \preceq 1
 \end{displaymath}
et pourtant $1\neq 2$.

 \item La bijectivité de $F$ et l'existence de la bijection réciproque permet de reformuler la relation $\sim$ sous une forme très proche de $\preceq$ en remplaçant seulement $\N$ par $\Z$..
\begin{displaymath}
 \forall (a,b)\in \Z^2, \; a \sim b \Leftrightarrow \exists k \in \Z \text{ tq } b = F^{k}(a)
\end{displaymath}
Sous cette forme, lestrois propriétés sont faciles à vérifier. La transitivité par exemple vient simplement de la possibilité d'ajouter deux entiers.\newline
La classe d'équivalence pur $\sim$ d'un entier $a$ est alors sous orbite pour $F$. On en déduit que les orbites forment une partition de $\Z$.

 \item Si On suppose que $x \leq F(x)$ pour tout entier $x$ alors $a \preceq b \Rightarrow a \leq b$. L'antisymétrie est alors immédiate.
\begin{displaymath}
\left. 
\begin{aligned}
 a &\preceq b \\ b &\preceq a
\end{aligned}
\right\rbrace   \Rightarrow 
\left. 
\begin{aligned}
 a &\leq b \\ b &\leq a
\end{aligned}
\right\rbrace   \Rightarrow 
a = b
\end{displaymath}

\end{enumerate}

\end{enumerate}

\subsection*{Partie II.}
\begin{enumerate}
 \item 
\begin{enumerate}
 \item En général une fonction définie sur une partie d'un ensemble admet plusieurs prolongements et une condition supplémentaire impose l'unicité du prolongement. Commençons par traduire cette condition en termes d'images.
\begin{displaymath}
F_\sigma \circ T_p = T_p \circ D_\sigma \Leftrightarrow
\forall x\in \Z,\; F_\sigma(x+p) = p + F_\sigma(x)
\end{displaymath}
Cette formule permet de prolonger $\sigma$ \og vers la droite\fg. Pour prolonger \og vers la gauche\fg, on peut utiliser $F_\sigma(x-p) = F_\sigma(x) - p$ qui s'obtient en appliquant la formule en $x-p$. Pour fixer les idées, insérons dans un tableau les premières valeurs du prolongement $F_\sigma$ de l'exemple de la question 2..
\begin{center}
\begin{tabular}{c|ccccccccccccccc}
$k$ & $-7$ & $-6$ & $-5$ & $-4$ & $-3$ & $-2$ & $-1$ & $0$ & $1$ & $2$ & $3$ & $4$ & $5$ & $6$ & $7$\\ \hline
$\sigma$ &  &  &  &  &  &  &  & $1$ & $2$ & $0$ &  &  &  &  & \\\hline
$F_\sigma$ & $-9$ & $-5$ & $-4$ & $-6$ & $-2$ & $-1$ & $-3$ & $1$ & $2$ & $0$ & $4$ & $5$ & $3$ & $7$ & $8$                                                                                                                                                                                                                                          \end{tabular}
\end{center}
En utilisant la division euclidienne de $x$ par $p$ avec les notations indiquées:
\begin{displaymath}
 x = (x//p)\times p + (x \% p),\hspace{0.5cm} F_\sigma(x) = (x//p)\times p + \sigma( x\% p)
 = x + \sigma( x\% p) - x\% p
\end{displaymath}

\item Par définition, la restriction $F_\sigma \circ F_{\sigma^{-1}}(x) = F_{\sigma^{-1}} \circ  F_\sigma(x) = x$ pour tout $x$ de $I_p$. Pour un $x$ qui n'est pas dans $I_p$, on peut le diviser par $p$ : $x = q p + r = T_p^{q}(r)$ avec $r\in I_p$. On écrit alors
\begin{multline*}
 F_\sigma \circ F_{\sigma^{-1}}(x)
 = F_\sigma \circ F_{\sigma^{-1}}\circ T_p^{q}(r) = F_\sigma \circ T_p^{q} \circ F_{\sigma^{-1}}(r)\\
 = T_p^{q} \circ F_\sigma \circ F_{\sigma^{-1}}(r)
 = T_p^{q}(r) = x
\end{multline*}
Le calcul est le même pour la composition dans l'autre sens. Cela traduit que $F_\sigma$ est bijective de bijection réciproque $F_{\sigma^{-1}}$.

\item La relation entre $T_p$ et $F_\sigma$ entraine que $f_\sigma$ est $p$-périodique.
\begin{displaymath}
\forall x \in\Z, \; f_\sigma(x+p) = F_\sigma(x+p) - (x+p) = F_\sigma(x) +p -(x+p) = F_\sigma(x) -x = f_\sigma(x) 
\end{displaymath}
Les orbites pour $F_\sigma$ sont les intervalles de $p$ entiers consécutifs qui commencent par un nombre congru à $0$ modulo $p$. Elles ont toutes $p$ éléments mais il y en a une infinité.
\end{enumerate} 

 \item 
\begin{enumerate}
 \item Les orbites sont les intervalles de la forme $\llbracket 3k,3k+2\rrbracket$. Les images par $f_\sigma$ de$0 , 1, 2$ sont $1, 1, -2$. Les autres valeurs s'en déduisent avec la période $3$.
 
 \item La fonction $T_b \circ F_\sigma$ est bijective car composée de deux fonctions bijectives. D'après le cours, sa bijection réciproque est
\begin{displaymath}
 (T_b \circ F_\sigma)^{-1} = F_{\sigma}^{-1} \circ T_b^{-1} = F_{\sigma^{-1}} \circ T_{-b}
\end{displaymath}

\end{enumerate}
 
 \item Le point important est que $T_{pq} = T_p^{q} = T_q^{p}$. Comme $T_p$ commute avec $F_\sigma$, la composée $T_p^q = T_{pq}$ aussi. On raisonne de même avec $T_q^p$ et $F_\varphi$. On en déduit que $T_{pq}$ commute avec $F_\sigma$ et $F_\varphi$ ce qui assure
\begin{displaymath}
 F_\sigma \circ F_\varphi \circ T_{pq} = T_{pq} \circ F_\sigma \circ F_\varphi
\end{displaymath}
L'intervalle $I_{pq}$ se décompose en $p$ intervalles consécutifs chacun de longueur $q$ (d'abord $I_q$ puis son $p$-translaté et ainsi de suite). Chacun de ces intervalles de longueur $q$ est stable par $\varphi$ ce qui assure que $I_{pq}$ est stable par $F_\varphi$. De même $I_{pq}$ se décompose en $q$ intervalles consécutifs chacun de longueur $p$, chacun étant $F_\sigma$-stable. Ceci assure que $I_{pq}$ est stable par $F_\sigma \circ F_\varphi$. La restriction de cette fonction est injective d'un ensemble fini dans lui même, elle est donc bijective. On la note $\theta$. On peut aussi noter
\begin{displaymath}
F_\sigma \circ F_\varphi = F_\theta 
\end{displaymath}
car à cause de la commutativité avec $T_{pq}$ elle est définie à partir de $\theta$ et $I_{pq}$ comme l'était $F_\sigma$ à partir de $\sigma$ et $I_p$. Pour préciser $\theta$, complétons le tableau déjà présenté
\begin{center}
\renewcommand{\arraystretch}{1.2}
\begin{tabular}{c|cccccc}
$k$            & $0$ & $1$ & $2$ & $3$ & $4$ & $5$  \\ \hline
$F_\sigma$     & $1$ & $2$ & $0$ & $4$ & $5$ & $3$  \\ \hline 
$F_\varphi$    & $1$ & $0$ & $3$ & $2$ & $5$ & $4$  \\ \hline 
$\theta$     & $0$ & $3$ & $1$ & $5$ & $4$ & $2$  
\end{tabular}
\end{center}

\end{enumerate}

\subsection*{Partie III.}
\begin{enumerate}
 \item On a vu en II. 1. que $f_\sigma = F_\sigma - \Id_\Z$ est périodique. Elle ne prend donc qu'un nombre fini de valeurs donc elle est minorée. On en déduit qu'il existe $b$ tel que $f_\sigma(x)+b >0$ pour tous les $x$. Pour $F= T_b \circ F_\sigma$ on a bien
 \begin{displaymath}
  \forall x \in \Z, \; F(x) = F_\sigma(x) + b = x + \underset{ > 0 }{\underbrace{f_\sigma(x) + b}}  > x
 \end{displaymath}

 \item La définition de l'orbite partielle $\mathcal{O}(a,p)$ fait apparaitre une somme télescopique
\begin{displaymath}
 \sum_{x \in \mathcal{O}(a,p)}f(x) = \sum_{k = 0}^{p-1}f(F^{k}(a))
 = \sum_{k = 0}^{p-1}\left( F^{k+1}(a) - F^{k}(x)\right) 
 = F^{p}(a) - a
\end{displaymath}

 \item
\begin{enumerate}
 \item Toutes les orbites sont infinies car $a < F(a) < F^(a) < \cdots$.
 \item Soit $I$ un intervalle entier et $\mathcal{O}$ une orbite qui le coupe. On note $a$ leplus petit élément de l'intersection. Comme la suite $a < F(a) < F^2(a) < \cdots $ est strictement croissante, il existe $p$ tel que $F^{p-1}(a) \leq v < F^p(a)$. De plus $u\leq a$ car $a \in I$ et $F^{-1}(a) \notin I$ (d'où $F^{-1}(a)<u$ car $a$ est le plus petit élément de l'intersection. 
 \item D'après les questions 2. et 3.b.
\begin{displaymath}
 \sum_{x\in I\cap \mathcal{O}}f(x) = \sum_{x\in \mathcal{O}(a,p)}f(x) = F^p(a) - a 
\end{displaymath}
D'après l'encadrement vérifié par $F$:
\begin{displaymath}
 \left. 
 \begin{aligned}
  u \leq a \\ F^{p-1}(a) \leq v < F^p(a) \leq F^{p-1}(a) + h \leq v + h
 \end{aligned}
\right\rbrace \Rightarrow F^p(a) - a \leq v - u + h
\end{displaymath}
\begin{displaymath}
 \left. 
 \begin{aligned}
  v < F^p(a) \\ F^{-1}(a) < u \leq a \leq F^{-1}(a) + h < u + h
 \end{aligned}
\right\rbrace \Rightarrow v - u - h < F^p(a) - a
\end{displaymath}

 \item Pour un intervalle $I_n= \left[ -n ,n \right]$, la \og longueur\fg~ de l'intervalle $v-u = 2n$. Lorsque $n$ est assez grand pour que $I_n$ coupe l'orbite, l'encadrement devient
\begin{displaymath}
 \frac{2n - h}{2n+1} < m_n(\mathcal{O}) \leq \frac{2n + h}{2n+1}
\end{displaymath}
Le théorème usuel d'encadrement assure alors que la suite des $m_n(\mathcal{O})$ converge vers $1$.
\end{enumerate}

 \item Lorsque $F$ admet $p$ orbites $\mathcal{O}_1, \cdots ,\mathcal{O}_p$ il existe des $n$ assez grands pour que $I_n$ les coupe toutes. Comme elles forment une partition, les éléments de $I_n$ se distribuent dans les $p$ orbites et la suite proposée s'écrit  la somme des $p$ suites
 \begin{displaymath}
  m_n(\mathcal{O}_1)+ \cdots + m_n(\mathcal{O}_p)
 \end{displaymath}
qui converge vers $p$ car chacune converge vers $1$.
\end{enumerate}
