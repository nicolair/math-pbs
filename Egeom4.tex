%<dscrpt>Un exercice de géométrie analytique dans l'espace.</dscrpt>
Un repère $(O,\overrightarrow{i},\overrightarrow{j},\overrightarrow{k})$ d'un espace étant fixé, on définit les points suivants par leurs coordonnées
\begin{align*}
 A:(0,0,0) &,& B:(0,1,0) &,& C:(0,0,1) \\
A^\prime:(a,0,0) &,& B^\prime:(b,1,0) &,& C^\prime:(c,0,1) 
\end{align*}
avec $abc\neq0$. On pose de plus
\begin{displaymath}
 s = \frac{1}{a}+\frac{1}{b}+\frac{1}{c}
\end{displaymath}
et on suppose $s\neq 0$. Il n'est pas nécessaire de  faire une figure.
\begin{enumerate}
 \item Montrer que les trois plans $(A^\prime BC)$, $(AB^\prime C)$, $(ABC^\prime )$ ont un point commun $S$ dont on déterminera les coordonnées.
\item Montrer que les trois plans $(AB^\prime C^\prime )$, $(A^\prime B C^\prime )$, $(A^\prime B^\prime C)$ ont un point commun $S^\prime $ dont on déterminera les coordonnées. Montrer que les droites $(SS^\prime )$ et $(AA^\prime )$ sont parallèles.
\item Soit $T$ le point d'intersection de la droite $(SS^\prime)$ avec le plan $(ABC)$ et $T^\prime $ le point d'intersection de la droite $(SS^\prime)$ avec le plan $(A^\prime B^\prime C^\prime )$. Vérifier que 
\begin{displaymath}
 \overrightarrow{TS} = \overrightarrow{SS^\prime } = \overrightarrow{S^\prime T^\prime }
\end{displaymath}

\end{enumerate}
