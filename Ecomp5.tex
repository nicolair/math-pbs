%<dscrpt>Exercice avec des nombres complexes.</dscrpt>
Soit $f$ la fonction \footnote{D'après Concours commun 2006 des écoles des mines d'Albi, ...} définie de $\C - \{2i\}$ dans $\C$ par :
\begin{displaymath}
 f(z) = \frac{z^2}{z-2i}
\end{displaymath}
\begin{enumerate}
 \item Déterminer les racines carrées de $8-6i$, en déduire les antécédents de $1+i$ par $f$.
 \item Soit $h\in \C$. Discuter suivant les valeurs de $h$, son nombre d'antécédents par $f$.\newline La fonction $f$ est-elle surjective, injective ?
 \item On définit une application $g$ de $\C - \{2i\}$ dans $\C$ par :
\begin{displaymath}
 g(z) = |z-2i|^2 \frac{z^2}{z-2i}+ z^3
\end{displaymath}
On note respectivement $x$ et $y$ les parties réelle et imaginaire de $z$. Exprimer en fonction de $x$ et $y$ les parties réeelles et imaginaires de $g(z)$.
 \item Soit $\mathcal P$ un plan rapporté à un repère orthonormé direct $\mathcal{R}=(O,\overrightarrow{e_1},\overrightarrow{e_2})$ et $\Gamma$ l'ensemble des points dont les affixes $z$ sont telles que $g(z)$ soit imaginaire pur.
\begin{enumerate}
  \item Montrer que $\Gamma$ est la réunion d'une droite $\Delta$ (privée d'un point) et d'un ensemble $\mathcal C$ dont on donnera une équation.
  \item Soit $A$ le point de $\mathcal{P}$ de coordonnées $(0,-1)$ dans $\mathcal{R}$. On définit deux vecteurs
\begin{displaymath}
  \overrightarrow{u_1} = \frac{1}{\sqrt{2}}\left( \overrightarrow{e_1} +\overrightarrow{e_2}\right),\hspace{0.5cm}
  \overrightarrow{u_2} = \frac{1}{\sqrt{2}}\left( -\overrightarrow{e_1} +\overrightarrow{e_2}\right)
\end{displaymath}
Montrer que $\mathcal{R}'=\left(A,\overrightarrow{u_1}, \overrightarrow{u_2}\right)$ est un repère orthonormé direct. Soit $M$ un point de coordonnées $(x,y)$ dans $\mathcal{R}$. Calculer les coordonnées $(X,Y)$ de $M$ dans $\mathcal{R}'$. 
  \item En considérant $(y+1)^2$, exprimer l'équation de $\mathcal{C}$ avec $X$ et $Y$. Présenter $\mathcal{C}$ et $\Delta$ sur une figure. 
\end{enumerate}
\end{enumerate}


