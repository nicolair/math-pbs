%<dscrpt>Introduction aux polynômes orthogonaux</dscrpt>
On note\footnote{d'après \textit{Fractions et Polynômes} éd Ellipses} $E$ le $\R$ espace vectoriel $\R[X]$ des polynômes à coefficients réels et $E_n$ le sous-espace formé par les polynômes dont le degré est inférieur ou égal à $n$. On identifiera dans ce texte un polynôme avec la fonction polynomiale qui lui est associée. Le $\R$ espace vectoriel $E$ est muni d'un produit scalaire vérifiant la propriété suivante :
\begin{displaymath}
 \forall (P,Q)\in E^2,\quad (XP/Q)=(P/XQ)
\end{displaymath}
Il est important de bien comprendre que $(P/Q)$ désigne le nombre réel produit scalaire des éléments $P$ et $Q$ de $E$.\\
Notons $1_E$ le polynôme constant 1 et $c_i=(X^i/1_E)$ pour tout entier naturel $i$.\\
On dira que $(P_n)_{n\in \N}$ est une \emph{suite de polynômes orthogonaux} lorsque
\begin{displaymath}
  \begin{aligned}
    \forall n\in \N &, \deg (P_n)=n \\
    \forall (i,j)\in \N^2 &, i<j \Rightarrow (X^i / P_j)=0
  \end{aligned}
\end{displaymath}

On définit une suite de fonctions polynomiales $(D_n)_{n\in \N}$ et une suite de nombres réels $(\Delta_n)_{n\in \N}$ par $\Delta_0=1$, $D_0=1_E$ et les formules suivantes (déterminants) qui sont valables pour tout entier non nul $n$ 
\begin{displaymath}
 \Delta_n = \left \vert
\begin{array}{ccccc}
 c_0 & c_1 & \cdots & c_{n-1} & c_n \\
 c_1 & c_2 & \cdots & c_{n} & c_{n+1} \\
 \vdots & \vdots & \ddots & \vdots & \vdots \\
 c_{n-1} & c_n & \cdots & c_{2n-2} & c_{2n-1} \\
c_n & c_{n+1} & \cdots & c_{2n-1} & c_{2n}
\end{array}
\right \vert
\end{displaymath}
\begin{displaymath}
\forall x\in \R:  D_n(x) = \left \vert
\begin{array}{ccccc}
 c_0 & c_1 & \cdots & c_{n-1} & c_n \\
 c_1 & c_2 & \cdots & c_{n} & c_{n+1} \\
 \vdots & \vdots & \ddots & \vdots & \vdots \\
 c_{n-1} & c_n & \cdots & c_{2n-2} & c_{2n-1} \\
1 & x & \cdots & x^{n-1} & x^n
\end{array}
\right \vert
\end{displaymath}

\subsection*{Partie I}
\begin{enumerate}
 \item 
  \begin{enumerate}
    \item Cas particulier. Montrer que l'on définit un produit scalaire vérifiant les conditions de l'énoncé en posant
\begin{displaymath}
 (P/Q)=\int_{-1}^{1}P(x)Q(x)dx
\end{displaymath}
pour tout couple $(P,Q)$ de polynômes.
    \item Calculer $c_n$ pour tout entier $n$ ainsi que $\Delta_1$, $\Delta_2$, $D_1$, $D_2$, $D_3$.
\end{enumerate}
 
\item Montrer que $(D_n)_{n\in \N}$ est une suite de polynômes orthogonaux. Quels sont les coefficients dominants ?

\item \begin{enumerate}
 \item Soit $(P_n)_{n\in \N}$ une suite de polynômes orthogonaux et $Q$ un polynôme de degré strictement inférieur à $n$, montrer que
 \begin{displaymath}
   (P_n / Q)=0
 \end{displaymath}
 \item Soit $(P_n)_{n\in \N}$ une suite de polynômes orthogonaux et $(\lambda_n)_{n\in \N}$ une suite de nombres réels non nuls. Montrer que $(\lambda_n P_n)_{n\in \N}$ est une suite de polynômes orthogonaux.
 \item Soit $(P_n)_{n\in \N}$ et $(Q_n)_{n\in \N}$ deux suites de polynômes orthogonaux, montrer qu'il existe une suite $(\lambda_n)_{n\in \N}$ de nombres réels non nuls tels que $Q_n=\lambda_n P_n$ pour tous les entiers $n$.
\end{enumerate}

Dans toute la suite, on notera $(Q_n)_{n\in \N}$ l'\emph{unique} suite de polynômes orthogonaux telle que le coefficient dominant de chaque $Q_n$ soit égal à 1.

\item \begin{enumerate}
 \item Exprimer $D_n$ en fonction de $\Delta_{n-1}$ et de $Q_n$.
 \item Montrer que, pour tout entier $n$ :
\begin{displaymath}
 (Q_n / X^n) = \Vert Q_n \Vert ^2
\end{displaymath}
Exprimer cette quantité en fonction de $\Delta_{n-1}$ et $\Delta_n$
\end{enumerate}

\item Relation de récurrence.
\begin{enumerate}
 \item Soit $(P_n)_{n\in \N}$ une suite de polynômes orthogonaux. Montrer qu'il exsite des suites $(\alpha_n)_{n\in \N}$, $(\beta_n)_{n\in \N}$, $(\gamma_n)_{n\in \N}$ telles que, pour tous les entiers $n\geq1$:
\begin{displaymath}
 XP_n = \alpha_n P_{n-1}+ \beta_n P_n +\gamma_n P_{n+1}
\end{displaymath}
\item Montrer qu'il existe des suites $(a_n)_{n\in \N}$ et $(b_n)_{n\in \N}$ telles que 
\begin{displaymath}
 \forall n\geq 2 : \quad Q_n = (a_n + X)Q_{n-1}+b_nQ_{n-2}
\end{displaymath}


\end{enumerate}

\end{enumerate}

\subsection*{Partie II}
Dans cette partie, on se propose de calculer explicitement les coefficients de la relation de récurrence vérifiée par les polynômes orthogonaux qui prennent en 1 la valeur 1 (polynômes de Legendre) pour le cas particulier de la question I.1.\newline
On garde les notations de la partie I et on désigne par $(L_n)_{n\in \N}$ l'unique suite de polynômes orthogonaux vérifiant $L_n(1)=1$ pour tout entier $n$. 
\begin{enumerate}
 \item Comment s'expriment les $L_n$ en fonction des $Q_n$ ?
 \item Montrer que pour tout entier $n$ : 
\begin{displaymath}
L_n(-x)= (-1)^n L_n(x)
\end{displaymath}
\item Montrer que $\beta_n=0$ pour tout entier $n$. (on pourra prendre la valeur en $1$ et en $-1$)
\item Soit $\lambda_n$ le coefficient dominant de $L_n$, exprimer $\alpha_n$ en fonction de $\lambda_{n-1}$, $\lambda_{n}$, $\Vert L_{n-1}\Vert ^2$, $\Vert L_{n}\Vert ^2$. Exprimer $\gamma_n$ en fonction de $\lambda_{n+1}$, $\lambda_{n}$.

\item En considérant $(L_n^\prime / L_{n-1})$, montrer que
\begin{displaymath}
 n\frac{\lambda_n}{\lambda_{n-1}} \Vert L_{n-1} \Vert^2 = 2
\end{displaymath}

\item Montrer que
\begin{displaymath}
 \Vert L_n \Vert ^2 = \frac{1}{2n+1}
\end{displaymath}
Préciser la relation de récurrence vérifiée par la suite $(Q_n)_{n\in \N}$.
\end{enumerate}
