%<dscrpt>Inégalité de Chebychev dans le cas continu.</dscrpt>
Pour tout fonction $h$ continue sur un segment réel et à valeur moyenne, on définit la \emph{valeur moyenne} (notée $\overline{h}$) par :
\begin{displaymath}
 \overline{h} = \frac{1}{b-a}\int_a^bh(t)\,dt
\end{displaymath}
Soit $f$ et $g$ deux fonctions continues sur le segment $[a,b]$ et à valeurs réelles.
\begin{enumerate}
 \item Soit $y\in [a,b]$ fixé. Pourquoi la fonction de $[a,b]$ dans $\R$
\begin{displaymath}
 x \mapsto (f(x)-f(y))(g(x)-g(y))
\end{displaymath}
est-elle intégrable sur $[a,b]$?

 \item Pourquoi la fonction de $[a,b]$ dans $\R$
\begin{displaymath}
 y \mapsto \int_a^b(f(x)-f(y))(g(x)-g(y))\, dx
\end{displaymath}
est-elle intégrable sur $[a,b]$? On note
\begin{displaymath}
 T = \int_a^b\left(\int_a^b(f(x)-f(y))(g(x)-g(y))\, dx \right)\, dy 
\end{displaymath}

\item Exprimer $T$ à l'aide des valeurs moyennes de $f$, $g$ et $fg$.

\item Inégalité de Chebychev. On suppose $f$ et $g$ \emph{croissantes}. Montrer que
\begin{displaymath}
 \overline{f}\,\overline{g} \leq \overline{fg}
\end{displaymath}

\end{enumerate}
