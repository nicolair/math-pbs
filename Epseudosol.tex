%<dscrpt>Pseudo-solution d'une équation linéaire dans un espace euclidien.</dscrpt>
Dans tout le problème \footnote{d'après Concours communs polytechniques 2006 maths 1} $E$ et $F$ désignent deux espaces vectoriels euclidiens de dimensions au moins 2. Le produit scalaire de deux vecteurs $u$ et $v$ de $E$ est noté
\begin{displaymath}
 (u/v)
\end{displaymath}
On se donne aussi une application linéaire $f\in\mathcal{L}(E,F)$ et un élément $v$ de $F$.
\begin{enumerate}
 \item En considérant la projection orthogonale de $v$ sur l'image de $f$, montrer qu'il existe un élément $x_0$ de $E$ tel que :
\begin{displaymath}
  \Vert f(x_0) -v\Vert = \min \left\lbrace \Vert f(x) -v\Vert, x\in E \right\rbrace 
\end{displaymath}
Dans la suite, $x_0$ sera appelé une \emph{pseudo-solution} de l'équation :
\begin{equation}
  f(x)=v
\end{equation} 

\item Montrer que si $f$ est injective, alors l'équation $(1)$ admet une pseudo-solution unique.

\item Montrer que $x_0$ est pseudo-solution de $(1)$ si et seulement si :
\begin{displaymath}
 \forall x \in E : \quad (f(x)/f(x_0)-v)=0
\end{displaymath}

\item Soit $\mathcal{B}$ et $\mathcal{C}$ deux bases orthonormales de $E$ et $F$ respectivement. On définit les matrices $A$, $V$ et $X_0$ par les relations :
\begin{displaymath}
 \begin{array}{ccc}
A=\underset{\mathcal{B}\mathcal{C}}{\mathrm{Mat}}\, f ,&
V=\underset{\mathcal{C}}{\mathrm{Mat}}\, v ,&
X_0=\underset{\mathcal{B}}{\mathrm{Mat}}\, x_0 
\end{array}
\end{displaymath}
\'Ecrire sous forme matricielle l'équation
\begin{displaymath}
 (f(x)/f(x_0)-v)=0
\end{displaymath}
et en déduire que $x_0$ est pseudo-solution de $(1)$ si et seulement si
\begin{displaymath}
 ^tAAX_0 = \,^tAV
\end{displaymath}

\item Exemple. Dans cette question, on prend $E=F=\R^3$ munis du produits scalaire usuel. Relativement à la base canonique de $\R^3$, les matrice de $f$ et $v$ sont respectivement :

\begin{eqnarray*}
 A=\left( 
 \begin{array}{ccc}
   1 & 1 & -1 \\
   1 & 1 & -1 \\
   -1& 2 & 1
 \end{array}
 \right) 
 &,&
 V=\left( 
 \begin{array}{c}
   1\\0\\1
 \end{array}
 \right) 
\end{eqnarray*}
Quel est le rang de $f$ ? Donner une équation de son image (expliquer et justifier). Le vecteur $v$ est-il dans l'image de $f$ ?
Déterminer les pseudo-solutions de l'équation $f(x)=v$.

\item Application. Soit $n$ un entier supérieur ou égal à deux, on considère trois éléments de $\R^n$
\begin{displaymath}
\begin{array}{ccc}
  a=(a_1,a_2,\cdots,a_n),&
  b=(b_1,b_2,\cdots,b_n),&
  c=(c_1,c_2,\cdots,c_n)
\end{array}
\end{displaymath}

et on souhaite trouver deux réels $\lambda$ et $\mu$ tels que la somme
\begin{displaymath}
 \sum_{k=1}^n (\lambda a_k +\mu b_k -c_k)^2
\end{displaymath}
soit minimale.
\begin{enumerate}
 \item Montrer que ce problème équivaut à la recherche des pseudo-solutions d'une équation $ f(x)=v$ où $f$ est un élément de $\mathcal{L}(\R^2,\R^n)$. Préciser le vecteur $v$ et donner la matrice de $f$ dans les bases canoniques de $\R^2$ et $\R^n$.
 \item Comment doit-on choisir $a$ et $b$ pour que l'application $f$ soit injective ?
 \item Lorsque cette dernière condition est réalisée, donner la solution du problème posé en exprimant $\lambda$ et $\mu$ à l'aide de produits scalaires dans $\R^n$.
\end{enumerate}

\end{enumerate}
 