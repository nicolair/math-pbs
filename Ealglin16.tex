%<dscrpt>Une bonne base pour des endomorphismes de trace nulle.</dscrpt>
Dans tout l'exercice, $E$ désigne un $\R$-espace vectoriel et $f$ un endomorphisme de $E$. La dimension de $E$ varie d'une question à l'autre.
\begin{enumerate}
\item Dans cette question $\mathcal{B}=(b_1,b_2,\cdots, b_n)$ est une base de $E$  ($n\geq 2$). Pour $i$ et $j$ deux entiers distincts entre 1 et $n$, on définit une famille
\[\mathcal{B}_{i,j}=(b^\prime _1,\cdots , b^\prime _n)\]
par les relations
\begin{displaymath}
\forall k \in \{1,\cdots,n\} : 
b^\prime _k = \left\{
\begin{array}{ccc}
	b_k & \mathrm{si} & k \neq i \\
	b_i +b_j & \mathrm{si} & k = i 
\end{array}
\right.
\end{displaymath}
Montrer que $\mathcal{B}_{i,j}$ est une base. Former les matrices de passage.

\item Dans cette question, $E$ est encore de dimension $n$. Soit $f\in \mathcal{L}(E)$ telle que ${\Mat}_{\mathcal{B}}\, f$ soit diagonale pour une certaine base $\mathcal B$ de $E$. Préciser 
\begin{displaymath}
 \Mat_{\mathcal{B}_{i,j}}\,f
\end{displaymath}

\item \begin{enumerate}
 \item Montrer que si, \emph{pour toute base} $\mathcal B$ de $E$, ${\Mat_{\mathcal{B}}\, f}$ est diagonale alors  $f$ est dans $\Vect (Id_E)$
 \item Montrer que si $f\not \in \Vect(Id_E)$ alors il  existe un $x\in E$ tel que $(x,f(x))$ est libre.
\end{enumerate}

\item Soit $\mathcal{U}=(u_1,u_2,u_3)$ une base de $E$ et 
\begin{displaymath}
\Mat_{\mathcal{U}}\, f = 
\begin{bmatrix}
 a & b & c \\
a^\prime & b^\prime & c^\prime \\
a^{\prime\prime} & b^{\prime\prime} & c^{\prime\prime}
\end{bmatrix}
\end{displaymath}

On note $p$ la projection sur $\Vect(u_2,u_3)$ parallelement à $\Vect(u_1)$ et on définit un endomorphisme $g\in \mathcal{L}(\Vect(u_2,u_3))$ par :
\[\forall x\in \Vect(u_2,u_3):\; g(x)=p\circ f(x)\]
Préciser 
\begin{displaymath}
 \Mat_{(u_2,u_3)}\,g
\end{displaymath}

\item On s'interesse maintenant aux endomorphismes de $E$ de trace nulle.
\begin{enumerate}
 \item Donner, en démontrant le résultat sous-jacent, la définition de la trace d'un endomorphisme.

\item Quels sont les éléments de $\Vect (Id_E)$ dont la trace est nulle ?

\item On suppose que la dimension de $E$ est 2. Soit $f$ un endomorphisme de $E$ de trace nulle et qui n'est pas dans $\Vect (Id_E)$. Montrer qu'il existe une base $\mathcal U$ de $E$ telle que tous les termes diagonaux de $\Mat_{\mathcal{U}}\, f$ soient nuls.


\item On suppose que la dimension de $E$ est 3. Soit $f$ un endomorphisme de trace nulle de $E$ et qui n'est pas dans $\Vect (Id_E)$.  Montrer qu'il existe une base $\mathcal U$ de $E$ telle que tous les termes diagonaux de $\Mat_{\mathcal{U}}\, f$ soient nuls.
\end{enumerate}

\end{enumerate}