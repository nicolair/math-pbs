%<dscrpt>Autour des racines carrées d'une matrice ou d'un endomorphisme.</dscrpt>
Les deux parties sont totalement indépendantes.\newline 
On ne confondra pas $(a,b,c)$ de $\R^3$ avec sa matrice colonne
$\begin{pmatrix} a \\ b \\ c \end{pmatrix}$ dans la base canonique.

\subsection*{Partie I}
Soit $u$ l'endomorphisme de $\R^3$ dont la matrice dans la base canonique est :
\begin{displaymath}
% use packages: array
A = \left[ \begin{array}{lll}
2 & 1 & 1 \\ 
1 & 2 & 1 \\ 
0 & 0 & 2
\end{array}
\right] 
\end{displaymath}
\begin{enumerate}
\item Discuter suivant le réel $\lambda$ du rang de la matrice $A-\lambda I_3$.
\item Déterminer pour chaque $i\in\{1,2,3\}$ le vecteur $e_i$ dont la deuxième composante vaut 1 et tel que $u(e_i)=ie_i$. 
Préciser $\ker (u-i\Id_{\R^3})$.

\item Justifier que $(e_1,e_2,e_3)$ est une base de $\R^3$ et écrire la matrice $\Delta$ de $u$ relativement à cette base. Former une relation entre $A$ et $\Delta$. 
\item Soit $B\in \mathcal{M}_{3}(\R)$ une matrice vérifiant $B^2=A$. On note $v$ l'endomorphisme de $\R^3$ dont la matrice dans la base canonique est $B$.
\begin{enumerate}
\item Justifier $v^2=u$ et $u\circ v = v\circ u$.
\item Pour chaque $i\in\{1,2,3\}$, montrer que $v(e_i)\in \Vect (e_i)$.
\item Montrer que la matrice de $v$ dans la base $(e_1,e_2,e_3)$ est de la forme
\begin{displaymath}
% use packages: array
\left[ 
\begin{array}{lll} 
\lambda_1 & 0 & 0 \\ 
0 & \lambda_2 & 0 \\ 
0 & 0 & \lambda_3
\end{array} 
\right] 
\end{displaymath}
 et préciser les valeurs possibles pour les $\lambda_i$.
\end{enumerate}
\item Former toutes les solutions dans $\mathcal{M}_{3}(\R)$ de l'équation $X^2=A$.
\end{enumerate}

\subsection*{Partie II}
Soit $E$ un $\R$ espace vectoriel de dimension $n$ et $u$ un endomorphisme de $E$ vérifiant
\[u\circ u = 0_{\mathcal{L}(E)}.\]
\begin{enumerate}
\item Montrer que $\rg u \leq \frac{n}{2}$.
\item Montrer que si $\rg u =r$, il existe une base de $E$ dans laquelle la matrice de $u$ s'écrit
\[
\begin{pmatrix}
0       & \cdots & 0        & 1 & 0       & \cdots  & 0 \\ 
        &        &          & 0 &  \ddots & \ddots  & \vdots \\ 
        &        &          &   &  \ddots & \ddots  & 0 \\ 
\vdots  &        &          &   &         &         & 1 \\ 
        &        &          &   &         &         & 0 \\ 
        &        &          &   &         &         & \vdots \\ 
0       &        & \cdots   &   &         &         & 0
\end{pmatrix}
\]
(toutes les cases contiennent 0 sauf $r$ qui contiennent 1).
\item Soit $M\in \mathcal{M}_{4}(\R)$ de rang 1 et telle que 
\[M^2=0_{\mathcal{M}_{4}(\R)}\]
Montrer qu'il existe des réels $(a,b,c,d)\neq (0,0,0,0)$ et $(x,y,z,t)\neq (0,0,0,0)$ tels que
\[
0 = xa+yb+zc+td, \hspace{0.5cm}
M = 
\begin{pmatrix}
xa & ya & za & ta \\
xb & yb & zb & tb \\
xc & yc & zc & tc \\
xd & yd & zd & td
\end{pmatrix}.
\]
 
\end{enumerate}

