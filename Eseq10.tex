%<dscrpt>Développement asymptotique d'une suite définie implicitement.</dscrpt>
\begin{enumerate}
  \item Montrer que pour tout $n\in \N$, l'équation
\begin{displaymath}
  e^x + n \ln(1+x) -2 = 0
\end{displaymath}
possède une unique solution $x_n \in \left] 0,1\right[$.
  \item Montrer que $\left( x_n\right)_{n\in \N}$ est décroissante et converge vers $0$.
  \item Montrer que $x_n \sim \frac{1}{n}$.
  \item 
\begin{enumerate}
  \item   Montrer qu'il existe des intervalles ouverts $I$ et $J$ contenant $0$ et tels que
\begin{displaymath}
  \left\lbrace 
  \begin{aligned}
    I &\rightarrow J \\ x&\mapsto \frac{\ln(1+x)}{2-e^{x}}
  \end{aligned}
\right. 
\end{displaymath}
soit bijective. On note $f$ cette fonction et $g$ sa bijection réciproque. 
  \item Montrer que $f$ et $g$ admettent des développements limités à tous les ordres.
  \item Former un développement limité à l'ordre $3$ en $0$ de $f(x)$.
  \item Former un développement limité à l'ordre $3$ en $0$ de $g(y)$.
\end{enumerate}

\item 
\begin{enumerate}
  \item Montrer que
\begin{displaymath}
  x_n = \frac{1}{n} - \frac{1}{2n^2} + \frac{5}{6n^3} + o(\frac{1}{n^3})
\end{displaymath}
  \item Déterminer un développement asymptotique de $\left( x_n\right)_{n\in \N}$ à la précision $o(\frac{1}{n^4})$.
\end{enumerate}

\end{enumerate}
