\begin{enumerate}
\item L'intersection de $\Gamma$ avec le plan $xOy$ a pour équation
\begin{displaymath}
\left\{\begin{array}{l} x^2+y^2=\frac 13 (z-2)^2\\ z=0
\end{array}\right. 
\Leftrightarrow
\left\{\begin{array}{l} x^2+y^2=\frac 43\\
z=0\end{array}\right.
\end{displaymath}
C'est un cercle de centre $O$ est de rayon $\frac{2\sqrt 3}3$ inclus dans le plan $xOy$.

\item
\begin{enumerate}
\item L'intersection de $\Gamma$ avec le plan d'équation $x=0$ a pour équation 
\begin{displaymath}
\left\{\begin{array}{l} x=0\\ x^2+y^2=\frac 13 (z-2)^2
\end{array}\right. 
\Leftrightarrow
\left\{\begin{array}{l} x=0 \\ y^2-\frac 13 (z-2)^2=0\end{array}\right.
\end{displaymath}
C'est la réunions de deux droites  d'équations 
\begin{displaymath}
\left\{\begin{array}{l} x=0 \\ y=\frac {\sqrt{3}} 3 (z-2)\end{array}\right.\hspace{1cm}
\left\{\begin{array}{l} x=0 \\ y=-\frac {\sqrt3} 3 (z-2)\end{array}\right.  
\end{displaymath}

\item \label{2} L'intersection de $\Gamma$ avec le plan d'équation $x=k$ pour $k\in\R^*$ a pour équation
\begin{displaymath}
\left\{\begin{array}{l} x=k\\ x^2+y^2=\frac 13 (z-2)^2
\end{array}\right. 
\Leftrightarrow
\left\{\begin{array}{l} x=k \\ \left(\frac {z-2}{\sqrt{3}k}\right)^2-\left(\frac yk\right)^2=1\end{array}\right. 
\end{displaymath}
 C'est une hyperbole.
\end{enumerate}
\item
\begin{enumerate}
\item Comme la base doit être orthonormée directe, on peut obtenir $\overrightarrow J_\varphi$ par produit vectoriel.
\begin{displaymath}
 \overrightarrow{J_\varphi}=
\overrightarrow{K_\varphi}\wedge\overrightarrow{K_\varphi}
=-\sin\varphi\overrightarrow{i}+\cos\varphi\overrightarrow{j}
\end{displaymath}

\item Exprimons les anciennes coordonnées en fonction des nouvelles :
\begin{displaymath}
 \left\lbrace 
\begin{aligned}
 &X_\varphi = \cos \varphi x + \sin \varphi y \\
 &Y_\varphi = -\sin \varphi x + \cos \varphi y\\
 &Z_\varphi = z 
\end{aligned}
\right. 
\Rightarrow
 \left\lbrace 
\begin{aligned}
 &x = \cos \varphi X_\varphi - \sin \varphi Y_\varphi \\
 &y = \sin \varphi X_\varphi + \cos \varphi Y_\varphi \\
 &z = Z_\varphi 
\end{aligned}
\right. 
\end{displaymath}
En remplaçant, on trouve comme équation de $\Gamma$ :
\begin{displaymath}
 X_\varphi^2+Y_\varphi^2=\frac 13 (Z_\varphi-2)^2
\end{displaymath}
\item  Soit $\cal P$ un plan parallèle à l'axe $Oz$. On peut trouver $\varphi$ tel que $\cal P$ ait pour équation  $X_\varphi=k$  avec $k\in\R$ dans le repère ${\cal R}_\varphi$. D'après l'équation trouvé pour $\Gamma$ dans  ${\cal R}_\varphi$ et la question \ref{2}, on en déduit que  l'intersection est une réunion de deux droites si $k=0$ et une hyperbole sinon. 
\end{enumerate}
\item
\begin{enumerate}
\item L'équation de $\Gamma$ dans ${\cal R}_A$ est $X^2+Y^2=\frac 13 Z^2$.
\item L'équation de $\mathcal P_{\overrightarrow n}$ est $aX+bY+cZ=0$.\newline
Plaçons nous d'abord dans le cas où $c\neq 0$.
\begin{multline*}
 M\in \Gamma \cap \mathcal P_{\overrightarrow n} 
\Leftrightarrow
\left\lbrace
\begin{aligned}
 &Z(M) = -\frac{a}{c}X(M)-\frac{b}{c}Y(M)\\
 &X(M)^2 + Y(M)^2 = \frac{1}{3}Z(M)^2
\end{aligned}
 \right. \\
\Leftrightarrow
\left\lbrace
\begin{aligned}
 &Z(M) = -\frac{a}{c}X(M)-\frac{b}{c}Y(M)\\
 &X(M)^2 + Y(M)^2 = \frac{1}{3c^2}(aX(M)+bY(M))^2 \quad (E)
\end{aligned}
 \right. 
\end{multline*}
L'équation $(E)$ s'écrit aussi
\begin{displaymath}
 (3c^2-a^2)X(M)^2 + (3c^2-b^2)Y(M)^2-2abX(M)Y(M)=0
\end{displaymath}
Considérons deux équations d'inconnue $x$
\begin{align*}
 &(3c^2-a^2)x^2 -2abx+ (3c^2-b^2)=0 & &(1)\\
 &(3c^2-b^2)x^2 -2abx+ (3c^2-a^2)=0 & &(2)
\end{align*}
Lorqu'elles sont de degré 2, elles ont le même discriminant
\begin{displaymath}
 4a^2b^2-4(3c^2-b^2)(3c^2-a^2)
= 12c^2(a^2+b^2-3c^2)
\end{displaymath}
Dans les cas particuliers $3c^2=a^2$ ou $3c^2=b^2$, la forme de l'équation $(E)$ montre clairement qu'il existe des points d'intersection autres que $A$. La condition $a^2+b^2-3c^2\geq 0$ est alors vérifiée.\newline
Dans le cas général, les équations sont du second degré.\newline
S'il existe un point $M$ autre que $A$ dans l'intersection, alors $X(M)$ et $Y(M)$ ne sont pas tous les deux nuls. 
\begin{align*}
 &\text{si } X(M)\neq 0 \text{ alors } \frac{Y(M)}{X(M)}\text{ est solution de }(1)\\
 &\text{si } Y(M)\neq 0 \text{ alors } \frac{X(M)}{Y(M)}\text{ est solution de }(2)
\end{align*}
Réciproquement, si le discriminant est positif, on peut considérer une solution $u_1$ de $(1)$. Le point de coordonnées
\begin{displaymath}
 (u_1,1,-\frac{a}{c}u_1-\frac{b}{c})
\end{displaymath}
est alors dans l'intersection.
\item D'après l'étude précédente, lorsque l'intersection ne se réduit pas à un point, elle est formée de l`union de deux droites éventuellemnt confondues lorsque le discriminant est nul.
\end{enumerate}
\end{enumerate}

