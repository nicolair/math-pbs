%<dscrpt>Comparaison suites et intégrales, développement limités et asymptotiques.</dscrpt>
\begin{enumerate}
\item
Quelques résultats sur les séries :\\
Soit $(x_n)_{n\in\N^*}$ une suite réelle, on appelle série de terme
général $x_n$ la suite
$\left(\sum\limits_{k=1}^nx_k\right)_{n\in\N^*}$. Si la suite
$\left(\sum\limits_{k=1}^nx_k\right)_{n\in\N^*}$ admet une limite
dans $\R$ on dit que la série converge, on note alors
$\sum\limits_{k=1}^{+\infty}x_k$ la limite.

\begin{enumerate}
                   \item Soit $f$ une fonction de $\R_+^*$ dans $\R$
                   décroissante. Encadrer $\sum\limits_{k=1}^n f(k)$
                   à l'aide d'intégrales de la fonction $f$ plus, éventuellement, une constante (indépendante de $n$).
                   \item Montrer que la série  $\sum\limits_{k=1}^n\frac 1k$
                   diverge et donner un équivalent de $\sum\limits_{k=1}^n\frac
                   1k$ lorsque $n$ tend vers $+\infty$.
                   %\item Montrer que la série $\sum\limits_{k=1}^n\frac{\ln k}{k^2}$ converge vers une constante $\lambda$ que l'on ne cherchera pas à déterminer.
\end{enumerate}

Soient $(x_n)$ et $(y_n)$ deux suites de réels telles que $y_n>0$ pour tout $n\in\N^*$ et
\begin{align*}
 x_n \sim_{n\to +\infty} y_n  &,& \lim\limits_{n\to +\infty}\sum\limits_{k = 1}^n y_k = + \infty
\end{align*}
On \textbf{admet} que  $\sum\limits_{k = 1}^n x_k \sim_{n\to
+\infty} \sum\limits_{k = 1}^n y_k $. \newline
On considère la suite $(u_n)$ définie par $u_0 \in \R$ et $u_{n + 1}
= u_n + e^{-u_n}$ pour $n\in\N$.

\item Montrer que $(u_n)$ est strictement croissante et tend vers $ + \infty$.

On pose alors, pour $n\in\N$,  $v_n = e^{u_n}$.

\item \begin{enumerate}
        \item Montrer que la suite $(v_{n + 1}-v_n)$ tend vers $1$.
        \item  En déduire un
équivalent de $v_n$ puis le premier terme du développement
asymptotique de $u_n$.
      \end{enumerate}
\item \begin{enumerate}
        \item  Montrer que $v_{n + 1}-v_n-1 \sim \dfrac 1{2n}$.
        \item  En déduire les deux premiers
termes du développement asymptotique de $v_n$.
\item Déterminer alors les deux premiers termes du développement asymptotique de $u_n$.
      \end{enumerate}


%\item Recommencer pour obtenir les trois premiers termes du développement asymptotique de $u_n$.

\end{enumerate}