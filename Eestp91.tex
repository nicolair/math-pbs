%<dscrpt>Intégrales généralisées et polynômes.</dscrpt>
\begin{enumerate}
\item On considère\footnote{partie de ESTP 1991} cinq nombres réels deux à deux distincts
$x_1,x_2,x_3,x_4,x_5$. \newline
Après avoir justifié l'intégrabilité de la fonction sur $\Bbb R$, montrer
qu'il existe des réels $\alpha_1,\alpha_2,\alpha_3,\alpha_4,\alpha_5$
tels que pour tout polynôme réel $P$ de degré inférieur ou égal à 4:
\[\int_{-\infty}^{+\infty}e^{-\frac{x^2}{2}}P(x)\,dx=\sum_{k=1}^{5}\alpha_k
P(x_k)\]
\item Soit $S$ le polynôme à coefficients réels défini par 
\[\forall x \in \Bbb R :\quad
S(x)=e^{\frac{x^2}{2}}\frac{d^5}{dx^5}(e^{\frac{-x^2}{2}})\]
Montrer que pour tout polynôme $P$ de degré inférieur ou égal à 4 on a 
\[\int_{-\infty}^{+\infty}e^{\frac{-x^2}{2}}S(x)P(x)\,dx=0\]
Donner les racines de $P$.
\item On se propose maintenant de choisir les nombres $x_1,x_2,x_3,x_4,x_5$. \begin{enumerate}
\item Montrer qu'il existe
$x_1,x_2,x_3,x_4,x_5,\alpha_1,\alpha_2,\alpha_3,\alpha_4,\alpha_5$
tels que pour tout polynôme réel de degré inférieur ou égal à 9 on ait
:
\[\int_{-\infty}^{+\infty}e^{-\frac{x^2}{2}}P(x)\,dx=\sum_{k=1}^{5}\alpha_k
P(x_k)\]
\item Trouver $x_1,x_2,x_3,x_4,x_5$. 
\end{enumerate} 
\end{enumerate} 
