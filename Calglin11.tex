\begin{enumerate}
 \item Si $P\in \ker u$ alors tous les coefficients du polynôme $u(P)$ sont nuls. On en déduit que, pour tous les $j$ entre 0 et $n$ :
\[\int_0^1t^jP(t)dt=0\]
En combinant linéairement ces relations, on en déduit que
\[\int_0^1Q(t)P(t)dt=0\]
pour n'importe quel polynome $Q$ de degré inférieur ou égal à $n$. Ceci est vrai en particulier pour $P$ lui même. Comme $t\rightarrow \tilde{P(t)}^2$ est une fonction continue et à valeurs positives,
\[\int_0^1P^2(t)dt=0\]
entraine que $P=0$.
\item Soit $M$ la matrice de $u$ dans la base $(1,X,\cdots,X^n)$  Avant même de la former on sait qu'elle est inversible puisque $u$ est bijective car son noyau se réduit au polynome nul.
\[m_{i,j}=\int_0^1t^{i-1}t^{j-1}=\frac{1}{i+j-1}\]
\[ M= \left[ 
\begin{array}{cccc}
1            & \frac{1}{2} & \cdots & \frac{1}{n+1} \\
\frac{1}{2}  & \frac{1}{3} &        & \\
\frac{1}{3}  &             &        & \\
\vdots       &             &        & \\
\frac{1}{n+1}&\frac{1}{n+2}&\cdots  & \frac{1}{2n+1}
\end{array}
\right] 
\]
\end{enumerate}
