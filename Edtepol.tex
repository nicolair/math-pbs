%<dscrpt>Forme complexe de la droite polaire d'un point par rapport à un cercle.</dscrpt>
On se place dans un plan euclidien orienté muni d'un repère orthonormé direct, à chaque point du plan est donc associée une affixe complexe.\newline
Dans ce problème\footnote{d'après \emph{Géométrie analytique classique} J-D Eiden (C\&M) p227}, aucun raisonnement géométrique ne sera pris en considération. Toutes les démonstrations doivent se faire à l'aide de calculs dans $\C$.

On note $\mathcal C$ le cercle (unité) formé par les points dont l'affixe appartient à $\U$. On rappelle que $\U$ est l'ensemble des nombres complexes de module 1.
\subsection*{I. Une nouvelle forme d'équation de droite.}
\begin{enumerate}
 \item Soit $\mathcal D$ une droite passant par un point $A$ (d'affixe  $a$) et orthogonale à un vecteur non nul $\overrightarrow u$ (d'affixe $u$). Soit $Z$ (d'affixe $z$) un point quelconque du plan.
 \begin{enumerate}
   \item Traduire $Z\in \mathcal D$ par une relation complexe.
   \item Déterminer des nombres complexes $w_0$, $w_1$ tels que :
\begin{displaymath}
 Z\in \mathcal D \Leftrightarrow z = w_0 + w_1 \overline{z}
\end{displaymath}
 Vérifier que si $w_0\neq 0$ alors
\begin{displaymath}
w_1 = - \frac{w_0}{\overline{w_0}} 
\end{displaymath}
 \end{enumerate}
 
 \item Soit $w$ un nombre complexe non nul fixé.
\begin{enumerate}
 \item Déterminer des nombres complexes $a$ et $u$ (avec $u$ non nul) tels que :
\begin{displaymath}
 \forall z \in \C : 
z - w + \frac{w}{\overline{w}}\overline{z}= \frac{1}{\overline{w}}\Re\left( (z-a)\overline{u}\right) 
\end{displaymath}
 
 \item Montrer que
\begin{displaymath}
 z = w - \frac{w}{\overline{w}}\overline{z}
\end{displaymath}
est l'équation d'une droite à préciser.
\end{enumerate}
\end{enumerate}

\subsection*{II. Droite polaire d'un point par rapport au cercle unité.}
Soit $M$ un point fixé du plan qui n'est pas l'origine du repère. Son affixe est $m\neq 0$. On note $\Delta_M$ l'ensemble des points $Z$ (d'affixe $z$) tels que
\begin{displaymath}
 z = \frac{2}{\overline{m}} - \frac{m}{\overline{m}}\, \overline{z}
\end{displaymath}
\begin{enumerate}
 \item Pourquoi $\Delta _M$ est-il une droite ?
 \item On considère l'équation $(E)$ d'inconnue $z$ :
\begin{align*}
(E) &  & z^2 - \frac{2}{\overline{m}}z + \frac{m}{\overline{m}}=0 
\end{align*}
En discutant suivant $|m|$, préciser les solutions complexes de $(E)$ et les modules de ces solutions.
\item Montrer que $\Delta_M \cap \mathcal C$ est vide lorsque $|m|<1$.
\item On suppose ici que  $|m|>1$.
\begin{enumerate}
 \item Montrer que $\Delta_M \cap \mathcal C$ est formé par deux points.
 \item Soit $U$ (d'affixe $u$) un des deux points de $\Delta_M \cap \mathcal C$. Calculer
\begin{displaymath}
 \Re\left( (u-m)\overline{u}\right) 
\end{displaymath}
 Quelle propriété géométrique peut-on en déduire? Faire un dessin.
\end{enumerate}
\end{enumerate}


\subsection*{III. Configuration géométrique}
\begin{figure}[ht]
 \centering
 \input{Edtepol_1.pdf_t}
 \caption{Configuration géométrique}
 \label{fig:Edtepol_1}
\end{figure}
\begin{enumerate}
 \item Soit $a$ et $b$ deux nombres complexes de module $1$ tels que $a+b\neq 0$. Montrer que
\begin{displaymath}
 \frac{a + b}{\overline{a} + \overline{b}} = ab
\end{displaymath}
\item Soit $A$ et $B$ deux points de $\mathcal C$ (respectivement d'affixes $a$ et $b$ avec $a+b\neq 0$). Soit $M$ (d'affixe $m$) un point quelconque du plan. Montrer que $M$ est sur la droite $(AB)$ si et seulement si :
\begin{displaymath}
 m = a+b -ab \,\overline{m}
\end{displaymath}
\item On se donne quatre points $A$, $B$, $C$, $D$ (affixes $a$, $b$, $c$, $d$ avec $a+b\neq 0$ et $c+d\neq 0$) sur $\mathcal C$.  et on suppose que les droites $(AB)$ et $(CD)$ se coupent en un point $M$ (affixe $m$).
\begin{enumerate}
 \item Montrer que $ab-cd\neq 0$.
 \item Montrer que 
\begin{displaymath}
 m = \frac{ab(c+d)-cd(a+b)}{ab - cd}
\end{displaymath}
\item Montrer que
\begin{displaymath}
 \overline{m} = \frac{a + b - c - d}{ab -cd}
\end{displaymath}
\item Montrer que 
\begin{align*}
 ab=\frac{a + b -m}{\overline{m}} & & cd=\frac{c + d -m}{\overline{m}} 
\end{align*}

\end{enumerate}
\item On suppose que les points du cercle sont dans la configuration de la figure \ref{fig:Edtepol_1}, En particulier, aucun de ces points n'est diamétralement opposé à un autre et les droites que l'on peut former se coupent deux à deux en $M$, $M'$, $M''$.
\begin{enumerate}
 \item Déterminer les affixes $m'$ et $m''$ de $M'$ et $M''$ ainsi que leurs conjugués.
 \item Montrer que 
\begin{align*}
 m' = \frac{2}{\overline{m}} - \frac{m}{\overline{m}}\, \overline{m'}
& &
 m'' = \frac{2}{\overline{m}} - \frac{m}{\overline{m}}\, \overline{m''}
\end{align*}
Que peut-on en conclure pour les points $M'$ et $M''$?
\end{enumerate}

\end{enumerate}
