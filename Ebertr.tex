%<dscrpt>Convergence au sens d'une famille de Bertrand.</dscrpt>
\subsection*{Famille de Bertrand}
On appelle \emph{famille de Bertrand} une suite $(f_{n})_{n\in \mathbb{N}}$ d'applications continues, intégrables de $\mathbb{R}^{+}$ dans $\mathbb{R}^{+}$, dont l'intégrale sur $\mathbb{R}^{+}$ vaut 1 et pour laquelle il existe un réel $M>0$ tel que
$$\forall n \in \mathbb{N}, \forall t \in \mathbb{R}^{+}:0\leq f_{n}(t)\leq M$$

Soit $(a_{n})_{n\in\mathbb{N}}$ une suite de réels. On dit que la série $\sum a_{n}$ \emph{converge au sens de} $(f_{n})_{n\in \mathbb{N}}$ si et seulement si:

\begin{itemize}
\item Pour tout $t>0$ la suite $(\sum _{k=0}^{n}a_{k}f_{k}(t))_{n\in \mathbb{N}}$ converge. On note alors $f(t)$ sa limite.
\item La fonction $f$ est intégrable sur$\mathbb{R}^{+}$ avec $S'=\int_{0}^{+\infty}f(t)\,dt$.
\end{itemize}

Le nombre $S'$ s'appelle alors la $(f_{n})$ somme de la série $\sum a_{n}$.

\begin{enumerate}
\item On pose, pour tout $n\in \mathbb{N}$ et $t\in \mathbb{R}^{+}$,
$$f_{n}(t)=\frac{t^{n}e^{-t}}{n!}.$$

\begin{enumerate}
\item Calculer pour tout $n\in \mathbb{N}$, $M_{n}=\sup_{t\geq 0}f_{n}(t)$.
\item Montrer que la suite $(M_n)_{n\in \mathbb{N}}$ est décroissante.
\item Montrer que $(f_n)_{n\in \mathbb{N}}$ est une famille de Bertrand.
\item Étudier la convergence au sens de $(f_{n})_{n\in \mathbb{N}}$ de la série $\sum a_{n}$ dans les cas suivants
\begin{enumerate}
\item $a_{n}=(-1)^{n}$
\item $a_{n}=(-1)^{n}n$
\end{enumerate}
\end{enumerate}

\item Pour tout $n\in \mathbb{N}$ on définit l'application $f_{n}$ de $\mathbb{R}^{+}$ dans $\mathbb{R}^{+}$ par:
\[f_{n}(t)=\left \{
\begin{array}{ccc}
0 & \mathrm{si} & |t-(n+1)|>1\\
1-|t-(n+1)| & \mathrm{si} & |t-(n+1)| \leq 1 \\
\end{array}
\right .
\]
\begin{enumerate}
\item Représenter graphiquement $f_{0},f_{1},f_{2},f_{3}$.
\item Montrer que la suite $(f_n)_{n\in \mathbb{N}}$ est une famille de Bertrand.
\item Soit $(a_n)_{n\in \mathbb{N}}$ une suite de réels. Montrer que $$(\sum _{k=0}^{n}a_{k}f_{k}(t))_{n\in \mathbb{N}}$$ converge pour tout $t\geq 0$ et donner sa limite sur l'intervalle $[p,p+1[$. En déduire $\int_{p}^{p+1}f(t)\,dt$.
\item Montrer que si la suite $(\sum _{k=0}^{n}a_k)_{n\in \mathbb{N}}$ converge alors la série $\sum a_{n}$ converge au sens de $(f_n)_{n\in \mathbb{N}}$ et donner un contre-exemple pour prouver que la réciproque est fausse.
\end{enumerate}

\end{enumerate}

