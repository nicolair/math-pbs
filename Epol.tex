%<dscrpt>Polynômes et tan : relations entre coefficients et racines.</dscrpt>
\begin{enumerate}
\item Pour tout entier $n \in \N$, on définit le polynôme
\[Q_{n}=(1+X)^{2n}-(1-X)^{2n}\]
Préciser le degré de $Q_{n}$, ses termes de plus haut et de plus bas degré.
\item \begin{enumerate} \item Factoriser $Q_{n}$ dans $\C[X]$ en produit de facteurs irréductibles.
\item En déduire sa factorisation dans $\R[X]$ en produit de facteurs irréductibles.
\item En déduire la valeur de 
\[\prod_{k=1}^{n-1}\tan ^{2}\frac{k\pi }{2n}\]
\end{enumerate}

\item Dans cette question $t$ est un réel strictement positif.
\begin{enumerate}
\item Exprimer 
\[
\prod_{k=1}^{n-1}\left( 1+\frac{t^{2}}{4n^{2}\tan ^{2}\frac{k\pi }{2n}}\right)
\]
à l'aide de $Q_{n}$
\item En déduire la convergence et la limite quand $n\rightarrow +\infty$ de
\[
\prod_{k=1}^{n-1}\left( 1+\frac{t^{2}}{4n^{2}\tan ^{2}\frac{k\pi }{2n}}\right)
\]
en fonction de $t$ et de $\sinh t$
\end{enumerate}
\end{enumerate}
