\subsection*{PARTIE I}
\begin{enumerate}
\item Il est évident que $E$ est un sous-espace vectoriel de dimension 4 de $\mathcal{M}_{3}(\C)$ dont
\[
\left [
\begin{array}{ccc}
0 & 1 & 1\\
1 & 0 & 1\\
1 & 1 & 1
\end{array}
\right ]
,\left [
\begin{array}{ccc}
1 & 0 & 0\\
0 & 0 & 0\\
0 & 0 & 0
\end{array}
\right ]
,\left [
\begin{array}{ccc}
0 & 0 & 0\\
0 & 1 & 0\\
0 & 0 & 0
\end{array}
\right ]
\left [
\begin{array}{ccc}
0 & 0 & 0\\
0 & 0 & 0\\
0 & 0 & 1
\end{array}
\right ]
\]
est une base. Le calcul du produit donne
\begin{eqnarray*}
M(a,a,a,b)M(a',a',a',b')=M(\alpha,\alpha,\alpha,\alpha,\beta)\\
\text{avec } \alpha=aa'+bb',\beta=ab'+a'b+bb'
\end{eqnarray*}
\item D'après la question précédente, $M(a,a,a,b)M(a',a',a',b')=I$ si et seulement si
\[
\left\{
\begin{array}{c@=c}
aa'+(2b)b'&1\\
ba'+(a+b)b'&0
\end{array}
\right.
\]
Le déterminant de ce système aux inconnues $a',b'$ est
\[a(a+b)-2b^{2}=a^{2}+ab-2b^{2}=(a-b)(a+2b)\]
On en déduit que $M(a,a,a,b)$ est inversible lorsque $a\neq b$ et $a+2b\neq 0$. Dans ce cas, la matrice inverse est $M(a',a',a',b')$ avec 

\begin{eqnarray*}
a'=\frac
{\left |
\begin{array}{cc}
1 & 2b\\
0 & a+b 
\end{array}
\right |
}
{(a-b)(a+2b)}
=\frac{a+b}{(a-b)(a+2b)}\\
b'=\frac
{\left |
\begin{array}{cc}
a & 1\\
b & 0 
\end{array}
\right |
}
{(a-b)(a+2b)}
=\frac{-b}{(a-b)(a+2b)}
\end{eqnarray*}
\item Matriciellement, le sytème proposé s'écrit
\[M(a,a,a,b)
\left (\begin{array}{c}x\\y\\z\end{array}\right )
=\left (\begin{array}{c}a^{2}-3\\2a-4\\-2\end{array}\right )
\]
\begin{itemize}
\item si $a\not \in \{1,-2\}$. La matrice $M(a,a,a,b)$ est inversible et les résultats de la question précédente conduisent à une seule solution
\[
\left (\begin{array}{c}x\\y\\z\end{array}\right )
=\frac{1}{a+2}
\left (\begin{array}{c}a^{2}+2a-3\\a-1\\-(a+5)\end{array}\right )
\]
\item si $a=1$. Le système se réduit à la seule équation $x+y+z=-2$. L'ensemble des solutions est 
\[
\{\left (\begin{array}{c}-2\\0\\0\end{array}\right )
+y\left (\begin{array}{c}-1\\1\\0\end{array}\right )
+z\left (\begin{array}{c}-1\\0\\1\end{array}\right )
,(y,z)\in \C^{2}
\}\]
\item si$a=-2$. Le système devient
\[\left\{
\begin{array}{c@=c}
-2x+y+z&1\\
x-2y+z&-8\\
x+y-2z&-2
\end{array}
\right.
\]
il est sans solution car la somme des trois équations donne $0=9$.
\end{itemize}
\end{enumerate}

\subsection*{PARTIE II}
\begin{enumerate}
\item Si $A=M(1,1,1,-1)$,$A^{2}=M(3,3,3,-1)=A+2I$. Posons $$u_{0}=0,v_{0}=1,u_{1}=1,v_{1}=0$$ pour que $A^{0}=u_{0}A+v_{0}I$ et $A^{1}=u_{1}A+v_{1}I$. Si $A^{n-1}=u_{n-1}A+v_{n-1}I$ alors
\[
A^{n}=u_{n-1}A^{2}+v_{n-1}A=(u_{n-1}+v_{n-1})A+2u_{n-1}I=u_{n}A+v_{n}I
\]
avec \[
u_{n}=u_{n-1}+v_{n-1},v_{n}=2u_{n-1}\]
\item On peut écrire les relations précédentes sous la forme
\[
v_{n}=v_{n-1}+2v_{n-2},u_{n}=\frac{1}{2}v_{n+1}\]
L'équation caractéristique de la relation de récurrence linéaire d'ordre 2 est $X^{2}-X-2=(X+1)(X-2)$. Les suites vérifiant cette relation sont des combinaisons de suites géométriques de raison -1 et 2. En tenant compte des conditions initiales, on trouve après calcul 
\[
v_{n}=\frac{2}{3}(-1)^{n}+\frac{1}{3}2^{n},
u_{n}=\frac{1}{3}(-1)^{n+1}+\frac{1}{3}2^{n}\]
On en déduit
\[
A^{n}=(\frac{1}{3}(-1)^{n+1}+\frac{1}{3}2^{n})A+(\frac{2}{3}(-1)^{n}+\frac{1}{3}2^{n})I
\]
\item La matrice $Q=2I-A$ n'est formée que de 1. Elle vérifie $Q^{2}=3Q,Q^{n}=3^{n-1}Q$. Comme $I$ et $Q$ commutent, on peut utiliser la formule du binôme dans $A^{n}=(2I-Q)^{n}$. On en déduit
\[
A^{n}=2^{n}I+(\sum _{k=1}^{n}\binom{n}{k}2^{n-k}(-1)^{k}3^{k-1})Q.
\]
La somme devant $Q$ s'écrit encore
\[\frac{1}{3}((2-3)^{n}-2^{n})=\frac{1}{3}(-1)^{n}-\frac{1}{3}2^{n}\]
En reinjectant $Q=2I-A$, on obtient
\begin{eqnarray*}
A^{n}&=&2^{n}I+(\frac{1}{3}(-1)^{n}-\frac{1}{3}2^{n})(2I-A)\\
&=& (\frac{1}{3}(-1)^{n+1}+\frac{1}{3}2^{n})A+(\frac{2}{3}(-1)^{n}+\frac{2}{3}2^{n})I
\end{eqnarray*}
qui est la même expression que dans la question précédente.
\end{enumerate}

\subsection*{PARTIE III}
\begin{enumerate}
\item Le déterminant d'une matrice 3,3 est une somme de produits. Chaque produit est formé de trois facteurs (un par colonne).Comme $\lambda$ figure une seule fois dans chaque colonne, le déterminant de $B_{a}-\lambda I$ est donc un polynôme de degré 3 en $\lambda$. Le terme de plus haut degré est $(-\lambda)^{3}$. Il est obtenu dans le produit des trois termes de la diagonale.
\newline Après calcul, on trouve
\[P(\lambda)=-\lambda^{3}+3\lambda^{2}+a^{2}\lambda-a^{2}-4\]
\item De plus, en $-\infty$ $P\rightarrow +\infty$, en 0 $P(0)=-a^{2}-4<0$ et en 2 $P(2)=a^{2}>0$. Le polynôme $P$ admet donc trois racines $\lambda_{1},\lambda_{2},\lambda_{3}$ telles que 
\[\lambda_{1}<0<\lambda_{2}<2<\lambda_{3}\]
\item Le nombre $\lambda$ est racine de $P$ si et seulement si $B_{a}-\lambda I$ n'est pas inversible c'est à dire si et seulement si il existe une colonne \emph{non nulle} $X$ telle que
\[(B_{a}-\lambda I)X=0\] ce qui s'écrit aussi $B_{A}X=\lambda X$.
\item Il s'agit simplement de chercher une solution $X$ de l'équation précédente en imposant la première ligne de $X$ égale à $2-\lambda$. On trouve après calcul
\[\left (
\begin{array}{c}
 2-\lambda\\
2+a-\lambda \\
\lambda^{2}-(a+2)\lambda+a
\end{array}
\right )
\]
\end{enumerate}
