%<dscrpt>Birapport de quatre droites dans un plan vectoriel.</dscrpt>
On utilisera les notations suivantes pour d{\'e}signer certains
vecteurs et certaines droites de $\R^{2}$.
\begin{eqnarray*}
U &=&(0,1),\quad D_{1}=\text{Vect}(U) \\
V &=&(1,0),\quad D_{2}=\text{Vect}(V) \\
\lambda  &\in &\R,\quad W_{\lambda }=(\lambda ,1),\quad
\Delta
_{\lambda }=\text{Vect}(W_{\lambda }) \\
T &=&(1,1),\quad D_{4}=\text{Vect}(T) \\
u_{\theta } &=&(\cos \theta ,\sin \theta ),\quad \delta _{\theta }=\text{Vect%
}(u_{\theta })
\end{eqnarray*}
La base canonique $\mathcal{C}$ de $\R^{2}$ s'{\'e}crit alors
$(V,U). $

On dira aussi qu'une famille $(d_{1},d_{2},\cdots ,d_{m})$ de
droites vectorielles est \emph{r{\'e}guli{\`e}re }si et seulement
si il existe $f\in GL(\R^{2})$ telle que
\[
\left\{ d_{1},d_{2},\cdots ,d_{m}\right\} =\left\{ f(\delta _{0}),f(\delta _{%
\frac{\pi }{m}}),\cdots ,f(\delta _{\frac{(m-1)\pi }{m}})\right\}
\]
Le dernier ensemble est form{\'e} par les images ensemblistes des droites
vectorielles.

Toutes les droites et tous les vecteurs consid{\'e}r{\'e}s dans le
probl{\`e}me sont dans $\R^{2}$.

\subsection*{I.\qquad Familles de droites}

Soient $d_{1},d_{2},d_{3},d_{4}$ quatre droites deux {\`a} deux distinctes.

\begin{enumerate}
\item
\begin{enumerate}
\item  Montrer que tout couple de $d_{1}\times d_{2}$ form{\'e} de vecteurs
non nuls est un syt{\`e}me libre.

\item  Montrer que $(d_{1},d_{2})$ est r{\'e}guli{\`e}re, citer
pr{\'e}cis{\'e}ment le th{\'e}or{\`e}me de cours utilis{\'e}.
\end{enumerate}

\item
\begin{enumerate}
\item  Montrer qu'on peut trouver des vecteurs $v_{1},v_{2},v_{3}$ non nuls
respectivement dans $d_{1},d_{2},d_{3}$ et tels que
\[
v_{1}+v_{3}=v_{2}
\]

\item  Exprimer $u_{0}+u_{\frac{2\pi }{3}}$ comme un $u_{\theta }$, en
d{\'e}duire que $(d_{1},d_{2},d_{3})$ est r{\'e}guli{\`e}re.
\end{enumerate}

\item  Montrer qu'il existe $\lambda \in \mathbf{R-}\left\{ 0,1\right\} $ et
des vecteurs $w_{1},w_{2},w_{3},w_{4}$ tels que
\begin{eqnarray*}
\forall i &\in &\left\{ 1,\cdots ,4\right\} \quad d_{i}=\text{Vect}(w_{i}) \\
w_{3} &=&w_{1}+\lambda w_{2} \\
w_{4} &=&w_{1}+w_{2}
\end{eqnarray*}

\item  Montrer qu'il existe $\lambda \in \mathbf{R-}\left\{ 0,1\right\} $ et
$f\in GL(\R^{2})$ tels que
\[
d_{1}=f(D_{1}),\quad d_{2}=f(D_{2}),\quad d_{3}=f(\Delta _{\lambda }),\quad
d_{4}=f(D_{4})
\]
\end{enumerate}

\subsection*{II.\qquad Birapport de quatre droites}

Etant donn{\'e}s quatre vecteurs $V_{i}=(x_{i},y_{i})$ ( i$\in \left\{
1,\cdots ,4\right\} $) suppos{\'e}s deux {\`a} deux ind{\'e}pendants, on
pose
\[
\forall (i,j)\in \in \left\{ 1,\cdots ,4\right\} ^{2}\quad \beta
_{ij}=x_{i}y_{j}-x_{j}y_{i}
\]

\begin{enumerate}
\item  Exprimer $\beta _{ij}$ {\`a} l'aide d'un d{\'e}terminant, en
d{\'e}duire $\beta _{ij}\neq 0$ pour $i\neq j$.

On note
\[
b(V_{1},V_{2},V_{3},V_{4})=\frac{\beta _{13}\,\beta _{24}}{\beta
_{14}\,\beta _{23}}
\]

\item  Pour $\lambda _{1},\lambda _{2},\lambda _{3},\lambda _{4}$ des
r{\'e}els non nuls, calculer $b(\lambda _{1}V_{1},\lambda _{2}V_{2},\lambda
_{3}V_{3},\lambda _{4}V_{4})$ et justifier la d{\'e}finition suivante.

\begin{quote}
Soit $d_{1},d_{2},d_{3},d_{4}$ des droites deux {\`a} deux distinctes et $V_{1},V_{2},V_{3},V_{4}$ des vecteurs non nuls respectivement dans $d_{1},d_{2},d_{3},d_{4}$;\newline
le birapport de $(d_{1},d_{2},d_{3},d_{4})$ est le nombre not{\'e} $\mathcal{B}(d_{1},d_{2},d_{3},d_{4})$ d{\'e}fini par
\begin{displaymath}
 \mathcal{B}(d_{1},d_{2},d_{3},d_{4})=b(V_{1},V_{2},V_{3},V_{4})
\end{displaymath}
\end{quote}


\item  Soit $f\in GL(\R^{2})$ et $d_{1},d_{2},d_{3},d_{4}$ des
droites deux {\`a} deux distinctes, montrer que
\[
\mathcal{B}(f(d_{1}),f(d_{2}),f(d_{3}),f(d_{4}))=\mathcal{B}%
(d_{1},d_{2},d_{3},d_{4})
\]

\item  Calculer $\mathcal{B}(\delta _{0},\delta _{\frac{\pi }{4}},\delta _{%
\frac{\pi }{2}},\delta _{\frac{3\pi }{4}})$

\item  Pour quelles valeurs de $\lambda $ les droites $D_{1},D_{2},\Delta
_{\lambda },D_{4}$ sont elles deux {\`a} deux distinctes ? Dans ce cas
calculer $\mathcal{B}(D_{1},D_{2},\Delta _{\lambda },D_{4})$.
\end{enumerate}

\subsection*{III.\qquad Quelques propri{\'e}t{\'e}s de groupes finis}

\begin{enumerate}
\item  Montrer que les quatre transpositions $(1,2),$ $(1,4),$ $(3,2),$ $%
(3,4)$ engendrent le groupe $\mathcal{S}_{4}$ des permutations de $\in
\left\{ 1,2,3,4\right\} $.

\item  On d{\'e}finit des bijections $i,a,b,c,d,e$ de $\mathbf{R-}\left\{
0,1\right\} $ dans $\mathbf{R-}\left\{ 0,1\right\} $ en posant
\begin{eqnarray*}
\forall \lambda  &\in &\mathbf{R-}\left\{ 0,1\right\} :\quad i(\lambda
)=\lambda ,\quad a(\lambda )=\frac{1}{\lambda },\quad b(\lambda )=1-\lambda
\\
c &=&a\circ b,\quad d=b\circ a,\quad e=a\circ b\circ a
\end{eqnarray*}
Montrer que toute compos{\'e}e de $a$ et $b$ est dans $\left\{
i,a,b,c,d,e\right\} .$ (on pourra utiliser le nombre de bijections
intervenant dans la composition)\newline
En d{\'e}duire que $\left\{ i,a,b,c,d,e\right\} $ est le sous-groupe du
groupe des bijections de $\mathbf{R-}\left\{ 0,1\right\} $ dans lui m{\^e}me
engendr{\'e} par $a$ et $b$.
\end{enumerate}

\subsection*{IV.\qquad Effet d'une permutation sur le birapport}

Soit $\lambda \in \mathbf{R-}\left\{ 0,1\right\} $ et $%
d_{1},d_{2},d_{3},d_{4}$ quatre droites deux {\`a} deux distinctes, on pose
ici $D_{3}=\Delta _{\lambda }$.

\begin{enumerate}
\item  Exprimer
\begin{displaymath}
\mathcal{B(}D_{\sigma (1)},D_{\sigma (2)},D_{\sigma (3)},D_{\sigma (4)}) 
\end{displaymath}
en fonction de $\lambda $ pour $\sigma \in \left\{(1,2),(1,4),(3,2),(3,4)\right\} $.

\item  Montrer que $\mathcal{B(}D_{\sigma (1)},D_{\sigma (2)},D_{\sigma
(3)},D_{\sigma (4)})$ peut prendre au plus six valeurs lorsque $\sigma $
d{\'e}crit $\mathcal{S}_{4}$. Pr{\'e}ciser ces valeurs en fonctions de $%
\lambda $.

\item  Montrer que $\mathcal{B(}d_{\sigma (1)},d_{\sigma (2)},d_{\sigma
(3)},d_{\sigma (4)})$ peut prendre au plus six valeurs lorsque $\sigma $
d{\'e}crit $\mathcal{S}_{4}$.

\item  Pour quelles valeurs r{\'e}elles de $\mathcal{B(}%
d_{1},d_{2},d_{3},d_{4})$ l'ensembles des $\mathcal{B(}d_{\sigma
(1)},d_{\sigma (2)},d_{\sigma (3)},d_{\sigma (4)})$ avec $\sigma \in
\mathcal{S}_{4}$ contient-il strictement moins de six {\'e}l{\'e}ments ?

\item  Donner une condition n{\'e}cessaire et suffisante assurant que $%
(d_{1},d_{2},d_{3},d_{4})$ est r{\'e}guli{\`e}re.
\end{enumerate}
