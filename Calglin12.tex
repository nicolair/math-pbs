\subsubsection*{Partie A}
\begin{enumerate}
 \item Pour montrer que $\sim$ est une relation d'équivalence, on doit montrer que la relation est réflexive, symétrique est transitive.
\begin{itemize}
 \item Réflexive. $A\sim A$ car on peut choisir $P=I$ (matrice unité).
\item Symétrique. Si $\sim B$, il existe $P$ inversible telle que 
\begin{displaymath}
 B=P^{-1}AP \Rightarrow A = Q^{-1}BQ
\end{displaymath}
avec $Q=P^{-1}$ inversible donc $B\sim A$.
\item Transitive. Si $A\sim B$ et $B\sim C$, il existe des matrices inversibles $P$ et $Q$ telles que 
\begin{displaymath}
 \left. 
\begin{aligned}
 B =& P^{-1}AP\\
 C =& Q^{-1}BQ
\end{aligned}
\right\rbrace 
\Rightarrow
C=(Q)^{-1}(P)^{-1}A(PQ)=(PQ)^{-1}A(PQ)
\end{displaymath}
donc $C\sim A$ car $PQ$ est inversible.
\end{itemize}

\item Il s'agit en fait de montrer que deux matrices semblables ont le même déterminant. Cela résulte de ce que le déterminant d'un produit est le produit des déterminants.
\begin{multline*}
 B=P^{-1}AP \Rightarrow \det (B) =\det(P^{-1}AP)=\det(P^{-1})\det(A)\det(P)\\=\det(A)\det(P)\det(P^{-1})
= \det(A)\det(PP^{-1})=\det(A)
\end{multline*}

\item \begin{enumerate}
 \item Si $x\in \Ima w$, il existe $y\in \ker (u^{i+j})$ tel que 
\begin{displaymath}
x=w(y)=u^j(y) \Rightarrow  u^i(x)=u^{i+j}(y)=0_E
\end{displaymath}
donc $x\in \ker u^i$. Ceci prouve 
\[\Ima w \subset \ker u^i\]
\item Appliquons à $w$ le théorème du rang:
\[\dim (\ker u^{i+j}) = \dim (\ker w) +\dim (\Ima w)\]
avec 
\begin{displaymath}
\ker w =\ker u^{i+j}\cap \ker u^j = \ker u^j 
\end{displaymath}
et $\Ima w \subset \ker u^i$. On en déduit
\[\dim (\ker u^{i+j}) \lq \dim (\ker u^i) +\dim (\ker u^j)\]
\end{enumerate}
\item \begin{enumerate}
 \item Soit $u$ un endomorphisme de rang 2 tel que $u^3=O$. Comme la dimension de $E$ est 3, son noyau est de dimension 1. D'après l'inégalité de la question précédente
\begin{eqnarray*}
 \dim (\ker u^2) &\leq& 2 \dim (\ker u)=2 \\
 3=\dim (\ker u^3) &\leq&  \dim (\ker u^2) + \dim (\ker u) \leq \dim (\ker u^2)+1
\end{eqnarray*}
On en déduit les ideux inégalités prouvant $\dim (\ker u^2)=2$.
\item Comme $\dim (\ker u^2)=2$, l'endomorphisme $u^2$ n'est pas identiquement nul. Il existe donc un vecteur $a$ tel que $u^2(a)\neq 0_E$. \`A fortiori $u(a)$ et $a$ sont non nuls.\newline
Pour montrer que la famille $(u^2(a),u(a),a)$ est une base, il suffit de prouver qu'elle est libre.\newline
On considère une combinaison nulle et on compose par $u^2$. On en déduit la nullité du coefficient de $a$. En composant ensuite par $u$ on obtient la nullité du coefficient de $u(a)$. Il ne reste plus qu'un coefficient qui est forcément nul.
\item Pour $v=u^2-u$, on désigne par $U$ la matrice de $u$ et par $V$ celle de $v$ dans la base $(u^2(a),u(a),a)$. On obtient:
\[
U= \begin{pmatrix}
    0 & 1 & 0 \\
0 & 0 & 1\\
0 & 0 & 0
   \end{pmatrix}
,\;
 U^2=\begin{pmatrix}
0 & 0 & 1 \\
0 & 0 & 0\\
0 & 0 & 0
   \end{pmatrix}
,\;
 V= \begin{pmatrix}
0 & -1 & 1 \\
0 & 0 & -1\\
0 & 0 & 0
   \end{pmatrix}
\]

\end{enumerate}

\end{enumerate}
\subsubsection*{Partie B}
\begin{enumerate}
 \item Deux matrices semblables ont le même déterminant donc $\det A = \det T = 1$ car $T$ est triangulaire avec des 1 sur la diagonale. Les deux matrices sont donc inversibles.
\item Le calcul montre que $N^3$ est la matrice nulle. On en déduit que 
\[(I_3+N)(I_3-N+N^2)=I_3-N^3=I_3\]
Les deux matrices sont donc inversibles et inverses l'une de l'autre. On en déduit
\[(P^{-1}AP)^{-1}=I_3-N+N^2\]
avec
\[(P^{-1}AP)^{-1}=P^{-1}A^{-1}P\]
\item Dans cette question $N=0$, donc $A$ est semblable à $I$. Comme $I$ commute avec $P$, $A$ est égal à $I$. Les matrices $A$ et $A^{-1}$ sont donc \emph{plus que semblables, elles sont égales} et égales à $I$.
\item \begin{enumerate}
 \item On a ici $\rg(A)=2$ et $M=N^{2}-N$. Comme $N^3=0$, on peut appliquer la question 4. de la partie A. Il existe une "bonne base" dans laquelle la matrice de l'endomorphisme représenté par $N$ est
\[
 \begin{pmatrix}
0 & 1 & 0 \\
0 & 0 & 1 \\
0 & 0 & 0
  \end{pmatrix}
\]
donc $M$ est semblable à
\[
- \begin{pmatrix}
0 & 1 & 0 \\
0 & 0 & 1 \\
0 & 0 & 0
  \end{pmatrix}
+
 \begin{pmatrix}
0 & 0 & 1 \\
0 & 0 & 0 \\
0 & 0 & 0
  \end{pmatrix}
=
 \begin{pmatrix}
0 & -1 & 1 \\
0 & 0 & -1 \\
0 & 0 & 0
  \end{pmatrix}
 \] 
\item Par le calcul, $M^{3}=0$, le rang de $M$ est clairement 2.
\item Pourquoi les matrices
\[
 \begin{pmatrix}
0 & 1 & 0 \\
0 & 0 & 1 \\
0 & 0 & 0
  \end{pmatrix}
,\;
\begin{pmatrix}
0 & -1 & 1 \\
0 & 0 & -1 \\
0 & 0 & 0
  \end{pmatrix}
 \] 
sont-elles semblables?\newline
Car on peut appliquer à l'endomorphisme représenté par $M$ et par la deuxième matrice la question 4.b. de la partie A. (il est de rang 2 et sa puissance d'ordre 3 est nulle, il existe alors une "bonne base")
\item Si $M$ et $N$ sont semblables, alors $I+M$ et $I+N$ sont aussi semblables. Or $A \sim I+N$ et 
\[A^{-1}\sim (I+N)^{-1}=I-N+N^2=I+M \sim A\]
\end{enumerate}
\item On a ici $\rg(A)=1$ et $M=N^{2}-N$. Notons $n$ l'endomorphisme dont la matrice dans la base canonique est $N$. Comme il est de rang 1 avec $n^3=0$ (immédiat à vérifier) son noyau est de dimension 2 avec
\begin{displaymath}
 \left\lbrace 
\begin{aligned}
 \Ima n \subset& \ker n \\
\text{ou}& \\
\Im n\text{ et }& \ker n \text{ supplémentaires}
\end{aligned}
\right. 
\end{displaymath}
La deuxième proposition est incompatible avec le caractère nilpotent. Il existerait en effet un vecteur $a$ non nul tel que $n(a)=\lambda a$ (l'image est une droite stable) avec $\lambda$ non nul (l'intersection noyau -image est réduite au vecteur nul). Mais alors 
\begin{displaymath}
 n^3(a)=\lambda ^3 a \neq 0
\end{displaymath}
On doit donc avoir
\[\Ima n \subset \ker n\]
Considérons alors une base $(a,b,c)$ avec $(c)$ base de $\Ima n$ et $(a,b)$ base de $\ker n$. La matrice de $n$ dans une telle base est
\[
 \begin{pmatrix}
0 & 0 & 1 \\
0 & 0 & 0 \\
0 & 0 & 0
  \end{pmatrix}
\]
On en déduit $N^2=0$, $M^2=-N$. De plus on a alors :
\[(I+N)^{-1}=I-N\]
Donc 
\begin{align*}
 A\sim I+N & & A^{-1}\sim I-N
\end{align*}
Pourquoi les deux matrices $I+N$ et $I-N$ sont-elles semblables ?\newline
Car, si $N$ est la matrice de $n$ dans $(a,b,c)$, alors $-N$ est la matrice de $n$ dans $(-a,b,c)$.\newline
Ceci prouve encore que 
\begin{displaymath}
 A \sim A^{-1}
\end{displaymath}
.
\item \begin{enumerate}
 \item On forme la matrice de $u-Id_E$ dans la base $(a,b,c)$ de l'énoncé.
\[
A= \begin{pmatrix}
0 & 0 & 0 \\
0 & -1 & -1 \\
0 & 1 & 1
          \end{pmatrix} \]
Son rang est 1 donc $\dim (\ker (u-Id_E))=2$.\newline
On lit facilement sur la matrice que 
\begin{displaymath}
 (e_1,e_2)=(a,b-c)
\end{displaymath}
est une base de $\ker (u-Id_E)$ et que $(b-c)$ est une base de $\Ima (u-Id_E)$.
\item La famille $(a,b-c,c)$ est une base car elle contient trois vecteurs et engendre $E$.\newline
En effet les vecteurs de $(a,b,c)$ s'expriment en fonction de $(a,b-c,c)$.
\[
\begin{array}{ccc}
a=a , & b= (b-c) +c , & c=c 
  \end{array}\]
De plus $u(c)=-b+2c=-(b-c)+c$. La matrice de $u$ dans $(a,b-c,c)$ est donc 
\[
\begin{pmatrix}
1 & 0 & 0 \\
0 & 1 & -1 \\
0 & 0 & 1
\end{pmatrix}
 \]
\item Les matrices $A$ et $A^{-1}$ sont semblables car on se trouve dans le cas de la question 5 avec $A$ semblable à une matrice $I+N$ avec $N$ de rang 1.
\item Toute matrice semblable à son inverse est-elle de la forme $T$ ?\newline
La réponse est \emph{non}. Exemple :
\begin{align*}
 A  =&
 \begin{pmatrix}
 1 & 0 & 0 \\
 0 & 2 & 0 \\
 0 & 0 & \frac{1}{2}
 \end{pmatrix} 
= \underset{(a,b,c)}{\mathop{\mathrm{Mat}}}u\\
A^{-1} =&
\begin{pmatrix}
 1 & 0 & 0 \\
 0 & \frac{1}{2} & 0 \\
 0 & 0 & 2
\end{pmatrix}
 = \underset{(a,b,c)}{\mathop{\mathrm{Mat}}}u^{-1}
 = \underset{(a,c,b)}{\mathop{\mathrm{Mat}}}u 
\end{align*}
La matrice $A$ est bien semblable à son inverse.\newline
Pourquoi n'est-elle pas semblable à une matrice de la forme $T$ ?\newline
Car deux matrices semblables ont la même trace alors que  
\[\tr A =1 +2 +\frac{1}{2}\neq \tr T = 3\]
\end{enumerate}

\end{enumerate}
