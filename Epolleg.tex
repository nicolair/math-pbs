%<dscrpt>Polynômes de Legendre.</dscrpt>
Pour tout $n\in\N$, on définit les polynômes $P_n$ et $L_n$ de $\R[X]$ par :
\begin{displaymath}
 P_n=(X^2-1)^n, \hspace{0,5cm} L_n=P^{(n)}_n
\end{displaymath}
Les polynômes $L_n$ sont appelés les \emph{polynômes de Legendre}.
\begin{enumerate}
\item Degré, coefficient dominant et parité
\begin{enumerate}
\item Expliciter $L_0$, $L_1$ et $L_2$.
\item Pour tout $n\in\N$, déterminer le degré, ainsi que le coefficient dominant du polynôme $L_n$.
\item Pour $n\in\N$, étudier la parité de $L_n$. (on dit qu'un polynôme est pair quand il est conservé par substitution de $X$ par $-X$ et qu'il est impair quand il est changé en son opposé)
\end{enumerate}
\item  Valeurs de $L_n(1)$ et de $L_n(-1)$.
\begin{enumerate}
\item A l'aide de la formule de Leibniz, donner, pour $n\in\N$, le polynôme $L_n$ sous forme d'une somme de polynômes.
\item  En déduire les valeurs de $L_n(1)$ et de $L_n(-1)$.
\end{enumerate}
\item Racines de $L_n$\begin{enumerate}
\item Pour tout $n\in\N^*$ et pour tout entier $k\in\{0,\dots,n-1\}$, déterminer les valeurs de $P^{(k)}_n(-1)$ et   $P^{(k)}_n(1)$.
\item Montrer que, pour tout $n\in\N$, le polynôme $L_n$ est scindé à racines simples et que toutes ses racines sont dans $]-1,1[$.
\end{enumerate}
\item Relation entre les polynômes de Legendre
\begin{enumerate}
\item Former une relation entre $P'_{n+1}$ et $P_n$ pour $n\in\N$, puis une relation entre $P''_{n+1}$, $P_n$ et $P_{n-1}$ pour $n\in\N^*$.
\item A l'aide de ces deux relations, déterminer une relation liant $L_{n+2}$, $L_{n+1}$ et $L_n$ pour $n\in\N$.
\end{enumerate}
\end{enumerate}
