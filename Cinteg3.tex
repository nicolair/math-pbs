\begin{enumerate}
 \item Une primitive de $\ln x$ est $x\ln x -x$. Une primitive de $\ln(x+K)$ est
\begin{displaymath}
 (x+k)\ln(x+K)-x
\end{displaymath}
\item Le calcul de $I_n$ se fait en utilisant les primitives de la question précédente :
\begin{multline*}
 I_n = \int_{0}^n \left( \ln(x+n+1) -\ln(x+1)\right) dx \\
 = \left[ (x+n+1)\ln(x+n+1)-x - (x+1)\ln(x+1)+ x\right]_{x=0}^{x=n}\\
= (2n+1)\ln(2n+1) - 2(n+1)\ln(n+1) 
\end{multline*}
\item Calcul du développement de $I_n$. On se ramène à des développements en $0$ en factorisant par $n$ dans les $\ln$.
\begin{align*}
 \ln(2n+1) &= \ln n + \ln 2 +\ln\left(1+\dfrac{1}{2n} \right)
=  \ln n + \ln 2 +\dfrac{1}{2n} -\dfrac{1}{8n^2}+o(\dfrac{1}{n^2}) \\
\ln(n+1) &= \ln n  +\ln\left(1+\dfrac{1}{n} \right)
=  \ln n + \ln 2 +\dfrac{1}{n} -\dfrac{1}{2n^2}+o(\dfrac{1}{n^2})
\end{align*}
On multiplie ensuite respectivement par $(2n+1)$ et $(n+1)$, le reste devient un $o(\frac{1}{n})$ et on obtient après calculs:
\begin{displaymath}
 I_n = (2\ln 2)n -\ln n + (\ln 2 -1) -\dfrac{3}{4}\dfrac{1}{n}+o(\dfrac{1}{n})
\end{displaymath}
\item \begin{enumerate}
 \item On utilise les propriétés élémentaires de la fonction inverse  :
\begin{displaymath}
 \forall x\geq i, \forall y \geq j : 
\dfrac{1}{x+y+1} \leq \dfrac{1}{i+j+1} 
\end{displaymath}
 puis on intégre sur des segments de longueur 1
\begin{displaymath}
 \int _{j}^{j+1} \dfrac{1}{x+y+1}dy \leq \dfrac{1}{i+j+1}
\end{displaymath}
\begin{displaymath}
 \int_{i}^{i+1}\left( \int _{j}^{j+1} \dfrac{1}{x+y+1}dy\right) dx \leq \dfrac{1}{i+j+1}
\end{displaymath}
\item Le raisonnement est analogue de l'autre coté sur $[i-1,i]\times[j-1,j]$.

\item En découpant en intervalles de longueur $1$, l'intégrale $I_n$ devient par relation de Chasles une somme double.
\begin{displaymath}
 I_n = \sum_{i=0}^{n-1}\left( \sum_{j=0}^{n-1}\int_{i}^{i+1}\left( \int _{j}^{j+1} \dfrac{1}{x+y+1}dy\right) dx \right) 
\end{displaymath}
Pour majorer, on utilise a., $S_n$ apparait naturellement comme majorant
\begin{displaymath}
 I_n \leq
\sum_{i=0}^{n-1}\left( \sum_{j=0}^{n-1} \dfrac{1}{i+j+1}\right) 
\end{displaymath}
Pour minorer on utilise b., le terme de droite de l'inégalité est formé par une partie des termes de $S_{n+1}$.
\begin{multline*}
 I_n \geq
\sum_{i=1}^{n}\left( \sum_{j=1}^{n} \dfrac{1}{i+j+1}\right) = S_{n+1} -
\sum_{i=1}^n\dfrac{1}{i+0+1} - \sum_{j=0}^n\dfrac{1}{0+j+1} \\
\geq S_{n+1} -2\sum_{k=0}^n\dfrac{1}{k+1}+1 
\end{multline*}
On en déduit
\begin{displaymath}
 I_{n-1} \geq S_n -2\sum_{k=1}^n\dfrac{1}{k}+1,\hspace{1cm}
 S_n \leq I_{n-1} +2\sum_{k=1}^n\dfrac{1}{k} - 1
\end{displaymath}
\end{enumerate}

\item D'après la question 3., les suites $(\frac{I_n}{n})$ et $(\frac{I_{n-1}}{n})$ convergent vers $2\ln2$. On obtiendra donc l'équivalence demandée par un simple théorème d'encadrement à condition de démontrer d'abord que la somme des $\frac{1}{k}$ est négligeable devant $n$. Cela se fait classiquement par comparaison avec une intégrale.
\begin{displaymath}
 1+\dfrac{1}{2}+\cdots+\dfrac{1}{n}\leq
1 +\int_{1}^{n}\dfrac{dx}{x}
= 1 + \ln n
\end{displaymath}
\item On peut écrire
\begin{displaymath}
 \left( \sum _{k=0}^{n-1}x^k\right)^2 = \left( \sum _{i=0}^{n-1}x^i\right)\left( \sum _{j=0}^{n-1}x^j\right)
\end{displaymath}
et tout développer pour obtenir une somme double puis intégrer en utilisant la linéarité. Les intégrales élémentaires se calculent et on obtient :
\begin{displaymath}
 J_n = S_n
\end{displaymath}
On en déduit
\begin{displaymath}
 J_n \sim 2n\ln 2
\end{displaymath}
ou encore
\begin{displaymath}
 \int_{0}^1\dfrac{1-x^n}{1-x}dx \sim 2n\ln 2
\end{displaymath}

\end{enumerate}
