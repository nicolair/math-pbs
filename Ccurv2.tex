\subsection*{PARTIE I}

\begin{enumerate}
\item  Comme $\rho _{\alpha }(\theta )=\frac{\cos \alpha }{\cos \alpha \cos
2\theta -\sin \alpha \sin 2\theta }=\frac{\cos \alpha }{\cos (2\theta
+\alpha )}=\cos \alpha \,\rho _{0}(\theta +\frac{\alpha }{2})$, on passe de $%
\mathcal{C}_{0}$ \`{a} $\mathcal{C}_{\alpha }$ par la similitude de centre $O
$, d'angle $-\frac{\alpha }{2}$ et de rapport $\cos \alpha $.

\item  Comme $\rho _{\alpha }(\theta +\frac{\pi }{2})=-\rho _{\alpha
}(\theta )$, $M_{\alpha }(\theta +\frac{\pi }{2})=r_{0,-\frac{\pi }{2}%
}(M_{\alpha }(\theta ))$.

\item  On r\'{e}duit l'intervalle d'\'{e}tude de la courbe
param\'{e}tr\'{e}e $M_{0}$ \`{a} $\left[ 0,\frac{\pi }{2}\right] -\{\frac{%
\pi }{4}\}$ en remarquant que $M_{0}(\theta +\pi )=s_{O}(M_{0}(\theta ))$ et
que $M_{0}(-\theta )=s_{Ox}(M_{0}(\theta )).$\newline
La fonction $\rho $ est positive dans $\left[ 0,\frac{\pi }{4}\right[ $ et
n\'{e}gative dans $\left] \frac{\pi }{4},\frac{\pi }{2}\right] $, elle
diverge vers $+\infty $ \`{a} droite de $\frac{\pi }{4}$ et vers $-\infty $
\`{a} gauche de $\frac{\pi }{4}$. La branche infinie correspondante admet
une direction asymptotique $\overrightarrow{u}_{\frac{\pi }{4}}$. Etudions
l'existence d'une asymptote en formant 
\[
\overrightarrow{OM_{0}}(\theta )\cdot \overrightarrow{u}_{-\frac{\pi }{4}}=%
\frac{a\left( \cos \theta -\sin \theta \right) }{\cos ^{2}\theta }=\frac{a}{%
\cos \theta +\sin \theta }=\frac{a}{\sqrt{2}\cos (\theta -\frac{\pi }{4})}
\]
Cette expression converge en $\frac{\pi }{4}$ vers $\frac{a}{\sqrt{2}}$ en
restant toujours $\geq \frac{a}{\sqrt{2}}$. La droite 
\[
x-y=\frac{a}{\sqrt{2}}
\]
est donc asymptote (pour la valeur $\frac{\pi }{4}$), la courbe est au
dessous cette droite.

\item 
\begin{enumerate}
\item  Dans les questions a.\ et b., $\theta _{0}$ est un r\'{e}el fix\'{e}.
Si $\theta _{0}$ est congru \`{a} 0 modulo $\pi $, $M_{\alpha }(\theta _{0})$
est d\'{e}fini pour tous les $\alpha $. Sinon, le seul $\alpha _{0}$ pour
lequel $M_{\alpha }(\theta _{0})$ n'est pas d\'{e}fini est $\arctan \cot
(2\theta _{0})$.

\item  La direction de la tangente en $M_{\alpha }(\theta _{0})$ \`{a} $%
\mathcal{C}_{\alpha }$ est $\rho _{\alpha }^{\prime }(\theta _{0})%
\overrightarrow{u}_{\theta _{0}}+\rho _{\alpha }(\theta _{0})\overrightarrow{%
v}_{\theta _{0}}$. Notons $(X_{M},Y_{M})$ les cordonn\'{e}es de $%
\overrightarrow{OM}$ dans la base $(\overrightarrow{u}_{\theta _{0}},%
\overrightarrow{v}_{\theta _{0}})$. Un point $M$ est sur la tangente $%
\mathcal{D}_{\alpha }(\theta _{0})$ si et seulement si 
\begin{eqnarray*}
\rho _{\alpha }(\theta _{0})(X_{M}-\rho _{\alpha }(\theta _{0}))-\rho
_{\alpha }^{\prime }(\theta _{0})Y_{M}=0 \\
(\cos 2\theta _{0}-\tan \alpha \sin 2\theta _{0})X_{M}-2(\sin 2\theta
_{0}-\tan \alpha \cos 2\theta _{0})Y_{M}=a
\end{eqnarray*}
Le point d'intersection de toutes les courbes $\mathcal{D}_{\alpha }(\theta
_{0})$ s'obtient en identifiant \`{a} 0 les coefficients de 1 et de $\tan
\alpha $ dans l'\'{e}quation de $\mathcal{D}_{\alpha }(\theta _{0})$. On
obtient le syst\`{e}me 
\begin{eqnarray*}
\cos 2\theta _{0}X_{M}-2\sin 2\theta _{0}Y_{M}=a \\
\sin 2\theta _{0}X_{M}+2\cos 2\theta _{0}Y_{M}=0
\end{eqnarray*}
qui conduit \`{a} $X_{M}=a\cos 2\theta _{0}$, $Y_{M}=-\frac{a}{2}\sin
2\theta _{0}$ soit 
\[
\overrightarrow{OP}(\theta _{0})=a\cos 2\theta _{0}\overrightarrow{u}%
_{\theta _{0}}-\frac{a}{2}\sin 2\theta _{0}\overrightarrow{v}_{\theta _{0}} 
\]
\end{enumerate}

\item  Par d\'{e}finition, $P(\theta )$ appartient \`{a} la tangente en $%
M_{\alpha }(\theta )$ \`{a} $\mathcal{C}_{\alpha }$.De plus 
\[
\overrightarrow{P^{\prime }}(\theta )=-\frac{3a}{2}\sin 2\theta 
\overrightarrow{u}_{\theta } 
\]
donc $\overrightarrow{OM}_{\alpha }(\theta )$ appartient \`{a} la direction
de la tangente des $P(\theta )$. On note $\Gamma $ le support de $P$.

\item  D'apr\`{e}s le calcul pr\'{e}c\'{e}dent, $P(\theta )$ est
stationnaire lorsque $\theta \equiv 0\quad (\frac{\pi }{2})$.\ Les points
correspondants sont 
\[
P(0)=O+a\overrightarrow{i},\quad P(\frac{\pi }{2})=O-a\overrightarrow{j}%
,\quad P(\pi )=O-a\overrightarrow{i},\quad P(\frac{3\pi }{2})=O+a%
\overrightarrow{j}
\]
La direction de la tangente en un point stationnaire est donn\'{e}e par la
premi\`{e}re d\'{e}riv\'{e}e non nulle or $\overrightarrow{P^{\prime \prime }%
}(\theta )=-3a\cos 2\theta \overrightarrow{u}_{\theta }-\frac{3a}{2}\sin
2\theta \overrightarrow{v}_{\theta }$ donc $\overrightarrow{P^{\prime \prime
}}(\theta )$ est dans la direction de$\overrightarrow{u}_{\theta }$ lorsque $%
\theta \equiv 0\quad (\frac{\pi }{2})$. La condition (2) est donc encore
v\'{e}rifi\'{e}e$.$

\item  Le calcul de la courbure se fait \`{a} l'aide de la base $(%
\overrightarrow{u}_{\theta },\overrightarrow{v}_{\theta })$%
\[
C(\theta )=\frac{1}{\left\| \overrightarrow{P^{\prime }}(\theta )\right\|
^{3}}\det (\overrightarrow{P^{\prime }}(\theta ),\overrightarrow{P^{\prime
\prime }}(\theta ))=\frac{2}{3a\left| \sin 2\theta \right| } 
\]

\item  Si on revient aux coordonn\'{e}es $(x(\theta ),y(\theta )),$ on
trouve $x(\theta )=a\cos ^{3}\theta $, $y(\theta )=-a\sin ^{3}\theta .$
Malgr\'{e} le moins, le support est obtenu \`{a} partir de l'astro\"{i}de
par l'homoth\'{e}tie de rapport $a$ car l'astro\"{i}de est sym\'{e}trique
par rapport \`{a} l'axe des $y$.
\end{enumerate}

\subsection*{PARTIE II}

\begin{enumerate}
\item  Comme $\overrightarrow{w^{\prime }}(\theta )=(\phi ^{\prime }(\theta
)+\psi (\theta ))\overrightarrow{u}_{\theta }+(\phi (\theta )-\psi ^{\prime
}(\theta ))\overrightarrow{v}_{\theta }$, la condition assurant que $\
\overrightarrow{w^{\prime }}(\theta )$ est colin\'{e}aire \`{a} $%
\overrightarrow{u}_{\theta }$ est 
\[
\phi (\theta )-\psi ^{\prime }(\theta )=0 
\]

\item  La d\'{e}finition de $P$ montre clairement que la condition (1) est
v\'{e}rifi\'{e}e. On remarque que, comme $\rho $ ne s'annule pas, la courbe $%
M$ n'admet pas de point stationnaire.\newline
Comme $\overrightarrow{OP}(\theta )=(\rho -h\rho ^{\prime })\overrightarrow{u%
}_{\theta }-h\rho \overrightarrow{v}_{\theta }$, on peut poser $\phi =\rho
-h\rho ^{\prime }$ et $\psi =h\rho $. La condition pr\'{e}c\'{e}dente
s'\'{e}crit alors 
\[
\rho -2h\rho ^{\prime }-h^{\prime }\rho =0 
\]

\item  L'ensemble des solutions de l'\'{e}quation sans second membre est $%
Vect(\frac{1}{\rho ^{2}})$. Cherchons une solution de l'\'{e}quation
compl\`{e}te sous la forme $\frac{\lambda }{\rho ^{2}}$. Une telle fonction
est solution lorsque $\lambda ^{\prime }=\rho ^{2}$. Comme $J$ d\'{e}signe
une primitive de $\rho ^{2}$, l'ensemble des solutions de l'\'{e}quation
compl\`{e}te est $\{\frac{J+\lambda }{\rho ^{2}},\lambda \in \mathbf{R}\}$.

\item  D'apr\`{e}s la question pr\'{e}c\'{e}dente, si $K_{\lambda }=\frac{%
J+\lambda }{\rho }$, les courbes d\'{e}finies par 
\[
\overrightarrow{OP}_{\lambda }=\overrightarrow{OM}-\frac{K_{\lambda }}{\rho }%
\overrightarrow{M^{\prime }} 
\]
v\'{e}rifient les deux conditions de l'introduction. En exprimant $%
\overrightarrow{M^{\prime }}$, il vient 
\[
\overrightarrow{OP}_{\lambda }=(\rho -K_{\lambda }\frac{\rho ^{\prime }}{%
\rho })\overrightarrow{u}_{\theta }-K_{\lambda }\overrightarrow{v}_{\theta
}=K_{\lambda }^{\prime }\overrightarrow{u}_{\theta }-K_{\lambda }%
\overrightarrow{v}_{\theta } 
\]
car $K_{\lambda }^{\prime }=\frac{J^{\prime }}{\rho }-\frac{(J+\lambda )\rho
^{\prime }}{\rho ^{2}}=\rho -\frac{\rho ^{\prime }}{\rho }K_{\lambda }$.

\item  Un point $P_{\lambda }(\theta _{0})$ est stationnaire si$%
\overrightarrow{\text{ }P^{\prime }}_{\lambda }(\theta _{0})=(K_{\lambda
}^{\prime \prime }(\theta _{0})+K_{\lambda }(\theta _{0}))\overrightarrow{u}%
_{\theta _{0}}=\overrightarrow{0}$. Or 
\[
K_{\lambda }^{\prime \prime }+K_{\lambda }=\frac{J+\lambda }{\rho ^{3}}%
\left( \rho ^{2}-\rho \rho ^{\prime \prime }+2\rho ^{\prime 2}\right) =\frac{%
J+\lambda }{\rho ^{3}}\det (\overrightarrow{M^{\prime }},\overrightarrow{%
M^{\prime \prime }}) 
\]
en consid\'{e}rant le d\'{e}terminant dans la base $(\overrightarrow{u}%
_{\theta },\overrightarrow{v_{\theta }})$. Le point $P_{\lambda }(\theta
_{0})$ est donc stationnaire lorsque $M(\theta _{0})$ n'est pas
bir\'{e}gulier ou bien lorsque $J+\lambda $ s'annule ce qui ne peut se
produire qu'une fois car $J$ est par d\'{e}finition croissante. Rappelons
que $M(\theta _{0})$ ne peut pas \^{e}tre stationnaire car $\rho $ ne
s'annule pas.

\item  Supposons $P_{\lambda }(\theta )$ est stationnaire avec
\[
\overrightarrow{P_{\lambda }^{\prime }}(\theta )=\overrightarrow{P_{\lambda
}^{\prime \prime }}(\theta )=\cdots =\overrightarrow{P_{\lambda }^{(n-1)}}%
(\theta )=\overrightarrow{0},\quad \overrightarrow{P_{\lambda }^{(n)}}%
(\theta )\neq \overrightarrow{0} 
\]
On sait alors que la formule de Taylor-Young montre que la tangente au
support de $P$ est de direction $\overrightarrow{P_{\lambda }^{(n)}}(\theta
) $. Il s'agit donc de v\'{e}rifier que ce vecteur est colin\'{e}aire \`{a} $%
\overrightarrow{OM}$ c'est \`{a} dire \`{a} $\overrightarrow{u_{\theta }}$.%
\newline
Il est clair que la formule de d\'{e}rivation de Leibniz est encore valable
pour le produit d'une fonction num\'{e}rique par une fonction vectorielle.
On en d\'{e}duit, en posant $\phi =K_{\lambda }^{\prime \prime }+K_{\lambda
} $ : 
\begin{eqnarray*}
\overrightarrow{P_{\lambda }^{\prime \prime }}=\phi ^{\prime }%
\overrightarrow{u}_{\theta }+\phi \overrightarrow{v}_{\theta } \\
\overrightarrow{P_{\lambda }^{\prime \prime \prime }}=(\phi ^{\prime \prime
}-\phi )\overrightarrow{u}_{\theta }+2\phi ^{\prime }\overrightarrow{v}%
_{\theta } \\
\overrightarrow{P_{\lambda }^{(4)}}=(\phi ^{(3)}-3\phi ^{\prime })%
\overrightarrow{u}_{\theta }+(3\phi ^{\prime \prime }-\phi )\overrightarrow{v%
}_{\theta } \\
\vdots \\
\overrightarrow{P_{\lambda }^{(n)}}=(\phi ^{(n-1)}-C_{n-1}^{2}\phi
^{(n-2)}+\cdots )\overrightarrow{u}_{\theta }+((n-1)\phi
^{(n-2)}-C_{n-1}^{3}\phi ^{(n-3)}+\cdots )\overrightarrow{v}_{\theta }
\end{eqnarray*}
A cause de la forme triangulaire des coefficients, la nullit\'{e} des
d\'{e}riv\'{e}es successives jusqu'\`{a} l'ordre $n-1$ entra\^{i}ne $\phi
(\theta )=\phi ^{\prime }(\theta )=\cdots =\phi ^{(n-2)}(\theta )=0.$ La non
nullit\'{e} de la d\'{e}riv\'{e}e n-i\`{e}me entra\^{i}ne $\phi
^{(n-1)}(\theta )\neq 0$ et 
\[
\overrightarrow{P_{\lambda }^{(n)}(\theta )}=\phi ^{(n-1)}(\theta )%
\overrightarrow{u}_{\theta } 
\]
\end{enumerate}

\subsection*{PARTIE III}

\begin{enumerate}
\item  D'apr\`{e}s II, une courbe param\'{e}tr\'{e}e $P$ v\'{e}rifiant les
conditions de l'introduction est de la forme $P=M-h\,\overrightarrow{%
M^{\prime }}$ avec $h$ solution de l'\'{e}quation diff\'{e}rentielle (3). De
cette \'{e}quation on d\'{e}duit que si $\rho (\alpha )=\rho (\beta )=0$ et $%
\rho ^{\prime }(\alpha )\neq 0$, $\rho ^{\prime }(\beta )\neq 0$ (pas de
point stationnaire), $h(\alpha )=h(\beta )=0$. On suppose aussi (ce que
l'\'{e}nonc\'{e} ne pr\'{e}cise pas) que $\rho $ ne s'annule pas entre $%
\alpha $ et $\beta $. Dans $\left] \alpha ,\beta \right[ $, $h=\frac{%
J+\lambda }{\rho ^{2}}$, $J$ est strictement monotone donc $h$ admet une
branche infinie en $\beta $.

\item 
\begin{enumerate}
\item  On applique la formule de Taylor-Young \`{a} $\rho $ puis on calcule
le d\'{e}veloppement de $\rho ^{2}$ que l'on int\`{e}gre. On aboutit \`{a} 
\begin{eqnarray*}
\psi (\theta ) &=&\frac{a}{3}(\theta -\alpha )^{2}+(\theta -\alpha
)^{2}\varepsilon (\theta ) \\
\psi ^{\prime }(\theta ) &=&\frac{2a}{3}(\theta -\alpha )+(\theta -\alpha
)\varepsilon (\theta )
\end{eqnarray*}

\item  La fonction $\psi $ se prolonge par continuit\'{e} en posant $\psi
(\alpha )=0$. La d\'{e}monstration du caract\`{e}re $\mathcal{C}^{\infty }$
de ce prolongement se fait en deux \'{e}tapes. On montre d'abord que toutes
les $\psi ^{(n)}(\theta )$ convergent quand $\theta \rightarrow 0$ avec $%
\theta \neq \alpha $. On raisonne ensuite par r\'{e}currence \`{a} l'aide du
th\'{e}or\`{e}me de la limite de la d\'{e}riv\'{e}e.\newline
La d\'{e}monstration de la convergence se fait en calculant $\psi
^{(n)}=(J\rho ^{-1})^{(n)}$ \`{a} l'aide de la formule de Leibniz puis en
interpr\'{e}tant le r\'{e}sultat \`{a} l'aide d'une formule de
Taylor-Lagrange. (Trait\'{e} en classe).

\item  Le raisonnement de la partie II reste valable dans $I-\{\alpha \}.$
Il montre que $P$ doit \^{e}tre de la forme $M-h\overrightarrow{M^{\prime }}$
avec $h=\frac{J+\lambda }{\rho ^{2}}$. Il est indispensable que $\lambda =0$
pour que le point $P(\alpha )$ soit d\'{e}fini. La question
pr\'{e}c\'{e}dente montre alors que la courbe param\'{e}tr\'{e}e $P$ est $%
\mathcal{C}^{\infty }$. De plus, ici $K_{0}=\psi $en utilisant a., 
\[
\overrightarrow{OP}(\theta )=\psi ^{\prime }(\theta )\overrightarrow{u}%
_{\theta }-\psi (\theta )\overrightarrow{v}_{\theta }=\frac{2a}{3}(\theta
-\alpha )\overrightarrow{u}_{\theta }+(\theta -\alpha )^{2}\overrightarrow{%
\varepsilon (\theta )} 
\]
avec $\overrightarrow{\varepsilon (\theta )}\rightarrow \overrightarrow{0}$.
Ceci montre que $\frac{1}{\left\| \overrightarrow{OP}(\theta )\right\| }%
\overrightarrow{OP}(\theta )\rightarrow \overrightarrow{u}_{\alpha }$ La
tangente \`{a} l'origine est bien colin\'{e}aire \`{a} $\overrightarrow{u}%
_{\alpha }$.
\end{enumerate}
\end{enumerate}

\subsection*{PARTIE IV}

\begin{enumerate}
\item  La fonction $\rho =\frac{1}{\theta }$ d\'{e}cro\^{i}t de +$\infty $
\`{a} 0, d'o\`{u} une allure de spirale pour le support avec convergence
vers $O$ en +$\infty $. Comme $\rho \sin \theta \rightarrow 1$ quand $\theta
\rightarrow 0$, la droite $y=1$ est asymptote.

\item  Choisissons la primitive $J(\theta )=-\frac{1}{\theta }$ de $\frac{1}{%
\rho ^{2}};$ ainsi $K_{\lambda }(\theta )=\frac{J+\lambda }{\rho }=\lambda
\theta -1$ et (II 4.) :
\[
\overrightarrow{OP}_{\lambda }(\theta )=\lambda \overrightarrow{u_{\theta }}%
-(\lambda \theta -1)\overrightarrow{v_{\theta }}
\]

\item  A cause des conditions impos\'{e}es \`{a} $P_{\lambda }$ le calcul de
la courbure est facilit\'{e} par le fait $\overrightarrow{P^{\prime }}%
_{\lambda }$ est port\'{e} par $\overrightarrow{u_{\theta }}$. Suivant le
signe de $\lambda \theta -1$, l'angle orient\'{e} $\alpha (\theta )$ entre $%
\overrightarrow{i}$ et $\overrightarrow{P^{\prime }}_{\lambda }$ est $\theta
$ ou $\theta +\pi $. On en d\'{e}duit la courbure, le vecteur normal et le
centre de courbure (malheureusement not\'{e} $K_{\lambda })$%
\[
c_{\lambda }(\theta )=\frac{1}{\left\| \overrightarrow{P^{\prime }}_{\lambda
}\right\| }=\frac{1}{\left| \lambda \theta -1\right| },\quad \overrightarrow{%
n}_{\theta }=sgn(\lambda \theta -1)\overrightarrow{v}_{\theta },\quad
K_{\lambda }(\theta )=0+\lambda \overrightarrow{u_{\theta }}
\]

\item  Le support de la courbe $K_{\lambda }$ est un cercle de centre $0$ et
de rayon $\lambda $. La courbe $P_{\lambda }$ est une d\'{e}veloppante de
cercle. On remarque que $P_{\lambda }(\frac{1}{\lambda })$ est stationnaire
car $\frac{1}{\lambda }$ annule le $J+\lambda $. Sur la figure c'est le
point de rebroussement intersection du cercle avec sa d\'{e}veloppante. En
ce point la spirale et la d\'{e}veloppante de cercle sont tangentes
conform\'{e}ment \`{a} II 6.
\end{enumerate}
