%<dscrpt>Courbes paramétrées (attention énoncé à retravailler).</dscrpt>
Le but de ce problème est d'étudier un procédé de transformation de courbes planes.\newline
Le plan est muni d'un repère orthonormé direct $(O,\overrightarrow{i}),\overrightarrow{j})$ qui permet d'identifier un point ou un vecteur et son affixe. On pourra utiliser librement la forme la plus commode. On utilisera par exemple un point $A$ ou son affixe complexe $a$, le vecteur $\overrightarrow{e_{\theta}}$ ou le complexe $e^{i\theta}$.


Soit $I$ un intervalle de $\R$ contenant 0, soit deux fonctions $r$ et $\theta$ de classe $\mathcal{C}^2(I,\R)$. On suppose de plus que $\theta'(t)>0$ pour tous le $t\in I$. Ces fonctions définissent une courbe paramétrée à valeurs complexes
\[\forall t \in I ,\medskip  f(t)=r(t)e^{i\theta(t)}\]
On pose $A_0=f(0)$ d'affixe $a_0$.\newline
On définit la courbe paramétrée $g$ dans $I$ en posant
\[g(t)=a_0+\int_{0}^{t}f'(u)e^{-i\theta(u)}\,du\]
Le support de $f$ est noté $\Phi$, celui de $g$ est noté $\Gamma$.
\begin{enumerate}
\item Soit $\varphi$ un changement de paramètre admissible d'un intervalle $J$ contenant 0 vers $I$. On suppose $\varphi(0)=0$. Montrer que la courbe paramétrée obtenue par le procédé décrit au dessus à partir de $f\circ \varphi$ est $g\circ \varphi$. Que peut-on en déduire ?
\item Soit $t_1$ et $t_2$ deux éléments de $I$, $A_1=f(t_1)$, $A_2=f(t_2)$, $B_1=g(t_1)$, $B_2=g(t_2)$. Montrer que les longueurs des arcs $A_1A_2$ et $B_1B_2$ sont égales.
\item Soit $t \in I$ et $A=f(t)$, $B=g(t)$.\newline
On note $M$ le point d'intersection de la normale à $\Phi$ en $A$ et de la droite perpendiculaire à $(OA)$ passant par $O$.\newline
On note $N$ le point d'intersection de la normale à $\Gamma$ en $B$ et de la droite perpendiculaire à $\overrightarrow{i}$ passant par $A_0$.
\begin{enumerate}
\item  Calculer les longueurs $AM$ et $BN$.
\item On note $C_\Phi$ et $C_\Gamma$ les fonctions courbures associées à $f$ et à $g$. Simplifier
\[C_F(A)AM - C_g(B)BN\]
\end{enumerate}

\item Dans les quatre cas suivants, $\theta(t)=t$ pour tous les $t\in I$
\begin{eqnarray}
I=\R,\medskip  r(t)&=&t \\
I=\R,\medskip  r(t)&=&e^t\\
I=]-\frac{\pi}{2},\frac{\pi}{2}[, \medskip r(t)&=&\cos t \\
I=]-\frac{\pi}{2},\frac{\pi}{2}[, \medskip r(t)&=&\frac{1}{\cos t}
\end{eqnarray}
Les 8 figures présentent les courbes $\Phi$ et $\Gamma$ pour les 4 cas proposés. Associer chaque figure à un cas et à une courbe. Attention, l'intervalle que parcourt le paramètre sur les dessins est plus petit que celui de la définition.
% include au lieu de inclure
%\begin{figure}
%   \centering
%   \incluregraphics[scale=0.3]{Etranscurv_f5.jpg}
%   \caption{}
%\end{figure}
%\begin{figure}
%   \centering
%   \incluregraphics[scale=0.3]{Etranscurv_f8.jpg}
%      \caption{}
%\end{figure}
%\begin{figure}
%   \centering
%   \incluregraphics[scale=0.3]{Etranscurv_f1.jpg}
%      \caption{}
%\end{figure}
%\begin{figure}
%   \centering
%   \incluregraphics[scale=0.3]{Etranscurv_f7.jpg}
%      \caption{}
%\end{figure}
%\begin{figure}
%   \centering
%   \incluregraphics[scale=0.3]{Etranscurv_f2.jpg}
%   \caption{}
% \end{figure}
%\begin{figure}
%   \centering
%   \incluregraphics[scale=0.3]{Etranscurv_f6.jpg}
%   \caption{}
%\end{figure}
%\begin{figure}
%   \centering
%   \incluregraphics[scale=0.3]{Etranscurv_f4.jpg}
%   \caption{}
%\end{figure}
%\begin{figure}
%   \centering
%   \incluregraphics[scale=0.3]{Etranscurv_f3.jpg}
%   \caption{}
%\end{figure}

\item Dans cette question,
\[I=[0,\frac{\pi}{2}[, \medskip r(t) = \cos t , \medskip \theta=\tan t -t\]
\begin{enumerate}
\item Calculer $g(t)$ en séparant les parties réelles et imaginaires. On sera amené à effectuer un calcul de primitive.
\item Calculer l'angle que fait la tangente en $A=f(t)$ à $\Gamma$ et le rayon de courbure.
\item Comment s'exprime la courbe paramétrée en fonction de cet angle ?
\end{enumerate}  
\end{enumerate}
