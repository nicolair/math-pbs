%<dscrpt>Combinaison de fonctions trigonométriques réciproques.</dscrpt>
On se propose d'étudier la fonction $f$ définie par :
\begin{displaymath}
 f(x)= \arccos \frac{1-x^2}{1+x^2} + \arcsin \frac{2x}{1+x^2}
\end{displaymath}

\begin{enumerate}
 \item Montrer que $f$ est définie dans $\R$.
 \item \begin{enumerate}
 \item Pour $u\in[-1,1]$, préciser $\arccos (-u)$ et $\arcsin (-u)$ en fonction de $\arcsin u$ et de $\arccos u$ .
  \item Soit $x$ un réel non nul, calculer
\begin{displaymath}
 f(\frac{1}{x})+f(-x)
\end{displaymath}
\end{enumerate}
\item Si $x =\tan \theta$, exprimer 
\begin{align*}
\frac{1-x^2}{1+x^2} & & \frac{2x}{1+x^2} 
\end{align*}
en fonction de $\theta$.
\item En dégageant les cas pertinents pour $x$, simplifier $f(x)$.  Tracer le graphe de $f$.
\end{enumerate}

