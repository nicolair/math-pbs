%<dscrpt>Intégrales de 0 à + l'infini.</dscrpt>
Pour $\alpha $ r{\'e}el et $\lambda $ r{\'e}el strictement positif, on note $%
f_{\alpha ,\lambda }$ l'application d{\'e}finie par
\[
\forall t\in \left] 0,+\infty \right[ ,\quad f_{\alpha ,\lambda
}(t)=t^\alpha e^{-\lambda t}.
\]

\subsubsection*{PARTIE I}

\begin{enumerate}
\item
\begin{enumerate}
\item[a.]  D{\'e}terminer l'ensemble $A$ des couples $(\alpha ,\lambda )$
tels que $f_{\alpha ,\lambda }$ converge en 0 {\`a} droite.

\item[b.]  D{\'e}terminer l'ensemble $B$ des couples $(\alpha ,\lambda )$
tels que $f_{\alpha ,\lambda }$ soit int{\'e}grable sur $\left] 0,+\infty
\right[ $.
\end{enumerate}

\item  Pour tout nombre r{\'e}el $x$, montrer que les fonctions
\[
t\rightarrow \frac{e^{-t}}{\sqrt{t}}\cos xt,\quad t\rightarrow \frac{e^{-t}}{%
\sqrt{t}}\sin xt
\]
sont int{\'e}grables sur $\left] 0,+\infty \right[ $.

Dans toute la suite du probl{\`e}me, on posera $u(0)=a$ et
\[
u(x)=\int_{0}^{+\infty }\frac{e^{-t}}{\sqrt{t}}\cos xt\,dt,\quad
v(x)=\int_{0}^{+\infty }\frac{e^{-t}}{\sqrt{t}}\sin xt\,dt.
\]

\item  {\'E}tudier la parit{\'e} de chacune des deux fonctions $u$ et $v$.
Montrer que $a>0$.

\item  Soit $x$ et $x^{\prime }$ deux nombres r{\'e}els, justifier
l'in{\'e}galit{\'e}
\[
\left| u(x)-u(x^{\prime })\right| \leq \left| x^{\prime }-x\right|
\int_{0}^{+\infty }e^{-t}\sqrt{t}\,dt.
\]
En d{\'e}duire que $u$ est continue.
\end{enumerate}

\subsubsection*{PARTIE II}

On se propose de d{\'e}montrer que les fonctions $u$ et $v$ sont
ind{\'e}finiment d{\'e}rivables.

\begin{enumerate}
\item
\begin{enumerate}
\item[a.]  Soit $x$ un nombre r{\'e}el, montrer que la fonction $%
t\rightarrow e^{-t}\sqrt{t}\sin xt$ est int{\'e}grable sur $\left] 0,+\infty
\right[ $. On note
\[
i(x)=-\int_{0}^{+\infty }e^{-t}\sqrt{t}\sin xt\,dt\text{.}
\]
Pour $h$ r{\'e}el non nul, justifier l'in{\'e}galit{\'e}
\[
\left| \frac{u(x+h)-u(x)}{h}-i(x)\right| \leq \frac{\left| h\right| }{2}%
\int_{0}^{+\infty }t^{\frac{3}{2}}e^{-t}dt\text{.}
\]

\item[b.]  En d{\'e}duire que $u$ est d{\'e}rivable au point $x$ et que $%
u^{\prime }(x)=i(x)$. D{\'e}montrer un r{\'e}sultat analogue pour $v$.
\end{enumerate}

\item  {\'E}tablir, pour tout nombre r{\'e}el $x$%
\begin{eqnarray*}
u(x) &=&2\left( v^{\prime }(x)-xu^{\prime }(x)\right) \\
v(x) &=&-2\left( u^{\prime }(x)+xv^{\prime }(x)\right)
\end{eqnarray*}
En d{\'e}duire que les fonctions $u$ et $v$ sont ind{\'e}finiment
d{\'e}rivables.
\end{enumerate}

\subsubsection*{PARTIE III}

On admet provisoirement le r{\'e}sultat suivant.
\begin{equation}
\forall x>0,\quad v(x)>0  \tag{R}
\end{equation}

\begin{enumerate}
\item  On d{\'e}finit des applications $r$ et $\theta $ par
\[
\forall x\in \left] 0,+\infty \right[ :\quad r(x)=\sqrt{u(x)^{2}+v(x)^{2}}%
,\quad \theta (x)=\arctan \frac{u(x)}{v(x)}\text{.}
\]
Calculer $\frac{r^{\prime }}{r}$ et $\theta ^{\prime }$ puis $r$ et $\theta $%
.

\item  A l'aide des questions pr{\'e}c{\'e}dentes, expliciter les fonctions $%
u$ et $v$. On donnera des expressions ne faisant pas appara\^{\i }tre les
fonctions $\sin $ et $\cos $. On ne cherchera pas {\`a} d{\'e}terminer la
valeur de la constante $a=u(0)$.
\end{enumerate}

\subsubsection*{PARTIE IV}

On se propose d'{\'e}tablir le r{\'e}sultat (R) de la partie III. Dans les
deux premi{\`e}res questions, $\lambda $ d{\'e}signe un nombre r{\'e}el
strictement positif.

\begin{enumerate}
\item  Montrer que
\[
t\rightarrow \frac{e^{-\lambda t}}{\sqrt{t}}\sin t
\]
est int{\'e}grable sur $\left] 0,+\infty \right[ $.

\item  {\'E}tudier le sens de variation de la suite de terme g{\'e}n{\'e}ral
\[
I_{p}=\int_{0}^{2p\pi }\frac{e^{-\lambda t}}{\sqrt{t}}\sin t\,dt\text{.}
\]
En d{\'e}duire que
\[
I(\lambda )=\int_{0}^{+\infty }\frac{e^{-\lambda t}}{\sqrt{t}}\sin t\,dt>0%
\text{.}
\]

\item  Montrer que $v(x)>0$ pour tout $x>0$.\newpage
\end{enumerate}
