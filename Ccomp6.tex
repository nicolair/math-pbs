\begin{enumerate}
 \item Les résultats sont présentés dans un tableau
\begin{displaymath}
% use packages: array
\renewcommand{\arraystretch}{2}
\begin{array}{c|c|c}
n &  E_0(n,a) & E_1(n,a) \\ 
\hline n=2 & 
\begin{array}{c}
 z^2 + a^2=0 \\
 \{ia,-ia\}
\end{array}

&
\begin{array}{c}
 2za=0  \\ \{ 0 \}
\end{array}\\
\hline n=3 & 
\begin{array}{c}
3z^2a+a^3=0 \\ \{ \frac{ia}{\sqrt{3}}, -\frac{ia}{\sqrt{3}}\}
\end{array}

&
\begin{array}{c}
 z^3+3za^2=0 \\ \{0, i\sqrt{3}a, - i\sqrt{3}a\}
\end{array}


  \\ 
\hline n=4 &
\begin{array}{c}
 z^4+6z^2a^2+a^4=0 \\ \mathcal S
\end{array}
 & 
\begin{array}{c}
4z^3a+4za^3=0 \\ \{ 0,ia, -ia\}
\end{array}
\end{array}
\end{displaymath}
avec
\begin{displaymath}
 \mathcal S = \left\lbrace ia\sqrt{3+2\sqrt{2}}, -ia\sqrt{3+2\sqrt{2}}, ia\sqrt{3-2\sqrt{2}}, -ia\sqrt{3-2\sqrt{2}},  \right\rbrace 
\end{displaymath}
L'ensemble $\mathcal S $ s'obtient en prenant les racines carrées des solutions de l'équation 
\begin{displaymath}
z^2+6u^2z+a^4 
\end{displaymath}
d'inconnue $z$.
On peut remarquer que $(1+\sqrt 2)^2 = 3+2\sqrt{2}$ et donc remplacer $\sqrt{3+2\sqrt{2}}$ par $1+\sqrt 2$ dans l'expression de $\mathcal S$.
\item Dans cette question, le point important est que l'on puisse écrire $\lambda^n=\lambda^k \lambda^{n-k}$. Comme $\lambda\neq 0$, on peut multiplier par $\lambda^n$ donc $w$ est solution de $E_0(n,a)$ si et seulement si
\begin{displaymath}
 \lambda^n \sum_{k\in \mathcal P _n}\binom{n}{k}w^ka^{n-k}=0
\Leftrightarrow
\sum_{k\in \mathcal P _n}\binom{n}{k}(\lambda w)^k(\lambda a)^{n-k}=0
\end{displaymath}
C'est à dire lorsque $\lambda w$ est solution de $E_0(n,\lambda a)$. Le raisonnement est exactement le même pour $E_1(n,a)$.
\item 
\begin{enumerate}
 \item 
   \begin{itemize}
     \item Si $w=-1$ l'équation n'admet pas de solution.
     \item Si $w\neq -1$ l'équation admet une unique solution
\begin{displaymath}
 a\frac{w-1}{w+1}
\end{displaymath}
   \end{itemize}
\item En utilisant $e^{i\frac{\alpha}{2}}$ comme dans le cours, on obtient
\begin{displaymath}
 \frac{w-1}{w+1} = i\tan \frac{\alpha}{2}
\end{displaymath}
\item L'ensemble cherché est formé par les solutions des équations 
\begin{displaymath}
 \frac{a+z}{a-z} = u
\end{displaymath}
pour les $u$ de $\U_n$.
\begin{itemize}
 \item Lorsque $n$ est impair, $-1$ n'est pas dans $\U_n$ donc d'après a. et b. l'ensemble des solutions est
\begin{displaymath}
 \left\lbrace ia \tan \frac{k\pi}{n}, n\in \{0,\cdots n-1\}\right\rbrace 
\end{displaymath}
\item  Lorsque $n$ est pair, $-1$ est pas dans $\U_n$ donc d'après a. et b. il y a une solution de moins. L'ensemble des solutions est
\begin{displaymath}
 \left\lbrace ia \tan \frac{k\pi}{n}, n\in \{0,\cdots n-1\}-\{\frac{n}{2}\}\right\rbrace 
\end{displaymath}
\end{itemize}
\item Le développement (selon la formule du binôme) de $(-z+a)^n$ s'obtient à partir de celui de $(z+a)^n$ en affectant chaque coefficient de $z^k$ d'un $-1$ lorsque $k$ est impair. On en déduit que
\begin{displaymath}
 (z+a)^n - (-z+a)^n = 2 \sum_{k\in \mathcal I _n}\binom{n}{k}w^k a^{n-k}
\end{displaymath}
puis que les solutions de $E_1(n,a)$ sont les mêmes que celles de l'équation du c.

\end{enumerate}

\item \begin{enumerate}
 \item En développant selon la formle du binôme, on obtient
\begin{eqnarray*}
 \Re \left( (x+iy)^n\right)  &=& \sum_{k\in \mathcal P _n}\binom{n}{k}x^{n-k} (iy)^{k} \\
 i \Im \left( (x+iy)^n\right)  &=& \sum_{k\in \mathcal I _n}\binom{n}{k}x^{n-k} (iy)^{k}
\end{eqnarray*}
\item On peut appliquer ces formules avec $x=\cos \theta$ et $y=\sin \theta$, comme de plus
\begin{displaymath}
 (\cos \theta + i\sin\theta)^n = e^{in\theta}
\end{displaymath}
On en déduit
\begin{eqnarray*}
 \cos n\theta  &=& \sum_{k\in \mathcal P _n}\binom{n}{k}(\cos \theta)^{n-k} (i\sin \theta)^{k} \\
 i \sin n\theta  &=& \sum_{k\in \mathcal I _n}\binom{n}{k}(\cos \theta)^{n-k} (i\sin \theta)^{k}
\end{eqnarray*}
En mettant $(\cos \theta )^n$ en facteur, on obtient :
\begin{eqnarray*}
\frac{\cos n\theta}{(\cos \theta )^n}   &=& \sum_{k\in \mathcal P _n}\binom{n}{k}\left( \frac{i\sin \theta}{\cos \theta}\right) ^{k} \\
 i \frac{\sin n\theta}{(\cos \theta )^n}   &=& \sum_{k\in \mathcal I _n}\binom{n}{k}\left( \frac{i\sin \theta}{\cos \theta}\right) ^{k} 
\end{eqnarray*}
\item D'après la question précédente, les nombres $i\tan \theta$ pour lesquels $\sin n\theta = 0$ sont des solutions de $E_1(n,1)$ (avec le paramêtre $a=1$). D'après la question 2. les nombres $ai\tan \theta$ pour lesquels $\sin n\theta = 0$ sont des solutions de $E_1(n,a)$. On retrouve donc l'ensemble
\begin{displaymath}
\left\lbrace ia \tan \frac{k\pi}{n}, n\in \{0,\cdots n-1\}-\{\frac{n}{2}\}\right\rbrace 
\end{displaymath}

\item D'après la question 4.b., les nombres $i\tan \theta$ pour lesquels $\cos n\theta = 0$ sont des solutions de $E_0(n,1)$ (avec le paramêtre $a=1$). Comme on en trouve le bon nombre suivant la parité de $n$ ce sont toutes les solutions. D'après la question 2., les solutions de $E_0(n,a)$ sont les $ai\tan \theta$ pour lesquels $\cos n\theta = 0$.
\end{enumerate}

\end{enumerate}
