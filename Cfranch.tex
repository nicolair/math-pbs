\subsubsection*{I. Pr{\'e}ambule}

1.  Le point important est la pr{\'e}sence du $\forall \varepsilon
$ dans la d{\'e}finition d'un point de franchissemement. Soit
$t_{0}$ un point de franchissement vers le haut et $\varepsilon
_{0}$ tel que $f<u$ dans $] t_{0}-\varepsilon _{0},t_{0}[ $, $f>u$
dans $] t_{0},t_{0}+\varepsilon _{0}[ $.\newline Pour tout
$\varepsilon >0$, $] t_{0}-\varepsilon ,t_{0}[ \cap ]
t_{0}-\varepsilon ,t_{0}[ =] t_{0}-\min (\varepsilon ,\varepsilon
_{0}),t_{0}[ \neq \emptyset $ et si $t_{1}$ est dans cet
intervalle $f(t_{1})<u$. De m{\^e}me, $] t_{0},t_{0}+\varepsilon [
\cap ] t_{0},t_{0}+\varepsilon [ =] t_{0},t_{0}+\min (\varepsilon
,\varepsilon _{0})[ \neq \emptyset $ et si $t_{2}$ est dans cet
intervalle $f(t_{2})>u$.

2.  Soit $t_{0}$ un point de franchissement. D'apr{\`e}s la
d{\'e}finition d'un tel point, pour tout entier $n>0$, il existe
$t_{n}^{+}$ et $t_{n}^{-}$ dans $]
t_{0}-\frac{1}{n},t_{0}+\frac{1}{n}[ $ tels que $f(t_{n}^{-})<u$
et $f(t_{n}^{+})>u$. Le th{\'e}or{\`e}me des valeurs
interm{\'e}daires prouve l'existence d'un $x_{n}$ entre
$t_{n}^{-}$ et $t_{n}^{+}$ donc dans $]
t_{0}-\frac{1}{n},t_{0}+\frac{1}{n}[ $
et tel que $f(x_{n})=u$. D'apr{\`e}s le th{\'e}or{\`e}me d'encadrement $%
(x_{n})_{n\in \mathbf{N}}$ converge vers $t_{0}$ et $f(t_{0})=u$
par continuit{\'e}.

3.  Dans les cas a., b., c. le point $\frac{1}{2}$ est un point de
franchissement de $0$ vers le haut. Dans les cas d.et e. il ne
s'agit pas d'un point de franchissement.

\begin{floatingfigure}{5cm}
\includegraphics[height=5cm,width=5cm]{F1franch.pdf}
\end{floatingfigure}
4.  Je vais montrer la proposition contrapos{\'e}e. Supposons que
$t_{0}$
soit un point de franchissement et qu'il existe un $\alpha >0$ tel que dans $%
] t_{0}-\alpha ,t_{0}+\alpha [ $ $f$ prenne un nombre fini de
fois la valeur $u$ (elle la prend au moins en $t_{0})$ par exemple en $%
\{ t_{0},t_{1},\cdots ,t_{s}\} $. Soit $\beta $ la distance
minimale entre $t_{0}$ et un autre de ces points, alors $f-u$
garde un signe constant dans $] t_{0}-\beta ,t_{0}[ $ et dans $]
t_{0},t_{0}+\beta [ $. Les deux signes sont diff{\'e}rents car
$t_{0}$ est un point de franchissement, on se trouve alors
obligatoirement dans le cas d'un franchissement vers le haut ou
vers le bas.\newline Par contraposition, si $t_{0}$ est un point
de franchissement qui n'est ni vers le haut ni vers le bas, la
fonction $f$ prend une infinit{\'e} de fois la valeur $u$ dans un
intervalle ouvert quelconque autour de $t_{0}$. Un exemple de
fonction de ce type est
\[
f(x)=\{
\begin{array}{ccc}
| x-\frac{1}{2}| \sin \frac{1}{x-\frac{1}{2}} & \text{si} & x\neq
\frac{1}{2} \\
0 & \text{si} & x=\frac{1}{2}
\end{array}
.
\]
cette fonction est continue dans $[ 0,1] $ car $|
f(x)| \leq | x-\frac{1}{2}| $ assure la continuit{\'e} en $%
\frac{1}{2}.$ Ce point est un point de franchissement de $0$ qui
n'est ni vers le haut ni vers le bas, son graphe est :

5.  Soit $t_{0}$ un point qui n'est pas de franchissement et tel que $%
f(t_{0})=u$. Il existe alors un $\alpha >0$ tel que $f-u$ garde un
signe constant dans $] t_{0}-\alpha ,t_{0}+\alpha [ $. Si $f-u$
est positif $t_{0}$ est un minimum local, si $f-u$ est n{\'e}gatif
c'est un maximum local.

6.  Supposons $f(t_{1})>u$ et $f(t_{2})<u$ avec $t_{1}<t_{2}$ pour
fixer les id{\'e}es (l'autre cas est analogue). Consid{\'e}rons
$A=\{ t\in [ t_{1},t_{2}] \text{ tels que }f(t)>u\} $. Cet
ensemble est non vide ( $t_{1}\in A$), major{\'e} par $t_{2}$ il
admet une borne sup{\'e}rieure que je note $t_{0}$. Montrons que
$t_{0}$ est un point de franchissement.

\begin{itemize}
\item  Il existe une suite d'{\'e}l{\'e}ments de $A$ qui converge vers $%
t_{0}.$ On en d{\'e}duit, par passage {\`a} la limite dans une
in{\'e}galit{\'e} et par continuit{\'e} que $f(t_{0})\geq u$. Ceci assure $%
t_{0}<t_{2}$

\item  $A\cap ] t_{0},t_{2}] $ est vide car $t_{0}=\sup A$ donc $%
\forall t\in ] t_{0},t_{2}] $, $f(t)\leq u$ en fait on a m{\^e}me
$f(t)<u$ sinon $f$ serait constante sur $] t_{0},t] $. Ceci prouve
que pour tout $\varepsilon >0$, il existe un $\theta _{1}\in ]
t_{0},t_{0}+\varepsilon [ $ tel que $f(\theta _{1})<u$.

\item  Pour tout $\varepsilon >0$, $t_{0}-\varepsilon $ n'est pas un
majorant de $A,$ il existe donc un $\theta _{2}\in ]
t_{0}-\varepsilon ,t_{0}[ \cap A$ donc tel que $f(\theta _{2})>u$.
\end{itemize}

\subsubsection*{II. Polygonation}

1.  Une fonction affine est continue, la restriction de $f_{n}$
sur chaque segment est donc continue. Ceci montre que $f_{n}$ est
continue sur chaque intrevalle ouvert $]
\frac{k}{2^{n}},\frac{k+1}{2^{n}}[ $. De plus la limite {\`a}
droite en $\frac{k}{2^{n}}$ est {\'e}gale {\`a} sa limite {\`a}
droite et {\`a} $f(\frac{k}{2^{n}})$ ce qui d{\'e}montre la
continuit{\'e} aux points de la subdivision. Elle n'est jamais
constante de
valeur $u$ car $f$ ne prend pas la valeur $u$ aux points de la forme $\frac{k%
}{2^{n}}$.

2.  Le segment $[ 0,1] $ se d{\'e}compose en un nombre fini
d'intervalles sur lesquels $f_{n}$ est strictement monotone ou
constante d'une valeur $\neq u$. Sur chacun de ces intervalles,
$f_{n}$ prend au plus
une fois la valeur $u$. Chacun de ces points est de franchissement, $%
F_{u}^{n}$ est donc le nombre d'intervalles o{\`u} $f_{n}$ prend la
valeur $u $.\newline Examinons un tel intervalle : aux extr{\'e}mit{\'e}s,
les valeurs de $f_{n}$ et de $f$ sont {\'e}gales et de part et d'autre
de $u$. D'apr{\`e}s la question I.6., $f$ admet au moins un point de
franchissement sur un tel intervalle donc $F_{u}^{n}\leq F_{u}$.

3.  La $n+1$ i{\`e}me subdivision s'obtient {\`a} partir de la $n$
i{\`e}me en divisant chaque intervalle en deux. Consid{\'e}rons un
intervalle de la subdivision d'ordre $n$ sur lequel $f_{n}$ prend la valeur $%
u$; les valeurs de $f$ aux extr{\'e}mit{\'e}s sont de part et d'autre de
$u$ donc il en est de m{\^e}me entre une extr{\'e}mit{\'e} et la valeur de
$f_{n} $ au milieu. Ceci montre que $f_{n+1}-u$ s'annule
exactement une fois sur un intervalle o{\`u} $f_{n}-u$ s'annule. Comme
$f_{n+1}-u$ peut s'annuler ailleurs, $F_{u}^{n}\leq F_{u}^{n+1}$.

4.a.  Num{\'e}rotons par ordre croissant les points de franchissement soit $%
t_{1},\cdots ,t_{F_{u}}$. Posons $\alpha =\min \{
t_{1},t_{2-}t_{1},\cdots ,t_{F_{u}}-t_{F_{u}-1},1-t_{F_{u}}\} $ et $%
I_{k}=] t_{k}-\alpha ,t_{k}+\alpha [ $. Par d{\'e}finition de $%
\alpha $, $I_{k}$ ne contient pas d'autre point de franchissement que $t_{k}$%
.

b.  Soit $i\in \{ 1,\cdots ,F_{u}\} $ $I_{i}$ l'intervalle
d{\'e}fini dans la question pr{\'e}c{\'e}dente. Comme $t_{i}$ est
un point de franchissement, il existe $\theta _{1}$ et $\theta
_{2}$ dans $I_{i}$ tels que $f(\theta _{1})>u$ et $f(\theta
_{2})<u$. Comme $f$ est continue en $\theta _{1}$, il existe
$J_{i}$ assez petit pour {\^e}tre inclus dans $I_{i} $ et pour que
$f-u$ reste strictement positive dans $J_{i}.$De m{\^e}me,
l'existence de $K_{i}$ est assur{\'e}e par la continuit{\'e} de $f$ en $%
\theta _{2}$.

c.  Lorsque $2^{-n}$ est plus petit que la plus petite longueur
des intervalles $J_{i}$ et $K_{i}$ de la question
pr{\'e}c{\'e}dente, chaque point de franchissement se trouve dans
un seul des intervalles de la subdivision associ{\'e}e {\`a}
$f_{n}$, chacun de ces intervalles contient
exactement un des points de franchissement donc $F_{u}^{n}=F_{u}$. La suite $%
(F_{u}^{n})_{n\in \mathbf{N}}$ est stationnaire de valeur $F_{u}$.

5  Donnons nous $K$ (entier arbitraire) points de franchissements $%
t_{1},t_{2},\cdots ,t_{K}$. Soit $\delta $ un nombre inf{\'e}rieur {\`a}
la
plus petite distance entre deux de ces points; il existe alors $y_{i}$, $%
z_{i}$ dans $] t_{i}-\delta ,t_{i}+\delta [ $ tels que $f(y_{i})>u
$, $f(z_{i})<u$. On suppose $y_{i}<z_{i}$ pour fixer les
id{\'e}es. Je me propose de montrer que, pour $n$ assez grand,
$f_{n}$ admet un point de
franchissement de $u$ entre $y_{i}$ et $z_{i}$. Ce qui entra\^{i}nera $%
F_{u}^{n}\geq K$ et donc $(F_{u}^{n})_{n\in \mathbf{N}}arrow +\infty $%
\newline
La d{\'e}finition de $\delta $ montre que les $y_{i}$ et $z_{i}$ sont
deux
{\`a} deux distincts. Soit $\varepsilon >0$ le plus petit des nombres $%
f(y_{i})-u$ et $u-f(z_{i})$. A cause de l'\emph{uniforme
continuit{\'e}}, il existe un $\alpha >0$ tel que $| f(x)-f(y)|
<\varepsilon $ d{\`e}s que $| x-y| <\alpha $.\newline Lorsque
$\frac{1}{2^{n}}<\alpha $ et que $a$ et $b$ sont deux entiers tels
que
\[
\frac{a-1}{2^{n}}\leq y_{i}<\frac{a}{2^{n}}<\cdots
<\frac{b}{2^{n}}\leq z_{i}<\frac{b+1}{2^{n}}
\]
on a aussi $f(\frac{a}{2^{n}})>u$ et $f(\frac{b}{2^{n}})<u$. Il
existe alors
un entier $k$ entre $a$ et $b$ tel que $f(\frac{k}{2^{n}})>u$ et $f(\frac{k+1%
}{2^{n}})<u$, ce qui prouve que $f_{n}$ admet un point de
franchissement dans $] \frac{k}{2^{n}},\frac{k+1}{2^{n}}[ $.

6.  La fonction $(x-\frac{1}{2})| \sin \frac{1}{(x-\frac{1}{2})}| $ admet en $\frac{1%
}{2}$ un point de franchissement qui n'est ni vers le haut ni vers
le bas. C'est le seul point de franchissement de $0$, tous les
points o{\`u} $f$ prend la valeur 0 sont des points de
tangence.\newline
Montrons que $F_{u}$ infini entra\^{i}ne que $f$
admet un point de franchissement de $u$ qui n'est ni vers le haut
ni vers le bas. On raisonne par dichotomie en s'inspirant de la
d{\'e}monstration du th{\'e}or{\`e}me de
Bolzano-Weirstrass.\newline
\begin{floatingfigure}{5cm}
\includegraphics[height=5cm,width=5cm]{F2franch.pdf}
\end{floatingfigure}
Coupons $I_{0}=[ 0,1] $ en deux, une des deux moiti{\'e}s doit
contenir une infinit{\'e} de points de franchissement, on
l'appelle $I_{1}$. De m{\^e}me une des deux moiti{\'e}s (soit
$I_{2}$) de $I_{1}$ contiendra une infinit{\'e} de points de
franchissements. On construit ainsi une suite de segments
embo\^{i}t{\'e}s dont les extr{\'e}mit{\'e}s $x_{n}$ et $y_{n}$
forment des suites adjacentes qui convergent vers un $t_{0}\in [
0,1] $.\newline
 Soit
$\theta _{n}$ un point de franchissement dans $[ x_{n},y_{n}] $
alors $f(\theta _{n})=u.$ Comme $(\theta _{n})_{n\in
\mathbf{N}}arrow t_{0}$, $f(t_{0})=u$ par continuit{\'e}. Ceci
prouve aussi que $t_{0}\in ] x_{n},y_{n}[ $ car $x_{n}$ et $y_{n}$
sont de la forme $\frac{k}{2^{n}}$ et que $f$ ne prend pas la
valeur $u$ en ces points.\newline Pour tout $\varepsilon >0$, il
existe un $n$ tel que $t_{0}-\varepsilon
<x_{n}<t_{0}<y_{n}<t_{0}+\varepsilon $; un des deux intervalles $]
x_{n},t_{0}[ $ ou $] t_{0},y_{n}[ $ contient une infinit{\'e} de
points de franchissements, par exemple $] x_{n},t_{0}[ .$ Soit
$\theta $ l'un d'entre eux.\newline
Pour $\alpha =\min (t_{0}-\theta ,\theta -x_{n})$, il existe $t_{1}$ et $%
t_{2}$ dans $] \theta -\alpha ,\theta +\alpha [ \subset ]
x_{n},t_{0}[ \subset ] t_{0}-\varepsilon ,t_{0}+\varepsilon [ $
tel que $f(t_{1})>u$, $f(t_{2})<u$. Ceci prouve que $t_{0}$ est un
point de franchissement et qu'il n'est ni vers le haut ni vers le
bas.
