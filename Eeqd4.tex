%<dscrpt>Equation differentielle linéaire</dscrpt>
Dans tout l'exercice, les solutions cherchées sont des fonctions à valeurs réelles. Cela n'interdit pas la considération de fonctions à valeurs complexes comme intermédiaire de calcul.

On étudie l'équation fonctionnelle
\begin{equation}
y''(x)+y(-x)=x+\cos x
\end{equation}

\begin{enumerate}
\item Soit $y_1$ et $y_2$ deux solutions de l'équation
\[y''(x)+y(-x)=0\]
et $\lambda$ un réel quelconque; $y_1+y_2$ et $\lambda y_1$ sont ils encore solutions de la même équation ?

\item Résoudre les équations suivantes en précisant pour chacune l'ensemble des solutions paires et impaires.

\begin{eqnarray}
y''(x)+y(x) = \cos x \\
y''(x)-y(x) = x
\end{eqnarray}

\item \emph{question de cours}\newline
Soit $f$ une fonction définie dans $\R$, montrer qu'il existe un unique couple de fonctions $(u,v)$ telles que $u$ soit paire, $v$ soit impaire et $f=u+v$. On prendra soin de rédiger séparément les argumentations assurant l'existence et l'unicité. On dit que $u$ est la partie paire et $v$ la partie impaire de $f$.

\item Soit $f$ une solution de (1), $u$ sa partie paire et $v$ sa partie impaire. Former une équation différentielle dont $u$ est solution, former une équation différentielle dont $v$ est solution.

\item Préciser l'ensemble des solutions de (1).
\end{enumerate}
