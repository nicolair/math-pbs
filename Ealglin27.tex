%<dscrpt>Plan stable pour le crochet de deux endomorphismes.</dscrpt>
Soit $E$ un $\R$ espace vectoriel de dimension finie. Pour un entier $n$ et un endomorphisme $f$ de $E$, on désigne par $f^{n}$ la composée de $f$ par lui même $n$ fois.\newline
On introduit aussi le \emph{crochet} de deux endomorphismes:
\begin{displaymath}
\forall (f,g)\in \mathcal{L}(E)^2,\;  [f,g] = f\circ g - g \circ f
\end{displaymath}
On s'intéresse aux familles libres $(f,g)$ d'endomorphismes de $E$ vérifiant 
\begin{displaymath}
  [f,g] \in \Vect(f,g)
\end{displaymath}
On considère plus particulièrement les cas où au moins l'un des deux endomorphismes est un projecteur. On notera $\mathcal{P}(E)$ l'ensemble des projecteurs de $E$.
\subsection*{Partie I. Trace, projecteurs, crochet.}
\begin{enumerate}
  \item Questions de cours.
\begin{enumerate}
  \item Justifier la possibilité de définir la trace d'un endomorphisme.
  \item Rappeler la définition d'un projecteur, d'une projection et la relation entre les deux. On ne demande pas de démonstration.
  \item Soit $p$ un projecteur. Montrer que 
\begin{displaymath}
  \tr (p) = \rg(p),\hspace{0.5cm} \Im(p) = \ker(p-\Id_E),\hspace{0.5cm} \ker(p) = \Im(p-\Id_E)
\end{displaymath}
\end{enumerate}

\item Propriétés des projecteurs. Soit $p$ et $q$ deux projecteurs.
\begin{enumerate}
  \item Montrer que 
\begin{displaymath}
  p \circ q = q \Leftrightarrow \Im q \subset \Im p, \hspace{1cm}
  p \circ q = p \Leftrightarrow \ker q \subset \ker p 
\end{displaymath}
  \item Soit $\mathcal{R}$, la relation binaire définie sur $\mathcal{P}(E)$ par :
\begin{displaymath}
\forall (p,q) \in \mathcal{P}(E)^{2}, \hspace{0.5cm} p \, \mathcal{R} \, q \Leftrightarrow  p \circ q=q \circ p=p  
\end{displaymath}
Comment caractériser $p \, \mathcal{R} \, q$ par des relations entre noyaux et images ? Montrer que $\mathcal{R}$ est une relation d'ordre sur $\mathcal{P}(E)$. 
  \item Soit $p$ un projecteur et $\lambda\in \R\setminus \{0,1\}$. Montrer que $p-\lambda \Id_E$ est un isomorphisme.
\end{enumerate}

  \item Propriétés du crochet.
\begin{enumerate}
  \item Montrer que: $\forall (f,g)\in \mathcal{L}(E)^2,\; \tr([f,g]) = 0,\; [g,f] = - [f,g]$. Montrer que l'application, pour $f\in \mathcal{L}(E)$ fixé,
\begin{displaymath}
  \left\lbrace 
\begin{aligned}
  \mathcal{L}(E) &\rightarrow \mathcal{L}(E)\\ g &\mapsto [f,g]
\end{aligned}
\right. 
\end{displaymath}
est linéaire.

  \item Le crochet n'est pas associatif mais il vérifie une autre relation. Calculer
\begin{displaymath}
  \left[ f, [g,h]\right] + \left[ g, [h,f]\right] + \left[ h, [f,g]\right] 
\end{displaymath}
pour tout $(f,g,h)\in \mathcal{L}(E)^3$ (identité de \emph{Jacobi}).
  \item Soit $(f,g)$ une famille libre d'endomorphismes de $E$ vérifiant $[f,g] \in \Vect(f,g)$. On note $V = \Vect(f,g)$, montrer que $V$ est stable pour l'opération crochet.
\end{enumerate}
\end{enumerate}

\subsection*{Partie II. Un exemple de projecteur.}
Dans cette partie, $E=\R^{4}$. La base canonique est notée $\mathcal{E}= (e_{1},e_{2},e_{3},e_{4})$.\newline
On définit $p_{0} \in \mathcal{L}(E)$ par
\begin{displaymath}
  \Mat_{\mathcal{E}}(p_0) = A \text{ avec }
A=-\frac{1}{3}
\begin{pmatrix}
-2 & -1 & 2 & 0 \\
0 & -3 & 0 & 0 \\
1 & -1 & -1 & 0 \\
0 & 0 & 0 & 0 
\end{pmatrix}
\end{displaymath}
\begin{enumerate}
  \item Montrer que $p_{0}$ est un projecteur.
\item \begin{enumerate}
\item Montrer que $(e_{1}+e_{3},e_{4})$ est une base de $\ker p_{0}$ et $(-2e_{1}+e_{3},e_{1}+3e_{2}+e_{3})$ est une base de $\Im p_{0}$
\item Montrer que $(-2e_{1}+e_{3},e_{1}+3e_{2}+e_{3},e_{1}+e_{3},e_{4})$ est une base de $E$. Exprimer la matrice de $p_{0}$ dans cette base.
\end{enumerate}
\item Soit $\mathcal{B}$ une base de $E$ telle que, en adoptant la notation de matrices par blocs et $I_{r}$ désignant la matrice carrée identité d'ordre $r$, 
\begin{displaymath}
\Mat_{\mathcal{B}}(p_{0}) =  
\begin{pmatrix}
I_{r} & 0 \\
 0    & 0 
\end{pmatrix}
\text{ avec } r = \rg p_{0}
\end{displaymath}
Déterminer, dans $\mathcal{B}$, la forme de la matrice d'un projecteur $q$ vérifiant $p_{0}\,\mathcal{R}\,q$.

\end{enumerate}

\subsection*{Partie III. Plans stables pour le crochet.}
Dans cette partie, $(f,g)$ est une famille libre d'endomorphismes de $E$ tels que 
\begin{displaymath}
[f,g]\neq 0_{\mathcal{L}(E)} \text{ et }  [f,g]\in V \text{ avec } V = \Vect(f,g)
\end{displaymath}

\begin{enumerate}
\item On suppose que $[f,g]\in \Vect(f)$. Plus précisément, $\exists \alpha \in \R^*\text{ tel que } [f,g] = \alpha f$.
\begin{enumerate}
\item Montrer que $\forall k\in \N^{*}, \; [f^k,g]  = \alpha k f^{k}$.
\item Montrer que, 
\begin{displaymath}
\forall k\in \N,\hspace{0.5cm} f^{k}\neq 0_{\mathcal{L}(E)}
\Rightarrow
(\mathrm{id},f,f^{2},\cdots,f^{k}) \text{ libre dans } \mathcal{L}(E)
\end{displaymath}
En déduire l'existence d'un entier $n$ tel que $f^{n}=0_{\mathcal{L}(E)}$. On dira que $f$ est \emph{nilpotent}.
\end{enumerate}

\item Montrer que $[f,g]\in \Vect(g)$ entraîne $g$ nilpotent.

\item On suppose que $f$ et $g$ sont deux projecteurs. Comme $[f,g]\in V$, il existe des réels $\alpha$, $\beta$ tels que 
\begin{displaymath}
  [f,g] = \alpha f + \beta g
\end{displaymath}

\begin{enumerate}
\item Montrer que $\alpha$ et $\beta$ sont non  nuls.
\item Montrer que 
\begin{displaymath}
  [f,g] = \alpha \left( f\circ g + g\circ f\right)  + 2\beta g
\end{displaymath}
En déduire 
\begin{displaymath}
\alpha(1-\alpha)\,f = 2\alpha g \circ f + \beta(1+\alpha)g \;\text{ et }\;
\beta(1-\alpha)\,g = -2\alpha f \circ g + \alpha(1+\alpha)f
\end{displaymath}
\item Montrer que $\alpha = 1$ entraîne $f\circ g = f$ et $g\circ f = g$. 
\item Montrer que $\alpha \neq 1$ entraîne $g\circ f = f$ et $f\circ g = g$.
\item Décrire, en précisant les relations entre les noyaux et les images, les projecteurs $f$ et $g$ satisfaisant aux conditions imposées dans cette question. Que vaut alors leur crochet?
\end{enumerate}
\item Soit $p_{0}$ le projecteur de la partie II. et $\mathcal{B}$ la base de la question II.3.. 
\begin{enumerate}
  \item Préciser la forme de la matrice dans $\mathcal{B}$ d'un projecteur $g$ vérifiant $[p_0,g] = -p_{0} + g$.
  \item Préciser la forme de la matrice dans $\mathcal{B}$ d'un projecteur $g$ vérifiant $[p_0,g] = p_{0} - g$.
\end{enumerate}

\end{enumerate}

