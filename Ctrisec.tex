\subsubsection*{Partie I}
\begin{enumerate}
\item Dans la relation entre $z_1$, $z_2$ et $z_3$, on utilise successivement :
\begin{eqnarray*}
j=-1-j^2 ,& j^2=-1-j ,& 1=-j-j^2
\end{eqnarray*}
On en déduit 
\begin{eqnarray*}
z_1-z_2+j^2(z_3-z_2)=0 ,& \frac{z_1-z_2}{z_3-z_2}=-j^2=e^{i\frac{\pi}{3}}\\
z_1-z_3+j(z_2-z_3)=0 ,& \frac{z_2-z_3}{z_1-z_3}=-j^{-1}=e^{i\frac{\pi}{3}}\\
j(z_2-z_1)+j^2(z_3-z_1)=0 ,& \frac{z_3-z_1}{z_2-z_1}=-j^{-1}=e^{i\frac{\pi}{3}}
\end{eqnarray*}
On en déduit que le triangle $(Z_1,Z_2,Z_3)$ est équilatéral puisque ses trois angles sont égaux à $\frac{\pi}{3}$.
\item Les nombres $\alpha$, $\beta$, $\gamma$ sont dans $]0,\frac{\pi}{3}[$. Donc $2(\alpha+\beta)\in ]0,\frac{4\pi}{3}[$. Comme $uv=e^{2i(\alpha+\beta)}$ et que $\frac{4\pi}{3}<2\pi$, on obtient bien $uv\neq1$. De même pour $uw$ et $vw$.\newline
De manière analogue :
\[uvw=e^{2i(\alpha+\beta+\gamma)}\]
avec $3(\alpha+\beta+\gamma)\equiv \pi \; (2\pi)$ donc $\alpha+\beta+\gamma=\frac{\pi}{3}$ et $uvw=j$.
\item Après calculs, on trouve
\begin{eqnarray*}
\frac{u(1-v)}{1-uv} &=& \frac{\sin \beta}{\sin(\alpha +\beta)}e^{i\alpha}\\
\frac{u-1)}{1-uv} &=& -\frac{\sin \alpha}{\sin(\alpha +\beta)}e^{-i\beta}
\end{eqnarray*}
\item On multiplie respectivement les lignes définissant $p$, $q$ et $r$ par $1$, $j$, et $j^2$ et on les ajoute. On obtient:
\begin{eqnarray*}
(1-uv)(1-vw)(1-wu)(p+jq+j^2r) =\\
(1-uv)(1-wu)\left[ \underbrace{(1-v)}_{}b +v(1-w)c\right]\\
+(1-uv)(1-vw)\left[ j(1-w)c + jw(1-u)a\right]\\
+(1-vw)(1-wu)\left[ j^2(1-u)a + j^2u\underbrace{(1-v)}_{}b\right]
\end{eqnarray*}
On regroupe deux par deux les 6 termes de cette somme. Par exemple ceux contenant $(1-v)$:
\begin{multline*}
b\left[(1-uv)(1-wu)(1-v)+(1-vw)(1-wu)j^2u(1-v)\right] \\
=
b(1-wu)(1-v)\left[ \underbrace{1-uv+(1-vw)j^2u}_{=1-uv+j^2u-uvwj^2=u(j^2-v)} \right] \\
 =
bu(1-wu)(1-v)u(j^2-v)
 =
bu(1-\frac{j}{v})(1-v)u(j^2-v)\\
 =
b\frac{u}{v}(v-j)(1-v)(j^2-v)
 =
bj^2u^2w(v^3-1)
 =
b\frac{u}{v}(v^3-1)
\end{multline*}
car $uvw=j$. On trouve de même :
\begin{align*}
a\left[(1-uv)(1-vw)jw(1-u)+(1-vw)(1-wu)j^2(1-u) \right] \\
=  a\frac{w}{u}j^2(u^3-1) \\
c\left[(1-uv)(1-wu)v(1-w)+(1-uv)(1-vw)j(1-w) \right] \\
= c\frac{v}{w}j(w^3-1)
\end{align*}
On en déduit la formule demandée
\[
E = \frac{u}{v}(v^3-1)b + \frac{w}{u}j^2(u^3-1)a + \frac{v}{w}j(w^3-1)c \\ 
\]
\end{enumerate}

\subsubsection*{Partie II}
\begin{figure}[ht]
	\centering
	\input{Ctrisec_1.pdf_t}
	\caption{$R$ comme intersection de deux trisectrices}
        \label{fig:Ctrisec_1}
\end{figure}
\begin{figure}[ht]
	\centering
	\input{Ctrisec_2.pdf_t}
	\caption{Triangle $(PQR)$}
        \label{fig:Ctrisec_2}
\end{figure}

\begin{enumerate}
	\item Les transformations $R_a$, $R_b$, $R_c$ sont des rotations. Leurs centres sont respectivement les points d'affixes $a$, $b$, $c$. Leurs angles sont respectivement $\alpha$, $\beta$, $\gamma$.
	\item En composant les rotations :
\[R_a \circ \R_b(z)=uv(z-b)+u(b-a)+a \]
donc $R_a \circ \R_b(r)=r$ si et seulement si
\[(1-uv)z=u(1-v)b+(1-u)a\]
Ceci prouve l'existence et l'unicité du point fixe.
\item Comme l'énoncé nous l'indique, soustrayons $(1-uv)a$ de chaque côté de la relation précédente. On obtient :
\[(1-uv)(r-a)=-u(1-v)a+u(1-v)b\]
d'où
\[\frac{r-a}{b-a}=\frac{u(1-v)}{1-uv}=\frac{\sin \beta}{\sin (\alpha + \beta)}e^{i\alpha}\]
On en déduit
\[(\overrightarrow{AB},\overrightarrow{AR})=\alpha\]
On obtient de manière analogue
\[\frac{r-b}{a-b}=\frac{1-u}{1-uv}=\frac{\sin \alpha}{\sin (\alpha + \beta)}e^{-i\beta}\] \[(\overrightarrow{BA},\overrightarrow{BR})=-\beta\]
\item De même le point fixe $P$ de $R_b \circ R_c$ est l'intersection de deux trisectrices issues de $B$ et $C$, le point fixe $Q$ de $R_c \circ R_a$ est l'intersection de deux trisectrices issues de $C$ et $A$. On en déduit la triangle $(PQR)$ sur la figure.
\end{enumerate}

\subsubsection*{Partie III}
\begin{enumerate}
 \item \begin{enumerate}
 \item $R_c^3$ est la rotation de centre $c$ et d'angle $6\gamma=2(\overrightarrow{CA},\overrightarrow{CB})$. On en déduit que $a^\prime=R_c^3(a)$ est le symétrique de $A$ par rapport à $(BC)$.
\item $R_a^3 \circ R_b^3 \circ R_c^3$  est la rotation de centre $C$ et d'angle $6(\alpha +\beta +\gamma)=2\pi$. C'est donc une translation ou l'identité. En fait c'est l'identité car le point $A$ est fixe.
\[R_a^3 \circ R_b^3 \circ R_c^3(a)=R_a^3 \circ R_b^3 (a')=R_a^3(a)=a\]
\end{enumerate}
\item On déduit de la question précédente que $R_a^3 \circ R_b^3 \circ R_c^3(0)=0$. Ceci s'écrit
\[(1-u^3)a+u^3(1-v^3)b+u^3v^3(1-w^3)c=0\]
\item Démonstration du théorème de Morley. On a vu que les intersections $P$, $Q$, $R$ des trisectrices sont aussi des points fixes pour des composées de rotations. Ils vérifient donc les trois relations de la question I.3. On a montré alors
\begin{eqnarray*}
E &=& (1-uv)(1-vw)(1-wu)(p+jq+j^2r)\\
&=&\frac{w}{u}j^2(u^3-1)a+\frac{u}{v}(v^3-1)b+\frac{v}{w}j(w^3-1)c
\end{eqnarray*}
En utilisant systématiquement $j=uvw$, on chasse les $j$ de l'expression précédente, on tombe alors sur la quantité nulle de la question précédente. On en déduit que 
\[p+jq+j^2r=0\]
c'est à dire que $PQR$ est équilatéral.
\end{enumerate}

\subsubsection*{Annexe. Comment faire les figures du théorème de Morley ?}
\begin{figure}[ht]
 \centering
 \input{Ctrisec_3.pdf_t}
 \caption{Tracé des figures pour le théorème de Morley}
 \label{fig:Ctrisec_3}
\end{figure}

Il n'est pas possible de trisecter un angle à la règle et au compas. Les figures pour illustrer le théorème de Morley ne sont donc pas très faciles à tracer.\\
On peut utiliser un logiciel de géométrie plane.
\begin{itemize}
 \item On fixe les points $A$ et $P$ et la droite qui porte $B$.
\item Cela détermine l'angle $(PAB)$. On en déduit deux autres angles égaux à celui là par symétrie axiale. On obtient donc en $A$ un angle trisecté en trois angles égaux.
\item On choisit un point $B$ arbitraire sur sa droite.
\item Cela détermine l'angle $(ABP)$, on en déduit deux autres angles égaux par symétrie. En $B$ on a donc un angle trisecté en trois angles égaux.
\item Les points $C$ et $Q$ sont alors déterminés. On reproduit par symétrie l'angle $(BCQ)$. On dispose donc d'un angle trisecté en $C$. sa droite extérieure est en pointillé sur la figure.
\item En déplaçant le point $B$ sur sa droite, on peut faire coïncider cette droite en pointillé avec $(AC)$. On se trouve alors dans la configuration du théorème.
\end{itemize}
 