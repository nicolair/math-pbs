%<dscrpt> Géométrie plane : cercles tangents.</dscrpt>
Dans un plan report{\'e} {\`a} un rep{\`e}re orthonorm{\'e}, on consid{\`e}re trois
cercles $\mathcal{C}$, $\mathcal{C}'$, $\mathcal{C}''$ d{\'e}finis par
les condition suivantes :
\begin{itemize}
  \item $\mathcal{C}$ est centr{\'e} en $(0,1)$ et passe par
  l'origine.
  \item $\mathcal{C}'$ est tangent ext{\'e}rieurement {\`a} $\mathcal{C}$,
  tangent {\`a} l'axe d'{\'e}quation $y=0$ et centr{\'e} en $A$ d'abscisse
  $a>0$
  \item $\mathcal{C}''$ est tangent ext{\'e}rieurement {\`a} $\mathcal{C}$
  et $\mathcal{C}'$,tangent {\`a} l'axe d'{\'e}quation $y=0$ et centr{\'e} en $B$ d'abscisse
  $b<a$.
\end{itemize}
Montrer qu'il existe une fonction $\varphi$ d'expression tr{\`e}s
simple telle que $b=\varphi (a)$. Calculer les rayons de
$\mathcal{C}'$ et $\mathcal{C}''$ en fonction de $a$. (faire une
figure explicative)
