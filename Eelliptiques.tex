%<dscrpt>Introduction aux fonctions elliptiques.</dscrpt>
\newcommand{\sn}{\operatorname{sn}}
\newcommand{\cn}{\operatorname{cn}}
\newcommand{\dn}{\operatorname{dn}}

Soit $k\in ]0, 1[$. Pour tout $x\in \R$, on note $F(x) = \displaystyle{\int_{0}^{x}\frac{dt}{\sqrt{1-k^{2}\sin^{2}(t)}}}$.

\begin{enumerate}
 \item Montrer que la fonction $F$ est bien définie, impaire et dérivable sur $\R$ et exprimer sa dérivée.
 \item Posons $K = F(\pi / 2)$ et $T = F(\pi)$. Montrer que $T = 2K$.
 \item Montrer que pour tout $x\in \R$, $F(x + \pi) = F(x) + T$.
 \item Montrer que $F(x) \geq x$ pour tout $x\geq0$. En déduire que $F$ réalise une bijection strictement croissante de $\R$ vers $\R$.
 
Notons $A:\R \to \R$ la bijection réciproque de $F$. Pour tout $x\in \R$, on notera également:
\[ \sn (x) = \sin(A(x)),\qquad \cn (x) = \cos(A(x)),\qquad \dn (x) = \sqrt{1-k^{2}\sn(x)^{2}}.\] 


\item Montrer que la fonction $\sn$ est $2T$-périodique.

\item Montrer que les fonctions $\sn$ et $\cn$ sont dérivables et que pour tout $x\in \R$:
\[ \sn'(x) = \cn(x)\dn(x) \ \text{ et } \ \cn'(x) = -\sn(x) \dn(x)\]

\item Soient $\omega \in \R_{+}$ et $\theta_{0}\in ]0, \pi / 2[$. Posons $k = \sin(\theta_{0}/2)$ et pour tout $x\in \R$:
\[ \theta(x) = 2\arcsin(k\sn(\omega x + K)).\]
\begin{enumerate}
 \item Montrer que $\theta$ vérifie:
 \[ \forall x\in \R,\ \theta''(x) + \omega^{2}\sin (\theta(x)) = 0,\ \theta(0) = \theta_{0},\ \theta'(0) = 0.\]
 \item Montrer que $\theta$ est périodique et déterminer une période.
\end{enumerate}
\end{enumerate}
