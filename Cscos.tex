Dans un triangle équilatéral chaque angle au sommet vaut $\frac{\pi}{3}$ de cosinus $\frac{1}{2}$. La somme des cosinus d'un triangle équilatéral vaut donc $\frac{3}{2}$.
\begin{enumerate}
\item L'examen de la figure et les propriétés élémentaires des angles orientés de vecteurs suffisent à justifier les relations
\begin{eqnarray*}
\widehat{(\overrightarrow{a},\overrightarrow{b})}&=&\pi - \gamma \\
\widehat{(\overrightarrow{b},\overrightarrow{c})}&=&\pi - \alpha \\
\widehat{(\overrightarrow{c},\overrightarrow{a})}&=&\pi - \beta \\
\end{eqnarray*}
\item On développe en utilisant le caractère bilinéaire du produit scalaire 
\[\Vert\overrightarrow{a}+\overrightarrow{b}+\overrightarrow{c}\Vert^2 = 3 + 2( \overrightarrow{a}\centerdot \overrightarrow{b} + \overrightarrow{a}\centerdot \overrightarrow{c} + \overrightarrow{b}\centerdot \overrightarrow{c})\]
Or 
\[\overrightarrow{a}\centerdot \overrightarrow{b} = \cos \widehat{(\overrightarrow{a},\overrightarrow{b})} = \cos (\pi - \gamma)= -\cos \gamma\]
avec des formules analogues pour les autres angles, donc 
\[\Vert\overrightarrow{a}+\overrightarrow{b}+\overrightarrow{c}\Vert = 3 - 2( \cos \alpha +\cos \beta +\cos \gamma\]
Il est alors clair que la somme des $\cos$ est $\frac{3}{2}$ si et seulement si
\[\overrightarrow{a}+\overrightarrow{b}+\overrightarrow{c}=\overrightarrow{0}\]
Montrons que cela entraine que le triangle est équilatéral. Pour cela on va montrer que les trois angles sont égaux en prouvant que les trois produits scalaires sont égaux.\newline
On suppose donc que la somme des trois vecteurs est nulle, on en déduit :
\begin{eqnarray*}
\overrightarrow{a}\centerdot \overrightarrow{b} = -\overrightarrow{a}\centerdot (\overrightarrow{a}+\overrightarrow{c})=
-1-\overrightarrow{a}\centerdot \overrightarrow{c}=
-1+(\overrightarrow{b}+\overrightarrow{c})\centerdot \overrightarrow{c}\\
= \overrightarrow{b}\centerdot \overrightarrow{c}
\end{eqnarray*} 
On démontre de même l'autre inégalité ce qui prouve bien que le triangle est équilatéral.
\end{enumerate}
