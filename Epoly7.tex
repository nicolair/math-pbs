%<dscrpt>Exercices sur les polynômes.</dscrpt>
Les deux questions sont indépendantes.
\begin{enumerate}
 \item 
\begin{enumerate}
 \item Former un polynôme du second degré à coefficients dans $\Z$ dont les racines sont
\begin{displaymath}
 a_1=\frac{1}{2}(1+\sqrt{5}),\hspace{0.5cm} a_2=\frac{1}{2}(1-\sqrt{5})
\end{displaymath}
\item Soit $A=X^5-X^4-2X^3+2X+2$. Montrer que $\widetilde{A}(a_1)=\widetilde{A}(a_2)$. Quelle est cette valeur commune ?
\end{enumerate}
\item On convient ici que $\binom{u}{v}=0$ lorsque $u$ et $v$ ne sont pas des entiers tels que $0\leq u \leq v$. Soit $n\in \N^*$ et $p$ entier entre $0$ et $n$. Montrer en considérant une puissance de $X^2+1$ que
\begin{displaymath}
 \binom{n}{p}= 
\sum_{k=0}^{2p}(-1)^{k+p}\binom{n}{k}\binom{n}{2p-k}
\end{displaymath}

\end{enumerate}
