%<dscrpt>Un exercice avec des intégrales.</dscrpt>
Soit $n$ et $m$ deux nombres entiers positifs ou nuls et $x$ un nombre r{\'e}el. On consid{\`e}re l'int{\'e}grale
\[
J_{n,m}(x)=\int_{0}^{x}(x-t)^{n}t^{m}\,dt .
\]

\begin{enumerate}
\item  Exprimer $J_{n,m}(x)$ en fonction de $J_{n-1,m+1}(x)$ pour $n\geq 1$.

\item  En d{\'e}duire une expression de $J_{n,m}(x).$ (on calculera $J_{0,n+m}(x)$ )

\item  En d{\'e}veloppant le bin{\^o}me $(x-t)^{n}$, donner une nouvelle expression de $J_{n,m}(x)$.

\item  D{\'e}duire des questions pr{\'e}c{\'e}dentes la valeur des sommes
\[
\sum_{p=0}^{n}\frac{(-1)^{p}}{p!(n-p)!(m+p+1)}\text{.}
\]

\item  Soit $x$ un nombre r{\'e}el. Calculer successivement sous les hypoth{\`e}ses a., b., c., d. l'int{\'e}grale
\[
F(x)=\int_{0}^{x}f(x-t)f(t)dt.
\]

\begin{enumerate}
\item  $f(t)=e^{t}$.

\item  $f(t)=t^{k}$ o{\`u} $k\in \N$.

\item  $f(t)=\frac{1}{1-t}$ en supposant $ x < 1$.

\item  $f(t)=\left\{
\begin{array}{ll}
1 & \text{si }t\in \left[ 0,1\right]  \\
0 & \text{si }t\notin \left[ 0,1\right]
\end{array}
\right. $
\end{enumerate}
\end{enumerate}
