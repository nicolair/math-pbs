%<dscrpt>Lemme de Hochschild (utilise un algorithme proche du pivot).</dscrpt>
On rappelle que le symbole de Kronecker $\delta_{ij}$ vaut 1 si $i=j$ et 0 sinon.\newline
L'objet de ce problème est de démontrer le \emph{Lemme de Hochschild} (question 1) et d'en déduire une application.\newline
Soit $X$ un ensemble quelconque et $V$ un sous espace vectoriel de dimension $p$ de l'espace de \emph{toutes} les fonctions de $X$
dans $\R$.

\begin{enumerate}
\item
\begin{enumerate}
  \item Montrer qu'il existe $x_1 \in X$ et une base $(a_1,a_2,\cdots,a_p)$ de $V$ telle que
\[
\forall i \in \{1,\cdots,p\} : a_i(x_1)=\delta_{i1}.
\]

\item Soit $k<p$, on suppose qu'il existe une famille $(x_1,x_2,\cdots,x_k)$ d'{\'e}l{\'e}ments de $X$ et une  base $(u_1,\cdots,u_p)$ de $V$ telle que
  \[
  \forall i \in \{1,\cdots,p\} , \forall j \in \{1,\cdots,k\} : u_i(x_j)=\delta_{ij}.
  \]
Montrer qu'il existe un {\'e}l{\'e}ment $x_{k+1}$ de $X$ et une base $(v_1,\cdots,v_p)$ de $V$ telle que
  \[
  \forall i \in \{1,\cdots,p\} , \forall j \in \{1,\cdots,k+1\} : v_i(x_j)=\delta_{ij}.
  \]

\item  Montrer qu'il existe une base $(w_1,\cdots ,w_p)$ de $V$ et une famille $(x_1,\cdots ,x_p)$ d'{\'e}l{\'e}ments de $X$ vérifiant
\begin{displaymath}
\forall (i,j)\in \{ 1,\ldots ,p\}^2 : w_i(x_j)=\delta _{ij}.
\end{displaymath}
\end{enumerate}

\item  Application. Soit $f$ une fonction \emph{d{\'e}rivable} de $\R$ dans $\R$ telle que l'espace engendré par ses translatées soit de dimension finie. On va montrer qu'elle vérifie une équation différentielle linéaire à coefficients constants.\newline
Pour tout r{\'e}el $a$, on note $f_a$ l'application définie par $f_a(t)=f(a+t)$ pour tout $t$ r{\'e}el. On pose
 \[V = \mathop{\mathrm{Vect}}(f_a,a\in \R)\]
et on suppose que $V$ (sous-espace de l'espace de toutes les applications dérivables de $\R$ dans $\R$) est \emph{de dimension finie }$p$.

\begin{enumerate}
\item  Montrer qu'il existe une base $(v_1,\cdots ,v_p)$ de $V$ et des
r{\'e}els $(x_1,\cdots ,x_p)$ tels que
\[
\forall (a,b)\in \R^2,f(a+b)=\sum_{i\in \left\{ 1,\ldots
,p\right\} }f(a+x_i)v_i(b).
\]

\item  Montrer que $f$ est ind{\'e}finiment d{\'e}rivable, que $%
f^{\prime }$ est dans $V$ et qu'il existe des r{\'e}els $a_0,a_1,\cdots a_p$
tels que
\[
a_pf^{(p)}+a_{p-1}f^{(p-1)}+\cdots +a_1f^{\prime }+a_0f=0.
\]
\end{enumerate}
\end{enumerate}
