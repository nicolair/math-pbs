%<dscrpt>Télescopage complexe.</dscrpt>
Soit $n\geq 2$ est un entier naturel. L'objet de cet exercice est de trouver une expression de
\begin{displaymath}
 S_n = \sum_{k=0}^n (-1)^k \frac{(1+2k)^3}{(1+2k)^4 + 4}.
\end{displaymath}

\begin{enumerate}
 \item Calculer sous forme algébrique les racines $4$-ièmes de $-4$.\newline
 On note $\mathcal{R}$ l'ensemble qu'elles forment et $z$ l'unique élément de $\mathcal{R}$ dont la partie réelle et la partie imaginaire sont strictement positives.\newline
 Préciser $z$ et exprimer les autres éléments de $\mathcal{R}$ en fonction de $z$ seulement.
 \item Simplifier, pour tout réel $x$, la somme
\begin{displaymath}
 T(x) = \frac{1}{x - z} + \frac{1}{x - \overline{z}} + \frac{1}{x + z} + \frac{1}{x + \overline{z}}
\end{displaymath}
On trouvera un quotient simple qui s'exprime uniquement en fonction de $x$.

\item Pour tout réel $y$, exprimer $y+2-z$ et $y+2 -\overline{z}$ en fonction de $y$, $z$, $\overline{z}$ seulement.\newline
En déduire, après simplification télescopique, une expression de
\begin{displaymath}
 T_n(x) = \sum_{k=0}^n (-1)^k\,T(x+2k)
\end{displaymath}
en fonction $z$, $\overline{z}$, $x$ et $n$.

\item Montrer que
\begin{displaymath}
 S_n = (-1)^n\frac{(n+1)}{4(n+1)^2 + 1}
\end{displaymath}


\end{enumerate}
