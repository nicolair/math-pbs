\subsection*{I. Nombre de partitions en $k$ parties.}
\begin{enumerate}
 \item Une partition d'un ensemble $E$ est un ensemble dont les éléments sont des parties de $E$ (ces parties devant vérifier certaines conditions). Une partition est donc un élément de $\mathcal{P}(\mathcal{P}(E))$. L'ensemble des partitions est donc une partie de $\mathcal{P}(\mathcal{P}(E))$ ou un élément de $\mathcal{P}(\mathcal{P}(\mathcal{P}(E)))$.\newline
 D'après le cours, si $E$ est fini alors $\mathcal{P}(E)$ est aussi fini. L'ensemble des partitions d'un ensemble fini est donc un ensemble fini.
 
 \item
 \begin{itemize}
  \item Comme tous les éléments d'un partition sont des parties non vides et disjointes, pour $k>n$, il n'existe pas de partition de $\llbracket 1,n \rrbracket$ en $k$ parties.
  \[
   k > n \Rightarrow S(n,k) = 0. 
  \]
  \item La seule partition de $E$ à un élément est le singleton $\left\lbrace E \right\rbrace \Rightarrow S(n,1)=1$.
  \item La seule partition de $\llbracket 1,n \rrbracket$ à $n$ éléments est $\left\lbrace \left\lbrace 1 \right\rbrace, \left\lbrace 2 \right\rbrace, \cdots \left\lbrace n \right\rbrace \right\rbrace $.
  \[
   S(n,n) = 1.
  \]
  \item Les partitions de $\llbracket 1,4 \rrbracket$ en $3$ parties sont constituées d'une paire et deux singletons. Il y en a donc autant que de paires dans un ensemble à $4$ éléments c'est à dire $6$. Ces partitions sont
  \begin{multline*}
     \left\lbrace \left\lbrace 1\right\rbrace, \left\lbrace 2\right\rbrace, \left\lbrace 3,4\right\rbrace \right\rbrace,
     \left\lbrace \left\lbrace 1\right\rbrace, \left\lbrace 3\right\rbrace, \left\lbrace 2,4\right\rbrace \right\rbrace,
     \left\lbrace \left\lbrace 1\right\rbrace, \left\lbrace 4\right\rbrace, \left\lbrace 2,3\right\rbrace \right\rbrace,\\
     \left\lbrace \left\lbrace 2\right\rbrace, \left\lbrace 3\right\rbrace, \left\lbrace 1,4\right\rbrace \right\rbrace,
     \left\lbrace \left\lbrace 2\right\rbrace, \left\lbrace 4\right\rbrace, \left\lbrace 1,3\right\rbrace \right\rbrace,
     \left\lbrace \left\lbrace 3\right\rbrace, \left\lbrace 4\right\rbrace, \left\lbrace 1,2\right\rbrace \right\rbrace
  \end{multline*}
 \end{itemize}

 \item Soit $n$ et $k$ dans $\N^*$ avec $k \leq n$. Classons les partitions de $\llbracket 1,n \rrbracket$ en $k$ parties selon qu'elles contiennent ou non le singleton $\left\lbrace n \right\rbrace $.
 \begin{itemize}
  \item Soit $\mathcal{P}_0$ l'ensemble des partitions en $k$ parties ne contenant pas $\left\lbrace n \right\rbrace $.\newline
  Soit $\mathcal{U} = \left\lbrace A_1, \cdots ,A_k\right\rbrace $ une partition appartenant à $\mathcal{P}_0$. Il existe $i$ tel que $n\in A_i$ et $A_i \setminus \left\lbrace  n \right\rbrace \neq \emptyset$. Si on remplace $A_i$ par $A_i \setminus \left\lbrace  n \right\rbrace \neq \emptyset$ dans $\mathcal{U}$, on obtient une partition de $\llbracket 1, n-1 \rrbracket$ en $k$ parties. On forme ainsi une application de $\mathcal{P}_0$ dans l'ensemble des partitions de $\llbracket 1, n-1 \rrbracket$ en $k$ parties. Cette application est surjective mais elle n'est pas injective. Chaque partition de $\llbracket 1, n-1 \rrbracket$ est l'image de $k$ partitions de $\llbracket 1,n \rrbracket$ selon le $A_i$ (avec $i$ entre $1$ et $k$) qui contient $n$. On en déduit
  \[
   \sharp \mathcal{P}_0 = k S(n-1,k).
  \]
  \item Soit $\mathcal{P}_1$ l'ensemble des partitions en $k$ parties contenant $\left\lbrace n \right\rbrace $.\newline
  \`A chaque partition $\mathcal{U}\in \mathcal{P}_1$, on peut associer $\mathcal{U}' = \mathcal{U} \setminus \left\lbrace n\right\rbrace $ qui est une partition de $\llbracket 1, n-1\rrbracket$ en $k-1$ parties. Cette fois l'application est bijective ce qui montre
  \[
   \sharp \mathcal{P}_1 = S(n-1,k-1).
  \]
 \end{itemize}
Avec la classification précédente,
\[
 S(n,k) = S(n-1,k-1) + k S(n-1,k).
\]

\end{enumerate}

\subsection*{II. Nombres de Bell.}
 \begin{enumerate}
  \item La formule résulte de la classification des partitions selon leur nombre de parties.
  \item On classe les partitions de $\llbracket 1 , n+1\rrbracket$ selon la partie contenant $n+1$.\newline
Soit $A$ une partie contenant $n+1$, notons $\mathcal{P}_A$ l'ensemble des partitions $\mathcal{U}$ telles que $A\in \mathcal{U}$. La classification considérée montre alors
\[
 B_{n+1} = \sum_{\stackrel{A \subset \llbracket 1,n+1 \rrbracket}{ n+1 \in A}} \sharp \mathcal{P}_A.
\]
Pour un $\mathcal{U}\in \mathcal{P}_A$, les éléments de $\mathcal{U}$ autres que $A$ forment une partition de $\llbracket 1, n \rrbracket \setminus A$. On en déduit que
\[
 \sharp \mathcal{P}_A = B_{n+1 - \sharp A}.
\]
On classe les $A$ suivant leur nombre d'éléments autres que $n+1$. Pour $i$ entre $0$ et $n$, il existe $\binom{n}{i}$ parties $A$ contenant $i$ éléments autres que $n+1$ et pour ces $A$, $\sharp \mathcal{P}_A = B_{n - i}$. On en déduit
\[
 B_{n+1} = \sum_{i=0}^n \binom{n}{i}B_{n - i} = \sum_{k=0}^n \binom{n}{k}B_{k} \text{ en posant } k = n-i.
\]

 \item
 \begin{enumerate}
  \item Avec la formule de dérivation d'une fonction composée,
\[
 f'(x) = e^x f(x)
\]

  \item D'après les questions 1 et 2, la suite $\left( B_n \right)_{n \in \N}$ est complètement déterminée par récurrence par 
\[
 B_0 = 1 , \hspace{0.5cm} B_{n+1} = \sum_{k=0}^n \binom{n}{k}B_k.
\]
Or $f(0)=e^0 = 1$ et, en utilisant la formule de Leibniz,
\[
 f^{(n+1)}(x) = \sum_{k=0}^n \binom{n}{k}e^xf^{(k)}(x)
 \Rightarrow f^{(n+1)}(0) = \sum_{k=0}^n \binom{n}{k}f^{(k)}(0).
\]
On en déduit que $B_n = f^{(n)}(0)$ pour tout $n\in \N$.
 \end{enumerate}

 \item D'après la formule de Taylor avec reste intégral à l'ordre $n$ entre $0$ et $x$,
\[
 R_n(x) = \int_0^x\frac{(x-t)^n}{n!}f^{(n + 1)}(t) \,dt, \hspace{0.5cm} r_n(x) = \int_0^x\frac{(x-t)^n}{n!}e^t \,dt
\]
La somme que l'on nous demande de majorer est la partie principale du développement de Taylor de l'exponentielle en $1$ écrite \ogà l'envers\fg.
\[
 \sum_{k=0}^n\frac{1}{(n-k)!} = \sum_{i=0}^n \frac{1}{i!} = e^1 - r_n(1) \Rightarrow 
 e - \sum_{k=0}^n\frac{1}{(n-k)!} = r_n(1)\geq 0
\]
car dans l'expression intégrale de $r_n$, les bornes sont dans le bon sens et les fonctions à intégrer sont positives. 
 
 \item Pour calculer $f^{(n+1)}(1)$, utilisons la formule de Leibniz dans l'expression de $f'$ trouvée en 3.a.
\begin{multline*}
  \frac{f^{(n+1)}(1)}{(n+1)!} = \frac{1}{(n+1)!}\sum_{k=0}^n\binom{n}{k}f^{(k)}(1)e
 = \frac{e}{n+1}\sum_{k=0}^n\frac{1}{(n-k)!} \frac{f^{(k)}(1)}{k!} \\
 \leq \frac{e}{n+1} \left( \sum_{k=0}^n\frac{1}{(n-k)!}\right)  M_n \leq \frac{e^2}{n+1} M_n
\end{multline*}
d'après l'inégalité de la question précédente.\newline
Pour $n$ assez grand, $\frac{e^2}{n+1} < 1$ (par exemple $n\geq 8$). On en déduit que $M_{n+1} = M_n = \cdots = M_9 = M_8$. 
\[
 \left( \forall n \geq 9, \; \frac{f^{(n)}(1)}{n!} \leq \frac{e^2 M_8}{n}\right) \Rightarrow \left( \frac{f^{(n)}(1)}{n!} \right)_{n \in \N} \rightarrow 0.
\]

 
 \item Toutes les dérivées de $f$ s'expriment avec la formule de Leibniz. On en déduit qu'elles sont positives donc croissantes. En particulier $f^{(n)}(x) \leq f^{(n)}(1)$ pour tous les $n\in \N$ et tous les $x \in \left[ -1,1\right]$. En intégrant, il vient, pour tout $x \in \left[ 0,1\right]$ 
\begin{multline*}
 R_n(x) = \int_0^x\frac{(x-t)^n}{n!}f^{(n+1)}(t)\,dt \\
  \leq (n+1)\left( \int_0^x(x-t)^n\,dt\right) \frac{f^{(n+1)}(1)}{(n+1)!}
  = \underset{ = x^{n+1} \leq 1}{\underbrace{\left[ -(x-t)^{n+1}\right]_{0}^{x}}} \,\frac{f^{(n+1)}(1)}{(n+1)!} \rightarrow 0.
\end{multline*}
Pour les $x\in \left[ -1,0\right]$, le reste n'est pas forcément positif mais la valeur absolue se majore de la même manière
\[
 \left|R_n(x)\right| \leq \int_{x}^{0} \frac{(t-x)^n}{n!} f^{(n+ 1)}(t)\,dt \\
 \leq |x|^{n+1}\frac{f^{(n+1)}(1)}{(n+1)!} \rightarrow 0.
\]

 \end{enumerate}

\subsection*{III. Une suite de polynômes.}
\begin{enumerate}
 \item La famille $(H_0,H_1, \cdots,H_n)$ contient $n+1 = \dim \R_n[X]$ vecteurs. Elle est libre car de degrés échelonnés, c'est donc une base de $\R_n[X]$.
 
 \item
 \begin{enumerate}
  \item On peut factoriser $H_k$:
\[
 H_{k+1} + k H_k = X(X-1) \cdots H_k\left[ (X-k) + k\right] = X H_k. 
\]

  \item Pour $n\in \N$, notons $\mathcal{P}_n$ l'égalité à démontrer. Pour de petites valeurs de $n$:
\[
 S(0,0) = 1 \Rightarrow \mathcal{P}_0, \hspace{0.5cm} \left( S(1,0)=0, S(1,1) = 1\right) \Rightarrow \mathcal{P}_1.
\]
Montrons que $\mathcal{P}_n\Rightarrow \mathcal{P}_{n+1}$. Supposons $\mathcal{P}_n$, alors:
\begin{multline*}
 X^{n+1}= \sum_{k=0}^nS(n,k)XH_k
 = \sum_{k=0}^nS(n,k)\left( H_{k+1} + kH_k\right) \\
 = \sum_{k=1}^{n+1}S(n,k-1)H_k + \sum_{k=0}^nkS(n,k)H_k\\
 = \sum_{k=1}^{n}\left( \underset{= S(n+1,k)}{\underbrace{S(n,k-1) + kS(n,k)}}\right) H_k + \underset{ = 1 = S(n+1,n+1)}{\underbrace{S(n,n)}}H_{n+1}\\
 = \sum_{k=0}^{n+1}S(n+1,k)H_k \; \text{ car $S(n+1,0) = 0$.}
\end{multline*}
 \end{enumerate}

 \item
 \begin{enumerate}
  \item D'après le cours:
  \[
   \sharp\mathcal{F} = \sharp \mathcal{F}(\llbracket 1,n \rrbracket,\llbracket 1,p \rrbracket) = p^n.
  \]

  \item Associons une relation d'équivalence $\mathcal{R}_f$ sur $\llbracket 1,n \rrbracket$ à une fonction $f\in \mathcal{F}$ par:
\[
 \forall(x,x') \in \llbracket 1,n\rrbracket, \; x \mathcal{R}_f x' \Leftrightarrow f(x) = f(x').
\]
  Cette relation d'équivalence induit une partition $\mathcal{U}_f$ associée à $f$. Un élément de cette partition est l'ensemble des antécédents d'une image par $f$. Il s'agit d'une partition de $\llbracket 1,n \rrbracket$ en $k$ parties où $k$ est le nombre d'images distinctes par $f$.
  
  \item D'après le cours, le nombre de fonctions injectives d'une ensemble à $k$ éléments dans $\llbracket 1,n \rrbracket$ est 
  \[
   p(p-1)\cdots(p-k+1) = H_k(p).
  \]
  \item Si on se donne une partition $\mathcal{U}$ en $k$ parties, pour combien de $f\in \mathcal{F}$ a-t-on $\mathcal{U} = \mathcal{U}_F$? Les ensembles d'antécédents par $f$ sont fixés mais pas les images. Il existe donc autant de $f$ que d'injections entre les éléments de la partition et $\llbracket 1,p \rrbracket$ c'est à dire $H_k(p)$. 
En classant les $f$ de $\mathcal{F}$ selon la partition $\mathcal{U}_f$ puis en regroupant les partitions suivant leur nombre de parties, on obtient
\[
p^n = \sharp \mathcal{F} = \sum_{k= }^n\left(\sum_{\mathcal{U} \text{ partition en $k$ parties}} H_k(p) \right) 
= \sum_{k= }^n S(n,k)H_k(p).
\]
On en déduit l'égalité polynomiale III.2b. car les deux fonctions polynomiales associées prennent les même valeurs pour une \emph{infinité} de $p$.
 
 \end{enumerate}
\end{enumerate}

\subsection*{IV. Somme de puissances.}
\begin{enumerate}
 \item Supposons $P = aX^b + bX^{p-1} + \cdots$ avec $p>0$. Les termes de degré $p$ et $p-1$ de $\Delta(P)$ sont
\begin{multline*}
 a(X+1)^p - aX^p + b(X+1)^{p-1} - bX^{p-1} + \text{ termes de degré } < p-1 \\
 = apX^{p-1} + \text{ termes de degré } < p-1 .
\end{multline*}
Donc $\Delta(P)$ est de degré $p-1$ et de coefficient dominant $ap$. On en déduit que $\Delta$ est surjective et de noyau $\R_0[X]$.

 \item
 \begin{enumerate}
  \item Comme $\Delta$ est surjective, il existe des polynômes $U$ (de degré $n+1$) tels que $\Delta(U) = (X+1)^n$. Le noyau de $\Delta$ étant formé des polynômes de degré $0$, ces polynômes $U$ sont égaux à une constante près. On en déduit que $U-U(0)$ est l'unique polynôme vérifiant les deux conditions. Il est noté $U_n$. 
  
  \item D'après les questions précédentes, $U_1$ est de la forme $aX^2 + b X$.
\[
 \Delta(U_1) = X +1 \Leftrightarrow 
 \left\lbrace 
 \begin{aligned}
  2a &= 1 \\ 2b &= 1
 \end{aligned}
 \right.
 \Leftrightarrow U_1 = \frac{1}{2}X(X+1).
\]
$U_2$ est de la forme $aX^3 + bX^2 + cX$.
\begin{multline*}
 \Delta(U_2) = (X +1)^2 \Leftrightarrow 
 \left\lbrace 
 \begin{aligned}
  3a &= 1 \\ 3a + 2b &= 2 \\ a + b + c &= 1
 \end{aligned}
 \right.\\
 \Leftrightarrow U_2 = \frac{1}{3}X^3 + \frac{1}{2}X^2 + \frac{1}{6}X = \frac{1}{6}(2X^2 + 3X + 1)X
 = \frac{1}{6}X(X+1)(2X+1).
\end{multline*}

  \item On montre la formule demandée par récurrence sur $p$.\newline 
Si $p=0$, alors $U_n(0) = 0 = 0^n$. 
Pour passer de $p$ à $p+1$, utilisons $\Delta(U_{n}) = (X+1)^{n}$.
\[
 U_{n}(p+1) = U_{n}(p) + \Delta(U_{n})(p) = \sum_{k=0}^{n}k^{n} + (p+1)^{n}.
\]

 \end{enumerate}

 \item L'encadrement est immédiant en intégrant les inégalités:
 \begin{align*}
k-1 \leq x \leq k \Rightarrow x^n \leq k^n &\Rightarrow \int_{k-1}^k x^n\,dx \leq  \int_{k-1}^k k^n\,dx = k^n \\
k \leq x \leq k+1 \Rightarrow k^n \leq x^n &\Rightarrow k^n = \int_{k}^{k+1} k^n\,dx \leq  \int_{k}^{k+1} x^n\,dx   
 \end{align*}
En sommant les inégalités de $k = 1$ à $p$, on obtient
\[
 \int_0^px^n\,dx \leq U_n(p) \leq \int_1^{p+1}x^n\, dx.
\]
En calculant les intégrales, il vient
\begin{multline*}
\frac{1}{n+1}p^{n+1} \leq U_n(p) \leq  \frac{1}{n+1}\left( (p+1)^{n+1} -1\right)
\Rightarrow
1 \leq \frac{U_n(p)}{\frac{p^{n+1}}{n+1}} \leq \left( \frac{p+1}{p}\right)^n \rightarrow 1 \\ 
\Rightarrow \left( U_n(p) \right)_{n \in \N^*} \sim \frac{p^{n+1}}{n+1}.
\end{multline*}

 \item Calculons $\Delta(H_k)$ pour $k\geq 1$ (car $\Delta(H_0)=0$):
\begin{multline*}
 \Delta(H_k) = (X+1)(X)\cdots(X-k+2) - X(X-1)\cdots(X-k+1)\\
 = X(X-1)\cdots (X-k+2)\left( (X+1) - (X-k+1)\right)
 = k H_{k-1}.
\end{multline*}
On en déduit
\[
 A = \Mat_{\mathcal{H}}\Delta_n
 = \begin{pmatrix}
    0      & 1      & 0      & \cdots & 0      \\
    \vdots & 0      & 2      & \ddots & \vdots \\
    \vdots & \vdots & \ddots & \ddots & 0      \\
    \vdots & \vdots &        & \ddots & n      \\
     0     &   0    & \cdots & \cdots & 0      
   \end{pmatrix}
\]

 \item La question III.2. exprime $X^n$ dans la base $\mathcal{H}$
\[
 X^n = \sum_{k=0}^nS(n,k)H_k = \sum_{k=0}^n\frac{S(n,k)}{k+1} \Delta(H_{k+1})
 = \Delta\left( \sum_{k=0}^n\frac{S(n,k)}{k+1} H_{k+1}\right) 
\]
car $\Delta(H_{k+1}) = (k+1)H_k$. Comme $\Delta$ commute avec la substitution de $X+1$ à $X$, on obtient
\[
 U_n = \sum_{k=0}^n\frac{S(n,k)}{k+1} \widehat{H_{k+1}}(X+1)
\]
à une constante près car ils ont la même image par $\Delta$. Cette constante est nulle, car, en prenant la valeur en $0$, comme $H_{k+1}(1) = 0$ dès que $k>0$, on a
\[
 \sum_{k=0}^n\frac{S(n,k)}{k+1} H_{k+1}(1) = S(n,0) = 0.
\]


\end{enumerate}
