\subsection*{Partie I}
\begin{enumerate}
  \item Les solutions sont $3+2i$ et $1+i$.\newline
En effet, le discriminant est
\begin{displaymath}
  \Delta = (4+3i)^2-4(1+5i) = 3 + 4i = (2 + i)^2
\end{displaymath}
On a cherché ses racines carrées sous la forme $x+iy$ avec
\begin{displaymath}
\left\lbrace 
\begin{aligned}
  x^2 - y^2 &= 3 \\ x^2 + y^2 &= \sqrt{9+16} = 5 \\2xy &= 4  
\end{aligned}
\right. \Rightarrow 2x^2 = 8
\end{displaymath}
On en déduit les solutions
\begin{displaymath}
  \frac{1}{2}\left( 4+3i \pm (2+i)\right) \longrightarrow 3+2i \text{ et } 1+i
\end{displaymath}

  \item Il s'agit en fait d'une équation du premier degré en $z$:
\begin{multline*}
h(z) = Z \Leftrightarrow (3 + 3i)z-(1+5i) = (z-1)Z
\Leftrightarrow (3 + 3i-Z)z = 1 + 5i - Z \\
\Leftrightarrow z = \frac{Z -(1+5i)}{Z-(3 + 3i)}
\end{multline*}

\item Les points fixes de $h$ sont $3+2i$ et $1+i$ car leur recherche revient à l'équation de la première question.
\begin{displaymath}
h(z) = z \Leftrightarrow  (4+3i)z-(1+5i) = (z-1)z
\Leftrightarrow z^2  -(5+3i)z + (1+5i) = 0
\end{displaymath}
\end{enumerate}

\subsection*{Partie II}
\begin{enumerate}
  \item Les points fixes sont $w$ et $w'$. En effet,
\begin{multline*}
h(z)=z
\Leftrightarrow (s-u)z -p =(z-u)z
\Leftrightarrow z^2 -(u -s +u)z + p =0 \\
\Leftrightarrow z^2 - (w+w')z +ww' = 0 
\Leftrightarrow (z-w)(z-w')=0
\end{multline*}

\item 
\begin{enumerate}
  \item On trouve
\begin{displaymath}
h_u(z)-h_u(z') = K(z-z') \hspace{0.5cm}\text{ avec }\hspace{0.5cm}  K = \frac{(u-w)(u-w')}{(u-z)(u-z')}
\end{displaymath}
car, en réduisant la différence au même dénominateur et 4 termes se simplifient
\begin{multline*}
h_u(z)-h_u(z')
=\frac{\left( z'-u\right)\left( (s-u)z-p\right) - \left( z-u\right)\left( (s-u)z'-p\right) }{(z-u)(z'-u)} \\
= \frac{-z'p - u(s-u)z + zp + u(s-u)z' }{(z-u)(z'-u)}
= \frac{(p-u(s-u))(z-z') }{(z-u)(z'-u)}
\end{multline*}
et on peut factoriser
\begin{displaymath}
  p-u(s-u) = u^2 -su + p = (u -w)(u-w')
\end{displaymath}

  \item On utilise la question précédente et le fait que $w$ et $w'$ sont des points fixes
\begin{displaymath}
h_u(z)-w = h_u(z) - h_u(w) = \frac{(u-w)(u-w')}{(u-z)(u-w)}(z-w) = \frac{(u-w')(z-w)}{(u-z)}  
\end{displaymath}
De même pour $w'$
\begin{displaymath}
h_u(z)-w' = h_u(z) - h_u(w') = \frac{(u-w)(u-w')}{(u-z)(u-w')}(z-w') = \frac{(u-w)(z-w')}{(u-z)}   
\end{displaymath}
On en tire
\begin{displaymath}
  \frac{h_u(z)-w}{h_u(z)-w'} = T\, \frac{z-w}{z-w'} \hspace{0.5cm}\text{ avec }\hspace{0.5cm} T=\frac{u-w'}{u-w}
\end{displaymath}
\end{enumerate}

\item 
\begin{enumerate}
  \item En posant $c=\frac{1}{2}(w+w')$ (affixe du milieu) et $r = \frac{1}{2}(w-w')$, on peut écrire
\begin{displaymath}
\frac{z-w}{z-w'} = \frac{(z-c)-d}{(z-c) + d} 
= \frac{1}{|z-w'|^2}\left(|z-c|^2-|d|^2 
+\underset{\in i\R}{\underbrace{(z-c)\overline{d}-\overline{(z-c)}d}} \right) 
\end{displaymath}
Ce qui assure que le quotient est imaginaire pur si et seulement si $|z-c|^2=|d|^2$ c'est à dire si le point d'affixe $z$ est sur le cercle $\mathcal{C}$ de diamètre $[W,W']$.

  \item Lorsque $U$ est sur la droite $\mathcal{D}$ (le pôle sur la droite des points fixes), le cercle $\mathcal{C}$ (de diamètre les points fixes) est stable par l'homographie.\newline
En effet le $T$ de la question 2.b est alors réel et
\begin{displaymath}
Z\in \mathcal{C}\Rightarrow \frac{z-w}{z-w'}\in i\R
\Rightarrow \frac{h_u(z)-w}{h_u(z)-w'} = 
\underset{\in \R}{\underbrace{T}}\,
\underset{\in i\R}{\underbrace{\frac{z-w}{z-w'}}}
\in i\R
\Rightarrow \text{pt d'affixe }h_u(z) \in \mathcal{C}
\end{displaymath}

  \item Lorsque $U$ est sur le cercle $\mathcal{C}$ (de diamètre les points fixes), ce cercle est envoyé par l'homographie sur la droite $\mathcal{D}$ (la droite des points fixes).\newline
En effet le $T$ de la question 2.b est alors imaginaire pur et
\begin{displaymath}
Z\in \mathcal{C}\Rightarrow \frac{z-w}{z-w'}\in i\R
\Rightarrow \frac{h_u(z)-w}{h_u(z)-w'} = 
\underset{\in i\R}{\underbrace{T}}\,
\underset{\in i\R}{\underbrace{\frac{z-w}{z-w'}}}
\in \R
\Rightarrow \text{pt d'affixe }h_u(z) \in \mathcal{D}
\end{displaymath}

\end{enumerate}


\end{enumerate}

