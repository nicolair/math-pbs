Cet ouvrage a été publié en 2014 chez l'éditeur "In Libro Veritas". Depuis cette date, les textes qui le composent ont continués à être mis à jour. La préface de l'ouvrage publié est reproduite au dessous.



La collection "MATH\'EMATIQUES EN MPSI" propose des documents pédagogiques (recueils de problèmes corrigés, livres de cours) en complément de ceux distribués en classe.\newline
Les ouvrages de la collection sont  disponibles sur internet. En fait, ils sont \emph{produits en ligne} à partir d'une base de données (le \emph{maquis documentaire}) accessible à l'adresse
\begin{center}
 \href{http://maquisdoc.net}{http://maquisdoc.net}
\end{center}
Cette base est conçue pour être très souple. Elle accompagne les auteurs et les utilisateurs en leur permettant de travailler librement et au jour le jour.

Il est devenu impossible de travailler sans internet (y compris pour rédiger des problèmes de mathématiques) mais il est également impossible de ne travailler que sur écran. Le papier garde donc toute sa validité et la publication de livres sous la forme imprimée habituelle (à coté d'autres types de services) est encore totalement justifiée.\newline
En revanche, le modèle économique de l'édition est devenu obsolète pour de tels ouvrages péri-scolaires produits à partir de structures web. L'éditeur (\emph{In Libro Veritas}) a accepté de diffuser cette collection sous licence Creative Commons. Les auteurs peuvent ainsi user plus libéralement de leur droit d'auteur et offrir davantage de liberté aux lecteurs.


\begin{center}
 \textbf{"\emph{Problèmes basiques}"}
\end{center}
 est un recueil de problèmes corrigés.\newline
L'objectif de cet ouvrage est d'aider le lecteur à maitriser certains éléments essentiels et à repérer ceux qu'il ne maitrise pas. Pour cela, les énoncés sont le plus souvent très brefs, plutôt des exercices que des problèmes. Ils mettent en oeuvre directement certains points de cours (définition, théorème usuel, technique de calcul, ...). De nombreux thèmes sont abordés qui couvrent l'essentiel du programme de MPSI.  Cet ouvrage pourra être complété utilement par un entrainement technique encore plus fractionné : les "\href{http://back.maquisdoc.net/v-1/index.php?act=chvueelt&vue=vue_rap_math&id_elt=1668}{rapidexo}" proposés par l'interface en ligne du maquis documentaire.\newline
Une attention particulière a été portée à une redaction soigneuse et complète des corrigés. L'étudiant ne doit pas se condamner à trouver. La lecture d'une solution, après un temps de recherche assez court, s'avère plus rentable qu'un acharnement infructueux.\newline
Pour cet ouvrage cependant, les questions posées sont si simples et si directes que ne pas trouver signale une lacune sérieuse dans la maitrise du cours. L'étudiant est alors invité à reprendre l'étude de ses notes.


D'autres ouvrages de la collection proposent des textes moins immédiats (\emph{Problèmes d'approfondissement}) ou plus spécifiques (\emph{Problèmes d'automne}). 


 
