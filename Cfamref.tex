\subsubsection*{Partie I. Existence d'une famille vérifiant $\mathcal M$.}
\begin{enumerate}
 \item Calcul du déterminant de $M$ pour $n=2$.
\begin{displaymath}
 M=
\begin{bmatrix}
 1 & m \\
 m & 1
\end{bmatrix}
\text{ avec }
m=-\cos \dfrac{\pi}{a}
\end{displaymath}
On en déduit
\begin{displaymath}
 \det M = 1 - m^2 = \sin^2 \dfrac{\pi}{2} 
\end{displaymath}
Pour $n=3$, le calcul se fait en développant selon la première ligne. On conserve l'expression en $m_{i,j}$. Après calculs cela donne
\begin{displaymath}
 \det M =
\begin{vmatrix}
 1 & m_{12} & m_{13} \\
m_{21} & 1 & m_{23}\\
m_{31} & m_{32} & 1
\end{vmatrix}
= 1 - m^2_{23}- m^2_{12}- m^2_{13}+2m_{12}m_{13}m_{23}
\end{displaymath}

\item Si $\mathcal B =(e_1,\cdots,e_n)$ vérifie $\mathcal M$ alors chaque $e_i$ est unitaire car $(e_i/e_i)$ est un terme diagonal (égal à $1$). De plus pour $i$ et $j$ distincts, d'après l'inégalité de Cauchy-Schwarz :
\begin{displaymath}
 |m_{ij}|=|(e_i/e)|\leq |e_i||e_j|=1
\end{displaymath}
L'égalité dans Cauchy-Schwarz se produit seulement si les vecteurs sont colinéaires. Ici les vecteurs de la base sont non colinéaires deux à deux donc $|m_{ij}|<1$. Comme 
\begin{displaymath}
 m_{ij}= - \cos \dfrac{\pi}{a_{ij}}
\end{displaymath}
avec $a_{ij}$ entier, cela entraine $a_{ij}\geq 2$.
\item Construction d'une base directe vérifiant $\mathcal M$ dans le cas $n=2$.\newline
On se fixe une base orthonormée directe $(a_1,a_2)$ et on pose :
\begin{align*}
 e_1 &= a_1 \\
 e_2 &= ma_1 + \sqrt{1-m^2}a_2
\end{align*}
alors $(e_1,e_2)$ répond bien à la question car :
\begin{align*}
 (e_1/e_2) &= m \\
 \det_{(a_1,a_2)}(e_1,e2) &=
\begin{vmatrix}
 1 & m \\
 0 & \sqrt{1-m^2}
\end{vmatrix}
=\sqrt{1-m^2}>0
\end{align*}

\item \begin{enumerate}
 \item Comme dans la question précédente, on peut poser
\begin{displaymath}
 e_2 = m_{12} + \sqrt{1-m_{12}^2}a_2
\end{displaymath}
ainsi $(e_1/e_2)=m_{12}$ et 
\begin{displaymath}
 \det_{(a_1,a-2,a_3)}(e_1,e_2,a_3)=
\begin{vmatrix}
 1 & m_{12} & 0 \\
 0 & \sqrt{1-m_{12}^2} & 0 \\
0 & 0 & 1
\end{vmatrix}
= \sqrt{1-m_{12}^2} >0
\end{displaymath}

\item La droite $\mathcal D$ est l'intersection de deux plans affines respectivement orthogonaux à $e_1$ et $e_2$. Ces deux vecteurs sont dans $\Vect(a_1,a_2)$ qui est le plan orthogonal à $a_3$. La direction de $\mathcal D$ est donc $\Vect(a_3)$.\newline
Soit $x=x_1a_1+x_2a_2\in \Vect(a_1,a_2)\cap \mathcal D$. Alors :
\begin{align*}
 (x/e_1) &= m_{13} = x_1 \\
 (x/e_2) &= m_{23} = m_{12}x_1 +\sqrt{1-m_{12}^2} x_2
\end{align*}
On en déduit :
\begin{align*}
 x_1=m_{13} & & x_2 = \dfrac{m_{23}-m_{12}m_{13}}{\sqrt{1-m_{12}^2}}
\end{align*}
Comme $\mathcal D$ est orthogonale à $\Vect(a_1,a_2)$, la distance de $0_E$ à $\mathcal D$ est la norme du vecteur d'intersection $x$ que l'on vient de calculer. On en déduit :
\begin{displaymath}
 d(0_E,\mathcal D) =
\sqrt{m_{13}^2+\dfrac{(m_{23}-m_{12}m_{13})^2}{1-m_{12}^2}} 
= \dfrac{1}{1-m_{12}^2}\left( m_{13}^2+m_{23}^2-2m_{12}m_{13}m_{23}\right) 
\end{displaymath}

\item Les conditions que doivent satisfaire un vecteur $e_3$ pour que la famille $(e_1,e_2,e_3)$ vérifie $\mathcal M$ sont :
\begin{itemize}
 \item $\Vert e_3\Vert =1$. C'est à dire que $e_3$ est sur la sphère de rayon $1$ et centrée à l'origine.
\item $(e_3/e_1)=m_{13}$ et $(e_3/e_2)=m_{23}$. C'est à dire que $e_3$ est sur la droite $\mathcal D$.
\end{itemize}
La condition géométrique assurant l'existence d'un tel vecteur $e_3$ est donc que la droite $\mathcal D$ \emph{coupe} la sphère unité.
\item D'après la question 1., la condition $\det M > 0$ entraine
\begin{multline*}
 \det M > 0 \Rightarrow 1-m_{12}^2 > m_{23}^2 + m_{13}^2-2m_{12}m_{13}m_{13} \\
\Rightarrow
\dfrac{m_{23}^2 + m_{13}^2-2m_{12}m_{13}m_{13}}{1-m_{12}^2}<1
\Rightarrow d(0_E,\mathcal D)<1
\end{multline*}
Ce qui signifie que $\mathcal D$ coupe la sphère en deux vecteurs $c$ et $c'$ symétriques par rapport au plan $\Vect(e_1,e_2)$. Les deux familles $(e_1,e_2,c)$ et $(e_1,e_2,c')$ vérifient $\mathcal M$. Une seule est directe car la réflexion par rapport au plan change l'orientation.
\end{enumerate}

\item Cas particulier. Le calcul des $\cos$ conduit à :
\begin{align*}
 A= \begin{bmatrix}
 1 & 3 & 2 \\
3 & 1 & 4 \\
2 & 4 & 1
\end{bmatrix}
& &
M=
\begin{bmatrix}
 1 & -\dfrac{1}{2} & 0 \\
-\dfrac{1}{2} & 1 & -\dfrac{1}{\sqrt{2}}\\
0 & -\dfrac{1}{\sqrt{2}} & 1
\end{bmatrix}
\end{align*}
Le calcul du déterminant conduit à 
\begin{displaymath}
 \det M = \dfrac{1}{4}>0
\end{displaymath}
Il existe donc, d'après la question d. une base vérifiant $\mathcal M$.
\end{enumerate}

\subsubsection*{Partie II. Famille de réflexions.}
\begin{enumerate}
 \item La base $\mathcal B$ n'est pas orthonormée, la matrice du produit scalaire dans cette base est $M$. On en déduit que deux vecteurs de coordonnées $(x_1,\cdots,x_n)$ et $(y_1,\cdots,y_n)$ sont orthogonaux si et seulement si :
\begin{displaymath}
 \begin{bmatrix}
  x_1 & \cdot & x_n
 \end{bmatrix}
M
\begin{bmatrix}
 y_1 \\ \vdots \\ y_n
\end{bmatrix}
=0
\end{displaymath}


 \item \begin{enumerate}
 \item Calcul de la matrice $S_1$ de $\sigma_1$. Par définition $\sigma_1(e_1)=-e_1$. Le vecteur $me_1-e_2$ est orthogonal à $e_1$ donc conservé par $\sigma_1$:
\begin{multline*}
 \sigma_1(me_1-e_2)=me_1-e_2 \Rightarrow -me_1 -\sigma_1(e_2)=me_1-e_2 \\
\Rightarrow \sigma_1(e_2) = -2me_1+e_2
\end{multline*}
 On en déduit :
\begin{displaymath}
 S_1=
\begin{bmatrix}
 -1 & -2m \\
0 & 1
\end{bmatrix}
\end{displaymath}
Calcul de la matrice $S_2$ de $\sigma_2$. Le raisonnement est le même en considérant le vecteur $e_1-me_2$ orthogonal à $e_2$ donc conservé par $\sigma_2$. Après calculs :
\begin{displaymath}
 S_2 =
\begin{bmatrix}
 1 & 0 \\
-2m & -1
\end{bmatrix}
\end{displaymath}
On peut calculer le produit matriciel :
\begin{displaymath}
 T =S_1S_2 =
\begin{bmatrix}
 -1 & -2m \\
0 & 1
\end{bmatrix}
\begin{bmatrix}
 1 & 0 \\
-2m & -1
\end{bmatrix}
\\
=
\begin{bmatrix}
 -1+4m^2 & 2m \\
-2m & -1
\end{bmatrix}
\end{displaymath}

\item Les vecteurs de $\mathcal C$ et $\mathcal B$ s'expriment les uns en fonction des autres :
\begin{align*}
 \left\lbrace 
\begin{aligned}
  e_1 &= a_1\\
 e_2 &= ma_1 +\sqrt{1-m^2}a_2 
\end{aligned}
\right. 
&\Leftrightarrow &
 \left\lbrace 
\begin{aligned}
 a_1 &= e_1 \\
 a_2 &= -\dfrac{m}{\sqrt{1-m^2}}e_1 +\dfrac{1}{\sqrt{1-m^2}}e_2 
\end{aligned}
\right. 
\end{align*}
On en déduit les matrices de passage
\begin{align*}
 P=P_{\mathcal C \mathcal B}=
\begin{bmatrix}
 1 & m \\
0 & \sqrt{1-m^2}
\end{bmatrix}
& &
P^{-1}=P_{\mathcal B \mathcal C}=
\begin{bmatrix}
 1 & -\dfrac{m}{\sqrt{1-m^2}} \\
0 & \dfrac{1}{\sqrt{1-m^2}}
\end{bmatrix}
\end{align*}
Pour la matrice de $\sigma_1$, il est inutile d'utiliser la formule de changement de base car $a_2$ est conservé par $\sigma_1$ (il est orthogonal à $a_1=e_1$).
\begin{displaymath}
 \Mat_{\mathcal C}\sigma_1 =
\begin{bmatrix}
 -1 & 0 \\
0 & 1
\end{bmatrix}
\end{displaymath}
Pour la matrice de $\sigma_2$, on utilise la formule de changement de base et $m=-\cos \frac{\pi}{a}$
\begin{displaymath}
\Mat_{\mathcal C}\sigma_2 =PS_2P^{-1}
=
\begin{bmatrix}
 1-2m^2 & -2m\sqrt{1-m^2} \\
2m\sqrt{1-m^2} & 2m^2-1
\end{bmatrix}
=
\begin{bmatrix}
 -\cos\dfrac{2\pi}{a} & \sin \dfrac{2\pi}{a} \vspace{4pt}\\
\sin \dfrac{2\pi}{a} & \cos \dfrac{2\pi}{a} \vspace{4pt}
\end{bmatrix}
\end{displaymath}
On en déduit la matrice de $\tau=\sigma_1\circ \sigma_2$
\begin{displaymath}
 \Mat_{\mathcal C}\tau =
\begin{bmatrix}
 -1 & 0 \\
0 & 1
\end{bmatrix}
\begin{bmatrix}
 -\cos\dfrac{2\pi}{a} & \sin \dfrac{2\pi}{a} \vspace{4pt}\\
\sin \dfrac{2\pi}{a} & \cos \dfrac{2\pi}{a} \vspace{4pt}
\end{bmatrix}
=
\begin{bmatrix}
\cos\dfrac{2\pi}{a} & -\sin \dfrac{2\pi}{a} \vspace{4pt} \\
\sin \dfrac{2\pi}{a} & \cos \dfrac{2\pi}{a} \vspace{4pt}
\end{bmatrix}
\end{displaymath}
Cela signifie que $\tau$ est la rotation d'angle $\frac{2\pi}{3}$.
\end{enumerate}
 
\item Cas $n=3$. Par définition, $\sigma_1$ est la symétrie orthogonale par rapport au plan $\left(\Vect(e_1) \right)^{\perp}$. Donc $e_1$ est transformé en son opposé alors que les vecteurs $m_{13}e_1-e_3$ et $m_{12}e_1-e_2$ sont fixés par $\sigma_1$ car ils sont orthogonaux à $e_1$. On peut exploiter cela pour obtenir les images de $e_3$ et $e_2$:
\begin{multline*}
 \sigma_1(m_{13}e_1-e_3)=m_{13}e_1-e_3 \Rightarrow -m_{13}e_1-\sigma_1(e_3)=m_{13}e_1-e_3 \\
\Rightarrow \sigma_1(e_3) = -2m_{13}e_1 +e_3
\end{multline*}
\begin{multline*}
 \sigma_1(m_{12}e_1-e_2)=m_{12}e_1-e_2 \Rightarrow -m_{12}e_1-\sigma_1(e_2)=m_{12}e_1-e_2 \\
\Rightarrow \sigma_1(e_2) = -2m_{12}e_1 +e_2
\end{multline*}
On en déduit la matrice de $\sigma_1$ :
\begin{displaymath}
 S_1 =
\begin{bmatrix}
 -1 & -2m_{12} & -2m_{13} \\
0 & 1 & 0 \\
0 & 0 & 1 
\end{bmatrix}
\end{displaymath}
De même, $e_2$ est transformée par $\sigma_2$ en son opposé alors que $m_{21}e_2-e_1$ et $m_{23}e_2-e_3$ sont fixés. On obtient comme au dessus :
\begin{align*}
 \sigma_2(e_1)=e_1-2m_{21}e_2 & & \sigma_2(e_3) = e_3 -2m_{23}e_2
\end{align*}
d'où la matrice de $\sigma_2$ :
\begin{displaymath}
 S_2 = 
\begin{bmatrix}
 1 & 0 & 0 \\
-2m_{12} & -1 & -2m_{32} \\
0 & 0 & 1
\end{bmatrix}
\end{displaymath}
De même, les vecteurs $e_1-2m_{31}e_3$ et $e_2-2m_{32}e_3$ sont fixés par $\sigma_3$ et conduisent à la matrice de $\sigma_3$.
\begin{displaymath}
 S_3=
\begin{bmatrix}
 1 & 0 & 0 \\
0 & 1 & 0 \\
-2m_{31} & -2m_{32} & -1
\end{bmatrix}
\end{displaymath}

\item \begin{enumerate}
 \item On a déjà calculé en I.5. la matrice $M$.
 \begin{displaymath}
 M=
\begin{bmatrix}
 1 & -\dfrac{1}{2} & 0 \\
-\dfrac{1}{2} & 1 & -\dfrac{1}{\sqrt{2}}\\
0 & -\dfrac{1}{\sqrt{2}} & 1
\end{bmatrix}
\end{displaymath}
On en déduit les matrices $S_1$, $S_2$ et $S_3$
\begin{align*}
 S_1=
\begin{bmatrix}
 -1 & 1 & 0 \\
 0 & 1 & 0 \\
0 & 0 & 1
\end{bmatrix}
& &
S_2=
\begin{bmatrix}
 1 & 0 & 0 \\
1 &-1 &\sqrt{2} \\
0 & 0 & 1
\end{bmatrix}
& &
S_3=
\begin{bmatrix}
1 & 0 & 0 \\
0 & 1 & 0 \\
0 & \sqrt{2} & -1
\end{bmatrix}
\end{align*}
puis leur produit
\begin{displaymath}
 T = S_1 S_2 S_3 =
\begin{bmatrix}
 0 & 1 & -\sqrt{2} \\
1 & 1 & -\sqrt{2} \\
0 & \sqrt{2} & -1
\end{bmatrix}
\end{displaymath}

\item D'après l'expression de $T$, un vecteur $u$ de coordonnées $(x,y,z)$ dans $\mathcal B$ vérifie $\tau(u)=-u$ si et seulement si
\begin{displaymath}
 \left\lbrace 
\begin{aligned}
 y  -\sqrt{2}z &= -x\\
 x+y -\sqrt{2}z &=-y \\
 \sqrt{2} y - z &=-z
\end{aligned}
\right. \Leftrightarrow
\left\lbrace 
\begin{aligned}
x+y-\sqrt{2}z &=0 \\
x+ 2y -\sqrt{2}z &= 0 \\
\sqrt{2}y &=0
\end{aligned}
\right. 
\Leftrightarrow
\left\lbrace 
\begin{aligned}
y &=0 \\
x-\sqrt{2}z &=0
\end{aligned}
\right. 
\end{displaymath}
On choisit donc le vecteur unitaire
\begin{displaymath}
 u = \dfrac{1}{\sqrt{3}}\left( \sqrt{2}e_1 + e_3\right) 
\end{displaymath}
On choisit un vecteur unitaire $v$ orthogonal à $u$. Par exemple
\begin{displaymath}
 v = \dfrac{1}{\sqrt{3}}\left( e_1 - \sqrt{2} e_3\right) 
\end{displaymath}
On ne peut pas complèter la famille $(u,v)$ par $w=u\wedge v$ pour former une base orthonormée directe $\mathcal D =(u,v,w)$ car, la base $\mathcal B$ n'étant pas orthonormée directe, le calcul du produit vectoriel n'est pas facile. On va trouver un $w$ en utilisant la condition d'orthogonalité de la question 1.\newline
On cherche donc un $w$ de coordonnées $(x,y,z)$ dans $\mathcal B$ orthogonal à $u$ et $v$.  Cela se traduit par :
\begin{align*}
 \begin{bmatrix}
  \sqrt{2} & 0 & 1
 \end{bmatrix}
M
\begin{bmatrix}
 x \\ y \\ z
\end{bmatrix}
 &= 0 
 & &
 \begin{bmatrix}
  1 & 0 & -\sqrt{2}
 \end{bmatrix}
M
\begin{bmatrix}
 x \\ y \\ z
\end{bmatrix}=0
\end{align*}
Avec l'expression de $M$ de la question I.5, cela donne :
\begin{align*}
 \left\lbrace
\begin{aligned}
 \sqrt{2}x -\sqrt{2}y +z &= 0\\
x+\dfrac{1}{2}y-\sqrt{2}z &=0
\end{aligned}
 \right. 
&\Leftrightarrow
\left\lbrace
\begin{aligned} 
x &= \dfrac{1}{\sqrt2}z \\
y &= \sqrt{2}z
\end{aligned}
\right. 
\end{align*}
On en déduit que 
\begin{displaymath}
 \left( \Vect(u,v)\right)^{\perp}=\Vect(e_1+2e_2+\sqrt{2}e_3) 
\end{displaymath}
Le calcul du déterminant
\begin{displaymath}
 \begin{vmatrix}
  \sqrt{2} & 1 & 1 \\
0 & 0 & 2 \\
1 & -\sqrt{2} & \sqrt{2}
 \end{vmatrix}
=-2
\begin{vmatrix}
 \sqrt{2} & 1 \\
1 & -\sqrt{2}
\end{vmatrix}
= 6
\end{displaymath}
montre que la famille $(u,v,e_1+2e_2+\sqrt{2}e_3)$ est directe.\newline
Attention, comme la base n'est pas orthonormée, le carré de la norme de $e_1+2e_2+\sqrt{2}e_3$ n'est pas $7$ mais
\begin{multline*}
\Vert e_1+2e_2+\sqrt{2}e_3 \Vert^2 = 1 + 4 + 2 + 2\times(-\dfrac{1}{2})\times(1\times 2)\\
 + 2\times 0 \times(1\times\sqrt{2}) + 2\times (-\dfrac{1}{\sqrt{2}})\times(2\times \sqrt{2}) = 1
\end{multline*}
On choisira donc comme  base orthonormée directe 
\begin{displaymath}
 \left\lbrace
\begin{aligned}
 u &= \dfrac{1}{\sqrt{3}}\left(\sqrt{2}e_1 + e_3 \right) \\
v &= \dfrac{1}{\sqrt{3}}\left(e_1 -\sqrt{2} e_3 \right) \\
w &=  e_1+2e_2+\sqrt{2}e_3
\end{aligned}
\right. 
\end{displaymath}


\item On connait la matrice $T$ de $\tau$ dans $\mathcal B$. Pour obtenir la matrice de $\tau$ dans $\mathcal D$ on utilise la formule de changement de base
\begin{displaymath}
 \Mat_{\mathcal D} \tau = P^{-1} T P
\end{displaymath}
avec
\begin{displaymath}
 P = P_{\mathcal B \mathcal D}=
\begin{bmatrix}
 \dfrac{\sqrt{2}}{\sqrt{3}} & \dfrac{1}{\sqrt{3}} & 1 \vspace{3pt}\\
0 & 0 &2 \vspace{3pt}\\
\dfrac{1}{\sqrt{3}} & -\dfrac{\sqrt{2}}{\sqrt{3}} & \sqrt{2} \vspace{3pt}
\end{bmatrix}
\end{displaymath}
Pour obtenir la matrice $P^{-1}$, on doit exprimer $e_1$, $e_2$, $e_3$ en fonction de $u$, $v$, $w$.
\begin{displaymath}
\left\lbrace
\begin{aligned}
 u &= \dfrac{1}{\sqrt{3}}\left(\sqrt{2}e_1 + e_3 \right) \\
v &= \dfrac{1}{\sqrt{3}}\left(e_1 -\sqrt{2} e_3 \right) \\
w &=  e_1+2e_2+\sqrt{2}e_3 
\end{aligned}
\right. 
\\
\Leftrightarrow
\left\lbrace 
\begin{aligned}
 e_1 &= \sqrt{\dfrac{2}{3}}u + \dfrac{1}{\sqrt{3}}v \\
 2e_2 &= w -e_1 -\sqrt{2}e_3 \\
 e_3 &= \dfrac{1}{\sqrt{3}}u - \sqrt{\dfrac{2}{3}}v
\end{aligned}
\right. 
\end{displaymath}
d'où
\begin{displaymath}
 e_2 = -\dfrac{\sqrt{2}}{\sqrt{3}}u + \dfrac{1}{2\sqrt{3}}v + \dfrac{1}{2}w
\end{displaymath}
\begin{displaymath}
 P^{-1}=
\begin{bmatrix}
 \dfrac{\sqrt{2}}{\sqrt{3}} & -\dfrac{\sqrt{2}}{\sqrt{3}} & \dfrac{1}{\sqrt{3}} \vspace{3pt}\\
\dfrac{1}{\sqrt{3}}& \dfrac{1}{2\sqrt{3}} & - \sqrt{\dfrac{2}{3}} \vspace{3pt}\\
0 & \dfrac{1}{2} & 0 \vspace{3pt}
\end{bmatrix}
\end{displaymath}
On en déduit :
\begin{displaymath}
\Mat_{\mathcal D}\tau =
P^{-1}\begin{bmatrix}
 0 & 1 & -\sqrt{2} \\
1 & 1 & -\sqrt{2} \\
0 & \sqrt{2} & -1
\end{bmatrix}
P 
=
\begin{bmatrix}
 -1 & 0 & 0 \vspace{3pt}\\
 0 & \dfrac{1}{2} & -\dfrac{\sqrt{3}}{2} \vspace{3pt}\\
 0 & \dfrac{\sqrt{3}}{2} & \dfrac{1}{2} \vspace{3pt}
\end{bmatrix}
\end{displaymath}
La transformation $\tau$ est donc une \emph{rotation miroir}, composée de la rotation d'angle $\frac{\pi}{3}$ autour de $u$ et de la réflexion par rapport au plan orthogonal à $u$.
\end{enumerate}
\end{enumerate}

