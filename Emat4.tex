%<dscrpt>Matrice semblable à son inverse: exemples.</dscrpt>
Dans tout ce problème, $E$ est un $\R$-espace vectoriel de dimension $3$.\newline
Soit $p$ un entier naturel non nul. Dans tout le problème sauf dans la qestion 1. l'entier $p$ est égal à $3$.\newline
Deux matrices quelconques $A$ et $B$ de $\mathcal M_p(\R)$ sont dites \emph{semblables} si et seulement si il existe une matrice inversible $P\in GL_p(\R)$ telle que $B=P^{-1}AP$. On notera alors $A\sim B$.\newline
L'objet de ce problème est de donner des exemples de matrices semblables à leurs inverses. 
\begin{enumerate}
 \item (question de cours) On rappelle que la trace d'une matrice est la somme des termes de sa diagonale. Montrer que deux matrices semblables ont la même trace.
 \item Soit 
\begin{displaymath}
 A=
\begin{pmatrix}
 1&1&1\\1&2&1\\1&2&3
\end{pmatrix}
\end{displaymath}
Montrer que $A$ est inversible, préciser son inverse. La matrice $A$ est-elle semblable à son inverse ?
\item Soit $f\in \mathcal L(E)$ telle que $f^3=f\circ f\circ f = 0_{\mathcal L(E)}$. On pose $g = -f + f^2=-f+f\circ f$.
\begin{enumerate}
  \item On suppose $f\neq 0_{\mathcal L(E)}$ et $f^2= 0_{\mathcal L(E)}$.\\ Montrer que $\rg(f)=1$ et qu'il existe une base $\mathcal A = (a_1,a_2,a_3)$ de $E$ telle que
\begin{displaymath}
 \mathop{\mathrm{Mat}}_{\mathcal A}f
=
\begin{pmatrix}
 0&0&1\\0&0&0\\0&0&0
\end{pmatrix}
\end{displaymath}
  \item On suppose $f^2\neq 0_{\mathcal L(E)}$.\\ Montrer qu'il existe une base $\mathcal A = (a_1,a_2,a_3)$ de $E$ telle que
\begin{displaymath}
 \mathop{\mathrm{Mat}}_{\mathcal A}f
=
\begin{pmatrix}
 0&1&0\\0&0&1\\0&0&0
\end{pmatrix}
\end{displaymath}
\item Calculer 
$(\mathop{\mathrm{id}_E}+f)\circ  (\mathop{\mathrm{id}_E}+g)$.
 Que peut-on en déduire pour $\mathop{\mathrm{id}_E}+f$?
\item Montrer que $g^3= 0_{\mathcal L(E)}$ et que  $g^2=0_{\mathcal L(E)}$ si et seulement si $f^2=0_{\mathcal L(E)}$.
\end{enumerate}
\item On va montrer ici que toute matrice de la forme $I_3+N$ avec
\begin{displaymath}
N=
 \begin{pmatrix}
  0&\alpha&\gamma\\0&0&\beta\\0&0&0
 \end{pmatrix}
\text{ et }
(\alpha,\beta,\gamma)\in \R^3
\end{displaymath}
est semblable à son inverse.
\begin{enumerate}
 \item Montrer que $I_3+N$ est inversible. Calculer $N^2$ et $N^3$.
 \item Soit $\mathcal A$ une base de $E$. On \emph{définit} un endomorphisme $f\in \mathcal L(E)$ par :
\begin{displaymath}
 \mathop{\mathrm{Mat}}_{\mathcal A}f = N
\end{displaymath}
En utilisant $f$, montrer que $I_3+N$ est semblable à son inverse.
\end{enumerate}
\item Donner un exemple de matrice dans $\mathcal M_3(\R)$ qui est inversible et semblable à son inverse mais qui n'est pas semblable à une matrice de la forme de la question 4.
\end{enumerate}
 
