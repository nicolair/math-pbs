\begin{enumerate}
  \item Cet exercice mélange les dominos multiplicatifs et additifs (sommes et produits télescopiques). On exprime $1 - x_{k+1}$ comme un quotient de termes consécutifs, puis $p_kx_{k+1}$ comme une différence:
\begin{displaymath}
1-x_{k+1} = \frac{p_{k+1}}{p_k}\Rightarrow x_{k+1} = 1-\frac{p_{k+1}}{p_k} \Rightarrow p_kx_{k+1} = p_k -p_{k+1}
\end{displaymath}
Cela permet de simplifier la somme
\begin{displaymath}
x_1+p_{n+1} + \sum_{k=1}^np_kx_{k+1} = x_1 + p_{n+1} + p_1 - p_{n+1}
= x_1 + p_1 =x_1 + 1-x_1 = 1
\end{displaymath}

  \item On veut appliquer la relation de la question précédente. Pour $k$ entre $1$ et $n$, posons
\begin{displaymath}
  x_k = \frac{k-1}{n}
\end{displaymath}
On a alors $x_1=0$ et $x_{n+1}=1$ donc $p_{n+1}=0$. La relation devient
\begin{displaymath}
  \sum_{k=1}^np_kx_{k+1} = 1
\end{displaymath}
avec 
\begin{multline*}
  p_k = (1-x_1)(1-x_2)\cdots \hspace{0.5cm} (k \text{ facteurs})
   = \frac{n-0}{n} \frac{n-1}{n} \cdots \hspace{0.5cm} (k \text{ facteurs}) \\
   = \frac{\overset{k \text{ facteurs}}{\overbrace{n(n-1)\cdots}}}{n^k}
\end{multline*}
On en déduit
\begin{displaymath}
1=\sum_{k=1}^n \frac{\overset{k \text{ facteurs}}{\overbrace{n(n-1)\cdots}}}{n^k}x_{k+1}
= \sum_{k=1}^n \frac{\overset{k \text{ facteurs}}{\overbrace{n(n-1)\cdots}}}{n^k}\frac{k}{n} \\
\Rightarrow 
S = n
\end{displaymath}

\end{enumerate}
