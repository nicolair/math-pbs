\begin{enumerate}
 \item \begin{enumerate}
 \item La fonction continue $f_p : x\rightarrow -1 +x + \cdots +x^{p}$ est strictement croissante entre $0$ et $1$. Sa valeur en $0$ est $-1<0$, sa valeur en $1$ est $p-1>0$. Elle s'annule donc exactement une fois en un nombre noté $x_p \in ]0,1[$ (pour $p\geq 2$).
\item En multipliant l'équation $(E_p)$ par $1-x_p$, on obtient la relation demandée
\begin{displaymath}
 x_p(1-x_p^p)=1-x_p
\end{displaymath}
\item Par définition de $x_p$ : $f_{p+1}(x_p)=x_p^{p+1}>0$. Comme $f_{p+1}$ est strictement croisante ceci prouve que $x^{p+1}<x_p$. La suite $(x_p)_{p\in\N}$ est donc décroissante et minorée par $0$, elle converge. Notons $l$ sa limite.
\item Comme la suite est décroissante, on peut écrire (pour tout $p\geq 2$):
\begin{displaymath}
 0\leq x_p^p \leq x_2^p
\end{displaymath}
Ce qui prouve ($x_2\in ]0,1[$ et théorème d'encadrement) que $(x_p^p)_{p\in\N}$ converge vers 0. En utilisant la relation du b., ontrouve alors que $l = 1-l$ ce qui entraine que la limite $l$ de $x_p$ est $\frac{1}{2}$.
\end{enumerate}
\item \begin{enumerate}
 \item Question de cours
 \item On trouve en remplaçant dans la formule du 1.b. :
\begin{displaymath}
 \varepsilon_p = x_p^{p+1}
\end{displaymath}
\item D'après 1. la suite $(\varepsilon_p)_{p\in\N}\rightarrow 0$. On peut en déduire
\begin{displaymath}
 (p+1)\ln (1+\varepsilon_p) \sim (p+1)\varepsilon_p = (p+1)x_p^{p+1}
\end{displaymath}
 et 
\begin{displaymath}
 0 \leq (p+1)x_p^{p+1} \leq (p+1)x_2^{p+1}
\end{displaymath}
On conclut alors avec la question 2.a. et le théorème d'encadrement.
\item Comme
\begin{align*}
 x_p=\dfrac{1+\varepsilon_p}{2} &\text{ et } \varepsilon_p = x_p^{p+1}
\end{align*}
on peut écrire :
\begin{displaymath}
 \varepsilon_p = \dfrac{1}{2^{p+1} (1+\varepsilon_p)^{p+1}} = \dfrac{1}{2^{p+1}}e^{(p+1)\ln (1+\varepsilon_p)}
\end{displaymath}
D'après 2.c., l'exponentielle à droite tend vers $0$ donc :
\begin{displaymath}
 (\varepsilon_p)_{p\in\N} \sim (\dfrac{1}{2^{p+1}})_{p\in\N}
\end{displaymath}
\end{enumerate}
\item \begin{enumerate}
 \item Par définition de $\alpha$ : $f(\alpha)=\alpha$.
\item La fonction $f$ est clairement décroissante et continue donc :
\begin{displaymath}
 f([\dfrac{1}{2},1]) = [f(1),f(\dfrac{1}{2})]=[\dfrac{1}{2},\dfrac{2}{3}]\subset [\dfrac{1}{2},1]
\end{displaymath}
\item Comme d'après la question précédente l'intervalle est stable, la suite est bien définie avec $u_n\in [\dfrac{1}{2},1]$ pour tous les $n$. On a vu au début que la racine $\alpha$ était elle aussi dans cet intervalle. On peut donc appliquer \emph{l'inégalité des accroissements finis} à $f$ entre $u_n$ et $\alpha$. La valeur maximale de la valeur absolue de la dérivée dans l'intervalle est obtenue en $\dfrac{1}{2}$. Elle vaut 
\begin{displaymath}
 (\dfrac{2}{3})^2
\end{displaymath}
On en déduit l'inégalité.
\item L'inégalité de la question précédente et le théorème d'encadrement montrent que
\begin{displaymath}
 (\varepsilon_p)_{p\in\N} \rightarrow \alpha
\end{displaymath}
\end{enumerate}
\item \begin{enumerate}
 \item La dérivée de $g$ s'annule seulement en $-\frac{1}{2}$, l'étude de son signe montrer que la fonction $g$ est décroissante dans $\R_+$.
\item D'après la question précédente :
\begin{displaymath}
 g([0,1])=[g(1),g(0)]=[\dfrac{1}{3},1]
\end{displaymath}
L'intervalle $[0,1]$ est donc stable par $g$ et toute suite dont le premier terme est dans cet intervalle est bien définie par récurrence avec $g$. Toutes ses valeurs restent dans l'intervalle.
\item Les deux suites extraites pour les indices pairs et impairs sont monotones car $g\circ g$ est \emph{croissante} comme composée de deux fonctions décroissantes. Comme elles restent dans l'intervalle bornée $[0,1]$, elles convergent.\newline
De plus, 
\begin{displaymath}
 v_0=1 , v_1=\dfrac{1}{3}, v_2=\dfrac{1}{\dfrac{1}{9}+\dfrac{1}{3}+1}<1
\end{displaymath}
On en déduit que la suite extraite pour les indices pairs est décroissante. En calculant numériquement $v_3$ on trouve que la suite extraite pour les indices impairs est croissante. On pourrait aussi considérer le terme \emph{avant} $u_0$ qui doit être $0$ et le comparer à $v_1$.
\item Par définition $g(u_{2n})=u_{2n+1}$, comme les suites convergent et que $g$ est continue en $l$. On obtient $g(l)=l^\prime$. Le raisonnement est identique pour l'autre relation.
\item On peut former l'équation $g\circ g (t)=t$ dont $l$ est une solution d'après la question précédente. Après réduction au même dénominateur, on obtient
\begin{displaymath}
 \frac{(t^2+t+1)^2}{1+(t^2+t+1)+(t^2+t+1)^2} = t
\Leftrightarrow
0=t^5 + t^4 +2t^3 +t -1
\end{displaymath}
En développant la relation donnée par l'énoncé, on trouve qu'elle égale à celle que l'on vient de trouver. 
\item La deuxième limite $l'$ vérifie la même relation que $l$. L'équation du dessus admet \emph{une seule racine réelle} car le facteur du troisième degré est l'équation $(E_3)$ du début de l'énoncé.\newline 
L'unique racine est donc $l=l^\prime = x_3$ ce qui assure la convergence.

\end{enumerate}

\end{enumerate}