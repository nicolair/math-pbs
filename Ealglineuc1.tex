%<dscrpt>Algèbre linéaire euclidienne : puissances, commutant.</dscrpt>
Dans tout le problème\footnote{d'après Mines Albi,Alès,... 1998 MPSI}, l'espace euclidien $\R^3$ est muni de sa structure euclidienne usuelle et rapportée à sa base canonique (orthonormée) notée $(e_1,e_2,e_3)$.
\subsubsection*{Partie I}
Soit $s$ l'endomorphisme de $\R^3$ de matrice
\begin{displaymath}
% use packages: array
S = \frac{1}{3}\left( \begin{array}{ccc}
5 & -1 & -1 \\ 
-1 & 5 & -1 \\ 
-1 & -1 & 5
    \end{array}\right) 
\end{displaymath}
dans la base canonique.
\begin{enumerate}
 \item Montrer que $s$ est un automorphisme de $\R^3$.
\item Soient $e^\prime _1=(1,1,1)$, $e^\prime _2=(1,-1,0)$, $e^\prime _3=(1,1,-2)$.
\begin{enumerate}
 \item Montrer que $(e^\prime _1, e^\prime _2, e^\prime _3)$ est une base de $\R^3$.
\item Déterminer la matrice $S^\prime$ de $s$ dans la base $(e^\prime _1, e^\prime _2, e^\prime _3)$.
\item Pour $n\in \N$, calculer ${S^{\prime}}^n$ et donner une méthode pour calculer $S^n$. (on ne demande pas d'effectuer les calculs)
\end{enumerate}
\item \begin{enumerate}
 \item La famille $(I_3,S)$ est-elle libre dans $\mathcal{M}_3(\R)$ ?
\item Montrer que $S^2$ peut s'exprimer comme combinaison linéaire de $I_3$ et $S$.
\item En déduire que pour tout $n\in \N$, il existe un unique couple $(a_n,b_n)$ de réels tels que
\[S^n=a_nI_3+b_nS\]
\item Donner les valeurs de $a_0$, $b_0$, $a_1$, $b_1$ et exprimer $a_{n+1}$ et $b_{n+1}$ en fonction de $a_n$ et $b_n$.
\item Montrer que la suite $(a_n+b_n)_{n\in \N}$ est constante et que la suite $(b_n+1)_{n\in \N}$ est géométrique. En déduire l'expression de $a_n$ et $b_n$ pour tous les $n$.
\end{enumerate}
\item Soit $B=S-2I_3$.
\begin{enumerate}
 \item Calculer $B^n$ pour $n\in \N$. En déduire l'expression de $S^n$ en fonction de $I_3$ et $B$.
\item Comparer avec le résultat de la question 3.
\end{enumerate}
\item L'expression de $S^n$ obtenue aux questions 3. et 4. est-elle valable pour $n\in \Z$?
\end{enumerate}
\subsubsection*{Partie II}
Soit $f$ l'endomorphisme de $\R^3$ de matrice
\begin{displaymath}
% use packages: array
A = \frac{1}{3}\left( \begin{array}{ccc}
-1 & -1 & 5 \\ 
5 & -1 & -1 \\ 
-1 & 5 & -1
    \end{array}\right) 
\end{displaymath}
dans la base canonique. On pose
\[u=f\circ s^{-1}\]
et on note $U$ la matrice de $u$ dans la base canonique.
\begin{enumerate}
 \item Calculer $U$, vérifier que $u$ est un automorphisme orthogonal et que 
\[u\circ s= s\circ u=f\]
\item Soit $(e^{\prime\prime}_1, e^{\prime\prime}_2, e^{\prime\prime}_3)$ la famille obtenue en normant les vecteurs $(e^\prime _1, e^\prime _2, e^\prime _3)$ de la question 2. de la première partie.
\begin{enumerate}
 \item Montrer que $(e^{\prime\prime}_1, e^{\prime\prime}_2, e^{\prime\prime}_3)$ est une base orthonormale.
\item \'Ecrire la matrice $U^\prime$ de $u$ dans cette base.
\end{enumerate}
\item \begin{enumerate}
 \item Exprimer la matrice de $s$ dans la base $(e^{\prime\prime}_1, e^{\prime\prime}_2, e^{\prime\prime}_3)$ en fonction de $S'$.
\item En déduire la matrice de $f$ dans la base $(e^{\prime\prime}_1, e^{\prime\prime}_2, e^{\prime\prime}_3)$.
\end{enumerate}
\item \begin{enumerate}
 \item Quel est l'ensemble des vecteurs invariants par $f$ ?
\item Soit $P=\Vect(e^{\prime\prime}_2, e^{\prime\prime}_3))$. Montrer que $f(P)=P$. Soit $g$ l'endomorphisme de $P$ tel que $g(x)=f(x)$ pour tout $x$ de $P$. Montrer que $g$ est la composée de deux applications linéaires simples que l'on précisera.
\end{enumerate}
\item On note $\mathcal{C}(f)$ l'ensemble des endomorphismes de $\R^3$ commutant avec $f$. C'est à dire l'ensemble des endomorphismes $g$ tels que $g\circ f= f\circ g$.
\begin{enumerate}
 \item Montrer que $\mathcal{C}(f)$ est une sous-algèbre de $\mathcal{L}(\R^3)$.
\item Soit $g\in\mathcal{C}(f)$.
\begin{enumerate}
 \item Montrer que le vecteur $g(e^{\prime\prime}_1)$ est invariant par $f$. Que peut-on en déduire ?
\item Soit $M$ la matrice de $g$ dans la base $(e^{\prime\prime}_1, e^{\prime\prime}_2, e^{\prime\prime}_3)$. Montrer que $M$ commute avec ${S^\prime}^3$.
\item En déduire la forme générale de la matrice d'un endomorphisme de $\mathcal{C}(f)$ dans la base $(e^{\prime\prime}_1, e^{\prime\prime}_2, e^{\prime\prime}_3)$.
\end{enumerate}
\item Quelle est la dimension de l'espace vectoriel $\mathcal{C}(f)$ ?
\end{enumerate}

\end{enumerate}

