%<dscrpt>Transformations d'équations différentielles.</dscrpt>
\subsubsection*{Partie I.}
Soit $A$ un nombre réel et $a$ la fonction définie dans $I=]0,+\infty[$ par :
\begin{displaymath}
 a(x)=\dfrac{A}{4x^2}
\end{displaymath}
On considère l'équation différentielle
\begin{align*}
 y''+ay = 0 & & (1)
\end{align*}
d'inconnue une fonction $y$ définie dans $I$.
\begin{enumerate}
 \item Former une équation différentielle, qui sera notée $(2)$, d'inconnue $y$ telle que :
\begin{displaymath}
 z \text{ solution de } (1) \Rightarrow t\rightarrow z(e^t)e^{-\frac{t}{2}} \text{ solution de } (2)
\end{displaymath}
On pourra poser 
\begin{displaymath}
 u(t)=z(e^t)e^{-\frac{t}{2}}
\end{displaymath}
 et chercher à exprimer successivement $z(e^t)$, $z'(e^t)$, $z''(e^t)$ en fonction de $u(t)$, $u'(t)$, $u''(t)$ et de fonctions exponentielles.
\item En discutant suivant le paramètre $A$, préciser l'ensemble des solutions de l'équation $(3)$ :
\begin{displaymath}
 y'' + \dfrac{A-1}{4}y = 0
\end{displaymath}
\item En discutant suivant le paramètre $A$, préciser l'ensemble des solutions de l'équation $(1)$.
\end{enumerate}
\subsubsection*{Partie II.}
Soient $p$ et $q$ des fonctions continues de $I=]0,+\infty[$ et à valeurs réelles. Soit $P$ la primitive de $p$ nulle en $1$ (aucune expression intégrale n'est nécessaire dans cet exercice). On considère l'équation différentielle $(1)$ dont l'inconnue est une fonction $y$ définie dans $I$:
\begin{align*}
 y'' + py'+qy = 0 & & (1)
\end{align*}
\begin{enumerate}
 \item Exprimer, à l'aide de $p$, $p'$, $q$, une fonction $v$ définie dans $I$ et à valeurs réelles telle que
\begin{displaymath}
 z \text{ solution de } (1) \Rightarrow x\rightarrow z(x)e^{\frac{1}{2}P(x)} \text{ solution de } (2)
\end{displaymath}
où $(2)$ est l'équation différentielle
\begin{align*}
 y'' + vy= 0 
\end{align*}
\item Former, l'équation différentielle $(2)$ associée à $(1)$ comme dans la question précédente dans le cas particulier
\begin{align*}
 p(x)=\dfrac{1}{x} & & q(x)=1-\dfrac{\lambda^2}{x^2} & & \lambda\in \R
\end{align*}
On trouvera une fonction $v$ de la forme 
\begin{displaymath}
 v(x) = 1 + \dfrac{A}{4x^2}
\end{displaymath}
pour un nombre réel $A$ à préciser.

\end{enumerate}
