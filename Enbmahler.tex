%<dscrpt>Nombre de Mahler.</dscrpt>
L'objet de ce problème\footnote{d'après \emph{Making Transcendance Transparent} E.B. Burger \& R. Tubbs (Springer)} est le nombre de Mahler\footnote{connu aussi sous le nom de nombre de \emph{Champernowne}} dont le développement décimal est obtenu en plaçant bout à bout l'écriture décimale de chaque entier naturel non nul. Ce nombre est noté $\mathcal{M}$
\begin{displaymath}
 \mathcal{M} = 0.123456789101112131415\cdots
\end{displaymath}
Les notations suivantes sont valables dans tout le problème. Pour tout $k\in \N^*$:
\begin{align*}
 &d_k = 9k\,10^{k-1} = k\left( 10^k- 10^{k-1}\right) \\
 &D_k = d_1 + d_2 + \cdots + d_k\\
 &m_k = \sum _{i=10^{k-1}}^{10^k -1}i\,10^{k(10^k - i -1)}\\
 &\mu_k = m_1\,10^{^{-D_1}} + m_2\,10^{^{-D_2}} + \cdots + m_k\,10^{^{-D_k}} 
\end{align*}

\subsection*{Question préliminaire}
 Soit $n$ un entier naturel non nul et $x$ un réel qui n'est pas égal à $1$. Montrer que
\begin{displaymath}
 \sum_{j=1}^{n}jx^{^{j-1}} = x^n\frac{nx-n-1}{(x-1)^2} + \frac{1}{(x-1)^2}
\end{displaymath}


\subsection*{Partie I. Autour des $d_k$.}
\begin{enumerate}
 \item Quel est le plus grand entier à $k$ chiffres en écriture décimale? Quel est le nombre d'entiers non nuls ayant exactement $k$ chiffres?
 \item Présenter dans un tableau les valeurs de $d_k$ et $D_k$ pour $k=1,2,3$.
 \item Montrer que $D_k = (k-\frac{1}{9})10^k+\frac{1}{9}$.
 \item Montrer que $D_k = 10\,D_{k-1} + 10^k -1$ pour $k\geq 2$.
\end{enumerate}


\subsection*{Partie II. Autour des $m_k$.}
Dans toute cette partie, $k$ désigne un élément non nul de $\N$.
\begin{enumerate}
 \item
\begin{enumerate}
 \item Montrer que
\begin{displaymath}
 \sum_{i=10^{k-1}}^{10^k -1}10^{k(10^k -i)} = \frac{10^k}{10^k -1}\left( 10^{d_k} - 1\right) 
\end{displaymath}
 \item Montrer que $10^{d_k -1}< m_k$.
 \item Montrer que
\begin{displaymath}
 m_k < \frac{10^k - 1}{10^k}\sum_{i=10^{k-1}}^{10^k -1}10^{k(10^k -i)}
\end{displaymath}
En déduire $m_k < 10^{d_k}$.
\end{enumerate}

\item 
Montrer les écritures décimales suivantes :
\begin{align*}
 &m_1 = 123456789\\
 &m_2 = 101112\, \cdots \,97 98 99
\end{align*}
On admettra que cette forme est valable pour tous les $k$ c'est à dire que $m_k$ est le nombre dont l'écriture décimale est obtenue en plaçant de gauche à droite les écritures décimales de tous les nombres à $k$ chiffres.
\begin{align*}
 m_3 &= 100101102\, \cdots\, 997 998 999\\ &\vdots
\end{align*}
Quel est le nombre de chiffres dans l'écriture de $m_k$?

 \item 
\begin{enumerate}
 \item \'Ecrire l'expression de $m_k$ obtenue en posant $j = 10^k -i$ dans la somme le définissant.
 \item Montrer qu'il existe un entier $a_k$ et un rationnel $r_k\in \left] 0, \frac{10}{9} \right[ $ tels que
\begin{displaymath}
\forall k \geq 2,\hspace{0.5cm} m_k = 10^{d_k}\frac{a_k}{(10^k -1)^2} - r_k .
\end{displaymath}
\end{enumerate}

\end{enumerate}

\subsection*{Partie III. Autour des $\mu_k$.}
On considère la suite $\left( \mu_k\right) _{k\in \N^*}$. Il pourra être utile de remarquer que $\frac{10}{9}< 1 +\frac{2}{10}$.
\begin{enumerate}
 \item Soit $k\geq 2$ et $l>k$. Montrer que
\begin{displaymath}
 m_k\,10^{-D_k} + m_{k+1}\,10^{-D_{k+1}} + \cdots + m_l\,10^{-D_l} < \frac{10}{9}\,10^{-D_{k-1}}  
\end{displaymath}
\item Montrer que la suite $\left( \mu_k\right) _{k\in \N^*}$ est convergente et que, en notant $\mathcal M$ sa limite,
\begin{displaymath}
  \mathcal M - \mu_k \leq \frac{10}{9}\,10^{-D_{k}}
\end{displaymath}
En déduire $\mathcal M \leq  0.1234567902$.\newline
On admet que le nombre $\mathcal M$ ainsi défini est bien le nombre de Mahler indiqué au début de l'énoncé. 
\item Pour $k\geq 2$, on note $q_k=10^{D_{k-1}}(10^k-1)^2$. Montrer qu'il existe $p_k\in \N$ tel que
\begin{displaymath}
 \left\vert \mathcal{M} - \frac{p_k}{q_k}\right| \leq \frac{10}{9\times10^{D_k}}
\end{displaymath}
Montrer que, pour tout réel $\gamma < 10$,
\begin{displaymath}
 \left\vert \mathcal{M} - \frac{p_k}{q_k}\right| \leq \frac{1}{q_k^\gamma}
\end{displaymath}
\end{enumerate}

\subsection*{Partie IV. Le théorème de Liouville}
Dans cette partie, on identifie un polynôme avec sa fonction polynomiale associée.\newline
Soit $P$ un polynôme à coefficients dans $\Z$, de degré $d$, sans racine dans $\Q$ et admettant une racine réelle $\alpha$ (donc forcément irrationnelle).\newline
 On note 
$M = \sup_{[\alpha - 1, \alpha +1]} |P'| $
\begin{enumerate}
\item Pourquoi $M$ est-il strictement positif ? On pose $C = \frac{1}{M}$.
\item Montrer que $\left|P(\frac{p}{q})\right| \geq \frac{1}{q^d}$ pour tout $p\in \Z$ et pour tout $q\in\N^*$.

\item \emph{Théorème de Liouville}. Montrer que $|\alpha - \frac{p}{q}|\geq \frac{C}{q^d}$ pour tout $p\in \Z$ et pour tout $q\in\N^*$ tels que $|\alpha - \frac{p}{q}|\leq 1$.

\item On admet que $\mathcal{M}$ est irrationnel. Montrer qu'il n'est racine d'aucun polynôme à coefficients entiers et de degré inférieur ou égal à $9$.
\end{enumerate}