\subsection*{I. \'Etudes locales en 0.}
\begin{enumerate}
  \item La fonction se prolonge par continuité en $0$ car, strictement à droite en 0  $\frac{\sin x}{x}\rightarrow 1$. on a donc $f(0)=0$.
 
  \item Calcul du développement asymptotique
\begin{multline*}
\sin t= t - \frac{1}{6}t^3 + o(t^4) \Rightarrow
\frac{1}{\sin t} = \frac{1}{t}\,\frac{1}{1-\frac{1}{6}t^2 + o(t^3)}
= \frac{1}{t}\left(1+\frac{1}{6}t^2 + o(t^3) \right)\\
= \frac{1}{t} +\frac{1}{6}t + o(t^2)
\end{multline*}

  \item On combine le développement usuel de $\cos$ avec le carré du précédent
\begin{displaymath}
\left. 
\begin{aligned}
&\cos t = 1 - \frac{1}{2}t^2 + o(t^3)\\
&\frac{1}{\sin^2 t} = \frac{1}{t^2} + \frac{1}{3} + o(t)
\end{aligned}
\right\rbrace 
\Rightarrow
\frac{\cos t}{\sin ^2 t}
= \frac{1}{t^2} +\left(\frac{1}{3}-\frac{1}{2} \right) + o(t)
= \frac{1}{t^2} -\frac{1}{6} + o(t)
\end{displaymath}

  \item Dans la question 2., strictement à droite de $0$, la fonction admet le développement limité
\begin{displaymath}
\frac{1}{t} - \frac{1}{\sin t} = \frac{1}{6}t + o(t^2)
\end{displaymath}
Elle converge donc vers $0$. On la prolonge en $0$ en lui assignant la limite comme image de $0$. On note $g$ la fonction ainsi prolongée.
\begin{displaymath}
  g:
\left\lbrace 
\begin{aligned}
  \left[ 0,\pi \right[ &\rightarrow \R \\
  t       &\mapsto
      \left\lbrace 
        \begin{aligned}
          &0                              &\text{ si } t = 0\\
          &\frac{1}{t} - \frac{1}{\sin t} &\text{ si } t\in ]0,\pi[
        \end{aligned}
      \right. 
\end{aligned}
\right. 
\end{displaymath}
D'après les résultats sur les compositions de fonctions, cette fonction est de classe $\mathcal{C}^{\infty}$ sur $]0,\pi[$ avec, en tenant compte des développements limités déjà calculés,
\begin{displaymath}
\forall x \in]0,\pi[,\;
g'(t) = -\frac{1}{t^2} + \frac{\cos t}{\sin^2t} = -\frac{1}{6} + o(t)
\end{displaymath}
On en déduit que $g'$ converge vers $-\frac{1}{6}$ en $0$ strictement à droite. Le théorème de la limite de la dérivée assure alors que $g$ est dérivable en $0$ et que $g'$ est continue en $0$. Cela montre que $g$ et de classe $\mathcal{C}^1$ sur $[0,\pi[$.\newline
On pourrait se passer du théorème de la limite de la dérivée pour justifier la dérivabilité en $0$ en remarquant que $g$ admet en $0$ un développement à l'ordre 1.
\end{enumerate}

\subsection*{II. Calcul de l'intégrale de Dirichlet.}
\begin{enumerate}
  \item Comme $\varphi$ est de classe $\mathcal{C}^1$, on peut effectuer une intégration par parties puis une majoration en utilisant le fait que la fonction continue $\varphi'$ est bornée et atteint ses bornes sur le segment $[0,\frac{\pi}{2}]$.
\begin{multline*}
\int_{0}^{\frac{\pi}{2}}\varphi(t)\sin(nt)\,dt
= \left[\varphi(t)\left(-\frac{1}{n}\cos (nt) \right)  \right]_{t=0}^{t=\frac{\pi}{2}}
+ \frac{1}{n}\int_0^{\frac{\pi}{2}}\varphi'(t)\cos(nt)\, dt \\
\Rightarrow
\left|\int_{0}^{\frac{\pi}{2}}\varphi(t)\sin(nt)\,dt\right|
\leq \frac{|\varphi(\frac{\pi}{2})|+|\varphi(0)|}{n}
+\frac{1}{n}\frac{\pi}{2}M_1
\text{ avec } M_1 = \max_{[0,\frac{\pi}{2}]}|\varphi'|
\end{multline*}
On en déduit que la suite tend vers $0$ à l'aide du théorème d'encadrement.

  \item 
\begin{enumerate}
  \item Il s'agit d'une formule usuelle le linéarisation (transformation de produit en somme)
\begin{displaymath}
2\sin(a) \cos(b) = \sin(a+b) + \sin(a-b)  
\end{displaymath}

  \item Notons $C$ la somme qui nous est proposée et 
\begin{displaymath}
S = 2\sin(2t) + 2\sin(4t) + \cdots + 2\sin(2nt)  
\end{displaymath}
En complexifiant, on fait apparaître une somme de puissances de $e^{2it}$.
\begin{multline*}
C+iS = 1 + 2 \, \frac{e^{2(n+1)it} - e^{2it}}{e^{2it} -1}
= 1 + 2 \,\frac{\sin(nt)}{\sin(t)}e^{(n+1)it} \\
\Rightarrow C = \Re(C+iS) = 1 +2\frac{\sin(nt)\cos((n+1)t)}{\sin(t)} = \frac{\sin((2n+1)t)}{\sin(t)}
\end{multline*}
car, d'après la question a.
\begin{displaymath}
2\sin(nt)\cos((n+1)t) = \sin((2n+1)t) - \sin(t)  
\end{displaymath}

\end{enumerate}

  \item D'après la question précédente et par linéarité de l'intégrale,
\begin{displaymath}
\int_0^{\frac{\pi}{2}}\frac{\sin((2n+1)t)}{\sin(t)}\, dt
= \int_0^{\frac{\pi}{2}}\left(1+2\cos(2t)+2\cos(4t)+\cdots+2\cos(2nt) \right)\,dt 
= \frac{\pi}{2}
\end{displaymath}
Chaque intégrale contenant un $\cos$ est nulle car les primitives en $\sin$ s'annulent en $0$ et $\frac{\pi}{2}$ à cause du coefficient multiplicatif pair.

  \item Effectuons le changement de variable $u=(2n+1)t$ dans l'expression intégrale de $F((2n+1)\frac{\pi}{2})$. Alors $\frac{dt}{t} = \frac{du}{u}$ et
\begin{displaymath}
F((2n+1)\frac{\pi}{2})
=\int_0^{(2n+1)\frac{\pi}{2}}\frac{\sin u}{u} \, du
=\int_0^{\frac{\pi}{2}}\frac{\sin((2n+1)t)}{t}\, dt
\end{displaymath}
Introduisons la fonction $g$ de classe $\mathcal{C}^1$ sur $[0,\pi[$. Pour $t$ non nul, elle vérifie
\begin{displaymath}
  g(t) = \frac{1}{t} - \frac{1}{\sin t} 
\end{displaymath}
On peut écrire
\begin{multline*}
F((2n+1)\frac{\pi}{2}) = \int_0^{\frac{\pi}{2}}\left(\frac{1}{t}-\frac{1}{\sin t}+\frac{1}{\sin t}\right) \sin((2n+1)t)\, dt \\
= \int_0^{\frac{\pi}{2}}g(t)\sin((2n+1)t)\, dt 
+ \underset{=\frac{\pi}{2}}{\underbrace{\int_0^{\frac{\pi}{2}}\frac{\sin((2n+1)t)}{\sin(t)}\, dt}}
\end{multline*}
Comme $g$ est de classe $\mathcal{C}^1$, la première suite converge vers $0$ d'après le lemme de Riemann-Lebegue (question II.1.). La deuxième est constante (question II.3). On en déduit
\begin{displaymath}
  \left( F\left( (2n+1)\frac{\pi}{2}\right) \right)_{n\in \N} \rightarrow \frac{\pi}{2}
\end{displaymath}

  \item
\begin{enumerate}
\item Par définition de $n$ comme partie entière de $\frac{x}{\pi}$,
\begin{displaymath}
  n\pi \leq x < (n+1)\pi
\end{displaymath}
Le réel $(2n+1)\frac{\pi}{2}=\frac{\pi}{2} + n\pi$ est aussi un élément de l'intervalle $[n\pi, (n+1)\pi[$. Dans cet intervalle, le $\sin$ garde un signe constant. On en déduit
\begin{multline*}
  F(x) - F(\frac{\pi}{2} + n\pi) = 
\int_{\frac{\pi}{2} + n\pi}^x \frac{\sin(t)}{t}\,dt\\
\Rightarrow
\left|F(x) - F(\frac{\pi}{2} + n\pi) \right|
\leq
\int_{\overleftrightarrow{[\frac{\pi}{2} + n\pi,x]}}\frac{|\sin(t)|}{t}\, dt
\leq
\int_{n\pi}^{(n+1)\pi}\frac{dt}{t} \leq \frac{\pi}{n \pi} = \frac{1}{n}
\end{multline*}

\item Pour tout $\varepsilon >0$, d'après la question II.4., il existe un entier $N$ tel que 
\begin{displaymath}
  \forall n \geq N,\; \left|F((2n+1)\frac{\pi}{2}) - \frac{\pi}{2}\right| < \frac{\varepsilon}{2} \text{ et } \frac{1}{n} < \frac{\varepsilon}{2}
\end{displaymath}
Pour $x> N\pi$, le $n$ associé à $x$ est plus grand que $N$. On en déduit
\begin{displaymath}
\left|F(x) - \frac{\pi}{2}\right| \leq  \left|F(x) - F(\frac{\pi}{2} + n\pi) \right| + \left|F((2n+1)\frac{\pi}{2}) - \frac{\pi}{2}\right|
\leq \frac{\varepsilon}{2} + \frac{\varepsilon}{2} = \varepsilon
\end{displaymath}
Ceci assure que la limite en $+\infty$ de $F$ (intégrale de Dirichlet) est $\frac{\pi}{2}$.
\end{enumerate}
\end{enumerate}


\subsection*{IV. \'Equation différentielle.}
\begin{enumerate}
  \item
\begin{enumerate}
  \item Si $P$ est un polynôme non nul, son degré est le même que celui de $P+P''$. Par conséquent, si l'équation admet une solution polynomiale, celle ci doit être de degré $n$.\newline
  Considérons l'application $\varphi$ de $\R_n[X]$ dans lui même qui à un polynôme $P$ associe $P+P''$. Cette application est clairement linéaire. C'est un endomorphisme de l'espace de dimension finie $\R_n[X]$. La considération du terme de plus haut degré montre que son noyau est réduit au polynôme nul. Il s'agit donc d'un endomorphisme injectif, ce qui, en dimension finie entraîne que $\varphi$ est bijectif.\newline
  En particulier, le polynôme $X^n$ admet un unique antécédent par $\varphi$ qui est l'unique solution polynomiale de l'équation. On note $A_n$ cette unique solution polynomiale.
  \item L'ensemble des solutions est le sous-espace affine (de l'espace de toutes les fonctions) contenant $A_n$ et de direction l'espace vectoriel engendré par $\sin$ et $\cos$ qui forment une base de l'espace des solutions de l'équation homogène. 
  \item Notons $A_n=\sum_{k=0}^{n} a_kX^k$. On peut écrire 
\begin{multline*} 
A_n+A''_n = 
\sum_{k=0}^{n} a_kX^k + \sum_{k=0}^{n} k(k-1)a_kX^k \\
= a_nX^n+a_{n-1}X^{n-1}  +\sum_{k=0}^{n-2}(a_k+(k+2)(k+1)a_{k+2})X^k = X^n 
\end{multline*}
Par suite $a_n=1$, $a_{n-1}=0$ et
\begin{displaymath}
\forall k \in\llbracket 0, n-2\rrbracket,\; a_k=-(k+2)(k+1)a_{k+2}   
\end{displaymath} 
Finalement  
\begin{displaymath}
\forall i \in\llbracket 0, \lfloor \frac{n}{2}\rfloor:\, a_{n-2i}=  (-1)^n \frac{n!}{(n-2i)!}
;\hspace{1cm}
\forall i \in \llbracket 0, \lfloor \frac{n+1}{2}\rfloor:\;a_{n-2i-1}=0 .
\end{displaymath}  
\end{enumerate}

  \item
\begin{enumerate}
  \item On obtient $P_0=1$, $P_1=X$, $P_2= X^2-2$ et $Q_0=0$, $Q_1=1$, $Q_2=2X$ sans qu'une relation évidente ne se dégage.
  \item On raisonne par récurrence. Comme les premières propositions sont vérifiées, il suffit de montrer l'implication.
En d\'erivant la relation donn\'ee par l'\'enonc\'e, on obtient, pour $x>0$:
\begin{multline*}
f^{(n+1)} 
= \frac{P'_n\sin^{(n)}+P_n\sin^{(n+1)} +Q'_n\sin^{(n+1)} +Q_n\sin^{(n+2)}}{x^{n+1}} \\
 -(n+1)\frac{P_n\sin^{(n)}+ Q_n\sin^{(n+1)}}{x^{n+2}}
\end{multline*}
comme $\sin^{(n)} = -\sin^{(n+2)}$, cela donne :
\begin{displaymath}
g^{(n+1)}(x)=  \frac{P_{n+1}(x)\sin^{(n+1)}(x) + Q_{n+1}(x)\sin^{(n+2)}(x)}{x^{n+2}}  
\end{displaymath}
avec
\begin{displaymath}
\left\lbrace 
  \begin{aligned}
P_{n+1}(x) &= xP_n(x)+xQ'_n(x)-(n+1)Q_n(x) \\
Q_{n+1}(x) &= xQ_n(x)-xP'_n(x)+(n+1)P_n(x)
  \end{aligned}
\right. 
\end{displaymath}
\end{enumerate}

  \item
\begin{enumerate}
  \item Soit ${\cal H}_n$ la propri\'et\'e: 
\begin{quote}\og $P_n$ de degr\'e $n$ de coefficient dominant 1 , $Q_n$ de degr\'e $n-1$ de coefficient dominant $n$, $P_n$ et $Q_n$ à coefficients entiers\fg
\end{quote}
 \begin{itemize}
 \item La propriété ${\cal H}_1$ est vérifiée.
 \item Supposons ${\cal H}_n$, alors $P_n$, $Q_n$, $P'_{n}$ et $Q'_{n}$ sont à coefficients entiers donc $P_{n+1}$ et $Q_{n+1}$ aussi.\newline 
De plus $XP_n$ est de degr\'e $n+1$ de coefficient dominant $1$ et $XQ'_n$ et $Q_n$ sont de degr\'e strictement inf\'erieur à $n+1$ donc $P_{n+1}$ est de degr\'e $n+1$ de coefficient dominant $1$.\newline
Enfin  $XQ_n$, $XP'_n$ et $(n+1)P_n$ sont de degr\'e $n$ et de coefficients dominants respectifs $n$, $n$ et $n+1$ donc $Q_{n+1}$ est de degr\'e $n$ de coefficient dominant $n+1$.\newline
Donc ${\cal H}_{n+1}$ est vérifiée.
\end{itemize}
On d\'emontre de mani\`ere analogue que, pour tout entier $p$, $P_{2p}$ est pair, $Q_{2p}$ est impair, $P_{2p+1}$ est impair, $Q_{2p+1}$ est pair.

  \item D'après les relations de 2.b.
\begin{displaymath}
\left\lbrace  
\begin{aligned}
P_3 &= XP_2+XQ'_2-3Q_2=X^3-6X \\
Q_3 &= XQ_2-XP'_2+3P_2=3X^2-6
\end{aligned}
\right. 
\end{displaymath}
\end{enumerate}

  \item Soit $\alpha_k=  \frac{\pi}{2}+k\pi$ et $\beta_k= k\pi$. De la relation
\begin{displaymath}
\forall x>0,\; U(x)\sin(x)+V(x)\cos(x)=0   
\end{displaymath}
on tire $U(\alpha_k)=0$ et $V(\beta_k)=0$ pour tous les entiers $k$. Le polynômes $U$ et $V$ admettent une infinit\'e de racines donc ils sont \'egaux au polynôme nul.

  \item
\begin{enumerate}
  \item En d\'erivant $n+1$ fois l'\'egalit\'e $xf(x)=\sin x$ avec la formule de Leibniz, on obtient:
\begin{displaymath}
\forall x >0,\; xf^{(n+1)}(x)+ (n+1)f^{(n)}(x)=\sin^{(n+1)}(x)
\end{displaymath}
En insérant dans cette relation les expressions de $f^{(n)}(x)$ et $f^{(n+1)}(x)$, il vient:
\begin{multline*}
\left( P_{n+1}(x)+(n+1)Q_n(x)-x^{n+1}\right) \sin^{(n+1)}(x) \\+ \left( (n+1)P_n(x)-Q_{n+1}(x)\right) \sin^{(n)}(x)=0  
\end{multline*}
\`A $n$ fix\'e, l'une des expressions $\sin^{(n+1)}(x)$ ou $\sin^{(n)}(x)$ vaut $\pm \sin(x)$ tandis que l'autre vaut $\pm \cos x$, on peut donc appliquer le r\'esultat de la question pr\'ec\'edente:
\begin{displaymath}
\forall x\in \R, \;  P_{n+1}(x)+(n+1)Q_n(x)-x^{n+1} = 0, \hspace{0.5cm} (n+1)P_n(x)-Q_{n+1}(x)=0.
\end{displaymath}

  \item En reportant $Q_{n+1}(x)=(n+1)P_{n}(x)$ dans l'expression de $Q_{n+1}$ de la question 2.b., on tire $x(Q_n(x)-P'_n(x))=0$ pour tous les $x$. Cela entraîne $Q_n(x)=P'_n(x)$ pour tous les $x \neq 0$. Le polynôme $P'_n-Q_n$ a une infinit\'e de racines donc il est nul. \newline
On a donc 
\begin{multline*}
\forall x\in \R,\; P_{n+1}(x)= x^{n+1}-(n+1)Q_n(x)\\
= xP_n(x)+xP''_n(x)-(n+1)Q_n(x) \hspace{0.5cm}\text{ (question 2.b)}\\
\Rightarrow
x^{n+1} = xP_n(x)+xP''_n(x)
\end{multline*}
\end{enumerate}
On en tire la relation $P_n + P''_n = x^n$ d'abord pour les $x$ non nuls, puis pour tous les autres par un argument polynomial. 

\end{enumerate}
