%<dscrpt>Condition polynomiale</dscrpt>
On cherche à déterminer les polynômes $P\in \C[X]$ vérifiant
\begin{displaymath}
 (*)\hspace{1cm}\widehat{P}(X^2-1)=\widehat{P}(X-1)\widehat{P}(X+1)
\end{displaymath}
Soit $a_0\in \C$, on définit la suite $\left( a_n\right) _{n\in \N}$ de nombres complexes en posant:
\begin{displaymath}
 \forall n\in \N,\hspace{0.5cm}a_{n+1}=a_n^2+2a_n
\end{displaymath}
\begin{enumerate}
 \item Quels sont les polynômes dans $\C_0[X]$ vérifiant $(*)$?

 \item 
\begin{enumerate}
\item Montrer que si $a_0$ est un réel strictement positif alors la suite $\left( a_n\right) _{n\in \N}$ est strictement croissante. Quelle est sa limite ? Discuter du comportement de la suite suivant la valeur de $a_0$ réel.

\item Exprimer $a_n +1$ comme une puissance de $a_0+1$. En déduire le bassin d'attraction de $-1$ c'est à dire la partie $\Omega$ du plan complexe telle que $a_0\in\Omega$ entraine $\left( a_n\right) _{n\in \N}$ converge vers $-1$.
\end{enumerate}

 \item Soit $P$ un polynôme de degré supérieur ou égal à $1$ vérifiant $(*)$. Montrer que si $a$ est une racine de $P$ alors $(a+1)^2-1$ est aussi une racine de $P$. Que peut-on en déduire pour la suite $\left( a_n\right) _{n\in \N}$ ?.
 
 \item 
\begin{enumerate}
 \item Montrer que $P$ n'admet pas de racine réelle strictement positive.
 \item Montrer que $-1$ n'est pas racine de $P$.
 \item Montrer que si $a$ est une racine complexe de $P$ alors $|a+1|=1$. 
\end{enumerate}
 
\item Montrer que si $a$ est une racine complexe de $P$ alors $|a-1|=1$.

\item Déterminer tous les polynômes de $\C[X]$ vérifiant $(*)$.
\end{enumerate}
