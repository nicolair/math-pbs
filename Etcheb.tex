%<dscrpt>Des polynômes proches de ceux de Tchebychev.</dscrpt>
\begin{enumerate}
\item Soit $n\in \N$, montrer qu'il existe \emph{au plus} un polyn{\^o}me $P_n$ tel que
\begin{displaymath}
(1)\hspace{2cm}     \forall z \in \C, \, \widetilde{P_n}(z+\frac{1}{z})=z^n+\frac{1}{z^n}
\end{displaymath}
\item Pr{\'e}ciser les polyn{\^o}mes $P_0, P_1, P_2$.
\item Montrer que pour tout entier $n$, il existe un unique polyn{\^o}me $P_n$ v{\'e}rifiant (1). \newline
  On pourra consid{\'e}rer
  \[(z+\frac{1}{z})(z^n+\frac{1}{z^n})\]

\item
 \begin{enumerate}
 \item Montrer que pour tout entier $n$ et tout r{\'e}el $t$ :
\begin{displaymath}
\widetilde{P_n}(2\cos t)= 2 \cos(nt) 
\end{displaymath}

 \item Former, {\`a} partir de $P_n$ un polyn{\^o}me $T_n$ tel que pour tout r{\'e}el $t$
\begin{displaymath}
 \widetilde{T_n}(\cos t)= \cos (nt)
\end{displaymath}
 \end{enumerate}
\end{enumerate}
