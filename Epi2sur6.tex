%<dscrpt>Polynômes et suites. Limite de la suite des 1/k^2</dscrpt>
Pour tout entier naturel $n$, on d{\'e}finit le polyn{\^o}me $Q_{n}$ {\`a} coefficient complexes par\footnote{pour les origines de cette idée, voir le chapitre de \emph{Raisonnements divins} de M. Aigner et G.M. Ziegler (Springer)} :
\[
Q_{n}=\frac{1}{2i}\left( (X+i)^{n+1}-(X-i)^{n+1}\right).
\]

\begin{enumerate}
\item \begin{enumerate} 
\item  D{\'e}terminer le degr{\'e} de $Q_{n}$ et son coefficient dominant.
\item Quel est le polynôme obtenu en substituant $-X$ à $X$ dans $Q_n$?\newline
Que peut-on en déduire pour l'ensemble des racines de $Q_n$?
      \end{enumerate}

\item  Soit $r \in \N^*$ et $p\in \llbracket 0, r \rrbracket$. Préciser le coefficient de $X^{2r-2p}$ dans $Q_{2r}$ puis un polynôme $S_r\in \R[X]$ permettant d'écrire 
\begin{displaymath}
 Q_{2r}=\widehat{S_r}(X^2).
\end{displaymath}
Le chapeau traduit la substitution de $X$ par $X^2$ dans $S_r$.
\item  En utilisant l'ensemble $\U_{n+1}$ des racines $n+1$-ièmes de l'unité, déterminer les racines de $Q_{n}$ dans $\C$. En d{\'e}duire la d{\'e}composition de $Q_{n}$ en facteurs irr{\'e}ductibles de $\R\left[ X\right]$.

\item  Soit $r\in \N^*$. Prouver les {\'e}galit{\'e}s suivantes :
\begin{displaymath}
\sum_{k=1}^{r} \left( \cot\frac{k\pi }{2r+1}\right) ^{2} =\frac{r(2r-1)}{3}, \hspace{1cm}
\sum_{k=1}^{r}\frac{1}{\left( \sin \frac{k\pi }{2r+1}\right)^{2} } =\frac{2r(r+1)}{3}.
\end{displaymath}

\item  \'Etablir les in{\'e}galit{\'e}s
\begin{displaymath}
\forall x\in \left] 0,\frac{\pi }{2}\right[ :\quad \left( \cot x\right)^{2}  \leq \frac{1}{x^{2}}\leq \frac{1}{(\sin x)^{2}} . 
\end{displaymath}

\item  Soit $r\in \N^*$. D{\'e}duire de la question pr{\'e}c{\'e}dente un encadrement de
\begin{displaymath}
\sum_{k=1}^{r}\frac{1}{\left( \frac{k\pi }{2r+1}\right) ^{2}} . 
\end{displaymath}

\item  Pour tout entier naturel non nul $n$, on pose $S_{n}=\sum_{k=1}^{n}\frac{1}{k^{2}}$. Montrer la convergence de $(S_{n})_{n\in \N^{*}}$ et pr{\'e}ciser sa limite.
\end{enumerate}
