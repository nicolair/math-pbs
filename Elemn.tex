%<dscrpt>Lemniscate.</dscrpt>
On se place\footnote{d'après CCP TSI maths 1 2009} dans un plan $\mathcal P$ muni d'un repère orthonormé $(O,\vec{i},\vec{j})$.
Si $M$ et $N$ sont deux points du plan, on note $d(M,N)$ la distance de $M$ à $N$.\\
Soit $a$ un nombre réel strictement positif. Notons $A$ et $B$ les points de coordonnées respectives $(-a,0)$ et $(a,0)$.\\
Soit $k$ un nombre réel strictement positif.
\begin{enumerate}
\item Soit $M\neq O$ un point de coordonnées cartésiennes $(x,y)$ et de coordonnées polaires $(r,\theta)$ (avec $r>0$ et $\theta\in ]-\pi,\pi]$). On considère la courbe $(C)$ formée par l'ensemble des points $M$ tels que $$d(A,M)d(B,M)=k^2$$
\begin{enumerate}
\item Montrer que la courbe $(C)$ est symétrique par rapport à l'axe des abscisses et à l'axe des ordonnées.
\item Déterminer l'équation cartésienne de la courbe $(C)$.
\item Déterminer une équation polaire $(E)$ de la courbe $(C)$.
\end{enumerate}
\item Dans cette question on se limite à $\theta \in \left[ 0,\frac{\pi}{2} \right] $, ce qui revient à considérer $(\tilde{C})$ la partie de la courbe comprise dans le quart de plan correspondant à $x\se 0$ et $y\se 0$.
   
\begin{enumerate}
\item Pour $\theta$ fixé, montrer que $(E)$ est un trinôme en $R=r^2$ (de paramètres $a, k$ et $\theta$).
\item Déterminer les conditions nécessaires et suffisantes vérifiées par $a$, $k$ et 
$\theta$ pour que le trinôme en $R$ admette deux solutions strictement positives, éventuellement confondues (on ne demande pas de calculer les racines).\\
Montrer que ces conditions sont équivalentes à :
$$
  \left\lbrace 
    \begin{aligned}
       k &\ie a \\ 
       \theta &\in I_k
    \end{aligned}
 \right.
 $$
où $I_k$ est un intervalle à préciser.
\item Si  $k< a$, montrer que la courbe $(\tilde{C})$ est la réunion de deux courbes $(C_i)$ (pour $i\in\{1,2\}$) admettant respectivement une équation polaire de la forme $r=r_i(\theta)$ avec $r_i$ une fonction définie sur $I_k$. Déterminer les fonctions $r_1$ et $r_2$.
\item Si $k>a$, déterminer le nombre de points d'intersections d'une droite passant par 
l'origine avec la courbe $(\tilde{C})$.
\end{enumerate}
\item
\begin{enumerate}
\item Vérifier que dans le cas particulier où $k= a$, une équation polaire de la courbe $(C)$ est $r=a\sqrt{2\cos(2\theta)}$.
\item Dans le cas où $k= a$ et $a=\frac{\sqrt{2}}{2}$. Montrer que la vitesse au point de paramètre $\theta$ est $$\frac{1}{\sqrt{\cos 2\theta}}\overrightarrow{e_{\alpha}} \text{ où }
\overrightarrow{e_{\alpha}}=\cos(\alpha)\overrightarrow{i} +\sin(\alpha)\overrightarrow{j}
$$
et $\alpha$ est à déterminer.\\  \'Etudier et tracer la courbe $(C)$ 
\end{enumerate}

\end{enumerate}

