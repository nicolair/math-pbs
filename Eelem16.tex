%<dscrpt>Discussion et étude de fonction.</dscrpt>
Soit $m$ un réel donné. L'objet de ce problème est de discuter suivant la valeur du paramètre $m$ du nombre de solutions dans $[0, \pi]$ de l'équation $\mathcal{E}_m$ d'inconnue $x$
\begin{displaymath}
  \mathcal{E}_m:\hspace{0.5cm} \cos(2x) + 2(1-m)\cos x +1 +4m = 0 
\end{displaymath}
On introduit pour cela une fonction auxiliaire
\begin{displaymath}
  \varphi:\;
\left\lbrace 
\begin{aligned}
  \R \setminus\{2\} &\rightarrow \R \\
  t &\mapsto \frac{t^2 + t}{t-2}
\end{aligned}
\right. 
\end{displaymath}

\begin{enumerate}
  \item Calculer et factoriser $\varphi'$. Former le tableau de variations de $\varphi$ et tracer son graphe.
  \item Déterminer un intervalle $I$ tel que
  \begin{displaymath}
    \mathcal{E}_m \text{ admet une solution dans $\R$} \Leftrightarrow m \in \varphi(I)
  \end{displaymath}
  \item Discuter suivant $m$ du nombre de solutions de $\mathcal{E}_m$ dans $[0,\pi]$.
\end{enumerate}
