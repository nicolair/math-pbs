%<dscrpt>Systèmes d'équations différentielles.</dscrpt>
\subsection*{Partie I.}
On considère un système différentiel $(\mathcal S_1)$ dont les inconnues sont des fonctions $x$ et $y$ dérivables dans $\R$ et une équation différentielle $(E_1)$ dont la fonction inconnue est deux fois dérivable dans $\R$.
\begin{align*}
 (\mathcal S_1)\hspace{0.5cm} 
\left\lbrace 
\begin{aligned}
 x'(t) &= 2y(t)\\
 y'(t) &= -x(t) +te^t
\end{aligned}
\right. 
& &
(E_1)\hspace{0.5cm} y''(t)+2y(t) = 2te^t
\end{align*}
\begin{enumerate}
 \item Soit $(f_1,f_2)$ un couple de solutions de $(\mathcal S_1)$. Montrer que $f_1$ est une solution de $(E_1)$.
 \item Déterminer l'ensemble des solutions de $(E_1)$.
 \item Déterminer l'ensemble des couples solutions de $(\mathcal S_1)$.
 \item Montrer qu'il existe un unique couple solution $(f_1, g_1)$ de $(\mathcal S_1)$ tel que $f_1(0)=g_1(0)=0$. 
\end{enumerate}

\subsection*{Partie II.}
On considère un système différentiel $(\mathcal S_2)$ dont les inconnues sont des fonctions $x$ et $y$ dérivables dans $\R$.
\begin{displaymath}
 (\mathcal S_2)\hspace{0.5cm} 
\left\lbrace 
\begin{aligned}
 x'(t) &= \frac{4t}{1+t^2}x(t)-t^2y(t)\\
 y'(t) &= \frac{2t}{1+t^2}y(t) +t(1+t^2)
\end{aligned}
\right. 
\end{displaymath}
\begin{enumerate}
 \item Déterminer l'ensemble des couples solutions de $(\mathcal S_2)$.
 \item Montrer qu'il existe un unique couple solution $(f_2, g_2)$ de $(\mathcal S_2)$ tel que $f_2(0)=g_2(0)=0$. 
\end{enumerate}

\subsection*{Partie III.}
On considère un système différentiel $(\mathcal S_3)$ dont les inconnues sont des fonctions $x$ et $y$ dérivables dans $\R$.
\begin{displaymath}
 (\mathcal S_3)\hspace{0.5cm} 
\left\lbrace 
\begin{aligned}
 x'(t) &= 7x(t)-5y(t)+\sh(t)\\
 y'(t) &= 10x(t)-8y(t)+\ch(t)
\end{aligned}
\right. 
\end{displaymath}
\begin{enumerate}
 \item Soit $(f,g)$ un couple solution de $(\mathcal S_3)$ et $u=2f-g$. Former une équation différentielle du premier ordre $(E_2)$ (inconnue notée $y$) dont $u$ est une solution.
 \item Soit $(f,g)$ un couple solution de $(\mathcal S_3)$ et $u=-f+g$. Former une équation différentielle du premier ordre $(E_3)$ (inconnue notée $y$) dont $u$ est une solution.
 \item Résoudre $(E_2)$. Résoudre $(E_3)$.
 \item Déterminer l'ensemble des couples solutions de $(\mathcal S_3)$.
 \item Montrer qu'il existe un unique couple solution $(f_3, g_3)$ de $(\mathcal S_3)$ tel que $f_3(0)=g_3(0)=0$. 
\end{enumerate}
