Cet ouvrage a été publié en 2014 chez l'éditeur "In Libro Veritas". Depuis cette date, les textes qui le composent ont continués à être mis à jour. La préface de l'ouvrage publié est reproduite au dessous.


La collection "MATH\'EMATIQUES EN MPSI" propose des documents pédagogiques (recueils de problèmes corrigés, livres de cours) en complément de ceux distribués en classe.\newline
Les ouvrages de la collection sont  disponibles sur internet. En fait, ils sont \emph{produits en ligne} à partir d'une base de données (le \emph{maquis documentaire}) accessible à l'adresse
\begin{center}
 \href{http://maquisdoc.net}{http://maquisdoc.net}
\end{center}
Cette base est conçue pour être très souple. Elle accompagne les auteurs et les utilisateurs en leur permettant de travailler librement et au jour le jour.

Il est devenu impossible de travailler sans internet (y compris pour rédiger des problèmes de mathématiques) mais il est également impossible de ne travailler que sur écran. Le papier garde donc toute sa validité et la publication de livres sous la forme imprimée habituelle (à coté d'autres types de services) est encore totalement justifiée.\newline
En revanche, le modèle économique de l'édition est devenu obsolète pour de tels ouvrages péri-scolaires produits à partir de structures web. L'éditeur (\emph{In Libro Veritas}) a accepté de diffuser cette collection sous licence Creative Commons. Les auteurs peuvent ainsi user plus libéralement de leur droit d'auteur et offrir davantage de liberté aux lecteurs.

\begin{center}
 \textbf{"\emph{Problèmes d'approfondissement}"}
\end{center}
est un recueil de problèmes corrigés.\newline
Les énoncés sont le plus souvent des adaptations pour la première année de problèmes de concours portant sur des sujets classiques. Divers thèmes sont abordés qui couvrent l'essentiel du programme de MPSI.  Les solutions mettent en \oe{}uvre de façon non immédiates des éléments de cours. Tous ces textes ont en commun de nécessiter une bonne maitrise des calculs, des concepts et de leurs relations.\newline
Une attention particulière a été portée à une redaction soigneuse et complète des corrigés. L'étudiant ne doit pas se condamner à trouver. La lecture de la solution, après un temps de recherche assez court, s'avèrera plus rentable qu'un acharnement infructueux. Lire un texte mathématique correct (un corrigé comme une copie d'élève doit en être un) aide à s'imprégner des conventions de rédaction, à dégager les concepts et les relations. L'étudiant ne doit pas non plus se contenter de trouver. Il faut s'obliger à repérer les tournures qui cristallisent les idées, à reproduire les présentations qui valorisent la copie. Il faut rédiger !


D'autres ouvrages de la collection proposent des textes plus simples (\emph{Problèmes basiques}) ou plus spécifiques (\emph{Problèmes d'automne}).
 
