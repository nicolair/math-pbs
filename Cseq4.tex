\begin{enumerate}
\item  On trouve facilement $\theta _{2}(x)=\frac{1}{1+x}$. On remarque que $%
0\leq \theta (x)\leq 1$ pour $x\geq 0.$
\item
\begin{enumerate}
\item  Utilisons l'indication de l'{\'e}nonc{\'e} avant de sommer
\[
\frac{1}{n^{2}+k}=\frac{1}{n^{2}}\left( 1-\frac{k}{n^{2}}+\frac{k^{2}}{n^{4}}%
\theta (\frac{k}{n^{2}})\right)
\]
\begin{eqnarray*}
\sum_{k=0}^{2n}\frac{1}{n^{2}+k} &=&\frac{1}{n^{2}}\sum_{k=0}^{2n}1-\frac{1}{%
n^{4}}\sum_{k=0}^{2n}k+\frac{1}{n^{6}}\sum_{k=0}^{2n}k^{2}\theta (\frac{k}{%
n^{2}})\\
&=&\frac{2}{n}-\frac{2n(2n+1)}{2n^{4}}+\frac{1}{n^{6}}%
\sum_{k=0}^{2n}k^{2}\theta (\frac{k}{n^{2}}) \\
&=&\frac{2}{n}-\frac{2}{n^{2}}-\frac{1}{n^{3}}+\frac{1}{n^{6}}%
\sum_{k=0}^{2n}k^{2}\theta (\frac{k}{n^{2}})
\end{eqnarray*}
Il reste {\`a} montrer que la derni{\`e}re somme est n{\'e}gligeable devant $%
\frac{1}{n^{2}}$. Ceci r{\'e}sulte de :
\[
0\leq \frac{1}{n^{6}}\sum_{k=0}^{2n}k^{2}\theta (\frac{k}{n^{2}})\leq \frac{1%
}{n^{6}}\sum_{k=0}^{2n}k^{2}\leq
\frac{1}{n^{6}}(2n)^{2}=\frac{4}{n^{3}}
\]
On majore $\theta (\frac{k}{n^{2}})$ par 1 puis chaque $k^{2}$ par
$n^{2}$. On obtient donc
\[
t_{n}=\frac{2}{n}-\frac{2}{n^{2}}+o(\frac{1}{n^{2}})
\]

\item Cherchons un d{\'e}veloppement de $t_{n+1}$ :
\[
t_{n+1}=\frac{2}{n+1}-\frac{2}{(n+1)^{2}}+o(\frac{1}{n^{2}})=\frac{2}{n(1+%
\frac{1}{n})}-\frac{2}{n^{2}}+o(\frac{1}{n^{2}})=\frac{2}{n}-\frac{4}{n^{2}}%
+o(\frac{1}{n^{2}})
\]
On en d{\'e}duit $t_{n+1}-t_{n}\sim -\frac{2}{n^{2}}$; ce qui montre $%
t_{n+1}-t_{n}<0$ {\`a} partir d'un certain rang. La suite
$(t_{n})_{n\in \N^{*}}$ est d{\'e}croissante {\`a} partir de ce rang.
\end{enumerate}

\item  Soit $i$ un entier alors $(i+1)^{2}-1=i^{2}+2i$. Si $j\in \left\{
i^{2},\ldots ,(i+1)^{2}-1\right\} $ alors $E(\sqrt{j})=i$ de sorte
que
\begin{eqnarray*}
U_{n^{2}+2n}&=&\sum_{i=1}^{n}\sum_{j=i^{2}}^{i^{2}+2i}\frac{(-1)^{E(\sqrt{j})}}{j}
=\sum_{i=1}^{n}(-1)^{i}\sum_{j=i^{2}}^{i^{2}+2i}\frac{1}{j}\\
&=&\sum_{i=1}^{n}(-1)^{i}\sum_{k=0}^{2i}\frac{1}{i^{2}+k}=\sum_{i=1}^{n}(-1)^{i}t_{i}=V_{n}
\end{eqnarray*}

\item
\begin{enumerate}
\item Comme $(t_{n})_{n\in \N^{*}}$ est d{\'e}croissante {\`a}
partir d'un certain rang et converge vers 0, la somme des
$(-1)^{n}t_{n}$ est convergente. Cet exercice a {\'e}t{\'e} trait{\'e} en
cours comme application de la notion de suite adjacente. On note
$l$ la limite$.$

\item Comme $u_{n}\rightarrow 0$ et $U_{(n+1)^{2}}=U_{n^{2}+2n}+u_{n}$,
on d{\'e}duit du a. la convergence de la suite extraite
$(U_{n^{2}})_{n\in
\N^{*}}$ vers $l$. D'autre part, si $p=E(\sqrt{n})$, on a aussi $%
U_{p^{2}}-t_{p}\leq U_{n}\leq U_{p^{2}}+t_{p}.$ On peut conclure
par le
th{\'e}or{\`e}me d'encadrement car $p\rightarrow +\infty $ lorsque $%
n\rightarrow +\infty $.
\end{enumerate}
\end{enumerate}
