\begin{enumerate}
\item On peut permuter circulairement les vecteurs d'un produit mixte, on en déduit :
\[\det (a,b,v)= \det(b,v,a)=(b\wedge v / a)=(d/a)\]
\[\det (a,b,v)=\det (v,a,b) = (v\wedge a / b)=-(c/b)\]
Lorsque le système d'équations vectorielles admet une solution, les deux expressions du déterminant sont égales d'où:
\[(d/a)+(c/b)=0\]
\item \begin{enumerate}
 \item 
L'étude de l'équation vectorielle d'inconnue $v$
\[a\wedge v = c\]
a été traitée en cours. Il existe des solutions lorsque $a$ et $c$ sont orthogonaux. Dans ce cas, l'ensemble des solutions est la droite vectorielle
\[-\frac{1}{\left\|a\right\|^2}\, a\wedge c + \Vect a\]
\item D'après la question précédente, le système $(*)$ admet une solution dès que les deux droites vectorielles se coupent. Introduisons des points pour utiliser la formule donnant la distance entre deux droites dans l'espace. \newline
Soit $O$ un point considéré comme origine. Définissons les points $A$ et $B$ par les relations:
\[\overrightarrow{OA}=-\frac{1}{\left\|a\right\|^2}\, a\wedge c + \Vect a \; , \; \overrightarrow{OB}=-\frac{1}{\left\|b\right\|^2}\, b\wedge d + \Vect d\]

Cherchons la distance entre les droites
\[\mathcal{D}=A+\Vect (a) \; ,\; \mathcal{D^\prime}=B+\Vect (b)\]

D'après le cours :
\[
d(\mathcal{D},\mathcal{D^\prime})=\frac{\det(\overrightarrow{AB},a,b)}{\left\|a\wedge b\right\|^2}
\]
avec
\begin{multline*}
\det(\overrightarrow{AB},a,b)
 = \det(-\frac{1}{\left\|b\right\|^2}\, b\wedge d + \frac{1}{\left\|a\right\|^2}\, a\wedge c,a,b)\\
 =  -\frac{1}{\left\|b\right\|^2}\det (b\wedge d,a,b) + \frac{1}{\left\|a\right\|^2} \det (a\wedge c,a,b)\\
 =  \frac{1}{\left\|b\right\|^2}\det (b\wedge d,b,a) + \frac{1}{\left\|a\right\|^2} \det (a\wedge c,a,b)\\
 =  \frac{1}{\left\|b\right\|^2} ((b\wedge d)\wedge b/b) + \frac{1}{\left\|a\right\|^2} ((a\wedge c)\wedge a/b)\\
= \frac{1}{\left\|b\right\|^2}\left (\left\|b\right\|^2 (d/a) -\underbrace{(b/d)}_{=0}(d/a) \right) 
 + 
\frac{1}{\left\|a\right\|^2}\left (\left\|a\right\|^2 (c/b) -\underbrace{(c/a)}_{=0}(a/b) \right)\\
= (d/a)+(c/b)
\end{multline*}
On en déduit que lorsque $(d/a)+(c/b)=0$, les deux droites se coupent et que le système admet une unique solution.
\end{enumerate}

\end{enumerate}