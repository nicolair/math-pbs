%<dscrpt>Sommations et nombres complexes.</dscrpt>
\begin{enumerate}
\item D{\'e}montrer que 
 \begin{displaymath}
\binom{n}{1}\frac{1}{1}- \binom{n}{2}\frac{1}{2}+\cdots+(-1)^{n-1} \binom{n}{n}\frac{1}{n}=1+\frac{1}{2}+\frac{1}{3}+\cdots+\frac{1}{n} 
 \end{displaymath}

\item Soit $n$ un entier naturel non nul, calculer $\sum_{(i,j)\in T}ij$ avec 
\begin{displaymath}
 T=\left\{ (i,j)\in \left\{ 1,\ldots ,n\right\} ^{2} \  \mathrm{ tq } \ i\leq j\right\}
\end{displaymath}

\item Soit $n$ un entier naturel non nul, calculer $R_{n}^{2}+I_{n}^{2}$ avec
\begin{align*}
 R_{n}=\sum_{k=0}^{\lfloor \frac{n}{2}\rfloor} (-1)^{k}\binom{n}{2k} & & I_{n}=\sum_{k=0}^{\lfloor \frac{n-1}{2}\rfloor} (-1)^{k}\binom{n}{2k+1}
\end{align*}

\item Soit $n$ un entier naturel non nul, $a_{1},a_{2},\cdots ,a_{n}$ des r{\'e}els strictement positifs, montrer
\begin{displaymath}
(a_{1}+a_{2}+\cdots +a_{n})(\frac{1}{a_{1}}+\frac{1}{a_{2}}+\cdots +\frac{1}{a_{n}}) \geq n^{2} 
\end{displaymath}

\item Soit $n$ un entier naturel non nul, calculer
\begin{displaymath}
 \binom{n}{0}-3\binom{n}{2}+3^{2}\binom{n}{4}-3^{3}\binom{n}{6}+\cdots 
\end{displaymath}

\item 
  \begin{enumerate}
    \item D{\'e}velopper $(z-a)(z-b)(z-c)$.
    \item Avec $w=e^{2i\pi /7}$, on considère
\begin{displaymath}
 a = w+w^{6},\, b = w^{2}+w^{5},\, c = w^{3}+w^{4}.
\end{displaymath}
Exprimer les coefficients du développement de la question a en fonction de puissances de $w$ seulement en les simplifiant à l'aide de relations vérifiées par $w$.\newline
En déduire une {\'e}quation de degr{\'e} 3 à coefficients entiers dont $\cos \frac{2\pi}{7}$, $\cos \frac{4\pi}{7}$, $\cos \frac{6\pi}{7}$ sont les racines.
\end{enumerate}


\end{enumerate}
