%<dscrpt>Une feuille de "calculs rapides".</dscrpt>
Dans toute cette partie, $x$ d{\'e}signe un {\'e}l{\'e}ment de $\left]
0,1\right[ $.

\begin{enumerate}
\item  Pour tout nombre entier naturel $n$, on pose $s(n,0)=1+x+\cdots
+x^{n}$. Calculer $s(n,0)$ et sa limite lorsque $n$ tend vers
$+\infty $.
\newline Pour tout nombre entier naturel $n$, on pose
$s(n,1)=1+2x+\cdots +(n+1)x^{n}$.

\item \begin{enumerate}
\item  Exprimer $(1-x)s(n,1)$ {\`a} l'aide de $s(n,0)$, en d{\'e}duire la
limite de $s(n,1)$ lorsque $n$ tend vers $+\infty $.

\item  Retrouver le r{\'e}sultat du a. {\`a} l'aide de la d{\'e}rivation.
\end{enumerate}
\item  Pour tout couple $(n,r)$ de nombres naturels, on pose
\[s(n,r)=\sum_{k=0}^{n}\binom{r+k}{r}x^{k}\]

\begin{enumerate}\item  On suppose que $n$ et $r$ sont des entiers non nuls.
Montrer que
\[
(1-x)s(n,r)=s(n,r-1)-\binom{r+n}{r}x^{n+1}\text{.}
\]

\item  D{\'e}terminer les limites des suites $(n^{r}x^{n})_{n\in \Bbb{N}}
$, $(\binom{r+n}{r}x^{n})_{n\in \Bbb{N}}$. D{\'e}terminer la limite de $%
(s(n,r))_{n\in \Bbb{N}}$.

\end{enumerate}
\end{enumerate}
