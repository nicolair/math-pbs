On va montrer que
\begin{displaymath}
 m' = \frac{b-a}{\overline{b-a}}\,\overline{(m-a)} +a
\end{displaymath}

On calcule d'abord, en utilisant un paramètre $\lambda$, l'affixe du projeté de $M$ sur la droite $(A,B)$.
L'affixe $h$ de ce projeté $H$ est de la forme $h=m+\lambda i (b-a)$ avec $\lambda$ réel. Pour calculer $\lambda$, on écrit que $H$ est sur la droite $(A,B)$.
\begin{displaymath}
 \frac{h-a}{b-a}=\frac{m-a}{b-a}+ i\lambda \in \R
\Rightarrow \lambda = -\Im\left(\frac{m-a}{b-a} \right) 
\end{displaymath}
On obtient alors facilement le symétrique :
\begin{multline*}
 m' = m + 2i\lambda (b-a) = m -2i(b-a)\Im\left(\frac{m-a}{b-a} \right)\\
= m-2i(b-a)\frac{1}{2i}\left(\frac{m-a}{b-a} -\overline{\frac{m-a}{b-a}} \right) 
=\frac{b-a}{\overline{b-a}}\,\overline{(m-a)} +a
\end{multline*}
On pouvait aussi chercher $m'$ sous la forme $u\overline{m}+v$ avec des coefficients indéterminés $u$ et $v$ que l'on calcule en écrivant que $a$ et $b$ sont invariants.