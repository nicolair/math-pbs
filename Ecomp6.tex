%<dscrpt>Propriété des nombres complexes. Etude d'une famille d'équations </dscrpt>
Dans tout l'exercice, $n$ désigne un entier naturel non nul, $\mathcal P _n$ désigne l'ensemble des entiers pairs entre $0$ et $n$ et $\mathcal I _n$ désigne l'ensemble des entiers impairs entre $0$ et $n$.

On associe à chaque paramètre complexe $a\neq 0$ deux équations d'inconnue $z$ notées $E_0(n,a)$ et $E_1(n,a)$
\begin{eqnarray*}
 E_0(n,a):\quad \quad \sum_{k\in \mathcal P _n} \binom{n}{k} z^k a^{n-k}&=& 0  \\
 E_1(n,a):\quad \quad \sum_{k\in \mathcal I _n} \binom{n}{k} z^k a^{n-k}&=& 0
\end{eqnarray*}

\begin{enumerate}
 \item Cas particuliers. Former les six équations obtenues pour $n=2$, $n=3$, $n=4$. Dans chaque cas, donner l'ensemble des solutions.
 \item Soit $\lambda$ un nombre complexe non nul, montrer que $w$ est solution de $E_0(n,a)$ si et seulement si $\lambda w$ est solution de $E_0(n,\lambda a)$
 \item 
  \begin{enumerate}
    \item Discuter selon le paramètre complexe $w$ et donner l'ensemble des solutions de l'équation d'inconnue $z$
\begin{displaymath}
 \frac{a+z}{a-z} = w
\end{displaymath}
     \item Pour $\alpha$ réel et $w=e^{i\alpha}$, simplifier
\begin{displaymath}
 \frac{w-1}{w+1}
\end{displaymath}
     \item Déterminer l'ensemble des solutions de l'équation d'inconnue $z$
\begin{displaymath}
 \left( \frac{a+z}{a-z}\right) ^n =1
\end{displaymath}
On exprimera chaque solution sous une forme simple.
     \item En considérant $(z+a)^n$ et $(-z+a)^n$ résoudre l'équation $E_1(n,a)$.
  \end{enumerate}
\item Une autre idée.
\begin{enumerate}
 \item Soit $x$ et $y$ deux nombres réels, exprimer avec des sommations les parties réelle et imaginaire de $(x+iy)^n$.
 \item Montrer que lorsque $\theta$ est un réel tel que $\cos \theta \neq 0$ :
\begin{displaymath}
 \frac{\cos(n\theta)}{\cos ^n \theta} = \sum_{k\in \mathcal P _n} \binom{n}{k} (i \tan \theta )^{k}, \hspace{0.5cm}
 i\,\frac{\sin(n\theta)}{\cos ^n \theta} = \sum_{k\in \mathcal I _n} \binom{n}{k} (i\tan \theta)^k
\end{displaymath}
\item En déduire les solutions de $E_1(n,1)$ puis retrouver celles de $E_1(n,a)$ déjà obtenues en 3.d.
 \item Déterminer les solutions de $E_0(n,i)$ puis de $E_0(n,a)$.
\end{enumerate}

\end{enumerate}


