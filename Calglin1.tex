\subsection*{Pr{\'e}liminaires}
\begin{enumerate}
 \item Comme $f^0$ est l'identité, son noyau $\{0_V\}$ est inclus dans $\ker f$. Pour $k\in \N^*$,
\begin{displaymath}
 \forall x\in V, \; x\in\ker f^k
\Rightarrow f^k(x)=0_V \Rightarrow f\left( f^k(x)\right) = f(0_V)=0_V
\Rightarrow x\in\ker f^{k+1}
\end{displaymath}
Ce qui montre la chaîne d'inclusions demandée.
\item Soit $p$ un entier tel que $\ker f^p = \ker f^{p+1}$, nous allons montrer que 
\begin{displaymath}
 \ker f^{p+2} \subset \ker f^{p+1}
\end{displaymath}
Cela entrainera que $\ker f^p = \ker f^{p+1} = \ker f^{p+2}$ à cause de l'inclusion  toujours valide $\ker f^{p+1} \subset \ker f^{p+2}$. On peut alors déduire par récurrence l'égalité de tous les noyaux suivants.\newline
Il s'agit donc de montrer que $\ker f^{p+2} \subset \ker f^{p+1}$.
Cela résulte de 
\begin{multline*}
 \forall x\in V, \; x\in \ker f^{p+2} : f^{p+1}(f(x))=0_V\Rightarrow f(x)\in \ker f^{p+1}=\ker f^{p} \\
\Rightarrow f^{p+1}(x)=f^p(\underset{\in \ker f^p}{\underbrace{f(x}}))=0_V 
\Rightarrow x\in \ker f^{p+1}
\end{multline*}
\item On suppose que $V$ est de dimension finie, tous les sous-espace de $V$ sont alors de dimension finie. La suite $\left( \dim \ker f^k \right) _{k\in \N}$ définit une fonction croissante de $\N$ dans l'ensemble fini $\llbracket 0, \dim V\rrbracket$. Une telle suite ne peut pas être strictement croissante car elle serait injective. Il existe donc des entiers $k$ tels que $\dim f^k < \dim f^{k+1}$ soit faux ce qui entraine $\dim f^k = \dim f^{k+1}$ car $\dim f^k \leq \dim f^{k+1}$. Soit $p$ le plus petit de ces $k$. Il vérifie
\begin{displaymath}
 0=\dim (\ker f^0) < \dim (\ker f^1)< \cdots<\dim (\ker f^p) = \dim (\ker f^{p+1}) \leq \dim V=n
\end{displaymath}
Comme les premières inégalités sont strictes et qu'il y en a $p$, on obtient
\begin{displaymath}
 p\leq \dim (\ker f^p) \leq n
\end{displaymath}
D'après un résultat de cours sur les sous-espaces en dimension finie:
\begin{displaymath}
 \left. 
\begin{aligned}
 \dim (\ker f^p) =& \dim (\ker f^{p+1}) \\
 \ker f^p \subset& \ker f^{p+1}
\end{aligned}
\right\rbrace 
\Rightarrow \ker f^p = \ker f^{p+1}
\end{displaymath}
L'égalité se propage alors (d'après 2.) à tous les $k\geq p$ parmi lesquels figure $n$ ce qui entraine $\ker f^n = \ker f^{n+1}$.
\item Dans cette question, l'endomorphisme $u$ est nilpotent. D'après la question précédente, la suite \og croissante\fg~ des $\ker u^k$ se stabilise avant $n$ à sa valeur finale qui est $V$ tout entier. On en déduit qu'il existe un $p\leq n$ tel que $V=\ker u^p$.\newline
On en tire que $V=\ker u^n$ c'est à dire que $u^n$ est l'endomorphisme nul.
\end{enumerate}

\subsection*{Partie I.}
\begin{enumerate}
  \item
     \begin{enumerate}
       \item La relation  $g_n^2=\lambda Id_{E_n} +D_n$ (entre des éléments de $\mathcal{L}(E_n)$) permet d'exprimer $D_n$ en fonction de $g_n$ :
\begin{displaymath}
 D_n=-\lambda Id_{E_n} + g_n^2
\end{displaymath}
Sous cette forme, il est {\'e}vident que $D_n$ commute avec $g_n$.\newline
Pour montrer qu'un sous-espace $E_p$ (avec $0\leq p \leq n$) est stable par $g$, on remarque que c'est un noyau. En effet: 
\begin{displaymath}
  E_p = \R_p[X] = \ker D_n^{p+1}
\end{displaymath}

Comme $g_n$ commute avec $D_n$, il commute aussi avec les puissance de $D_n$. En particulier
\begin{multline*}
 x\in E_p = \ker D_n^{p+1} \Rightarrow D_n^{p+1}(g_n(x))= g_n( D_n^{p+1}(x)) = g_n(0_{E_n}) = 0_{E_n}\\
 \Rightarrow g_n(x) \in \ker D_n^{p+1} = E_p
\end{multline*}
Une fois prouv{\'e}e la stabilit{\'e} de $E_p$ par $g_n$, on peut consid{\'e}rer la restriction $g_p$ de $g_n$ {\`a} $E_p$. Elle v{\'e}rifie {\'e}videmment la m{\^e}me relation que $g_n$.

 \item Le raisonnement est le m{\^e}me que pour la question pr{\'e}c{\'e}dente. Le fait que $E$ ne soit pas de dimension finie ne change rien. Si $g$ v{\'e}rifie la relation, il commute donc avec l'op{\'e}rateur de d{\'e}rivation.\newline
Comme plus haut, $E_n$ est stable par $g$ car c'est un noyau d'une puissance de $D_n$ et la restriction $g_n$ de $g$ v{\'e}rifie la m{\^e}me relation avec la restriction $D_n$ de $D$.
    \end{enumerate}
    
 \item
  \begin{enumerate}
 \item L'opérateur $D_F$ est la restriction {\`a} $F$ de l'op{\'e}rateur de d{\'e}rivation. Comme $F$ est de dimension finie, il existe un entier $k$ qui est le degr{\'e} maximal d'un polyn{\^o}me quelconque de $F$. Alors $D_F^{k+1}$ est nul.\newline
D'apr{\`e}s la partie pr{\'e}liminaire, comme $D_F$ est nilpotent dans un espace de dimension $n+1$, l'endomorphisme $D_F^{n+1}$ est nul. Ceci montre que $F\subset \R_n[X]$. Comme les deux espaces sont de m{\^e}me dimension, ils sont {\'e}gaux.\newline
On peut en conclure que les seuls sous-espaces de dimension finie stables par $D$ sont les $\R_n[X]$. \newline
Un seul sous-espace de dimension infinie est stable par $D$, il s'agit de $\R[X]$ lui m{\^e}me. En effet, un tel espace doit contenir des polyn{\^o}mes de degr{\'e} arbitrairement grand (sinon il serait de dimension finie) et tous leurs polynômes d{\'e}riv{\'e}s.

\item Soit $G$ un sous-espace de $E$. Supposons $G$ stable par $g$ et exploitons la relation fondamentale pour montrer que $G$ est stable par $D$.
\begin{displaymath}
\forall P \in G,\; D(P) = \underset{\in G}{\underbrace{g^2(P)}} - \underset{\in G}{\underbrace{\lambda P}} \in G 
\end{displaymath}
Réciproquement, supposons $G$ stable par $D$. D'après la question précédente (I.1.b) $G=\R[X]$ ou il existe $n\in \N$ tel que $G=E_n$.
\begin{itemize}
  \item Si $G=\R[X]$, il est évidemment stable par $g$.
  \item Si $G = E_n$, on a montré en I.1.b que $E_n = \ker D^{n+1}$ est stable par $g$.
\end{itemize}
          \end{enumerate}

  \item Cas $\lambda<0$.

\begin{enumerate}
       \item Dans $E_0=\R$ qui est un espace de dimension 1, les seules applications lin{\'e}aires sont les multiplications par un scalaire et $D_0$ est l'application nulle. S'il existe un $g_0$ (la multiplication par un $\mu\in \R$) vérifiant la relation, on peut écrire
\begin{displaymath}
  g_0^2 = \lambda {\Id}_{E_0} + D_0 \Leftrightarrow \mu^2=\lambda \Rightarrow \lambda \geq 0
\end{displaymath}
La condition nécessaire à l'existence d'un $g_0$ vérifiant la relation est donc $\lambda \geq 0$.

       \item D'apr{\`e}s 1.a., lorsqu'il existe un entier $n$ et un $g_n\in \mathcal{L}(E_n)$ vérifiant la relation, tous les sous-espaces $E_p$ avec $p \in      \llbracket 0,n \rrbracket$ sont stables par $g_n$. D'après 1.b., lorsqu'il existe un $g$ dans $\mathcal{L}(E)$ vérifiant la condition, tous les sous-espaces $E_p$ avec $p$ entier sont stables par $g$.\newline
       Dans les deux cas, $E_0$ est stable donc $\lambda \geq 0$. Ainsi, lorsque $\lambda<0$, il n'existe pas d'application $g$ v{\'e}rifiant la condition {\'e}tudi{\'e}e, ni dans $\mathcal{L}(E)$, ni dans un $\mathcal{L}(E_n)$.
     \end{enumerate}

  \item
     \begin{enumerate}
\item Soit $f$ lin{\'e}aire de $V$ dans $V$ telle que $f^{n+1}$ soit nulle mais pas $f^n$. Il existe alors un $y\in V$ tel que
\begin{displaymath}
f^n(y)\neq 0  
\end{displaymath}
Montrons que $\mathcal{B}=(y,f(y),\cdots,f^n(y))$ est libre.\newline
Si $(\lambda_{0},\lambda_{1},\cdots\lambda_{n})$ sont des r{\'e}els tels que
 \begin{displaymath}
\lambda_{0}y+\lambda_{1}f(y)+\cdots+\lambda_{n}f^n(y)=0   
 \end{displaymath}
en composant par $f^n$, on obtient $\lambda_{0}f^n(y)=0$ avec $f^n(y)\neq 0$ d'o{\`u} $\lambda_{0}=0$ et ainsi de suite. En composant successivement par $f^{n-1},f^{n-2},\cdots$ on obtient la nullit{\'e} de tous les coefficients. La famille est donc libre.\newline
Cette famille est une base car elle contient autant de vecteurs que la dimension de l'espace.
       
\item Dans le cas où l'endorphisme nilpotent est la dérivation restreinte à un espace $E_n = \R_n[X]$, 
\begin{displaymath}
  Y = X^n \Rightarrow \left( Y,D_n(Y),\cdots,D_n^n(Y)\right) 
\end{displaymath}
est une base de $E_n$. En fait n'importe quel polynôme de degré $n$ aurait fait l'affaire.
     \end{enumerate}

  \item Un exemple avec $n=2$ et $\lambda >0$.

\begin{enumerate}
  \item Il est bien {\'e}vident que les endomorphismes $h$ de la forme
\begin{displaymath}
  h = a \Id_{E_2} + bD_2 + c D_2^2
\end{displaymath}
commutent avec $D_2$. On va montrer que ce sont les seuls.\newline
Considérons $X^2\in \R_2[X]$, la famille $(X^2,D(X^2),D^2(X^2)) = (X^2,2X,2)$ est une base de $E_2$ d'après la question 4.b ou simplement car il s'agit d'une famille de 3 polynômes de degrés échelonnés dans $\R_2[X]$. Comme $f(X^2)\in E_2$, 
\begin{displaymath}
  \exists (a,b,c)\in \R^3 \text{ tel que } f(X^2) = aX^2 + bD(X^2) + cD^2(X^2)
\end{displaymath}
Définissons $F\in \mathcal{L}(E_2)$ par $F =a \Id_{E_2} + bD_2 + cD_2^2$ et comparons le à $f$. Pour cela, il suffit de les comparer sur les vecteurs d'une base (théorème du prolongement linéaire).\newline
Par d{\'e}finition de $F$:
\begin{align*}
 f(X^2)=& F(X^2) \\
f(D(X^2))=& D(f(X^2))=aD(X^2)+bD^2(X^2)=F(D(X^2)) \text{ car } D^3(X^2)=0 \\
f(D^2(X^2))=& D^2(f(X^2))=aD^2(X^2)=F(D^2(X^2))
\end{align*}
Les deux fonctions co{\"\i}ncident sur une base, elles sont donc {\'e}gales.

  \item Montrons que $\left( \Id_{E_2}, D_2, D_2^2\right)$ est une famille libre dans $\mathcal{L}(E_2)$. Considérons des réels $\alpha$, $\beta$, $\gamma$ tels que
\begin{displaymath}
  \alpha \Id_{E_2} + \beta D_2 + \gamma D_2^2 = 0_{\mathcal{L}(E_2)}
\end{displaymath}
et prenons la valeur de cette expression (endomorphis nul) successivement aux polynômes $1$, $X$ et $X^2$. On en tire dans l'ordre $\alpha = 0$, $\beta=0$, $\gamma =0$. La famille est donc bien libre.

  \item On doit chercher les $g_2$ telles que $g_2^2=\lambda \Id_{E_2} +D_2$ parmi les applications qui commutent avec $D_2$. Cherchons donc des conditions sur $a,b,c$ assurant que
\begin{displaymath}
g_2 = a\Id_{E_2} + bD_2 + cD_2^2  
\end{displaymath}
v{\'e}rifie $g_2^2=\lambda \Id_{E_2} + D_2$. Calculons $g^2$ :
\begin{displaymath}
g^2 = a^2 \Id_{E_2} + 2abD_2^2 + (b^2+2ac)D_2^2 = \lambda \Id_{E_2} + D_2  
\end{displaymath}
Comme $\left( \Id_{E_2}, D_2, D_2^2\right)$ est une famille libre, on peut identifier les coefficients. On trouve donc exactement deux endomorphismes répondant à la question, l'un étant déterminé par
\begin{displaymath}
a = \sqrt{\lambda},\hspace{0.5cm} b=\frac{1}{2\sqrt{\lambda}},\hspace{0.5cm} c=-\frac{1}{8\lambda \sqrt{\lambda}} 
\end{displaymath}
L'autre {\'e}tant son oppos{\'e}e.
     \end{enumerate}
\end{enumerate}

\subsection*{Partie II.}
\begin{enumerate}
  \item
\begin{enumerate}
\item On suppose ici $g_n^2 = D_n$. Comme $D_n$ est nilpotent $g_n$ l'est aussi. Si un endomorphisme $f$ est injectif, alors tous les $f^k$ le sont également,  par cons{\'e}quent $g_n^2$ et $g_n$ ne sont pas injectifs (une de leurs puissance est l'endomorphisme nul). Mais pourquoi $\ker g_n^2$ est-il de dimension au moins 2?\newline
Si ce n'était pas le cas, on aurait, avec $\ker g_n \subset \ker g_n^2$, 
\begin{displaymath}
  0 < \dim (\ker g_n) \leq \dim (\ker g_n^2) < 2 \Rightarrow \dim (\ker g_n) = \dim (\ker g_n^2) = 1 
\end{displaymath}
La suite des noyaux de $g_n^k$ est alors constante d{\`e}s le premier rang. Mais d'apr{\`e}s la partie pr{\'e}liminaire, sa valeur finale est $E_n$ autrement dit $g_n$ est nulle ce qui est absurde.
 \item Il n'existe pas de $g_n$ tel que $g_n^2 = D_n$ car le noyau de $D_n$ est de dimension 1 (polynômes constants) alors que celui de $g_n^2$ devrait {\^e}tre au moins 2.\newline
D'après la partie I, si $g^2 = D$, les espaces $E_n$ sont stables par $g$ et les restrictions $g_n$ vérifient $g_n^2 = D_n$ ce qui est impossible. 
     \end{enumerate}

  \item \begin{enumerate}
\item Tout polyn{\^o}me admet plusieurs polyn{\^o}mes \emph{primitifs} (c'est à dire dont le polynôme dérivé est égal au polynôme donné) qui diff{\`e}rent d'une constante. L'application $D$ et les applications $D^m$ sont donc surjectives.\newline 
La surjectivit{\'e} de $g^k$ entra{\^\i}ne celle de $g$ car $\Im g^k \subset \Im g$.

\item Pour $q\leq k$, $\ker g^q \subset \ker g^k=\ker D^m=E_{m-1}$ qui est de dimension finie $m$. On conclut avec le résultat de cours: tout sous-espace d'un espace de dimension finie est de dimension finie.

\item L'application $\Phi$ est lin{\'e}aire car c'est la restriction de $g$ à $\ker g^p$. Elle prend ses valeurs dans $\ker g^{q-1}$ car
\begin{displaymath}
  \forall x\in E, \; x\in \ker g^p \Rightarrow 0_E = g^p(x)=g^{p-1}(g(x)) \Rightarrow g(x)\in \ker g^{p-1}
\end{displaymath}
Montrons la surjectivit{\'e} de $\Phi$.\newline
Soit $x\in \ker g^{p-1}$. Comme $g$ est surjective, il existe un $y\in E$ tel que $x=g(y)$ et
\begin{displaymath}
0_E = g^{p-1}(x) = g^{p-1}(g(y)) = g^p(y)  
\end{displaymath}
donc $y\in \ker g^p$ et $y$ est un ant{\'e}c{\'e}dent par $\Phi$ de $x$.\newline
Ainsi $\Phi$ est surjective de $\ker g^p$ vers $\ker g^{p-1}$ de noyau $\ker g$. Le th{\'e}or{\`e}me du rang donne alors
\begin{displaymath}
\dim(\ker g^p) = \dim(\ker g^{p-1}) + \dim(\ker g)  
\end{displaymath}
La suite des dimensions est arithm{\'e}tique d'o{\`u}
\begin{displaymath}
\dim(\ker g^p)=p\dim(\ker g)
\end{displaymath}
     \end{enumerate}

  \item D'après la question précédente, comme $\dim(\ker D^m)=m$,
\begin{displaymath}
  g^k = D^m \Rightarrow
\dim (\ker g^k) = \dim (\ker D^m) \Rightarrow k\dim(\ker g) = m \Rightarrow k \text{ divise } m
\end{displaymath}
Réciproquement, si $k$ divise $m$, il existe $q\in \N$ tel que $m=qk$. En posant $g = D^q$, on vérifie
\begin{displaymath}
  g^k = D^{qk} = D^m
\end{displaymath}
La condition nécessaire et suffisante demandée est donc
\begin{displaymath}
  k \text{ divise } m
\end{displaymath}

\end{enumerate}

\subsection*{Partie III.}
\begin{enumerate}
  \item 
\begin{enumerate}
\item La fonction $\varphi$ est $\mathcal{C}^{\infty}$ dans $]-1,+\infty[$, elle admet donc, d'après la formule de Taylor avec reste de Young, des développements limités à tous les ordres.

\item \`A l'ordre $3$, le développement est usuel:
\begin{displaymath}
\varphi(x) = (1+x)^{\frac{1}{2}} = 1 + \frac{1}{2}x -\frac{1}{8}x^2 + \frac{1}{16}x^3 + o(x^3)  
\Rightarrow b_0=1, b_1=\frac{1}{2}, b_2=-\frac{1}{8}, b_3=\frac{1}{16}
\end{displaymath}
Pour $b_k$, on utilise l'expression venant de la formule de Taylor
\begin{displaymath}
b_k = \frac{\varphi^{(k)}(0)}{k!}
= \frac{1}{k!}\underset{k \text{ facteurs consécutifs}}{\underbrace{(\frac{1}{2})(\frac{1}{2}-1)(\frac{1}{2}-2) \cdots }}
= \frac{(-1)^{k-1}}{2^k\,k!}\prod_{i=1}^{k-1}(2i-1)
\end{displaymath}
Le produit étant constitué des impairs entre $1$ et $2k-3$. On le transforme
\begin{displaymath}
\prod_{i=1}^{k-1}(2i-1)
= \frac{(2k-2)!}{\text{pdt des pairs entre $2$ et $2k-2$}}
= \frac{(2k-2)!}{2^{k-1}(k-1)!}
\end{displaymath}
On en déduit
\begin{displaymath}
  b_k = \frac{(-1)^{k-1} (2k-2)!}{2^{2k-1}k!\,(k-1)!}
  = \frac{(-1)^{k-1} (2k-1)!}{2^{2k-1}(2k-1)k!\,(k-1)!}
  = \frac{(-1)^{k-1}}{2^{2k-1}(2k-1)}\binom{2k-1}{k}
\end{displaymath}

\item Pour montrer la formule demandée, il ne faut surtout pas chercher à combiner les coefficients du binôme mais utiliser le produit du développement limité par lui même. Pour tout $n$ entier, notons
\begin{displaymath}
a_m = \sum_{k=0}^{m}b_k\,b_{m-k}  
\end{displaymath}
En fait $a_m$ est le coefficient de $x^m$ dans le développement limité de $\varphi(x)^2 = 1+x$. On en déduit que $a_m=1$ pour $m=1$ ou $2$ et $a_n=0$ pour les autres valeurs.
\end{enumerate}

  \item On calcule $g_n^2$ en utilisant le fait que $\Id_{E_n}$ et $D_n$ commutent et que $D_n^{n+1}$ est l'endomorphisme nul:
\begin{displaymath}
g_n^2 = \sum_{(k,k')\in \llbracket 0,n \rrbracket^2} b_kb_k'D_n^{k+k'}  
\end{displaymath}
Pour $m$ entre $0$ et $2n$, on regroupe les $(k,k')$ tels que $k+k'=m$. En fait, seuls les $m \leq n$ contribuent significativement car $D_n^{n+1}$ est l'endomorphisme nul.\newline
Pour $m$ entre 0 et $n$, on retrouve les sommes $a_m$ de la question précédente
\begin{displaymath}
  g_n^2 = \sum_{m=0}^{n} a_m\,D_n^{m} = D_m^0 + D_m = \Id_{E_m} + D_m
\end{displaymath}

  \item On cherche à se rapprocher du cas précédent en factorisant par $\lambda$ lorsqu'il est non nul:
\begin{displaymath}
\lambda \Id_{E_n} + D_n = \lambda \left( \Id_{E_n} + \frac{1}{\lambda}D_n\right)   
\end{displaymath}
En remplaçant $D_n$ par $\frac{1}{\lambda}D_n$, un calcul analogue au précédent montre que 
\begin{displaymath}
\left( \sum_{k=0}^{n}\frac{b_k}{\lambda^{k}}D_n^k\right)^2 = \Id_{E_n} + \frac{1}{\lambda}D_n   
\end{displaymath}
Si $\lambda>0$, on peut poser
\begin{displaymath}
  g_n = \sqrt{\lambda}\left( \sum_{k=0}^{n}\frac{b_k}{\lambda^{k}}D_n^k\right)
\end{displaymath}
qui vérifie $g_n^2 = \lambda \Id_{E_m} + D_m$.\newline
L'expression suivant d'un endomorphisme de $E$ semble n'avoir aucun sens car elle fait intervenir une somme infinie
\begin{displaymath}
  g = \sqrt{\lambda}\left( \sum_{k=0}^{+\infty}\frac{b_k}{\lambda^{k}}D_n^k\right)
\end{displaymath}
En fait elle définit bien un endomorphisme car, pour chaque polynôme $P$, le calcul de $g(P)$ ne fait intervenir que les $k$ inférieurs ou égaux au degré de $P$. Si $n=\deg(P)$, tout se passe dans $E_n$ et $g(P)=g_n(P)$ d'où 
\begin{displaymath}
  g^2(P) = g_n^2(P) = \lambda P + D(P)
\end{displaymath}
Ceci étant valable pour tous les $P$, on a bien $g^2 = \lambda \Id_E + D$.
\end{enumerate}
