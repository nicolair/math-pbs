%<dscrpt>Calculs de sommes.</dscrpt>

\subsection*{Exercice 1}
Soit $a,b,n$ des nombres entiers, on pose
\begin{align*}
D_a &= \{(x,y)\in \N^2 \, \text{ tq }\, x+y=a\}\\
T_n &= \{(x,y)\in \N^2 \, \text{ tq }\, x+y \leq n\}\\
C_n &= \{0,1,\cdots,n\}^2
\end{align*}
Donner une expression simple de chacune des sommes suivantes
\begin{align*}
A_a&=\sum_{(x,y)\in D_a}\binom{x+y}{x} & & B_n = \sum_{(x,y)\in T_n}\binom{x+y}{x}\\
G_{b,n}&= \sum_{x=0}^n\binom{x+b}{x}   & & D_n = \sum_{(x,y)\in C_n}\binom{x+y}{x}
\end{align*}

%ex exo5 de elem4
\subsection*{Exercice 2}
Pour $k$ entier naturel et $x$ réel non congru à $0$ modulo $\pi$, linéariser
\begin{displaymath}4\sin^2x\sin (2kx)\end{displaymath}
et l'exprimer comme la différence de deux termes consécutifs d'une suite. En déduire, pour des entiers $p$ et $q$ fixés tels que $p\leq q$, une autre expression de 
\begin{displaymath}\sum_{k=p}^{q}4\sin^2x\sin (2kx)\end{displaymath}

%ex exo9 de elem4
\subsection*{Exercice 3}
Soit $n$ un entier strictement positif, exprimer, pour $k\in\{0,2,\ldots,n-1\}$,
\begin{displaymath}\frac{\binom{n}{k}}{ \binom{2n}{k}}-\frac{\binom{n}{k+1}}{ \binom{2n}{k+1}}\end{displaymath}
{\`a} l'aide d'un quotient de deux coefficients du bin{\^o}me.\newline
En déduire une expression de
\begin{displaymath}\sum _{k=0}^{n}\frac{\binom{n}{k}}{ \binom{2n-1}{k}}\end{displaymath}
