\begin{enumerate}
 \item \begin{enumerate}
 \item Les deux équations du système sont équivalentes car elles sont conjuguées.
 \item On résoud le système de la question précédente par les formules de Cramer. Le déterminant égal à $|a|^2 -|b|^2$ est non nul.
On obtient une seule solution :
\begin{displaymath}
  \frac{\overline{a}z^\prime + b\overline{z^\prime} -\overline{a}c + b\overline{c}}{|a|^2 -|b|^2}
\end{displaymath}
\end{enumerate}
\item Ici, l'équation $(E)$ devient
\begin{displaymath}
 re^{i\alpha}z + re^{i\beta}\overline{z}=z^\prime _c
\end{displaymath}
On introduit alors les relations suivantes dans $(E)$
\begin{align*}
 e^{i\alpha} = e^{i\frac{\alpha + \beta}{2}}e^{i\frac{\alpha - \beta}{2}} &,&
 e^{i\beta} = e^{i\frac{\alpha + \beta}{2}}e^{-i\frac{\alpha - \beta}{2}}
\end{align*}
ce qui conduit  à :
\begin{displaymath}
 re^{i\frac{\alpha + \beta}{2}}2 \Re (e^{i\frac{\alpha - \beta}{2}}z) = z^\prime -c
\end{displaymath}
Comme l'affixe de $\overrightarrow{AB}$ est
\begin{displaymath}
 e^{i\beta}-e^{i\alpha}= e^{i\frac{\alpha + \beta}{2}}2i \sin \frac{\beta - \alpha}{2} 
\end{displaymath}
On obtient bien que $\overrightarrow{CM^\prime}$ est othogonal à $\overrightarrow{AB}$. Le point $M^\prime$ d'affixe $z^\prime$ est donc sur la droite (notée $D^\prime$) passant par $C$ et orthogonale à $(AB)$.
\item Si le point $M^\prime$ d'affixe $z^\prime$ est sur la droite (notée $D^\prime$) passant par $C$ et orthogonale à $(AB)$, il existe un réel $\lambda$ tel que
\begin{displaymath}
 z^\prime -c = 2\lambda e^{i\frac{\alpha + \beta}{2}}
\end{displaymath}
On peut alors simplifier par $e^{i\frac{\alpha + \beta}{2}}$ l'équation $(E)$ ce qui donne
\begin{displaymath}
 \Re (e^{i\frac{\alpha - \beta}{2}}z) = \frac{\lambda}{r}
\end{displaymath}
En notant $x$ et $y$ les coordonnées dee $M$ qui  sont aussi les parties réelle et imaginaire de l'affixe $z$, cette relation s'écrit
\begin{displaymath}
 \cos \frac{\alpha - \beta}{2}x -\sin \frac{\alpha - \beta}{2}y = \frac{\lambda}{r}
\end{displaymath}
C'est l'équation d'une droite dont la direction est
\begin{displaymath}
\overrightarrow{e}_{ \frac{\pi}{2}-\frac{\alpha - \beta}{2}}
\end{displaymath}
On a vu qu'un vecteur directeur de $D^\prime$ (orthogonal à $(AB)$) est 
\begin{displaymath}
 \overrightarrow{e}_{\frac{\alpha + \beta}{2}}
\end{displaymath}
L'angle de droites $(D^\prime , D_{z^\prime})$ est donc
\begin{displaymath}
 \frac{\pi}{2}-\frac{\alpha - \beta}{2} - \frac{\alpha + \beta}{2} = \frac{\pi}{2} -\alpha   \mod(\pi)
\end{displaymath}

\end{enumerate}
