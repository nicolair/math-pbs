\begin{enumerate}
\item Voir les figures \ref{Cplanicom_1} et \ref{Cplanicom_2}.
\begin{figure}[h!t]
 \centering
 \input{Cplanicom_1.pdf_t}
 \caption{Configuration 1}
 \label{Cplanicom_1}
\end{figure}
\begin{figure}[h!t]
 \centering
 \input{Cplanicom_2.pdf_t}
 \caption{Configuration 2}
 \label{Cplanicom_2}
\end{figure}

\item \begin{enumerate}
\item On étudie les variations de $f$ et $g$ dans $]0,+\infty[$.\newline
On calcule d'abord la dérivée de $f$ en l'exprimant comme une somme:
\[
f(x)=\frac{x}{2r_1}+\frac{r_1^2-r_2^2}{2r_1}\frac{1}{x}=
\frac{x}{2r_1}+\frac{r_1 \cos^2\Phi}{2x}
\]
On en déduit
\[
f'(x) = \frac{(x-r_1\cos \Phi)(x+r_1\cos \Phi)}{2r_1x^2}
\]
Pour étudier $g$ il est inutile de calculer sa dérivée. L'expression
\begin{align*}
 g(x)=\frac{x}{2r_2}-\frac{r_1^2-r_2^2}{2r_2}\frac{1}{x}
\end{align*}
suffit à montrer que la fonction est croissante car $r_1^2-r_2^2>0$. On en déduit les tableaux suivants:
\begin{displaymath}
\renewcommand{\arraystretch}{1.5}
\begin{array}{|c|ccccc|}
\hline & 0 &  & r_1 \cos \Phi &  & +\infty \\ 
\hline & + \infty &  &  &  & +\infty \\ 
f &  & \searrow &  & \nearrow &  \\ 
 &  &  & \cos \Phi &  & \\ \hline
\end{array}
\hspace{1cm}
\begin{array}{|c|ccc|}
\hline & 0 &    & +\infty \\ 
\hline & &   & +\infty \\ 
g &    & \nearrow &  \\ 
 & -\infty  &  & \\ \hline
\end{array} 
\end{displaymath}
\item \'Etudions l'équation $f(x)=1$. Le tableau des variations et le théorème de la valeur intermédiaire montrent clairement que cette équation admet deux solutions. Cette équation revient à 
\[
 (x-r_1)^2 = r_2^2
\]
Les deux solutions sont donc $r_1-r_2$ et $r_1+r_2$. On en déduit que 
\begin{align*}
\{x \,\text{ tq }\, f(x)\in [-1,+1]\} &= [r_1-r_2,r_1+r_2] \\
\{f(x) \,\text{ tq }\, f(x)\in [-1,+1]\} &= [\cos \Phi,1]
\end{align*}
D'après le tableau des variations de $g$, pour déterminer les $x$ tels que $-1 \leq g(x) \leq 1$, nous devons étudier les équations $g(x)=-1$ et $g(x)=1$. Elles ont chacune une seule solution : $r_1-r_2$ et $r_1+r_2$ respectivement. On en déduit :
\begin{align*}
\{x \,\text{ tq }\, g(x)\in [-1,+1]\} &= [r_1-r_2,r_1+r_2] \\
\{g(x) \,\text{ tq }\, g(x)\in [-1,+1]\} &= [-1,1]
\end{align*}
\item Pour factoriser $f(x)-g(x)$, il est plus commode d'utiliser $r_1$ et $r_2$ que $\Phi$. On obtient
\[
f(x)-g(x) =\frac{1}{2r_1r_2\,x}(r_1-r_2)(r_1+r_2-x)(r_1+r_2+x)
\]
On en déduit le tableau de signes
\begin{displaymath}
\renewcommand{\arraystretch}{1.5}
\begin{array}{|c|ccccc|} \hline
 & 0 &  & r_1+r_2 &  & +\infty \\ 
\hline f(x)-g(x) &  & + & 0 & - &\\ \hline 
\end{array} 
\end{displaymath}

\end{enumerate}
\item \begin{enumerate}
\item Il s'agit simplement d'utiliser l'inégalité triangulaire De $z=z_1+z_2$, on déduit $|z|\leq |z_1|+|z_2|=r_1+r_2$. De $z_1=z-z_2$, on déduit $|z_1|\leq |z|+|z_2|=|z|+r_2$ puis $|z|\geq r_1-r_2$.

\item Considérons le carré du module de $z-z_1 = z_2$.
\[
|z-z_1|^2 = \left||z|e^{i\theta} - r_1e^{i\theta_1}\right|^2 
= |z|^2-2|z|r_1\cos(\theta - \theta_1)+r_1^2 = |z_2|^2 = r_2^2
\]
puis $\cos(\theta - \theta_1)=f(|z|)$.\newline
On obtient de même $\cos(\theta - \theta_2)=g(|z|)$ en considérant $z-z_2=z_1$ 
\end{enumerate}

\item \begin{enumerate}
\item 
Supposons que $\theta_1$ et $\theta_2$ vérifient les conditions de l'énoncé. Alors, d'après 3.b. et les conditions sur l'intervalle $[0,\pi]$,
\[
\left\lbrace 
\begin{aligned}
\theta - \theta_1 &= \arccos(f(|z|)\\
\theta_2-\theta &= \arccos(g(|z|) 
\end{aligned}
\right. \Rightarrow
\left\lbrace 
\begin{aligned}
\theta_1 &= \theta -\arccos(f(|z|)\\
\theta_2 &= \theta + \arccos(g(|z|) 
\end{aligned}
\right. 
\]
Ces formules assurent l'unicité.\newline
En ce qui concerne l'existence : remarquons d'abord que si $z=|z|e^{i\theta}$ et $z_1=r_1e^{i\theta_1}$ alors: 
\[
|z-z_1|^2=|z|^2-2|z|r_1\cos(\theta - \theta_1)+r_1^2
\] 
Par conséquent, si on pose $\theta_1 = \theta -\arccos(f(|z|)$ ce qui est possible d'après l'étude de $f$ pour $r_1-r_2\leq|z|\leq r_1+r_2$ on aura
\[
|z-z_1|^2=r_2^2
\]
On peut remarquer que  $\theta - \theta_1 = \arccos(f(|z|)\leq \Phi$ car la valeur minimale que prend $f$ est $\cos \Phi$,\newline
Une fois $\theta_1$ défini, on est certain de l'existence d'un $\theta_2$. Il suffit de prendre un argument de $z-z_1$.\newline
Parmi tous les arguments de $z-z_1$ possibles pourquoi en existe-t-il un (noté $\theta_2$) tel que  $\theta_2-\theta \in [0,\pi]$ ?\newline
Divisons la relation $z=z_1+z_2$ par $e^{i\theta}$ puis prenons la partie imaginaire, il vient :
\[
r_1\sin(\theta_1-\theta) + r_2\sin(\theta_2-\theta)=0
\]
donc $r_1\sin(\theta_2-\theta)= - r_1\sin(\theta_1-\theta)$ or $\theta_1-\theta \in [-\Phi,0]$ donc $\sin(\theta_1-\theta)<0$ et $\sin(\theta_2-\theta)>0$. On peut donc bien choisir un argument $\theta_2$ tel que $\theta_2-\theta \in [0,\pi]$.\newline
On a vu que $\cos(\theta_2 - \theta)=g(|z|)$ comme de plus $\theta_2-\theta \in [0,\pi]$ on a $\theta_2-\theta=\arccos g(|z|)$. Finalement :
\[
\theta_1 = \theta - \arccos f(|z|), \hspace{0.5cm}
\theta_2 = \theta + \arccos g(|z|)
\]
\item Un tel choix conduit à une configuration 1.
\item D'après 2.c., pour $r_1-r_2\leq|z|\leq r_1+r_2$, $f(|z|)\geq g(|z|)$ donc $\theta - \theta_1 \leq \theta_2-\theta$ car la fonction $\arccos$ est décroissante et 
\[
2\theta \leq \theta_1 + \theta_2.
\]
\end{enumerate}
\end{enumerate}