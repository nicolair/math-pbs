%<dscrpt>Racines sixièmes de -1.</dscrpt>
Le plan complexe $P$ est rapporté à un repère direct $(O,\overrightarrow{u},\overrightarrow{u})$.\newline
Les nombres complexes $z_1$, $z_2$, $z_3$, $z_4$, $z_5$, $z_6$ que l'on va calculer seront tous exprimés sous forme algébrique et sous forme trigonométrique.
\begin{enumerate}
\item (\emph{Question de cours}) Démontrer l'expression algébrique de $j=e^{i\frac{2\pi}{3}}$
\begin{displaymath}
 j = -\dfrac{1}{2}+i\dfrac{\sqrt{3}}{2}
\end{displaymath}

\item Résoudre l'équation
\[z^2 - \sqrt{3} z +1 = 0\]
Soit $z_1$ la solution de partie imaginaire positive et $z_2$ l'autre. Exprimer $z_1$ et $z_2$ sous forme algébrique et trigonométrique. Placer les points $M_1$ et $M_2$ d'affixes $z_1$ et $z_2$.

\item Soit $M_3$ l'image de $M_2$ par la rotation de centre O et d'angle $\frac{2\pi}{3} $. Placer $M_3$ sur la même figure et calculer son affixe notée $z_3$.

\item Soit $M_4$ l'image de $M_2$ par la translation de vecteur $\overrightarrow{w}$ dont l'affixe est
\[-\frac{\sqrt{3}+i}{2}\]
Placer le point $M_4$ sur la même figure et calculer son affixe $z_4$.

\item Soit
\begin{align*}
z_5=\frac{i}{2}(1+i\sqrt{3}) & &  z_6 = \frac{2}{i-\sqrt{3}} 
\end{align*}
Exprimer $z_5$ et $z_6$ sous forme algébrique et exponentielle. Placer les points $M_5$ et $M_6$ d'affixes $z_5$ et $z_6$ sur la figure.

\item Développer
\[(z-z_1)(z-z_2)(z-z_3)(z-z_4)(z-z_5)(z-z_6)\]
en regroupant d'abord les $z_k$ conjugués. Développer encore pour obtenir une expression très simple. Quel est l'ensemble
\begin{displaymath}
 \left\lbrace z_1,z_2,z_3,z_4,z_5,z_6\right\rbrace  ?
\end{displaymath}

\end{enumerate}
