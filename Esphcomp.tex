%<dscrpt>Paramétrisation d'une sphère complexe.</dscrpt>
On note $\Delta$ la partie de $\C^2$ formée par les couples $(u,v)$ de complexes tels que $u+v=0$ soit $\Phi$ la fonction définie par :
\begin{displaymath}
\Phi :
\left\lbrace 
 \begin{aligned}
  \C^2 \setminus \Delta &\rightarrow  \C^3 \\
 (u,v) &\rightarrow \left(\frac{1+uv}{u+v}, i\frac{1-uv}{u+v}, \frac{u-v}{u+v} \right) 
 \end{aligned}\right. 
\end{displaymath}
Pour tous nombres complexes $x$, $y$, $z$, on considère le système $(S)$ aux inconnues  $\alpha$, $\beta$, $\gamma$ complexes.
\begin{displaymath}
 (S):
\left\lbrace 
\begin{aligned}
 x\alpha + x\beta - \gamma &=& 1\\
 y\alpha + y\beta + i\gamma &=& i\\
 (z-1)\alpha + (z+1)\beta &=& 0
\end{aligned}
\right. 
\end{displaymath}
\begin{enumerate}
 \item En discutant sur $x$, $y$, $z$, préciser l'ensemble des solutions de $(S)$.
 \item On définit la partie $\Im \Phi$ de $\C^3$ par :
\begin{displaymath}
 \forall (x,y,z)\in \C^3  : \left( 
(x,y,z)\in \Im \Phi \Leftrightarrow \exists(u,v)\in \C^2\setminus \Delta \text{ tq } (x,y,z)=\Phi(u,v) \right) 
\end{displaymath}

Déterminer une équation cartésienne de $\Im \Phi$.
\item Quels sont les couples $(u,v)$ pour lesquels $\Phi(u,v)\in\R^3$ ?
\end{enumerate}
