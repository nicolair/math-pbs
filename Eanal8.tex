%<dscrpt>Continuité et dérivablité : noyau et image d'un opérateur</dscrpt>
Une rédaction très précise est exigée pour cet exercice. Les théorèmes et les objets mathématiques utilisés devront être exactement cités.\newline
Soit $E=\mathcal{C}^{0}(\R,\R)$ l'ensemble des fonctions continues de $\R$ dans $\R$ et $\mathcal{F}(\R,\R)$ l'ensemble de toutes les fonctions de $\R$ dans $\R$. On définit une fonction $\Phi$ :
\begin{displaymath}
 \fonc {\Phi}{\mathcal{C}^{0}(\R,\R)}{\mathcal{F}(\R,\R)}{f}{g}\hspace{0.4cm} \text{ avec } \fonc {g}{\R}{\R}{x}{xf(x)}.
\end{displaymath}
\begin{enumerate}
 \item Montrer que $\Phi$ est linéaire.
\item Montrer que $\Phi$ est injective.
\item Quelle propriété caractérise, pour une fonction quelconque $h$ définie dans $\R$, le fait d'être dans l'image de $\Phi$?
\end{enumerate}