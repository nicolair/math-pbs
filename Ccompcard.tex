\subsection*{I. Méthode de Cardan}
\begin{enumerate}
 \item Question de cours. Par définition $j$ est l'unique racine cubique de $1$ dont la partie imaginaire est strictement positive. Il vérifie $1+j+j^2 = 0$ et
\[
 \U_3 = \left\lbrace 1, j, j^2\right\rbrace. 
\]
Tout nombre complexe non nul admet 3 racines cubiques. Si $w$ est l'une d'entre elles, les autres sont $jw$ et $j^2w$. Par exemple, l'ensemble des racines cubique de $-\frac{p^3}{27}$ est
\[
 \left\lbrace -\frac{p}{3}, -j\,\frac{p}{3},-j^2\,\frac{p}{3}\right\rbrace .
\]

 \item
\begin{enumerate}
 \item Les racines cubiques de $U$ sont $u_0$, $ju_0$, $j^2u_0$, celles de de $V$ sont $v_0$, $jv_0$, $j^2v_0$. On peut donc former $9$ couples de racines cubiques de $U$ et $V$ mais seulement trois de ces couples vérifient la condition supplémentaire sur le produit.
\begin{displaymath}
 \mathcal{C} = \left\lbrace (u_0,v_0), (ju_0,j^2v_0), (j^2u_0,jv_0)\right\rbrace 
\end{displaymath}

 \item Soit $(u,v)\in \mathcal{C}$, d'après la formule du binôme
\begin{multline*}
 (u+v)^3 = u^3 + 3u^2v + 3uv^2 + v^3 = U +3uv(u+v) + V = S +3p_0(u+v)\\
\Rightarrow
(u+v)^3  -3p_0(u+v) -S=0
\end{multline*}
\end{enumerate}
 
 \item Toute équation du second degré dans $\C$ admet des solutions. Notons $U$ et $V$ les solutions de l'équation proposée. On a alors $U+V = -q$ et $UV = -\frac{p^3}{27}$. Ceci entraîne que les complexes $U$ et $V$ sont non nuls. Ils admettent chacun trois racines cubiques. Soit $u_0$ et $v_1$ des racines cubiques de $U$ et $V$. Alors $u_0v_1$ est une racine cubique de $UV$ donc
\begin{displaymath}
 u_0v_1 \in \left\lbrace -\frac{p}{3}, -j\frac{p}{3}, -j^2\frac{p}{3}\right\rbrace 
\end{displaymath}
 On choisit alors $v_0$ égal à $v_1$ ou à $j^2v_1$ ou à $jv_1$ pour réaliser $u_0v_0= -\frac{p}{3}=p_0$. La relation de la question précédente s'écrit alors
\begin{displaymath}
 (u+v)^3  +p(u+v) +q=0
\end{displaymath}


 \item
\begin{enumerate}
 \item D'après les questions précédentes, on peut procéder de la manière suivante (\emph{méthode de Cardan}).
\begin{itemize}
 \item Calculer les solutions de l'équation $(2)$.
 \item Choisir deux racines cubiques $u$ et $v$ de ces solutions telles que $uv=-\frac{p}{3}$.
 \item Les nombres $u+v$, $ju + j^2v$, $j^2u +jv$ sont des solutions de l'équation $(1)$.
\end{itemize}

 \item Lorsque $4p^3+27q^2$ est nul, le discriminant de l'équation $(2)$ est nul aussi. Elle admet donc une seule solution $U$. 
Les calculs précédents restent valables mais les solutions obtenues ne sont pas distinctes, elle deviennent : $2u$, $-u$, $-u$.
\end{enumerate}

 \item
\begin{enumerate}
 \item Dans le cas particulier considéré, $p=-3$, $q=1$, on forme l'équation $(2)$
\begin{displaymath}
 (2)\hspace{0.5cm} z^2 + z + 1 = 0
\end{displaymath}

 \item Les solutions de $(2)$ sont $j=e^{\frac{2i\pi}{3}}$ et $j^2=\overline{j}=e^{-\frac{2i\pi}{3}}$. Les racines cubiques des solutions dont le produit vaut $-\frac{p}{3}=1$ sont inverses l'une de l'autre c'est à dire conjuguées. On obtient
\begin{displaymath}
\mathcal{C} = \left\lbrace 
 \left( e^{\frac{2i\pi}{9}},e^{-\frac{2i\pi}{9}}\right), 
 \left( e^{\frac{8i\pi}{9}},e^{-\frac{8i\pi}{9}}\right),
 \left( e^{-\frac{4i\pi}{9}},e^{\frac{4i\pi}{9}}\right)
\right\rbrace 
\end{displaymath}
en utilisant
\begin{displaymath}
 \frac{2\pi}{9} + \frac{2\pi}{3} = \frac{8\pi}{9} \text{ et }
 \frac{2\pi}{9} - \frac{2\pi}{3} = -\frac{4\pi}{9}
\end{displaymath}

 \item La méthode donne alors trois solutions pour l'équation $(1)$ :
\begin{displaymath}
 2\cos \frac{2\pi}{9}, \hspace{0.5cm} 2\cos \frac{8\pi}{9}, \hspace{0.5cm} 2\cos \frac{4\pi}{9}
\end{displaymath}

\end{enumerate}

\end{enumerate}

\subsection*{II. Tableau de variations}
\begin{enumerate}
 \item La dérivée de $f$ est $f'(x) = 3x^2+p$. On obtient deux tableaux de variations distincts suivant le signe de $p$.

Si $p>0$. Dans ce cas $f'(x)>0$ pour tous les $x$ et la fonction est strictement croissante de $-\infty$ vers $+\infty$.

Si $p<0$. On peut former le tableau suivant
\begin{center}
\begin{tabular}{c|ccccccc}
    & $-\infty$ &            & $-\sqrt{-\frac{p}{3}}$ &           &$-\sqrt{-\frac{p}{3}}$ &            & $+\infty$\\ \hline
$f$ & $-\infty$ & $\nearrow$ &   $v_-$                &$\searrow$ &  $v_+$                & $\nearrow$ & $+\infty$
\end{tabular}
\end{center}

 \item L'équation admet toujours au moins une solution réelle à cause des limites en $-\infty$ et $+\infty$ et du théorème des valeurs intermédiaires. Les tableaux ne sont pas utiles ici, toute argumentation les invoquant est une erreur.

 \item Calculons $v_-$ et $v_+$:
\begin{displaymath}
v_- = f(-\sqrt{-\frac{p}{3}}) = \frac{p}{3}\sqrt{-\frac{p}{3}}-p\sqrt{-\frac{p}{3}} +q
=  -\frac{2p}{3}\sqrt{-\frac{p}{3}}+q
\end{displaymath}
de même :
\begin{displaymath}
v_+ = f(\sqrt{-\frac{p}{3}}) = -\frac{p}{3}\sqrt{-\frac{p}{3}}+p\sqrt{-\frac{p}{3}} +q
=  \frac{2p}{3}\sqrt{-\frac{p}{3}}+q
\end{displaymath}
On en déduit que $v_- v_= \frac{1}{3}(4p^3+27q^2)$.

Si $4p^3+27q^2<0$. Alors $p<0$ donc on est dans le cas du deuxième tableau de variations. On applique trois fois le théorème des valeurs intermédiaires. On obtient trois solutions distinctes $x_1$, $x_2$, $x_3$ telles que
\begin{displaymath}
 x_1 < -\sqrt{-\frac{p}{3}} < x_2 < \sqrt{-\frac{p}{3}} < x_3
\end{displaymath}

Si $4p^3+27q^2>0$. Deux cas sont possibles.
\begin{enumerate}
 \item Si $p>0$ alors le premier tableau de variations montre qu'il existe une seule solution réelle.
 \item Si $p<0$ alors on est dans le cas du deuxième tableau avec $v_-$ et $v_+$ de même signe. Dans ce cas aussi, une seule solution: soit avant $-\sqrt{-\frac{p}{3}}$ soit après $-\sqrt{\frac{p}{3}}$.
\end{enumerate}

\end{enumerate}
