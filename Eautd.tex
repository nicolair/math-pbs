%<dscrpt>Automorphismes du disque unité.</dscrpt>
On note $\mathcal D$ l'ensemble des nombres complexes de module strictement plus petit que $1$.
\begin{enumerate}
 \item Montrer que, pour des éléments $p$ et $z$ quelconques de $\mathcal D$, le nombre complexe $1-\overline{p}z$ est non nul et :
\begin{displaymath}
 1-\left\vert  \dfrac{p-z}{1-\overline{p}z} \right\vert ^2
=
\dfrac{(1-|z|^2)(1-|p|^2)}{|1-\overline{p}z|^2}
\end{displaymath}
Que peut-on en déduire pour 
\begin{displaymath}
 \dfrac{p-z}{1-\overline{p}z}
\end{displaymath}

\item Pour tout $p\in \mathcal D$, on considère l'application $\alpha_p$
\begin{displaymath}
 \left\lbrace 
\begin{aligned}
 \mathcal D \rightarrow& \mathcal D \\
 z \rightarrow& \dfrac{p-z}{1-\overline{p}z}
\end{aligned}
\right. 
\end{displaymath}
Montrer que cette application est bijective. Quelle est sa bijection réciproque ?
\item On suppose ici que
\begin{displaymath}
 p=(\sin \varphi) e^{i\theta} \text{ avec } \varphi\in [0,\frac{\pi}{2}[ \text{ et } \theta \in ]-\pi,\pi]
\end{displaymath}
Déterminer la forme trigonométrique des nombres complexes $z$ tels que
\begin{displaymath}
 \dfrac{p-z}{1-\overline{p}z}=z
\end{displaymath}
Un seul de ces nombres est dans $\mathcal D$, préciser lequel.
\end{enumerate}

