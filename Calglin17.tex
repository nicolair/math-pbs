\begin{enumerate}
 \item L'application $g$ est linéaire de $A$ dans $E$.
\item Montrons que le noyau de $g$ est réduit au vecteur nul:
\begin{displaymath}
 a\in \ker g \Rightarrow a+f(a)=0 \Rightarrow a=-f(a)\Rightarrow a\in A \cap B \Rightarrow a=0
\end{displaymath}
\item  L'application $g$ est injective donc $\dim g(A)= \dim A$. On en déduit que
\begin{displaymath}
 \dim g(A) + \dim B = \dim E
\end{displaymath}
D'après un exercice traité en classe (le refaire ce n'est pas directement dans le programme), pour montrer que $g(A)$ est un supplémentaire de $B$, il suffit de montrer que l'intersection est réduite au vecteur nul. Soit $x\in g(A)\cap B$. Il exista alors $a\in A$ tel que :
\begin{displaymath}
 x=a+f(a) \Rightarrow a = x -f(a) \in B
\end{displaymath}
Donc $a\in A\cap B$ donc $a$ est nul et $x$ également.
\end{enumerate}
