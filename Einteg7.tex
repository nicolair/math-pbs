%<dscrpt>Un exercice avec des intégrales.</dscrpt>
Pour tout entier $n\geq 0$, on d{\'e}finit
\[
I_{n}=\int_{0}^{1}\frac{(1+t)^{n}}{1+t^{2}}dt
\]

\begin{enumerate}
\item  Calculer $I_{0}$ et $I_{1}$.

\item  Etablir l'existence d'un polyn{\^o}me $P_{n}$ et de r{\'e}els $a_{n}$
et $b_{n}$ tels que
\[
\frac{(1+t)^{n}}{1+t^{2}}=P_{n}(t)+\frac{a_{n}+b_{n}t}{1+t^{2}}
\]
Montrer que $a_{n}$ et $b_{n}$ v{\'e}rifient la m{\^e}me relation de
r{\'e}currence lin{\'e}aire d'ordre 2 {\`a} coefficients constants. Les
exprimer au moyen de $2^{\frac{n}{2}}$, $\cos (n\frac{\pi }{4})$, $\sin (n%
\frac{\pi }{4})$.

\item  Montrer que $I_{n}$ peut s'{\'e}crire sous la forme $p_{n}+q_{n}\ln
2+r_{n}\pi $ o{\`u} $(p_{n})_{n\in \N}$, $(q_{n})_{n\in \N}$%
, $(r_{n})_{n\in \N}$ sont trois suites de nombres \emph{rationnels}%
. Pour quelles valeurs de l'entier $n$ a-t-on $q_{n}=0$.

\item  Calculer
\begin{eqnarray*}
&&I_{n+2}-2I_{n+1}+2I_{n} \\
&&q_{n+2}-2q_{n+1}+2q_{n} \\
&&r_{n+2}-2r_{n+1}+2r_{n}
\end{eqnarray*}
et {\'e}tablir une r{\'e}currence entre les $p_{n}$. Que vaut $p_{5}$ ? (on
mettra le r{\'e}sultat sous la forme d'une fraction irr{\'e}ductible)

\item  En int{\'e}grant par parties, trouver des constantes $A$ et $B$
telles que
\[
I_{n}=\frac{2^{n}}{n}(A+\frac{B}{n}+\circ (\frac{1}{n}))
\]
\end{enumerate}
