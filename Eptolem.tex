%<dscrpt>Inégalité de Ptolémée.</dscrpt>
Ce problème porte sur l'inégalité de Ptolémée (question 1.b) et ses cas d'égalité.\newline
On désigne par $m_1$, $m_2$, $m_3$, $m_4$ des nombres complexes deux à deux distincts et on introduit les notations\footnote{$B(m_1,m_2,m_3,m_4)$ est appelé le \emph{birapport} des quatre nombres complexes.} suivantes :
\begin{align*}
 P(m_1,m_2,m_3,m_4) &= (m_2-m_1)(m_4-m_3) + (m_4-m_1)(m_3-m_2) \\
 B(m_1,m_2,m_3,m_4) &= \frac{(m_2-m_1)(m_4-m_3)}{(m_4-m_1)(m_3-m_2)}
\end{align*}
\begin{align*}
 z_2 = \frac{1}{m_2-m_1}, & & z_3 = \frac{1}{m_3-m_1}, & & z_4 = \frac{1}{m_4-m_1}
\end{align*}
\begin{enumerate}
 \item 
\begin{enumerate}
\item Développer, simplifier puis factoriser $P(m_1,m_2,m_3,m_4)$.
\item Démontrer l'inégalité de Ptolémée qui porte sur les distances et se formule de la manière suivante.\newline
 Si $M_1$, $M_2$, $M_3$, $M_4$ sont des points deux à deux distincts d'un plan  alors:
\begin{displaymath}
 M_1 M_2\cdot M_3 M_4 + M_1 M_4\cdot M_2 M_3  \geq M_1 M_3\cdot M_2 M_4 
\end{displaymath}
 \end{enumerate}
 \item Cas d'égalité.
\begin{enumerate}
 \item Soit $a$ et $b$ des nombres complexes non nuls. Démontrer, \emph{par un calcul}, que
\begin{displaymath}
 |a+b| = |a| + |b| \Leftrightarrow \frac{a}{b} \in \R_+^*
\end{displaymath}
\item Formuler une condition équivalente à l'égalité dans l'inégalité de Ptolémée. On dira alors que $(M_1,M_2,M_3,M_4)$ est une configuration d'égalité.
\item Soit $s$ une similitude quelconque du plan et $(M_1,M_2,M_3,M_4)$ une configuration d'égalité. Montrer que $(s(M_1),s(M_2), s(M_3),s(M_4))$ est encore une configuration d'égalité.
\end{enumerate}
\item On suppose que les points $M_1$, $M_2$, $M_3$, $M_4$ sont \emph{dans cet ordre} sur une même droite. Montrer que $(M_1,M_2,M_3,M_4)$ est une configuration d'égalité.

\item On suppose ici que $m_1$, $m_2$, $m_3$, $m_4$ sont de module $1$ avec $m_1=e^{i\alpha_1}$, $m_2=e^{i\alpha_2}$, $m_3=e^{i\alpha_3}$, $m_4=e^{i\alpha_4}$.
\begin{enumerate}
 \item Exprimer $B(m_1,m_2,m_3,m_4)$ avec des $\sin$.
 \item Montrer que si $M_1$, $M_2$, $M_3$, $M_4$ sont sur un même cercle et dans cet ordre pour le sens trigonométrique habituel alors $(M_1,M_2,M_3,M_4)$ est une configuration d'égalité.
\end{enumerate}
\item \begin{enumerate}
\item Exprimer $B(m_1,m_2,m_3,m_4)$ à l'aide de $z_2$, $z_3$, $z_4$.
\item On considère des points $M_1$, $M_2$, $M_3$, $M_4$ d'affixe $m_1$, $m_2$, $m_3$, $m_4$ et les points $Z_2$, $Z_3$, $Z_4$ d'affixes $z_2$, $z_3$, $z_4$. Que peut-on dire de $Z_2$, $Z_3$, $Z_4$ lorsque $(M_1,M_2,M_3,M_4)$ est une configuration d'égalité?
\end{enumerate}
\end{enumerate}
