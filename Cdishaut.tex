La hauteur issue de $A$ est la droite $A+\Vect(\overrightarrow{BC}\wedge \overrightarrow{BD})$.
\begin{enumerate}
 \item La distance entre les deux droites et
\begin{displaymath}
 \frac{\left|\det(\overrightarrow{PQ},\overrightarrow u ,\overrightarrow v )\right|}
{\Vert \overrightarrow{u}\wedge \overrightarrow{v}\Vert}
\end{displaymath}

 \item La formule du double produit vectoriel et la définition du produit vectoriel avec le déterminant conduisent à
\begin{displaymath}
 (\overrightarrow{BC}\wedge \overrightarrow{BD})\wedge (\overrightarrow{AC}\wedge \overrightarrow{AD})
= \det(\overrightarrow{AC},\overrightarrow{AD},\overrightarrow{BC}) \overrightarrow{BD}
- \det(\overrightarrow{AC},\overrightarrow{AD},\overrightarrow{BD}) \overrightarrow{BC}
\end{displaymath}
Or 
\begin{displaymath}
 \det(\overrightarrow{AC},\overrightarrow{AD},\overrightarrow{BD})
= \det(\overrightarrow{AC},\overrightarrow{AD},\overrightarrow{BC})
+ \det(\overrightarrow{AC},\overrightarrow{AD},\overrightarrow{CD})
\end{displaymath}
avec  $\det(\overrightarrow{AC},\overrightarrow{AD},\overrightarrow{CD})=0$ 
car $\overrightarrow{CD} = -\overrightarrow{AC}+\overrightarrow{AD}$.\newline
De même
\begin{displaymath}
\det(\overrightarrow{AC},\overrightarrow{AD},\overrightarrow{BC}) = \det(\overrightarrow{AC},\overrightarrow{AD},\overrightarrow{BA})
= -\det(\overrightarrow{AB},\overrightarrow{AC},\overrightarrow{AD}) 
\end{displaymath}
On en déduit finalement que le vecteur proposé est égal à
\begin{displaymath}
 \det(\overrightarrow{AB},\overrightarrow{AC},\overrightarrow{AD})\overrightarrow{DC}
\end{displaymath}

 \item La hauteur issue de $A$ est dirigée par $\overrightarrow{BC}\wedge \overrightarrow{BD}$ et celle issue de $B$ par $\overrightarrow{AC}\wedge \overrightarrow{AD}$. On peut donc appliquer la formule citée en question 1 pour évaluer la distance entre ces hauteurs. L'introduction du produit mixte et le résultat de la question précédente conduisent à l'expression suivante de cette distance
\begin{displaymath}
\frac 
{\left|\det(\overrightarrow{AB},\overrightarrow{AC},\overrightarrow{AD})(\overrightarrow{DC}/\overrightarrow{AB})\right|}
{\left|\det(\overrightarrow{AB},\overrightarrow{AC},\overrightarrow{AD})DC)\right|}
= AB \,\cos \delta
\end{displaymath}
où $\delta$ est l'écart angulaire entre les vecteurs $\overrightarrow{DC}$ et $\overrightarrow{AB}$.
 \item D'après la question précédente, les quatre hauteurs se coupent deux à deux si et seulement si les arêtes opposées sont orthogonales.\newline
Montrons que lorsque les arêtes opposées sont deux à deux orthogonales les quatre hauteurs sont concourantes.\newline
Notons $I_{AB}$ le point d'intersection des hauteurs issues de $A$ et de $B$. Il est de la forme
\begin{displaymath}
I_{AB} = A + \lambda \overrightarrow{BC}\wedge\overrightarrow{BD} 
\end{displaymath}
pour un certain $\lambda$ réel. On peut déterminer ce nombre en écrivant (avec un produit scalaire) que $\overrightarrow{I_{AB}B}$ est orthogonal au plan $(ACD)$ c'est à dire à $\overrightarrow{AC}\wedge \overrightarrow{AD}$. On trouve
\begin{displaymath}
 I_{AB} = A + \frac{(\overrightarrow{AB}/\overrightarrow{AD})}{\Delta} \overrightarrow{BC}\wedge\overrightarrow{BD}
\;\text{ avec }\;
\Delta = \det(\overrightarrow{AB},\overrightarrow{AC},\overrightarrow{AD})
\end{displaymath}
Montrons maintenant que $I_{AB}$ est sur la hauteur issue de $C$ en montrant que $\overrightarrow{ CI_{AB}}$ est orthogonal au plan $(ABD)$. Il suffit pour cela de montrer que les produits scalaires
\begin{displaymath}
 (\overrightarrow{ CI_{AB}}/\overrightarrow{AB})\hspace{0.5cm} \text{ et }
(\overrightarrow{ CI_{AB}}/\overrightarrow{AD})
\end{displaymath}
sont nuls. Or
\begin{multline*}
 (\overrightarrow{ CI_{AB}}/\overrightarrow{AB})=
 (\overrightarrow{CA}/\overrightarrow{AB}) 
+ \frac{(\overrightarrow{AB}/\overrightarrow{AD})}{\Delta} (\overrightarrow{BC}\wedge \overrightarrow{BD}/\overrightarrow{AB})\\
= (\overrightarrow{CA}/\overrightarrow{AB}) + (\overrightarrow{AB}/\overrightarrow{AD})
= (\overrightarrow{AB}/\overrightarrow{CD}) = 0
\end{multline*}
Les calculs sont analogues pour le deuxième produit scalaire ainsi que pour les deux autres montrant que $I_{AB}$ est sur la quatrième hauteur.
 \item Notons $\mathcal{P}$ le plan contenant $B$, $C$, $D$ et considérons les plans contenant les hauteurs du triangle $B,C,D$ et un vecteur normal à ce plan. Ils se coupent selon une droite perpendiculaire au plan et qui perce le plan en l'orthocentre. Lorsque le point $A$ est sur cette droite, les arêtes opposées sont deux à deux orthogonales. 
\end{enumerate}
