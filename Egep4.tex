%<dscrpt> Géométrie plane: droites, courbe.</dscrpt>
Dans un plan muni d'un rep{\`e}re orthonormal $(O,\overrightarrow{i},\overrightarrow{j})$, le point $B$ est un
point variable de l'axe des ordonn{\'e}es, le point $A$ est fix{\'e} de coordonn{\'e}es $(4,0)$.

\begin{enumerate}
\item Montrer que l'ensemble $\mathcal{P}$ des points $M$ v{\'e}rifiant

\begin{displaymath}
 \left\lbrace 
\begin{aligned}
 BM =& 4 \\
\overrightarrow{MB}\cdot\overrightarrow{MA} =& 0
\end{aligned}
\right. 
\end{displaymath}
et est la r{\'e}union d'une droite $\mathcal{D}$ et d'une courbe $\mathcal{C}$

\item Construire la courbe $\mathcal{C}$.

\item Montrer que $\mathcal{C}$ est le support de la courbe param{\'e}tr{\'e}e d{\'e}finie dans $\R$
\[t\rightarrow (4\frac{1-t^2}{1+t^2},4t\frac{1-t^2}{1+t^2})\]

\item  Deux droites orthogonales passant par l'origine coupent en g{\'e}n{\'e}ral la courbe $\mathcal{C}$ en deux points $C$ et $D$ distincts.
\begin{enumerate}
\item D{\'e}terminer l'{\'e}quation de la droite $(CD)$ en fonction du coefficient directeur de l'une des droites
orthogonales.

\item Montrer que toutes les droites $(CD)$ passent par un m{\^e}me point lorsque les droites orthogonales
varient.
\end{enumerate}
\item  Deux droites passant par l'origine et sym{\'e}triques par rapport {\`a} la premi{\`e}re bissectrice des axes coupent en g{\'e}n{\'e}ral la courbe $\mathcal{C}$ en deux points distincts $E$ et $F$. \newline
D{\'e}terminer l'{\'e}quation de la droite $(EF)$ en fonction du coefficient directeur de l'une des deux droites sym{\'e}triques.

\end{enumerate}
