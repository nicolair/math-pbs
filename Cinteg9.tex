\begin{enumerate}
\item[ ]  D{\'e}composons en {\'e}l{\'e}ments simples la fraction 
\begin{displaymath}
\frac{4X-3}{X(X-2)(X+2)} 
\end{displaymath}
il vient :
\[
\frac{4X-3}{X(X-2)(X+2)}=\frac{1}{8}\left( \frac{6}{X}-\frac{11}{X+2}+\frac{5%
}{X-2}\right)
\]
On en d{\'e}duit
\begin{multline*}
\sum_{k=3}^{n}\frac{4k-3}{k(k-2)(k+2)} 
= \frac{1}{8}\left( 6\sum_{k=3}^{n}\frac{1}{k}-11\sum_{k=3}^{n}\frac{1}{k+2}+5\sum_{k=3}^{n}\frac{1}{k-2}\right)
\\
= \frac{1}{8}\left( 6\sum_{k=3}^{n}\frac{1}{k}-11\sum_{k=5}^{n+2}\frac{1}{k}+5\sum_{k=1}^{n-2}\frac{1}{k}\right)
\end{multline*}
Comme $6-11+5=0$, les termes des sommes entre $5$ et $n-2$
disparaissent. Il reste :
\[
\sum_{k=3}^{n}\frac{4k-3}{k(k-2)(k+2)}=\frac{1}{8}\left( 6(\frac{1}{3}+\frac{%
1}{4})+5(1+\frac{1}{2}+\frac{1}{3}+\frac{1}{4})-\varepsilon
_{n}\right)
\]
o{\`u} $\varepsilon _{n}$ est form{\'e} de termes qui tendent vers 0. On
en d{\'e}duit que
\[
\sum_{k=3}^{n}\frac{4k-3}{k(k-2)(k+2)}\rightarrow \frac{1}{8}\left( 6(\frac{1%
}{3}+\frac{1}{4})+5(1+\frac{1}{2}+\frac{1}{3}+\frac{1}{4})\right) =\frac{167%
}{96}
\]
\end{enumerate}
