%<dscrpt>Théorème de Sperner.</dscrpt>
Dans tout le probl{\`e}me, $E$ d{\'e}signe un ensemble {\`a} $n$ {\'e}l{\'e}ments.
\subsubsection*{Partie I. Pr{\'e}liminaire}
Pour un entier $n$ non nul, on note $\omega (n)=\max(\binom{n}{0}, \binom{n}{1}, \cdots, \binom{n}{n})$
\begin{enumerate}
\item \`A quel coefficient du binôme est égal $\omega(n)$ ?
\item
Montrer que $\omega(2n-1)=\frac{1}{2}\omega(2n)$
\end{enumerate}

\subsubsection*{Partie II. Ensembles de Sperner}
On dit qu'une partie $\mathcal{S}$ de ${\mathcal P}(E)$ est \emph{de Sperner} lorsque
\begin{displaymath}
\forall (A,B) \in {\mathcal S}^2 : A \not= B \Rightarrow
A\not\subset B \text{ et } B \not\subset A 
\end{displaymath}
\begin{enumerate}
\item
Montrer que si $p$ est un entier entre 1 et $\mathrm{Card}(E)-1$, l'ensemble $\mathcal{P}_p$ des parties  de $E$ {\`a} $p$ {\'e}l{\'e}ments est de Sperner.
\item Dans cette question, $E=\{1,\ldots,n\}$ et $a_1,\ldots,a_n$ sont des r{\'e}els strictement positifs (non n{\'e}cessairement distincts).
On d{\'e}finit une application $f$ de ${\mathcal P}(E)$ dans $\R$ en posant 
\begin{displaymath}
 \forall A \in \mathcal{P}(E) : f(A)=\sum_{i\in A}a_i
\end{displaymath}
Montrer que pour tout r{\'e}el $t$, $f^{-1}(\{t\})$ est une partie de Sperner lorsqu'elle n'est pas vide.
\item Déterminer le nombre de parties de Sperner {\`a} deux {\'e}l{\'e}ments $\mathcal{S} =\{A_1,A_2\}$.
\end{enumerate}
\subsubsection*{Partie III. Cha{\^\i}nes}
On appelle {\it cha{\^\i}ne} de $E$ une famille $(C_1,C_2,\ldots,C_n)$ de parties de $E$ telle que
\begin{align*}
    \forall i \in \{1,\ldots ,n\} &: \hbox{Card}(C_i)= i\\
    \forall i \in \{1,\ldots ,n-1\} &: C_i \subset C_{i+1}
\end{align*}
Soit $A$ est une partie fix{\'e}e {\`a} $k$ {\'e}l{\'e}ment de $E$. Calculer le nombre de chaines $(C_1,C_2,\ldots,C_n)$ telles que $C_k=A$.

\subsubsection*{Partie IV. Th{\'e}or{\`e}me de Sperner}
\begin{enumerate}
\item Soit $(C_1,C_2,\ldots,C_n)$ une chaine. Montrer que l'intersection d'une partie de Sperner avec $\{C_1,C_2,\ldots,C_n\}$ contient \emph{au plus} un élément.
\item En consid{\'e}rant toutes les cha{\^\i}nes $(C_1,C_2,\ldots,C_n)$ telles que 
\begin{displaymath}
\{C_1,C_2,\ldots,C_n\} \cap \mathcal{S} \neq \emptyset
\end{displaymath}
montrer que
\begin{displaymath}
\sum_{A \in \mathcal{S}}\dfrac{1}{ \binom{n}{\mathrm{Card} \,A}} \leq 1 
\end{displaymath}
\item
En d{\'e}duire le th{\'e}or{\`e}me de Sperner c'est {\`a} dire
\begin{displaymath}
\mathrm{Card} \mathcal{S} \leq \omega(n) 
\end{displaymath}
\end{enumerate}
