
Notons $R^{\prime }$ et $R^{\prime \prime }$ les rayons respectifs de $%
\mathcal{C}^{\prime }$ et $\mathcal{C}^{\prime \prime }$. Les
tangences aux
axes entra\^{\i }nent que les coordonn{\'e}es de $A$ et $B$ sont $%
(a,R^{\prime })$ et $(b,R^{\prime \prime })$.

%\begin{floatingfigure}{7cm}
  %\centering
%  \incluregraphics[width=7cm]{Cgep3f1.pdf} include au lieu de inclure
%  \caption{$a=3, b=\frac{6}{5}$}
%\end{floatingfigure}

La tangence de $\mathcal{C}$ et $\mathcal{C}^{\prime }$ se traduit par $%
CA=R^{\prime }+1$ soit : $a^{2}+(R^{\prime }-1)^{2}=(R^{\prime }+1)^{2}$.%
\newline
La tangence de $\mathcal{C}$ et $\mathcal{C}^{\prime \prime }$ se
traduit par $CB=R^{\prime \prime }+1$ soit : $a^{2}+(R^{\prime
\prime }-1)^{2}=(R^{\prime \prime }+1)^{2}$.\newline La tangence
de $\mathcal{C}^{\prime }$ et $\mathcal{C}^{\prime \prime }$ se
traduit par $AB=R^{\prime }+R^{\prime \prime }$ soit :
$(a-b)^{2}+(R^{\prime }-R^{\prime \prime })^{2}=(R^{\prime
}+R^{\prime \prime })^{2}.$ \newline On cherche {\`a} {\'e}liminer
$R^{\prime }$ et $R^{\prime \prime }$ entre
ces trois {\'e}quations, c'est {\`a} dire {\`a} former une relation entra\^{%
\i }n{\'e}e par ce syst{\`e}me et o{\`u} ne figurent plus ni $R^{\prime }$ ni
$R^{\prime \prime }$. De (1) et (2), on tire $R^{\prime
}=\frac{1}{4}a^{2} $, $R^{\prime \prime }=\frac{1}{4}b^{2}$. En
reportant dans (3), il vient
\[
(a-b)^{2}+\left( \frac{a^{2}-b^{2}}{4}\right) ^{2}=\left( \frac{a^{2}+b^{2}}{%
4}\right) ^{2}
\]
On en d{\'e}duit $(a-b)^{2}=\frac{1}{4}a^{2}b^{2}$ puis, en tenant
compte de $b<a$, $a-b=\frac{1}{2}ab$ et finalement
\[
b=\frac{2a}{a+2}
\]
