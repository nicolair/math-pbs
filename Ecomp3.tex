%<dscrpt>Théorème de l'angle au centre par un calcul complexe.</dscrpt>
On considère trois nombres réels $\alpha_1$, $\alpha_2$, $\alpha_3$ dans $]-\pi, \pi]$ avec $\alpha_1 <\alpha_2$. On pose 
\begin{displaymath}
z_1=e^{-\alpha_1},\;z_2=e^{-\alpha_2},\;z_3=e^{-\alpha_3} 
\end{displaymath}
En discutant suivant la position de $\alpha_3$, préciser le module et un argument de
\[\frac{z_2-z_3}{z_1-z_3}\] 
En déduire un théorème connu de géométrie.