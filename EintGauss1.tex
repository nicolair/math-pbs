%<dscrpt>Calcul de l'intégrale de Gauss.</dscrpt>
Définissons diverses fonctions dans $\R$. Pour tout réel $x$:
\begin{displaymath}
\lambda (x) = \int_{0}^{x}e^{-t^{2}}\ dt,
\hspace{0.5cm}
f(x) = \int_{0}^{1}\frac{e^{-x(1+t^{2})}}{1+t^{2}}\ dt ,
\hspace{0.5cm}
g(x) = \int_{0}^{1}e^{-x(1+t^{2})}\ dt
\end{displaymath}
Notons que $\lambda$ est l'unique primitive de $t\in \R\mapsto e^{-t^{2}}$ s'annulant en $0$.\newline
Le but de cet exercice est de démontrer que:
\begin{displaymath}
\lim_{x\to + \infty} \lambda (x) = \frac{\sqrt{\pi}}{2} \hspace{0.5cm} \text{ (intégrale de Gauss)}  
\end{displaymath}

\begin{enumerate}
\item Soit $a\in [1,2]$. Définissons une fonction $\varphi$ dans $\R$ par :
\[
\forall x \in \R, \; \varphi (x) = e^{-ax} - 1 + ax .
\]
On pourra utiliser une formule de Taylor à préciser.

\begin{enumerate}
\item Montrer que $\varphi$ est à valeurs positives sur $\R$.
\item Montrer que pour tout $x$ réel:
\[
x\geq  -\frac{\ln (2)}{a} \Rightarrow  \varphi(x) \leq a^2x^2 \Rightarrow \left|\frac{e^{-ax} - 1}{x} + a\right|\leq a^{2}|x| \text{ pour } x\neq 0.
\]
\end{enumerate}
Ces inégalités permettent de montrer que $f$ est dérivable sur $\R_{+}$ et que:
\[\forall x\in \R_{+},\ f'(x) = -g(x).\]
Cette propriété est \emph{admise} et sera utile dans la fin du problème.

\item Pour tout $x\in \R_{+}$, posons:
\[h(x) = f(x^{2}) + \lambda (x)^{2}.\]
\begin{enumerate}
\item Calculer $h(0)$.
\item Montrer que pour tout $x>0$:
\[\lambda (x) = x\int_{0}^{1}e^{-x^{2}t^{2}}\ dt.\]
\item En déduire que $h$ est constante sur $\R_{+}$.
\item Montrer que pour tout $x\in \R_{+}$, $0\leq f(x) \leq e^{-x}$. En déduire la limite de $f(x)$ quand $x$ tend vers $+\infty$.
\item Montrer enfin que:
\[\lim_{x\to + \infty} \lambda (x) = \frac{\sqrt{\pi}}{2}.\]

\end{enumerate}

\end{enumerate}

