%<dscrpt>Presque un sous-anneau !</dscrpt>
Dans cet exercice, $E$ est un espace vectoriel sur un corps $\K$.
On rappelle que les applications linéaires de $E$ dans $\K$ sont aussi appelées \emph{formes linéaires} et que leur ensemble peut être noté $E^*$. \newline 
On admet que pour tout élément non nul $a\in E$, il existe des formes linéaires $\alpha \in \mathcal{L}(E,\K)=E^*$ telles que $\alpha(a)\neq 0_K$.\newline
Pour tout $u\in E$ et tout $\alpha\in E^*$, on définit une fonction $f_{\alpha, u}$ par :
\begin{displaymath}
 f_{\alpha, u} :
\left\lbrace 
\begin{aligned}
 &E\rightarrow E\\ &x \mapsto \alpha(x)u
\end{aligned}
\right. 
\end{displaymath}
On considère un sous-espace vectoriel fixé $A$ de $E$ qui n'est ni $\{0_E\}$ ni $E$. On note $\mathcal{A}$ l'ensemble des endomorphismes de $E$ dont l'image est incluse dans $A$.
\begin{displaymath}
 \forall f\in\mathcal{L}(E), \left( f\in \mathcal{A} \Leftrightarrow \Im f \subset A\right) 
\end{displaymath}
\begin{enumerate}
 \item Vérifier que $\mathcal{A}$ est stable pour l'addition, la multiplication par un élément de $\K$ et la composition.
 \item 
\begin{enumerate}
 \item Soit $u\in E$ et $\alpha\in E^*$, montrer que $f_{\alpha, u}\in \mathcal{L}(E)$. Si $u\neq 0_E$ et $\alpha \neq O_{E^*}$, quels sont les $v\in E$ et $\beta\in E^*$ tels que  $f_{\alpha, u} = f_{\beta, v}$.
 \item Soit $u$ et $v$ dans $E$, $\alpha$ et $\beta$ dans $E^*$. Déterminer $\gamma \in E^*$ et $w\in E$ tels que $f_{\alpha, u} \circ f_{\beta, v} = f_{\gamma, w}$.
\item Dans quel cas $f_{\alpha, u}$ appartient-il à $\mathcal{A}$? En déduire que $\mathcal{A}$ ne se réduit pas à la fonction nulle. 
\end{enumerate}
\item Supposons qu'il existe un élément neutre $e$ pour la composition dans $\mathcal{A}$.
\begin{enumerate}
  \item Montrer que $\ker e$ et $\Im e$ sont supplémentaires dans $E$.
  \item Montrer que $\Im e = A$.
  \item Montrer que $\ker e \subset \ker f$ pour tout $f\in\mathcal{A}$.
  \item Soit $a$ un élément non nul de $A$ et $b$ un vecteur non nul dans $\ker e$. Il existe alors $\alpha\in E^*$ telle que $\alpha(b)\neq 0_K$. Montrer que $f_{\alpha, a}\in \mathcal A$. Que peut-on en conclure ?
\end{enumerate}
\item Supposons qu'il existe un supplémentaire  $B$ de $A$ et notons $\mathcal{A}'$ la partie de $\mathcal{A}$ définie par:
\begin{displaymath}
 \forall f\in\mathcal{L}(E),\left(  f\in \mathcal{A}' \Leftrightarrow \Im f \subset A \text{ et } B\subset \ker f\right) 
\end{displaymath}
\begin{enumerate}
 \item Montrer que $\mathcal{A}'$ admet un élément neutre pour la composition. Préciser la nature de cette application.
 \item Montrer que $\mathcal{A}'$ est un anneau pour l'addition et la composition. Est-il un sous-anneau de $\mathcal{L}(E)$? 
\end{enumerate}
\end{enumerate}
