%<dscrpt>Complément de l'inégalité de Cauchy-Schwarz.</dscrpt>
Soit $n$ naturel non nul et $a_1, \cdots, a_n,\, b_1,\cdots ,b_n$ des réels strictement positifs.\newline
On définit $m$, $M$, $a$, $g$ par
\[
 m = \min(\frac{a_1}{b_1}, \cdots, \frac{a_n}{b_n}), \; M = \max(\frac{a_1}{b_1}, \cdots, \frac{a_n}{b_n}),\; a = \frac{1}{2}(m + M), \; g = \sqrt{m M}.
\]

\begin{enumerate}
 \item Soit $u$ et $v$ réels strictement positifs. Montrer que $\sqrt{u v} \leq \frac{1}{2}(u + v)$.\newline
Que peut-on en déduire pour $\frac{a}{g}$?

 \item Pour $x$ réel, factoriser $f(x) = -x^2 + (m+M)x -mM$. En déduire
\[
 \forall k \in \llbracket 1, n \rrbracket, \; f(\frac{a_k}{b_k}) \geq 0.
\]

 \item 
 \begin{enumerate}
 \item Montrer que 
\[
 \sum_{k=1}^{n}a_k^2 + m\,M \sum_{k=1}^{n}b_k^2 \leq (m+M) \sum_{k=1}^{n}a_kb_k. 
\]
  \item En déduire
\[
 \sqrt{\sum_{k=1}^{n}a_k^2}\, \sqrt{\sum_{k=1}^{n}b_k^2} \leq \frac{a}{g}\sum_{k=1}^{n}a_kb_k. 
\]

 \end{enumerate}


\end{enumerate}

