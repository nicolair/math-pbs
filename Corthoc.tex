\begin{figure}
   \centering
   \input{Corthoc_1.pdf_t}
   \caption{Orthocentre d'un triangle inscrit sur une hyperbole}
   \label{fig:Corthoc_1}
\end{figure}

\begin{enumerate}
\item On considère trois points quelconques $(A,B,C)$ dans le plan. Les coordonnées de ces points sont respectivement $(a,a'), (b,b'), (c,c')$. Un point quelconque $M$ est sur la hauteur issue de $A$ si et seulement si $\overrightarrow{AM}$ est colinéaire à un vecteur orthogonal à $\overrightarrow{BC}$. Le calcul des coordonnées de ce vecteur donne l'équation
\[\left \vert
\begin{matrix}
x(M)-a & b'-c'\\
y(M)-a'& c-b
\end{matrix}
\right \vert
=0
\]
\item Les trois points $(A,B,C)$ sont sur l'hyperbole $\Gamma$, on en déduit que leurs ordonnées sont respectivement :
\[a'=\frac{k}{a},b'=\frac{k}{b},c'=\frac{k}{c}\]
Les coordonnées de l'orthocentre sont solutions du système aux deux inconnues $x(H)$ et $y(H)$ formé à partir des équations des hauteurs issues de $A$ et de $B$.
\begin{eqnarray*}
\left \vert
\begin{matrix}
x(H)-a & -k(\frac{1}{b}-\frac{1}{c})\\
y(H)-\frac{k}{a}& b-c
\end{matrix}
\right \vert
=0 \\
\left \vert
\begin{matrix}
x(H)-b & -k(\frac{1}{c}-\frac{1}{a})\\
y(H)-\frac{k}{b}& c-a
\end{matrix}
\right \vert
=0
\end{eqnarray*}
En développant les déterminants, le système devient
\[
\left\lbrace
\begin{matrix}
(b-c)x+k(\frac{1}{b}-\frac{1}{c})y &=& a(b-c)+\frac{k^2}{a}(\frac{1}{b}-\frac{1}{c}) \\
(c-a)x+k(\frac{1}{c}-\frac{1}{a})y &=& b(c-a)+\frac{k^2}{b}(\frac{1}{c}-\frac{1}{a})
\end{matrix}
\right.
\]
On divise la première ligne par $b-c$, la seconde par $c-a$ puis on multiplie les deux par $abc$. Le système devient :
\[
\left\lbrace
\begin{matrix}
abcx-aky &=& a^2bc-k^2 \\
abcx-bky &=& ab^2c-k^2
\end{matrix}
\right.
\]
On résoud ce système par les formules de Cramer. Le déterminant $D$ au dénominateur est :
\[
D =
\left\vert
\begin{matrix}
abc & -ak \\
abc & -bk
\end{matrix}
\right\vert
= abck
\left\vert
\begin{matrix}
1 & -a \\
1 & -b
\end{matrix}
\right\vert
=abck(a-b)
\]
Les déterminants donnant $x(H)$ et $y(H)$ s'écrivent :
\begin{eqnarray*}
Dx(H)=
\left\vert
\begin{matrix}
a^2bc-k^2 & -ak \\
ab^2c-k^2 & -bk
\end{matrix}
\right\vert
=(b-a)k^3 \\
Dy(H)=
\left\vert
\begin{matrix}
abc & a^2bc-k^2 \\
abc & ab^2c-k^2
\end{matrix}
\right\vert
=(abc)^2(b-a)
\end{eqnarray*}
On en déduit finalement les coordonnées de l'orthocentre
\begin{eqnarray*}
x(H) &=& -\frac{k^2}{abc}\\
y(H) &=& -\frac{abc}{k}
\end{eqnarray*}
Ces expressions montrent bien que $x(H)y(H)=k$ c'est à dire $H$ est encore sur l'hyperbole. On a démontré la proposition suivante
\begin{quotation}
L'orthocentre d'un triangle inscrit sur une hyperbole équilatère est encore sur l'hyperbole.
\end{quotation} 
\end{enumerate}
