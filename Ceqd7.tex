\begin{enumerate}
\item \begin{enumerate}
\item L'équation de la tangente $T_\lambda$ en $(1,\lambda)$ à la courbe de $f$ est
\[ y-\lambda =f'(1)(x-1)\]
Comme $f$ est solution de l'équation differentielle 
\[f'(1)=\frac{1-2\lambda}{2}\]
l'équation s'écrit
\[(1-2\lambda)x-2y-1+4\lambda=0\]
\item Réordonnons suivant $\lambda$ l'équation précédente:
\[(2(1-x)+2)\lambda+x-1-2y=0\]
Lorsque les coordonnées d'un point annulent les deux coefficients, ce point est sur toutes les droites. C'est le cas ici pour le point de coordonnées
\[(2,\frac{1}{2})\] 
\end{enumerate}
\item Résolution de
\[(1+x^2)y'(x)+2xy(x)=\frac{1}{x}\]
Une primitive de $\frac{-2x}{1+x^2}$ est $-\ln(1+x^2)$. Une solution de l'équation sans second membre est donc
\[\frac{1}{1+x^2}\]
On cherche une solution particulière de l'équation complète par la méthode de variation de la constante sous la forme
\[\frac{\lambda(x)}{1+x^2}\]
Cela donne
\[(1+x^2)\frac{\lambda'(x)}{1+x^2}=\frac{1}{x}\]
d'où $\lambda(x)=\ln x$. On en déduit que l'ensemble des solutions est
\[\left\lbrace x\rightarrow \frac{\lambda + \ln x}{1+x^2}, \lambda \in \R \right\rbrace \]
En particulier
\[f-\lambda (x)=\frac{2\lambda + \ln x}{1+x^2}\]
\item Comme en 1., l'équation de $D_\lambda$ est
\[y-\lambda=f'(x_0)(x-x_0)\]
avec
\[f'(x_0)=\frac{c(x_0)-\lambda b(x_0)}{a(x_0)}\]
elle s'écrit encore
\[a(x_0)(y-\lambda)(c(x_0)-\lambda b(x_0))(x-x_0)\]
On réordonne suivant $\lambda$ :
\[(b(x_0)(x-x_0)-a(x_0))\lambda+a(x_0)y-c(x_0)(x-x_0)=0\]
On considère le système de deux équations
\begin{eqnarray*}
\left\lbrace \begin{array}{ccc}
b(x_0)(x-x_0)-a(x_0) & = & 0 \\ 
c(x_0)(x-x_0) & = & 0
\end{array} \right.
\end{eqnarray*} 
\begin{itemize}
\item[Si $b(x_0)\neq 0$ : ] toutes les droites $D_\lambda$ passent par le point de coordonnées
\[(x_0+\frac{a(x_0)}{b(x_0)})\]
\item[Si $b(x_0)= 0$ : ] l'équation de $D_\lambda$ s'écrit
\[c(x_0)x-a(x_0)y+\lambda a(x_0)-c(x_0)x_0=0\]
toutes les droites $D_\lambda$ sont paralleles.
\end{itemize}
\item 
\begin{enumerate}
 \item On considère deux solutions $f_0$ et $f_1$ de l'équation $(E)$ avec $f_0(x_0)=y_0$ et  $f_1(x_0)=y_1$. Les équations des tangentes sont :
\begin{align*}
 \begin{vmatrix}
  x-x_0 & 1 \\
  y-y_0 & F(x_0,y_0)
 \end{vmatrix} = 0
& &
 \begin{vmatrix}
  x-x_0 & 1 \\
  y-y_1 & F(x_0,y_1)
 \end{vmatrix} = 0
\end{align*}
 Ces tangentes sont paralleles si et seulement si les vecteurs directeurs sont colinéaires c'est à dire ici lorsque $F(x_0,y_0)=F(x_0,y_1)$. Dans ce cas :
\begin{displaymath}
 \frac{F(x_0,y_0)=F(x_0,y_1)}{y_0 - y_1} = 0
\end{displaymath}
Si les tangentes sont toutes concourantes, il existe un point $I$ de coordonnées $(x_I,y_I)$ qui appartient à toutes les tangentes. On peut remarquer que $x_I\neq x_0$ car chaque tangente passe par un point d'abscisse $x_0$ et d'ordonnée différente. \'Ecrivons que $I$ appartient aux tangentes en $(x_0,y_0)$ et en $(x_0,y_1)$.
\begin{align*}
 \begin{vmatrix}
  x_I-x_0 & 1 \\
  y_I-y_0 & F(x_0,y_0)
 \end{vmatrix} = 0
& &
 \begin{vmatrix}
  x_I-x_0 & 1 \\
  y_I-y_1 & F(x_0,y_1)
 \end{vmatrix} = 0
\end{align*}
On en déduit :
\begin{multline*}
 \left. 
\begin{aligned}
 (x_I-x_0)F(x_0,y_0) = y_I-y_0\\
 (x_I-x_0)F(x_0,y_1) = y_I-y_1
\end{aligned}
\right\rbrace \\
\Rightarrow
 (x_I-x_0)F(x_0,y_0) + y_0 = (x_I-x_0)F(x_0,y_1) + y_1 \\
\Rightarrow
(x_I-x_0)(F(x_0,y_0)-F(x_0,y_1)) = y_1 - y_0 \\
\Rightarrow
\frac{F(x_0,y_0)-F(x_0,y_1)}{y_1-y_0} = \frac{1}{x_0 - x_I}
\end{multline*}

 \item \`A la question précédente, on a vu que l'on pouvait noter 
\begin{displaymath}
 \frac{F(x_0,y_0)-F(x_0,y_1)}{y_1-y_0} = \alpha(x_0)
\end{displaymath}
On en déduit
\begin{multline*}
 F(x_0,y_0)-F(x_0,y_1)= \alpha(x_0)(y_0 - y_1) \\
\Rightarrow
F(x_0,y_0) -\alpha(x_0)y_0 = F(x_0,y_1) -\alpha(x_0)y_1 
\end{multline*}
Ceci montre que $F(x_0,y)-\alpha(x_0)y$ est indépendant de $y$. On note :
\begin{displaymath}
 \beta(x_0) = F(x_0,y)-\alpha(x_0)y
\end{displaymath}

 \item D'après l'expression de la fonction $F$ trouvée au dessus,  l'équation $(E)$ est linéaire c'est à dire qu'elle est de la forme :
\begin{displaymath}
 \forall x\in I : y'(x) = \alpha(x)y(x) + \beta(x)
\end{displaymath}

\end{enumerate}

\end{enumerate}
