%<dscrpt>Strophoïde et cissoïde.</dscrpt>
L'objet du probl{\`e}me\footnote{d'après Ecole de l'Air MP 2002} est
la recherche de lieux g{\'e}om{\'e}triques conduisant {\`a}
l'{\'e}tude de courbes planes (appel{\'e}es en g{\'e}n{\'e}ral
cubiques circulaires). Les parties I et II donnent deux exemples de
telles courbes. Dans la troisi{\`e}me partie, on consid{\`e}re le
cas g{\'e}n{\'e}ral.

Dans toute la suite, le plan est rapport{\'e} {\`a} un rep{\`e}re
orthonorm{\'e} d'origine not{\'e}e $O$, d'axes $Ox$ et $Oy$ et on
d{\'e}signe par $a$ un nombre r{\'e}el strictement positif
donn{\'e}.

\subsection*{PARTIE I. \'Etude de la stropho{\"i}de droite}
\begin{figure}[ht]
 \centering
  \input{Ecubcirc_1.pdf_t}
\caption{Définition de la strophoïde droite}
\end{figure}

On d{\'e}signe par $D$ la droite d'{\'e}quation $x = 2a$ et par $C$
le cercle de centre $M_{0}$ de coordonnées $(-2a, 0)$, de rayon $2a$.\newline
Pour tout nombre r{\'e}el $\theta$, on d{\'e}signera par :
\begin{itemize}
\item $H(\theta)$ le point d'intersection, lorsqu'il existe, de la droite (notée $D_\theta$) d'angle polaire $\theta$ et de la droite $D$.
\item $M(\theta)$ le point d'intersection de la droite d'angle polaire $\theta$ et du cercle $C$ (avec la convention que lorsqu'il y a deux points d'intersection, $M(\theta)$ d{\'e}signe le point d'intersection distinct de $O$).
\end{itemize}

\begin{enumerate}
\item Donner une {\'e}quation cart{\'e}sienne, puis une {\'e}quation polaire du cercle $C$.
\item D{\'e}terminer des coordonn{\'e}es polaires de $M(\theta)$ et $H(\theta)$, puis du milieu $I(\theta)$ du segment $[M(\theta)H(\theta)]$.\newline
En d{\'e}duire, lorsque $\theta$ varie, que $I(\theta)$ d{\'e}crit la courbe d'{\'e}quation polaire :
$$r(\theta) = -a \dfrac{\cos{(2\theta)}}{\cos{(\theta)}}.$$

\item Exprimer $r(\theta + 2\pi)$, $r(\pi + \theta)$, $r(-\theta)$  en fonction de $r(\theta)$. Interpr{\'e}ter g{\'e}om{\'e}triquement ces r{\'e}sultats et indiquer sur quelle partie $E$ de $\R$ il suffit d'{\'e}tudier la courbe.

\item D{\'e}terminer la limite de $r(\theta) \sin{(\theta - \pi/2)}$ lorsque $\theta$ tend vers $\pi/2$. Qu'en d{\'e}duit-on g{\'e}om{\'e}triquement ?

\item \'Etudier le signe de $r(\theta)$ pour $\theta \in E$, repr{\'e}senter sur une m{\^e}me figure la droite $D$, le cercle $C$, et
le support de cette courbe $\theta \mapsto I(\theta)$.


\item Donner enfin une {\'e}quation cart{\'e}sienne du support de la courbe $\theta \mapsto I(\theta)$.
\end{enumerate}

\subsection*{PARTIE II. \'Etude de la cisso{\"i}de droite}
\begin{figure}[ht]
 \centering
  \input{Ecubcirc_2.pdf_t}
\caption{Définition de la cissoïde droite}
\end{figure}

\medskip
On d{\'e}signe par $D$ la droite d'{\'e}quation $x = 2a$ et par $C$
le cercle de centre $M_{0} (-a, 0)$, de rayon $a$.\newline
Pour tout nombre r{\'e}el $t$, on d{\'e}signera par :
\begin{itemize}
\item $H(t)$ le point d'intersection de la droite (notée $D_t$) d'{\'e}quation $y = tx$ et de la droite $D$.
\item $M(t)$ le point d'intersection de la droite d'{\'e}quation $y = tx$ et du cercle $C$ (avec la convention que lorsqu'il y a deux points d'intersection, $M(t)$ d{\'e}signe le point d'intersection distinct de $O$).
\end{itemize}
\begin{enumerate}
\item Donner une {\'e}quation cart{\'e}sienne du cercle $C$.
\item D{\'e}terminer les coordonn{\'e}es de $M(t)$ et $H(t)$, puis du milieu $J(t)$ du segment $[M(t)H(t)]$.
\item D{\'e}terminer le vecteur-d{\'e}riv{\'e} {\`a} la courbe $t \mapsto J(t)$, puis en d{\'e}duire le(s) point(s) stationnaire(s)
 (c'est-{\`a}-dire non r{\'e}gulier(s)) de celle-ci. Etudier la tangente au(x) point(s) stationnaire(s).

En d{\'e}duire que la tangente {\`a} la courbe $t \mapsto J(t)$ au
point $J(t_{0})$ a pour {\'e}quation 
$$ t_{0} (t_{0}^{2} + 3)x - 2y = a t_{0}^{3} $$

\item Etudier les variations des coordonn{\'e}es $x(t)$, $y(t)$ du point $J(t)$ pour $t \in \R_{+}$, et repr{\'e}senter sur une m{\^e}me figure la droite $D$, le cercle $C$, et le support de cette courbe $t \mapsto J(t)$.

\item Donner enfin une {\'e}quation cart{\'e}sienne du support de la courbe $t \mapsto J(t)$.
\end{enumerate}



\subsection*{PARTIE III. \'Etude g{\'e}n{\'e}rale des cubiques
circulaires}

On consid{\`e}re un point $M_{0} (x_{0}, y_{0})$ distinct de $O$ et
on d{\'e}signe alors par $D$ la droite d'{\'e}quation $x = 2a$ et
par $C (x_{0}, y_{0})$ le cercle de centre $M_{0} (x_{0}, y_{0})$
passant par $O$.\newline
Pour tout nombre r{\'e}el $t$, on d{\'e}signera par :
\begin{itemize}
\item $H(t)$ le point d'intersection de la droite (notée $D_t$) d'{\'e}quation $y = tx$ et de la droite $D$.
\item $M(t)$ le point d'intersection de la droite d'{\'e}quation $y = tx$ et du cercle $C (x_{0}, y_{0})$ (avec la convention que lorsqu'il y a deux points d'intersection, $M(t)$ d{\'e}signe le point d'intersection distinct de $O$).
\end{itemize}
\begin{enumerate}
\item D{\'e}terminer les coordonn{\'e}es de $M(t)$ et $H(t)$, puis du milieu $K(t)$ du segment $[M(t)H(t)]$.
\item \'Etudier les branches infinies de la courbe $t \mapsto K(t)$.
\item \`A quelle condition sur $M_{0} (x_{0}, y_{0})$ l'origine $O$ appartient-elle au support de la courbe paramétrée $t \mapsto K(t)$ ?\newline
Repr{\'e}senter graphiquement la zone du plan correspondante.

\item \`A quelle condition sur $M_{0} (x_{0}, y_{0})$ la courbe a-t-elle un point double ? \newline
Quel est alors ce point double ? Repr{\'e}senter graphiquement la
zone du plan correspondante.

\item \`A quelle condition sur $M_{0} (x_{0}, y_{0})$ la courbe a-t-elle un point stationnaire (c'est {\`a} dire non r{\'e}gulier) ?\newline
Quel est alors ce point stationnaire ? Repr{\'e}senter graphiquement la zone du plan correspondante.
\end{enumerate}
