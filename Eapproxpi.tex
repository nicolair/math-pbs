%<dscrpt>Approximation de pi, accélération de convergence.</dscrpt>
On définit\footnote{d'après E.S.M de Saint Cyr 1995} des suites $\left( c_n\right) _{n\in \N^*}$ et $\left( \lambda_n\right) _{n\in \N^*}$ par les relations
\begin{align*}
 c_1 = 0, \; c_{n+1}=\sqrt{\frac{1+c_n}{2}}
& &
\lambda_1 = 2,\; \lambda_{n+1} = \frac{\lambda_n}{c_{n+1}}
\end{align*}

\begin{enumerate}
 \item 
\begin{enumerate}
\item Montrer que, pour tout entier $n$ non nul, il existe $\theta_n\in [0,\frac{\pi}{2}]$ et $\alpha_n\in\R^+$ tels que
\begin{displaymath}
 c_n = \cos(\theta_n),\hspace{1cm} \lambda_n = \alpha_n \sin \theta_n
\end{displaymath}
\item Exprimer $\theta_n$ en fonction de $n$. Vérifier que $\left( \alpha_n\right) _{n\in \N^*}$ est géométrique. En déduire que $\left( \lambda_n\right) _{n\in \N^*}$ converge vers $\pi$.
\end{enumerate}
  
 \item En utilisant une formule de Taylor, (préciser laquelle) montrer l'inégalité
\begin{displaymath}
 |\pi - \lambda_n| \leq \frac{\pi^3}{6\times 4^n}
\end{displaymath}

 \item Soit $p\in \N^*$ fixé. Montrer que $\left( \lambda_n\right) _{n\in \N^*}$ admet le développement
\begin{displaymath}
 \lambda_n = \pi - \frac{\pi^3}{3!}\frac{1}{4^n} + \frac{\pi^5}{5!}\frac{1}{4^{2n}} +\cdots 
+(-1)^p\frac{\pi^{2p+1}}{(2p+1)!}\frac{1}{4^{pn}} + o(\frac{1}{4^{pn}})
\end{displaymath}

 \item Accélération de convergence.\newline
On définit une nouvelle suite $\left( \lambda_n^{(1)}\right) _{n\in \N^*}$ par
\begin{displaymath}
 \lambda_n^{(1)} = \frac{1}{3}\left( -\lambda_n + 4\lambda_{n+1}\right) 
\end{displaymath}
\begin{enumerate} 
 \item Donner un équivalent de $\lambda_n^{(1)} - \pi$. En déduire que $\lambda_n^{(1)} - \pi$ est négligeable devant $\lambda_n - \pi$ lorsque $n$ tend vers l'infini. 
 \item Déterminer un réel $\alpha$ tel que la suite $\left( \lambda_n^{(2)}\right) _{n\in \N^*}$ définie par
\begin{displaymath}
 \lambda_n^{(2)} = \alpha \lambda_n^{(1)} + (1-\alpha)\lambda_{n+1}^{(1)}
\end{displaymath}
vérifie, lorsque $n$ tend vers l'infini,
\begin{displaymath}
 \lambda_n^{(2)} - \pi \in o(\lambda_n^{(1)} - \pi)
\end{displaymath}

\end{enumerate}

\end{enumerate}
