%<dscrpt>Variables aléatoires: calculs de covariance.</dscrpt>
\noindent Le but du probl\`eme\footnote{d'après ESSEC, option éco, maths 2, 2001} est l'\'etude du coefficient de corr\'elation lin\'e%
aire de deux variables al\'eatoires qu'on aborde d'abord de fa\c con g\'en\'erale (%
\textbf{partie I}), puis dans un cas particulier (\textbf{partie II}).

\subsection*{PARTIE I}

On consid\`ere deux variables al\'eatoires $X$ et $Y$ d\'efinies sur un m\^eme
espace probabilis\'e fini. La fonction probabilité est notée $P$.\\
 On rappelle que $V(X)=0$
si et seulement si, avec une probabilit\'e \'egale \`a $1$, $X$ est constante ; on dit alors que $X$ est constante presque surement.\\
On suppose ici que $V(X)>0$. 

\begin{enumerate}
\item \textbf{Covariance des variables al\'eatoires }$X$\textbf{\ et }$Y$

\begin{enumerate}
\item Donner la définition de la covariance de $X$ et $Y$, notée $\mathrm{Cov}(X,Y)$ . 
\item Soit $\lambda$ un réel. Exprimer $\mathrm{Cov}(\lambda X+Y,\lambda X+Y)$ en fonction de $V(\lambda X+Y)$
et en d\'eduire la formule :
\begin{equation*}
V(\lambda X+Y)=\lambda ^{2}V(X)+2\lambda \mathrm{Cov}(X,Y)+V(Y).
\end{equation*}

\item En d\'eduire que $$\left(\mathrm{Cov}(X,Y)\right)^{2}\leqslant V(X)V(Y).$$
A quelle condition n\'ecessaire et suffisante a-t-on l'\'egalit\'e $$%
(\mathrm{Cov}(X,Y))^{2}=V(X)V(Y)$$
\end{enumerate}

\item \textbf{Coefficient de corr\'elation lin\'eaire des variables al\'eatoires }$%
X$\textbf{\ et }$Y.$\newline
On suppose dans cette question les variances $V(X)$ et $V(Y)$ strictement positives. On rappelle que  le coefficient de corr\'elation lin\'eaire $\rho $ des variables
al\'eatoires $X$ et $Y$ est 
 $$\rho =\frac{\mathrm{Cov}%
\left( X,Y\right) }{\sigma \left( X\right) \sigma \left( Y\right) }.$$

\begin{enumerate}
\item  
 Montrer que $\rho \in[-1,+1]$. Pr\'eciser, de plus, \`a quelle condition n\'ecessaire et suffisante $\rho $ est \'egal \`a $-1$ ou $+1$.

\item Quelle est la valeur de $\rho $ lorsque les variables al\'eatoires $X$ et $Y$ sont ind\'ependantes ?

\end{enumerate}
\end{enumerate}

\subsection*{PARTIE II}

\begin{enumerate}
\item \textbf{Calculs pr\'eliminaires}

\begin{enumerate}
\item On consid\`ere deux entiers naturels $q$ et $n$ tels que $n\geqslant q$. \'Etablir la formule
suivante :
\begin{equation*}
\sum\limits_{k=q}^{n}\binom{k}{q}=\binom{n+1}{q+1}
\end{equation*}

\item En d\'eduire une expression factoris\'ee des quatre
sommes suivantes :%
\begin{equation*}
\sum\limits_{k=1}^{n}k,\quad \quad \sum\limits_{k=2}^{n}k(k-1)\quad \text{ et } \quad
\sum\limits_{k=1}^{n}k^{2},\quad \quad \sum\limits_{k=3}^{n}k(k-1)(k-2).
\end{equation*}%
On consid\`ere dans toute la suite de cette partie un nombre entier $%
n\geqslant 2$ et une urne contenant $n$ jetons num\'erot\'es de $1$ \`a $n$.%
\newline
On extrait de cette urne successivement et sans remise deux jetons et on d\'e%
signe alors par :

\begin{itemize}
\item[$\bullet$] $N_{1}$ la variable al\'eatoire indiquant le num\'ero du premier jeton tir\'e.

\item[$\bullet$] $N_{2}$ la variable al\'eatoire indiquant le num\'ero du second jeton tir\'e.

\item[$\bullet$] $X$ la variable al\'eatoire indiquant le plus petit des num\'eros des deux
jetons tir\'es.

\item[$\bullet$] $Y$ la variable al\'eatoire indiquant le plus grand des num\'eros des deux
jetons tir\'es.
\end{itemize}


\end{enumerate}

\item \textbf{Lois conjointe et marginales de $N_{1}$ et $N_{2}$.}

\begin{enumerate}
\item D\'eterminer les probabilit\'es $P(N_{1}=i)$ pour $1\leqslant i\leqslant n$
et $P(N_{2}=j/N_{1}=i)$ pour $1\leqslant j\leqslant n$.\newline
En d\'eduire $P(N_{2}=j)$ pour $1\leqslant j\leqslant n$, puis comparer les
lois de $N_{1}$ et $N_{2}$.

\item Calculer les esp\'erances $E(N_{1})$ et $E(N_{2})$, les variances $%
V(N_{1})$ et $V(N_{2})$.

\item D\'eterminer les probabilit\'es $P(N_{1}=i\cap N_{2}=j)$ pour $1\leqslant
i\leqslant n$ et $1\leqslant j\leqslant n$ en distinguant les deux cas $i=j$
et $i\neq j$ et déterminer
$ E(N_{1}N_{2})$.
En d\'eduire la covariance et le coefficient de corr\'elation lin\'eaire de $N_{1}$
et $N_{2}$.

\item Exprimer enfin sous forme factoris\'ee la variance $V(N_{1}+N_{2})$.
\end{enumerate}

\item \textbf{Lois conjointe, marginales et conditionnelles de $X$ et $Y$.}

\begin{enumerate}
\item Soit $(i,j)\in[\![1,n]\!]^2$. Déterminer la probabilit\'e $P((X=i)\cap (Y=j))$.  

\item En d\'eduire les probabilit\'es $P(Y=j)$ pour $2\leqslant j\leqslant n$ et 
$P(X=i)$ pour $1\leqslant i\leqslant n-1$.\newline
(On v\'erifiera que les formules donnant $P(Y=j)$ et $P(X=i)$ restent valables
si $j=1$ ou $i=n$).

\item D\'eterminer les probabilit\'es $P_{Y=j}(X=i)$ et $P_{X=i}(Y=j)$ pour $%
1\leqslant i<j\leqslant n$, puis reconna\^\i tre la loi de $X$ conditionn\'ee par $%
Y=j$ et la loi de $Y$ conditionn\'ee par $X=i.$

\item Comparer les lois des variables al\'eatoires $n+1-X$ et $Y$.
En  d\'eduire les
expressions de $E(X)$ en fonction de $E(Y)$ et de $V(X)$ en fonction de $V(Y)
$.
\end{enumerate}

\item \textbf{Esp\'erances et variances de }$X$\textbf{ et }$Y$.

\begin{enumerate}
\item Exprimer les esp\'erances $E(Y)$ et $E(X)$ en fonction de $n$.

\item Exprimer sous forme factoris\'ee $E[(Y(Y-2)]$, puis $E(Y^{2}),V(Y)$ et $%
V(X)$ en fonction de $n$.
\end{enumerate}

\item \textbf{Covariance et coefficient de corr\'elation lin\'eaire de }$X$\textbf{ et }$Y$.

\begin{enumerate}
\item V\'erifier que $X+Y=N_{1}+N_{2}$, puis en d\'eduire sous forme factoris\'ee
la variance de $X+Y$ et la covariance de $X$ et $Y$.

\item En d\'eduire le coefficient de corr\'elation de $X$ et $Y$.\newline
\textit{On remarquera que ce coefficient de corr\'elation lin\'eaire de $X$ et $Y
$ est ind\'ependant de $n$.}
\end{enumerate}


\end{enumerate}




