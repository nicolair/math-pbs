%<dscrpt>Matrices de projecteurs.</dscrpt>
Soit $\mathcal{E}=(e_1,e_2,e_3)$ une base d'un $\R$-espace vectoriel $E$. On définit trois vecteurs $a_1$, $a_2$, $a_3$ de $E$ par :
\begin{displaymath}
\left\lbrace
\begin{aligned}
a_1 &= e_1 + e_2 + e_3 \\
a_2 &= e_1  + e_3 \\
a_3 &= -e_1 + e_2 + 2e_3
 \end{aligned}
\right. 
\end{displaymath}

\begin{enumerate}
\item Montrer que 
\[\mathcal{A}=(a_1,a_2,a_3), \mathcal{A}_1=(e_1,a_2,a_3), \mathcal{A}_2=(a_1,e_2,a_3)\]
sont des bases. Préciser les matrices de passage 
\[P_{\mathcal{A E}}, P_{\mathcal{A}_1\mathcal{E}}, P_{\mathcal{A}_2\mathcal{E}} \]

\item On note $p_1$ le projecteur sur $\Vect (e_2,e_3)$ parallèlement à $\Vect (e_1)$. Calculer :
\[
\underset{\mathcal{E}}{\Mat}\, p_1,\;
\underset{\mathcal{A}}{\Mat}\, p_1, \;
\underset{\mathcal{E A}}{\Mat}\, p_1,\;
\underset{\mathcal{A E}}{\Mat}\, p_1 \] 

\item On note $p_2$ le projecteur sur $\Vect (e_2,e_3)$ parallèlement à $\Vect (a_1)$. Calculer :
\[
\underset{\mathcal{E}}{\Mat}\, p_2,\;
\underset{\mathcal{A}}{\Mat}\, p_2, \;
\underset{\mathcal{E A}}{\Mat}\, p_2, \;
\underset{\mathcal{A E}}{\Mat}\, p_2
\] 


\end{enumerate}