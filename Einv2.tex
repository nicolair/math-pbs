%<dscrpt>Inversion.</dscrpt>
On se place\footnote{d'après Banque PT épreuve B 2009} dans un plan $\mathcal P$ muni d'un produit scalaire noté $(\ /\ )$ et d'un repère orthonormé $(O,\vec{i},\vec{j})$.\\
Soit $\Omega$ un point du plan et $R$ un réel strictement positif.\\
\subsection*{PARTIE I}
\begin{enumerate}
\item \'Etant donné un point $M$ du plan, combien existe-t-il de points $M'$ vérifiant les deux conditions:
\begin{displaymath}
 \left\lbrace 
\begin{aligned}
 &\text{ Les points $M,M'$ et $\Omega$ sont alignés.} \\
 &(\overrightarrow{\Omega M}/\overrightarrow{\Omega M'})=R^2
\end{aligned}
\right. 
\end{displaymath}
\item On considère l'application $\phi$, de $\mathcal P\setminus \{\Omega\}$ dans lui-même, qui, à tout point $M$ distinct de $\Omega$ associe le point $M'$ tel que :
\begin{displaymath}
 \left\lbrace 
\begin{aligned}
 & \text{ Les points $M,M'$ et $\Omega$ sont alignés.} \\
 & (\overrightarrow{\Omega M}/\overrightarrow{\Omega M'})=R^2
\end{aligned}
\right. 
\end{displaymath}
Montrer que l'application $\phi$ est bijective et déterminer sa bijection réciproque.
\item Quelle est l'image par $\phi$ d'un cercle de centre $\Omega$ ?
\item Quelle est l'image par $\phi$ d'une droite passant par $\Omega$ privée de $\Omega$ ?
\item Soit $M$ un point du plan d'affixe $z$. On note $z_\Omega$ l'affixe de $\Omega$ et $z'$ l'affixe de $\phi(M)$.
Exprimer $z'$ en fonction de $z, z_\Omega$ et $R$.
\end{enumerate} 
\subsection*{PARTIE II}
\begin{enumerate}
\item On suppose dans cette question que $\Omega$ est le point d'affixe $1$ et que $R=1$.
\begin{enumerate}
\item Déterminer l'ensemble $A$ des réels $\theta$ pour lesquels le point 
$M(\theta)$ d'affixe $$z=\frac12(1+e^{i\theta})$$ n'est pas le point $\Omega$.
\item Pour $\theta\in A$ déterminer l'image par $\phi$ du point $M(\theta)$.
\item Déterminer l'image par $\phi$ du cercle de diamètre $[O\Omega]$ privé de $\Omega$.
\item Déterminer l'image par $\phi$ de l'axe des ordonnées.
\end{enumerate}
\item Soit $(E)$ l'ellipse d'équation $$3x^2+4y^2=12$$
\begin{enumerate}
\item Déterminer l'excentricité et les foyers de cette ellipse.
\item On note $F$ le foyer d'abscisse positive. Donner une équation polaire de $(E)$ dans le repère $(F,\vec{i},\vec{j})$.
\item On suppose dans cette question que $\Omega$ est le point $F$ et que $R=1$.\\ Déterminer une équation polaire dans le repère $(F,\vec{i},\vec{j})$ de l'image de $(E)$ par $\phi$.
\end{enumerate}
\item On suppose dans cette question que $\Omega$ est le point $O$ et que $R=\sqrt{2}$. Soit $(H)$ l'hyperbole d'équation $xy=1$.\\ Déterminer l'image de $(H)$ par $\phi$ (on pourra commencer par déterminer une équation polaire de $(H)$).
\end{enumerate}
