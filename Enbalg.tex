%<dscrpt>Nombres de Liouville ou algébriques. Transformation de Tschirnaus. Problème de Routh-Hurwitz.</dscrpt>
Soit $\alpha \in \C$. On dit que $\alpha$ est \textit{algébrique} s'il existe un polynôme $P\in \Q[X]$ non nul tel que
$P(\alpha)=0$. Un nombre complexe qui n'est pas algébrique est \textit{transcendant}. Tous les espaces vectoriels de ce sujet
sont des $\Q$-espaces vectoriels.

\subsection*{Partie I. Le théorème de Liouville}

Dans cette partie, $\alpha$ est un nombre algébrique réel irrationnel. On admettra que le développement décimal d'un nombre rationnel est périodique au delà d'un certain rang.\footnote{voir l'élément de cours \href{http:\/back.maquisoc.net\/data\/cours_nicolair/C2142.pdf}{périodicité du développement dans une base}}
\begin{enumerate}
\item Montrer qu'il existe $P \in \Z[X]$ non nul et sans racine rationnelle tel que $P(\alpha)=0$.  On fixe un tel $P$
et on appelle $d$ son degré. On note aussi 
\begin{displaymath}
M_\alpha = \sup_{[\alpha - 1, \alpha +1]} |P'| 
\end{displaymath}

\item Soit $(a_k)_{k\geq 1}$ une suite de chiffres,
\textit{i.e.} $a_k\in\{0,1,\cdots,9\}$.
On suppose que cette suite n'est pas constamment nulle à partir d'un certain rang.
\begin{enumerate}
  \item Soit $m$ et $n$ entiers tels que $m> n$. Montrer que
\begin{displaymath}
\sum_{k=n+1}^{m}10^{-k!}<\frac{10}{9}10^{-(n+1)!} 
\end{displaymath}

  \item Montrer que la suite $(s_n)_{n\geq 1}$ où $\displaystyle s_n = \sum_{k=1}^{n}a_k 10^{-k!}$
  est convergente. On note $s$ sa limite.
  \item Préciser les 720 premières décimales de $s$. Ce nombre est-il rationnel ?
\end{enumerate}
\item \textbf{Théorème de Liouville}. Montrer que pour tout nombre rationnel $r$:
\begin{displaymath}
|r-\alpha|\geq \frac{C_\alpha}{q^d} 
\end{displaymath}
lorsque $C_\alpha=\min (1,\frac{1}{M_\alpha})$ et $r=\frac{p}{q}$ avec $p\in \Z$ et $q\in \N^*$ (minorer $q^dP\left(\frac{p}{q}\right)$).

\item Montrer que le nombre $s$ est transcendant. On pourra raisonner par l'absurde et majorer et
minorer $s_m - s_n$ avec $n$ assez grand bien choisi.
\end{enumerate}

\subsection*{Partie II. Le théorème de l'élément primitif}
\begin{enumerate}
\item Montrer que $\alpha$ est algébrique si et seulement si il existe un entier $p$ tel que la famille $(1,\alpha, \cdots ,\alpha^p)$ est liée.
\item Soit $\alpha$ algébrique irrationnel et $d$ le plus petit des entiers $p$ tels que la famille $(1,\alpha, \cdots ,\alpha^p)$ soit liée. On dira que $d$ est le \emph{degré} de $\alpha$. On note
\begin{displaymath}
\Q [\alpha]=\Vect(1,\alpha, \cdots \alpha^{d-1}) 
\end{displaymath}
  \begin{enumerate}
  \item Montrer que $(1,\alpha, \cdots ,\alpha^{d-1})$ est une base de $\Q[\alpha]$.
  \item Montrer que $\alpha^n\in \Q[\alpha]$ pour tout entier naturel $n$.
  \item Montrer que $\alpha^{-1}\in \Q[\alpha]$.
  \item Montrer qu'il existe un unique $P\in \Q[X]$ unitaire et irréductible tel que $P(\alpha)=0$. Quel est son degré?
  \end{enumerate}
  \item Soit $\alpha$ algébrique de degré $p$ et $\beta$ algébrique de degré $q$.\newline
On note $\Q[\alpha,\beta]$ le sous-espace vectoriel engendré par les $\alpha^m \beta^n$ avec $m$ entier entre 0 et $p-1$ et $n$ entier entre 0 et $q-1$. Montrer que 
\begin{align*}
 \forall l \in \N :& &  (\alpha+\beta)^l \in \Q[\alpha,\beta] & & , & & (\alpha\beta)^l \in \Q[\alpha,\beta]
\end{align*}
En déduire que $\alpha + \beta$ et $\alpha \beta$ sont algébriques.
\end{enumerate}

\subsection*{Partie III. Transformation de Tchirnhaus}
On rappelle que deux polynômes ont une racine commune (dans $\C$) si et seulement si ils ne sont pas premiers entre eux.\newline
Soit $P$ et $Q$ deux polynômes à coefficients rationnels. Pour tout nombre complexe $s$, on désigne par $Q_s$ le polynôme obtenu à partir de $Q$ en substituant $s-X$ à $X$ dans $Q$. Soit 
\begin{displaymath}
Q_s=\widehat{Q}(s-X)=Q\circ (s-X) 
\end{displaymath}
\begin{enumerate}
\item  Exemple.\newline
Soient $P=X^2+X+1$ et $Q=X^2-2$. Donner une condition nécessaire et suffisante portant sur $s$ pour
que $P$ et $Q_s$ aient une racine en commun.
\item Cas général.\newline
On admet que la condition assurant que $P$ et $Q_s$ ont une racine en commun s'écrit $C(s)=0$ où $C$ est un polynôme. Quelles sont les racines de $C$?
\item Application.\newline
Former un polynôme irréductible dans $\Q[X]$ à coefficients entiers dont $j+\sqrt{2}$ est racine. Former un polynôme irréductible dans $\Q[X]$ à coefficients entiers dont $\sqrt{3}+\sqrt{2}$ est racine.
\end{enumerate}

\subsection*{Partie IV. Problème de Routh-Hurwitz}
On désigne par $\mathcal{H}$ l'ensemble des nombres complexes dont la partie réelle est négative ou nulle. On dira qu'un polynôme non nul et à coefficients réels est \emph{positif} lorsque tous ses coefficients sont positifs ou nuls. On dira qu'un polynôme à coefficients réels ou complexe est \emph{stable} lorsque toutes ses racines sont dans $\mathcal{H}$. \par On considère une famille
de $n$ nombres complexes $(z_1,z_2,\cdots,z_n)$. Elle définit deux polynômes
\begin{displaymath}
A=\prod_{i=1}^{n}(X-z_i) \quad S_A=\prod_{(i,j) \in \{1,\cdots,n\}^2} (X-z_i-z_j) 
\end{displaymath}
\begin{enumerate}
\item Quel est le degré de $S_A$? Si $A$ est à coefficients réels, en est-il de même pour $S_A$?
\item Montrer que $\mathcal{H}$ est stable par addition.
\item Montrer que l'ensemble des polynômes positifs est stable par multiplication.
\item Montrer qu'une éventuelle racine réelle d'un polynôme positif est forcément négative ou nulle.
\item (Cours) Quels sont les polynômes irréductibles de $\R[X]$ ?
Montrer qu'un polynôme de $\R[X]$ unitaire, irréductible et stable est positif.
\item Montrer que tout diviseur d'un polynôme stable est stable.
\item Montrer que si $A$ et $S_A$ sont positifs et à coefficients réels alors $A$ est stable.
\item Montrer que si $A$ est stable et à coefficients réels alors $A$ et $S_A$ sont positifs.
\item Expliquez la suite des calculs à effectuer pour déterminer pratiquement si un polynôme à coefficients complexes est stable.
\end{enumerate}
