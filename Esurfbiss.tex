%<dscrpt>Surface bissectrice de deux droites dans l'espace.</dscrpt>
On considère l'espace euclidien orienté $E$ muni d'un repère orthonormal \repere.\newline 
Soit $D$ et $D'$ deux droites non coplanaires; on note $\overrightarrow{u}$ (respectivement $\overrightarrow{u'}$)
un vecteur directeur unitaire de $D$ (respectivement $D'$) et $A$ (respectivement $A'$) un point de $D$ (respectivement $D'$).\newline
Soit $\Delta$ la perpendiculaire commune à $D$ et $D'$. On note $H$ (respectivement $H'$) le point d'intersection de $\Delta$ avec $D$ (respectivement $D'$).\newline
Dans tout le problème $X$ désigne l'ensemble des points $M$ équidistants des droites $D$ et $D'$.\newline
Le produit mixte des vecteurs $\overrightarrow{a}$, $\overrightarrow{b}$, $\overrightarrow{c}$ est noté $\det(\overrightarrow{a},\overrightarrow{b},\overrightarrow{c})$ ou $vol(\overrightarrow{a},\overrightarrow{b},\overrightarrow{c})$.
\begin{figure}
	\centering
	\input{Esurfbiss_1.pdf_t}
\end{figure}

\begin{enumerate}
\item \begin{enumerate}
\item Citer et démontrer une formule donnant la distance notée $d(M,D)$ d'un point $M$ à la droite $D$.
\item On note $d=HH'$ la distance entre $D$ et $D'$. Montrer que 
\[ d=\frac{|\det(\overrightarrow{AA'},\overrightarrow{u},\overrightarrow{u'})|}{\|\overrightarrow{u}\wedge \overrightarrow{u'}\|}\]
\end{enumerate}
\item Dans cette question, on suppose que l'on a $\overrightarrow{u}=\overrightarrow{j}$, $\overrightarrow{u'}=\overrightarrow{i}$, $A=O$ et que $A'$
est le point de coordonnées $(0,0,1)$.
	\begin{enumerate}
	\item Déterminer une équation de l'ensemble $X$.
	\item Préciser une équation de l'intersection de $X$ avec chacun des plans de coordonnées. Construire les courbes associées après avoir déterminé leur nature.
	\end{enumerate}
\item On revient maintenant au cas général.
	\begin{enumerate}
	\item Soit $O'$ soit le milieu de $[HH']$ et $\overrightarrow{I}$, $\overrightarrow{J}$ définis par 
\[\overrightarrow{I}=\frac{1}{\|\overrightarrow{u}+\overrightarrow{u'}\|}(\overrightarrow{u}+\overrightarrow{u'}) , \; 
\overrightarrow{J}=\frac{1}{\|\overrightarrow{u}-\overrightarrow{u'}\|}(\overrightarrow{u}-\overrightarrow{u'})\]
Comment doit-on choisir $\overrightarrow{K}$ pour que $(O',\overrightarrow{I},\overrightarrow{J},\overrightarrow{K})$ soit un repère orthonormal direct ? 
\item Démontrer que dans ce nouveau repère, les droites $D$ et $D'$ admettent un système d'équations de la forme
\begin{displaymath}
 \left \{
	\begin{aligned}
	y=\varepsilon mx\\
	z=\varepsilon \frac{d}{2}
	\end{aligned}
	\right.
\end{displaymath}
où $\varepsilon \in \{-1,1\}$ et $m \in \R$.
\item Déterminer une équation de l'ensemble $X$ des points équidistants de $D$ et de $D'$.
\end{enumerate}
\item Dans cette question on utilise le repère de la question précédente pour étudier les droites contenues dans $X$.\newline
Montrer que par tout point de $X$ passent deux droites incluses dans $X$. Préciser les coordonnées des vecteurs directeurs.

\item Dans cette question on étudie encore les droites contenues dans $X$ mais sans utiliser de repère, tous les calculs sont vectoriels.\newline
Soit $M\in X$, $\overrightarrow{w}$ un vecteur non nul et $\lambda$ un nombre réel. On définit le point $P_\lambda$ par :
\[\overrightarrow{MP_\lambda}=\lambda \overrightarrow{w}\]
\begin{enumerate}
\item Exprimer $d(P_\lambda,D)^2$ comme une fonction du second degré en $\lambda$.
\item Montrer que les deux relations suivantes :
\begin{displaymath}
\left\lbrace
  \begin{aligned}
\Vert \overrightarrow{w}\wedge \overrightarrow{u}\Vert^2 =& 
   \Vert\overrightarrow{w}\wedge \overrightarrow u'\Vert^2 & & (1)\\ 
\det(\overrightarrow{AM},\overrightarrow{u},\overrightarrow{w}\wedge\overrightarrow{u})  =&  \det(\overrightarrow{A'M},\overrightarrow{u'},\overrightarrow{w}\wedge\overrightarrow{u'}) & & (2) 
\end{aligned}
\right. 
\end{displaymath}
entraînent que $P_\lambda \in X$ pour tout $\lambda$ réel.
\item On définit deux plans $\mathcal P^+$ et $\mathcal P^-$ 
\begin{align*}
 P^+ =\left( \Vect(\overrightarrow u + \overrightarrow u')\right) ^\perp 
& & 
 P^- =\left( \Vect(\overrightarrow u - \overrightarrow u')\right) ^\perp
\end{align*}
Montrer que l'ensemble des $\overrightarrow{w}$ vérifiant $(1)$ est $\mathcal P^+ \cup \mathcal P^-$.
\item Montrer que l'ensemble des $\overrightarrow{w}$ vérifiant $(2)$ est un plan vectoriel à préciser.
\item Montrer que par tout point $M$ de $X$ passent deux droites incluses dans $X$. Préciser les vecteurs directeurs.
\end{enumerate}

\end{enumerate}