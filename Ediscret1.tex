%<dscrpt>Formule du crible de Poincaré.</dscrpt>
Soit $E$ un ensemble fini. Si $B$ est une partie de $E$, on note $f_B$ la fonction caractéristique de la partie $B$. C'est la fonction de $E$ dans $\{0,1\}$ définie par :
\[x \mapsto f_B(x)=\left\lbrace \begin{array}{ccc}
0 & \mathrm{si} & x \not \in B\\
1 & \mathrm{si} & x\in B
\end{array} \right. \]

Soit $n \geq 2$ un entier fixé et $B_1,B_2,\cdots , B_n$ des parties de $E$. Pour toute partie $I$ de $\{1,\cdots, n\}$, on note
\[B_I=\bigcap_{i\in I}B_1\]

\begin{enumerate}
\item \begin{enumerate}
\item Soit $B$ une partie de $E$, préciser la somme
\[\sum_{x \in E} f_B(x)\]
\item Préciser la fonction définie dans $E$ par :
\[x \mapsto 1-f_B(x)\]
\item Soit $I$ une partie de $\{1,\cdots,n\}$ et $B_1,\cdots,B_n$ des parties de $E$, préciser la fonction  dans $E$ par :
\[x\mapsto \prod _{i\in I}f_{B_i}(x)\]
\end{enumerate}
\item On considère une famille $A_1,A_2, \cdots , A_n$ de parties de $E$.\begin{enumerate}
\item Exprimer $A_1 \cup \cdots \cup A_n$ comme le complémentaire d'une intersection.
\item En déduire directement (sans récurrence) la formule du crible de Poincaré.
\[\sharp(A_1 \cup \cdots \cup A_n)=\sum_{p=1}^n(-1)^{p-1}\sum _{I\in \mathcal{P}_p}\sharp \bigcap_{i\in I} A_i\]
où $\sharp B$ désigne le nombre d'éléments de $B$ et $\mathcal{P}_p$ l'ensemble des parties à $p$ éléments de $\{1,\cdots ,n\}$(on pourra développer un produit).
\end{enumerate}
\item Applications.
\begin{enumerate}
 \item Déterminer le nombre d'applications non surjectives d'un ensemble à $p$ éléments dans un ensemble à $n$ éléments.
\item Déterminer le nombre de permutations (bijection d'un ensemble fini dans lui même) d'un ensemble à $n$ éléments ayant au moins un point fixe. 
\end{enumerate}

\end{enumerate}
