%<dscrpt>Formule de Plouffe</dscrpt>
Ce texte porte sur une série convergente pour laquelle on dispose à la fois d'une expression exacte simple de la somme et d'un algorithme efficace permettant de calculer le développement hexadécimal (en base 16) de cette somme (formule de S. Plouffe 1997).\newline
On considère la série $\left( \sum x_n \, 16^{-n}\right)_{n \in \N}$ avec :
\begin{displaymath}
\forall n\in \N, \; 
x_n = \frac{4}{8n+1} - \frac{2}{8n+4} - \frac{1}{8n+5} - \frac{1}{8n+6}.
\end{displaymath}
Les premières valeurs de $x_n$ sont
\begin{displaymath}
 x_0 = \frac{47}{15}, \hspace{0.5cm} x_1 = \frac{106}{819}, \hspace{0.5cm} x_2 = \frac{829}{19635}
\end{displaymath}


\subsection*{Partie I. Convergences.}
\begin{enumerate}
 
 \item Montrer que $\left( \sum x_n \, 16^{-n}\right)_{n \in \N}$ est une série à termes positifs convergente.\newline  On note $s$ sa somme, $s_n$ la somme partielle et $\varepsilon_n$ le reste à l'ordre $n$:
\begin{displaymath}
\forall n \in \N, \; s = s_n + \varepsilon_n, \hspace{0.5cm} s_n = \sum_{k=0}^{n}x_k\,16^{-k} .
\end{displaymath}

 \item Montrer que 
\begin{displaymath}
\forall n \in \N, \;  0 < \varepsilon_n < 16^{-(n + 1)}.
\end{displaymath}
\end{enumerate}


\subsection*{Partie II. Développements hexadécimaux.}
Soit $x\in \R$ et $n\in \N$. On note $a_n(x)$ l'approximation hexadécimale par défaut de $x$ à l'ordre $n$ :
\begin{displaymath}
 x = a_n(x) + b_n(x) \text{ avec } a_n \in \Z\times 16^{-n} \text{ et } 0\leq b_n(x) < 16^{-n}
\end{displaymath}
\begin{enumerate}
 \item 
On suppose qu'il existe $u$ et $v$ dans $\N^*$ tels que $x=\frac{u}{v}$.
\begin{enumerate}
 \item Exprimer $a_n(x)$ et $b_n(x)$ en fonction du quotient $q_n$ et du reste $r_n$ de la division de $16^n\,u$ par $v$.
 \item Exprimer, en fonction de $q_n(x)$, le coefficient de $16^{-n}$ dans le développement de $x$ en base 16.
 \item Exprimer, en fonction de $r_n(x)$, le coefficient de $16^{-(n+1)}$ dans le développement de $x$ en base 16.
\end{enumerate}

 \item Former la suite des restes des divisions de $16^n\times 47$ par $15$. En déduire le développement hexadécimal de $x_0$.
 
 \item Les divisions suivantes sont données:
\begin{displaymath}
 16 \times 106 = 2\times 819 + 58, \hspace{1cm} 16 \times 58 = 1\times 819 + 109, \hspace{0.5cm} 16\times 829 = 13264
\end{displaymath}
\begin{enumerate}
 \item Former un tableau donnant les coefficients de $16^{-n}$ (avec $n\in \llbracket 0, 4\rrbracket$) des développements hexadécimaux de $x_0 16^{-0}$, $x_116^{-1}$, $x_216^{-2}$, $\varepsilon_2$ en plaçant un \og ?\fg~ pour les coefficients que vous ne pouvez pas calculer facilement.
 \item Quel développement hexadécimal de $s$ peut-on en déduire ?
\end{enumerate}
 \item \'Ecrire à l'aide de puissances de $16$ l'encadrement de $s$ correspondant à la question précédente. En déduire une approximation décimale de $s$.
\end{enumerate}


\subsection*{Partie III. Calculs formels de sommes.}
Dans cette partie, $c$ représente un réel de $]0,1[$ et $l$ un entier naturel.
\begin{enumerate}
 \item
\begin{enumerate}

 \item Calculer $\int_{0}^{c} t^{8n + l}\, dt$. Comment doit-on choisir $c$ pour faire apparaitre les suites figurant dans $16^{-n}x_n$ ?
 
 \item Montrer la convergence de la série 
\begin{displaymath}
 \left( \sum \frac{c^{8n+l+1}}{8n+l+1}\right)_{n\in \N}.
\end{displaymath}

\end{enumerate}
 
 \item Expression des sommes élémentaires.
\begin{enumerate}
 \item Montrer que 
\begin{displaymath}
 \forall c \in \,]0,1[, \;\forall t \in [0,c],\hspace{0.5cm} \frac{t^8}{1 - t^8} \leq \frac{c^8}{1 - c^8}
\end{displaymath}
 \item Montrer que 
\begin{displaymath}
\forall t \in [0,c], \forall n \in \N,\hspace{0.5cm} \left| \frac{t^l}{1 - t^8} - \sum_{i=0}^n t^{8i+l}\right| \leq \frac{c^{8n+8+l}}{1-c^8} 
\end{displaymath}
  \item En déduire 
\begin{displaymath}
 \sum_{i=0}^{+\infty} \frac{c^{8i+l+1}}{8i + l + 1} = \int_{0}^{c}\frac{t^l}{1-t^8}\,dt
\end{displaymath}
\end{enumerate}

 \item Montrer que 
\begin{displaymath}
 s = 
\int_0^{\frac{1}{\sqrt{2}}}\frac{8t^5 + 4 \sqrt{2}\,t^4 + 8t^3 - 4\sqrt{2}}{t^8 -1}\, dt
\end{displaymath}

 \item Quelles sont les racines complexes du polynôme $X^8 - 1$?\newline
 En déduire sa factorisation dans $\R[X]$. On admet
\begin{displaymath}
8X^5 + 4 \sqrt{2}\,X^4 + 8X^3 - 4\sqrt{2} = 8(X^2+1)(X^2 + \sqrt{2}\,X +1)(X - \frac{1}{\sqrt{2}}) 
\end{displaymath}
En déduire une nouvelle expression de $s$ comme une intégrale simplifiée entre $0$ et $1$.

 \item Déterminer les réels $\alpha$, $\beta$, $\gamma$, $\delta$ tels que 
\begin{displaymath}
 16\frac{X - 1}{(X^2 - 2X + 2)(X^2 - 2)} = 
 \frac{\alpha X + \beta}{X^2 - 2X + 2} 
 + \frac{\gamma X +\delta }{2-X^2} 
\end{displaymath}


\item En déduire $s$.
\end{enumerate}

