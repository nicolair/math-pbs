\begin{enumerate}
 \item Comme $\Im f =\{f(x) , x \in E\}$, d'après la propriété de cours relative à la projection orthogonale sur le sous-espace vectoriel $\Im f$ :
\begin{displaymath}
 \min \left\lbrace \Vert f(x)-v \Vert , x\in E \right\rbrace = \Vert p_{\Im f}(v)-v \Vert 
\end{displaymath}
Comme $p_{\Im f}(v)\in \Im f$, il existe un $x_0\in E$ tel que $f(x_0)=p_{\Im f}(v)$.

\item Une psudo solution est un antécédent de la projection orthogonale du second membre sur l'image de l'application. L'équation admet donc \emph{toujours} des pseudo-solutions. Lorsque $f$ est injective, chaque élément de l'image admet un \emph{unique} antécédent. Dans ce cas il existe donc une unique pseudo solution.

\item $f(x_0)=p_{\Im f}(v)$ si et seulement si $v-f(x_0)$ est orthogonal à $\Im f$ c'est à dire si et seulment si :
\begin{displaymath}
 \forall x\in E : (f(x)/f(x_0)-v)=0
\end{displaymath}


\item La traduction matricielle de $(f(x)/f(x_0)-v)=0$ est
\begin{displaymath}
 ^tX \left( ^tA A X_0 - V \right) = O_{\mathcal M_{n,1}(\R)} 
\end{displaymath}
Alors $x_0$ est pseudo solution de (1) si et seulement si la relation précédente est valable pour toutes les matrices colonnes $X$. En choisissant successivement les colonnes $X$ formées avec un seul 1 et que des 0 on en déduit que cela entraine 
\begin{displaymath}
  ^tA A X_0 - V  = O_{\mathcal M_{n,1}(\R)} 
\end{displaymath}
la réciproque est évidente.

\item Calcul du rang de $f$ pour l'exemple.
\begin{displaymath}
 \rg f = \rg A = \rg 
\begin{pmatrix}
 1 & 1 &-1 \\ 0 & 0 & 0 \\ 0 & 3 & 0 
\end{pmatrix}
= 2 
\end{displaymath}
\'Equation de l'image.\newline
$(a,b,c)\in \Im f$ si et seulement si il existe des réels $(x,y,z)$ tels que
\begin{displaymath}
 \left\lbrace 
\begin{aligned}
 x+y-z &= a \\
x+y-z &= b \\
-x+2y +z &= c
\end{aligned}
\right. 
\Leftrightarrow
 \left\lbrace 
\begin{aligned}
 x+y-z &= a \\
0 &= b-a \\
3y &= c +a
\end{aligned}
\right. 
\end{displaymath}
On en déduit que $(a,b,c)\in \Im f$ si et seulement si $a-b=0$.\newline
Calcul des pseudo solutions.\newline
Ici $v=(1,0,0) \not \in \Im f$. On forme $^tAAX = ^tAV$ :
\begin{align*}
 \begin{pmatrix}
 1 &  1 &-1  \\
 1 & 1  & 2 \\
-1 & -1 & 1
 \end{pmatrix}
 \begin{pmatrix}
 1 & 1 &-1 \\
 1 & 1 & -1 \\
-1 & 2 & 1
 \end{pmatrix}
 \begin{pmatrix}
 x \\
 y \\
 z
 \end{pmatrix}
&=
 \begin{pmatrix}
 1 &  1 &-1 \\
 1 & 1  & 2 \\
-1 & -1 & 1
 \end{pmatrix}
 \begin{pmatrix}
 1  \\ 0 \\ 1
 \end{pmatrix}  \\
 \begin{pmatrix}
  3 & 0  & -3 \\
  0 & 6 & 0 \\
-3 & 0 & 3
 \end{pmatrix}
\begin{pmatrix}
 x \\ y \\ z
 \end{pmatrix}
&=\begin{pmatrix}
 0 \\  3 \\ 0
 \end{pmatrix}
\end{align*}
Cela conduit au système
\begin{displaymath}
 \left\lbrace
\begin{aligned}
3x & &-3z &= 0 \\
 & 6y &  &= 3 \\
-3x & &+ 3z &= 0
\end{aligned}
\right. 
\end{displaymath}
L'ensemble des solutions est donc 
\begin{displaymath}
 \left\lbrace (x,\frac{1}{2},x) , x\in \R \right\rbrace 
\end{displaymath}

\item \begin{enumerate}
 \item Considérons une fonction $f$ définie dans $\R^2$ et à valeurs dans $\R^n$ :
\begin{displaymath}
 f(\lambda,\mu) = \lambda a + \mu b
\end{displaymath}
et l'équation $(2)$ : $f(x)=c$. Une pseudo solution de cette équation est un couple $(\lambda_0,\mu_0)$ tel que
\begin{displaymath}
 \Vert \lambda_0 a + \mu b -c \Vert = \min 
\left\lbrace
\Vert \lambda a + \mu b -c \Vert , (\lambda, \mu) \in \R^2
 \right\rbrace 
\end{displaymath}
Comme l'application $t \rightarrow t^2$ est strictement croissante dans $\R^2$, cela revient à minimiser le carré de la norme. La matrice de $f$ dans la base canonique de $\R^n$ est :
\begin{displaymath}
 \begin{pmatrix}
  a_1 & b_1 \\
 \vdots & \vdots \\
 a_n & b_n
 \end{pmatrix}
\end{displaymath}
\item La fonction est injective si et seulement si $(a,b)$ est libre.
\item On suppose $(a,b)$ libre. On trouve $\lambda$ et $\mu$ en exprimant que $\lambda a +\mu b -c$ est orthogonal à $a$ et $b$.
\begin{displaymath}
 \left\lbrace 
\begin{aligned}
 \Vert a \Vert^2 \lambda - (a/c) \mu &= (a/c) \\
 (a/b)\lambda + \Vert b \Vert ^2 \mu &= (b/c)
\end{aligned}
\right. 
\end{displaymath}
On résoud par les formules de Cramer :
\begin{align*}
 \lambda = \dfrac{(b/c)\Vert b \Vert^2-(a/c)(a/b)}{\Vert a \Vert^2 \Vert b \Vert^2 -(a/b)^2}
&,&
 \mu = \dfrac{(b/c)\Vert a \Vert^2-(a/c)(a/b)}{\Vert a \Vert^2 \Vert b \Vert^2 -(a/b)^2}
\end{align*}
On remarque que le dénominateur des formules précédentes est strictement négatif d'après l'inégalité de Cauchy-Schwarz (l'égalité ne se produisant que pour un couple lié de vecteurs).
\end{enumerate}

\end{enumerate}
