%<dscrpt>Géométrie plane : construction d'un polygone dont les milieux forment un polygone donné.</dscrpt>
\subsection*{Etude d'un problème de géométrie du plan \footnote{Université de Provence, licence de Maths, atelier de pratique mathématique, mars 1997}}

Problème
\begin{quote}
Soient $n$ points $a_{1}, a_{2}, \cdots a_{n}$ du plan complexe.\newline
On veut étudier (existence, unicité, construction) les polygones (non nécessairement convexes) $m_{1}, m_{2}, \cdots m_{n}$ tels que $a_{1}$ soit milieu de $m_{1}m_{2}$,$\cdots$, $a_{i}$ milieu de $m_{i}m_{i+1}$,$\cdots$, $a_{n}$ milieu de $m_{n}m_{1}$.
\end{quote} 

\begin{enumerate}
\item \begin{enumerate}
\item Faire une étude géométrique du cas $n=3$ discuter de l'existence, de l'unicité et de la construction des polygones (triangles) solutions. On supposera les $a_{i}$ non alignés.
\item Identifier la figure formée par les milieux des côtés d'un quadrilatère. En déduire une condition nécessaire et suffisante pour qu'il existe des solutions an problème dans le cas $n=4$. Quelles sont alors ces solutions ?
\end{enumerate}
\item Dans le cas général, le problème peut se ramener à la résolution d'un système de $n$ équations linéaires à $n$ inconnues. Ecrire un tel système. Le résoudre et discuter. En déduire une reponse au problème.

\item\begin{enumerate}
\item Si $a$ et $b$ sont deux points du plan, identifier la transformation géométrique composée de la symétrie par rapport au point $a$ et de la symétrie par rapport au point $b$.
\item Résoudre le problème en étudiant des produits de symétries.
\end{enumerate}
\item À partir de l'étude menée au 1. et d'un raisonnement par récurrence ramenant le problème de $n$ points à $n-2$ points, retrouver les résultats précédents. et proposer une construction géométrique des solutions lorsqu'elles existent.
\end{enumerate}
NB. Il serait apprécié que les candidats résument dans une conclusion ce qu'ils sont parvenus à démontrer.
