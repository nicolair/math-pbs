%<dscrpt>Calcul matriciel.</dscrpt>
On d{\'e}finit des ensembles $\mathcal{T}$, $\mathcal{T^+}$, $\mathcal{D}$ de matrices carr{\'e}es dans $\mathcal M_3(\R)$.\newline
L'ensemble $\mathcal{T}$ est form{\'e} par les matrices \emph{triangulaires
sup{\'e}rieures}
\[\begin{pmatrix}
  \alpha & a & b \\
  0 & \beta & c \\
  0 & 0 & \gamma
\end{pmatrix} \quad a , b , c  , \alpha , \beta , \gamma \in \R
\]
L'ensemble $\mathcal{T^+}$ est form{\'e} par les matrices \emph{triangulaires
sup{\'e}rieures strictes}
\[\begin{pmatrix}
  0 & a & b \\
  0 & 0 & c \\
  0 & 0 & 0
\end{pmatrix} \quad a , b , c  \in \R
\]
L'ensemble $\mathcal{D}$ est form{\'e} par les matrices \emph{diagonales}
\[\begin{pmatrix}
  \alpha & 0 & 0 \\
  0 & \beta & 0 \\
  0 & 0 & \gamma
\end{pmatrix} \quad \alpha , \beta , \gamma \in \R\]

\subsubsection*{Partie I}
\begin{enumerate}
  \item Montrer l'existence d'entiers $n$ tels que :
  \[\forall A \in \mathcal{T^+} : A^n=
  0_{\mathcal{M}_3(\R)}\]
  Pr{\'e}ciser le plus petit de ces entiers.
  \item
\begin{enumerate}
  \item D{\'e}terminer les matrices diagonales $D$ qui commutent avec
  toutes les matrices triangulaires sup{\'e}rieures strictes.
  \item Pour une telle matrice $D$, calculer pour $n$ entier et
  $A\in\mathcal{T}^+$
  \[(D+A)^n\]

\end{enumerate}

\end{enumerate}

\subsubsection*{Partie II}
Soit $\mathcal{E}$ l'ensemble des matrices de la forme
\[M(a,b)=\begin{pmatrix}
  1 & a & b \\
  0 & 1 & a \\
  0 & 0 & 1
\end{pmatrix} \quad a,b \in \R\]

Pour toute application $f$ de $\R$ dans $\R$, on note
\[\widehat{M}(x)=M(x,f(x))\]
\begin{enumerate}
  \item Montrer que $\mathcal{E}$ est un sous-groupe de
  $\mathrm{GL}_3(\R)$.
  \item On cherche les fonctions $f$ telles que

\begin{equation}
  \forall (x,y)\in \R^2 : \widehat{M}(x)\widehat{M}(y)=\widehat{M}(x+y)
\end{equation}

\begin{enumerate}
  \item Montrer que si $f$ est une telle fonction : $f(0)=0$,
  $\widehat{M}(0)=I_3$ et pour tout r{\'e}el $x$
  \[\widehat{M}^{-1}(x)=\widehat{M}(-x)\]
  \item Caract{\'e}riser les fonctions v{\'e}rifiant (1) par une relation
  fonctionnelle.
  \item V{\'e}rifier que, pour tout r{\'e}el $m$,
  \[x\rightarrow\frac{1}{2}x^2+mx\]
  v{\'e}rifie la condition (1)
  \item Montrer que toute fonction d{\'e}rivable v{\'e}rifiant (1) est de
  cette forme.
\end{enumerate}

\end{enumerate}
