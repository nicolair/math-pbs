%<dscrpt>Exercices de géométrie dans l'espace.</dscrpt>
Pour tous les exercices, on se place dans un espace $\mathcal{E}$ muni d'un repère orthonormé direct $\mathcal R = (O,(\overrightarrow i , \overrightarrow j , \overrightarrow k))$ dont les fonctions coordonnées sont notées $x$, $y$, $z$.
\subsection*{Exercice 1.}
On se donne un plan $\mathcal{P}$ de manière paramétrique. Pour tout point $M$ de l'espace,
\begin{displaymath}
 M\in \mathcal P \Leftrightarrow \exists(\lambda,\mu)\in \R^2 \text{ tq }
\left\lbrace 
\begin{aligned}
 x(M) &=& 2+\lambda-\mu\\
 y(M) &=& 3-\lambda+2\mu \\
 z(M) &=& 1 + 2\lambda +\mu
\end{aligned}
\right. 
\end{displaymath}
Calculer la distance d'un point $M$ au plan $\mathcal{P}$.

\subsection*{Exercice 2.}
On se donne une droite $\mathcal{D}$ et un plan $\mathcal{P}$ par des équations. Pour tout point $M$ de l'espace,
\begin{align*}
 M\in \mathcal{P}&\Leftrightarrow x(M) + 2y(M) + 3z(M)+6 = 0
\\
 M\in \mathcal{D}&\Leftrightarrow
\left\lbrace 
\begin{aligned}
 x(M) + y(M) + z(M) -1 = 0 \\
 x(M) - y(M)-2z(M) = 0
\end{aligned}
\right. 
\end{align*}
Former un système d'équations cartésiennes pour la droite $\mathcal{D}'$ projection orthogonale de $\mathcal D$ sur $\mathcal P$.

\subsection*{Exercice 3.}
On se donne deux vecteurs $\overrightarrow J$ et $\overrightarrow K$ :
\begin{align*}
 \overrightarrow J = \frac{1}{\sqrt{2}}\left(\overrightarrow i - \overrightarrow j \right)
& &
 \overrightarrow K = \frac{1}{\sqrt{3}}\left(\overrightarrow i + \overrightarrow j + \overrightarrow k \right) 
\end{align*}
et un cercle $\mathcal C$ dans l'espace par des équations cartésiennes. Pour tout point $M$ de l'espace,
\begin{displaymath}
 M\in \mathcal{C} \Leftrightarrow
\left\lbrace 
\begin{aligned}
 x(M) + y(M) + z(M)  = 3 \\
 x(M)^2 + y(M)^2 + z(M)^2 = 5
\end{aligned}
\right.  
\end{displaymath}
\begin{enumerate}
 \item Déterminer l'unique vecteur $\overrightarrow I$ tel que $(\overrightarrow I, \overrightarrow J, \overrightarrow K)$ soit une base orthonormée directe. On note $X$, $Y$, $Z$ les fonctions coordonnées dans le repère $\mathcal R' = (O,(\overrightarrow I , \overrightarrow J , \overrightarrow K))$.
 \item Exprimer $X$, $Y$, $Z$ en fonction de $x$, $y$, $z$.
 \item Exprimer les équations de $\mathcal C$ avec $X$, $Y$, $Z$. En déduire le rayon et les coordonnées du centre d'abord dans $\mathcal R'$ puis dans $\mathcal R$.
\end{enumerate}

\subsection*{Exercice 4.}
On se donne une droite $\mathcal{D}$ et un plan $\mathcal{P}$ par des équations. Pour tout point $M$ de l'espace,
\begin{align*}
 M\in \mathcal{P}&\Leftrightarrow x(M) - y(M) - 3z(M) = 1
\\
 M\in \mathcal{D}&\Leftrightarrow
\left\lbrace 
\begin{aligned}
 x(M) - z(M) -2 = 0 \\
  y(M) + z(M) -1 = 0
\end{aligned}
\right. 
\end{align*}
\begin{enumerate}
 \item Pour tout point $M$ de l'espace, calculer les coordonnées du symétrique $s(M)$ de $M$ par rapport à la droite $\mathcal{D}$.
 \item Former l'équation du plan $\mathcal{P}'$ symétrique de $\mathcal{P}$ par rapport à $\mathcal{D}$.
\end{enumerate}
