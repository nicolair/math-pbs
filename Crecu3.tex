\begin{enumerate}
 \item \begin{enumerate}
 \item Il s'agit d'une simple vérification. On développe et ordonne d'abord le crochet de droite, on obtient :
\begin{displaymath}
 2(a^2+b^2+c^2) -2(ab+ac+bc)
\end{displaymath}
Quand on multiplie par $a+b+c$, on obtient :
\begin{multline*}
 2(a^3+b^3+c^3) + 2(ab^2+ac^2+ba^2+bc^2+ca^2+cb^2)\\
-2(a^2b+abc+ca^2+ab^2+b^2c+abc+abc+bc^2+c^2a)\\
= 2(a^3+b^3+c^3) -6abc
\end{multline*}
\item Dans la relation précédente, en remplaçant $a$ par $a^{\frac{1}{3}}$, $b$ par $b^{\frac{1}{3}}$, $c$ par $c^{\frac{1}{3}}$, on obtient (tout est $>0$)
\begin{displaymath}
 a+b+c = 3(abc)^{\frac{1}{3}} + \text{terme positif avec des puissances }\frac{1}{3}
\end{displaymath}
De même, en remplaçant $a$ par $a^{-\frac{1}{3}}$, $b$ par $b^{-\frac{1}{3}}$, $c$ par $c^{-\frac{1}{3}}$, on obtient
\begin{displaymath}
 \frac{1}{a} + \frac{1}{b} + \frac{1}{c} = 3(abc)^{-\frac{1}{3}} + \text{terme positif avec des puissances }-\frac{1}{3}
\end{displaymath}
Cela prouve les inégalités demandées.
\end{enumerate}
\item Les deux inégalités de la question précédentes se reformulent en :
\begin{displaymath}
 \frac{3}{\frac{1}{a}+\frac{1}{b}+\frac{1}{c}} \leq (abc)^{\frac{1}{3}} \leq \frac{a+b+c}{3}
\end{displaymath}
Il s'agit de la comparaison classique entre moyennes \emph{harmonique}, \emph{géométrique} et \emph{arithmétique}.\newline
Les suites sont bien définies car chaque nouveau terme est strictement positif ce qui permet la poursuite du processus. La comparaison des moyennes montre par récurrence que 
\begin{displaymath}
 \forall n \geq 1 : c_n \leq b_n \leq a_n
\end{displaymath}
\item Montrons que $(c_n)_{n\in\N}$ et $(a_n)_{n\in\N}$ sont adjacentes. \newline
Preuve de la croissance de $(c_n)_{n\in\N}$.
\begin{displaymath}
 c_n \leq b_n \leq a_n \Rightarrow
\left\lbrace 
\begin{aligned}
 \frac{1}{a_n} &\leq \frac{1}{c_n}\\
 \frac{1}{b_n} &\leq \frac{1}{c_n}
\end{aligned}
\right. 
\Rightarrow
c_{n+1}= \frac{3}{\frac{1}{a_n}+\frac{1}{b_n}+\frac{1}{c_n}} \geq c_n 
\end{displaymath}
Preuve de la décroissance de $(a_n)_{n\in\N}$.
\begin{displaymath}
 c_n \leq b_n \leq a_n \Rightarrow
\left\lbrace 
\begin{aligned}
 c_n &\leq a_n\\
 b_n &\leq a_n
\end{aligned}
\right. 
\Rightarrow
a_{n+1}= \frac{a_n+b_n+c_n}{3} \leq a_n 
\end{displaymath}
Majoration de la différence.
\begin{displaymath}
 b_n\leq a_n \Rightarrow a_{n+1} = \frac{a_n+b_n+c_n}{3} \leq \frac{2a_n+c_n}{3}
\end{displaymath}
D'autre part $c_{n+1}\geq c_n$ donc
\[
 a_{n+1} - c_{n+1} \leq \frac{2a_n + c_n}{3} - c_n = \frac{2}{3}(a_n - c_n).
\]
On en déduit que $(a_n - c_n)_{n\in\N}$ est majorée par une suite géométrique de raison $\frac{2}{3}  < 1$. Elle converge donc vers $0$ par le théorème d'encadrement.\newline
Il est alors évident, d'après le théorème d'encadrement encore, que $(b_n)_{n\in\N^*}$ converge vers la limite commune de $(a_n)_{n\in\N}$ et $(c_n)_{n\in\N}$.

\item 
Le point essentiel dans les deux questions suivantes est la formule
\begin{equation}
 a_{n+1}c_{n+1} = \frac{a_nb_nc_n(a_n+b_n+c_n)}{a_nb_n+b_nc_n+c_na_n}
\end{equation}
 \begin{enumerate} \item En particulier, si $a_nc_n=b_n^2$, la formule devient
\begin{displaymath}
 a_{n+1}c_{n+1} = \frac{b_n^3(a_n+b_n+c_n)}{a_nb_n+b_nc_n+b_n^2}=b_n^2 
\end{displaymath}
Comme tout est positif, lorsque $a_1c_1=b_1^2$ on obtient $a_2c_2=b_1^2=b_2^2$ et la relation se propage par récurrence, la suite des $b_n$ est alors constante. Les trois suites convergent vers $b_1$ qui est la moyenne géométrique de $a_1$ et $c_1$.

\item On va montrer que si $a_1c_1<b_1^2$, la suite des $b_n$ est décroissante. Remarquons que
\begin{displaymath}
 b_2^3 = a_1b_1c_1 < b_1^3 \Rightarrow b_2 < b_1.
\end{displaymath}
Il s'agit donc de montrer que $a_nc_n<b_n^2$ pour tous les entiers $n$.\newline
La relation (1) peut encore s'écrire $a_{n+1}c_{n+1} = f(a_nc_n)$ avec 
\[
 f: x\rightarrow\frac{ux}{x+v}, \hspace{1cm} u=b_n(a_n+b_n+c_n), \hspace{1cm} v = a_nb_n + b_nc_n.
\]
La fonction $f$ est croissante car elle peut s'écrire 
\begin{displaymath}
 f(x) = u - \frac{uv}{x+v}
\end{displaymath}
et que tout est strictement positif. Alors :
\begin{displaymath}
 a_nc_n<b_n^2 \Rightarrow a_{n+1}c_{n+1} = f(a_nc_n) <f(b_n^2)= b_n^2
\end{displaymath}
puis :
\begin{displaymath}
 b_{n+1}^3= a_{n+1}b_{n+1}c_{n+1} < b_n^3 \Rightarrow b_{n+1}<b_n
\end{displaymath}
Le raisonnement est analogue lorsque $a_1c_1 > b_1^2$ et conduit à une suite décroissante.
\end{enumerate}

\end{enumerate}
