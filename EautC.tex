%<dscrpt>Automorphismes de sous-groupes du cercle unité.</dscrpt>
On désigne par $(\U,.)$ le groupe des nombres complexes de module $1$ pour la multiplication dans $\C$. On s'intéresse ici aux morphismes de certains de ses sous-groupes.
\begin{enumerate}
 \item Soit $n$ un entier naturel supérieur ou égal à $2$ et $\U_n$ l'ensemble des racines $n$-ièmes de l'unité dans $\C$. On note $u=e^{\frac{2i\pi}{n}}$.
 \begin{enumerate}
  \item Montrer que $\U_n$ est un sous-groupe de $\U$.
  \item Soit $f$ un morphisme de groupe de $\U_n$ dans $\U_n$. Montrer qu'il existe un unique $m\in\{0,\cdots,n-1\}$ tel que:
\begin{displaymath}
 \forall v\in \U_n :\; f(v)=v^m
\end{displaymath}
 \end{enumerate}

\item Soit $f$ un morphisme de groupe de $\U$ dans $\U$. On définit $g$ de $\R$ dans $\C$ par :
\begin{displaymath}
 \forall t\in \R;\; g(t) = f(e^{it})
\end{displaymath}
et on suppose qu'elle est dérivable en $0$.
\begin{enumerate}
 \item Montrer que $g$ est dérivable dans $\R$ et que $g'(t)=g'(0)g(t)$ pour tous les réels $t$.
 \item Montrer qu'il existe un $m$ dans $\Z$ tel que $f(v)=v^m$ pour tous les $v$ dans $\U$.
\end{enumerate}

\end{enumerate}
