\subsection*{I. Préliminaires}
\begin{enumerate}
  \item Comme $1+j+j^2=0$ par définition; $j^2 = -1 - j$ c'est à dire $p=q=-1$.
  
  \item Si $q=0$, alors $\alpha$ est $0$ ou $p$ qui sont des entiers. Comme $\alpha$ est irrationnel, $q\neq0$.\newline
  Le complexe $\alpha$ est solution de l'équation du second degré $x^2-px-q$ dont le discriminant est $p^2+4q$. Si ce discriminant est nul, l'équation admet une unique solution et $\alpha = \frac{p}{2}\in \Q$. Comme $\alpha$ est irrationnel, $p^2+4q\neq 0$ et l'équation admet deux solutions distinctes. On note $\alpha'$ l'autre solution. D'après les résultats de cours sur les équations du second degré,
  \begin{displaymath}
    \alpha + \alpha' = p, \hspace{0.5cm} \alpha \alpha' = -q
  \end{displaymath}
On en déduit $\alpha'\in \Z[\alpha]$ car $\alpha' = p +(-1)\alpha$. Lorsque $\alpha \notin \R$, l'autre solution $\alpha'$ est la conjuguée complexe : $\alpha' = \overline{\alpha}$. \newline
Si $\alpha = \sqrt{d}$ avec $d\geq 2$ dans $\N$, alors $\alpha^2 = q$ donc $p=0$ et $\alpha' = -\alpha=-\sqrt{d}$.
  
  \item
  \begin{enumerate}
    \item Pour montrer que $\Z[\alpha]$ est un sous anneau de $\C$, on vérifie que $\Z[\alpha]$ contient l'unité et qu'il est stable pour les deux opérations.
    \begin{displaymath}
    1 = 1 + 0\alpha \Rightarrow 1 \in \Z[\alpha]\\
    \end{displaymath}
    Pour tout $z$ et $z'$ dans $\Z[\alpha]$, il existe $a$, $b$, $a'$, $b'$ dans $\Z$ tels que $z=a+b\alpha$, $z'=a'+b'\alpha$:
    \begin{align*}
      z+z' = \underset{\in \Z}{\underbrace{(a+a')}} + \underset{\in \Z}{\underbrace{(b+b')}}\alpha &\in \Z[\alpha]\\
      zz' = aa' + (ab'+ba')\alpha + bb'\alpha^2 
      = \underset{\in \Z}{\underbrace{(aa'+qbb')}} + \underset{\in \Z}{\underbrace{(ab'+ba'+paa')}}\alpha &\in \Z[\alpha]
    \end{align*}

    \item La décomposition d'un élément de $\Z[\alpha]$ est unique car $\alpha$ est irrationnel. En effet, pour $a$, $b$, $a'$, $b'$ dans $\Z$,
    \begin{displaymath}
      a+b\alpha = a' + b'\alpha \Rightarrow (b-b')\alpha = a' -a 
    \end{displaymath}
    Ce qui ne peut se produire que si $b=b'$ et $a=a'$.
  \end{enumerate}

  \item
  \begin{enumerate}
    \item La fonction $N_{\alpha}$ est à valeurs entières car, pour tous $a$ et $b$ dans $\Z$,
    \begin{multline*}
      N_{\alpha}(a+b\alpha) = (a+b\alpha)(a+b\alpha') = a^2+b^2\alpha\alpha' + ab(\alpha+\alpha')\\
      = a^2-qb^2 + pab \in \Z
    \end{multline*}
Pour $z_1$ et $z_2$ dans $\Z[\alpha]$, il existe des $a_1$, $b_1$, $a_2$, $b_2$ dans $\Z$ tels que $z_1=a_1+b_1\alpha$ et $z_2=a_2+b_2\alpha$. Alors:
\begin{multline*}
  N(z_1)N(z_2) = (a_1+b_1\alpha)(a_1+b_1\alpha')(a_2+b_2\alpha)(a_2+b_2\alpha')\\
  = \underset{=(z_1z_2)}{\underbrace{(a_1+b_1\alpha)(a_2+b_2\alpha)}} \left((a_1+b_1\alpha')(a_2+b_2\alpha') \right) 
\end{multline*}
De plus, comme $\alpha'$ est solution de la même équation que $\alpha$,
\begin{align*}
  z_1z_2 = (a_1+b_1\alpha)(a_2+b_2\alpha) &= a_1a_2+qb_1b_2 + (a_1b_2+b_1a_2+pb_1b_2)\alpha\\
  (a_1+b_1\alpha')(a_2+b_2\alpha') &= a_1a_2+qb_1b_2 + (a_1b_2+b_1a_2+pb_1b_2)\alpha'
\end{align*}
donc
\begin{displaymath}
  N(z_1)N(z_2) = N\left( a_1a_2+qb_1b_2 + (a_1b_2+b_1a_2+pb_1b_2)\alpha\right)
  = N(z_1z_2)
\end{displaymath}

    \item On a vu en question 2. que $\alpha'$ est le conjugué de $\alpha$ lorsque $\alpha\notin \R$. Donc, pour tous les $a$, $b$ entiers , $a+b\alpha'$ est le conjugué de $a+b\alpha$. On en déduit
    \begin{displaymath}
     \alpha\notin \R \Rightarrow \left(  \forall z\in \Z, \; N_{\alpha}(z) = |z|^2\right) 
    \end{displaymath}

    \item Le cas où $\varphi$ est le nombre d'or ne présente rien de particulier. On reprend l'expression de la question a. avec $p=q=1$,
    \begin{displaymath}
      \forall(a,b)\in \Z^2, \; N_{\varphi}(a+b\varphi) = a^2-b^2+ab
    \end{displaymath}
  \end{enumerate}
\end{enumerate}

\subsection*{II. Divisibilité}
\begin{enumerate}
  \item Si $z'$ divise $z$ dans $\Z[\alpha]$, il existe $q\in \Z[\alpha]$ tel que $z=qz'$. Alors $N(z) = N(q) N(z')$ est une relation entre entiers relatifs donc $N(z')$ divise $N(z)$ dans $\Z$.
  
  \item Inversibles de $\Z[\alpha]$.
    \begin{enumerate}
    \item Soit $u\in \Z[\alpha]$ tel que $|N_{\alpha}(u)|=1$.\newline
  Notons $N_{\alpha}(u)=\epsilon$ pour bien garder à l'esprit que sa valeur est $1$ ou $-1$. Il existe $a$ et $b$ dans $\Z$ tels que $u=a+b\alpha$ et, par définition de $N_{\alpha}$,
  \begin{displaymath}
    u\,(a+b\alpha') = \epsilon \Rightarrow u\,\left( \epsilon (a+b\alpha')\right)=1 
  \end{displaymath}
On en déduit que $u$ est un inversible de $\Z[\alpha]$ d'inverse $\epsilon (a+b\alpha')$.\newline
Soit $u\in I_{\alpha}$ c'est à dire inversible dans $\Z[\alpha]$.\newline
Il existe $v\in \Z[\alpha]$ tel que $uv=1$. Remarquons que $N_\alpha(1)=1$ par définition de $N_\alpha$. On en déduit $1=N_{\alpha}(u)N_{\alpha}(v)$ dans $\Z$ qui entraine que $N_{\alpha}(u)$ est inversible dans $\Z$ donc égal à $1$ ou $-1$.\newline
Si $\alpha\notin \R$, $\N_{\alpha}(z) = |z|^2$ donc seule la valeur $1$ est possible. L'ensemble des inversibles est l'intersection de $\Z[\alpha]$ avec l'ensemble $\U$ (complexes de module $1$).

    \item Les éléments de $I_i$ sont les $a+ib$ tels que $a^2+b^2=1$ avec $a$ et $b$ dans $\Z$. On en déduit
    \begin{displaymath}
      I_i = \left\lbrace 1, -1, i, -i\right\rbrace = \U_4 
    \end{displaymath}
Les éléments de $I_j$ sont les $a+bj$ tels que $|a+bj|^2=1$ avec $a$ et $b$ dans $\Z$. Or
\begin{displaymath}
  |a+bj|^2 = a^2 + b^2 +2ab\Re (j) = a^2+b^2-ab = (a-\frac{1}{2}b)^2 +\frac{3}{4}b^2
\end{displaymath}
Donc $|a+bj|^2 = 1$ entraine $|b|\leq 1$. Le cas $b=0$ conduit à $1$ et $-1$. Le cas $b=-1$ conduit à $\overline{j}$ ($a=0$) et $-j$ ($a=1$). Le cas $b=1$ conduit à $j$ ($a=0$) et $-\overline{j}$ ($a=1$). Comme $-\overline{j}=e^{\frac{2i\pi}{3}}$, on peut vérifier
\begin{displaymath}
  U_j = \left\lbrace 1, -1, j, -j, \overline{j}, -\overline{j}\right\rbrace = \U_6 
\end{displaymath}
Soit $d\in \N$ et $d\geq 2$. Un élément $u\in \Z[i\sqrt{d}$ est inversible si et seulement si il existe $a$ et $b$ dans $\Z$ vérifiant $u=a+ib\sqrt{d}$ avec $a^2 + db^2=1$. Comme $d\geq 2$, on doit avoir $b=0$ donc $a=\pm 1$. On en déduit
\begin{displaymath}
  I_{i\sqrt{d}} = \left\lbrace -1, 1\right\rbrace 
\end{displaymath}

    \item On a vu que $\varphi \varphi' = 1$ donc $\varphi$ est inversible d'inverse $-\varphi'$ et $\varphi'$ est inversible d'inverse $-\varphi$. On en déduit que toutes les puissances de ces nombres sont inversibles. Elles sont deux à deux distinctes car $\varphi$ et $\varphi'$ ne sont pas de module $1$. Il existe donc une infinité d'inversibles dans ce cas. 
    \item Soit $z$ un diviseur de $z'$ dans $\Z[\alpha]$ tel que $|N_{\alpha}(z)|=|N_{\alpha}(z')|$. Il existe alors $u\in \Z[\alpha]$ tel que $z'=uz$. On en déduit
    \begin{displaymath}
      N_{\alpha}(z') = N_{\alpha}(u) N_{\alpha}(z) = N_{\alpha}(u) N_{\alpha}(z')\Rightarrow N_{\alpha}(u)=1 
    \end{displaymath}
donc $u$ est inversible et $z=vz'$ avec $v$ l'inverse de $u$.
  \end{enumerate}

  \item Irréductibles de $\Z[\alpha]$.\newline
  Soit $z\in \Z[\alpha]$ avec $|N_{\alpha}(z)|$ premier (notons le $p$) et $d$ un diviseur de $z$.\newline
  Il existe alors $q\in\Z[\alpha]$ tel que $z=dq$. Alors, dans $\Z$, $N_{\alpha}(d)N_{\alpha}(q)=N_{\alpha}(z)$. Donc $|N_{\alpha}(d)|$ est $1$ ou $p$.
  \begin{itemize}
    \item Si $|N_{\alpha}(d)|=1$, alors $d$ est inversible.
    \item Si $|N_{\alpha}(d)|=p$ alors $|N_{\alpha}(q)|=1$ donc $q$ est inversible. Notons $q'$ son inverse. On en déduit $d=q'z\in Iz$.
  \end{itemize}
  On a bien montré que $\mathcal{D}(z) = I \cup Iz$ c'est à dire que $z$ est irréductible.\newline
  Un nombre naturel premier $p$ n'est pas forcément irréductible dans $\Z[\alpha]$ car cet anneau \emph{étend} $\Z$. Il est possible qu'il contienne des diviseurs de $p$. Par exemple dans $Z[\sqrt{p}]$, le nombre $p$ est le carré de $\sqrt{p}$ donc il n'est pas irréductible.
  
  \item Exemple avec $\Z[i\sqrt{6}]$.
  \begin{enumerate}
    \item Déterminer les $z\in\Z[i\sqrt{6}]$ tels que $N_{i\sqrt{6}}(z) = v$ revient à déterminer les couples $(a,b)\in \Z^2$ tels que $a^2 + 6b^2=v$.\newline
    Pour $v\in\left\lbrace 2, 3\right\rbrace$, on doit avoir $b=0$ sinon $6b^2>v$. Comme de plus ni $2$ ni $3$ ne sont des carrés d'entiers, les ensembles cherchés sont vides.

    \item On a déjà vu que si $z$ est un diviseur de $z'$ tel que $N_{\alpha}(z)=N_{\alpha}(z')$, il existe $u$ inversible tel que $z=uz'$. On utilisera plusieurs fois cette remarque.\newline
    Montrons que $2$ est irréductible.\newline
    Soit $z\in \Z[i\sqrt{6}]$ un diviseur de $2$. Alors $N_{i\sqrt{6}}(d)$ divise $N_{i\sqrt{6}}(2)=4$.
    \begin{align*}
       &N_{i\sqrt{6}}(d)=1 &\Rightarrow d\in I \\
       &N_{i\sqrt{6}}(d)=2 &\text{ impossible} \\
       &N_{i\sqrt{6}}(d)=4 &\Rightarrow d\in I\times 2 \text{ (d'après 2.d.) }
    \end{align*}
    Montrons que $-3$ est irréductible.\newline
    Soit $z\in \Z[i\sqrt{6}]$ un diviseur de $-3$. Alors $N_{i\sqrt{6}}(d)$ divise $N_{i\sqrt{6}}(-3)=9$.
    \begin{align*}
       &N_{i\sqrt{6}}(d)=1 &\Rightarrow d\in I \\
       &N_{i\sqrt{6}}(d)=3 &\text{ impossible} \\
       &N_{i\sqrt{6}}(d)=9 &\Rightarrow d\in I\times 3 \text{ (d'après 2.d.) }
    \end{align*}
    Montrons que $i\sqrt{6}$ est irréductible.\newline
    Soit $z\in \Z[i\sqrt{6}]$ un diviseur de $i\sqrt{6}$. Alors $N_{i\sqrt{6}}(d)$ divise $N_{i\sqrt{6}}(i\sqrt{6})=6$.
    \begin{align*}
       &N_{i\sqrt{6}}(d)=1 &\Rightarrow d\in I \\
       &N_{i\sqrt{6}}(d)=2 &\text{ impossible} \\
       &N_{i\sqrt{6}}(d)=3 &\text{ impossible} \\
       &N_{i\sqrt{6}}(d)=6 &\Rightarrow d\in I\times(i\sqrt{6}) \text{ (d'après 2.d.) }
    \end{align*}

    \item La relation $2\times (-3) = (i\sqrt{6})^2$ montre que le théorème de Gauss n'est pas valide dans l'anneau $\Z[i\sqrt{6}]$. En effet, d'après cette relation $i\sqrt{6}$ divise $2\times (-3)$ mais il n'a pas de diviseur commun (non inversible) avec $2$ et pas non plus avec $-3$.  
  \end{enumerate}
  
  \item Exemple avec $\Z[\sqrt{10}]$.
  \begin{enumerate}
    \item Il suffit de chercher les restes modulo $10$ des carrés des nombres de $0$ à $9$. Présentons les dans un tableau:
    \begin{center}
\begin{tabular}{c|c|c|c|c|c|c|c|c|c|}
 $0$ & $1$ & $2$ & $3$ & $4$ & $5$ & $6$ & $7$ & $8$ & $9$\\ \hline
 $0$ & $1$ & $4$ & $9$ & $6$ & $5$ & $6$ & $1$ & $4$ & $1$
    \end{tabular}
    \end{center}
L'ensemble des restes modulo $10$ des carrés d'entiers est $\left\lbrace 0, 1, 4, 5, 6, 9\right\rbrace$. 
    
    \item Dans ce cas, pour tous $a$ et $b$ entiers,
  \begin{displaymath}
    N_{\sqrt{10}}(a+b\sqrt{10}) = (a+b\sqrt{20})(a-b\sqrt{10}) = a^2 -10 b^2
  \end{displaymath}
Comme dans le cas de $\Z[i\sqrt{6}]$, pour chaque $w$ que l'on veut prouver irréductible, on calcule $N_{\sqrt{10}}(z)$ et on forme l'ensemble de ses diviseurs. Présentons les résultats dans un tableau
\begin{center}
\vspace{0.2cm}
\begin{tabular}{l|c|c}
$w$ & $N_{\sqrt{10}}(z)$ & diviseurs de $N_{\sqrt{10}}(z)$ dans $\N$\\ \hline
$2$ & $4$                & $1$, $2$, $4$\\ \hline
$3$ & $9$                & $1$, $3$, $9$\\ \hline
$4+\sqrt{10}$            & $6$ & $1$, $2$, $3$, $6$\\ \hline
$4-\sqrt{10}$            & $6$ & $1$, $2$, $3$, $6$
\end{tabular}
\end{center}
On cherche ensuite les $z\in \Z[\sqrt{10}]$ tels que $N_{\sqrt{10}}(z)=v$ pour les \og vrais\fg~ diviseurs trouvés c'est à dire $\pm 2$ ou $\pm 3$. Il s'agit chaque fois d'étudier l'équation
\begin{displaymath}
  a^2 - 10 b^2 = v ,\hspace{0.5cm} (a,b)\in \Z^2
\end{displaymath}
Mais une telle relation entraine $a^2 \equiv v \mod 10$ ce qui n'est pas possible car $2$, $3$, $-2 \equiv 8$, $-3 \equiv 7$ sont justement les nombres qui \emph{ne sont pas} des restes de carrés modulo $10$.\newline
Il n'existe donc pas de $z\in \Z[\sqrt{10}]$ tels que $N_{\sqrt{10}}(z)$ soit $\pm2$ ou $\pm3$. On en déduit comme dans la question $4$ que $2$, $3$, $4\pm\sqrt{10}$ sont irréductibles.
  \item La relation montre que le théorème de Gauss n'est pas vrai dans $\Z[\sqrt{10}]$.
  \end{enumerate}

\end{enumerate}

\subsection*{III. Division euclidienne}
\begin{enumerate}
  \item pour un $x$ réel, il existe $a\in\left\lbrace \lfloor x \rfloor , \lceil x \rceil \right\rbrace$ tel que $|x-a|\leq \frac{1}{2}$.  
  
  \item Supposons que $\Z[\alpha]$ vérifie la propriété d'approximation de l'énoncé. Soit $z$ et $z'$ dans $\Z[\alpha]$ avec $z'\neq 0$. Considérons $w = \frac{z}{z'}\in\C$. Il existe alors $w_{\alpha}\in \Z[\alpha]$ tel que 
  \begin{multline*}
    \left|\frac{z}{z'}-w_{\alpha}\right| < 1 \Rightarrow z = w_{\alpha}z' + r \text{ avec } r = z-w_{\alpha}z' \in \Z[\alpha] \\
    \text{ et } |r|^2 = |z'|^2 \left|\frac{z}{z'}-w_{\alpha}\right|^2 < |z'|^2
  \end{multline*}
Autrement dit, le quotient de la division euclidienne dans l'anneau est une bonne approximation du quotient complexe, le reste étant simplement le reste du \og développement idiot\fg~ pour ce quotient. 
  
  \item
  \begin{enumerate}
    \item Considérons un nombre complexe $w$ quelconque de partie réelle $u$ et de partie imaginaire $v$. Notons $\alpha_r = \Re(\alpha)$ et $\alpha_i = \Im(\alpha)$. Par hypothèse de cette partie, $\alpha_i\neq 0$. La condition sur $x$ et $y$ revient au système de deux équations aux inconnues $x$ et $y$ obtenu en identifiant les parties réelles et imaginaires.
\begin{displaymath}
  \left\lbrace 
  \begin{aligned}
    x + \alpha_r y &= u \\ \alpha_i y &= v
  \end{aligned}
  \right. 
\end{displaymath}
Ce système admet une unique solution car $\alpha_i \neq 0$. Il existe des réels $x$ et $y$ tels que $w = x +y \alpha$.    
    
    \item Considérons les $x$ et $y$ de la question précédente. D'après 1., il existe $a$ et $b$ dans $\Z$ tels que $|x-a| \leq \frac{1}{2}$ et $|y-b| \leq \frac{1}{2}$. Définissons $w_{\alpha}\in \Z[\alpha]$ par $w_{\alpha} = a+b\alpha$ et exprimons $|w-w_{\alpha}|^2$ avec la formule trouvée en I.4.a. 
\begin{multline*}
  |w-w_{\alpha}|^2 = (x-a)^2 - q(y-b)^2 + p(x-a)(x-b) \\
  \leq |x-a|^2 + |q||y-b|^2 + |p||x-a||x-b| \leq \frac{1 + |q| + |p|}{4}
\end{multline*}

    \item Les questions 3.a. et 2. montrent que si $\alpha$ est tel que $1+|p|+|q|<4$, alors $\Z[\alpha]$ est euclidien.
    \begin{itemize}
      \item Pour $\alpha= i$, $p=0$, $q=-1$, $1+|p|+|q|=2$.
      \item Pour $\alpha= i$, $p=-1$, $q=-1$, $1+|p|+|q|=3$.
      \item Pour $\alpha= i\sqrt{2}$, $p=0$, $q=-2$, $1+|p|+|q|=3$.
    \end{itemize}
Les anneaux $\Z[i]$, $\Z[j]$, $\Z[i\sqrt{2}$ sont donc bien euclidiens.
  \end{enumerate}

  \item
  \begin{enumerate}
    \item Ici $\alpha$ est le nombre complexe de partie imaginaire positive vérifiant $\alpha^2 = \alpha - 2$. En résolvant l'équation du second degré, on obtient
    \begin{displaymath}
      \alpha = \frac{1+i\sqrt{7}}{2}
    \end{displaymath}
Notons $x=\Re(w)$ et $y=\Im(w)$ et introduisons l'expression trouvée pour $\alpha$
\begin{multline*}
  |z-(u+v\alpha)|^2 = \left(x-u-\frac{v}{2} \right)^2 + \left( y-\frac{v\sqrt{7}}{2}\right)^2\\
  = \left(x -\frac{v}{2} - u\right)^2 + \frac{7}{4}\left( \frac{2y}{\sqrt{7}} -v\right)^2  
\end{multline*}
Il existe un entier $v$ tel que 
\begin{displaymath}
 \left|\frac{2y}{\sqrt{7}} -v\right|\leq \frac{1}{2} 
\end{displaymath}
Ce $v$ étant fixé, il existe un entier $u$ tel que
\begin{displaymath}
  \left|x -\frac{v}{2} - u\right|\leq \frac{1}{2}
\end{displaymath}
On définit ainsi $w_{\alpha} = u + v\alpha \in \Z[\alpha]$ tel que 
\begin{displaymath}
  \left|z-w_{\alpha}\right|^2 \leq \frac{1}{4} + \frac{7}{16} = \frac{11}{16} < 1
\end{displaymath}
On peut donc conclure avec la question 2.

    \item Ici $\alpha$ est le nombre complexe de partie imaginaire positive vérifiant $\alpha^2 = \alpha - 3$. En résolvant l'équation du second degré, on obtient
    \begin{displaymath}
      \alpha = \frac{1+i\sqrt{11}}{2}
    \end{displaymath}
Notons $x=\Re(w)$ et $y=\Im(w)$ et introduisons l'expression trouvée pour $\alpha$
\begin{multline*}
  |z-(u+v\alpha)|^2 = \left(x-u-\frac{v}{2} \right)^2 + \left( y-\frac{v\sqrt{11}}{2}\right)^2\\
  = \left(x -\frac{v}{2} - u\right)^2 + \frac{11}{4}\left( \frac{2y}{\sqrt{11}} -v\right)^2  
\end{multline*}
Il existe toujours de bonnes approximations entières, d'abord $v$ puis $u$. On définit ainsi $w_{\alpha} = u + v\alpha \in \Z[\alpha]$ tel que 
\begin{displaymath}
  \left|z-w_{\alpha}\right|^2 \leq \frac{1}{4} + \frac{11}{16} = \frac{15}{16} < 1
\end{displaymath}
On peut donc encore conclure avec la question 2.
  \end{enumerate}

\end{enumerate}

\subsection*{IV. Applications}
\begin{enumerate}
  \item Reproduire le cours, la terminaison est assurée par le fait que la suite des carrés des normes est strictement décroissante à valeurs dans $\N$. La preuve de l'invariance de l'intersection des ensembles des diviseurs de deux restes consécutifs est la même que dans $\Z$. On convient d'appeler pgcd des valeurs initiales l'entier de Gauss non nul renvoyé par l'algorithme.
  
  \item Le premier quotient est donné par le tableau, on compléte la troisième colonne
\begin{displaymath}
  8+9i - (2+i)(5+3i) = 8+9i -10 -6i -5i +3 = 1 - 2i \\
\end{displaymath}
  \begin{center}
\begin{tabular}{|l|l|l|l|} \hline
$N$ & $145$  & $34$  & $5$    \\ \hline
$a$ & $8+9i$ & $5+3i$& $1-2i$ \\ \hline
$q$ &        & $2+i$ & .      \\ \hline
$u$ & $1$    & $0$   & $1$    \\ \hline
$v$ & $0$    & $1$   & $-2-i$ \\ \hline
  \end{tabular}
  \end{center}
Pour trouver le quotient d'une division euclidienne, on approche les parties réelles et imaginaires par des entiers
\begin{displaymath}
  \frac{5+3i}{1-2i}=\frac{(5+3i)(1+2i)}{5} = \frac{-1+13i}{5} \simeq 3i
\end{displaymath}
Ce qui justifie l'approximation, c'est que le module du reste diminue
\begin{displaymath}
  5+3i-(3i)(1-2i) = 5+3i -3i -6 = -1 \text{ (carré module $1$)} 
\end{displaymath}
On complète la colonne suivante qui fournit les coefficients de Bezout et l'algorithme s'arrête
  \begin{center}
\begin{tabular}{|l|l|l|l|l|} \hline
$N$ & $145$  & $34$  & $5$    & $1$    \\ \hline
$a$ & $8+9i$ & $5+3i$& $1-2i$ & $-1$  \\ \hline
$q$ &        & $2+i$ & $3i$   & .      \\ \hline
$u$ & $1$    & $0$   & $1$    & $-3i$  \\ \hline
$v$ & $0$    & $1$   & $-2-i$ & $-2+6i$ \\ \hline
  \end{tabular}
  \end{center}
L'algorithme s'arrête car $-1$, de module $1$, est inversible.\newline
Les entiers de Gauss $a_0 = 8+9i$ et $a_1 = 5+3i$ sont étrangers. De plus,
\begin{displaymath}
  u_3 a_0 + v_3a_1 = (-3i)(8+9i) + (-2+6i)(5+3i) = -1
\end{displaymath}
L'algorithme d'Euclide étendu permet encore d'exprimer le pgcd de $a_0$ et $a_1$ comme combinaison de $a_0$ et $a_1$.

  \item Question de cours. Reproduire dans le cadre de $\Z[i]$ les preuves formulées dans $\Z$. Le théorème de Gauss se formule comme dans $\Z$.
  \begin{quote}
    Soit $u$, $v$, $w$ dans $\Z[\alpha]$ euclidien. Si $u$ divise $vw$ et si $u$ est étranger à $v$ alors $u$ divise $w$. 
  \end{quote}


  \item
  \begin{enumerate}
    \item Soit $x$ et $y$ dans $\Z$ tel que $x^2 + 2 = y^3$. Montrons par l'absurde que $x$ est impair. Si $x$ est pair, il existe $x'\in\Z$ tel que $x=2x'$:
\begin{displaymath}
4x'^2 + 2 =y^3 \Rightarrow y \text{ pair} \Rightarrow \exists y'\in \Z\text{ tq } 2x'^2 + 1 = 4 y'^2 \text{ impossible }  
\end{displaymath}

    \item Remarquons que $N_{i\sqrt{2}}(2i\sqrt{2})=8$ et $N_{i\sqrt{2}}(x-i\sqrt{2})=x^2+2$. Si $\delta \in \Z[i\sqrt{2}]$ est un diviseur commun, $N_{i\sqrt{2}}(\delta)$ doit diviser $8$ et $x^2+2$ dans $\Z$. Or ils sont premiers entre eux car $2$ est le seul diviseur premier de $8$ alors que $x^2+2$ est impair car $x^2$ est impair (car $x$ est impair). On doit donc avoir $N_{i\sqrt{2}}(\delta)=1$ donc $\delta$ inversible.\newline
     Comme $x-i\sqrt{2}$ et $2i\sqrt{2}$ sont étrangers, $x-i\sqrt{2}$ et $(x-i\sqrt{2}) + (2i\sqrt{2})=x+i\sqrt{2}$ le sont aussi.
    
    \item Si $y$ n'est pas irréductible, il admet un diviseur irréductible $z_1$.\newline
    Il existe alors $y_1\in \Z[i\sqrt{2}]$ tel que $y=z_1 y_1$. De plus $|y_1|^2 < |y|^2$ car un irréductible est de module strictement plus grand que $1$. On recommence avec $y_1$. Le processus s'arrêtera car la suite des carrés des modules est strictement décroissante dans $\N$.\newline
    Le caractère euclidien de l'anneau ne joue aucun rôle dans ce raisonnement.
    
    \item Considérons un facteur irréductible $z_i$ de $y$. On sait que $z_i^3m_i$ divise $(x+i\sqrt{2})(x-i\sqrt{2})$.
    \begin{itemize}
      \item Si $z_i$ ne divise pas $x+i\sqrt{2}$, alors $z_i^3m_i$ et $x+i\sqrt{2}$ sont étrangers donc $z_i^3m_i$ divise $x-i\sqrt{2}$.
      \item Si $z_i$ divise $x+i\sqrt{2}$, alors il ne divise pas $x-i\sqrt{2}$ (ils sont étrangers) donc $z_i^3m_i$ et $x-i\sqrt{2}$ sont étrangers donc $z_i^3m_i$ divise $x+i\sqrt{2}$.
    \end{itemize}
En regroupant les $z_i$ qui divisent le même facteur, on obtient que $x+i\sqrt{2}$ et $x-i\sqrt{2}$ sont des produits de cubes de $z_i$. Ils sont donc eux même des cubes. Il existe $a$ et $b$ entiers tels que $x+i\sqrt{2} = (a+ib\sqrt{2})^3$.\newline
Le caractère euclidien de l'anneau joue un rôle capital dans ce raisonnement car il utilise le théorème de Gauss qui est une conséquence de l'existence d'une division euclidienne.
    
    \item Traduisons la condition précédente par un système en séparant partie réelle et imaginaire
\begin{displaymath}
  x+i\sqrt{2} = (a+ib\sqrt{2})^3
\Leftrightarrow 
\left\lbrace 
\begin{aligned}
  a^3-6ab^2 &= x \\ 3a^2b - 2b^3 &= 1
\end{aligned}
\right. 
\Leftrightarrow 
\left\lbrace 
\begin{aligned}
  a(a^2-6b^2) &= x \\ (3a^2 - 2b^2)b &= 1
\end{aligned}
\right. 
\end{displaymath}
On en déduit $b=\epsilon\in\left\lbrace -1,+1\right\rbrace$ et le système devient
\begin{displaymath}
\left\lbrace 
\begin{aligned}
  a(a^2-6) &= x \\ 3a^2 - 2 &= \epsilon
\end{aligned}
\right.   
\Rightarrow \epsilon = 1, \, a = \pm 1, x = \pm 5
\end{displaymath}
Comme $25 +2 = 3^3$, les deux seuls couples solutions sont $(5,3)$ et $(-5,3)$.\newline
Ici, comme l'anneau est euclidien, $5+i\sqrt{2}$ est bien un cube
\begin{displaymath}
  (-1-i\sqrt{2})^3 = 5 + i\sqrt{2}
\end{displaymath}
  \end{enumerate}
  
  \item 
  \begin{enumerate}
      \item Il s'agit d'un exemple d'anneau non euclidien dans lequel le théorème de Gauss est faux. Les carrés des normes de $3$ et $1 \pm i\sqrt{26}$ sont $9$ et $27$. On raisonne comme dans II 4. et 5. L' équation dans $\Z$
  \begin{displaymath}
    a^2 + 26b^2 = v \text{ avec } v = 3 \text{ ou } 9
  \end{displaymath}
n'a pas de solution si $v=3$ et les solutions $-3$, $+3$ si $v=9$. On en déduit les irréductibilités demandées.

    \item Evidemment $(1+i\sqrt{26})(1-i\sqrt{26})= 27 = 3^3$. Pour autant, si on cherche $1+i\sqrt{26}$ comme un cube, on arrive aux équations
\begin{displaymath}
\left\lbrace 
\begin{aligned}
  a(a^2-48b^2) &= 1 \\ (3a^2 - 26b^2)b &= 1
\end{aligned}
\right. 
  \Rightarrow a=\pm 1 \text{ et } b = \pm 1 \Rightarrow \text{ contradiction }
\end{displaymath}
Ainsi, malgré la relation $(1+i\sqrt{26})(1-i\sqrt{26})= 3^3$, l'élément $1+i\sqrt{26}$ \emph{n'est pas un cube}. L'anneau $\Z[i\sqrt{26}]$ n'est pas euclidien.
  \end{enumerate}

\end{enumerate}
