%<dscrpt>Matrice semblable à son inverse</dscrpt>
Dans tout le problème \footnote{d'après Mines Albi,Alès,... 2002 MPSI}, $E$ est un $\R$-espace vectoriel de dimension 3.\newline
On notera $0$ l'endomorphisme nul, la matrice nulle et le vecteur nul.\newline
Pour deux matrices $A$ et $B$ de $\mathcal{M}_3(\R)$, on dira que la matrice $A$ est \emph{semblable} à la matrice $B$ s'il existe une matrice $P\in GL_3(\R)$ telle que 
\[B=P^{-1}AP\]
On notera $A\sim B$ lorsque la matrice $A$ est semblable à la matrice $B$.\newline
L'objet de ce problème est d'étudier des exemples de matrices semblables à leur inverse.
\subsubsection*{Partie A}
\begin{enumerate}
 \item Montrer que la relation $\sim$ est une relation d'équivalence sur $\mathcal{M}_3(\R)$.
\item Montrer que deux matrices de déterminants différents ne sont pas semblables.
\item Soit $u$ un endomorphisme de $E$ et $i$, $j$ deux entiers naturels. On considère l'application $w$ de $\ker u^{i+j}$ vers $E$ définie par :
\[w(x)=u^j(x)\]
\begin{enumerate}
 \item Montrer que $\Ima w \subset \ker u^i$.
\item En déduire que
\[\dim(\ker u^{i+j})\leq \dim(\ker u^{i}) + \dim(\ker u^{j})\]
\end{enumerate}
\item Soit $u$ un endomorphisme de $E$ vérifiant $u^3=0$ et $\rg u =2$.
\begin{enumerate}
 \item Montrer que $\dim (\ker u^2)=2$.
 \item Montrer qu'il existe un vecteur $a$ tel que $u^3(a)\neq 0$ et que la famille $(u^2(a),u(a),a)$ est alors une base de $E$.
 \item \'Ecrire la matrice $U$ de $u$  et la matrice $V$ de $v=u^2-u$ dans cette base.
\end{enumerate}

\end{enumerate}

\subsubsection*{Partie B}
Dans la suite de ce problème, la matrice $A$ de $\mathcal{M}_3(\R)$ est semblable à une matrice du type
\begin{displaymath}
% use packages: array
T = \left( \begin{array}{ccc}
1 & \alpha & \beta \\ 
0 & 1 & \gamma \\ 
0 & 0 & 1
    \end{array}\right) 
\end{displaymath}
On se propose de montrer que $A$ est semblable à son inverse $A^{-1}$.\newline
On pose
\begin{displaymath}
% use packages: array
N = \left( \begin{array}{ccc}
0 & \alpha & \beta \\ 
0 & 0 & \gamma \\ 
0 & 0 & 0
    \end{array}\right) 
\end{displaymath}
 et soit $P\in GL_3(\R)$ telle que 
\[P^{-1}AP=T=I_3+N\]
\begin{enumerate}
\item Expliquer pourquoi la matrice $A$ est bien inversible.
\item Calculer $N^3$ et montrer que
\[P^{-1}A^{-1}P=I_3-N+N^2\]
\item On suppose dans cette question que $N=0$. Montrer alors que les matrices $A$ et $A^{-1}$ sont semblables.
\item On suppose dans cette question que $\rg (N)=2$. On pose $M=N^2-N$.
\begin{enumerate}
 \item Montrer que la matrice $N$ est semblable à la matrice
\begin{displaymath}
% use packages: array
\left( \begin{array}{ccc}
0 & 1 & 0 \\ 
0 & 0 & 1 \\ 
0 & 0 & 0
    \end{array}\right) 
\end{displaymath}
et en déduire une matrice semblable à la matrice $M$.
\item Calculer $M^3$ et déterminer $\rg (M)$.
\item Montrer que les matrices $M$ et $N$ sont semblables.
\item Montrer que les matrices $A$ et $A^{-1}$ sont semblables.
\end{enumerate}
\item On suppose dans cette question que $\rg (N)=1$. On pose $M=N^2-N$. Montrer que les matrices $A$ et $A^{-1}$ sont semblables.
\item Exemple. Soit la matrice 
\begin{displaymath}
% use packages: array
A = \left( \begin{array}{ccc}
1 & 0 & 0 \\ 
0 & 0 & -1 \\ 
0 & 1 & 2
    \end{array}\right) 
\end{displaymath}
On note $(a,b,c)$ une base de $E$ et $u$ l'endomorphisme de $E$ de matrice $A$ dans cette base.
\begin{enumerate}
 \item Montrer que $\ker (u- Id_E)$ est un sous-espace vectoriel de $E$ de dimension 2 dont on donnera une base $(e_1,e_2)$.
\item Justifier que la famille $(e_1,e_2,c)$ est une base de $E$ et écrire la matrice de $u$ dans cette base.
\item Montrer que les matrices $A$ et $A^{-1}$ sont semblables.
\item Réciproquement, soit 
\begin{displaymath}
% use packages: array
T = \left( \begin{array}{ccc}
1 & \alpha & \beta \\ 
0 & 1 & \gamma \\ 
0 & 0 & 1
    \end{array}\right) 
\end{displaymath}
Toute matrice de $\mathcal{M}_3(\R)$ semblable à son inverse est-elle semblable à une matrice de la forme $T$ ?
\end{enumerate}

\end{enumerate}
