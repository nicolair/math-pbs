%<dscrpt>Intersections d'un cône par des plans.</dscrpt>
Dans un espace muni d'un repère orthonormé direct $\mathcal R = (O,(\overrightarrow i ,\overrightarrow j ,\overrightarrow k))$, on note $\Gamma$ l'ensemble des points $M$ dont les coordonnées $(x(M),y(M),z(M))$ vérifient
\begin{displaymath}
 x(M)^2 + y(M)^2 = \frac{1}{3}(z(M)-2)^2
\end{displaymath}
\begin{enumerate}
 \item Déterminer l'équation cartésienne de la courbe obtenue par l'intersection de $\Gamma$ avec le plan $xOy$. Quelle est la nature de cette courbe ?

 \item 
\begin{enumerate}
 \item Déterminer la nature de l'intersection de $\Gamma$ avec le plan d'équation $x=0$.
 \item Déterminer la nature de l'intersection de $\Gamma$ avec le plan d'équation $x=k$ où $k$ est un réel non nul fixé.
\end{enumerate}

 \item Soit $\varphi$ un réel quelconque. On considère un nouveau repère orthonormé direct
\begin{displaymath}
\mathcal R_\varphi = (O,(\overrightarrow I_\varphi ,\overrightarrow J_\varphi ,\overrightarrow K_\varphi))
\text{ avec }
 \overrightarrow I_\varphi = \cos \varphi \overrightarrow i + \sin \varphi \overrightarrow j  
\text{ et } \overrightarrow K_\varphi = \overrightarrow k
\end{displaymath}
On note $X_\varphi (M)$, $Y_\varphi (M)$, $Z_\varphi (M)$ les coordonnées d'un point $M$ dans ce nouveau repère.
\begin{enumerate}
 \item Que vaut $\overrightarrow J_\varphi$? 
 \item Exprimer l'équation de $\Gamma$ à l'aide des coordonnées $X_\varphi(M)$, $Y_\varphi(M)$, $Z_\varphi(M)$.
 \item Déterminer la nature de l'intersection de $\Gamma$ avec un plan parallèle à l'axe $Oz$.
\end{enumerate}

 \item Soit $A$ le point de coordonnées $(0,0,2)$ dans $\mathcal{R}$.\newline
 On note $(X(M),Y(M),Z(M))$ les coordonnées de $M$ dans le repère $\mathcal R _A = (A,(\overrightarrow i ,\overrightarrow j ,\overrightarrow k))$. Pour $a$, $b$, $c$ réels (non tous nuls), on note $\overrightarrow n$ le vecteur
\begin{displaymath}
 \overrightarrow n = a\overrightarrow i + b\overrightarrow j + c\overrightarrow k 
\end{displaymath}
et $\mathcal P_{\overrightarrow n}$ le plan passant par $A$ et orthogonal à $\overrightarrow n$.
\begin{enumerate}
 \item Exprimer l'équation de $\Gamma$ à l'aide des coordonnées $X$, $Y$, $Z$.
 \item Montrer que l'intersection de $\Gamma$ avec $\mathcal P_{\overrightarrow n}$ contient un point autre que $A$ si et seulement si
\begin{displaymath}
 a^2 + b^2 \geq 3c^2
\end{displaymath}
 On commencera par traiter le cas $c\neq 0$.
 \item  Déterminer la nature de l'intersection de $\Gamma$ avec un plan passant par $A$ lorsqu'elle ne se réduit pas au point $A$.
\end{enumerate}

\end{enumerate}
