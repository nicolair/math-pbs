\subsection*{Partie I - Angles d'Euler}
\begin{enumerate}
 \item Par définition des rotations :
\begin{displaymath}
\left( r_2(\overrightarrow{i})=\overrightarrow{u}, \;
r_1(\overrightarrow{u})=\overrightarrow{u}, \;
r_3(\overrightarrow{u})=\overrightarrow{i_1}\right) 
\hspace{0.3cm} \Rightarrow \hspace{0.3cm}
r_3 \circ r_2 \circ r_1 (\overrightarrow{i})=\overrightarrow{i}
\end{displaymath}
De même 
\begin{displaymath}
 \left( r_2(\overrightarrow{k})=\overrightarrow{k}, \;
 r_1(\overrightarrow{k})=\overrightarrow{k_1},\;
 r_3(\overrightarrow{k_1})=\overrightarrow{k_1}\right) 
 \hspace{0.3cm} \Rightarrow \hspace{0.3cm}
r_3 \circ r_2 \circ r_1 (\overrightarrow{k})=\overrightarrow{k_1}. 
\end{displaymath}
La rotation composée transforme la base orthonormée directe $(\overrightarrow{i},\overrightarrow{j},\overrightarrow{k})$ en une base orthonormée directe $(\overrightarrow{i_1},\overrightarrow{w},\overrightarrow{k_1})$. La seule base orthonormée directe dont le premier et le troisième vecteur sont $\overrightarrow{i_1}$ et $\overrightarrow{k_1}$ est $(\overrightarrow{i_1},\overrightarrow{j_1},\overrightarrow{k_1})$. On en déduit
\begin{displaymath}
 r_3 \circ r_2 \circ r_1 = r
\end{displaymath}

\item La fonction $R=f\circ r_{\overrightarrow{w},\alpha}\circ f^{-1}$ est une rotation car elle est composée de plusieurs rotations (les rotations forment un groupe pour la composition). On va montrer que c'est une rotation d'angle $\alpha$ autour de l'axe $f(\overrightarrow{w})$.\newline
On vérifie facilement que $R(f(\overrightarrow{w}))=f(\overrightarrow{w})$. Pour montrer que l'angle est $\alpha$, on utilise la formule de changement de base.\newline
Soit $\mathcal U=(\overrightarrow{u},\overrightarrow{v},\overrightarrow{w})$ une base orthonormée directe et $\mathcal V=(f(\overrightarrow{u}),f(\overrightarrow{v}),f(\overrightarrow{w}))$. La famille $\mathcal V$ est également orthonormée directe car $f$ est une rotation. Alors:
\begin{displaymath}
\Mat_{\mathcal U} r_{\overrightarrow{w},\alpha} =
\begin{pmatrix}
 \cos \alpha & -\sin \alpha & 0 \\
 \sin \alpha & \cos \alpha & 0\\
0 & 0 & 1
\end{pmatrix}
\end{displaymath}
On peut alors regarder la matrice de $f$ comme une matrice de passage entre deux bases (elle exprime les vecteurs de $\mathcal V$ en fonction de ceux de $\mathcal U$) puis utiliser la formule de changement de base.
\begin{multline*}
 \Mat_{\mathcal U}f = P_{\mathcal V \mathcal U} \Rightarrow 
 \Mat_{\mathcal V}R=  P_{\mathcal U \mathcal V}\,\Mat_{\mathcal U}R\,P_{\mathcal V \mathcal U}\\
=P_{\mathcal U \mathcal V} 
    \left( \Mat_{\mathcal U}f \, \Mat_{\mathcal U}r_{\overrightarrow{w},\alpha} \, \Mat_{\mathcal U}f^{-1}\right) P_{\mathcal V \mathcal U}\\
= \left( P_{\mathcal U \mathcal V}\,P_{\mathcal V \mathcal U}\right)
  \Mat_{\mathcal U}r_{\overrightarrow{w},\alpha}
\left(  P_{\mathcal U \mathcal V}\,P_{\mathcal V \mathcal U} \right) 
=\Mat_{\mathcal U}r_{\overrightarrow{w},\alpha}
\end{multline*}
La forme de cette matrice montre que $R$ est la rotation d'angle $\alpha$ autour de $f(\overrightarrow{w})$.
\item On remarque que $r_\varphi$ et $r_\psi$ sont des rotations de même axe. Elles vont donc commuter. Comme $r_1$ et $R_\theta$ sont des rotations d'angle $\theta$ mais respectivement autour de $\overrightarrow{u}$ et $\overrightarrow{i}$ avec $\overrightarrow{u}=r_\varphi(\overrightarrow{i})$, la question précédente montre que
\begin{displaymath}
 r_1 = r_\varphi \circ R_\theta \circ r_\varphi ^{-1}
\end{displaymath}
De même, comme $r_1(\overrightarrow{k})=\overrightarrow{k_1}$ :
\begin{displaymath}
 r_3 = r_{\overrightarrow{k_1},\varphi}=r_1\circ r_\varphi \circ r_1^{-1}
\end{displaymath}
On en déduit :
\begin{multline*}
 r = r_3\circ r_1 \circ r_2 = r_1 \circ r_\psi \circ r_1^{-1}\circ r_1 \circ r_2 
= r_1 \circ r_\psi \circ r_2 \\
= r_\varphi \circ R_\theta \circ r_\varphi^{-1}\circ r_\psi \circ r\varphi 
=  r_\varphi \circ R_\theta \circ r_\psi
\end{multline*}
car $r_\varphi$ et $r_\psi$ commutent. Le point intéressant dans cette décomposition est que les axes des trois rotations soient dirigés par les vecteurs $\overrightarrow{i}$ et $\overrightarrow{k}$ de la base de départ.
\end{enumerate}

\subsection*{Partie II - Quaternions}
Les questions de cette partie se traitent par de simples vérifications. Leur correction ne sera pas détaillée.

\subsection*{Partie III - Multiplications}
\begin{enumerate}
 \item La vérification de ce que $S$, $g_q$, $d_q$, $C_q$ sont des endomorphismes ne pose pas de difficulté. Bien remarquer qu'il s'agit d'une structure de $\R$-espace vectoriel.\newline
Soit $q^\prime$ un quaternion quelconque, on peut écrire :
\begin{equation*}
 d_{q^{-1}}(q^\prime) = q^\prime q^{-1} = \dfrac{1}{N(q)}q^\prime \overline{q} = \dfrac{1}{N(q)}\overline{q\overline{q^\prime}}
\end{equation*}
donc
\begin{displaymath}
 d_{q^{-1}} = \dfrac{1}{N(q)} \,S \circ g_q \circ S
\end{displaymath}
De même :
\begin{displaymath}
 C_q=g_q \circ d_{q^{-1}}= \dfrac{1}{N(q)} \,g_q \circ S \circ g_q \circ S
\end{displaymath}
\item \begin{enumerate}
 \item Pour former la matrice de $g_q$, on exprime les images des vecteurs de base en fonction de $(1_{\mathbb{H}},\overrightarrow{i},\overrightarrow{j},\overrightarrow{k})$. On peut se permettre de ne pas écrire complètement certaines matrices car on \emph{sait} qu'il s'agit de quaternions.
\begin{displaymath}
 g_q(1)= \alpha 1_{\mathbb{H}} + \delta \overrightarrow{i} + \gamma \overrightarrow{j} + \beta \overrightarrow{k}
\end{displaymath}
\begin{multline*}
 g_q(\overrightarrow{i}) = 
 \begin{bmatrix}
a  & -\overline{b} \\
b  & \overline{a}
\end{bmatrix}
\begin{bmatrix}
0  & i \\
i  & 0 
\end{bmatrix}
=
\begin{bmatrix}
 -i\overline{b} & . \\
 i\overline{a} &  .
\end{bmatrix}
=
\begin{bmatrix}
 -i\gamma -\delta & . \\
 i\alpha + \beta & .
\end{bmatrix}
 \\
 = -\delta 1_{\mathbb{H}} + \alpha \overrightarrow{i} + \beta \overrightarrow{j}  -\gamma \overrightarrow{k}
\end{multline*}
\begin{multline*}
 g_q(\overrightarrow{j}) = 
 \begin{bmatrix}
a  & -\overline{b} \\
b  & \overline{a}
\end{bmatrix}
\begin{bmatrix}
0  & -1 \\
1  & 0 
\end{bmatrix}
=
\begin{bmatrix}
 -\overline{b} & . \\
 \overline{a} &  .
\end{bmatrix}
=
\begin{bmatrix}
 -\gamma +i\delta & . \\
 \alpha -i\beta & .
\end{bmatrix}
 \\
 = -\gamma 1_{\mathbb{H}} - \beta \overrightarrow{i} + \alpha \overrightarrow{j} + \delta \overrightarrow{k}
\end{multline*}
\begin{multline*}
 g_q(\overrightarrow{k}) = 
 \begin{bmatrix}
a  & -\overline{b} \\
b  & \overline{a}
\end{bmatrix}
\begin{bmatrix}
i  & 0 \\
0  & -i 
\end{bmatrix}
=
\begin{bmatrix}
 ia & . \\
 ib &  .
\end{bmatrix}
=
\begin{bmatrix}
 i\alpha -\beta & . \\
 i\gamma - \delta & .
\end{bmatrix}
 \\
 = -\beta 1_{\mathbb{H}} + \gamma \overrightarrow{i} - \delta \overrightarrow{j} + \alpha \overrightarrow{k}
\end{multline*}
On en déduit :
\begin{displaymath}
 \mathop{\mathrm{Mat}}_{\mathcal B}g_q = 
\begin{bmatrix}
 \alpha & -\delta & -\gamma & -\beta \\
 \delta & \alpha & -\beta & \gamma \\
 \gamma & \beta & \alpha & -\delta \\
 \beta & -\gamma & \delta & \alpha
\end{bmatrix}
\end{displaymath}
\item La matrice précédente s'écrit avec des blocs $2\times 2$ $A$, $B$ :
\begin{displaymath}
 \det g_q =
\begin{vmatrix}
A  & -B \\
B  & A 
\end{vmatrix}
\end{displaymath}
Ce déterminant n'est pas modifié par des opérations élémentaires sur les blocs :
\begin{multline*}
 \det g_q =
\begin{vmatrix}
A  & -B + iA \\
B  & A + iB 
\end{vmatrix}
= \begin{vmatrix}
A  & i(A + iB) \\
B  & A + iB 
\end{vmatrix}
= \begin{vmatrix}
A - iB  & 0 \\
B  & A + iB 
\end{vmatrix}
\\
=\vert \det(A+iB)\vert ^2
= \left\vert \begin{vmatrix}
\alpha + i\gamma  & -\delta + i\beta \\
\delta + i\beta  & \alpha - i\gamma 
\end{vmatrix} \right\vert ^2 
\\
= \vert (\alpha + i\gamma)(\alpha - i\gamma)-(\delta - i\beta)(\delta + i\beta)\vert^2
=(\alpha^2 + \beta^2 + \gamma^2 + \delta^2)^2 = N(q)^2
\end{multline*}
On en déduit :
\begin{displaymath}
 \det g_q = N(q)^2
\end{displaymath}
\end{enumerate}
\item L'égalité entre applications linéaires 
\begin{displaymath}
 C_q = \dfrac{1}{N(q)} g_q \circ S \circ g_q \circ S
\end{displaymath}
se traduit par l'égalité suivante entre les déterminants (attention, l'espace est de dimension $4$):
\begin{displaymath}
 \det C_q = \dfrac{1}{N(q)^4} (\det g_q)^2 (\det S)^2
\end{displaymath}
Or $(\det S)^2=1$ car $S\circ S$ est l'identité. On en déduit :
\begin{displaymath}
 \det C_q = 1
\end{displaymath}
\end{enumerate}

\subsection*{Partie IV - Produit scalaire}
Dans cette partie $\overrightarrow{u}$ et $\overrightarrow{v}$ sont deux quaternions purs respectivement de coordonnées $(\gamma , \delta , \beta)$ et $(\gamma^\prime , \delta^\prime , \beta^\prime)$ dans la base 
$(\overrightarrow{i} , \overrightarrow{j} , \overrightarrow{k})$.
\begin{align*}
 \overrightarrow{u} =&
\gamma \overrightarrow{i} +\delta \overrightarrow{j} + \beta \overrightarrow{k}
=
\begin{bmatrix}
i\beta  & -\gamma + i\delta \\
\gamma + i\delta  & -i\beta
\end{bmatrix}
\\
 \overrightarrow{v}
=& 
\gamma^\prime \overrightarrow{i} +\delta^\prime \overrightarrow{j} + \beta^\prime \overrightarrow{k}
=
\begin{bmatrix}
i\beta^\prime  & -\gamma^\prime + i\delta^\prime \\
\gamma^\prime + i\delta^\prime  & -i\beta^\prime
\end{bmatrix}
\end{align*}

\begin{enumerate}
 \item Pour vérifier que $(./.)$ définit un produit scalaire, formons le produit matriciel des quaternions.
\begin{displaymath}
 \overrightarrow{u}\overrightarrow{u^\prime} 
=
\begin{bmatrix}
-\beta \beta^\prime -\gamma \gamma^\prime -\delta \delta^\prime +i(\delta \gamma^\prime - \gamma \delta^\prime) & . \\
-\delta \beta^\prime + \beta \delta^\prime + i(\gamma \beta^\prime -\gamma^\prime \beta)  & .
\end{bmatrix}
\end{displaymath}
On en déduit l'expression du produit scalaire
\begin{displaymath}
 (\overrightarrow{u}/\overrightarrow{v}) = \dfrac{1}{2}\tr (\overrightarrow{u}\overrightarrow{v}) = 
\beta\beta^\prime + \gamma\gamma^\prime + \delta \delta^\prime
\end{displaymath}
Ceci montre en même temps que $(\overrightarrow{i}, \overrightarrow{j},\overrightarrow{k})$ est une base orthonormée. Elle est directe par définition de l'orientation de l'espace $E$ des quaternions purs.

\item Dans l'espace vectoriel euclidien oriené $E$, le calcul en coordonnées (dans une base orthonormée directe) du produit vectoriel $\overrightarrow{u}\overrightarrow{v}$ donne
\begin{displaymath}
 \begin{pmatrix}
  \delta \\
  \gamma \\
 \beta
 \end{pmatrix}
\wedge
 \begin{pmatrix}
  \delta^\prime \\
  \gamma^\prime \\
 \beta^\prime
 \end{pmatrix}
=
 \begin{pmatrix}
  \gamma \beta^\prime -\gamma^\prime \beta \\
  \beta \delta^\prime - \beta^\prime \delta \\
 \delta \gamma^\prime - \delta^\prime \gamma
 \end{pmatrix}
\end{displaymath}
On retrouve des expressions figurant dans le produit matriciel calculé plus haut, on en déduit :
\begin{displaymath}
\overrightarrow{u} \overrightarrow{v} = 
-(\overrightarrow{u}/\overrightarrow{v})1_{\mathbb H} + \overrightarrow{u} \wedge \overrightarrow{v}
\end{displaymath}
Les autres expressions demandées par l'énoncé en découlent immédiatement.
\end{enumerate}

\subsection*{Parties V - Rotations}
\begin{enumerate}
 \item \begin{enumerate}
 \item On doit montrer que l'image par l'application $C_q$ d'un quaternion pur $\overrightarrow{u}$ est encore un quaternion pur. On utilise la conjugaison (un quaternion est pur lorsqu'il est égal à l'opposé de son conjugué).
\begin{displaymath}
 C_q(\overrightarrow{u}) = \dfrac{1}{N(q)} q \overrightarrow{u} \overline{q}
\end{displaymath}
\begin{displaymath}
\overline{C_q(\overrightarrow{u})} = \dfrac{1}{N(q)} q \overline{\overrightarrow{u}} \overline{q} = \dfrac{1}{N(q)} q (-\overrightarrow{u}) \overline{q} = -C_q(\overrightarrow{u})
\end{displaymath}
On en déduit que $C_q(\overrightarrow{u})$ est un quaternion pur.
\item On note $c_q$ la restriction de $C_q$ à $E$. Comme $C_q(1_{\mathbb H}) = 1_{\mathbb H}$, la matrice de $C_q)$ dans la base $(1_{\mathbb H}, \overrightarrow i,\overrightarrow j,\overrightarrow k)$ est de la forme
\begin{displaymath}
 \begin{bmatrix}
  1 &  
\begin{matrix}
 0 & 0 & 0
\end{matrix}

\\
\begin{matrix}
 0 \\
 0 \\
 0
\end{matrix}
  & \mathop{\mathrm{Mat}}_{(\overrightarrow i,\overrightarrow j,\overrightarrow k)} c_q
 \end{bmatrix}
\end{displaymath}
D'après la définition du déterminant d'une matrice (ou en développant suivant la première colonne), on obtient
\begin{displaymath}
 \det c_q = \det C_q =1
\end{displaymath}
\item Comme $c_q$ est de déterminant 1, pour montrer que c'est une rotation, il suffit de montrer qu'il conserve le produit scalaire.
\begin{multline*}
 (c_q(\overrightarrow u)/c_q(\overrightarrow v)) = -\dfrac{1}{2} \tr (c_q(\overrightarrow u)c_q(\overrightarrow v))
= -\dfrac{1}{2} \tr (q \overrightarrow u q^{-1} q \overrightarrow v q^{-1}) \\
= -\dfrac{1}{2} \tr (q \overrightarrow u \overrightarrow v q^{-1}) 
= -\dfrac{1}{2} \tr (\overrightarrow u \overrightarrow v q^{-1} q) 
= -\dfrac{1}{2} \tr (\overrightarrow u \overrightarrow v )
= (\overrightarrow u / \overrightarrow v)
\end{multline*}
en utilisant le fait que la trace d'un produit de deux matrices ne change pas si on les permute 
\end{enumerate}
\item \begin{enumerate}
 \item Rappelons que
\begin{align*}
 q = \begin{bmatrix}
 a & -\overline{b} \\
 b & \overline{a}
     \end{bmatrix}
&,&
\overline{q} = \begin{bmatrix}
\overline{a}  & \overline{b} \\
 -b & a
\end{bmatrix}
\end{align*}
\begin{displaymath}
c_q(\overrightarrow i) = \dfrac{1}{N(q)} q 
\begin{bmatrix}
 0 & i \\
 i & 0
\end{bmatrix}
\overline{q}
= \dfrac{1}{N(q)} q 
\begin{bmatrix}
 -ib & ia \\
 i\overline{a} & i\overline{b}
\end{bmatrix} 
= \dfrac{1}{N(q)} 
\begin{bmatrix}
 . & . \\
 -ib^2+i\overline{a}^2 & 0
\end{bmatrix}
\end{displaymath}
\begin{displaymath}
 (c_q(\overrightarrow i)/\overrightarrow i) 
= \dfrac{\Im(-ib^2+i\overline{a}^2)}{N(q)}
=  \dfrac{\alpha^2 -\beta^2 -\gamma^2 +\delta^2}{N(q)}
\end{displaymath}

\begin{displaymath}
c_q(\overrightarrow j) = \dfrac{1}{N(q)} q 
\begin{bmatrix}
 0 & -1 \\
 1 & 0
\end{bmatrix}
\overline{q}
= \dfrac{1}{N(q)} q 
\begin{bmatrix}
 b & -a \\
 \overline{a} & \overline{b}
\end{bmatrix} 
= \dfrac{1}{N(q)} 
\begin{bmatrix}
 . & . \\
 b^2+\overline{a}^2 & 0
\end{bmatrix}
\end{displaymath}
\begin{displaymath}
 (c_q(\overrightarrow j)/\overrightarrow j)
= \dfrac{\Re(b^2+\overline{a}^2)}{N(q)}
= \dfrac{\gamma^2 -\delta^2 +\alpha^2 -\beta^2}{N(q)}
\end{displaymath}

\begin{displaymath}
c_q(\overrightarrow k) = \dfrac{1}{N(q)} q 
\begin{bmatrix}
 i & 0 \\
 0 & -i
\end{bmatrix}
\overline{q}
= \dfrac{1}{N(q)} q 
\begin{bmatrix}
 i\overline{a} & i\overline{b} \\
 ib & -ia
\end{bmatrix} 
= \dfrac{1}{N(q)} 
\begin{bmatrix}
 i|a|^2 -i|\overline{b}|^2 & . \\
 . & .
\end{bmatrix}
\end{displaymath}
\begin{displaymath}
 (c_q(\overrightarrow k)/\overrightarrow k)
= \dfrac{\Im(i|a|^2 -i|\overline{b}|^2)}{N(q)}
= \dfrac{\alpha^2 +\beta^2 -\gamma^2 -\delta^2}{N(q)}
\end{displaymath}
\item On déduit de la question précédente que
\begin{displaymath}
 \tr c_q = \dfrac{3\alpha^2 -\beta^2 -\gamma^2 -\delta^2}{\alpha^2 +\beta^2 +\gamma^2 +\delta^2}
\end{displaymath}
Cette trace est égale à 3 si et seulement si
\begin{displaymath}
 3\alpha^2 -\beta^2 -\gamma^2 -\delta^2 = 3(\alpha^2 +\beta^2 +\gamma^2 +\delta^2)
\end{displaymath}
c'est à dire lorsque $\beta^2 +\gamma^2 +\delta^2=0$ ou encore que $q\in \Vect (1_{\mathbb H})$.
\end{enumerate}
\item Lorsque $q\not\in \Vect (1_{\mathbb H})$, $c_q$ n'est pas l'identité car la trace de $c_q$ n'est pas égale à la trace de l'identité (qui est 3). De plus :
\begin{align*}
 C_q(q) =& q q q^{-1} =q\\
C_q(\overline{q}) =& \dfrac{1}{N(q)}qq^{-1}q^{-1}  = \dfrac{1}{N(q)}q^{-1} = \overline{q}\\
c_q(\overrightarrow V_q) =& \dfrac{1}{2}C_q(q-\overline{q})= \dfrac{1}{2}(q-\overline{q})= \overrightarrow V_q
\end{align*}
\item En utilisant les calculs de la partie IV et les décompositions
\begin{align*}
 q = \alpha 1_{\mathbb H} + \overrightarrow V_q
&,&
 \overline{q} = \alpha 1_{\mathbb H} - \overrightarrow V_q
\end{align*}
on obtient
\begin{align*}
 q\overrightarrow u \overline{q} =&
\alpha^2 \overrightarrow u + 2\alpha \overrightarrow V_q \wedge \overrightarrow u -(\overrightarrow V_q \wedge \overrightarrow u)\wedge \overrightarrow V_q \\
 \overline{q}\overrightarrow u q =&
\alpha^2 \overrightarrow u - 2\alpha \overrightarrow V_q \wedge \overrightarrow u -(\overrightarrow V_q \wedge \overrightarrow u)\wedge \overrightarrow V_q \\
(c_q-c_q^{-1})(\overrightarrow u) =& \dfrac{4\alpha}{N(q)} \overrightarrow V_q\wedge \overrightarrow u
\end{align*}
On sait déjà que $c_q$ est une rotation, cette rotation est un demi-tour lorsque $c_q\circ c_q$ est l'identité c'est à dire lorsque $c_q = c_q^{-1}$. Comme $\overrightarrow V_q$ n'est pas nul, ceci se produit si et seulement si $\alpha=0$ c'est à dire lorsque $q\in E$ ($q$ est un quaternion pur).

On suppose dans toute la suite que $q\not \in \Vect 1_{\mathbb{H}}$ et $q\not \in E$. Il existe alors un unique $\theta \in ]-\pi , \pi[$ tel que  $c_q=r_{\theta , \overrightarrow{V}_q}$ car $c_q$ est une rotation qui n'est pas un demi-tour.
\item
\begin{enumerate}
 \item Lorsque $c_q=r_{\theta , \overrightarrow{V}_q}$ la matrice de $c_q$ dans une base orthonormée directe de la forme $\mathcal U = (\overrightarrow{a}, \overrightarrow{b}, \frac{1}{N(\overrightarrow{V}_q)}\overrightarrow{V}_q)$ est 
\begin{displaymath}
\mathop{\mathrm{Mat}}_{\mathcal U} c_q
 \begin{bmatrix}
  \cos \theta & -\sin \theta & 0 \\
  \sin \theta & \cos \theta & 0 \\
  0 & 0 & 1 
 \end{bmatrix}
\end{displaymath}
\item On déduit de la matrice précédente et du calcul de la trace de $c_q$ (V.2) que :
\begin{align*}
 \tr c_q =& 2\cos \theta + 1 = \dfrac{3\alpha^2 - \Vert \overrightarrow V_q \Vert^2}{\alpha^2 + \Vert \overrightarrow V_q \Vert^2} \\
\cos \theta =& \dfrac{\alpha^2 - \Vert \overrightarrow V_q \Vert^2}{\alpha^2 + \Vert \overrightarrow V_q \Vert^2}
=  \dfrac{\alpha^2 - \Vert \overrightarrow V_q \Vert^2}{N(q)}
\end{align*}
Dans la base $\mathcal U$, la matrice de $\overrightarrow u \rightarrow \overrightarrow V_q \wedge \overrightarrow u$ se calcule avec l' expression usuelle du produit vectoriel :
\begin{displaymath}
 \begin{bmatrix}
  0 \\
  0 \\
 \Vert \overrightarrow V_q \Vert
 \end{bmatrix}
\wedge
 \begin{bmatrix}
  x \\
  y \\
 z
 \end{bmatrix}
= \begin{bmatrix}
   -\Vert \overrightarrow V_q \Vert y \\
  \Vert \overrightarrow V_q \Vert x \\
0
  \end{bmatrix}
\end{displaymath}
la matrice cherchée est donc :
\begin{displaymath}
 \begin{bmatrix}
  0 &   - \Vert \overrightarrow V_q \Vert & 0 \\
\Vert \overrightarrow V_q \Vert & 0 & 0 \\
0 & 0 & 0
 \end{bmatrix}
\end{displaymath}
En identifiant les expressions des matrices de $c_q - c_q^{-1}$ dans $\mathcal U$ obtenues à partir de a. et de V.4., on obtient
\begin{displaymath}
 \sin \theta = \dfrac{2\alpha \Vert \overrightarrow V_q \Vert}{N(q)}
\end{displaymath}
\item On utilise
\begin{displaymath}
 \tan \frac{\theta}{2} = \dfrac{\sin \theta}{1 + \cos \theta}
\end{displaymath}
On en déduit
\begin{displaymath}
  \tan \frac{\theta}{2} = \dfrac{2\alpha \Vert \overrightarrow V_q \Vert}{N(q)+\alpha^2-\Vert \overrightarrow V_q \Vert} 
= \dfrac{\Vert \overrightarrow V_q \Vert}{\alpha}
\end{displaymath}
Cette expression détermine un unique $\frac{\theta}{2}$ dans $]-\frac{\pi}{2},\frac{\pi}{2}[$ donc un unique $\theta$ dans $]-\pi, \pi[$.
\end{enumerate}
\end{enumerate}

\subsection*{Partie VI - Quaternions et angles d'Euler}
\begin{enumerate}
 \item Lorsque $q$ est de la forme 
\begin{displaymath}
 q = \begin{bmatrix}
e^{i\omega} & 0 \\
0 & e^{-i\omega}
     \end{bmatrix}
=\cos \omega 1_{\mathbb H} + \sin \omega \overrightarrow k
\end{displaymath}
$c_q$ est une rotation d'axe $\overrightarrow k$ (car $\sin \theta >$) et d'angle $\theta\in]-\pi,\pi[$ défini par :
\begin{displaymath}
 \tan \frac{\theta}{2} = \dfrac{\sin \omega}{\cos \omega} = \tan \omega
\end{displaymath}
Lorsque $\cos \omega \neq \frac{\pi}{2}$.
\begin{itemize}
 \item si $\omega \in ]0,\frac{\pi}{2}[$ : $c_q = r_{\overrightarrow k , 2\omega}$.
\item  si $\omega = \frac{\pi}{2}$ : $c_q$ est le demi-tour d'axe $\Vect \overrightarrow k$.
\item si $\omega \in ]\frac{\pi}{2}, \pi[$ : $c_q = r_{\overrightarrow k , 2\omega -2\pi}=r_{\overrightarrow k , 2\omega}$
\end{itemize}
Lorsque $q$ est de la forme
\begin{displaymath}
 q= \begin{bmatrix}
\cos \omega & i\sin \omega \\
i\sin \omega & \cos \omega
    \end{bmatrix}
=\cos \omega 1_{\mathbb H} + \sin \omega \overrightarrow i
\end{displaymath}
\begin{itemize}
\item  si $\omega = \frac{\pi}{2}$ : $c_q$ est le demi-tour d'axe $\Vect \overrightarrow i$.
\item si $\omega \in ]0,\pi[-\{\frac{\pi}{2}\}$, $\sin \omega >0$ : $c_q = r_{\overrightarrow i , 2\omega}$
\end{itemize}
\item Effectuons le calcul matriciel qui donne la matrice d'une rotation en fonction de ses angles d'Euler:
\begin{multline*}
 \begin{bmatrix}
  e^{i\frac{\varphi}{2}} & 0\\
0 & e^{-i\frac{\varphi}{2}}
 \end{bmatrix}
\begin{bmatrix}
 \cos\frac{\theta}{2} & i\sin\frac{\theta}{2}  \\
i\sin\frac{\theta}{2} & \cos\frac{\theta}{2}
\end{bmatrix}
 \begin{bmatrix}
  e^{i\frac{\psi}{2}} & 0\\
0 & e^{-i\frac{\psi}{2}}
 \end{bmatrix}
\\
=  \begin{bmatrix}
  e^{i\frac{\varphi}{2}} & 0\\
0 & e^{-i\frac{\varphi}{2}}
 \end{bmatrix}
\begin{bmatrix}
 \cos\frac{\theta}{2}e^{i\frac{\psi}{2}} & i\sin\frac{\theta}{2}e^{-i\frac{\psi}{2}}  \\
i\sin\frac{\theta}{2}e^{i\frac{\psi}{2}} & \cos\frac{\theta}{2}e^{-i\frac{\psi}{2}}
\end{bmatrix} 
= 
\begin{bmatrix}
 \cos\frac{\theta}{2}e^{i\frac{\varphi+\psi}{2}} & . \\
 \sin\frac{\theta}{2}e^{i\frac{\psi - \varphi}{2}} & . \\
\end{bmatrix}
\end{multline*}
\item Comme $q$ est un quaternion de norme 1 : $|a|^2 + |b|^2=1$, il existe donc un réel $\lambda\in ]0,\frac{\pi}{2}[$ qui permet d`'exprimer les modules sous forme trigonométrique.
\begin{displaymath}
 |a| = \cos \lambda , |b| = \sin \lambda
\end{displaymath}
Introduisons des arguments $\mu$ et $\nu$ dans $[0,2\pi]$ :
\begin{displaymath}
 a = \cos \lambda e^{i\mu}, |b| = \sin \lambda e^{i\nu}
\end{displaymath}
Il ne reste plus qu'à identifier les deux matrices :
\begin{align*}
 \dfrac{\theta}{2}= \lambda &,& \dfrac{\varphi + \psi}{2}=\mu &,& \dfrac{\psi -\varphi}{2} =\nu \\
\theta = 2\lambda &,& \psi=\mu+\nu &,& \varphi = \mu - \nu
\end{align*}

\end{enumerate}