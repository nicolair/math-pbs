%<dscrpt>Fractions rationnelles : décomposition en éléments simples.</dscrpt>
Soit $n$ un entier non nul. On se propose de d{\'e}composer en {\'e}l{\'e}ments simples dans $\C[X]$ puis dans $\R[X]$ la fraction rationnelle
\begin{displaymath}
F=\frac{1}{(X^n-1)^2}. 
\end{displaymath}
\begin{enumerate}

\item Pr{\'e}ciser les p{\^o}les de $F$. La partie polaire de $F$ relative au p{\^o}le $u$ est not{\'e}e
\begin{displaymath}
 \frac{\alpha(u)}{(X-u)^2}+\frac{\beta(u)}{X-u}
\end{displaymath}
Comment s'écrit la décomposition en éléments simples de $F$ dans $\C(X)$ ?
\item Soit  $k$ un entier entre $1$ et $n$. Former des développements limités à l'ordre $1$ en $1$ des fonctions suivantes
\begin{displaymath}
 x^k, \hspace{0.5cm} 1+x+x^2+\cdots+x^{n-1}, \hspace{0.5cm} \left( 1+x+x^2+\cdots+x^{n-1}\right)^{-2} .
\end{displaymath}

\item Pour un pôle $u$ autre que $1$, exprimer $\alpha(u)$ en fonction de $u$ et $\alpha(1)$, exprimer $\beta(u)$ en fonction de $u$ et $\beta(1)$.

\item
\begin{enumerate}
  \item Montrer que, au voisinage de $1$,
\begin{displaymath}
 \frac{1}{(1+x+x^2+\cdots+x^{n-1})^{2}} = \alpha(1) + \beta(1)(x-1) + o(x-1).
\end{displaymath}
En d{\'e}duire la partie polaire de $F$ relative au p{\^o}le 1.
  \item Former la décomposition en éléments simples de $F$ dans $\C(X)$.
\end{enumerate}

\item  On note $w=e^{\frac{2i\pi}{n}}$.
 \begin{enumerate}
  \item Soit $k\in\{1,2,\cdots,n-1\}$, pr{\'e}ciser $k'\in\{1,2,\cdots,n-1\}$ tel que
\begin{displaymath}
\overline{w^k}=w^{k'} 
\end{displaymath}
 \item En distinguant $n$ pair et $n$ impair, factoriser $X^n-1$ en polyn{\^o}mes irr{\'e}ductibles de $\R[X]$.
 \item En distinguant $n$ pair et $n$ impair, d{\'e}composer $F$ en {\'e}l{\'e}ments simples de $\R(X)$.
\end{enumerate}

\end{enumerate}
