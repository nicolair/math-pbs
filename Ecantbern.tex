%<dscrpt>Théorème de Cantor-Bernstein.</dscrpt>
Soit $E$ et $F$ deux ensembles, $f$ une fonction injective de $E$ dans $F$ et $g$ une fonction injective de $F$ dans $E$.\newline
Dans ce texte, pour toute partie $A$ de $E$, on désigne par $E\setminus A$ le complémentaire de $A$ dans $E$. De même, pour toute partie $X$ de $F$, on désigne par $F\setminus X$ le complémentaire de $X$ dans $F$.\newline
On définit une partie $\mathcal{H}$ de $\mathcal{P}(E)$ par:
\begin{displaymath}
  \forall A \in \mathcal{P}(E), \; 
  A\in \mathcal{H} \Leftrightarrow g(F\setminus f(A)) \subset E \setminus A .
\end{displaymath}

\begin{enumerate}
  \item Question de cours (image directe, image réciproque).\newline
  Soit $A\subset E$, $X\subset F$, $b\in E$, $y\in F$.\newline
  Caractériser la propriété $y\in f(A)$ à l'aide d'un quantificateur. Caractériser la propriété $b\in g(X)$ à l'aide d'un quantificateur. Caractériser $b\in f^{-1}(X)$. Caractériser $x\in g^{-1}(A)$.
  
  \item Dans cette question, on suppose qu'il existe $B\subset E$ telle que 
\begin{displaymath}
  E\setminus B = g(F\setminus f(B)).
\end{displaymath}
On définit des fonctions $f_1$ et $g_1$:
\begin{displaymath}
  f_1:
\left\lbrace 
\begin{aligned}
  B &\rightarrow f(B)\\ b &\mapsto f(b)
\end{aligned}
\right. 
\hspace{1cm}
  g_1:
\left\lbrace 
\begin{aligned}
  F \setminus f(B) &\rightarrow E\setminus B \\ x &\mapsto g(x)
\end{aligned}
\right. .
\end{displaymath}

\begin{enumerate}
  \item Montrer que, $f_1$ et $g_1$ sont bijectives.
  \item On définit des fonctions $\varphi$ et $\psi$ par:
\begin{displaymath}
\varphi :\,
\left\lbrace 
  \begin{aligned}
    E &\rightarrow F \\
    a &\mapsto
      \left\lbrace 
        \begin{aligned}
          f_1(a) &\text{ si } a\in B \\ g_1^{-1}(a) &\text{ si } a\notin B 
        \end{aligned}
      \right. 
  \end{aligned}
\right.
,\hspace{1cm}
\psi :\,
\left\lbrace 
  \begin{aligned}
    F &\rightarrow E \\
    x &\mapsto
      \left\lbrace 
        \begin{aligned}
          f_1^{-1}(x) &\text{ si } x\in f(B) \\ g_1(x) &\text{ si } x\notin f(B) 
        \end{aligned}
      \right. 
  \end{aligned}
\right.
\end{displaymath}
Montrer qu'elles sont bijectives.
\end{enumerate}

\item 
\begin{enumerate}
  \item  Soit $A\in \mathcal{P}(E)$, compléter la proposition suivante
\begin{displaymath}
A \in \mathcal{H} \Leftrightarrow \left( \forall x \in F, x \notin \;?\; \Rightarrow g(x) \; ?\; ?\right)  . 
\end{displaymath}
Puis démontrer cette équivalence\footnote{Toute rédaction prétendant démontrer en même temps les deux implications sera considérée incorrecte sans être lue!}.
  \item Montrer que, pour tout $A\in \mathcal{P}(E)$,
\begin{displaymath}
  A \in \mathcal{H} \Leftrightarrow g^{-1}(A) \subset f(A) .
\end{displaymath}
\end{enumerate}

\item 
\begin{enumerate}
  \item Montrer que $\emptyset \in \mathcal{H}$. Montrer que $E\setminus g(F) \in \mathcal{H}$.
  \item On note
\begin{displaymath}
  B = \bigcup_{A \in \mathcal{H}} A .
\end{displaymath}
Montrer 
\begin{displaymath}
  g^{-1}(B) = \bigcup_{A \in \mathcal{H}} g^{-1}(A),\hspace{1cm} f(B) = \bigcup_{A \in \mathcal{H}}f(A) .
\end{displaymath}
En déduire que $B\in \mathcal{H}$.
\end{enumerate}

\item 
\begin{enumerate}
  \item Pour tout $a\in E$, montrer que $a \notin B \Rightarrow B \cup \left\lbrace a\right\rbrace \notin \mathcal{H}$ .
  
  \item Montrer que $E \setminus B = g(F \setminus f(B))$ .
\end{enumerate}

\item Démontrer le théorème de Cantor-Bernstein
\begin{quote}
  Deux ensembles étant donnés, s'il existe des applications injectives entre chacun des deux alors il existe des applications bijectives entre chacun des deux.
\end{quote}

\end{enumerate}
