%<dscrpt>Premiers pas vers la théorie analytique des nombres.</dscrpt>
On rappelle qu'aucune notion de \og somme infinie\fg \phantom{n}ou de \og produit infini\fg \phantom{n}ne figure dans le programme de MPSI. Aucun raisonnement faisant intervenir de telles notions ne sera pris en compte.

Dans tout ce problème\footnote{d'après \emph{Introduction à la théorie analytique et probabiliste des nombres}, G. Tenenbaum}, $x$ et $\delta$ désignent des  nombres réels strictement supérieurs à $1$. On introduit diverses notations particulières.
\begin{itemize}
\item La partie entière de $x$ est notée $\lfloor x \rfloor$. La partie fractionnaire de $x$ est notée $\{x\}$. Par définition :
\begin{displaymath}
 x = \lfloor x \rfloor + \{x\}\hspace{1cm} \{x\}\in [0,1[ 
\end{displaymath}

\item L'ensemble des entiers naturels non nuls inférieurs ou égaux à $x$ est noté $\mathcal E(x)$.
 \item L'ensemble des nombres premiers inférieurs ou égaux à $x$ est noté $\mathcal P(x)$. Le nombre d'éléments de $\mathcal P(x)$ est noté $\pi(x)$.
\item L'ensemble des entiers dont la décomposition en facteurs premiers ne contient que des éléments de $\mathcal P(x)$ est noté $\mathcal N(x)$.
\item Pour tout entier naturel non nul $m$, l'ensemble des entiers dont la décomposition en facteurs premiers ne contient que des éléments de $\mathcal P(x)$ avec des exposants inférieurs ou égaux à $m$ est noté $\mathcal N_m(x)$.
\item Pour tout $\delta\geq1$ fixé, on définit:
\begin{displaymath}
 Z_x=\sum_{n\in \mathcal E(x)}\frac{1}{n^\delta}\hspace{1cm} \hspace{1cm}
 S_m(x) = \sum_{n\in \mathcal N_m(x)}\frac{1}{n^\delta}
\end{displaymath}
\end{itemize}
\subsection*{Partie I. Une formule d'Euler.}
Dans cette partie $\delta>1$ et $x\geq 2$.
\begin{enumerate}
 \item Parmi les ensembles $\mathcal E(x)$, $\mathcal N(x)$, $\mathcal N_m(x)$ lesquels sont finis ? Pour chacun de ceux là, préciser le cardinal avec les notations de l'énoncé.\newline
Montrer que, pour $x$ fixé, il existe un entier $m$ à préciser tel que $\mathcal E(x)\subset \mathcal N_m(x)$.  Montrer que, pour $x$ et $m$ fixés, il existe un $y>1$ à préciser tel que $\mathcal N_m(x)\subset \mathcal E(y)$ ? Que peut-on en déduire pour les sommes $S_m(x)$, $Z_x$, $Z_y$ ?

\item Montrer que
\begin{displaymath}
 \prod_{p\in \mathcal P(x)}\left( \sum_{k=0}^{m}\frac{1}{p^{k\delta}}\right)  = S_m(x)
\end{displaymath}

\item \'Etude de la suite $\left( Z_i \right)_{i\in\N^*}$.
\begin{enumerate}
 \item Pour tout entier naturel non nul $j$, montrer que
\begin{displaymath}
 \frac{\delta -1}{(j+1)^\delta}\leq \frac{1}{j^{\delta -1}} - \frac{1}{(j+1)^{\delta -1}}\leq 
\frac{\delta -1}{j^\delta}
\end{displaymath}
\item Montrer que
\begin{displaymath}
 \frac{1}{\delta -1} - \frac{1}{(\delta -1)(i+1)^{\delta -1}}\leq Z_i
\leq \frac{1}{\delta -1} +1 - \frac{1}{(\delta -1)i^{\delta -1}}
\end{displaymath}
\item Montrer que la suite $\left( Z_i \right)_{i\in\N^*}$ est convergente. On note $\zeta(\delta)$ sa limite\footnote{il s'agit de la très célèbre fonction zeta de Riemann}. Montrer que
\begin{displaymath}
 \frac{1}{\delta -1} \leq \zeta(\delta) \leq \frac{1}{\delta -1} +1
\end{displaymath}
\item Pour $x$ fixé, montrer que la suite $\left(S_m(x) \right)_{m\in\N^*}$ est convergente. On note $S(x)$ sa limite. Montrer que
\begin{displaymath}
 S(x)\leq \zeta(\delta)
\end{displaymath}
\end{enumerate}

 \item Soit $p$ un nombre premier, montrer que la suite
\begin{displaymath}
 \left( \sum_{k=0}^{m}\frac{1}{p^{k\delta}}\right)_{m\in \N^*}
\end{displaymath}
est convergente et préciser sa limite.
\item Montrer que :
\begin{displaymath}
 Z_x \leq \prod_{p\in \mathcal P(x)}\frac{1}{1-\frac{1}{p^\delta}}\leq \zeta(\delta)
\end{displaymath}
En déduire une formule d'Euler :
\begin{displaymath}
 \lim_{x\rightarrow \infty }\prod_{p\in \mathcal P(x)}\frac{1}{1-\frac{1}{p^\delta}}=\zeta(\delta)
\end{displaymath}
\end{enumerate}
\subsection*{Partie II. Constante d'Euler.}
Dans cette partie on prend $\delta=1$ et on note 
\begin{displaymath}
Z_x=\sum_{n\in \mathcal E(x)}\frac{1}{n} 
\end{displaymath}
\begin{enumerate}
 \item Pour tout entier naturel non nul $n$, on pose $u_n=Z_n-\ln n$.
\begin{enumerate}
 \item Montrer que pour tout entier naturel non nul $i$:
\begin{displaymath}
 \frac{1}{i+1}\leq \ln(i+1) - \ln(i)\leq \frac{1}{i}
\end{displaymath}
\item Déduire de la question a. que la suite $\left(u_n \right)_{n\in\N^*}$ est décroissante.
\item Déduire de la question a. que 
\begin{displaymath}
 \forall n\in \N^* : \frac{1}{n}\leq u_n
\end{displaymath}
\item Montrer que la suite  $\left(u_n \right)_{n\in\N^*}$ est convergente. On note $\gamma$ sa limite\footnote{constante d'Euler} et $v_n=u_n-\gamma$ pour tout naturel non nul $n$.
\end{enumerate}
\item
  Montrer que 
\begin{displaymath}
 \forall x\in \,[0,1[ \;: x+\ln(1-x)+\frac{x^2}{2(1-x)}\geq 0
\end{displaymath}
En déduire que
\begin{displaymath}
 \forall n\in \N^* \;: \frac{1}{n+1}+\ln\left(1-\frac{1}{n+1}\right)+\frac{1}{2n(n+1)}\geq 0 
\end{displaymath}
\item Pour tout naturel non nul $n$, on pose $w_n=v_n-\frac{1}{2n}$. Montrer que la suite $\left(w_n \right)_{n\in\N^*}$ est croissante puis que :
\begin{displaymath}
 \forall n\in \N^* : 0\leq Z_n - \ln n -\gamma \leq \frac{1}{2n}
\end{displaymath}
\item   Montrer que 
\begin{displaymath}
 \forall x\in \,[0,1[ \;: x+\ln(1-x)+\frac{x^2}{2(1+x)}\leq 0
\end{displaymath}
En déduire que
\begin{displaymath}
 \forall n\in \N^* \;: \frac{1}{n+1}+\ln\left(1-\frac{1}{n+1}\right)+\frac{1}{2(n+1)(n+2)}\leq 0 
\end{displaymath}
puis que la suite $\left(v_n -\frac{1}{2(n+1)}\right)_{n\in\N^*}$ est décroissante. Montrer l'équivalence des suites :
\begin{displaymath}
 Z_n - \ln n -\gamma \sim \frac{1}{2n}
\end{displaymath}

\end{enumerate}

\subsection*{Partie III. Valeur moyenne du nombre de diviseurs.}
\begin{figure}
 \centering
 \input{Ethean_1.pdf_t}
 \caption{Hyperbole de Dirichlet}
 \label{fig:Ethean_1}
\end{figure}

Dans cette partie, $\delta=1$. Pour tout entier naturel $n$ supérieur ou égal à $1$, on note $\tau(n)$ le nombre de diviseurs de $n$ et $D(x)$ la somme des nombres de diviseurs des entiers inférieurs ou égaux à $x$. On se propose de montrer que, en $+\infty$,
\begin{displaymath}
 \frac{D(x)}{x}=\frac{1}{x}\sum_{n\in \mathcal E(x)}\tau(n) = \ln x +2\gamma -1 +O(\frac{1}{\sqrt{x}})
\end{displaymath}
\begin{enumerate}
\item Question de cours.\newline
Soient $f$, $g$, $h$ des fonctions définies dans $]1,+\infty[$ et qui ne s'annulent pas. \'Enoncer la définition de :
\begin{displaymath}
 \text{en }+\infty:\hspace{1cm}f(x)=g(x)+O(h(x))
\end{displaymath}
 
\item Sur la figure \ref{fig:Ethean_1}, les points représentés  par les petits disques sont à coordonnées entières et on note $\mathcal{H}_x$ la courbe (hyperbole de Dirichlet). Comment s'interprètent $\tau(x)$ et $D(x)$ pour cette figure ? Préciser l'ordonnée marquée par un point d'interrogation.\newline
Montrer que :
\begin{displaymath}
 2\sum_{m\in \mathcal E(\sqrt{x})}\lfloor \frac{x}{m}\rfloor = D(x) +\lfloor\sqrt{x}\rfloor ^2
\end{displaymath}
\item Montrer que 
\begin{displaymath}
 xZ_{\sqrt{x}}-\sqrt{x} \leq \sum_{m\in \mathcal E(\sqrt{x})}\lfloor \frac{x}{m}\rfloor \leq x Z_{\sqrt{x}}
\end{displaymath}
\item Former un encadrement montrant le résultat annoncé.
\end{enumerate}

\subsection*{Partie IV. Une inégalité de Chebychev.}
On se propose dans cette partie de montrer l'inégalité de Chebychev :
\begin{displaymath}
 \forall n \in \N\setminus\{0,1\}, \; \theta (n) \leq n \ln 4\hspace{0.5cm}\text{ avec }\hspace{0.5cm} \theta (n) = \sum_{p\in \mathcal P(n)} \ln p
\end{displaymath}
\begin{enumerate}
\item Montrer ce résultat pour $n = 2$.

\item Montrer que si $n \geq 4$ est pair et si l'inégalité est vraie au rang $n-1$ alors
elle l'est au rang $n$.

\item On suppose maintenant que $n$ est impair et on l'écrit $n = 2m + 1$ avec $m \in \N$.
\begin{enumerate}
\item En considérant le développement de $ (1 + 1)^{2m + 1}$, montrer que 
\begin{displaymath}
\binom{2m + 1}{m} \leq 4^m 
\end{displaymath}

\item Soit $p$ un nombre premier vérifiant $m+1  < p \leq 2m + 1$.\newline
 Montrer que $p$ divise $\binom{2m + 1}{m}$.
\item En déduire que
\begin{displaymath}
 \theta (2m + 1) - \theta (m+1) \leq \ln \binom{2m + 1}{m}
\end{displaymath}
 \end{enumerate}

\item Montrer que l'inégalité est vraie pour tout naturel non nul $n\geq 2$.

\end{enumerate}


