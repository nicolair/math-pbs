\begin{enumerate}
  \item 
\begin{enumerate}
  \item Tout nombre rationnel $\frac{a}{b}$ avec $a\in\Z$ et $b\in \Z^*$ est algébrique car il est racine du polynôme du premier  degré $bX -a \in \Z[X]$. 
  \item Le nombre réel $\sqrt{2}$ est algébrique car racine de $X^2 - 2$ mais il n'est pas rationnel (cours).
\end{enumerate}

  \item
\begin{enumerate}
  \item La fonction associée au polynôme est de classe $\mathcal{C}^{\infty}$. Sa dérivée est continue donc bornée dans le segment $[x-1,x+1]$. Il existe donc des $M>0$ tels que $|P'(t)|\leq M$ pour tous les $t\in [x-1,x+1]$. On peut appliquer l'inégalité des accroissement finis dans cet intervalle entre la racine $x$ et un $y$ quelconque:
\begin{displaymath}
  \left|P(y)-P(x)\right| \leq M\left|y-x\right| \Rightarrow \left|P(y)\right| \leq M\left|y-x\right|
\end{displaymath}

  \item L'expression à minorer est le numérateur de la valeur du polynôme après réduction au même dénominateur
\begin{multline*}
0\neq  P(\frac{p}{q}) = a_0+a_1\frac{p}{q}+\cdots + a_d\frac{p^d}{q^d}
= \frac{a_0q^d + a_1p^1q^{d-1} + \cdots + a_dp^d}{q^d}\\
  = \frac{\sum_{k=0}^{d}a_kp^kq^{d-k}}{q^d}
  \Rightarrow \underset{\in \Z}{\underbrace{\sum_{k=0}^{d}a_kp^kq^{d-k}}} \neq 0 \Rightarrow \left|\sum_{k=0}^{d}a_kp^kq^{d-k}\right| \geq 1
\end{multline*}
car $p$, $q$ et les $a_i$ sont entiers.
  \item Dans cette question, on s'occupe de rationnels $\frac{p}{q}$ \emph{qui ne sont pas racines } de $P$.\newline
Considérons d'abord ceux qui sont proches de $x$ c'est à dire dans $[x-1, x+1]$ et exploitons les questions 2.a et b.
\begin{displaymath}
\left|x - \frac{p}{q}\right| \geq \frac{1}{M}\left|P(\frac{p}{q})\right| =\frac{\left|\sum_{k=0}^{d}a_kp^kq^{d-k}\right|}{M q^d} \geq \frac{1}{Mq^d}
\end{displaymath}
Pour les autres, comme $q$ et $d$ sont des naturels non nuls, $q^d\geq 1$ et
\begin{displaymath}
  |x-\frac{p}{q}|\geq 1 \Rightarrow |x-\frac{p}{q}|\geq \frac{1}{q^d}.
\end{displaymath}
Pour couvrir les deux cas, on choisit $K = \min(\frac{1}{M},1)$.
\end{enumerate}

  \item
\begin{enumerate}
  \item L'inégalité est évidente car, sous les conditions de l'énoncé, $\frac{u_k}{9}\leq 1$ et $10 ^{k - k!}\leq 1$. 
  \item Par définition la suite $\left( x_n\right)_{n\in \N}$ est croissante. Pour prouver sa convergence, il suffit de la majorer en utilisant la question précédente
\begin{displaymath}
x_n \leq 9\left( 1 + \frac{1}{10} + \cdots + \frac{1}{10^{n}}\right) =9\,\frac{1-10^{n+1}}{1-10^{-1}}\leq \frac{9}{1-10^{-1}}= 10 . 
\end{displaymath}
On note $x$ sa limite.
  \item On forme une inégalité analogue à la précédente mais au delà d'un entier $n$ fixé
\begin{multline*}
\forall p>n,\;
x_p - x_n \leq \frac{9}{10^{(n+1)!}} \left( 1 + \frac{1}{10^{(n+1)!-n!}} + \cdots + \frac{1}{10^{p!-n!}}\right)\\
\leq \frac{9}{10^{(n+1)!}} \left( 1 + \frac{1}{10^{(n+2)!-(n+1)!}} + \cdots + \frac{1}{10^{p!-(n+1)!}}\right)\\
\leq \frac{9}{10^{(n+1)!}} \left( 1 + \frac{1}{10} + \frac{1}{10^2} + \cdots + \frac{1}{10^{p!}}\right)
\leq \frac{9}{10^{(n+1)!}}\frac{10}{9}=\frac{1}{10^{(n+1)!-1}}\leq \frac{1}{10^{n\,n!}}.
\end{multline*}
Pour justifier l'inégalité du début de la troisième ligne, il suffit de réaliser que la somme considérée est une somme de puissance de $\frac{1}{10}$ très \emph{lacunaire} c'est à dire qu'il manque beaucoup de termes (seuls figurent ces exposants avec des différences compliquées de factorielles). On majore simplement en ajoutant toutes les puissances qui manquent.
  \item On peut écrire $x_n$ sous la forme
\begin{displaymath}
  x_n = \frac{p_n}{q_n} \; \text{ avec }\; q_n = 10^{n!} \;\text{ et }\; p_n\in \N.
\end{displaymath}
S'il existait $P\in \Z[X]$ tel que $P(x)=0$, il existerait aussi un réel $K$ fixé tel que 
\begin{displaymath}
  \left|x-\frac{p}{q}\right| \geq \frac{K}{q^d}
\end{displaymath}
pour tous les rationnels $\frac{p}{q}$ qui \emph{ne sont pas racines} de $P$.
La suite $\left( x_n\right)_{n\in \N}$ étant strictement croissante, elle prend une infinité de valeurs différentes. Comme $P$ admet au plus $d$ racines, il existe un rang $N$ tel que $x_n$ n'est pas racine de $P$ dès que $n\geq N$. On devrait alors avoir
\begin{displaymath}
\forall n\geq N,\;  \frac{1}{10^{n\,n!}}\leq \left|x-x_n\right|\geq \frac{K}{q_n^d}= \frac{K}{10^{d\,n!}}
\Rightarrow 10^{(n-d)n!} \leq \frac{1}{K}
\end{displaymath}
ce qui est absurde car la suite à gauche diverge vers $+\infty$.\newline
Il ne peut exister de polynôme à coefficients entiers dont $x$ soit racine. Tous les nombres obtenus comme limites de ces suites sont transcendants.

\end{enumerate}

\end{enumerate}
