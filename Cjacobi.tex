D'après X-ens 2013 PC
\subsection*{Partie I. Questions préliminaires.}
\begin{enumerate}
 \item Notons $f$ la fonction considérée, en exprimant les valeurs absolues dans les différents cas, on obtient
\begin{center}
\renewcommand{\arraystretch}{1.4}
\begin{tabular}{|l|l|l|l|l|l|}
\hline
$x$    & $]-\infty , a ]$ & $[a,b]$  & $[b,c]$  & $[c,d]$  & $[d , +\infty[$ \\ \hline
$f(x)$ & 0                & $2(x-a)$ & $2(d-c)$ & $2(d-x)$ & 0 \\ \hline
\end{tabular}
\end{center}
L'hypothèse $a+d = b+c$ intervient pour $f(x)=0$ dans les deux intervalles infinis. On en déduit que $f$ est à valeurs positives.

 \item L'équivalence entre les trois propriétés est un résultat de cours. La base obtenue à partir de $\mathcal{E}$ en permutant les vecteurs de manière à ce que les deux premiers soient $e_p$ et $e_q$ est orthogonale. Dans cette base, la matrice de $r_{p,q,\theta}$ est diagonale par blocs orthogonaux :
\begin{displaymath}
 \begin{pmatrix}
  \cos \theta & -\sin \theta &  \\
  \sin \theta & \cos \theta &  \\
   & & I_{n-2}
 \end{pmatrix}
\end{displaymath}
Le bloc en haut à gauche est la matrice d'une rotation du plan. Avec le produit matriciel par blocs, on vérifie facilement que cette matrice est orthogonale donc que $r_{p,q,\theta}$ conserve le produit scalaire.
\end{enumerate}


\subsection*{Partie II. Conjugaison par une matrice de rotation.}
\begin{enumerate}
 \item La matrice $S'$ est symétrique d'après les propriétés usuelles de la transposition.\newline
En regardant $R_{p,q}(\theta)$ comme la matrice de passage (orthogonale) de $\mathcal{E}$ dans $\mathcal{E}'$, la définition de $S'$ apparait comme une formule de changement de base. La matrice $S'$ est donc la matrice de $f$ dans la base orthonormée $\mathcal{E}'$. On en déduit
\begin{displaymath}
 s_{i,j}' = \text{ coordonnée selon $e'_i$ de $f(e'_j)$ } = <e_i' / f(e'_j)>.
\end{displaymath}


 \item Calcul de $S'$.\label{formules}
\begin{enumerate}
 \item D'après 1. et l'orthogonalité de la famille:
\begin{multline*}
 s'_{p,p} = <e_p' / f(e'_p)>
 = <\cos \theta e_p + \sin \theta e_q / \cos \theta f(e_p) + \sin \theta f(e_q)>\\
 = <\cos \theta e_p + \sin \theta e_q / \cos \theta \left( s_{p,p}e_p+s_{q,p}e_q\right)  + \sin \theta \left( s_{p,q}e_p+s_{q,q}e_q\right) >\\
 = \cos^2\theta s_{p,p} + \cos\theta \sin\theta s_{p,q} 
  +\sin \theta \cos \theta s_{q,p} + \sin^2\theta s_{q,q} \\
 = \cos^2\theta s_{p,p} + 2 \cos\theta \sin\theta s_{p,q} + \sin^2\theta s_{q,q}.
\end{multline*}
Par un calcul analogue avec $e'_q = -\sin \theta e_p + \cos \theta e_q$, on trouve
\begin{displaymath}
 s'_{q,q} = \sin^2\theta s_{p,p} - 2 \cos\theta \sin\theta s_{p,q} + \cos^2\theta s_{q,q}. 
\end{displaymath}
En sommant les deux expressions, on obtient
\begin{displaymath}
 s'_{p,p} + s'_{q,q} = s_{p,p} + s_{q,q}
\end{displaymath}

 \item Le principe du calcul est le même qu'au dessus. On trouve
\begin{displaymath}
 s'_{p,q} = -\cos \theta \sin \theta s_{p,p} + \left( \cos^2 \theta - \sin^2 \theta\right)s_{p,q} + \cos\theta\sin \theta s_{q,q}. 
\end{displaymath}

 \item De même pour $k\notin \{ p,q \}$, on trouve :
\begin{equation} \label{Sprimepqk}
 s'_{p,k} = \cos \theta s_{p,k} + \sin \theta s_{q,k}, \hspace{0.5cm}
 s'_{q,k} = -\sin \theta s_{p,k} + \cos \theta s_{q,k}
\end{equation}

\end{enumerate}

 \item 
\begin{enumerate}
 \item Comme $-\frac{\pi}{2} < \cos \theta < \frac{\pi}{2}$, on peut diviser par $\cos^2\theta \neq 0$. D'après 2.a.,
\begin{multline*}
 s'_{p,q} = 0 \Leftrightarrow 
-\tan \theta \,s_{p,p} + (1-\tan^2 \theta)s_{p,q} + \tan \theta\,s_{q,q} = 0 \\
\Leftrightarrow s_{p,q} \tan^2 \theta +(s_{p,p} - s_{q,q})\tan \theta -s_{p,q} = 0
\end{multline*}
Comme $s_{p,q}\neq 0$, on a bien prouvé que $s'_{p,q}=0$ si et seulement si $\tan \theta$ est solution de l'équation en $t$:
\begin{equation}
 t^2 + \frac{s_{p,p}-s_{q,q}}{s_{p,q}}\,t -1 = 0 \label{eqtan}
\end{equation} 

 \item  L'équation \ref{eqtan} admet deux racines réelles car son discriminant est strictement positif. Le produit de ces deux racines est $-1$ donc le produit des deux modules est $1$. Ceci montre que exactement une des deux est dans $]-1, 1[$. On la note $t_0$ et on pose $\theta_0 = \arctan t_0$.
\end{enumerate}

 \item La première relation est une simple reformulation de l'équation 
\begin{displaymath}
 \tan^2\theta_0 + \frac{s_{p,p}-s_{q,q}}{s_{p,q}}\,\tan \theta_0 -1 = 0
\Leftrightarrow
s_{p,p} - s_{q,q} = s_{p,q} \frac{1-\tan^2\theta_0}{\tan \theta_0}
\end{displaymath}
 On reprend les formules de la question \ref{formules} pour les exprimer avec $t_0 = \tan \theta_0$.
\begin{multline*}
s'_{p,p} - s_{p,p} = (\cos^2\theta_0 -1)s_{p,p} + 2\cos\theta_0\sin\theta_0 s_{p,q} + \sin^2\theta_0 s_{q,q} \\
= -\sin^2\theta_0\left( s_{p,p}-s_{q,q}\right) + 2\cos\theta_0\sin\theta_0 s_{p,q} \\
= \cos^2 \theta_0\left( -t_0(1-t_0^2) + 2t_0\right) s_{p,q} 
= \frac{t_0(t_0^2 + 1)}{1+t_0^2}s_{p,q} = t_0 s_{p,q}.
\end{multline*}
De plus
\begin{displaymath}
s'_{p,p} + s'_{q,q} = s_{p,p} + s_{q,q} 
\Rightarrow 
s'_{q,q} - s_{q,q} = s_{p,p} - s'_{p,p} = -t_0 s_{p,q}. 
\end{displaymath}
 
 \item 
\begin{enumerate}
 \item Lorsque ni $i$ ni $j$ ne sont dans $\{p,q\}$, $e'_i = e_i$ et $f(e'_j) = f(e_j)$ donc $s'_{i,j} = s_{i,j}$. On en déduit,
\begin{displaymath}
\left\| E' \right\|^2 =  \left\| E \right\|^2 + \sum_{\empil{k = 1}{k\neq p}}^{n}({s'_{p,k}}^{2}-{s_{p,k}}^{2})
 + \sum_{\empil{k = 1}{k\neq q}}^{n}({s'_{q,k}}^{2}-{s_{q,k}}^{2})
\end{displaymath}
En utilisant les relations \ref{Sprimepqk}, on obtient pour $k\notin\{ p,q\}$:
\begin{displaymath}
 {s'_{p,k}}^{2} + {s'_{q,k}}^{2} = {s_{p,k}}^{2} + {s_{q,k}}^{2}
\end{displaymath}
Il ne reste donc plus que 
\begin{displaymath}
\left\| E' \right\|^2 =  \left\| E \right\|^2 
+ ({s'_{p,q}}^{2}-{s_{p,q}}^{2}) +({s'_{q,p}}^{2}-{s_{q,p}}^{2})  
= \left\| E \right\|^2 -2{s_{q,p}}^{2}
\end{displaymath}
car $s'_{p,q}=0$.
 \item La relation $\left\|S'\right\| = \left\|S \right\|$ vient de l'expression de la norme avec la trace et de la possibilité de permuter les matrices dans la trace d'un produit. Les autres formules sont évidentes à partir de l'expression comme somme des carrés des coefficients.\newline
On en déduit
\begin{displaymath}
\left\|D'\right\|^2 = \left\|S'\right\|^2 - \left\|E'\right\|^2 = \left\|S\right\|^2 - \left\|E\right\|^2 + 2 s^{2}_{q,p}
= \left\|D \right\|^2 + 2s^2_{q,p}.
\end{displaymath}

\end{enumerate}

 \item Les expressions des coefficients de $S'$ sont donnés par la question 2. Ils font intervenir des $\sin \theta$ et $\cos \theta$que l'on peut exprimer avec $t_0$
\begin{displaymath}
 \cos \theta_0 = \frac{1}{\sqrt{1+t_0^2}}, \sin \theta_0 = \frac{t_0}{\sqrt{1+t_0^2}}
\end{displaymath}
car $\theta_0 \in \left] -\frac{\pi}{2}, \frac{\pi}{2}\right[$ donc $\cos \theta_0 >0$. En fait on a même $\theta_0 \in \left] -\frac{\pi}{4}, \frac{\pi}{4}\right[$ à cause du choix de la racine dans $]-1,1[$.

 \item On suppose dans cette question que $s_{p,q}$ est le coefficient de $E$ de plus grande valeur absolue parmi les $s_{i,j}$ avec $i\neq j$.
\begin{enumerate}
 \item \`A cause du choix du couple $p_m,q_m$, on peut majorer les $\frac{n(n-1)}{2}$ coefficients intervenant dans la somme :
\begin{displaymath}
 \left\|E\right\|^2 \leq \frac{n(n-1)}{2} s_{p,q}^2
 \Rightarrow 
 s_{p,q}^2 \geq \frac{2}{n(n-1)}\left\|E\right\|^2
 . 
\end{displaymath}

 \item On peut en déduire une majoration de la norme de $E'$:
\begin{displaymath}
\left\|E'\right\|^2 
= \left\|E\right\|^2- 2s_{p,q} 
\leq \left( 1-\frac{2}{n(n-1)}\right)\left\|E\right\|^2 
\end{displaymath}
Posons 
\begin{displaymath}
 \rho = \sqrt{1-\frac{2}{n(n-1)}}.
\end{displaymath}
Rappelons que $n\geq2$ est la dimension de l'espace. C'est un nombre fixé indépendant de $S$ pour lequel l'expression dans la racine est dans $]0,1[$ tel que 
\begin{displaymath}
\left\|E'\right\| \leq \rho \left\|E\right\|. 
\end{displaymath}

 \item Sur la diagonale, seuls les coefficients $p,p$ et $q,q$ interviennent:
\begin{displaymath}
 \left\|D' - D\right\|^2 = (s'_{p,p} - s_{p,p})^2 + (s'_{q,q} - s_{q,q})^2
 = 2 t_0^2 s^2_{p,q}
\end{displaymath}
d'après la question 4.
\end{enumerate}
Comme $|t_0| \leq 1$,
\begin{displaymath}
 \left\|D' - D\right\|^2 \leq 2 s^2_{p,q} \leq \left\|E\right\|^2
\end{displaymath}
car $s_{p,q}$ figure deux fois parmi les coefficients de $E$.

 \item
\begin{enumerate}
 \item D'après II.2. puis II.4.
\begin{multline*}
\left. 
\begin{aligned}
 s'_{p,p} =& \cos^2\theta s_{p,p} + 2 \cos\theta \sin\theta s_{p,q} + \sin^2\theta s_{q,q} \\
 s'_{q,q} =& \sin^2\theta s_{p,p} - 2 \cos\theta \sin\theta s_{p,q} + \cos^2\theta s_{q,q}
\end{aligned}
\right\rbrace \\
\Rightarrow
s'_{q,q} - s'_{p,p}
= \left( \cos^2 \theta_0 - \sin^2 \theta_0\right)\left( s_{q,q} - s_{p,p}\right) -4 \sin\theta_0 \cos\theta_0 s_{p,q} \\
= -\left( \frac{1-t_0^2}{1+t_0^2}\,\frac{1-t_0^2}{t_0} + 4 \frac{t_0}{1+t_0^2}\right)s_{p,q} 
= - \frac{1+t_0^2}{t_0} s_{p,q}. 
\end{multline*}

 \item  D'après la question précédente et II.4.,
\begin{multline*}
 (s'_{q,q}-s'_{p,p})^2 - (s_{q,q}-s_{p,p})^2 
= \left( \frac{(1+t_0^2)^2}{t_0^2} - \frac{(1-t_0^2)^2}{t_0^2}\right)s_{p,q}^2 
= 4 s_{p,q}^2 \geq 0\\
\Rightarrow 
\left| s'_{q,q}-s'_{p,p} \right| \geq \left| s_{q,q}-s_{p,p} \right|.
\end{multline*}
\end{enumerate}

 \item
\begin{enumerate}
  \item On sait déjà d'après 2.a que $s_{p,p} - s'_{p,p} = s'_{q,q} - s_{q,q}$. Il suffit donc de montrer que l'un des deux est du signe de $s_{q,q} - s_{p,p}$.\newline
Cela résulte de 4. et du choix de $t_0$
\begin{displaymath}
 s_{p,p} - s_{q,q} = \frac{1-t_0^2}{t_0}s_{p,q},\hspace{0.5cm}
 s'_{p,p} - s_{p,p} = t_0 s_{p,q},\hspace{0.5cm}
 1-t_0^2 > 0
\end{displaymath}
 
 \item Supposons par exemple que les deux expressions de la question précédentes soient positives. Alors
\begin{displaymath}
 s'_{p,p} \leq s_{p,p} \leq s_{q,q} \leq s'_{q,q}.
\end{displaymath}
On peut appliquer le résultat de la question préliminaire et conclure en prenant la valeur en $s_{i,i}$
\begin{displaymath}
 |s_{i,i} - s'_{q,q}| + |s_{i,i} - s'_{p,p}| - |s_{i,i} - s_{p,p}| - |s_{i,i} - s_{q,q}|\geq 0
\end{displaymath}
L'autre cas se traite de manière analogue.
\end{enumerate}

 \item On a défini
\begin{displaymath}
R = \sum_{(i,j)\in \llbracket 1,n \rrbracket^2} \left|s_{i,i} - s_{j,j}\right|, \hspace{0.5cm}
R' = \sum_{(i,j)\in \llbracket 1,n \rrbracket^2} \left|s'_{i,i} - s'_{j,j}\right|
\end{displaymath}
Pour $i\notin\{p,q\}$, on a $s'_{i,i} = s_{i,i}$. Dans la différence, il ne reste donc que les autres termes
\begin{displaymath}
 R' - R =
\sum_{i=1}^{n}\left( |s'_{i,i} - s'_{p,p}| - |s_{i,i} - s_{p,p}| + |s'_{i,i} - s'_{q,q}| - |s_{i,i} - s_{q,q}|\right) 
\end{displaymath}
D'après la question précédente, tous les termes de cette somme sont positifs. On ne garde que ceux avec $i$ égal à $p$ ou $q$:
\begin{displaymath}
R' - R \geq   2\left( |s'_{p,p} - s'_{q,q}| - |s_{p,p} - s_{q,q}|\right) 
\end{displaymath}
Or
\begin{displaymath}
s'_{q,q} - s'_{p,p} = (s'_{q,q} - s_{q,q}) + (s_{q,q} - s_{p,p}) + (s_{p,p} - s'_{p,p}) 
\end{displaymath}
avec les trois parenthèses de même signe d'après 9.a. On en déduit
\begin{multline*}
|s'_{q,q} - s'_{p,p}| = |s'_{q,q} - s_{q,q}| + |s_{q,q} - s_{p,p}| + |s_{p,p} - s'_{p,p}| \\
\Rightarrow
|s'_{q,q} - s'_{p,p}| - |s_{q,q} - s_{p,p}| = |s'_{q,q} - s_{q,q}| + |s_{p,p} - s'_{p,p}| \\
\Rightarrow
R' - R \geq 2\left( |s'_{q,q} - s_{q,q}| + |s_{p,p} - s'_{p,p}|\right)
= 2 \sum_{i=1}^n |s'_{i,i} - s_{i,i}|
\end{multline*}
car les autres termes sont nuls.
\end{enumerate}


\subsection*{Partie III. Algorithme.}
\begin{enumerate}
 \item 
\begin{enumerate}
 \item On majore grossièrement (pour les termes non nuls) la valeur absolue de la différence par la somme des valeurs absolues.
\begin{multline*}
R_m \leq \sum_{\empil{(i,j)\in \llbracket 1,n \rrbracket^2}{i\neq j}} \left| \sigma_{j,j}^{(m)}\right|
 + \sum_{\empil{(i,j)\in \llbracket 1,n \rrbracket^2}{i\neq j}} \left|\sigma_{i,i}^{(m)}\right|\\
\leq \sum_{j=1}^n(n-1) \left| \sigma_{j,j}^{(m)}\right| + \sum_{i=1}^n(n-1) \left| \sigma_{i,i}^{(m)}\right|
= 2 (n-1)\sum_{j=1}^n \left| \sigma_{j,j}^{(m)}\right|
\end{multline*}
La deuxième inégalité est l'inégalité de Cauchy-Schwarz en écrivant
\begin{displaymath}
 \left( \sum_{j=1}^n \left|\sigma_{j,j}^{(m)}\right|\right)^2
 = \left( \sum_{j=1}^n 1 \times \left| \sigma_{j,j}^{(m)}\right|\right)^2.
\end{displaymath}

 \item On a vu en II.5.b. que la norme est conservée:
\begin{displaymath}
\forall m\in \N,\; \left\| \Sigma^{(m)} \right\| = \left\| \Sigma \right\|. 
\end{displaymath}
D'après la question précédente
\begin{displaymath}
R_m \leq 2 (n-1) \sqrt{n} \, \sqrt{\sum_{j=1}^n (\sigma_{j,j}^{(m)})^2 } 
\leq 2 (n-1) \sqrt{n} \left\| \Sigma \right\|
\end{displaymath}
en négligeant les coefficients qui ne sont pas sur la diagonale et en utilisant la conservation de la norme.
\end{enumerate}
 
 \item On peut appliquer la dernière question de la partie II. avec $S = \Sigma^{(m)}$, $S' = \Sigma^{(m+1)}$.
\begin{displaymath}
R_{m+1} - R_m = R' - R \geq 2 \sum_{i=1}^n |s'_{i,i} - s_{i,i}| = 2\epsilon_m. 
\end{displaymath}
La série des $\epsilon_m$ étant à termes positifs, il suffit de montrer que les sommes partielles sont majorées. Or l'inégalité précédente entraîne
\begin{displaymath}
 \sum_{k=0}^{m} \epsilon_k \leq \frac{1}{2}\left(R_{m+1} -R_0\right) 
 \leq \frac{R_{m+1}}{2} \leq (n-1)\sqrt{n} \left\|\Sigma \right\|.
\end{displaymath}
La série est donc convergente (rappelons que $n$ est la dimension de l'espace).
 
 \item
\begin{enumerate}
 \item On peut regarder chaque coefficient de la suite de matrices $D^{(m)}$ comme une somme partielle de la série 
\begin{displaymath}
 \left( \sum \sigma_{i,i}^{(m+1)} - \sigma_{i,i}^{(m)}\right). 
\end{displaymath}
D'après la question précédente, cette série est absolument convergente donc convergente. On en déduit la convergence de chaque suite de coefficients donc de la partie diagonale.

 \item D'après la question 7.b. $S = \Sigma^{(m)}$, $S' = \Sigma^{(m+1)}$,
\begin{displaymath}
 \left\| E^{(m+1)} \right\| \leq \rho \left\| E^{(m)} \right\|.
\end{displaymath}
La suite des $\left\| E^{(m)} \right\|$ est dominée par une suite géométrique de raisons $\rho \in ]0,1[$. Elle converge donc vers $0$. On en déduit que la suite des matrices $E^(m)$ converge vers la matrice nulle. Toutes les suites de coefficients hors de la diagonale convergent vers $0$. 
\end{enumerate}

\end{enumerate}
