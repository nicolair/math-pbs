\subsubsection*{Partie I.}
\begin{enumerate}
 \item On se conforme aux indications de l'énoncé en introduisant une fonction $u$, puis en exprimant $z$ puis ses dérivées en fonction des dérivées de $u$. On multiplie à chaque étape par une exponentielle. On obtient
\begin{align*}
 u(t) =& z(e^t)e^{-\frac{t}{2}} & &\\
z(e^t) =& u(t)e^{\frac{t}{2}} & \times& \frac{A}{4e^{2t}}=a(e^t) \\
e^t z'(e^t) =& u'(t)e^{\frac{t}{2}}+\frac{1}{2}u(t)e^{\frac{t}{2}} & & \\
z'(e^t) =& u'(t)e^{-\frac{t}{2}}+\frac{1}{2}u(t)e^{-\frac{t}{2}} & & \\
e^t z''(e^t) =& u''(t)e^{-\frac{t}{2}}-\frac{1}{4}u(t)e^{-\frac{t}{2}} & & \\
z''(e^t) =& u''(t)e^{-\frac{3t}{2}}-\frac{1}{4}u(t)e^{-\frac{3t}{2}} & \times& 1 \\
\end{align*}
\'Ecrivons ensuite la relation différentielle vérifiée par $z$ en $e^t$ qui est bien dans $I$. La colonne de droite dans le calcul précédent indique les coefficients par lesquels on multiplie. On obtient
\begin{displaymath}
 0=u''(t)e^{-\frac{3t}{2}} -\frac{1}{4}u(t)e^{-\frac{t}{2}} +\frac{A}{4}u(t)e^{\frac{t}{2}}
\Leftrightarrow u''(t)+\frac{A-1}{4}u(t)=0
\end{displaymath}
L'équation cherchée est donc
\begin{align*}
 y''+\frac{A-1}{4}y=0 & & (2)
\end{align*}
L'intérêt de ce calcul est d'obtenir une équation différentielle à coefficients constants à partir d'une équation dont les coefficients n'étaient pas constants.
\item Pour toutes les valeurs de $A$, l'ensemble des solutions de $(2)$ est de la forme
\begin{displaymath}
 \left\lbrace \lambda_1u_1 + \lambda_2 u_2, (\lambda_1,\lambda_2)\in \R^2\right\rbrace 
\end{displaymath}
L'expression des fonctions de la base $(u_1,u_2)$ dépend des valeurs de $A$.
\begin{itemize}
 \item[Cas $A<1$]
\begin{align*}
 u_1(t)=e^{-kt} & & u_2(t)=e^{kt} & &\text{ avec } k=\frac{1}{2}\sqrt{1-A} 
\end{align*}
On pourait aussi choisir
\begin{align*}
 u_1(t)=\ch(kt) & & u_2(t)=\sh(kt) & &\text{ avec } k=\frac{1}{2}\sqrt{1-A} 
\end{align*}
\item[Cas $A=1$.]
\begin{align*}
 u_1(t)=1 & & u_2(t)=t 
\end{align*}
\item[Cas $A>1$.]
\begin{align*}
 u_1(t)=\cos(\omega t) & & u_2(t)=\sin(\omega t) & &\text{ avec } \omega=\frac{1}{2}\sqrt{A-1} 
\end{align*}
\end{itemize}

\item D'après la question 1., si $z$ est une solution de $(1)$ alors $u$ est solution de $(2)$ avec
\begin{displaymath}
 \forall t \in \R : u(t)= z(e^t)e^{-\frac{t}{2}}
\end{displaymath}
Mais on connait la forme des solutions de $(2)$ et on peut exprimer $z$ en fonction de $u$
\begin{displaymath}
 \forall x >0 : z(x)=u(\ln x)\sqrt{x}
\end{displaymath}
 On peut donc former des fonctions $y_1$, $y_2$ à partir de $u_1$ et $u_2$. Il convient de vérifier que ces fonctions sont effectivement solutions de $(1)$. Ce calcul qui ne présente pas de problème n'est pas présenté ici. On en déduit que l'ensemble des solutions de $(1)$ est
\begin{displaymath}
 \left\lbrace \lambda_1y_1 + \lambda_2 y_2, (\lambda_1,\lambda_2)\in \R^2\right\rbrace 
\end{displaymath}
L'expression des fonctions $y_1$ et $y_2$ dépend des valeurs de $A$.
\begin{itemize}
 \item[Cas $A<1$.]
\begin{align*}
 y_1(x)=x^{\frac{1}{2}-k} & & y_2(x)=x^{\frac{1}{2}+k} & &\text{ avec } k=\frac{1}{2}\sqrt{1-A} 
\end{align*}

\item[Cas $A=1$.]
\begin{align*}
 y_1(x)=\sqrt{x} & & y_2(x)=\sqrt{x}\ln x & &\text{ avec } k=\frac{1}{2}\sqrt{1-A} 
\end{align*}
\item[Cas $A>1$.]
\begin{align*}
 y_1(t)=\sqrt{x}\cos(\omega \ln x) & & y_2(t)=\sqrt{x}\sin(\omega \ln x) & &\text{ avec } \omega=\frac{1}{2}\sqrt{A-1} 
\end{align*}
\end{itemize}

\end{enumerate}

\subsubsection*{Partie II.}
\begin{enumerate}
 \item On procède comme dans la partie I.. On note $u$ la nouvelle fonction et on exprime $z$ en fonction de $u$ pour pouvoir utiliser l'équation de départ. Une exponentielle se met en facteur.
\begin{align*}
& & u(x)=& z(x)e^{\frac{1}{2}P(x)}\\
q(x) & \times& z(x) =& u(x)e^{-\frac{1}{2}P(x)}\\
p(x) & \times& z'(x) =& \left( u'(x)-\frac{1}{2}u(x)p(x) \right) e^{-\frac{1}{2}P(x)}\\
1 & \times& z''(x) =& \left( u''(x)-p(x)u'(x)-\frac{1}{2}u(x)p'(x) +\frac{1}{4}u(x)p^2(x)\right) e^{-\frac{1}{2}P(x)}
\end{align*}
Quand on combine avec les coefficients de la colonne de gauche, les termes en $u'$ diparaissent. On obtient
\begin{displaymath}
 0 = u''(x) +(-\frac{1}{2}p'-\frac{1}{4}p^2+q)u(x)
\end{displaymath}
La fonction $u$ est donc bien solution d'une équation
\begin{displaymath}
 y'' +vy=0 \text{ avec } v = -\frac{1}{2}p'-\frac{1}{4}p^2+q
\end{displaymath}
L'intérêt de cette question est de fournir une méthode permettant d'obtenir une nouvelle équation sans terme en $y'$.
\item Dans le cas particulier de cette question, les calculs conduisent à :
\begin{displaymath}
 v= 1 + \frac{A}{4x^2} \text{ avec }  A = 1-4\lambda^2
\end{displaymath}
On retombe donc sur une équation analogue à l'équation $(1)$ de la partie I.
\end{enumerate}
