%<dscrpt>Un calcul de déterminant.</dscrpt>
Soit $\beta\in \R^*$ et $n\in \N^*$, on définit la \emph{puissance factorielle montante} \footnote{d'après Mathématiques concrètes Graham, Knuth, Patashnik} par :
\begin{displaymath}
 \beta^{\overline{0}} = 1,\hspace{0.5cm}
\beta^{\overline{n}} = \beta (\beta +1)\cdots (\beta + n-1)
\hspace{0.5cm}\text{(produit de $n$ facteurs)}
\end{displaymath}
\begin{enumerate}
\item 
\begin{enumerate}
 \item Simplifier $\beta^{\overline{k+1+j}}-(\beta+k)\beta^{\overline{k+j}}$.
 \item Soit $j\geq i$, exprimer $\frac{\beta^{\overline{j}}}{\beta^{\overline{i}}}$ comme une puissance factorielle montante.
\end{enumerate}

 \item Soient $m\in \N$ et $p\in \N^*$, on définit :
\begin{align*}
&\Delta(m,p,\beta) = \text{ la matrice $p\times p$ dont le terme $k,l$ est } \beta^{\,\overline{m+k+l-2}}\\
&P(m,p,\beta) = \text{ la matrice $p\times p$ dont le terme $k,l$ est } \beta^{m+k+l-2}\\
& \delta(m,p,\beta)=\det (\Delta(m,p,\beta))
\end{align*}

\begin{enumerate}
 \item Calculer $\delta(m,1,\beta)$ et $\delta(1,3,1)$.
 \item On note $L_1,\cdots, L_p$ les lignes de la matrice $\Delta(m,p,\beta)$.\newline
Pour $k$ entre $1$ et $p-1$, quel est le réel $\lambda$ pour lequel le premier terme de la ligne $L_{k+1}-\lambda L_k$ est nul? Préciser le reste de cette ligne.

 \item Former une relation entre $\delta(m,p,\beta)$ et $\delta(m+1,p-1,\beta)$. En déduire une expression de $\delta(m,p,\beta)$ avec des factorielles et des puissances factorielles montantes.
 \item Calculer $\det(P(m,p,\beta))$.
\end{enumerate}

\item  Soient $p\in \N^*$ et $\beta_1,\cdots \beta_p$ des réels deux à deux distincts, on définit :
\begin{align*}
&V(\beta_1,\cdots,\beta_p) = \text{ la matrice $p\times p$ dont le terme $k,l$ est } \beta_l^{\,\overline{k-1}}\\
& v(\beta_1,\cdots,\beta_p)=\det (V(\beta_1,\cdots,\beta_p))
\end{align*}
Calculer $v(\beta_1,\cdots,\beta_p)$.
\end{enumerate}
