
 \subsection*{I. Calculs préliminaires}
\begin{enumerate}
  \item Les solutions de l'équation homogène sont les fonctions 
\[
\fonc{y}{I}{\R}{x}{\lambda \sqrt x}\quad\text{avec}\ \lambda\in\C.  
\]
On cherche une solution  particulière de l'équation complète de la forme  
\[
\fonc{y}{I}{\R}{x}{\lambda(x) \sqrt x}\quad\text{avec $\lambda$ fonction dérivable de $I$ dans $\C$}.
\]
Si $\lambda'(x)=\frac \mu x$ pour tout $x$ dans $I$ alors $y$ est solution. D'où
$$\fonc{y}{I}{\R}{x}{\mu\sqrt x\ln x}$$ est solution.
Les solutions de cette équation différentielle sont les fonctions 
\[
\fonc{}{I}{\R}{x}{\lambda \sqrt x+\mu\sqrt x\ln x}\;\text{ avec }\ \lambda \in\C.
\]

  \item 
  \begin{enumerate}
  \item On vérifie que $\cos(t) = \frac{1-u^{2}}{1+u^{2}}$.
  \item Changement de variable $u = \tan \frac{t}{2}$ dans l'intégrale. On la note $I(\theta)$.
\begin{itemize}
  \item Les bornes : $t$ en $0\leftrightsquigarrow u \text{ en } 0$ , $t \text{ en }\theta \leftrightsquigarrow u \text{ en } \tan \frac{\theta}{2}$.
  \item L'élément différentiel : $du = \frac{1}{2}(1+u^2)\,dt \Rightarrow dt =\frac{2}{1+u^2}\,du$.
  \item Expression en fonction de $u$ : on utilise $\cos t = \frac{1-u^2}{1+u^2}$
\end{itemize}
On en tire
\begin{multline*}
I(\theta) = \int_0^{\tan\frac{\theta}{2}}\frac{2}{1+u^2+\cos\alpha(1-u^2)}\\
= \int_0^{\tan\frac{\theta}{2}}\frac{2}{1+\cos\alpha +u^2(1-\cos\alpha)}
= \frac{2}{1+\cos\alpha}\int_0^{\tan\frac{\theta}{2}}\frac{dt}{1+u^2\frac{1-\cos \alpha}{1+\cos \alpha}}
\end{multline*}
Or:
\begin{displaymath}
  1+\cos \alpha = 2 \cos^2 \frac{\alpha}{2},\hspace{0.5cm} 1-\cos\alpha = 2\sin^2\frac{\alpha}{2}
  ,\hspace{0.5cm } \frac{1-\cos \alpha}{1+\cos \alpha} = \tan^2\frac{\alpha}{2}
\end{displaymath}
l'expression de $I(\theta)$ est donc
\begin{displaymath}
\frac{1}{\cos^2 \frac{\alpha}{2}}\frac{\cos\frac{\alpha}{2}}{\sin \frac{\alpha}{2}}\arctan\left(\tan\frac{\alpha}{2}\tan\frac{\theta}{2} \right) 
  = \frac{2}{\sin \alpha}\arctan\left(\tan\frac{\alpha}{2}\tan\frac{\theta}{2} \right)
\end{displaymath}

  \item  D'après la question précédente:
\begin{displaymath}
\psi_{\alpha}(\theta) = \pi -2\arctan\left(\tan\frac{\alpha}{2}\tan\frac{\theta}{2} \right)
\end{displaymath}
Vérifions que $\psi_\alpha$ est une involution:
\begin{multline*}
\tan \frac{\psi_\alpha(\theta)}{2} = \frac{1}{\tan\frac{\alpha}{2}\tan\frac{\theta}{2}}
\Rightarrow
\tan \frac{\alpha}{2}\tan \frac{\psi_\alpha(\theta)}{2} = \frac{1}{\tan\frac{\theta}{2}}\\
\Rightarrow \arctan\left(\tan \frac{\alpha}{2}\tan \frac{\psi_\alpha(\theta)}{2} \right)
= \frac{\pi}{2} -\frac{\theta}{2}\\
\Rightarrow
\psi_\alpha \circ \psi_\alpha(\theta)
= \pi -2\arctan\left(\tan \frac{\alpha}{2}\tan \frac{\psi_\alpha(\theta)}{2} \right) = \theta
\end{multline*}
On a pu simplifier les $\arctan \circ \tan$ car tout se passe entre $0$ et $\frac{\pi}{2}$.
\end{enumerate}

\end{enumerate}

\subsection*{II \'Etude d'un cas particulier}
Dans cette partie, $a>0$, $I = ]0, +\infty[$ et $\varphi$ est définie sur $I$ par:
\[ 
\forall x \in I, \; \varphi(x) = \frac{a}{x}.
\]

\begin{enumerate}
\item Soit $x\in I$. Alors $\varphi(x)=x \, \Leftrightarrow \, \frac ax=x \, \Leftrightarrow \, x=\sqrt{a}$.
Il existe un unique point fixe $c=\sqrt a$.
\item La fonction $ \sqrt{} $ est dérivable dans $I$. Elle est solution de $\mathcal{F}_{\varphi}$ si et seulement si
\[
\forall x\in I\;  \frac{1}{2\sqrt{x}}=\sqrt{\frac ax} \Leftrightarrow a=\frac{1}{4}.  
\]

Soit $a=\frac 14$. La fonction  $\sqrt{}$ est deux fois dérivable dans $I$ et
\[
\forall x \in I, \; \sqrt{x}'' =-\frac 1{4x^{\frac 32}}=-\frac 1{4x^{2}}\times \frac 1{\sqrt x}=\varphi'(x)f(x).  
\]
Elle est donc bien solution de $\mathcal E_\varphi$.
\item 
\begin{enumerate}
\item On a 
\[
W' = f_1' f_2' + f_1f_2'' -f_1''f_2 - f_1'f_2'
    = f_1(\varphi'f_2)-(\varphi'f_1)f_2   = 0
\]
car $f_1$ et $f_2$ sont solutions de $\mathcal E_\varphi$. La fonction $W$ est donc constante.
\item Soit $f_1$ une solution de  $\mathcal E_\varphi$ qui ne s'annule pas. Si $f_1y'-f_1'y$ est constante de valeur $v$
alors $y'$ est dérivable car il s'exprime avec $v$, $f_1$, $f_1'$ et $y$.\newline
En dérivant l'égalité, il vient
\begin{multline*}
  f_1'y'+f_1y''-f_1'y'-f_1''y = 0
\Rightarrow y''-\frac{f_1''}{f_1}y = 0 \text{ (car $f_1$ ne s'annule pas)} \\
\Rightarrow y'' -\varphi'y = 0 \text{ (car $f_1$ est solution de $\mathcal E_\varphi$)}.
\end{multline*}

\item Nommons $f_1$ la fonction racine carrée sur $I$. C'est une solution de $\mathcal E_\varphi$ qui ne s'annule pas.\newline
Pour $\mu$ complexe, considérons l'équation différentielle en $y$ 
\[
(E_\mu) \hspace{0.5cm} f_1 y' - f_1' y = \mu \Leftrightarrow \forall x > 0, \; y'(x) - \frac{1}{2x}\,y = \frac{\mu}{x}.
\]
On retrouve l'équation différentielle de la question 1. de la partie I.\newline
D'après a., si $y$ est solution de $\mathcal{E}_\varphi$, il existe $\mu\in \C$ tel que $y$ soit solution de $E_\mu$.\newline
D'après b., si $y$ est solution d'une équation $E_\mu$ c'est une solution de $\mathcal{E}_\varphi$.\newline
L'ensemble des solutions de $\mathcal{E}_\varphi$ est donc l'union des ensembles de solutions des $E_\mu$ pour tous les $\mu$ complexes. \newline
D'après I.1. les solutions de $\mathcal E_\varphi$ sont les fonctions 
\[
\fonc{y}{I}{\R}{x}{\lambda \sqrt x+\mu\sqrt x\ln x}\; \text{ avec }\ (\lambda,\mu)\in\C^2.  
\]

\end{enumerate}
\item Par définition $z=f\circ \exp$, elle est donc deux fois dérivable sur $\R$. De plus 
\[
  \forall t\in I,\hspace{0.5cm} f'(t) = \frac 1t z'\circ \ln t, \qquad f''(t) = \frac 1{t^2}z''\circ \ln (t)-\frac 1{t^2}z'\circ \ln (t).
\]
Donc $f$ est solution de $\mathcal E_\varphi$ si et seulement si
\[
\forall t\in I,\hspace{0.5cm} \frac 1{t^2}z''\circ \ln (t)-\frac 1{t^2}z'\circ \ln (t)+\frac a{t^2}z\circ \ln (t)=0\ (*).
\]
Lorsque $t$ décrit $I$, $\ln(t)$ décrit $\R$, donc $(*)$ est équivalent à 
\[
\forall t\in\R,\hspace{0.5cm} z''(t)-z'(t) + az(t) = 0 .\]

\item Avec les règles de dérivation des fonctions composées et des fonctions usuelles, la dérivée de cette puissance généralisée reste $ux^{u-1}$. 

\item 
\begin{enumerate}
\item 
Les complexes $u_1$ et $u_2$ sont les racines du trinôme $x^2-x+a=0$. Comme $a\neq\frac 14$ son discriminant, $\Delta=1-4a$, est non nul et donc $u_1\neq u_2$.\\
Si $0<a<\frac 14$ alors $\Delta>0$, $u_1$ et $u_2$ sont donc réels.\\
Si $a>\frac 14$ alors $\Delta<0$, $u_1$ et $u_2$ sont donc complexes conjugués.

\item Les solutions de l'équation différentielle $z''  - z' + az = 0$ d'inconnue $z$ sont les fonctions
\begin{displaymath}
\fonc{z}{\R}{\C}{x}{\lambda e^{u_1 x}+\mu e^{u_2 x}}\text{ avec }(\lambda,\mu)\in\C^2  
\end{displaymath}
D'après la question précédente, les solutions de $\mathcal E_\varphi$ sont donc les fonctions 
$\fonc{f}{I}{\R}{x}{\lambda e^{u_1 \ln x}+\mu e^{u_2 \ln x}}$ avec $(\lambda,\mu)\in\C^2$.

\item Soit $y$ une solution $\mathcal E_\varphi$. Il existe $(\lambda,\mu)\in\C^2$ tel que
\[
 \fonc{y}{I}{\R}{x}{\lambda e^{u_1 \ln x}+\mu e^{u_2 \ln x}} 
\]
Cette fonction $y$ est solution du problème de Cauchy si et seulement si $\lambda$ et $\mu$ sont solutions du système 
\[
\left\{\begin{array}{lll}
\lambda c^{u_1}+\mu c^{u_2}&=&1\\
\lambda u_1c^{u_1-1}+\mu u_2c^{u_2-1}&=&1
\end{array}\right.
\]

C'est un système de Cramer (car $c^{u_1+u_2-1}(u_2-u_1)\neq 0$). On détermine $\lambda$ et $\mu$ grâce aux formules de Cramer. On obtient que l'unique fonction solution est $f_c$.

\item Rappelons les relations en jeu: $a = c^2$, $u_1+u_2 = 1$, $u_1u_2 = a$.\newline
D'après II.5. une fonction puissance se dérive selon la formule usuelle. On en déduit
\begin{multline*}
  f_c'(x)
= \frac{u_2-c}{u_2-u_1}\frac{u_1}{c}\left( \frac{x}{c}\right)^{u_1-1} 
 +\frac{c-u_1}{u_2-u_1}\frac{u_2}{c}\left( \frac{x}{c}\right)^{u_2-1} \\
= \frac{c-u_1}{u_2-u_1}\left( \frac{x}{c}\right)^{u_1-1} 
 +\frac{u_2 - c}{u_2-u_1}\left( \frac{x}{c}\right)^{u_2-1}
\end{multline*}
car
\begin{align*}
  (u_2-c)\frac{u_1}{c} = \frac{u_1u_2-cu_1}{c}=\frac{a-cu_1}{c}\frac{c^2-cu_1}{c}=c-u_1 \\
  (c-u_1)\frac{u_2}{c} = \frac{cu_2 - u_1u_2}{c}=\frac{cu_2-a}{c}\frac{ccu_2-c^2}{c}= u_2 - c
\end{align*}
D'autre part,
\begin{displaymath}
  \frac{\varphi(x)}{c} = \frac{a}{cx} = \frac{c}{x}
\Rightarrow
\left(\frac{\varphi(x)}{c} \right)^{u_1}  = \left(\frac{c}{x} \right)^{u_1} = \left(\frac{x}{c} \right)^{-u_1} 
= \left(\frac{x}{c} \right)^{u_2 -1}.
\end{displaymath}
La transformation est analogue pour la puissance $u_2$, on en tire
\begin{displaymath}
f_c\circ \varphi(x)
=\frac{u_2-c}{u_2-u_1}\left(\frac{x}{c} \right)^{u_2 -1}
+\frac{c-u_1}{u_2-u_1}\left(\frac{x}{c} \right)^{u_1 -1} = f_c'(x)
\end{displaymath}
\end{enumerate}
\end{enumerate}

\subsection*{III. Involutions conjuguées}
\begin{enumerate}
\item
\begin{enumerate}
\item Comme $\varphi$ est une involution,   
\[
\psi\circ\psi=h^{-1}\circ \varphi\circ( h\circ h^{-1})\circ \varphi\circ h=h^{-1}\circ (\varphi\circ \varphi)\circ h=h^{-1}\circ h=Id_I.  
\]

$\psi$ est donc une involution de $J$.
\item La fonction $g$ est dérivable et 
\begin{multline*}
g'= h'\times f'\circ h
= h'\times f\circ \varphi\circ h\quad \text{ car $f$ est solution de } \mathcal F_\varphi \\
= h'\times f\circ h\circ h^{-1}\circ \varphi\circ h
= h'\times  g\circ h^{-1}\circ \varphi\circ h
= h'\times  g\circ\psi .
\end{multline*}
\end{enumerate}

\item On veut montrer que l'involution $\psi_\alpha$ de $J=]0,\pi[$ est conjuguée d'une involution déjà rencontrée. Rappelons que :
\begin{displaymath}
\psi_{\alpha}(\theta) = \pi -2\arctan\left(\tan\frac{\alpha}{2}\tan\frac{\theta}{2} \right)
\end{displaymath}
Considérons une application $h$:
\begin{displaymath}
  h:\hspace{0.5cm}
\left\lbrace 
\begin{aligned}
  &J = \left]0,\pi \right[ &\rightarrow& I = \left]0,+\infty \right[ \\
  &\theta &\mapsto& \tan \frac{\theta}{2}
\end{aligned}
\right. 
\end{displaymath}
Elle est bijective, de bijection réciproque $h^{-1}$
\begin{displaymath}
  h^{-1}:\hspace{0.5cm}
\left\lbrace 
\begin{aligned}
  &I = \left]0,+\infty \right[ &\rightarrow&  J = \left]0,\pi \right[ \\
  &y &\mapsto& 2\arctan y
\end{aligned}
\right. 
\end{displaymath}
On peut alors décomposer:
\begin{multline*}
\psi_\alpha(\theta) = 2\left( \frac{\pi}{2}-\arctan\left(\tan\frac{\alpha}{2}\tan \frac{\theta}{2} \right) \right) 
= 2\arctan\left( \frac{1}{\tan\frac{\alpha}{2}\tan \frac{\theta}{2}} \right) \\
= 2\arctan\left(\varphi(\tan \frac{\theta}{2})\right)\hspace{0.2cm}\text{ (avec $a =\cot\frac{\alpha}{2}$ )}  
= h^{-1}\circ \varphi \circ h(\theta)
\end{multline*}
L'involution $\psi_\alpha$ est donc conjuguée d'une involution $\varphi$ de la partie II avec $a =\cot\frac{\alpha}{2}$.

Dans ce calcul, on a utilisé librement la relation valable pour tous les réels $x>0$:
\begin{displaymath}
\arctan x + \arctan \frac{1}{x} = \frac{\pi}{2}
\end{displaymath}
\end{enumerate}
