%<dscrpt>Nombre de partitions.</dscrpt>
Soient $k$, $m$, $n$ trois entiers naturels non nuls avec $k\leq m$ et $k\leq n$.\newline
On note $m^{\underline{k}}$ le produit de $k$ entiers consécutifs décroissants à partir de $m$ 
\begin{displaymath}
 m^{\underline{k}} = \underset{k \text{ facteurs}}{\underbrace{m(m-1) \cdots}} = m(m-1) \cdots (m-k+1)
\end{displaymath}
On dira que $m^{\underline{k}}$ est une \emph{puissance descendante} de $m$.\newline
On note
$ \left\lbrace \begin{matrix}
 n \\ k
\end{matrix}\right\rbrace 
$ 
le nombre de partitions d'un ensemble à $n$ éléments en $k$ parties non vides. Par exemple
$ \left\lbrace \begin{matrix}
 3 \\ 2
\end{matrix}\right\rbrace 
=3$
car les partitions de $\{a,b,c\}$ en deux parties non vides sont
\begin{displaymath}
 \left\lbrace \{a,b\},\{c\}\right\rbrace,\hspace{0.5cm} 
 \left\lbrace \{b,c\},\{a\}\right\rbrace,\hspace{0.5cm}
 \left\lbrace \{c,a\},\{b\}\right\rbrace
\end{displaymath}
\begin{enumerate}
 \item En précisant dans chaque cas l'ensemble des partitions à considérer, calculer
\begin{displaymath}
 \left\lbrace \begin{matrix}
 4 \\ 1
\end{matrix}\right\rbrace , \hspace{0.5cm}
 \left\lbrace \begin{matrix}
 4 \\ 2
\end{matrix}\right\rbrace , \hspace{0.5cm}
 \left\lbrace \begin{matrix}
 4 \\ 3
\end{matrix}\right\rbrace , \hspace{0.5cm}
 \left\lbrace \begin{matrix}
 4 \\ 4
\end{matrix}\right\rbrace
\end{displaymath}
\item  Soit $A$ et $X$ deux ensembles, respectivement de cardinal $n$ et $m$, soit $k$ entier entre $1$ et $\min(m,n)$. On note
\begin{itemize}
 \item $\Pi_k$: l'ensemble des partitions de $A$ en $k$ parties non vides.
 \item $\mathcal{F}_k$ : l'ensemble des fonctions de $A$ dans $X$ telles que $\sharp(f(A))=k$.
\end{itemize}
\begin{enumerate}
 \item Soit $f\in \mathcal{F}_k$ et $f(A)=\{y_1,y_2,\cdots,y_k\}$. On note
\begin{displaymath}
 \pi(f)=\left\lbrace f^{-1}(\{y_i\}) , i\in\{1,\cdots,k\}\right\rbrace
\end{displaymath}
 Montrer que $\pi(f)\in \Pi_k$ c'est à dire une partition de $A$ en $k$ parties non vides.
 \item  Soit $f\in \mathcal{F}_k$. Quel est le cardinal de l'ensemble des $g\in \mathcal{F}_k$ telles que $\pi(g)=\pi(f)$? 
\end{enumerate}

\item Montrer que
\begin{displaymath}
 m^n = \sum_{k=1}^{\min(m,n)}
 \left\lbrace \begin{matrix}
 n \\ k
\end{matrix}\right\rbrace
m^{\underline{k}}
\end{displaymath}

\end{enumerate}
