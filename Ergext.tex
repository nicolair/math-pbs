%<dscrpt>Rang et matrices extraites.</dscrpt>
Soit $p$ et $q$ deux entiers et $A\in \mathcal M_{p,q}(\R)$, pour toutes parties (non vides) $I$ de $\{1,2,\cdots,p\}$ et $J$ de $\{1,2,\cdots,q\}$ :
\begin{align*}
 I &= \{i_1,i_2,\cdots,i_s\}\subset \{1,2,\cdots,p\} \\
 J &= \{j_1,j_2,\cdots,j_t\}\subset \{1,2,\cdots,q\} 
\end{align*}
on définit une matrice $A_{IJ}\in \mathcal M_{s}(\R)$ dite \emph{extraite} de $A$ par :
\begin{displaymath}
 \forall (u,v)\in \{1,\cdots, s\}\times \{1,\cdots, t\} :\text{ terme $u,v$ de }A_{IJ} 
= a_{i_uj_v} 
\end{displaymath}
Par exemple :
\begin{align*}
 A =
\begin{bmatrix}
 1 & 2 & 3 & 4 \\
 -1 & 3 & 5 & 7 \\
 8 & -6 & -5 & 10
\end{bmatrix}
& &
I=\{2,3\} & & J=\{3,4\} & &
A_{IJ}=
\begin{bmatrix}
 5 & 7 \\
 -5 & 10
\end{bmatrix}
\end{align*}
L'objet de cet exercice est de montrer que le rang de $A$ est la taille de la plus grande matrice carrée inversible extraite c'est à dire le plus grand des $s$ pour lesquels il existe des parties à $s$ éléments $I$ et $J$ telles que $A_{IJ}$ soit inversible.

Soit $E$ un $\R$-espace vectoriel de dimension $p$, soit $\mathcal U =\{u_1,\cdots,u_p\}$ une base de $E$ et $\mathcal V =\{v_1,\cdots,v_q\}$ une famille de vecteurs de $E$ tels que :
\begin{displaymath}
 A = \Mat_{\mathcal U}\mathcal V \neq 0_{\mathcal M_{p,q}(\R)}
\end{displaymath}
Pour toutes parties $I$ de $\{1,\cdots,p\}$ et $J$ de $\{1,\cdots,q\}$, on définit :
\begin{itemize}
 \item $\overline{I}$ est le complémentaire de $I$ dans $\{1,\cdots,p\}$.
\item $\overline{J}$ est le complémentaire de $J$ dans $\{1,\cdots,q\}$.
\item $\mathcal U _I = \{u_i, i\in I\}$, $E_I = \Vect\left( \mathcal U_I\right)$. On utilisera librement le fait que $E_I$ et $E_{\overline{I}}$ sont des sous-espaces supplémentaires de $E$.
\item $p_I$ est la projection sur $E_I$ parallélement à $E_{\overline{I}}$.
\item $\mathcal V _J = \{v_j, j\in J\}$ , $V_J = \Vect\left( \mathcal V_J\right)$.
\end{itemize}
On définit enfin un entier $r$ par :
\begin{itemize}
 \item il existe des parties à $r$ éléments $I$ de $\{1,2,\cdots,p\}$ et $J$ de $\{1,2,\cdots,q\}$ telles que $A_{IJ}$ inversible.
\item pour toutes parties à $r+1$ éléments $I$ de $\{1,2,\cdots,p\}$ et $J$ de $\{1,2,\cdots,q\}$ (s'il en existe), $A_{IJ}$ n'est pas inversible.
\end{itemize}

\begin{enumerate}
 \item Montrer que $r\geq 1$.
\item \begin{enumerate}
\item Pour toutes parties (non vides) $I$ de $\{1,2,\cdots,p\}$ et $J$ de $\{1,2,\cdots,q\}$, montrer que $A_{IJ}$ est la matrice dans une certaine base d'une certaine famille de vecteurs. On précisera soigneusement l'espace vectoriel, la base et la famille.
\item En déduire que $r\leq \rg(A)$. 
\end{enumerate}
\item Pour toutes parties (non vides) $I$ de $\{1,2,\cdots,p\}$ et $J$ de $\{1,2,\cdots,q\}$, montrer que la restriction de $p_I$ à $V_J$ est injective si et seulement si 
\begin{displaymath}
 E_{\overline{I}} \cap V_J = \{0_E\}
\end{displaymath}
\item Soit $J$ une partie (non vide) de $\{1,2,\cdots,q\}$ telle que $\mathcal V_J$ soit libre. Montrer que $\mathcal V_J$ est une base de $E$ ou que l'on peut compléter cette famille par des vecteurs de $\mathcal U$ pour former une base de $E$.
\item Montrer que $r=\rg(A)$.
\end{enumerate}
