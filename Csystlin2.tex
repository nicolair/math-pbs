\begin{enumerate}
  \item Pour passer de $(S)$ à $(S')$, on effectue les opérations élémentaires suivantes:
\begin{itemize}
  \item échanger les lignes 1 et 2
  \item échanger les lignes 2 et 3
  \item échanger les colonnes 2 et 3 (inconnues $y$ et $z$)
  \item $L_2 \leftarrow L_2 -  L_1$
  \item $L_3 \leftarrow L_3 - aL_1$  
\end{itemize}
Le système obtenu est alors
\begin{displaymath}
\left\lbrace  
\begin{alignat}{6}
x &+&     z      &+ &aby       &= b \\
  & &     (a-1)z &+ &b(1-a)y   &= 1-b \\
  & &     (1-a)z &+ &b(1-a^2)y &= 1-ab
\end{alignat}
\right. 
\end{displaymath}
La dernière opération $L_3 \leftarrow L_3 + L_2$ conduit au système annoncé
\begin{displaymath}
(S')
\left\lbrace  
\begin{aligned}
x + &z + &ab y &= b \\
&(a-1)z + &b(1-a)y &= 1-b \\
&  &b(1-a)(2+a)y &= 2-ab -b
\end{aligned}
\right. 
\end{displaymath}

  \item La forme triangulaire du système permet la discussion
\begin{itemize}
  \item Si $a \notin \left\lbrace 1, -2\right\rbrace$ et $b\neq 0$, le système admet un unique triplet solution.
  \item Si $a=1$, le système devient
\begin{displaymath}
\left\lbrace  
\begin{aligned}
x + &z + &b y &= 1 \\
&  & 0 &= 1-b \\
&  & 0 &= 2-2b 
\end{aligned}
\right. 
\end{displaymath}
On en déduit
\begin{itemize}
  \item Si $a=1$ et $b\neq 1$, le système n'admet pas de solutions.
  \item Si $a=1$ et $b=1$, l'ensemble des solutions est
\begin{displaymath}
  \left\lbrace (1-\lambda - \mu, \lambda,  \mu),  (\lambda,\mu) \in \R^2 \right\rbrace 
\end{displaymath}
\end{itemize}

\item Si $a=-2$, le système devient
\begin{displaymath}
\left\lbrace  
\begin{aligned}
x + &z - &2b y &= b \\
-&3z + &3by &= 1-b \\
&  &0 &= 2 +b
\end{aligned}
\right. 
\end{displaymath}
On en déduit
\begin{itemize}
  \item Si $a=-2$ et $b\neq -2$, le système n'admet pas de solution.
  \item Si $a=-2$ et $b=-2$, le système devient
\begin{displaymath}
\left\lbrace  
\begin{aligned}
x + &z + &4y &= -2 \\
-&3z - &6y &= 3
\end{aligned}
\right. 
\end{displaymath}
L'ensemble des solutions est
\begin{displaymath}
\left\lbrace (-1-2\lambda, \lambda, -1-2\lambda), \lambda \in \R\right\rbrace 
\end{displaymath}
\end{itemize}

\item Si $b=0$, le système devient
\begin{displaymath}
\left\lbrace  
\begin{aligned}
x + &z  & &= 0 \\
&(a-1)z  & &= 1 \\
&  &0 &= 2
\end{aligned}
\right. 
\end{displaymath}
Le système n'admet donc pas de solutions.
\end{itemize}

\end{enumerate}
