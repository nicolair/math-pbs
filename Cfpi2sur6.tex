\begin{enumerate}
 \item \begin{enumerate}
 \item Le calcul des intégrales se fait avec des intégrations par parties. On obtient :
\begin{align*}
 \int_0^1 t^2\cos(k\pi t) dt = \frac{2(-1)^k}{(k\pi)^2} & &, & &
\int_0^1 t\cos(k\pi t) dt = \frac{(-1)^k-1}{(k\pi)^2}
\end{align*}
On en déduit
\[\int_0^1 (ct^2 +dt)\cos(k\pi t) dt = \frac{(2c+d)(-1)^k-d}{(k\pi)^2}\]
\item La relation 
\[(2c+d)(-1)^k-d=\pi^2\]
est valable pour \emph{tous les} $k$ si et seulement si $2c+d=0$ et $d=-\pi^2$. On en déduit le couple $(a,b)$ cherché
\begin{displaymath}
  a=\frac{\pi^2}{2}\;,\;b=-\pi^2
\Rightarrow
\forall k\in \N^*,\; \frac{\pi^2}{2}\int_0^1 (t^2 -2t)\cos(k\pi t) dt =\frac{1}{k^2}
\end{displaymath}

\item Utilisons les $a$ et $b$ de la question précédente :
\begin{multline*}
 \int_0 ^1(at^2+bt)\left( \frac{1}{2} + \sum_{k=1}^n \cos (k\pi t)\right) dt 
= \frac{1}{2}\int_0 ^1(at^2+bt)dt + \sum_{k=1}^n \frac{1}{k^2} \\
= \frac{a}{6}+\frac{b}{4} + \sum_{k=1}^n \frac{1}{k^2} 
= -\frac{\pi^2}{6} + \sum_{k=1}^n \frac{1}{k^2}
\end{multline*}

\end{enumerate}
\item En considérant le $cos$ comme la partie réelle de l'exponentielle complexe, on peut symétriser la somme pour la voir comme une suite géométrique de raison $e^{2i\theta}\neq1$.
\begin{multline*}
 1+2\sum_{k=1}^n\cos 2k\theta
  = \sum_{k=-n}^{n} \left( e^{2i\theta}\right)^k
  =\frac{e^{2(n+1)i\theta} - e^{-2ni\theta}}{e^{2i\theta} - 1}\\
  =\frac{e^{i\theta}\left( e^{(2n+1)i\theta} - e^{-(2n+1)i\theta}\right)}{e^{i\theta}\left( e^{i\theta} - e^{-i\theta}\right) }
  = \frac{\sin(2n+1)\theta}{\sin \theta}
\end{multline*}

\item Comme $f$ est $\mathcal{C}^1$, on peut procéder à une intégration par parties :
\[\int_0^1f(t)\sin (\lambda t)dt=\frac{f(0)-\cos \lambda f(1)}{\lambda} + \frac{1}{\lambda} \int_0 ^1 \cos \lambda t f^\prime (t)dt\]
On en déduit
\[\left \vert \int_0^1f(t)\sin (\lambda t)dt \right\vert \leq \frac{\vert f(0)\vert + \vert f(1)\vert + \underset{[0,1]}{\sup}\vert f^\prime(t)\vert }{\lambda}\]
ce qui prouve bien la convergence vers $0$ pour $\lambda$ en $+\infty$.

\item
\begin{enumerate}
 \item La fonction $f$ est clairement de classe $\mathcal{C}^{\infty}$ sur $]0,1]$. \`A l'aide d'une étude locale en 0, on va montrer que $f$ est continue en 0 et que $f^\prime_{\vert ]0,1[}$ converge en 0. Ceci prouvera le caractère $\mathcal C ^1$ de la fonction sur $[0,1]$ d'après le théorème de la limite de la dérivée.\newline
Les équivalences, limites et développements suivants sont tous en $0$.\newline
Montrons d'abord la continuité en étudiant la limite de $f$.
\begin{displaymath}
\left. 
\begin{aligned}
  t^2-2t  \sim&  -2t \\
\sin \frac{\pi}{2}t  \sim&  \frac{\pi}{2}t
\end{aligned}
\right\rbrace 
\Rightarrow f \rightarrow \frac{-2\pi^2}{4\frac{\pi}{2}} =-\pi 
\end{displaymath}
\'Etudions ensuite la limte de la dérivée
\[f^\prime (t)=\frac{\pi^2}{4}\left( \frac{2t-2}{\sin \frac{\pi}{2}t} - \frac{\pi}{2}\frac{(t^2-2t)\cos \frac{\pi}{2}t}{\sin ^2 \frac{\pi}{2}t}\right). \]
\[
\frac{2t-2}{\sin \frac{\pi}{2}t}=\frac{-2+2t}{\frac{\pi}{2}t+o(t^2)}=-\frac{4}{\pi t}\frac{1-t}{1+o(t)} = -\frac{4}{\pi t}(1-t+o(t))
\]
\begin{multline*}
 \frac{(t^2-2t)\cos \frac{\pi}{2}t}{\sin ^2 \frac{\pi}{2}t}
= \frac{(-2t+t^2)(1+o(t))}{\frac{\pi^2}{4}t^2+o(t^3)} 
= \frac{-2t+t^2+o(t^2)}{\frac{\pi^2}{4}t^2+o(t^3)}\\
= -\frac{8}{\pi^2 t}\frac{1-\frac{t}{2}+o(t)}{1+o(t)}
= -\frac{8}{\pi^2 t}(1-\frac{t}{2}+o(t))
\end{multline*}
d'où en combinant les deux parties :
\begin{displaymath}
f^\prime (t)=\frac{\pi ^2}{4}(\frac{2}{\pi}+o(1))\rightarrow \frac{\pi}{2} 
\end{displaymath}
C'est à dire que la dérivée de la restriction de $f$ à $]0,1[$ converge en 0 vers $\frac{\pi}{2}$ ce qui entraine que $f$ est dérivable en $0$ avec \[f^\prime(0)=\frac{\pi}{2}\] et que $f^\prime$ est continue en $0$.

\item Notons $s_n=\sum_{k=1}^n\frac{1}{k^2}$. D'après 1.c :
\[\int_0 ^1(\frac{\pi^2}{2}t^2-\pi^2t)\left( \frac{1}{2} + \sum_{k=1}^n \cos (k\pi t)\right) dt = -\frac{\pi^2}{6}+s_n\]
Utilisons 2. avec $\theta=\frac{\pi t}{2}$ puis la fonction $f$ définie en 4.:
\begin{align*}
\int_0 ^1(\frac{\pi^2}{2}t^2-\pi^2t)\frac{\sin (2n+1)\frac{\pi t}{2}}{2\sin \frac{\pi t}{2}} &= -\frac{\pi^2}{6}+s_n \\
\int_0^1f(t)\sin (2n+1)\frac{\pi t}{2} &=-\frac{\pi^2}{6}+s_n
\end{align*}
La question 3 montre (avec $\lambda=\frac{(2n+1)\pi}{2}$) la convergence de $(s_n)_{n\in\N^*}$ vers
\[\frac{\pi^2}{6}\]
\end{enumerate}
\end{enumerate}