%<dscrpt>Exercice sur les polynômes : relation entre coefficients et racines.</dscrpt>
On d\'{e}signe par $\cot $ la fonction cotangente $\frac{\cos }{\sin }$. Soit $x$ un nombre réel non entier.
\begin{enumerate}
\item
\begin{enumerate}
 \item Préciser les racines 5-èmes de $e^{2i\pi x}$.
 \item Soit $\theta$ réel ($\theta \not \equiv 0 \mod 2\pi$), simplifier $i\frac{Z+1}{Z-1}$ pour $Z=e^{i\theta}$.
\end{enumerate}
 
\item  D\'{e}terminer les racines du polyn\^{o}me complexe 
\begin{displaymath}
(X-i)^{5}(i+\cot (\pi x))+(X+i)^{5}(i-\cot (\pi x))  
\end{displaymath}

\item  En déduire des expressions simples pour la somme et le produit 
\begin{displaymath}
 \sum_{k=0}^{4}\cot ((x+k)\frac{\pi}{5}),\hspace{1cm}
 \prod_{k=0}^{4}\cot ((x+k)\frac{\pi}{5}) 
\end{displaymath}
\end{enumerate}

