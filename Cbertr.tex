\begin{enumerate}
\item \begin{enumerate}
\item La fonction $f_{n}$ est croissante de 0 à $n$, décroissante ensuite
$$f'_{n}(t)=\frac{1}{n!}(n-t)t^{n-1}e^{-t}$$
Elle atteint sa borne supérieure en 
$$M_{n}=f_{n}(n)=\frac{n^{n}e^{-n}}{n!}$$
\item On forme le quotient de deux termes consécutifs
\begin{eqnarray*}
\frac{M_{n+1}}{M_{n}}&=&\frac{(n+1)^{n+1}e^{-(n+1)}}{(n+1)!}\frac{n!}{n^{n}e^{-n}}=\left(\frac{n+1}{n}\right)^{n}e^{-1}\\&=&\left(1+\frac{1}{n}\right)^{n}e^{-1}=e^{-1+n\ln (1+\frac{1}{n})}
\end{eqnarray*}
On sait que $\ln (1+x)\leq x$ pour $x\geq 0$ (concavité de $\ln$) donc si $x=\frac{1}{n}$,$-1+n\ln (1+\frac{1}{n})\leq 0$ et $\frac{M_{n+1}}{M_{n}}\leq 1$.
\item Comme la suite est décroissante, elle est bornée par son premier terme. On peut prendre $$M=M_{1}=\frac{1}{n}$$
Le calcul de $\int_{0}^{+\infty}\frac{t^{n}e^{-t}}{n!}\,dt$ se fait par récurrence à l'aide d'une intégration par parties. On trouve en fait 
$$\int_{0}^{+\infty}\frac{t^{n}e^{-t}}{n!}\,dt = 1$$
\item Cas $a_{n}=(-1)^{n}$ On sait que 
$$(\sum _{k=0}^{n}\frac{(-t)^{k}}{k!})_{n\in \Bbb{N}}\rightarrow e^{-t}$$
donc
$$(\sum _{k=0}^{n}\frac{(-1)^{k}t^{k}e^{-t}}{k!})_{n\in \Bbb{N}}\rightarrow e^{-2t}$$
qui est intégrable.

Cas $a_{n}=(-1)^{n}n$. La somme démarre alors en fait en $k=1$ et on peut décaler les indices.
\[\sum _{k=0}^{n}\frac{(-1)^{k} k t^{k}e^{-t}}{k!}= (-t)\sum _{k=1}^{n-1}\frac{(-1)^{k-1}t^{k-1}e^{-t}}{(k-1)!}\]
cette suite converge vers $-te^{-2t}$ qui est intégrable. Les deux séries alternées considérées convergent donc au sens de $(f_{n})$
\end{enumerate}
\item \begin{enumerate}
\item Les fonctions sont affines par morceaux
\begin{figure}
   \centering
   \includegraphics[scale=0.25]{Cbertr0.pdf}
   \includegraphics[scale=0.25]{Cbertr1.pdf}
   \includegraphics[scale=0.25]{Cbertr2.pdf}
   \includegraphics[scale=0.25]{Cbertr3.pdf}
\end{figure}
\item La famille est de Bertrand car les bornes supérieures et les intégrales valent 1 (aire du triangle).
\item Pour tout $t$ fixé, la suite $(a_n f_{n}(t))_{n\in \Bbb{N}}$ est nulle à partir d'un certain rang. La suite $(\sum _{k=0}^{n}a_k f_{k}(t))_{n\in \Bbb{N}}$ est stationnaire donc convergente.

Si $t\in [p,p+1]$, seules $f_{p}(t)$ et $f_{p+1}(t)$ ne sont pas nulles parmi les $f_{k}(t)$, donc $f(t)=a_{p}f_{p}(t)+ a_{p+1}f_{p+1}(t)$ et
$$\int_{p}^{p+1}f(t)\,dt=\frac{a_{p}+a_{p+1}}{2}$$
\item Si $(\sum _{k=0}^{n}a_{k})_{n\in \Bbb{N}}$ converge, il en est de même de $(\sum _{k=0}^{n} \frac{ a_{k}+ a_{k+1}}{2})_{n\in \Bbb{N}}$ donc la série converge au sens de $(f_{n})$

On a vu en 1.d. que $\sum (-1)^{n}$ convergeait au sens de $(\frac{t^{n}e^{-t}}{n!})$ mais divergeait au sens usuel.
\end{enumerate}
\end{enumerate}
