
\subsubsection*{Partie I. Exemples. Une inégalité générale.}

Dans toute cette partie, on utilisera souvent le r\'{e}sultat suivant.

\begin{quote}
Si une fonction continue, d\'{e}finie dans $\R$ est monotone au
voisinage de $+\infty $ et de $-\infty $ alors elle major\'{e}e si et
seulement si elle ne diverge pas vers $+\infty$ ni en $-\infty $ ni en $+\infty $.
\end{quote}

Ce r\'{e}sultat est une cons\'{e}quence des d\'{e}finitions des limites et
du fait qu'une fonction continue sur un segment est born\'{e}e$.$

\begin{enumerate}
\item Ici $X=\R$, $f(x)=Kx^{2}$, $h_m(x)=mx-Kx^2$.
\begin{itemize}
 \item Si $K<0$, $h_m$ n'est majorée pour aucune valeur de $m$.
 \item Si $K>0$, $h_m$ est majorée pour tous les $m$ réels. On obtient facilement la valeur minimale de la fonction du second degré. On en déduit :
\begin{displaymath}
 X^\circ =\R ;\; f^\circ (m)=\dfrac{m^2}{4K}
\end{displaymath}
On vérifie que $f=f^\circ$ si et seulement si $K=\frac{1}{2}$.
\end{itemize}


\item  Lorsque $f$ est une fonction continue sur un segment $X=\left[
a,b\right] $, il en est de m\^{e}me des fonctions $h_{m}$ pour n'importe
quelle valeur de $m$. Ces fonctions sont donc toujours born\'{e}es ce qui
montre $x^\circ =\R$ \newline
 Chaque fonction continue $h_{m}$ atteint ses bornes sur le segment $
\left[ a,b\right] $. Il existe donc $x_{m}\in \left[ a,b\right] $ tel que $
f^{\circ }(m)=h_{m}(x_{m})$. Dans ce cas rien n'assure l'unicité de $x_m$.


\item  Ici $X=\R$ et 
\[f(x)=e^{x} , h_{m}(x)=mx-e^{x}\]
Examinons les limites 
\begin{eqnarray*}
\text{en }-\infty  &:&\quad h_{m}(x)\rightarrow \left\{
\begin{array}{ccc}
-\infty  & \text{si} & m>0 \\
0 & \text{si} & m=0 \\
+\infty  & \text{si} & m<0
\end{array}
\right.  \\
\text{en }+\infty  &:&\quad h_{m}(x)\rightarrow -\infty
\end{eqnarray*}
Le r\'{e}sultat pr\'{e}cis\'{e} au début permet de conclure que $X^{\circ }=\left[ 0,+\infty \right[ $.\newline
Si $m=0$, la fonction $h_{0}(x)=-e^{x}$ admet $0$ comme borne sup\'{e}rieure donc $f^{\circ }(0)=0$.\newline
Si $m>0$, formons le tableau de variations de $h_{m}(x)=mx-e^{x}$. On a $h_{m}^{\prime }(x)=m-e^{x}$ et
\[
\begin{array}{c|ccccc}
& 0 &  & \ln m &  & +\infty  \\
\hline
&  &  & m\ln m-m &  &  \\
h_{m} &  & \nearrow  &  & \searrow  &  \\
& -1 &  &  &  & -\infty
\end{array}
\]
On en d\'{e}duit (en notant \`{a} nouveau $x$ la variable)
\begin{eqnarray*}
X^{\circ } &=&\left[ 0,+\infty \right[  \\
f^{\circ }(x) &=&\left\{
\begin{array}{ccc}
0 & \text{si} & x=0 \\
x\ln x-x & \text{si} & x>0
\end{array}
\right.
\end{eqnarray*}
On forme alors $k_{u}(x)=ux-x\ln x-x$ pour $x>0$ dont on cherche le tableau
de variations $k_{u}^{\prime }(x)=u-\ln x$ :
\[
\begin{array}{c|ccccc}
& 0 &  & e^{u} &  & +\infty  \\
\hline
&  &  & e^{u} &  &  \\
k_{u} &  & \nearrow  &  & \searrow  &  \\
& 0 &  &  &  & -\infty
\end{array}
\]
On en d\'{e}duit $X^{\circ \circ }=\R$ avec $f^{\circ \circ
}(x)=e^{x}$.

\item  Ici $X=\R$ et $f(x)=\alpha x+\beta $, $h_{m}(x)=mx-\alpha
x-\beta =(m-\alpha )x-\beta $. Si $m\neq \alpha ,$ une des limites en $\pm
\infty $ est $+\infty $ et donc $h_{m}$ n'est pas major\'{e}e$.$ On en d\'{e}%
duit
\[
X^{\circ }=\left\{ \alpha \right\} ,\quad f^{\circ }(\alpha)=-\beta
\]
Alors $h_{u}$ est toujours born\'{e}e puisque son domaine de d\'{e}finition
est r\'{e}duit au seul point $\alpha $ avec $h_{u}(\alpha )=u\alpha -(-\beta
)$. Donc en revenant \`{a} la lettre $x$ pour d\'{e}signer la variable :
\[
X^{\circ \circ }=\R,\quad f^{\circ \circ }(x)=\alpha x+\beta
\]


\item  Pour $x\in X$ et $m\in X^{\circ },$ $mx-f(x)=h_{m}(x)\leq f^{\circ
}(m)=\sup_{X}h_{m}$ d'o\`{u}
\[
mx\leq f(x)+f^{\circ }(m)
\]

\end{enumerate}

\subsubsection*{Partie II. Espaces $\mathcal{N}$ et $\mathcal{N}_0$ de fonctions convexes.}
Les conditions imposées aux fonctions de $\mathcal{N}$ et $\mathcal{N}_0$ entraînent clairement qu'elles sont convexes et croissantes.
\begin{enumerate}
\item Comme $f(0)=0$, on peut poser $\tau(x)=\frac{f(x)}{x}$ et interpréter $\tau$ comme le taux d'accroissement en 0 de la fonction $f$. D'après le théorème des accroissements finis, il existe $c_x \in ]0,x[$ tel que $\tau(x)=f'(c_x)$. Comme $\tau$ diverge vers $+\infty$, la fonction $f'$ n'est pas majorée, comme $f'$ est strictement croissante, elle diverge vers $+\infty$.
\item Appliquons le théorème des accroissements finis entre $x$ et $2x$. Il existe $c\in]x,2x[$ tel que
\[f(2x)-f(x)=xf'(c)\geq xf'(x)\]
car $f'$ est croissante. Comme de plus $f$ est aussi croissante avec $f(0)=0$, $f(x)$ est positif donc
\[xf'(x)\leq f(2x)\]
\item On suppose ici que $f'\rightarrow +\infty$ en $+\infty$. Comme
\[f'(x)\leq 2 \frac{f(2x)}{2x}\]
On en déduit $\frac{f(2x)}{2x}\rightarrow +\infty$. Comme d'autre part $\tau$ est croissante car $f$ est convexe, cela prouve que $\tau \rightarrow +\infty$
\end{enumerate}

\subsubsection*{Partie III. Transformée de Legendre dans $\mathcal{N}_0$.}
\begin{enumerate}
\item Les propriétés des fonctions dans $\mathcal{N}_0$  en particulier $f'$ croissante, $f'\rightarrow +\infty$ en $+\infty$ et le calcul de $h'_m(x)=m-f'(x)$ conduisent au tableau suivant pour $h_m$
\[
\begin{array}{c|ccccc}
& 0 &  & x_m &  & +\infty  \\
\hline
&  &  & f^{\circ}(m) &  &  \\
h_{m} &  & \nearrow  &  & \searrow  &  \\
& 0 &  &  &  & -\infty
\end{array}
\]
La fonction $h_m$ atteint son maximum en un unique point $x_m$ qui est noté $\varphi(m)$.
\item \begin{enumerate}
\item Les conditions de $\mathcal{N}_0$ entraînent clairement que $f'$ est strictement croissante de $\R_+$ dans $\R_+$, elle est bijective car continue avec les "bonnes limites".
\item Comme $x_m$ annule la dérivée de $h_m$, $m=f'(\varphi(m))$ donc $\varphi$ est la \emph{bijection réciproque} de $f'$. On en déduit que $\varphi$ est continue (d'après un résultat du cours). Elle est également croissante ce qui conduit aux "bonnes limites" grace aux limites de $f'$.
      \end{enumerate}
 \item D'après les questions précédentes, on peut écrire
 \[f^{\circ}(m)=m\varphi(m)-f\circ\varphi(m)\]
 Comme $f'$ est strictement croissante avec $f''>0$, la bijection réciproque de $f'$ est dérivable (résultat de cours) avec
 \[\varphi'=\frac{1}{f''\circ \varphi}\]
 On en déduit que $\varphi$ est $\mathcal{C}^1$ puis que $f^\circ$ est $\mathcal{C}^1$ avec
 \[f^{\circ\prime}(m)=\varphi(m)+m\varphi'(m)-\varphi'(m)\underbrace{f'(\varphi(m))}_{=m}=\varphi(m)\]
 car $f'(\varphi(m))=m$. Comme $\varphi$ est $\mathcal{C}^1$, cela prouve que $f^\circ$ est $\mathcal{C}^2$. D'après le calcul de $\varphi'$ déjà effectué,
 \[f^{\circ\prime \prime}=\frac{1}{f''\circ \varphi}\]
 \item Les calculs précédents et la définition de $\mathcal{N}_0$ montrent clairement que $f^\circ \in \mathcal{N}_0$. D'autre part, $f^{\circ\circ\prime}$ est la bijection réciproque de $f^{\circ\prime}$ c'est à dire $f'$. Ainsi $f^{\circ\circ}$ et $f$ ont la même dérivée, la condition en 0 prouve qu'elles sont égales.

\item \begin{enumerate}
 \item En utilisant des notations évidentes à partir des définitions :
\begin{displaymath}
 f\leq g \Rightarrow h_{g,m}\leq h_{f,m} \Rightarrow g^{\circ} \leq f^{\circ}
\end{displaymath}

\item La résolution de cette question ne repose pas sur la question précédente mais sur l'utilisation de $\varphi$ et la remarque suivante.
\begin{quote}
 La seule involution strictement croissante d'un intervalle $I$ dans lui même est l'identité.
\end{quote}
 En effet notons $g$ une telle involution  ($g\circ g =Id_{I}$). Pour tout $x\in I$, $g(x)<x$ est impossible car il entraîne
$x<g(x)$. De même $x<g(x)$ est impossible.

Introduisons la fonction $\varphi$ des questions 1 2. 3.. On a montré en particulier que :
\begin{itemize}
 \item $\varphi$ est la bijection réciproque de $f^\prime$.
 \item $f^{\circ \prime}= \varphi$
\end{itemize}
Lorsque $f=f^{\circ}$ alors $f^{\prime}=\varphi$ est involutive et strictement croissante donc $f^\prime(x)=x$ pour tous les $x\geq 0$ donc, en tenant compte de $f(0)=0$ :
\begin{displaymath}
 f(x)=\dfrac{1}{2}x^2
\end{displaymath}

\end{enumerate}

\end{enumerate}
