\subsection*{I. Outils.}
\begin{enumerate}
 \item Un coefficient du binôme est un quotient de deux produits chacun étant constitué d'un même nombre de facteurs consécutifs:
\begin{displaymath}
 \frac{k}{n}\binom{n}{k} = \frac{k}{n} \frac{n(n-1)...}{k(k-1)...} = \frac{(n-1)(n-2)...}{(k-1)(k-2)...}
=\binom{n-1}{k-1} 
\end{displaymath}

 \item Les propriétés $\mathcal{P}_2$ et $\mathcal{P}_3$ sont logiquement équivalentes car $\mathcal{P}_3$ formule la \emph{définition} de la convergence des suites $\left( f_n(x)\right) _{n\in \N}$ pour chaque $x$.\newline
La proposition $\mathcal{P}_1$ implique les deux autres car elle signifie que pour chaque $\varepsilon$, il existe un $\alpha$ valable pour \emph{tous} les $x$. Dans les autres propositions, cet $\alpha$ peut dépendre de $x$, les propositions ne sont pas équivalentes.
   
 \item Remarquons que si $x=y$ n'importe quel $z$ dans $[0,1]$ convient. Supposons donc $x\neq y$ et  notons $\varphi$ la fonction suggérée par l'énoncé.
\begin{displaymath}
 \varphi(t) = f(t)+(y-t)f'(t) + (y-t)^2M
\end{displaymath}
Elle dépend de $y$, on aurait pu le marquer en la notant $\varphi_y$.  Elle a les mêmes propriétés de régularité que $f'$ c'est à dire $\mathcal{C}^1[0,1]$. On peut donc lui appliquer le théorème de Rolle entre $x$ et $y$ si on choisit un $M$ tel que $\varphi(x)=\varphi(y)$.\newline
Cette contrainte se traduit par une équation très simple d'inconnue $M$:
\begin{displaymath}
 f(y) = f(x)+(y-x)f'(x) + (y-x)^2M 
\end{displaymath}
 qui admet une solution car $y\neq x$. Il n'est pas utile de l'exprimer.\newline
Pour ce $M$ là, il existe $z$ strictement entre $x$ et $y$ (donc dans $[0,1]$) tel que $\varphi'(z)=0$. Or
\begin{displaymath}
 \varphi'(t) = f'(t)-f'(t)+(y-t)f''(t)-2(y-t)M =(y-t)(f''(t)-2M)
\end{displaymath}
De $z\neq y$, on tire $f''(z) = 2M$. En réinjectant dans la relation $\varphi(x)=\varphi(y)$, on obtient
\begin{displaymath}
 f(y) = f(x)+(y-x)f'(x) + (y-x)^2\frac{f''(z)}{2} 
\end{displaymath}

\item La fonction $x\mapsto x(1-x)$ est croissante de $0$ à $\frac{1}{2}$, décroissante de $\frac{1}{2}$ à $1$. On en tire en particulier:
\begin{displaymath}
 \forall x\in [0,1],\hspace{0.5cm} 0\leq x(1-x) \leq \frac{1}{4}
\end{displaymath}

\end{enumerate}

\subsection*{II. Propriétés.}
\begin{enumerate}
 \item On perçoit l'analogie avec la relation fondamentale entre les coefficients du binôme:
\begin{multline*}
 (1-t)B^{n-1}_k(t)+tB^{n-1}_{k-1}=
\binom{n-1}{k}t^k(1-t)^{n-k}\\
 + \binom{n-1}{k-1}t^k(1-t)^{n-1-(k-1)}\text{ avec } n-1-(k-1)=n-k\\
=\left(\binom{n-1}{k} + \binom{n-1}{k-1} \right)t^k(1-t)^{n-k}
= B^n_k 
\end{multline*}

\item Dans les trois cas, le calcul repose sur la formule du binôme.

Cas $f(x)=1$. On applique directement la formule du bin{\^o}me :
\begin{displaymath}
f_{n}(x)=\sum_{k=0}^{n}\binom{n}{k}x^{k}(1-x)^{n-k}=(1-x+x)^n=1 
\end{displaymath}

Cas $f(x)=x$. On remarque d'abord que le terme associ{\'e} {\`a} $k=0$ est nul. Puis on applique la relation obtenue en I1.
\begin{multline*}
f_{n}(x)=\sum_{k=0}^{n}\frac{k}{n}\binom{n}{k}x^{k}(1-x)^{n-k}
= \sum_{k=1}^{n}\binom{n-1}{k-1}x^{k}(1-x)^{n-k} \\
=x\sum_{k=1}^{n}\binom{n-1}{k-1}x^{k-1}(1-x)^{n-1-(k-1)}
= x(1-x+1)^{n-1}=x 
\end{multline*}
en tenant compte du d{\'e}calage d'indice et de la formule du bin{\^o}me.

Cas $f(x)=e^{x}$.
\begin{displaymath}
f_{n}(x)  =  \sum_{k=0}^{n}e^{\frac{k}{n}}\binom{n}{k}x^{k}(1-x)^{n-k}
=\sum_{k=0}^{n}\binom{n}{k}(e^{\frac{1}{n}}x)^{k}(1-x)^{n-k}
 = \left( e^{\frac{1}{n}}x+1-x\right) ^{n}
\end{displaymath}

\item  On veut montrer ici que $x(1-x){B^n_k}'(x) = (k-nx)B^n_k(x)$.\newline
Si $k=0$.
\begin{displaymath}
 x(1-x){B^n_0}'(x) = -nx(1-x)(1-x)^{n+1} = -nx(1-x)^n = (k-nx)B^n_0(x)
\end{displaymath}
Si $k=n$.
\begin{displaymath}
 x(1-x){B^n_n}'(x) = nx(1-x)x^{n-1} = n(1-x)x^n =(k-nx)B^n_n(x)
\end{displaymath}
Si $1\leq k \leq n-1$.
\begin{multline*}
 x(1-x){B^n_k}'(x) = x(1-x)\binom{n}{k}\left(kx^{k-1}(1-x)^{n-k} -(n-k)x^{k}(1-x)^{n-k-1} \right) \\
= \left( k(1-x)-(n-k)x\right) \binom{n}{k}x^{k}(1-x)^{n-k}
= (k-nx) B^n_k(x)
\end{multline*}

\item
\begin{enumerate}
\item De la définition de la fonction $g$, on déduit $g(\frac{k}{n})=\frac{k}{n}f(\frac{k}{n})$. On peut alors exprimer $g_n(x)-xf_n(x)$ comme une somme, factoriser partiellement et utiliser le résultat de la question précédente
\begin{multline*}
g_n(x)-xf_n(x) =
\sum_{k=0}^n\left(\frac{k}{n}-x \right)f(\frac{k}{n})B^n_k(x)
= \frac{1}{n}\sum_{k=0}^nf(\frac{k}{n})\left(k-nx \right)B^n_k(x)\\
= \frac{1}{n}\sum_{k=0}^nf(\frac{k}{n})x(1-x){B^n_n}'(x)
= \frac{x(1-x)}{n}\left( \sum_{k=0}^nf(\frac{k}{n}){B^n_n}(x)\right)' 
= \frac{x(1-x)}{n}f'_n(x)
\end{multline*}

\item  Si $f(x)=x$ pour tous les $x$ alors $g(x)=x^{2}$ et $f_{n}(x)=x$. La formule pr{\'e}c{\'e}dente donne donc
\begin{displaymath}
 \frac{x(1-x)}{n}=-x^{2}+g_{n}(x)
\end{displaymath}
On en d{\'e}duit que si $f(x)=x^{2}$ pour tous les $x$ alors
\begin{displaymath}
f_{n}(x)=\frac{x(1-x)}{n}+x^{2} 
\end{displaymath}

\end{enumerate}
\item  Examinons les trois sommes obtenues {\`a} partir du d{\'e}veloppement
\begin{displaymath}
 (\frac{k}{n}-x)^{2}=(\frac{k}{n})^{2}-2\frac{k}{n}x+x^{2}
\end{displaymath}
Elles correspondent {\`a} des calculs d{\'e}j{\`a} effectu{\'e}s
\begin{align*}
&\sum_{k=0}^{n}(\frac{k}{n})^{2}\binom{n}{k}x^{k}(1-x)^{n-k} =\frac{x(1-x)}{n}+x^{2} \\
&\sum_{k=0}^{n}(-2x)(\frac{k}{n})\binom{n}{k}x^{k}(1-x)^{n-k} =-2x^{2} \\
&\sum_{k=0}^{n}x^{2}\binom{n}{k}x^{k}(1-x)^{n-k} = x^{2}
\end{align*}
On en d{\'e}duit
\begin{displaymath}
 \sum_{k=0}^{n}(\frac{k}{n}-x)^{2}\binom{n}{k}x^{k}(1-x)^{n-k}=\frac{x(1-x)}{n}
\end{displaymath}
\end{enumerate}

\subsection*{III. Monotonie.}
\begin{enumerate}
 \item Si $k=0$, $B^n_0(t)= (1-t)^n$ donc ${B^n_0}'=-nB^{n-1}_0$.\newline
Si $k=n$, $B^n_n(t)= t^n$ donc $B'^n_n=nB^{n-1}_{n-1}$.\newline
Si $1\leq k \leq n-1$, on utilise encore la relation obtenue en I1. en particulier associée à la symétrie des coefficients
\begin{displaymath}
 (n-k)\binom{n}{k} = (n-k)\binom{n}{n-k} = n \binom{n-1}{n-k-1} = n \binom{n-1}{k}  
\end{displaymath}
Il vient:
\begin{multline*}
 {B^n_k}'(t)= \binom{n}{k}kt^{k-1}(1-t)^{n-k}-\binom{n}{k}(n-k)t^{k}(1-t)^{n-k-1}\\
= n\left( \binom{n-1}{k-1}t^{k-1}(1-t)^{n-k} - \binom{n-1}{k}t^{k}(1-t)^{n-k-1}\right) 
\end{multline*}
Comme $n-k = (n-1)-(k-1)$ et $n-k-1 = (n-1)-k$, on obtient
\begin{displaymath}
 {B^n_k}'= n \left(B^{n-1}_{k-1}-B^{n-1}_{k} \right) 
\end{displaymath}

 \item En utilisant les relations trouvées à la question précédente, il vient
\begin{multline*}
 \frac{1}{n}f'_n(x)=
 \sum_{k=1}^nf(\frac{k}{n}) B^{n-1}_{k-1}
-\sum_{k=0}^{n-1}f(\frac{k}{n}) B^{n-1}_{k}
= \sum_{k=0}^{n-1}f(\frac{k+1}{n}) B^{n-1}_{k}
-\sum_{k=0}^{n-1}f(\frac{k}{n}) B^{n-1}_{k}\\
= \sum_{k=0}^{n-1}\left(f(\frac{k}{n}+\frac{1}{n})-f(\frac{k}{n}) \right)B^{n-1}_{k}
=  \sum_{k=0}^{n-1}(\Delta_n f)(\frac{k}{n}) B^{n-1}_{k}
\end{multline*}


 \item Si on suppose $f$ croissante, alors $\Delta_n f$ est à valeurs positives d'après sa définition même. Comme les fonctions de Bernstein sont à valeurs positives, la formule de la question précédente montre que $f'_n$ est à valeurs positives donc $f_n$ est croissante.
\end{enumerate}

\subsection*{IV. Approximations.}
\begin{enumerate}
 \item
\begin{enumerate}
 \item Comme la fonction est de classe $\mathcal{C}^2$, sa dérivée seconde est continue. Elle est donc bornée. Le fait que ses bornes sont atteintes n'est pas utile ici.

 \item Considérons un $x\in[0,1]$ et utilisons la question I.3. avec $y=\frac{k}{n}$ pour chaque $k$. Il existe donc des $z_k$ tels que 
\begin{displaymath}
 f(\frac{k}{n}) = f(x) +(\frac{k}{n}-x)f'(x) + (\frac{k}{n}-x)^2\frac{f''(z_k)}{2} 
\end{displaymath}
En multipliant par les fonctions de Bernstein et en sommant, on obtient
\begin{multline*}
 f_n(x)
= f(x)\underset{=1}{\underbrace{\sum_{k=0}^nB^n_k(x)}} 
+ f'(x)\underset{=x}{\underbrace{\sum_{k=0}^n \frac{k}{n}B^n_k(x)}}
- xf'(x)\underset{=1}{\underbrace{\sum_{k=0}^nB^n_k(x)}}\\
+ \sum_{k=0}^n(\frac{k}{n}-x)^2\frac{f''(z_k)}{2}B^n_k(x)
= f(x) + \frac{1}{2}\sum_{k=0}^n(\frac{k}{n}-x)^2 f''(z_k)B^n_k(x)
\end{multline*}
 d'après les questions précédentes. On peut aussi écrire $f(x)$ comme une somme:
\begin{displaymath}
 f(x) = f(x)\sum_{k=0}^nB^n_k(x) = \sum_{k=0}^n f(x)B^n_k(x)
\end{displaymath}
Les fonctions de Bernstein étant à valeurs positives dans $[0,1]$, on peut majorer la différence:
\begin{multline*}
 \left|f(x)-f_n(x)\right| \leq \sum_{k=0}^n \left|f(x)-f_n(x)\right|B^n_k(x)
=  \frac{1}{2}\sum_{k=0}^n(\frac{k}{n}-x)^2 |f''(z_k)| B^n_k(x)\\
\leq \frac{M_2}{2}\sum_{k=0}^n(\frac{k}{n}-x)^2  B^n_k(x)
\leq \frac{M_2}{2} \frac{x(1-x)}{n}
\end{multline*}

 \item Une étude rapide de fonction (maximum en $\frac{1}{2}$) montre que $0\leq x(1-x) \leq \frac{1}{4}$ pour tous les $x$ de $[0,1]$. On en tire donc:
\begin{displaymath}
 \forall x\in [0,1],\;
\left|f(x)-f_n(x)\right| \leq \frac{M_2}{8n}
\end{displaymath}
La suite des $\frac{M_2}{8n}$ tend vers $0$ et le $N$ associé à un $\varepsilon$ par la définition de la convergence de la suite sera valable pour tous les $x$ de la condition $\mathcal{P}_1$.
\end{enumerate}
 
 \item
\begin{enumerate}
 \item Dans cette question, le point important est simplement que les valeurs de $B^n_k$ sont toujours positives. Une somme qui porte sur les $k$ de $K_\alpha(x)$ est plus petite qu'une somme obtenue en ajoutant des termes positifs pour les autres $k$.\newline
 On considère la somme de la question II5. Par définition de $K_\alpha(x)$ et d'après la remarque précédente,
\begin{displaymath}
 \alpha^2\sum_{k\in K_\alpha(x)}B^n_k(x) \leq \sum_{k\in K_\alpha(x)}(\frac{k}{n}-x)^2 B^n_k(x)
\leq \sum_{k = 1}^n(\frac{k}{n}-x)^2 B^n_k(x)
=\frac{x(1-x)}{n}
\end{displaymath}
Comme $x(1-x)\leq \frac{1}{4}$, on en déduit l'inégalité demandée
\begin{displaymath}
 \sum_{k\in K_\alpha(x)}B^n_k(x) \leq \frac{1}{4n\alpha^2}
\end{displaymath}

 \item On a déjà vu que l'on peut écrire $f(x)$ comme une somme
\begin{displaymath}
 f(x) = \sum_{k=0}^nf(x)B^n_k(x)
\end{displaymath}
On factorise partiellement la différence et on sépare les $k$ avant de majorer
\begin{multline*}
 \left| f(x)-f_n(x)\right|
= \left| \sum_{k=0}^n\left(f(x) -f_n(x) \right) B^n_k(x)\right|
\leq  \sum_{k=0}^n\left|f(x) -f_n(x) \right| B^n_k(x)\\
\leq  \sum_{k\in K_\alpha(x)}\left|f(x) -f_n(x) \right| B^n_k(x)
+ \sum_{k\in K'_\alpha(x)}\left|f(x) -f_n(x) \right| B^n_k(x)
\end{multline*}
Pour les $k\in K_\alpha(x)$, majorons par $\left|f(x) -f_n(x) \right|\leq 2M_0$ et ne faisons rien pour les autres.  On en déduit:
\begin{multline*}
\left| f(x)-f_n(x)\right| 
\leq  2M_0\sum_{k\in K_\alpha(x)} B^n_k(x) + \sum_{k\in K'_\alpha(x)}\left|f(x) -f_n(x) \right| B^n_k(x)\\
\leq \frac{M_0}{2n\alpha^2} + \sum_{k\in K'_\alpha(x)}\left|f(x) -f_n(x) \right| B^n_k(x)
\end{multline*}
en utilisant l'inégalité de la question précédente.
\end{enumerate}
Pour tout $x\in [0,1]$ et tout $\varepsilon>0$, comme $f$ est continue en $x$, il existe un $\alpha>0$ (qui dépend de $x$ à priori) tel que $\left|f(y)-f(x)\right|\leq \frac{\varepsilon}{2}$ pour $y\in[0,1]$ vérifiant $|x-y|\leq \alpha$. Ceci se produit pour les $\frac{k}{n}$ lorsque $k\in K'_\alpha(x)$. On en tire
\begin{displaymath}
 \sum_{k\in K'_\alpha(x)}\left|f(x) -f_n(x) \right| B^n_k(x)
\leq \frac{\varepsilon}{2} \sum_{k\in K'_\alpha(x)} B^n_k(x)
\leq \frac{\varepsilon}{2} \sum_{k=0}^1 B^n_k(x) = \frac{\varepsilon}{2}
\end{displaymath}
Pour ce $\alpha$ là, on peut donc écrire que pour tout $n$,
\begin{displaymath}
 \left| f(x)-f_n(x)\right| \leq \frac{M_0}{2n\alpha^2} + \frac{\varepsilon}{2}
\end{displaymath}
Comme la suite $\left( \frac{M_0}{2n\alpha^2}\right) _{n\in \N^*}$ converge vers $0$, il existe un $N$ à partir duquel 
\begin{displaymath}
 \frac{M_0}{2n\alpha^2} \leq \frac{\varepsilon}{2}
\end{displaymath}
 Ce qui montre la proposition $\mathcal{P}_3$.

En fait on pourrait montrer aussi de cette manière la proposition $\mathcal{P}_1$ mais il faudrait utiliser le théorème de Heine sur la continuité uniforme sur un segment. 
\end{enumerate}
