%<dscrpt>Triangles pour lesquels la somme des cos est 3/2.</dscrpt>
\begin{figure}
\centering
\input{Escos_1.pdf_t}
\caption{Exercice II}
\end{figure} 
On cherche les triangles $(ABC)$ tels que la somme des cosinus des angles $\widehat{A}=\alpha$, $\widehat{B}=\beta$, $\widehat{C}=\gamma$. soit $\frac{3}{2}$. (citez un tel triangle)\newline
Pour un triangle $(ABC)$ quelconque, on définit des vecteurs 
\[
\overrightarrow{a}=\frac{1}{\Vert\overrightarrow{BC}\Vert}\overrightarrow{BC} , 
\overrightarrow{b}=\frac{1}{\Vert\overrightarrow{CA}\Vert}\overrightarrow{CA} , 
\overrightarrow{c}=\frac{1}{\Vert\overrightarrow{AB}\Vert}\overrightarrow{AB} 
\]
\begin{enumerate}
\item Préciser les angles entre les vecteurs $\overrightarrow{a}$, $\overrightarrow{b}$, $\overrightarrow{c}$ en fonction de $\widehat{A}=\alpha$, $\widehat{B}=\beta$, $\widehat{C}=\gamma$.
\item Transformer le carré de la norme de $\overrightarrow{a}+\overrightarrow{b}+\overrightarrow{c}$. En déduire les triangles cherchés.
\end{enumerate}
