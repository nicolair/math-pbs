%<dscrpt>Irrationalité de Pi^2$.</dscrpt>
On se propose de montrer que $\pi^2$ est irrationnel en raisonnant par l'absurde. On suppose donc qu'il existe des entiers naturels non nuls $a$ et $b$ tels que $\pi^2=\frac{a}{b}$.

Pour tout $n\in \N^*$, on définit dans $\R$ une fonction polynomiale $f_n$ et un réel $A_n$ par
\begin{displaymath}
 f_n(x)=\frac{1}{n!}x^n(1-x)^n,\hspace{1cm}
A_n = \pi \int_{0}^{1}a^nf_n(x)\sin(\pi x)\,dx
\end{displaymath}
On pose, pour $n\in\N^*$ et $x$ réel,
\begin{align*}
 F_n(x)&=b^n\sum_{k=0}^{n}(-1)^k\pi^{2(n-k)}f_n^{(2k)}(x) \\
 g_n(x)&=F_n'(x)\sin(\pi x)-\pi F_n(x)\cos(\pi x)
\end{align*}


\begin{enumerate}
 \item Montrer que $f_n(x)=\frac{1}{n!}\sum_{i=n}^{2n}e_i x^i$ avec $e_n,e_{n+1},\cdots,e_{2n}$ dans $\Z$.
 \item Montrer que, pour tout $k\in\N$, les valeurs des dérivées $f_n^{(k)}(0)$ et $f_n^{(k)}(1)$ sont des entiers relatifs.
\item
\begin{enumerate}
 \item Montrer que $F_n(0)$ et $F_n(1)$ sont des entiers.
 \item Montrer que
\begin{displaymath}
 g_n'(x)=\pi^2a^nf_n(x)\sin(\pi x)
\end{displaymath}
 puis que $A_n$ est entier.
\end{enumerate}

\item On pose $u_n=\frac{1}{n!}a^n$.
\begin{enumerate}
 \item Montrer qu'il existe $n_0\in \N$ tel que $u_n< \frac{1}{2}$ pour $n$ entier supérieur ou égal à $n_0$.
 \item Montrer que $0\leq f_n(x)\leq \frac{1}{n!}$ pour $x\in[0,1]$ et $n\in \N^*$.
 \item Montrer que $\pi^2$ est irrationnel. En déduire que $\pi$ est irrationnel.
\end{enumerate}

\end{enumerate}
