\begin{enumerate}
 \item \begin{enumerate}
 \item La relation de récurrence étant linéaire, il est immédiat que si $(x_n)_{n\in\N}$ et $(y_n)_{n\in\N}$ sont dans $E_a$ et $\lambda$ et $\mu$ dans $\R$, la suite $(\lambda x_n + \mu y_n)_{n\in\N} \in E_a$.
\item Chaque suite dans $E_a$ est définie de manière unique par ses trois premiers termes et la relation de  récurrence. L'application
\begin{displaymath}
 \fonc{\phi}{E_a}{\R^3}{(x_n)_{n\in\N}}{(x_0,x_1,x_2)}
\end{displaymath}
est un isomorphisme linéaire. La dimension de $E_a$ est donc $3$.
\end{enumerate}
\item \begin{enumerate}
 \item On vérifie (simplement en remplaçant dans la relation de récurrence) que les fonctions constantes sont dans $E_a$.
\item La suite $(v_n)_{n\in\N}$ est définie par la différence entre deux termes consécutifs. En \og cassant\fg~ les coefficients $(1+a)$ et $(1+4a)$ de la relation $(1)$, celle-ci s'écrit
\begin{align*}
 4(u_{n+3}-u_{n+2}) &= 4a(u_{n+2}-u_{n+1}) -(u_{n+1}-u_n) \\
4v_{n+2} &= 4a v_{n+1}-v_n
\end{align*}
La suite $(v_n)_{n\in\N}$ vérifie donc
\begin{align*}
 4v_{n+2}-4av_{n+1}+v_n= 0 \hspace{2cm}(2)
\end{align*}
\item On désigne par $F_a$ l'ensemble des suites vérifiant $(2)$. Montrons d'abord qu'une suite vérifiant $(2)$ vérifie aussi $(1)$.\newline
Soit $(x_n)_{n\in\N}\in F_a$, d'après $(2)$ on a :
\[
 x_n = 4ax_{n+1}-4x_{n+2} 
 \Rightarrow x_{n+1} = 4a x_{n+2} - 4x_{n+3}
 \Rightarrow 4ax_{n+2}-x_{n+1} = 4x_{n+3}.
\]
En remplaçant dans le membre de droite de $(1)$, on a:
\begin{multline*}
4(1+a)x_{n+2}-(1+4a)x_{n+1}+x_n \\ 
= 4ax_{n+2}-x_{n+1}+ \underset{= -x_n \text{ d'après (2)}}{\underbrace{4x_{n+2}-4ax_{n+1}}} + x_n  
= 4ax_{n+2}-x_{n+1} = 4x_{n+3}.
\end{multline*}
Ceci prouve que $(x_n)_{n\in\N} \in E_a$ d'où $F_a\subset E_a$. Cet ensemble $F_a$ est stable par combinaison linéaire comme tout ensemble de suites vérifiant une relation de récurrence linéaire. On en conclut que $F_a$ est un sous-espace vectoriel de $E_a$.

La relation $(2)$ est vérifiée par les suites fabriquées à partir de celles de $E_a$ en prenant la différence de deux termes consécutifs. Mais les suites de $F_a$ sont elles toutes de cette forme ? \newline
En fait oui. L'application qui a une suite associe la différence de deux termes consécutifs est un endomorphisme de $E_a$ dont l'image est \emph{incluse} dans $F_a$ qui est de dimension $2$. Le noyau de cette application linéaire est $K$ (espace des suites constantes) qui est de dimension $1$. Le rang est donc $2$ ce qui prouve que $F_a$ est \emph{l'image} de l'application.
\end{enumerate}
\item Pour déterminer une base de $F_a$, on forme l'équation caractéristique
\begin{displaymath}
4x^2-4ax+1=0 
\end{displaymath}
dont le discriminant est $16(a^2-1)$.\newline
Lorsque $0\leq a <1$, l'équation a deux racines complexes conjuguées. On pose $a=\cos \theta$ avec $\theta \in ]0,\frac{\pi}{2}[$. Les racines sont $\frac{1}{2}e^{i\theta}$ et $\frac{1}{2}e^{-i\theta}$. Une base est alors (cours) :
\begin{displaymath}
 ((2^{-n}\cos n\theta)_{n\in\N},(2^{-n}\sin n\theta)_{n\in\N})
\end{displaymath}
Lorsque $a=1$. L'équation a une racine double $\frac{1}{2}$.  Une base est alors (cours) :
\begin{displaymath}
 ((2^{-n})_{n\in\N},(2^{-n} n)_{n\in\N})
\end{displaymath}
Lorsque $1<a$, l'équation a deux racines réelles. On pose $a=\ch \theta$ avec $\theta >0$. Les racines sont $\frac{1}{2}e^{\theta}$ et $\frac{1}{2}e^{-\theta}$. Une base est alors (cours) :
\begin{displaymath}
 ((2^{-n}e^{n\theta})_{n\in\N},(2^{-n}e^{-n\theta})_{n\in\N})
\end{displaymath}

\item L'ensemble $K$ des suites constantes est dans $F_a$ si et seulement si la suite constante de valeur $1$ est dans $E_a$ c'est à dire
\begin{align*}
 \forall k\in \N &: 4 = 4(1+a)-(1+4a)+1
\end{align*}
Ce qui ne se produit que pour
\begin{displaymath}
 a_0 = \dfrac{5}{4}
\end{displaymath}
\item Dans cette question, on suppose $a\neq\frac{5}{4}$.
\begin{enumerate}
 \item 
L'hypothèse $a\neq\frac{5}{4}$ entraîne que $K$ n'est pas inclus dans $F_a$. Comme c'est une droite vectorielle, on en tire $K\cap F_a=\{(0)\}$. Ici $(0)$ désigne la suite nulle. De plus :
\begin{itemize}
 \item $K$ est de dimension $1$ (la famille constituée de la suite constante de valeur $1$ en est une base) 
\item $F_a$ est de dimension $2$ (question 3.)
\item  $E_a$ est de dimension $3$  (question 1.b.)
\end{itemize}
Il en résulte que $K$ et $F_a$ sont supplémentaires dans $E_a$.\newline 
Soit $\left( u_n\right) _{n\in \N}$ une suite quelconque de $E_a$. Comment se décompose-t-elle sur ces deux supplémentaires?\newline
Elle est la somme d'une suite constante de valeur $c$ et d'une suite $\left( v_n\right) _{n\in \N}$ de $F_a$. Cette dernière suite est caractérisée par ses deux premiers termes $v_0$ et $v_1$. Pour calculer $c$, $v_0$, $v_1$, formons les relations venant des trois premiers termes
\begin{displaymath}
\left\lbrace  
\begin{aligned}
 c + v_0 &= u_0 & &\times\frac{1}{4}\\
 c + v_1 &= u_1 & &\times(-a)\\
 c -\frac{1}{4} v_0 + av_1 &= u_2 & &\times 1
\end{aligned}
\right. 
\end{displaymath}
et combinons les pour trouver $c$. En sommant avec les coefficients indiqués, on obtient
\begin{displaymath}
  c=\frac{\frac{1}{4}u_0-au_1+u_2}{\frac{5}{4}-a},\hspace{0.5cm} v_0=u_0 -c,\hspace{0.5cm} v_1=u_1 -c
\end{displaymath}

\item On obtient des bases de $E_a$ simplement en insérant $(1)$ (la suite constante de valeur $1$) dans les familles trouvées en 3.
\end{enumerate}
\item Dans le cas particulier $a_0=\frac{5}{4}$ la relation de récurrence définissant $E_{a_0}$ devient 
\begin{displaymath}
 4x_{n+3}=9x_{n+2}-6x_{n+1}+x_n
\end{displaymath}
Elle est vérifiée par $(n)_{n\in\N}$. Les racines de l'équation caractéristique (de $F_a$) sont alors $1$ et $\frac{1}{4}$. En fait 1 est une racine double de l'équation caractéristique de degré 3 de $E_a$. Bien que ce ne soit pas vraiment plus compliqué que pour les récurrences d'ordre 2, les récurrences linéaires d'ordre 3 ou plus ne sont pas au programme.
Pour montrer que la famille
\begin{displaymath}
 ((n)_{n\in\N},(1)_{n\in\N},(4^{-n})_{n\in\N})
\end{displaymath}
est une base de $E_{a_0}$, il suffit (dimension) de prouver qu'elle est libre. Supposons donc que
\begin{displaymath}
 \alpha(n)_{n\in\N} +\beta(1)_{n\in\N}+\gamma(4^{-n})_{n\in\N})=(0)
\end{displaymath}
\'Ecrivons la nullité des trois premiers termes et transformons le système par opérations élémentaires :
\begin{multline*}
% use packages: array
\renewcommand{\arraystretch}{2}
\left\lbrace \begin{array}{lclclcl}
& & \beta &+& \gamma & = & 0\\ 
\alpha &+&\beta &+& \dfrac{1}{4}\gamma &= & 0 \\ 
2\alpha &+&\beta &+& \dfrac{1}{16}\gamma &= &0
\end{array}\right. 
\Leftrightarrow
\left\lbrace \begin{array}{lclclcl}
\alpha &+&\beta &+& \dfrac{1}{4}\gamma &= & 0 \\ 
& & \beta &+& \gamma & = & 0\\ 
 & -&\beta  &+& -\dfrac{7}{16}\gamma &= &0
\end{array}\right. 
\\ \Leftrightarrow
\left\lbrace \begin{array}{lclclcl}
\alpha &+&\beta &+& \dfrac{1}{4}\gamma &= & 0 \\ 
& & \beta &+& \gamma & = & 0\\ 
 & &  &+& (1-\dfrac{7}{16})\gamma &= &0
\end{array}\right. 
\end{multline*}
ce qui entraîne $\alpha = \beta = \gamma$ et donc que la famille est libre.

\item Une suite de $E_a$ se décompose comme une somme d'une suite constante et d'une suite de $F_a$. La question 5.a. donne la valeur $c$ de la suite constante de cette décomposition. On trouve ici $c=1$ avec les conditions initiales particulières. Notons $\left( v_n\right) _{n\in \N}$ la composante dans $F_a$. Elle admet pour conditions initiales
\begin{displaymath}
 v_0=-\sqrt{|a^2-1|},\hspace{0.5cm} v_1 = 0,\hspace{0.5cm} v_2 = \frac{1}{4}\sqrt{|a^2-1|}
\end{displaymath}
Lorsque $a=1$, cette suite est identiquement nulle donc $u_n=1$ pour tous les $n$.\newline
Lorsque $a=\cos \theta\in [0,1[$ les conditions initiales deviennent
\begin{displaymath}
 v_0=-\sin \theta,\hspace{0.5cm} v_1 = 0,\hspace{0.5cm} v_2 = \frac{1}{4}\sin \theta
\end{displaymath}
Comme la suite est dans $F_a$, il existe $\lambda$ et $\mu$ tels que 
\begin{displaymath}
 \forall n\in \N,\hspace{0.5cm}v_n = \lambda 2^{-n}\cos(n\theta) + \mu2^{-n}\sin(n\theta)
\end{displaymath}
Les conditions initiales conduisent à
\begin{displaymath}
 \left\lbrace 
\begin{aligned}
\lambda &= -\sin \theta\\
\frac{\cos \theta}{2}\lambda + \frac{\sin \theta}{2}\mu &=0 
\end{aligned}
\right. 
\Rightarrow
 \left\lbrace 
\begin{aligned}
\lambda &= -\sin \theta\\
\mu &= \cos \theta 
\end{aligned}
\right. 
\end{displaymath}
On en déduit
\begin{displaymath}
 \forall n\in \N,\hspace{0.5cm}v_n = 2^{-n}\sin((n-1)\theta),\hspace{0.5cm} u_n= 1 +2^{-n}\sin((n-1)\theta)
\end{displaymath}
Dans le cas $a=\ch\theta >1$, des calculs analogues conduisent à 
\begin{displaymath}
 \forall n\in \N,\hspace{0.5cm}v_n = 2^{-n}\sh((n-1)\theta),\hspace{0.5cm} u_n= 1 +2^{-n}\sh((n-1)\theta)
\end{displaymath}

\end{enumerate}