%<dscrpt>Irrationnalité de e. Généralisation de la formule du binôme avec un élément nilpotent.</dscrpt>
Dans tout le problème, on confondra un polynôme à coefficients réels avec la fonction polynomiale définie dans $\R$ qui lui est associée.
\subsubsection*{Partie I. Irrationnalité de $e^m$}
Dans cette partie, on \emph{admet} que pour tout entier naturel $n$, il existe des polynômes $A_n$ et $B_n$ à coefficients dans $\Z$ et de degré inférieur ou égal à $n$ définissant une fonction $f_n$
\[\forall x \in \R\; : \; f_n(x)= A_n(x)+B_n(x)e^x\]
telle que :
\begin{eqnarray*}
 \forall k \in \{0,\cdots,n\} &:& f_n^{(k)}(0)=0 \\
\forall x \in \R &:& f_n^{(n+1)}(x)=x^ne^x
\end{eqnarray*}
\begin{enumerate}
 \item Calculer des polynômes $A_n$ et $B_n$ satisfaisant aux conditions pour $n$ égal à 1 ou 2.
 \item \begin{enumerate}
   \item Calculer, à l'aide de la formule de Leibniz, la dérivée $n+1$ ème de la fonction
\[x \rightarrow \frac{x^{2n+1}e^x}{(n+1)!}\]
Montrer que le coefficient de $x^ne^x$ est un entier à préciser.
  \item Montrer que :
\[\forall x >0 \;:\; 0<f_n(x)<\frac{x^{2n+1}e^x}{(n+1)!}\]
(on pourra utiliser des tableaux de variations et des dérivations successives) 
  \end{enumerate}
 \item Soit $m$ un entier naturel non nul, on suppose qu'il existe un entier $q$ non nul tel que $qe^m$ soit entier. Montrer que pour tous les entiers $n$ :
\[qf_n(m)\in \Z\]
En déduire une contradiction et conclure.
\end{enumerate}

\subsubsection*{Partie II. Généralisation de la formule du binôme.}
Pour tout couple $(m,k)\in \Z\times\N$, on définit des nombres $c_{m,k}$ par les relations
\begin{displaymath}
% use packages: array
\left\lbrace 
\begin{array}{lll}
\forall m \in \Z & : & c_{m,0}=1 \\ 
\forall k \geq 1 & : & c_{0,k}=0
\end{array}
\right. 
\end{displaymath}
et
\[\forall k\geq1 \; : \; c_{m,k}=c_{m-1,k} + c_{m-1,k-1}\]
\begin{enumerate}
 \item Former le tableau des $c_{m,k}$ avec $m$ comme numéro de la ligne et $k$ comme numéro de la colonne pour $m$ entre $-4$ et $+4$ et $k$ entre $0$ et $4$. \newline
Formulez des remarques intéressantes relativement à ces coefficients.
\item On considère un anneau $A$ dont le neutre additif est noté $0_A$ et le neutre multiplicatif (élément unité) est noté $i$. Cet anneau $A$ contient un élément $d$ (dit \emph{nilpotent}) pour lequel il existe un entier $n\geq 1$ tel que 
\[d^{n+1} = 0_A\]
\begin{enumerate}
 \item Calculer
\[\left( \sum _{k=0}^{n}c_{-1,k}d^k \right) (i+d)\]
En déduire que $i+d$ est un élément inversible de $A$.
\item Montrer que pour tout $m\in \Z$ :
\[(i+d)^m = \sum _{k=0}^{n}c_{m,k}d^k\]
\end{enumerate}

\end{enumerate}

\subsubsection*{Partie III. Existence de $A_n$ et $B_n$.}
On désigne par $\R_n [X]$ l'espace des polynômes à coefficients réels et dont le degré est inférieur ou égal à $n$. On considère l'anneau des endomorphismes de $\R_n [X]$. On rappelle que, dans cet anneau, la loi multiplicative est la composition $\circ$ des endomorphismes.\newline
L'unité est l'application linéaire identité notée ici $i$:
\begin{displaymath}
i : 
\left\lbrace
\begin{aligned}
\R_n[X] & \rightarrow  \R_n[X] \\ 
P & \rightarrow  P
\end{aligned}
 \right. 
\end{displaymath}
L'élément nilpotent considéré est la dérivation notée ici $d$:
\begin{displaymath}
d : 
\left\lbrace 
\begin{aligned}
\R_n[X] & \rightarrow & \R_n[X] \\ 
P & \rightarrow & P^\prime
\end{aligned}
\right. 
\end{displaymath}
\begin{enumerate}
 \item Montrer que $i+d$ est un automorphisme de $\R_n[X]$.
\item Pour tout entier $n$ on pose :
\[B_n=(i+d)^{-(n+1)}(X^n)\]
et on définit la fonction $\beta_n$ par :
\[\forall x\in \R \; : \; \beta_n (x)=B_n(x)e^x\]
\begin{enumerate}
 \item Préciser, à l'aide d'une puissance de $i+d$ la dérivée $m$ ième de $\beta_n$ pour un entier naturel $m$ quelconque. Que se passe-t-il pour $m=n+1$ ?
\item Pour $m\in \{0, \cdots, n\}$, montrer que 
\[\frac{\beta_n^{(m)}(0)}{m!} \in \Z\]
Conclure.
\end{enumerate}

\end{enumerate}

