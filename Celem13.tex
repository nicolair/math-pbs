\subsection*{Exercice 1}
Remarquons que $D_a$ est formé par les couples $(k,a-k)$ o{\`u} $k$
décrit $\{0,\cdots,n\}$. On en déduit que
\begin{displaymath}
A_a=\sum_{k=0}^a \binom{a}{k}=2^a
\end{displaymath}
Comme $T_n$ est l'union disjointe des $D_a$,
\begin{displaymath}
B_n=\sum_{a=0}^nA_a=1+2+\cdots+2^n=2^{n+1}-1
\end{displaymath}
La somme $G_{b,n}$ se calcule \og par dominos\fg
\begin{multline*}
G_{b,n} = 
\binom{b}{0}+(\binom{b+2}{1}-\binom{b+1}{0})+(\binom{b+3}{2}-\binom{b+2}{1})+\cdots+(\binom{n+b+1}{n}-\binom{n+b}{n-1})\\
 = 1-1+\binom{n+b+1}{n} = \binom{n+b+1}{n}
\end{multline*}

\begin{multline*}
D_n = 
\sum_{y=0}^n \left(  \sum_{x=0}^n\binom{x+y}{x} \right)
= \sum_{y=0}^nG_{y,n} = \sum_{y=0}^n\binom{n+y+1}{y} 
= \sum_{y=1}^{n+1}\binom{n+y}{y}\\
 = G_{n,n+1}-1 = \binom{2n+2}{n+1}-1
\end{multline*}

%ex exo5 de elem4
\subsection*{Exercice 2}
On linéarise par les formules transformant les produits en sommes :
\begin{multline*}
4\sin^{2}x \sin(2kx) = 2\sin (2kx)-\sin (2(k+1)x)-\sin(2(k-1)x)\\
 = (\sin (2kx)-\sin (2(k+1)x))-( \sin (2(k-1)x)-\sin (2kx)) = u_{k}-u_{k-1}
\end{multline*}
avec $u_{k}=\sin (2kx)-\sin (2(k+1)x)$.
On en déduit
\begin{multline*}
\lefteqn{\sum _{k=p}^{q}4\sin^{2}x \sin(2kx)=u_{q}-u_{p-1}}\\
 = \sin(2qx)- \sin(2(q+1)x)- \sin(2(p-1)x)+ \sin(2px)
\end{multline*}


%ex exo9 de elem4
\subsection*{Exercice 3}
Notons $X$ la différence des deux quotients, comme
\begin{displaymath}
\frac{\binom{n}{k}}{ \binom{2n}{k}}=\frac{n(n-1)\cdots(n-k+1)}{(2n)(2n-1)\cdots(2n-k+1)}
\end{displaymath}
on a
\begin{multline*}
X = \frac{n(n-1)\cdots(n-k+1)}{(2n)(2n-1)\cdots(2n-k+1)}-\frac{n(n-1)\cdots(n-k)}{(2n)(2n-1)\cdots(2n-k)}\\
  = \frac{n\cdots(n-k+1)}{ (2n)\cdots(2n-k)}(2n-k-n+k) 
  = \frac{1}{2}\frac{n\cdots(n-k+1)}{ (2n-1)\cdots(2n-k)}\\
  = \frac{1}{2}\frac{n\cdots(n-k+1)}{ k!} \frac{k!}{ (2n-1)\cdots(2n-1-k+1)} 
  = \frac{1}{2}\frac{\binom{n}{k}}{\binom{2n-1}{k}}
\end{multline*}
Comme cette décomposition n'est valable que pour $k<n$, on en déduit (par simplification télescopique)
\begin{displaymath}
\sum _{k=0}^{n-1}\frac{\binom{n}{k}}{\binom{2n-1}{k}}
= 2\left(\frac{\binom{n}{0}}{\binom{2n}{0}}-\frac{\binom{n}{n}}{\binom{2n}{n}}\right) 
= 2 - 2\frac{1}{\binom{2n}{n}}
\end{displaymath}
Pour la somme complète, on ajoute le dernier terme
\begin{displaymath}
\sum _{k=0}^{n}\frac{\binom{n}{k}}{\binom{2n-1}{k}} = 2 - 2\frac{1}{\binom{2n}{n}} + \frac{1}{\binom{2n-1}{n}}
\end{displaymath}
En fait cette somme est égale à $2$ car
\begin{displaymath}
 \binom{2n-1}{n} = \frac{\overset{n \text{ facteurs}}{\overbrace{(2n-1)(2n-2)\cdots (2n-1-n)}}}{n!}
 = \frac{1}{2}\,\frac{2n\,\overset{n-1 \text{ facteurs}}{\overbrace{(2n-1)(2n-2)\cdots}}}{n!}
 = \frac{1}{2}\,\binom{2n}{n}.
\end{displaymath}
